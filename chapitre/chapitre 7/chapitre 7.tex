\chapter{Calcul de primitives}

\minitoc
\begin{nota}
    \begin{itemize}
        \item \(I\) et \(J\) désige des intervalles de \(\R\), non vides et non réduits à un point
        \item \(\K\) désigne l'ensemble \(\R\) ou \(\C\)
    \end{itemize}
\end{nota}

\section{Primitives}

\begin{defprop}
    Soit \(f:I\to\K\) une fonction quelconque. \\
    On dit qu'une fonction \(F:I\to\K\) est une primitive de \(f\) sur \(I\) si \(F\) est dérivable sur \(I\) de dérivée \(f\) \\
    Si \(f\) admet une primitive \(F\) sur \(I\) alors l'ensemble des primitives de \(f\) sur \(I\) est \(\accol{x\mapsto F(x)+ \lambda \tq \lambda \in \K}\)
\end{defprop}

\begin{theo}[Théorème fondamental de l'analyse]
    Si \(f\) \textbf{CONTINUE} sur \(I\) alors : 
    \begin{itemize}
        \item pour tout \(x_0\) réel, la fonction \(F:\int_{x_0}^{x} f(t) dt\) est une primitive de \(f\) sur \(I\)
        \item la fonction \(f\) admet des primitives sur \(I\)
    \end{itemize}
\end{theo}

\begin{defprop}[Application au calcul d'intégrales sur un segment]
    Si \(f\) est \textbf{CONTINUE} sur \(I\) et \(F\) uine primitive de \(f\) sur \(I\) alors, pour tout réels \(a\) et \(b\) dans \(I\), on a :
    \[\int_{a}^b f(t) dt = F(b)-F(a) \underset{\mathrm{notation}}{=} \croch{F}^a_b \]
\end{defprop}

\section{Primitives usuelles}
\begin{defprop}[Puissances entière ou réelles]
    ~\\
    \renewcommand{\arraystretch}{2.75}
	\begin{tabular}{|l|l|c|}

		\hline
		Si la fonction \(f\) est \(\dots\) & alors une primitive de \(f\) est \(\dots\) & sur tout intervalle \(I\) inclus dans \(\dots\)\\
        \hline
        \(x\mapsto x^n\) avec \(n \in \N\) & \(x\mapsto \frac{1}{n+1} x^{n+1}\) & \(\R\) \\
        \(x\mapsto x^n\) avec \(n \in \Zm \pd\accol{-1}\) & \(x\mapsto \frac{1}{n+1} x^{n+1}\) & \(\Rs\) \\
        \(x\mapsto \frac{1}{x}\) & \(x\mapsto \ln\paren{\abs{x}}\) & \(\Rs\) \\
        \(x\mapsto \frac{1}{2\sqrt{x}}\) & \(x\mapsto \sqrt{x}\) & \(\Rps\) \\
        \(x\mapsto x^{\alpha}\) avec \(\alpha \in \R \pd\Z\)& \(x\mapsto \frac{1}{\alpha + 1}x^{\alpha+1}\) & \(\Rps\) \\

        \hline
	\end{tabular}
\end{defprop}

\begin{defprop}[Exponentielle à valeurs réelles ou complexes et logarithme népérien]
    ~\\
    \renewcommand{\arraystretch}{2.75}
	\begin{tabular}{|l|l|c|}

		\hline
		Si la fonction \(f\) est \(\dots\) & alors une primitive de \(f\) est \(\dots\) & sur tout intervalle \(I\) inclus dans \(\dots\)\\
        \hline
        \(x\mapsto e^{\lambda x}\) avec \(\lambda \in \Ks\) & \(x\mapsto \frac{1}{\lambda} e^{\lambda x}\) & \(\R\) \\
        \(x\mapsto e^x\) & \(x\mapsto e^x\) & \(\R\) \\
        \(x\mapsto \ln(x)\) & \(x\mapsto x\ln(x)-x\) & \(\Rps\) \\
        \hline
	\end{tabular}
\end{defprop}
\vspace{10cm} 
\begin{defprop}[Fonctions hyperboliques]
    ~\\
    \renewcommand{\arraystretch}{2.75}
	\begin{tabular}{|l|l|c|}

		\hline
		Si la fonction \(f\) est \(\dots\) & alors une primitive de \(f\) est \(\dots\) & sur tout intervalle \(I\) inclus dans \(\dots\)\\
        \hline
        \(x\mapsto \ch(x)\) & \(x\mapsto \sh(x) \) & \(\R\) \\
        \(x\mapsto \sh(x)\) & \(x\mapsto \ch(x) \) & \(\R\) \\
        \(x\mapsto 1-\tth^2(x)\) & \(x\mapsto \tth(x) \) & \(\R\) \\
        \(x\mapsto \frac{1}{\ch^2(x)}\) & \(x\mapsto \tth(x) \) & \(\R\) \\        
        \hline
	\end{tabular}
\end{defprop}
\begin{defprop}[Fonctions circulaires et fonctions circulaires réciproques]
    ~\\
    \renewcommand{\arraystretch}{2.75}
	\begin{tabular}{|l|l|c|}

		\hline
		Si la fonction \(f\) est \(\dots\) & alors une primitive de \(f\) est \(\dots\) & sur tout intervalle \(I\) inclus dans \(\dots\)\\
        \hline
        \(x\mapsto \cos(x)\) & \(x\mapsto \sin(x) \) & \(\R\) \\
        \(x\mapsto \sin(x)\) & \(x\mapsto -\cos(x) \) & \(\R\) \\
        \(x\mapsto 1+\tan^2(x)\) & \(x\mapsto \tan(x) \) & \(\R\pd\accol{\frac{\pi}{2}+k\pi\tq k\in\Z}\) \\
        \(x\mapsto \frac{1}{\cos^2(x)}\) & \(x\mapsto \tan(x) \) & \(\R\pd\accol{\frac{\pi}{2}+k\pi\tq k\in\Z}\) \\
        \(x\mapsto \frac{-1}{\sqrt{1-x^2}}\) & \(x\mapsto \Arccos(x) \) & \(\intervee{-1}{1}\) \\
        \(x\mapsto \frac{1}{\sqrt{1-x^2}}\) & \(x\mapsto \Arcsin(x) \) & \(\intervee{-1}{1}\) \\
        \(x\mapsto \frac{1}{1+x^2}\) & \(x\mapsto \Arctan(x) \) & \(\R\) \\

        \hline
	\end{tabular}
\end{defprop}

\section{Calculs de primitives}
\begin{defprop}
    \begin{itemize}
        \item \underline{Primitives d’une combinaison linéaire de fonctions}\\
        Si \(f : I \mapsto \K\) et \(g : I \mapsto \K\) sont des fonctions qui admettent des primitives sur \(I\) notées \(F\) et \(G\) alors, pour tous \(\alpha\) et \(\beta\) dans \(\K\), la fonction\( \alpha f + \beta g : I \mapsto \K\) admet pour primitive sur \(I\) la fonction \( \alpha F + \beta G\)
        \item \underline{Primitives d’une fonction dérivée de fonctions composées} \\
        Si \(u : I \mapsto \R\) est une fonction dérivable sur \(I\) tel que pour tout \(x\) de \(I\), \(u(x)\) appartient à \(J\) et si \(g : J \mapsto \K \) est une fonction dérivable sur \(I\) alors une primitive de la fonction \(f : x  \mapsto u'(x)g'(u(x))\) sur \(I\) est la fonction \(F : x  \mapsto g (u(x))\).\\
        Dans le tableau ci-dessous (à savoir retrouver à partir des primitives usuelles), \(I\) désigne un intervalle sur lequel \(u\) est dérivable et tel que, pour tout \(x\) de \(I\), \(u(x)\) appartient au domaine de dérivabilité de \(F\) .
    \end{itemize}
    \renewcommand{\arraystretch}{2.5}
    \begin{center}
        
	    \begin{tabular}{|l|l|}
	    	\hline
	    	Si la fonction \(f\) est \(\dots\) & alors une primitive de \(f\) est \(\dots\) \\
            \hline
            \(x\mapsto u'(x)\paren{u(x)}^{\alpha}\) avec \(\alpha \in \R\pd\accol{-1}\)  & \(x\mapsto \frac{1}{\alpha+1}\paren{u(x)}^{\alpha+1} \) \\
            \(x\mapsto \frac{u'(x)}{u(x)}\)  & \(x\mapsto \ln(\abs{u(x)}) \) \\
            \hline
            \(x\mapsto u'(x)e^{\lambda u(x)}\) avec \(\lambda \in \Ks\) & \(x\mapsto \frac{1}{\lambda}e^{\lambda u(x)} \) \\
            \(x\mapsto u'(x)\ln\paren{u(x)}\) & \(x\mapsto u(x)\ln(u(x))-u(x) \) \\
            \hline
            \(x\mapsto u'(x)\ch\paren{u(x)}\) & \(x\mapsto \sh(u(x)) \) \\
            \(x\mapsto u'(x)\sh\paren{u(x)}\) & \(x\mapsto \ch(u(x)) \) \\
            \(x\mapsto u'(x)\paren{1+\tth^2\paren{u(x)}}\) & \(x\mapsto \tth(u(x)) \) \\
            \hline

            \(x\mapsto u'(x)\cos\paren{u(x)}\) & \(x\mapsto \sin(u(x)) \) \\
            \(x\mapsto u'(x)\sin\paren{u(x)}\) & \(x\mapsto -\cos(u(x)) \) \\
            \(x\mapsto u'(x)\paren{1+\tan^2\paren{u(x)}}\) & \(x\mapsto \tan(u(x)) \) \\
            \hline
            \(x\mapsto \frac{-u'(x)}{\sqrt{1-u^2(x)}}\) & \(x\mapsto \Arccos(u(x)) \)\\
            \(x\mapsto \frac{u'(x)}{\sqrt{1-u^2(x)}}\) & \(x\mapsto \Arcsin(u(x)) \)  \\
            \(x\mapsto \frac{u'(x)}{1+u^2(x)}\) & \(x\mapsto \Arctan(u(x)) \) \\
            \hline
	    \end{tabular}
    \end{center}
\end{defprop}

\subsection{Deux théorème important}
\begin{defi}[préliminaire]
    Une fonction \(f : I \mapsto \K\) est dite de classe \(\classe{1}\) sur \(I\) si \(f\) est dérivable sur \(I\) et de dérivée continue sur \(I\)
\end{defi}

\begin{theo}[Intégration par parties]
    Si \(u\) et \(v\) sont deux fonctions de classe \(\classe{1}\) sur \(I\) alors, pour tous réels \(a\) et \(b\) dans \(I\), on a :
    \[\int_{a}^{b}u'(t)v(t)dt = \croch{u(t)v(t)}^b_a - \int_{a}^{b}u(t)v'(t)dt\]
\end{theo}

\begin{dem}
    Soit \(u\) et \(v\) deux applications de \(\ensclasse{1}{I}{\R}\) alors \(\forall (a,b) \in I^2\) : 
    \[
    \begin{aligned}
        \int_{a}^{b}(uv)'(t)dt &= \int_{a}^{b}\paren{u'v+uv'}(t)dt \\
         \croch{uv}^b_a &= \int_{a}^{b}(u'v)(t)dt + \int_{a}^{b}(uv')(t)dt \\
          \int_{a}^{b}u'(t)v(t)dt &= \croch{uv}^b_a- \int_{a}^{b}(uv')(t)dt 
    \end{aligned}
    \]
\end{dem}
\begin{theo}[Changement de variable]
    Si \(\phi  :J \mapsto \R\) est fonction de classe \(\classe{1}\) sur \(J\) tel que, pour tout \(t\) de \(J\), \(\phi(t)\) appartient à \(I\) \\
    et\\
    Si \(f  :I\mapsto \K\) est fonction continue sur \(I\) tel que, pour touts \(\alpha\) et \(\beta\) dans \(J\), on a: 
    \[\int_{\alpha}^{\beta} f(\phi(t))\phi'(t) dt = \int_{\phi(\alpha)}^{\phi(\beta)}f(x)dx\]
\end{theo}

\begin{dem}
    Soit \(\phi  :J \mapsto \R\) une fonction de classe \(\classe{1}\) sur \(J\) tel que, pour tout \(t\) de \(J\), \(\phi(t)\) appartient à \(I\) et \(f  :I\mapsto \K\) une fonction continue sur \(I\) tel que, pour touts \(\alpha\) et \(\beta\) dans \(J\), alors : \\

    \(f\) possède une primitive sur \(I\) (car \(f\) est continue \(I\) ) que l'on note \(F\). \\
    On note aussi \(G:t\mapsto F(\phi(t))\) qui est dérivable sur \(J\) par composition ainsi \(G':t\mapsto F'(\phi(t))\times \phi'(t)\), alors :
    \[
    \begin{aligned}
        \int_{\alpha}^{\beta} f(\phi(t))\phi'(t)dt &= \int_{min}^{max} G'(t)dt \\
        &=\croch{G(t)}_{\alpha}^{\beta} \\
        &=F(\phi(\beta)) - F(\phi(\alpha)) \\
        &= \croch{F}_{\phi(\alpha)}^{\phi(\beta)} \\
        &= \int_{\phi(\alpha)}^{\phi(\beta)} f(x)dx
    \end{aligned}
    \]
\end{dem}
\subsection{Primitives de \(x \mapsto e^{ax} \cos(bx)\) ou \(x \mapsto e^{ax} \sin(bx)\)}
\begin{defprop}[]
    \begin{itemize}
        \item \underline{Préliminaire} \\
        Soit \(f\) et \(F\) des fonctions définies sur un intervalle \(I\) à valeurs complexes.
        \begin{enumerate}
            \item \(f\) admet des primitives sur \(I\) si, et seulement si, \(\Reel{f}\) et \(\Ima{f}\) admettent des primitives sur \(I\).
            \item \(F\) est une primitive de \(f\) sur \(I\) si, et seulement si, 
            \(\begin{cases*}
                \Reel{F}\text{ est une primitive de } \Reel{f}\text{ sur  } I \\
                \Ima(F) \text{ est une primitive de } \Ima{f} \text{ sur } I
            \end{cases*}\).
        \end{enumerate}
        \item \underline{Une application usuelle du résultat précédent} \\
        Soit \(a\) et \(b\) des réels tels que \((a,b) \neq (0,0)\). \\
        On note \(\lambda = a+\i b\) et \(f_{\lambda}\) la fonction définie sur \(\R\) par, pour tout \(x\) réel 
            \[f_{\lambda}(x) = e^{ax}\cos(bx) + \i e^{ax}\sin(bx) = e^{ax}e^{bx} \underset{\mathrm{def}}{=} e^{(a+\i b)x} = e^{\lambda x}\]
        La fonction \(F_{\lambda}:x\mapsto \frac{1}{\lambda} e^{\lambda x}\) est une primitive de \(f_{\lambda}\) sur \(\R\) donc :
        \begin{itemize}
            \item la fonction \(\Reel{F_{\lambda}}\) est une primitive de la fonction \(\Reel{F_{\lambda}}:x\mapsto e^{ax}\cos(bx)\) sur \(\R\)
            \item la fonction \(\Ima{F_{\lambda}}\) est une primitive de la fonction \(\Ima{F_{\lambda}}: x\mapsto e^{ax}\sin(bx)\) sur \(\R\)
        \end{itemize}
    \end{itemize}
\end{defprop}

\subsection{Primitives de \(x\mapsto \frac{1}{ax^2+bx+c}\) avec \(a,b\) et \(c\) des réels et \(a\) non nul}
\begin{appl}
    Soit \(a,b\) et \(c\) des réels avec \(a\) non nul et \(g\) la fonction \(g:\R \mapsto \R\) définie par \(g(x) = ax^2+bx+c\) \\
    Trois cas se présentent : 
    \begin{enumerate}
        \item Si \(g\) admet deux racines réelles distinctes \(r_1\) et \(r_2\) alors il existe deux réels \(\alpha_1\) et \(\alpha_2\) tel que :
        \[\forall x \in R\pd\accol{r_1,r_2}, \frac{1}{ax^2+bx+c} = \frac{\alpha_1}{x-r_1}+\frac{\alpha_2}{x-r_2}\]
        Dans ce cas, \\
        une primitive de \(x\mapsto \frac{1}{ax^2+bx+c}\) sur tout intervalle \(I\) inclus dans \(R\pd\accol{r_1,r_2}\) est :
        \[x\mapsto \alpha_1 \ln\abs{x-r_1} + \alpha_2\ln\abs{x-r_2}\]


        \item si \(g\) admet une racine réelle double \(r\) alors il existe un réel \(\alpha\) tel que :
        \[\forall x \in R\pd\accol{r}, \frac{1}{ax^2+bx+c} = \frac{\alpha}{(x-r)^2}\]
        Dans ce cas,\\
        une primitive de \(x\mapsto \frac{1}{ax^2+bx+c}\) sur tout intervalle \(I\) inclus dans \(\R\pd\accol{r}\) est :
        \[x\mapsto \frac{-\alpha}{x-r}\]


        \item Si \(g\) n'admet pas de racines réelles alors, en écrivant \(g\) sous forme canonique, on peut trouver trois réels \(\alpha,\beta \) et \(\gamma\) tel que :
        \[\forall \in \R, \frac{1}{ax^2+bx+c} = \frac{\alpha}{\paren{\frac{x+\beta}{\gamma}}^2+1}\]
        Dans ce cas,\\
        une primitive de \(x\mapsto \frac{1}{ax^2+bx+c}\) sur tout intervalle \(I\) inclus dans \(\R\) est :
        \[x\mapsto \alpha \gamma \arctan\paren{\frac{x+\beta}{\gamma}}\]
    \end{enumerate}
\end{appl}
