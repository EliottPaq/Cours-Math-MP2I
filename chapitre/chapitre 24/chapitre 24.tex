\chapter{Arithmétique des Polynômes}

\minitoc

Dans ce chapitre, \( \K\) désigne le corps \(\R\) ou \( \C\).\\~\\
On rappelle une définition et une notation vues dans le chapitre “Polynômes” :
\begin{itemize}
    \item  deux polynômes \(A\) et \(B\) sont dits associés, s’il existe un scalaire non nul\( \lambda\)  tel que \(A = \lambda B\).
    \item  \(\cal{D} (A)\) désigne l’ensemble des diviseurs du polynôme \(A\) de \(\K\croch{X}\) dans \(\K\croch{X}\) .
\end{itemize}

\section{Arithmétique dans \(\K\croch{X}\)}
\subsection{PGCD de deux polynômes}
\begin{defi}[Définition des PGCD]
    Soit \((A, B) \in \paren{\K\croch{X}}^2\) tel que \(B\neq 0_{\K\croch{X}}\) .\\
    Tout diviseur commun de \(A\) et \(B\) dans \(\K\croch{X}\) de degré maximal est dit PGCD des polynômes \(A\) et \(B\).\\~\\
    \underline{Remarques} \\
    \begin{itemize}
        \item Pour tout \(A \in \K\croch{X}\) , les PGCD de \(A\) et \(1_{\K\croch{X}}\) sont les associés du polynôme \(1_{\K\croch{X}}\).
        \item Pour tout \(B \in \K\croch{X} \pd \accol{0_{\K\croch{X}}}\)  , les PGCD de \(0_{\K\croch{X}}\)  et \(B\) sont les associés du polynôme \(B\).
        \item Pour tout \((A, B) \in \paren{\K\croch{X}}^2\) tel que \((A, B)\neq (0_{\K\croch{X}} , 0_{\K\croch{X}} )\), les PGCD de \(A\) et \(B\) sont les PGCD de \(B\) et \(A\).
    \end{itemize}
\end{defi}

\begin{prop}[Propriété importante des PGCD]
    Soit \((A, B) \in \paren{\K\croch{X}}^2\) tel que \(B\neq 0_{\K\croch{X}}\) .\\
    Si \(R\) est le reste de la division euclidienne de \(A\) par \(B\) alors
    \[\cal{D}(A) \inter \cal{D}(B) = \cal{D}(B) \inter \cal{D}(R)\]
    donc les PGCD de \(A\) et \(B\) sont les PGCD de \(B\) et de \(R\).\\
    \underline{Remarque} \\
    Plus généralement si \(A = BC + D\) avec \((A, B, C, D) \in \paren{\K\croch{X}}^4\) alors \(\cal{D}(A) \inter \cal{D}(B) = \cal{D}(B) \inter \cal{D}(D)\).
\end{prop}

\begin{defprop}[Algorithme d’Euclide]
    Soit \((A, B) \in \paren{\K\croch{X}}^2\) tel que \(B\neq 0_{\K\croch{X}}\) . \\~\\
    On pose :
    \[R_0 = A \text{ et }R_1 = B\]
    Pour tout \(i \in \Ns\) tel que \(R_i\neq 0_{\K\croch{X}}\) , on définit \(R_{i+1}\) comme suit :
    \[R_{i+1} \text{ est le reste de la division euclidienne de } R_{i-1} \text{ par } R_i.\]
    Alors :
    \begin{itemize}
        \item il existe \(n \in \N\) tel que \(R_{n+1} = 0_{\K\croch{X}}\)  et \(R_n\neq 0_{\K\croch{X}}\) 
        \item pour tout \(i \in \interventierii{0}{n}\) , les PGCD de \(R_{i-1}\) et \(R_i\) sont les PGCD de \(R_i\) et \(R_{i+1}\) .
    \end{itemize}
    En particulier, les PGCD de \(A\) et \(B\) sont les PGCD de \(R_n\) et \(R_{n+1}\) donc sont les polynômes associés à \(R_n\).
    \underline{Remarques} \\
    \begin{itemize}
        \item Les PGCD de \(A\) et \(B\) sont les polynômes associés au dernier reste non nul dans l’algorithme d’Euclide encore appelé algorithme des divisions euclidiennes successives.
        \item Parmi les polynômes associés au dernier reste non nul \(R_n\), un seul est unitaire.
    \end{itemize}
\end{defprop}

\begin{defprop}[    Caractérisation des PGCD
]
Soit \((A, B) \in \paren{\K\croch{X}}^2\) tel que \(B\neq 0_{\K\croch{X}}\) . \\~\\
L’ensemble des diviseurs communs à \(A\) et \(B\) est égal à l’ensemble des diviseurs d’un de leurs PGCD.
\end{defprop}

\begin{defprop}[Définition et caractérisation du PGCD]
    Soit \((A, B) \in \paren{\K\croch{X}}^2\) tel que \(B\neq 0_{\K\croch{X}}\) .\\~\\
    \begin{itemize}
        \item Le polynôme unitaire de plus haut degré de \(\K\croch{X}\) diviseur commun de \(A\) et \(B\) est appelé le PGCD de \(A\) et \(B\) et noté \(A \wedge B\).
        \item Le polynôme \(A \wedge B\) est caractérisé par les trois propriétés suivantes :
        \begin{enumerate}
            \item \(A \wedge B\) est un polynôme unitaire de \(\K\croch{X}\).
            \item \(A \wedge B \divise A\) et \(A \wedge B \divise B\).
            \item \(\forall  D \in \K\croch{X} , D \divise A\) et \(D \divise B \imp D \divise A \wedge B\).
        \end{enumerate}
    \end{itemize}
\end{defprop}
\begin{defprop}[Relation de Bézout]
    Soit \((A, B) \in \paren{\K\croch{X}}^2\) tel que \(B\neq 0_{\K\croch{X}}\) .\\~\\
    Il existe un couple de polynômes (\(U, V ) \in \paren{\K\croch{X}}^2\), dit couple de Bézout, tel que \(AU + BV = A \wedge B\).\\
    \underline{Remarque} \\
    Un tel couple, qui n’est pas unique, peut être déterminé par l’algorithme d’Euclide étendu selon le même principe que celui vu dans \(\Z\).
\end{defprop}
\subsection{PPCM de deux polynômes}
\begin{defprop}
    Soit \((A, B) \in \paren{\K\croch{X}}^2\) avec\( A\neq 0_{\K\croch{X}}\)  et \(B\neq 0_{\K\croch{X}}\) .
    \begin{itemize}
        \item Tout multiple commun non nul de \(A\) et \(B\) de degré minimal est dit PPCM de A\(\) et \(B\).
        \item L’unique polynôme unitaire de plus bas degré de \(\K\croch{X}\) qui est multiple commun de \(A\) et \(B\) est appelé le PPCM de \(A\) et \(B\) et noté \(A \wedge B\)
    \end{itemize}
\end{defprop}
\subsection{Couple de polynômes premiers entre eux}
Soit \((A, B, C) \in (\K\croch{X})^3\) avec \(A\neq 0_{\K\croch{X}}\) , \(B\neq 0_{\K\croch{X}}\)  et \(C\neq 0_{\K\croch{X}}\) .

\begin{defi}
    On dit que les polynômes \(A\) et \(B\) sont premiers entre eux si \(A \wedge B = 1_{\K\croch{X}}\)
\end{defi}

\begin{theo}[Théorème de Bézout]
    \(A\) et \(B\) sont premiers entre eux si, et seulement si, il existe \((U, V ) \in (\K\croch{X})^2\) tel que \(AU + BV = 1_{\K\croch{X}}\).
\end{theo}
\begin{defprop}[Lemme de Gauss]
Si \(C\) divise \(AB\) et si \(C\) est premier avec \(A\) alors \(C\) divise \(B\).
\end{defprop}
\begin{defprop}[Propriétés sur le produit]
    \begin{itemize}
        \item Si \(A\) et \(B\) sont premiers entre eux et divisent \(C\) alors \(AB\) divise \(C\).
        \item Si \(A\) et \(C\) sont premiers entre eux et si \(B\) et \(C\) sont premiers entre eux alors \(AB\) et \(C\) sont premiers entre eux.
    \end{itemize}
\end{defprop}

\subsection{PGCD d’un nombre fini de polynômes}
Soit \(n \in \N\) avec \(n \geq 2\) et \((A_1, \dots , A_n) \in (\K\croch{X})^n\) tel que l’un au moins des \(A_i\) est différent de \(0_{\K\croch{X}}\) .

\begin{defprop}[PGCD]
    On appelle PGCD des polynômes \(A_1, A_2, \dots\) et \(A_n\) et on note \(A_1 \wedge \dots \wedge A_n\) le polynôme unitaire de degré maximal de \(\K\croch{X}\) diviseur commun de \(A_1, A_2, \dots\) et \(A_n\).\\~\\
    On a alors :
    \[\cal{D}(A_1 \wedge \dots \wedge A_n) = \cal{D}(A_1) \inter \dots \inter \cal{D}(A_n)\].
\end{defprop}

\begin{defprop}[Relation de Bézout]
    Il existe un \(n\)-uplet de polynômes \((U_1, \dots , U_n) \in (\K\croch{X})^n\) tel que \(A_1U_1 + \dots + A_nU_n = A_1 \wedge \dots \wedge A_n\).
\end{defprop}

\begin{defprop}[Polynômes premiers entre eux]
Les polynômes\( A_1, A_2, \dots\) et \(A_n\) sont dits :
    \begin{itemize}
        \item premiers entre eux dans leur ensemble si \(A_1 \wedge \dots \wedge A_n = 1_{\K\croch{X}}\).
        \item premiers entre eux deux à deux si \(\forall (i, j) \in \interventierii{1}{n} , i\neq j \imp A_i \wedge A_j = 1_{\K\croch{X}}\).
    \end{itemize}
\end{defprop}

\section{Polynômes irréductibles}
\subsection{Théorème de D’Alembert-Gauss }
\begin{theo}
    Tout polynôme non constant de \( \C\croch{X}\) admet au moins une racine complexe. \\
    \underline{Remarques} \\
    \begin{itemize}
        \item Tout polynôme non constant de \(\C\croch{X}\) est donc scindé sur \(\C\).
        \item Deux polynômes de \(\C\croch{X}\) sont premiers entre eux si, et seulement si, ils n’ont pas de racine commune.
    \end{itemize}
\end{theo}

\subsection{Polynômes irréductibles de \(\K\croch{X}\)}
\begin{defi}
    Un polynôme \(P\) de \(\K\croch{X}\) est dit irréductible s’il n’est pas constant et que ses seuls diviseurs dans \(\K\croch{X}\) sont les polynômes constants et les polynômes associés à \(P\). \\
    \underline{Remarque} \\
    Un polynôme \(P\) de \( \K\croch{X}\) irréductible dans\( \K\croch{X}\) n’a donc comme seuls diviseurs, à une constante multiplicative près, que \(1_{\K\croch{X}}\) et lui-même (analogie avec la notion de nombre premier vue dans \(\Z\)).
\end{defi}
\begin{defprop}[Caractérisation des polynômes irréductibles]
    \begin{itemize}
        \item Les polynômes irréductibles de \(\C \croch{X}\) sont les polynômes de degré \(1\).
        \item Les polynômes irréductibles de \(\R \croch{X}\) sont les polynômes de degré \(1\) et les polynômes de degré \(2\) qui n’ont pas de racine réelle.
    \end{itemize}
\end{defprop}

\subsection{Décomposition en facteurs irréductibles}
\begin{theo}
    \begin{itemize}
        \item Tout polynôme non constant de \(\K\croch{X}\) admet au moins un diviseur irréductible dans \(\K\croch{X}\).
        \item Tout polynôme non constant \(P\) de \(\K\croch{X}\) peut s’écrire, de manière unique à l’ordre près des facteurs, sous la forme \[P = \lambda  (P_1)\alpha_1 (P_2)\alpha_2 \dots (P_m)\alpha_m\]
        avec
        \begin{enumerate}
            \item  \(\lambda\)  le coefficient dominant de \(P\) .
            \item \(m, \alpha_1, \alpha_2, \dots , \alpha_m\) des entiers naturels non nuls.
            \item \(P_1, P_2, \dots , P_m\) des polynômes irréductibles unitaires deux à deux distincts de \(\K\croch{X}\).
        \end{enumerate}
    \end{itemize}
\end{theo}

\begin{defprop}[Caractérisation de la divisibilité dans \(\C \croch{X}\)]
    Soit \((A, B) \in (\C \croch{X})^2\) avec \(B\neq 0_{\C\croch{X}}\).\\
    \(B\) divise \(A\) si, et seulement si, toute racine de \(B\) de multiplicité \(\beta\) est racine de \(A\) de multiplicité \(\alpha \geq \beta\).
\end{defprop}

\begin{defprop}[Racines conjuguées des polynômes de \(\R\croch{X}\)]
    Deux racines complexes conjuguées d’un polynôme de \(\R \croch{X}\) ont même multiplicité.
\end{defprop}