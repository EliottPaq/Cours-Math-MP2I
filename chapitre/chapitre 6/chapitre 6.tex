\chapter{Nombres complexes (\(2\))}

\minitoc

\section{Équations algébreiques}
\subsection{Préliminaires}
\begin{defi}[Définition d'une fonction polynomiale]
    Une fonction \(P:\C\to\C\) est dite fonction polynomiale à coefficients complexes s'il existe un entier naturel \(n\) et un \(n+1\)-uplet de nombres complexes \((b_0,b_1,\dots,b_n)\) tel que pour tout \(z\) de \(\C\), 
    \[P(z) = b_0+b_1z+\dots+b_nz^n = \sum_{k=0}^n b_k z^k\]
\end{defi}
\begin{prop}[Propriétés de factorisation]
    Soit \(P\) une fonction polynomiale à coefficients complexes et \(a\) un nombre complexe.\\
    Si \(a\) est une racine de \(P\) , autrement dit si \(P (a) = 0\), alors il existe une fonction polynomiale à coefficients complexes \(Q\) tel que, pour tout \(z\) de \(\C\), on a :
    \[P(z) = (z-a)Q(z)\]
\end{prop}

\subsection{Résolution des équations du second degré dans \(\C\)}
\begin{defprop} [cas particulier des équations du type \(z^2 = z_0\)]
    Soit \(z_0\) et \(z\) des nombres complexes de formes algébriques respectives \(x_0 + \i y_0\) et \(x+\i y\) 
    \[z^2 = z_0 \text{ six et seulement si ,} 
    \begin{cases}
        x^2-y^2 &= x_0 \\
        x^2+y^2 &= \sqrt{x_0^2 + y_0^2}\\
        2xy &= y_0 \\
    \end{cases}
    \]
\end{defprop}

\begin{defprop}[Cas général]
    soit \(a,b\) et \(c\) des nombres complexes avec \(a\) non nul. \\
    \begin{itemize}
        \item \underline{Racines} \\Les solutions de l'équations polynomiale \(az^2+bz+c=0\) d'inconnue le nombre complexe \(z\) sont : 
        \[z_1 = \frac{-b-\delta}{2a} \text{ et } z_2 = \frac{-b+\delta}{2a}\]
        où \(\delta\) est une "racine carré" de \(\Delta = b^2 -4ac\), autrement dit où \(\delta\) est un nombre complexe vérifiant : 
        \[\delta^2 = \Delta\]
        \item \underline{Somme et produit des racines (formules de Viète)} \\
        Les racines \(z_1\) et \(z_2\) de la fonction polynomiale \(P:z\mapsto az^2 + bz +c \) vérifient :
        \[z_1+z_2 = -\frac{b}{a} \text{ et } z_1z_2 = \frac{c}{a}\]

    \end{itemize}
\end{defprop}

\begin{dem}[Formule des solutions du cas général]
    soit \(a,b\) et \(c\) des nombres complexes avec \(a\) non nul. \\
    Soit \(z \in \C\)
    \begin{align*}
        az^2+bz+c &= a\paren{z^2+\frac{b}{a}z+\frac{c}{a}} \\
        &=a\paren{\paren{z+\frac{b}{2a}}^2+\frac{c}{a} - \frac{b^2}{4a^2}} \\
        &=a\paren{\paren{z+\frac{b}{2a}}^2 - \frac{b^2-4ac}{4a^2}} \\
        &=a\paren{\paren{z+\frac{b}{2a}}^2 - \frac{\Delta}{\paren{2a}^2}}  \tag*{ on pose \(\Delta = b^2-4ac\)}\\
        &=a\paren{\paren{z+\frac{b}{2a}}^2 - \paren{\frac{\delta}{2a}}^2} \tag*{ on pose \(\delta\)  comme étant la "racine carré" de \(\Delta\)}\\
        &=a\paren{z+\frac{b}{2a}-\frac{\delta}{2a}}\paren{z+\frac{b}{2a}+\frac{\delta}{2a}} \\
        &=a\paren{z-z_1}\paren{z-z_2} \text{ avec } 
        \begin{cases}
            z_1 &= \frac{-b-\delta}{2a} \\\\
            z_2 &= \frac{-b+\delta}{2a}
        \end{cases}
    \end{align*}
\end{dem}

\begin{dem}[Formule de viète]
    soit \(a,b\) et \(c\) des nombres complexes avec \(a\) non nul. \\
    Soit \(P:z\mapsto az^2+bz+c\) 
    \[P(z) = az^2+bz+c = a(z-z_1)(z-z_2) = a(z^2-(z_1+z_2)z+z_1z_2)\]
    donc par identification : 
    \[
    \begin{cases}
        b &=-a(z_1+z_2) \\
        c &= az_1z_2 
    \end{cases} \iff\begin{cases}
        -\frac{b}{a} &=z_1+z_2 \\
        \frac{c}{a} &= z_1z_2 
    \end{cases} 
    \]
\end{dem}

\subsection{Résolution des équations du type \(z^n = z_0\) dans \(\C\) avec \(n\in \Ns\)}

\begin{defi}
    Soit \(n\) un entier naturel non nul et \(z_0\) un nombre complexe. \\
    On appelle racine \(n\)- ième de \(z_0\) tout nombre complexe tel que \(z^n = z_0\)
\end{defi}

\begin{defprop}[Cas particulier où \(z_0 = 1\)]
    \begin{itemize}
        \item \underline{Racines} \\
        Il y a \(n\) racine \(n\)-ième de l'unité qui sont les nombres complexes suivants : 
        \[\omega_k = e^{\i\frac{2k\pi}{n}} \text{ avec } k \in \interventierii{0}{n-1}\] 
        \item \underline{L'ensemble des raicnes} \\
        \begin{itemize}
            \item L'ensemble des racines \(n\)-ièmes de l'unité est noté 
            \[\U_n = \accol{z\in\R \tq z^n = 1}\]
            \item Les points dont les affixes sont les racines \(n\)-ièmes de l’unité sont les sommets d’un polygone régulier à \(n\) côtés, de centre \(O\) et inscrit dans \(\U\).
        \end{itemize}
    \end{itemize}
\end{defprop}
\begin{dem}
    Soit \(z \in\C\) tel que \(z^n = 1\) \\
    \(z=0\) n'est pas solution donc \(\quantifs{\exists (r,\theta) \in \Rps\times\R}z = re^{\i \theta}\) 
    \begin{align*}
        z^n=1 &\iff r^ne^{\i \theta n }= 1e^{\i \times 0} \\
        &\iff  
            \begin{cases}
                r^n &= 1 \\
                n \theta &\equiv 0 [2\pi]
            \end{cases}\\
        &\iff \begin{cases}
                r &= 1 \\
                \theta \equiv 0 \croch{\frac{2\pi}{n}}
            \end{cases}\\
    \end{align*}
    Ainsi \(S = \U_n =\accol{e^{\i \frac{k2\pi}{n}}\tq k \in \Z}\) \\
    On note \(\fonction{f}{\Z}{\C}{k}{e^{\i \frac{k2\pi}{n}}}\) alors on sait que \(f\) est \(n\) périodique car \(\forall  k \in \Z, \begin{cases}
        k+n &\in \Z \\
        k-n &\in \Z
    \end{cases} \)
    et 
    \begin{align*}
        f(k+n) &= e^{\i \frac{2(k+n)\pi}{n}} \\
               &= e^{\i \frac{2k\pi}{n}}\times e^{\i \frac{2n\pi}{n}} \\
               &= e^{\i \frac{2k\pi}{n}}\times 1 \\
               &= f(k)
    \end{align*}
    Donc \(S = \U_n =\accol{e^{\i \frac{k2\pi}{n}}\tq k \in \interventierii{0}{n-1}}\). \\
    Montrons que \(\U_n\) contient \(n\) élément autrement dit que: 
    \[\quantifs{\forall (k,k') \in\interventierii{0}{n-1}^2;k<k'} \imp e^{\i \frac{k2\pi}{n}} \neq e^{\i \frac{k'2\pi}{n}}\]
    \underline{Par l'absurde :}\\
    Soit \(k\) et \(k'\) dans \(\interventierii{0}{n-1}\) avec \(k<k'\), supposons que \(e^{\i \frac{k2\pi}{n}} = e^{\i \frac{k'2\pi}{n}}\) \\
    alors \(\frac{k2\pi}{n} \equiv \frac{k'2\pi}{n} \croch{2\pi}\)\\
    donc il existe \(k'' \in \Ns\) tel que \(\frac{k2\pi}{n} -\frac{k'2\pi}{n} = 2 k'' \pi\) car \(k'-k >0\)\\
    Ainsi \(k'-k = nk''\) avec \(\begin{cases}
        k'-k \in \interventierii{1}{n-1} &\text{ car } 0\leq k<k'\leq n-1 \\
        nk'' \in \interventierie{n}{\pinf} & \text{ car } k''\in \Ns
    \end{cases}\) \\
    Ce qui est absurde et prouve que \(e^{\i \frac{k2\pi}{n}} \neq e^{\i \frac{k'2\pi}{n}}\)\\
    \conclusion \\
    Il y as exactement \(n\) racine \(n\)-ièmes de l'unité qui sont les \( \omega_k = e^{\i \frac{k2\pi}{n}}\) pour \(k\in \interventierii{0}{n-1}\)
\end{dem}


\begin{defprop}[Cas général]
    Il y a \(n\) racines \(n\)- ièmes pour le nombre complexe non nul \(z_0\) de forme trigonométrique \(z_0 = r_0e^{\i\theta_0}\) qui sont les nombres complexes suivants :
    \[\sqrt[n]{r_0}e^{\i \paren{\frac{\theta_0}{n}+\frac{2k\pi}{n}}} \text{ avec }k \in \interventierii{0}{n-1}\]
\end{defprop}

\begin{ex}
\[\U_3 = \accol{1,\exp\paren{\frac{2\i \pi}{3}},\exp \paren{\frac{4 \i \pi}{3}}}\]

\[\U_4 = \accol{1,\exp\paren{\frac{2\i \pi}{4}},\exp \paren{\frac{4 \i \pi}{4}},\exp \paren{\frac{6 \i \pi}{4}}} = \accol{1,\i,-1,-\i}\]

\[\U_4 = \accol{1,\exp\paren{\frac{2\i \pi}{5}},\exp \paren{\frac{4 \i \pi}{5}},\exp \paren{\frac{6 \i \pi}{5}},\exp \paren{\frac{8\i \pi}{5}}} \]

\end{ex}

\section{Exponentielle complexe}
\begin{defi}
Pour tout nombre complexe \(z\), on appelle exponentielle de \(z\) le nombre complexe noté \(e^z\) le nombre complexe \(e^z\) défini par : 
\[e^z = e^{\Reel{z}}e^{\i \Ima{z}}\]
dont le module est \(\abs{e^z} = e^{\Reel{z}}\) et les arguments vérrfient \(\arg(e^z)\equiv \Ima{z} [2\pi]\)
\end{defi}

\begin{prop}
    Soit un couple de nombres complexe \((z,z')\)
    \begin{itemize}
        \item on as l'égalité suivante : 
        \[e^{z+z'} = e^ze^{z'}\]
        on en déduit les propriétés suivantes :
        \begin{itemize}
            \item \(\frac{1}{e^z} = e^{-z}\)
            \item pour tout entier relatif \(n\), on a: \(e^{nz} = \paren{e^z}^n\)
        \end{itemize}
        \item \(e^z = e^{z'}\) si et seulement si, \(z-z' \in 2\i\pi \Z\) en notant \(2\i \pi \Z =\accol{2\i k \pi \tq k\in \Z}\)
    \end{itemize}
\end{prop}
\begin{defprop}[Résolution de l'équations \(e^z = a\) avec \(a\) un nombre complexe]
    Soit \(a\) un nombre complexe. \\
    \begin{itemize}
        \item Si \(a\) est nul alors l'équation \(e^z = a\) n'a pas de solution dans \(\C\)
        \item Si \(a\) est non nul alors l'équation \(e^z = a\) possède une infinité de solutions dans \(\C\) qui sont les nombres complexes 
        \[z= \ln(z)+\i \theta\]
        avec \(r\) le module de \(a\) et \(\theta\) un argument de \(a\).
    \end{itemize}
\end{defprop}

\section{Interprétations géométriques}
\begin{defprop}[Module et arguments de \(\frac{z'-\omega}{z-\omega}\)]
    Soit \(\omega,z \) et \(z'\) des nombres complexes tel que \(\omega \neq z\) et \(\omega \neq z'\) de points images notés \(\Omega,M\) et \(M'\). \\
    Alors : 
    \[\abs{\frac{z'-\omega}{z-\omega}} = \frac{\Omega M'}{\Omega M} \text{ et } \arg\paren{\frac{z'-\omega}{z-\omega}} = \paren{\overrightarrow{\Omega M},\overrightarrow{\Omega M'}}[2\pi]\]

\end{defprop}
\begin{defprop}[Traduction de l’alignement et l’orthogonalité]
     Soit \(\Omega,M \) et \(M'\) trois points du plan tels que \(\Omega \neq M\) et \(\Omega \neq M'\) d'affixes respectivement notées \(\omega,z\) et \(z'\)
     \begin{itemize}
        \item Les points  \(\Omega,M \) et \(M'\) sont alignés si, et seulement si,  \(\frac{z'-\omega}{z-\omega}\) est un réel
        \item Les droites \(\Omega M\) et \(\Omega M'\) sont orthogonales si, et seulement si, \(\frac{z'-\omega}{z-\omega}\) est un imaginaire pur.
     \end{itemize}
\end{defprop}
\begin{defprop}[Ecriture complexe de transformations du plan vues au collège]
    Dans ce paragraphe, \(M\) et \(M'\) sont deux points du plan complexe d’affixes respectives \(z\) et \(z'\).
    \begin{itemize}
        \item \underline{Translation} \\
        Soit \(b\) un nombre complexe. \\
        \(M'\) est l'image par \(M\) par la translation de vecteur d'affixe \(b\) si, et seulement si \[z' = z+b\]
        \item \underline{Homothétie} \\
        Soit \(\alpha\) un nombre réel et \(\Omega\) un point du plan d'affixe \(\omega\). \\
        \(M'\) est l'image par \(M\) par l'Homothétie de centre \(\Omega\) et de rapport \(\alpha\)  si, et seulement si \[z'-\omega = \alpha(z-\omega)\]
        \item \underline{Rotation} \\
        Soit \(\theta\) un nombre réel et \(\Omega\) un point du plan d'affixe \(\omega\). \\
        \(M'\) est l'image par \(M\) par la rotation de centre \(\Omega\) et d'angle \(\theta\)  si, et seulement si \[z'-\omega = e^{\i \theta}(z-\omega)\]
    \end{itemize}
\end{defprop}

\begin{defprop}[Applicaitons \(z\mapsto az+b\) avec \((a,b) \in \Cs\times\C\)]
    Soit  \((a,b) \in \Cs\times\C\).L'application \(f\) de \(\C\) dans \(\C\) définie par 
    \[f(z) = az+b\]
    est dite similitude directe. \\
    \underline{Interprétation géométrique :}
    Pour tout \(z\in \C\), on note \(M\) le point d'affixe \(z\) et \(M'\) le point d'affixe \(z' = f(z)\)
    \begin{itemize}
        \item \underline{Cas où \(a=1\)} \\
        On a alors l'équivalence suivante : \(z' = f(z)\) so et seulement si, \(z'-z = b\) \\
        L'application \(f\) est donc la translation de vecteur d'affixe \(b\).
        \item\underline{Cas où \(a\neq1\)}\\ 
        \(f\) admet alors un point fixe \(\omega\) donné par \(\omega = \frac{b}{1-a} \) dont le point image image est noté \(\Omega\) \\
        On en déduit les équivalences suivantes :  \\
        \begin{align*}
            z' = f(z) &\text{ si, et seulement si, } z'-\omega = a'(z-\omega) \\
                      &\text{ si, et seulement si, } z'-\omega  = \abs{a}\paren{e^{\i \arg(a)}(z-\omega)} \\
                      &\text{ si, et seulement si, } z'-\omega  = e^{\i \arg(a)}\paren{\abs{a}(z-\omega)} \\
        \end{align*}
    \end{itemize}
    L'application \(f\) est donc la composée commutative : 
    \begin{itemize}
        \item de l'Homothétie de centre \(\Omega\) et de rapport \(\abs{a}\)
        \item de la rotation de centre \(\Omega\) et d'angle \(\arg(a)\)
    \end{itemize}
\end{defprop}

\begin{defprop}[Applicaitons \(z\mapsto a\conj{z}+b\) avec \((a,b) \in \Cs\times\C\)]
    Soit  \((a,b) \in \Cs\times\C\). \\
    L'application \(g\) de \(\C\) dans \(\C\) définie par 
    \[g(z) = a\conj{z}+b\]
    est dite similitude indirect. Elle peut s'écrire sous la forme de la composée non commutative. 
    \[g = f \circ s\]
    avec :
    \begin{itemize}
        \item \(s:z\mapsto \conj{z}\) qui est la symétrie axiale d'axe de la droite des réels
        \item \(f:z\mapsto az+b\) qui est une similitude directe.
    \end{itemize}
\end{defprop}