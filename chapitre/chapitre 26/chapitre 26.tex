\chapter{Analyse Asymptotique (2)}

\minitoc

Dans ce chapitre, \(\K\) désigne le corps \(\R\) ou \(\C\).

\section{Relations de comparaison pour les fonctions}
    Se reporter au chapitre \hyperref[chap:analyse-asymptotique1]{"Analyse asymptotique \((1)\)"}
\section{Relation de comparaison pour les suites}
    Sauf mention contraire, dans cette partie, \(\paren{u_n}\) et \(\paren{v_n}\) désignent deux suites à valeurs dans \(\K\).
\subsection{Définitions}
\begin{defprop}{Domination}
    On dit que \(\paren{u_n}\) est dominée par \(\paren{v_n}\) s’il existe un entier naturel \(n_0\) et une suite \(\paren{w_n}\) bornée tel que
    \[\forall n \in \interventierie{n_0}{\pinf} , u_n = v_nw_n\]
    On note alors \(u_n = \O{v_n}\).
\end{defprop}
\begin{defprop}[Négligeabilité]
    On dit que \(\paren{u_n}\) est négligeable devant \(\paren{v_n}\) s’il existe un entier naturel \(n_0\) et une suite \(\paren{w_n}\) de limite nulle tel que
    \[\forall n \in  \interventierie{n_0}{\pinf}   , u_n = v_nw_n\]
    On note alors un = \(\o{v_n}\).  
\end{defprop}
\begin{defprop}[Equivalence]
    On dit que \(\paren{u_n}\) est équivalente à \(\paren{v_n}\) s’il existe un entier naturel \(n_0\) et une suite \(\paren{w_n}\) de limite égale à \(1\) tel que
    \[\forall n \in  \interventierie{n_0}{\pinf}, u_n = v_nw_n\]
    On note alors \(u_n \sim v_n\).
\end{defprop}

\subsection{Caractérisations pratiques}
\begin{defprop}
    Dans le cas où \(\paren{v_n}\) ne s’annule pas à partir d’un certain rang, on a les équivalences suivantes :
    \begin{enumerate}
        \item \(u_n = \O{v_n}\) si, et seulement si, la suite \(\paren{\frac{u_n}{v_n}}\) est bornée.
        \item \(u_n = \o{v_n}\) si, et seulement si, la suite \(\paren{\frac{u_n}{v_n}}\) a pour limite \(0\).
        \item  \(u_n \sim v_n\) si, et seulement si, la suite \(\paren{\frac{u_n}{v_n}}\) a pour limite \(1\).
    \end{enumerate}
\end{defprop}

\subsection{Lien entre les relations de comparaison}
\begin{defprop}
    \begin{enumerate}
        \item \(u_n = \o{v_n} \imp  u_n = \O{v_n}\) ;
        \item \(u_n \sim v_n \imp  u_n = \O{v_n}\) ;
        \item \(u_n \sim v_n \iff u_n = v_n + \o{v_n}\).
    \end{enumerate}
\end{defprop}

\subsection{Croissances comparées de suites usuelles}
\begin{defprop}
    Pour tous réels strictement positifs \(\alpha\), \(\beta\) et \(\gamma\) , on a :
    \begin{enumerate}
        \item \(\paren{\ln(n)}^{\beta} = \o{n^{\alpha}}\)
        \item \(n^{\alpha} = \o{e^{\gamma n}}\)
        \item \(e^{\gamma n} = \o{n!}\) .
        \item \(n^{\alpha} = \o{n^{\beta}}\text{ dans le cas }\alpha < \beta\)
    \end{enumerate}
\end{defprop}

\subsection{Règles usuelles de manipulation des relations de comparaison}
\begin{defprop}[Cas des \(\mathscr{O}\) (et des \(o\))]
    \begin{enumerate}
        \item Si \(u_n = \O{v_n}\) et \(\lambda  \in  \Ks\) alors \(u_n = \O{\lambda v_n}\) et \(\lambda u_n = \O{v_n}\).
        \item Si \(u_n = \O{v_n}\) et \(w_n = \O{v_n}\) alors \(u_n + w_n = \O{v_n}\).
        \item Si \(u_n = \O{v_n}\) et \(v_n = \O{w_n}\) alors \(u_n = \O{w_n}\).
        \item Si \(u_n = \O{v_n}\) alors \(u_nw_n = \O{v_nw_n}\).
        \item Si \(u_n = \O{v_n}\) et \(w_n = \O{x_n}\) alors \(u_nw_n = \O{v_nx_n}\).
        \item Si \(u_n = \O{v_n}\) et \(\phi : \N \to \N\) strictement croissante alors \(u_{\phi(n)} = \O{v_{\phi(n)}}\).
    \end{enumerate}
    \underline{Remarques} \\
    \begin{itemize}
        \item Dans tout ce qui précède, on peut remplacer \(\mathscr{O}\) par \(o\).
        \item JAMAIS de "composition des \(\mathscr{O}\) (ou des \(o\)) à gauche" par une fonction sans preuve directe.
    \end{itemize}
\end{defprop}

\begin{defprop}[Cas des équivalents]
    \begin{enumerate}
        \item Si \(u_n \sim v_n\) alors \(v_n \sim u_n\).
        \item Si \(u_n \sim v_n\) et \(v_n \sim w_n\) alors \(u_n \sim w_n\).
        \item Si \(u_n = \O{\paren{v_n}}\) et \(v_n \sim w_n\) alors \(u_n = \O{w_n}\).
        \item Si \(u_n = \o{v_n}\) et \(v_n \sim w_n\) alors \(u_n = \o{w_n}\)
        \item Si \(u_n \sim v_n\) alors \(u_nw_n \sim v_nw_n\) .
        \item Si \(u_n \sim v_n\) et \(w_n \sim x_n\) alors \(u_nw_n \sim v_nx_n\) .
        \item Si \(u_n \sim v_n\) avec \(\paren{u_n}\) et \(\paren{v_n}\) strictement positives à partir d’un certain rang alors \(u^{\beta}_n \sim v^{\beta}_n\) pour tout réel \(\beta\).
        \item Si \(u_n \sim v_n\) avec \(\paren{u_n}\) et \(\paren{v_n}\) qui ne s’annulent pas à partir d’un certain rang alors \(\frac{1}{u_n}\sim \frac{1}{v_n}\)
        \item Si \(u_n \sim v_n\) et \(\phi : \N \to \N\) strictement croissante alors \(u_{\phi(n)} \sim v_{\phi(n)}\).
    \end{enumerate}
    JAMAIS de "composition à gauche" par une fonction ni de somme d’équivalents sans preuve directe.
\end{defprop}

\subsection{Obtention et utilisation des équivalents}
\begin{defprop}[Obtention d’un équivalent par encadrement]

    Si \(\paren{u_n}\) , \(\paren{v_n}\) et \(\paren{w_n}\) sont à valeurs réelles et vérifient \(v_n \leq u_n \leq w_n\) à partir d’un certain rang avec \(v_n \sim w_n\) alors \(u_n \sim v_n\).
\end{defprop}
\begin{defprop}[Limite, signe et équivalent]
    \begin{itemize}
        \item Si \(u_n \sim v_n\) alors \(\paren{u_n}\) et \(\paren{v_n}\) ont même "comportement" c’est-à-dire que :
        \begin{itemize}
            \item \(\paren{u_n}\) a pour limite \(l\) si, et seulement si, \(\paren{v_n}\) a pour limite \(l\).
            \item \(\paren{u_n}\) n’a pas de limite si, et seulement si, \(\paren{v_n}\) n’a pas de limite.
        \end{itemize}
        \item Si \(u_n \sim v_n\) alors \(u_n\) et \(v_n\) ont le même signe à partir d’un certain rang.
    \end{itemize}
\end{defprop}
\subsection{Formule de Stirling (à connaître)}
\begin{defprop}
    Un équivalent de la suite \((n!)\) est donné par la formule de Stirling suivante :
    \[n! \sim \sqrt{2\pi n} \paren{\frac{n}{e}}^n\]
    \underline{Remarques}\\
    \begin{itemize}
        \item La démonstration de ce résultat de cours n’est pas exigible mais on en verra une dans le TD sur les séries numériques.
        \item On peut déduire de cet équivalent un développement asymptotique de la suite \((\ln n!)\) qui est :
        \[\ln n! = n \ln (n) - n + \frac{1}{2} \ln \paren{2 \pi n} + \o{1}.\]
    \end{itemize}
\end{defprop}

\subsection{Utilisation des développements limités et équivalents usuels}
\begin{defprop}
    Une connaissance parfaite des développements limités et équivalents usuels vus pour les fonctions dans le chapitre \hyperref[chap:analyse-asymptotique1]{"Analyse asymptotique (1)"} est nécessaire pour traiter les problèmes d’analyse asymptotique sur les suites, étudier la nature des séries numériques en MP2I ainsi que les différents types de convergence des séries de fonctions et les problèmes d’intégrabilité des fonctions sur un intervalle en MPI.
\end{defprop}