\chapter{trigonométrie (Rappels et compléments)}

\minitoc

Dans ce chapitre, on rappelle ce qui a été vu en trigonométrie au lycée et on complète avec les formules
d’addition et de duplication ainsi que l’étude de la fonction tangente.

\section{Cercle trigonométrique}

On se place dans le plan muni d'un repère orthonormé \(\paren{O,\vec{i},\vec{j}}\)

\begin{defi}[Cercle trigonométrique]

	On appelle cercle trigonométrique le cercle de centre \(O\) et de rayon \(1\)

\end{defi}

\begin{prop}[enroulement de la droite des réels sur le cercle trigonométrique]
	Soit \(M\) un point du plan. \\
	Le point \(M\) appartient au cercle trigonométrique si, et seulement si, il existe un réel \(t\) tel que les coordonnées de \(M\) dans le repère orthonormé \(\paren{O,\vec{i},\vec{j}}\) sont \(\paren{\cos t ; \sin t}\)
\end{prop}

\subsection{Relation de congruence modulo \(2\pi\) sur \(\R\)}

\begin{defi}
	Deux réels \(a\) et \(b\) sont dits congrus modulo \(2\pi\) s'il existe un entier relatif \(k\) tel que \(a-b = 2k\pi\)
	\underline{Notation} : \(a \equiv b \croch{2 \pi} \)
\end{defi}

\begin{defprop}
	On dit que la relation \(\equiv\) est une relation d'équivalence sur \(\R\) car elle vérifie les propriétés suivantes :
	\begin{enumerate}
		\item Pour tout réel x, on a : \(x \equiv x \croch{2 \pi}\). \hfill (réfléxivité)
		\item Pour tout couple de réels \(\paren{x,y}\) tel que \( x \equiv y \croch{2 \pi} \), on a :\( y \equiv x \croch{2 \pi} \) \hfill (symétrie)
		\item Pour tout triplet de réels \(\paren{x,y,z}\) tel que \(x \equiv y \croch{2 \pi} \) et \( y \equiv z \croch{2 \pi} \), on a : \( x \equiv z \croch{2 \pi} \) \hfill (transitivité)
	\end{enumerate}
\end{defprop}



\section{Cosinus et sinus}
\subsection{Formules et valeur remarquables}

\begin{formu}[Formule de base]
	Pour tout réel \(t\), on a :
	\begin{enumerate}
		\item \( \cos\paren{\pi - t} = -\cos t \) et \( \sin\paren{\pi - t} = \sin t \) \\
		\item \( \cos\paren{\pi + t} = -\cos t \) et \( \sin\paren{\pi + t} = -\sin t \) \\
		\item \( \cos\paren{\frac{\pi}{2} - t} = \sin t \) et \( \sin\paren{\frac{\pi}{2} - t} = \cos t \) \\
		\item \( \cos\paren{\frac{\pi}{2} + t} = -\sin t \) et \( \sin\paren{\frac{\pi}{2} + t} = \cos t \) \\
	\end{enumerate}
	\begin{tabular}{|c|c|c|c|c|c|}

		\hline
		\(t\)       & \(0\) & \(\frac{\pi}{6}\)      & \(\frac{\pi}{4}\)      & \(\frac{\pi}{3}\)       & \(\frac{\pi}{2}\) \\
		\hline
		\(\cos t \) & \(1\) & \(\frac{\sqrt{3}}{2}\) & \(\frac{\sqrt{2}}{2}\) & \(\frac{1}{2}\)        & \(0\)             \\
		\hline
		\(\sin t \) & \(0\) & \(\frac{1}{2}\)        & \(\frac{\sqrt{2}}{2}\) & \(\frac{\sqrt{3}}{2}\) & \(1\)             \\
		\hline
	\end{tabular}
\end{formu}

\begin{rem}
	Soient \(a\) et \(b\) des réels :
	\begin{itemize}
		\item
		      \(
		      \begin{aligned}
			      \cos a = \cos b
			      \iff\
			       & \left\{
			      \begin{aligned}
				      a & \equiv b \croch{2\pi}  \\
				        & \text{ou}              \\
				      a & \equiv -b \croch{2\pi}
			      \end{aligned}
			      \right.
			      \iff\
			       & \left\{
			      \begin{aligned}
				      \quantifs{\exists k \in \Z}  & a = b +2 k \pi   \\
				                                   & \text{ou}        \\
				      \quantifs{\exists k' \in \Z} & a = -b +2 k' \pi
			      \end{aligned}
			      \right.
		      \end{aligned}
		      \)\\
		      \item\(
		      \begin{aligned}
			      \sin a = \sin b
			      \iff\
			       & \left\{
			      \begin{aligned}
				      a & \equiv b \croch{2\pi}     \\
				        & \text{ou}                 \\
				      a & \equiv \pi-b \croch{2\pi}
			      \end{aligned}
			      \right.
			      \iff\
			       & \left\{
			      \begin{aligned}
				      \quantifs{\exists k \in \Z}  & a = b +2 k \pi      \\
				                                   & \text{ou}           \\
				      \quantifs{\exists k' \in \Z} & a = \pi-b +2 k' \pi
			      \end{aligned}
			      \right.
		      \end{aligned}
		      \)
	\end{itemize}

\end{rem}

\begin{formu} [Formule d'addition]
	Pour tout couple de réels \(\paren{a,b}\) on a :
	\begin{enumerate}
		\item \( \cos\paren{a+b} = \cos\paren{a} \cos\paren{b} - \sin\paren{a} \sin\paren{b} \) \\
		\item \( \cos\paren{a-b} = \cos\paren{a} \cos\paren{b} + \sin\paren{a} \sin\paren{b} \)\\
		\item \( \sin\paren{a+b} = \sin\paren{a} \cos\paren{b} + \cos\paren{a} \sin\paren{b} \) \\
		\item \( \sin\paren{a-b} = \sin\paren{a} \cos\paren{b} - \cos\paren{a} \sin\paren{b} \)\\
	\end{enumerate}
\end{formu}

\begin{formu}[Formule de simpson]
	Pour tout couple de réels \(\paren{a,b}\) on a :
	\begin{enumerate}
		\item \( \sin\paren{a+b} + \sin\paren{a-b} = 2\sin\paren{a} \cos\paren{b} \iff \frac{1}{2}\paren{\sin\paren{a+b} + \sin\paren{a-b}} = \sin\paren{a} \cos\paren{b}\) \\
		\item \( \cos\paren{a+b} + \cos\paren{a-b} = 2\cos\paren{a} \cos\paren{b} \iff \frac{1}{2}\paren{\cos\paren{a+b} + \cos\paren{a-b}} = \cos\paren{a} \cos\paren{b}\)

	\end{enumerate}

\end{formu}

\begin{appl}
	Calcul : \[\int_{0}^{\pi} \sin\paren{x} \cos\paren{3x} dx = \int_{0}^{\pi} \frac{1}{2} \paren{\sin\paren{4x}+\sin(2x)}dx = 0\]
\end{appl}

\begin{formu}[Formule de duplication]
	Pour tout réel \(a\), on a :
	\begin{enumerate}
		\item \(\cos\paren{2a} = \cos^2\paren{a} - \sin^2\paren{a} = 2\cos^2(a)-1 = 1-\sin^2(a) \)
		\item \(\sin(2a) = 2\cos(a)\sin(a) \)
	\end{enumerate}
\end{formu}

\begin{prop}[Sinus et Cosinus]
	\begin{itemize}
		\item La fonction \(\cos\) est définie sur \(\R\), paire et périodique de période \(2\pi\). Elle est dérivable sur \(\R\) et sa dérivée vérifie \(\cos' = -\sin\)
		\item La fonction \(\sin\) est définie sur \(\R\), impaire et périodique de période \(2\pi\). Elle est dérivable sur \(\R\) et sa dérivée vérifie \(\sin' = \cos\)
	\end{itemize}
\end{prop}

\begin{prop}[Inégalité remarquable]
	Pour tout réel \(t\), on a : \(\abs{\sin(t)} \leq \abs{t}\)
\end{prop}

\section{La fonction tangente}
\begin{defi}
	La fonction \(\frac{\sin}{\cos} \) est appelée la fonction tangente et notée \(\tan\)
\end{defi}

\begin{prop}
	La fonction \(\tan\) est définie sur \(\R\backslash\accol{\frac{\pi}{2}+k\pi\tq k\in \Z}\), impaire et périodique de période \(\pi\). Elle est dérivable sur \(\R\)\(\R\backslash\accol{\frac{\pi}{2}+k\pi\tq k\in \Z}\) et sa dérivée vérifie \(\tan' = 1+\tan = \frac{1}{tan^2}\)
\end{prop}

\begin{formu}
	Pour tout réel \(t\), on a :
	\begin{enumerate}
        \item \(tan(\pi-t) = -\tan(t)\)
        \item \(tan(\pi+t) = \tan(t) \)
        \item \begin{tabular}{|c|c|c|c|c|c|}

		\hline
		\(t\)       & \(0\) & \(\frac{\pi}{6}\)      & \(\frac{\pi}{4}\)      & \(\frac{\pi}{3}\)       & \(\frac{\pi}{2}\) \\
		\hline
		\(\tan t \) & \(0\) & \(\frac{1}{\sqrt{3}}\) & \(1\) & \(\sqrt{3}\)       & NULL             \\
		\hline
	\end{tabular}
    \end{enumerate}
\end{formu}

\begin{formu}[addition et duplication]
    Pour tout couple de réels \(\paren{a,b}\) n'appartenant pas à l'ensemble \(\accol{\frac{\pi}{2}+k\pi\tq k\in \Z}\), on a :
    \begin{enumerate}
        \item Si \(a+b\) n'appartient pas à l'ensemble \(\accol{\frac{\pi}{2}+k\pi\tq k\in \Z}\) alors \(\tan(a+b) = \frac{\tan(a)+\tan(b)}{1-\tan(a) \tan(b)}\)
        \item Si \(a-b\) n'appartient pas à l'ensemble \(\accol{\frac{\pi}{2}+k\pi\tq k\in \Z}\) alors \(\tan(a-b) = \frac{\tan(a)-\tan(b)}{1+\tan(a) \tan(b)}\)
        \item Si \(2a\)  n'appartient pas à l'ensemble \(\accol{\frac{\pi}{2}+k\pi\tq k\in \Z}\) alors \(\tan(2a) = \frac{2\tan(a)}{1-\tan^2(a)} \)
    \end{enumerate}
\end{formu}

\begin{exoex}
Soit \(t\) réel n'appartenant pas à \(\accol{\frac{\pi}{4}+k\frac{\pi}{2}\tq k\in \Z}\) : 
    \begin{align*}
        \sin(t) &= 2\sin\paren{\frac{t}{2}}\cos\paren{\frac{t}{2}} \\
        &= \frac{2\sin\paren{\frac{t}{2}}}{\cos\paren{\frac{t}{2}}}\cos^2\paren{\frac{t}{2}}\\
        &= \frac{1}{1+\tan^2\paren{\frac{t}{2}}}\times 2 \tan\paren{\frac{t}{2}} \\
        &=\frac{2 \tan\paren{\frac{t}{2}}}{1+\tan^2\paren{\frac{t}{2}}} 
    \end{align*}
\end{exoex}