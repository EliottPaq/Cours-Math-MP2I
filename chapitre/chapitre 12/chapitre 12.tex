\chapter{Limite et continuité}

\minitoc
\begin{nota}
    Dans ce chapitre, \(I\) et \(J\) désignent des intervalles de \(\R\), non vides et non réduits à un point.
\end{nota}
\section{étude locale des fonctions à valeurs réelles}
\subsection{Limite en un point \(a\) de \(\Rb\) appartenant à \(I\) ou extrémité de \(I\)}

\begin{defi}
    Soit \(f\) une fonction définie sur \(I\) à  valeur dans \(\R\)
    \begin{itemize}
        \item Cas où \(a\) est un réel, appartenant à \(I\) ou extrémité de \(I\).\\
        On dit que \(f\) admet pour limite \(l\) en \(a\) si : \(\quantifs{\forall \epsilon \in \Rps;\exists\delta \in \Rps;\forall x \in I} \abs{x-a}\leq \delta \imp \abs{f(x)-l}\leq \epsilon\)
        \item cas où \(a = \pinf\) est extrémité de \(I\)\\
            On dit que \(f\) admet pour limite \(l\) en \(\pinf\) si : \(\quantifs{\forall \epsilon \in \Rps;\exists B \in \Rps;\forall x \in I} x\geq B \imp \abs{f(x)-l}\leq \epsilon\)
        \item cas où \(a = \minf\) est extrémité de \(I\)\\
         On dit que \(f\) admet pour limite \(l\) en \(\minf\) si : \(\quantifs{\forall \epsilon \in \Rps;\exists B \in \Rms;\forall x \in I} x\leq B \imp \abs{f(x)-l}\leq \epsilon\)
    \end{itemize}
\end{defi}

\begin{defi}[Définitions d’une limite infinie]
    \begin{itemize}
        \item cas où \(a\) est un réel, appartenant à \(I\) ou extrémité de \(I\).\\~\\
            On dit que \(f\) admet pour limite \(\pinf \) en \(a\) si : \(\quantifs{\forall A \in \Rps;\exists\delta \in \Rps;\forall x \in I} \abs{x-a}\leq \delta \imp f(x)\geq A\)\\
            On dit que \(f\) admet pour limite \(\minf \) en \(a\) si : \(\quantifs{\forall A \in \Rms;\exists\delta \in \Rps;\forall x \in I} \abs{x-a}\leq \delta \imp f(x)\leq A\)
        \item cas où \(a = \pinf\) est extrémité de \(I\)\\~\\
            On dit que \(f\) admet pour limite \(\pinf \) en \(\pinf\) si : \(\quantifs{\forall A \in \Rps;\exists B \in \Rps;\forall x \in I} x\geq B \imp f(x)\geq A\)\\
            On dit que \(f\) admet pour limite \(\minf \) en \(\pinf\) si : \(\quantifs{\forall A \in \Rms;\exists B \in \Rps;\forall x \in I} x\geq B \imp f(x)\leq A\)
        \item cas où \(a = \minf\) est extrémité de \(I\)\\~\\
            On dit que \(f\) admet pour limite \(\pinf \) en \(\minf\) si : \(\quantifs{\forall A \in \Rps;\exists B \in \Rms;\forall x \in I} x\leq B \imp f(x)\geq A\)\\
            On dit que \(f\) admet pour limite \(\minf \) en \(\minf\) si : \(\quantifs{\forall A \in \Rms;\exists B \in \Rms;\forall x \in I} x\leq B \imp f(x)\leq A\)
    \end{itemize}
\end{defi}

\begin{defprop}[Unicité]
    Si \(f\) admet une limite \(l\) en \(a\) alors celle-ci est unique et on note \(f(x)\underset{x\to a}{\to} l\) ou \(\lim_{x\to a} f(x) = l\).
\end{defprop}
\begin{defprop}[Existence d’une limite en un point où la fonction est définie]
    Si \(f\) est définie en \(a\) et possède une limite en \(a\) alors \(\lim_{x\to a} f(x) = f(a)\).
\end{defprop}

\begin{defprop}[condition nécessaire d'existence de limite]
    Si \(f\) possède une limite finie en \(a\) alors \(f\) est bornée au voisinage de \(a\).
\end{defprop}

\subsection{Limite à gauche et à droite en un réel appartenant à \(I\) ou extrémité de \(I\).}

\begin{nota}
    Soit \(f\) une fonction définie sur \(I\), à valeurs dans \(\R\).
\end{nota}
\begin{defi}
    Soit \(a\) un point de \(\R\), appartenant à \(I\) ou extrémité de \(I\).
    \begin{enumerate}
        \item On dit que \(f\) admet une limite à gauche en \(a\) si la restriction \(f_{| I \inter \intervee{\minf}{a}}\) admet une limite en \(a\)\\
        Dans ce cas, on note \(\lim_{x\to a^-}f(x)\)  ou \(\lim_{\substack{x \to a \\ x < a}} f(x)\) la limite obtenue.
        \item On dit que \(f\) admet une limite à droite en \(a\) si la restriction \(f_{| I \inter \intervee{a}{\pinf}}\) admet une limite en \(a\)\\
        Dans ce cas, on note \(\lim_{x\to a^+}f(x)\)  ou \(\lim_{\substack{x \to a \\ x > a}} f(x)\) la limite obtenue.
    \end{enumerate}
\end{defi}

\begin{defprop}[Condition nécessaire et suffisante d’existence de limite]
    Soit \(a\) un point de \(\R\) appartenant à \(I\) mais pas extrémité de \(I\)\\~\\
    \(f\) admet une limite en \(a\) \ssi les trois conditions suivantes sont réunies :
    \begin{enumerate}
        \item \(f\) a une limite à gauche en \(a\).
        \item \(f\) a une limite à droite en \(a\)
        \item \(\lim_{x\to a^-} f(x) = \lim_{x \to a^+} f(x) = f(a)\)
    \end{enumerate}
\end{defprop}
\subsection{Caractérisation séquentielle de la limite}
\begin{theo}
    Soit \(f\) une fonction définie sur \(I\), à valeurs dans \(\R\)\\
    Soit \(a\) un point de \(\Rb\), appartenant à \(I\) ou extrémité de \(I\), et \(l\) un point de \(\Rb\)\\~\\
    \(f\) admet une limite \(l\) en \(a\) \ssi pour toute suite \((x_n)\) d'éléments de \(I\) qui admet pour limite \(a\), la suite réelle \((f(x_n))\) admet pour limite \(l\)
\end{theo}

\subsection{Opérations sur les limites}
\begin{defprop}
    Soit \(a\) un point de \(\Rb\), appartenant à \(I\) ou extrémité de \(I\).
    Soit \(f\) et \(g\) deux fonctions définies sur \(I\) et à valeurs réelles et \(\lambda\) un réel
    \begin{enumerate}
        \item \underline{Addition} \\~\\
        \begin{enumerate}
            \item Si \(f(x) \underset{x\to a}{\to} l\) avec \(l \in \R\) et \(g \underset{x\to a}{\to} l'\) avec \(l' \in \R\) alors \((f+g)(x) \underset{x\to a}{\to} l + l'\)
            \item Si \(f(x) \underset{x\to a}{\to} \pinf\)  et \(g(x)\underset{x\to a}{\to} l'\) avec \(l' \in \R \union \accol{\pinf}\) alors \((f+g)(x) \underset{x\to a}{\to} \pinf\)
            \item     Si \(f(x) \underset{x\to a}{\to}\minf\)  et \(g(x)\underset{x\to a}{\to} l'\) avec \(l' \in \R \union \accol{\minf}\) alors \((f+g)(x)\underset{x\to a}{\to} \minf\)
        \end{enumerate}
        \item \underline{Multiplication par un réel}.\\~\\
            \begin{enumerate}
                \item Si \(f(x) \underset{x\to a}{\to} l \) avec \(l \in \R\) alors \(\lambda f(x) \underset{x\to a}{\to} \lambda l\)
                \item Si \(f(x) \underset{x\to a}{\to}\pinf\) alors \(\lambda f(x)\underset{x\to a}{\to}\begin{cases}
                    \pinf &\text{ si } \lambda>0 \\
                    0 &\text{ si } \lambda=0 \\
                    \minf &\text{ si } \lambda<0 \\
                \end{cases}\)
                \item Si \(f(x) \underset{x\to a}{\to} \minf\) alors \(\lambda f(x) \underset{x\to a}{\to} \begin{cases}
                    \pinf &\text{ si } \lambda<0 \\
                    0 &\text{ si } \lambda=0 \\
                    \minf &\text{ si } \lambda>0 \\
                \end{cases}\)
            \end{enumerate}
        \item \underline{Produit} 
            \begin{enumerate}
                \item Si \(f(x) \underset{x\to a}{\to} l\) avec \(l \in \R\) et \(g(x) \underset{x\to a}{\to} l'\) avec \(l' \in \R\) alors \((fg)(x)\underset{x\to a}{\to}ll'\)
                \item Si \(f(x) \underset{x\to a}{\to} \pinf\) et \(g(x) \underset{x\to a}{\to} l'\) avec \(l' \in \Rb \pd \accol{0}\) alors \((fg)(x) \underset{x\to a}{\to} \begin{cases}
                    \pinf &\text{ si } l' > 0 \\
                    \minf &\text{ si } l'<0
                \end{cases}\)
                \item Si \(f(x) \underset{x\to a}{\to} \minf\) et \(g(x) \underset{x\to a}{\to} l'\) avec \(l' \in \Rb \pd \accol{0}\) alors \((fg)(x) \underset{x\to a}{\to} \begin{cases}
                    \minf &\text{ si } l'> 0 \\
                    \pinf &\text{ si } l' < 0 
                \end{cases}\)
            \end{enumerate}
        \item \underline{Inverse}\\
            On suppose que \(f\) ne s'annule pas sur un voisinage de \(a\) sauf éventuellement en \(a\).
            \begin{enumerate}
                \item Si \(f(x) \underset{x\to a}{\to} l\) avec \( l \in Rs\) alors \(\frac{1}{f(x)} \underset{x\to a}{\to} \frac{1}{l}\)
                \item Si \(f(x) \underset{x\to a}{\to} l\) avec \( l\in \accol{\pinf, \minf}\) alors \(\frac{1}{f(x)} \underset{x\to a}{\to} 0\)
                \item Si \(f(x) \underset{x\to a}{\to} 0\) avec les termes \(f(x)\) strictement positifs au voisinage de \(a\) alors \(\frac{1}{f(x)}\underset{x\to a}{\to} \pinf\)
                \item Si \(f(x) \underset{x\to a}{\to} 0\) avec les termes \(f(x)\) strictement négatifs au voisinage de \(a\) alors \(\frac{1}{f(x)} \underset{x\to a}{\to} \minf\)
            \end{enumerate}
        \item \underline{Composition}\\
            Soit \(f\) une fonction définie sur \(I\) et à valeurs réelles telle que \(f (I) \subset J\).\\
            Soit \(g\) une fonction définie sur \(J\) et à valeurs réelles.\\
            Soit \(a\) un point de \(\Rb\), appartenant à \(I\) ou extrémité de \(I\).\\
            Soit \(b\) un point de \(\Rb\), appartenant à \(J\) ou extrémité de \(J\).\\
            Soit \(l\) un point de \(\Rb\).\\
            Si \(f\) admet pour limite \(b\) en \(a\) et si \(g\) admet pour limite \(l\) en \(b\) alors \(g \circ f\) admet pour limite \(l\) en \(a\). Autrement dit,
            \[f(x) \underset{x\to a}{\to} b \text{ et } g(y) \underset{y\to b}{\to} l \imp g \circ f (x)\underset{x\to a}{\to} l\]
    \end{enumerate}
\end{defprop}

\subsection{Limites et relation d’ordre}
Soit \(a\) un point de \(\Rb\), appartenant à \(I\) ou extrémité de \(I\).
\begin{defprop}[Passage à la limite d’une inégalité large]
    Soit \((l,l')\in \Rb \times \Rb\)\\~\\
    Si \(f\) et \(g\) sont deux fonctions définies sur \(I\), à valeurs réelles telles que \(f\leq g\) au voisinage  \(a\) avec \(f\) de limite \(l\) en \(a\) et \(g\) de limite \(l'\) en \(a\) alors \(l \leq l'\)
\end{defprop}

\begin{defprop}[Signe de la fonction et signe de la limite]
    Soit \(f\) une fonction définie sur \(I\), à valeurs réelles, de limite \(l \in \R\) en \(a\).
    \begin{itemize}
        \item Si \(l > 0\) alors \(f\) est strictement positive au voisinage de \(a\).
        \item Si \(l < 0\) alors \(f\) est strictement négative au voisinage de \(a\).
    \end{itemize}
\end{defprop}
\subsection{Existence d’une limite finie}
Soit \(a\) un point de \(\R\), appartenant à \(I\) ou extrémité de \(I\).
\begin{theo}[Théorème d’encadrement]
    Soit \(f\) une fonction définie sur \(I\) et à valeurs réelles, et \(l\) un nombre réel. S’il existe deux fonctions \(g\) et \(h\) définies sur \(I\), à valeurs réelles telles que \(g \leq f \leq h\) au voisinage de \(a\) avec \(g\) et \(h\) de même limite finie \(l\) en \(a\) alors \(f\) admet pour limite \(l\) en \(a\).
\end{theo}

\begin{defprop}[Propriété pratique]
    Soit \(f\) et \(g\) deux fonctions définies sur \(I\), à valeurs réelles, et \(l\) un nombre réel. S’il existe un voisinage de \(a\) sur lequel on \(a\) pour tout \(x\), \(\abs{f(x)-l}\leq g(x)\) avec g de limite \(0\) en \(a\) alors \(f\) a pour limite \(l\) en \(a\).
\end{defprop}

\begin{defprop}[Corollaires de la propriété pratique]
Soit \(f\) et \(g\) deux fonctions définies sur \(I\), à valeurs réelles.
    \begin{itemize}
        \item Si \(f\) a pour limite le réel \(l\) en \(a\) alors \(\abs{f}\) a pour limite \(\abs{l}\) en \(a\).
        \item Si \(f\) a pour limite \(0\) en \(a\) et si \(g\) est bornée au voisinage de \(a\) alors \(f g\) a pour limite \(0\) en a
    \end{itemize}
\end{defprop}

\subsection{Existence d’une limite infinie}
Soit \(f\) une fonction définie sur \(I\) et à valeurs réelles.\\~\\
Soit \(a\) un point de \(\R\), appartenant à \(I\) ou extrémité de \(I\).

\begin{theo}[Théorème de minoration]
    S’il existe une fonction \(g\) définie sur \(I\), à valeurs réelles, telle que \(g \leq f\) au voisinage de \(a\) avec \(g\) de limite \(\pinf\) en \(a\) alors \(f\) admet pour limite \(\pinf\) en \(a\).
\end{theo} 

\begin{theo} [Théorème de majoration]
S’il existe une fonction \(h\) définie sur \(I\), à valeurs réelles telle que \(f \leq h\) au voisinage de \(a\) avec \(h\) de limite \(\minf\) en \(a\) alors \(f\) admet pour limite \(\minf\) en \(a\).
\end{theo}

\subsection{Théorèmes de limite monotone}
\begin{theo}
    Soit \((a, b) \in \R \times \R\) avec \(a < b\).\\~\\
    \begin{itemize}
        \item Cas où la fonction f\( : \intervee{a}{b} \to \R\) définie sur \(\intervee{a}{b}\) est \underline{CROISSANTE}\\
        \begin{itemize}
            \item Si \(f\) est croissante et majorée alors \(f\) admet une limite finie en \(b\) et \( \lim_{x \to b^-} f(x) = \sup_{x \in \intervee{a}{b}} (f(x))\)
            \item Si \(f\) est croissante et non majorée alors \(f\) admet pour limite \(\pinf \) en \(b\).\\
            \item Si \(f\) est croissante et minorée alors \(f\) admet une limite finie en \(a\) et \( \lim_{x \to a^+} f(x) = \inf_{x \in \intervee{a}{b}} (f(x))\)
            \item Si \(f\) est croissante et non minorée alors \(f\) admet pour limite \(\minf\) en \(a\).
        \end{itemize}
        \item Cas où la fonction f\( : \intervee{a}{b} \to \R\) définie sur \(\intervee{a}{b}\) est \underline{DECROISSANTE}\\
        \begin{itemize}
            \item Si \(f\) est décroissante et minorée alors \(f\) admet une limite finie en \(b\) et \( \lim_{x \to b^-} f(x) = \inf_{x \in \intervee{a}{b}} (f(x))\)
            \item Si \(f\) est décroissante et non minorée alors \(f\) admet pour limite \(\minf \) en \(b\).\\
            \item Si \(f\) est décroissante et majorée alors \(f\) admet une limite finie en \(a\) et \( \lim_{x \to a^+} f(x) = \sup_{x \in \intervee{a}{b}} (f(x))\)
            \item Si \(f\) est décroissante et non majorée alors \(f\) admet pour limite \(\pinf\) en \(a\).
        \end{itemize}
\end{itemize}
\end{theo}

\section{Continuité des fonctions à valeurs réelles en un point}
Soit \(f\) une fonction définie sur \(I\), à valeurs dans \(\R\) et \(a\) un réel appartenant à \(I\).
\subsection{Définition}
\begin{defi}
    \begin{enumerate}
        \item \(f\) est dite continue en \(a\) si \(f\) admet pour limite \(f (a)\) en \(a\).
        \item \(f\) est dite continue à gauche en \(a\) si la restriction \(f_{|I\inter\intervee{\minf}{a}}\) est continue en \(a\) c’est-à-dire si \(\lim_{ x\to a^-}f(x)\) existe et vaut \(f (a)\).
        \item \(f\) est dite continue à droite en \(a\) si la restriction \(f_{|I\inter\intervee{a}{\pinf}}\) est continue en \(a\) c’est-à-dire si \(\lim_{ x\to a^+}f(x)\) existe et vaut \(f (a)\).
    \end{enumerate}
\end{defi}
\subsection{Condition nécessaire et suffisante de continuité en un point}
\begin{defprop}
     \(f\) est continue en \(a\) si, et seulement si, elle est continue à gauche et à droite en \(a\).
\end{defprop}
\subsection{Caractérisation séquentielle de la continuité en un point}
\begin{defprop}
     \(f\) est continue en \(a\) si, et seulement si, pour toute suite \((x_n)\) d’éléments de \(I\) qui admet pour limite \(a\), la suite réelle \((f (x_n))\) admet pour limite \(f (a)\).
\end{defprop}

\subsection{Opérations sur les fonctions continues en un point}
\begin{defprop}
    Soit \(f\) et \(g\) deux fonctions définies sur \(I\), à valeurs réelles.
    \begin{enumerate}
        \item \underline{Combinaison linéaire}\\~\\
            Si \(f\) et \(g\) sont continues en \(a\) et \((\lambda, \mu)\) est un couple de réels alors \(\lambda f + \mu g\) est continue en \(a\).
        \item \underline{Produit}\\~\\
            Si \(f\) et \(g\) sont continues en \(a\) alors \(f g\) est continue en \(a\).
        \item \underline{Quotient}\\~\\
            Si \(f\) et \(g\) sont continues en \(a\) et si \(g\) ne s’annule pas au voisinage de \(a\) alors \(f g\) est continue en \(a\).
    \end{enumerate}
\end{defprop}

\subsection{Composition de fonctions continues en un point}
    \begin{defprop}
        Soit \(f\) une fonction définie sur \(I\) et à valeurs réelles tel que, pour tout \(x\) de \(I\), \(f (x)\) appartient à \(J\).\\
        Soit \(g\) une fonction définie sur \(J\) et à valeurs réelles.\\
        Soit \(a\) un réel de \(I\).\\
        Si \(f\) est continue en \(a\) et si \(g\) est continue en\( f (a)\) alors \(g \circ f\) est continue en \(a\).\\
\end{defprop}


\subsection{Prolongement par continuité} 
\begin{defprop} 
    Soit \(b\) un réel n’appartenant pas à \(I\) mais extrémité de \(I\). \\
    Si \(f\) admet une limite finie \(l\) en \(b\) alors le prolongement de \(f\) à \(I \union \accol{b}\) noté \(\tilde{f} : I \union {b} \to \R\) défini par \(\forall x \in  I, \tilde{f} (x) = f (x)\) et \(\tilde{f} (b) = l \) est continu en \(b\) et appelé prolongement par continuité de \(f\) en \(b\).
\end{defprop}

\section{Continuité des fonctions sur un intervalle}
\subsection{Définition}
\begin{defi}
    Une fonction définie sur \(I\), à valeurs dans \(R\) est dite continue sur \(I\) si elle est continue en tout \(a\) de \(I\). \\
    L’ensemble des fonctions continues sur \(I\) à valeurs dans \(R\) est souvent noté \(\mathscr{C}\paren{I,\R} \) ou \(\classe{I}\)
\end{defi}
\subsection{Théorèmes généraux : combinaison linéaire, produit, quotient, composée}
\begin{theo}
    \begin{itemize}
        \item \(\forall \paren{f,g} \in \paren{\mathscr{C}\paren{I,\R}}^2,\forall \paren{\alpha,\beta}\in\R^2,\alpha f + \beta g \in \mathscr{C}\paren{I,\R}\)
        \item \(\forall \paren{f,g} \in \paren{\mathscr{C}\paren{I,\R}}^2,fg \in \mathscr{C}\paren{I,\R}\)
        \item \(\forall \paren{f,g} \in \paren{\mathscr{C}\paren{I,\R}}^2,g(I)\subset \Rs,\frac{f}{g}\in \mathscr{C}\paren{I,\R}\)
        \item \(\forall f \in \mathscr{C}\paren{I,\R},\forall g \in \mathscr{C}\paren{J,\R},f(I)\subset J \imp g \circ f \in \mathscr{C}\paren{I,\R}\)
    \end{itemize}
\end{theo}

\subsection{Théorème des valeurs intermédiaires et corollaires}
\begin{theo} [Théorème des valeurs intermédiaires]
Soit \(f\) une fonction définie sur \(I\) à valeurs dans \(\R\) et, \(a\) et \(b\) deux points de \(I\).\\
Si \(f\) est continue sur \(I\) avec \(f (a) \leq f (b)\) alors \(f\)\(\) atteint toute valeur intermédiaire entre \(f (a)\) et \(f (b)\)
\end{theo}
\begin{dem}
    On suppose les hypothèses réunies. Dans le cas \(a = b\), le résultat attendu est immédiat. On se place donc dans le cas \(a < b\) (sans perte de généralité) avec \(f (a) < f (b)\) (car le cas \(f (a) = f (b)\) est immédiat).\\~\\
Soit \(y\) un réel de l’intervalle \(\intervee{f(a)}{f(b)}\).\\~\\
\underline{Montrons, en suivant le principe de dichotomie, qu’il existe un réel \(x\) dans \(\intervii{a}{b} \) tel que \(y = f (x)\).
}
\begin{itemize}
    \item On note \(a_0 = a, b_0 = b\) ; on a alors \( f (a_0) < y < f (b_0)\).\\
    \item Etape \(1\) : on pose \( m_0 = \frac{1}{2}(a_0 + b_0)\).
        \begin{itemize}
            \item si \(y = f (m_0)\) alors on a bien trouvé un réel \(x\) dans \(\intervii{a}{b} \) tel que \(y = f (x)\) : c’est terminé !
            \item si \(f (a_0) < y < f (m_0)\), on pose \((a_1, b_1) = (a_0, m_0)\) et on continue la recherche de \(x\) dans \(\intervii{a_1}{b_1} \).
            \item si \(f (m_0) < y < f (b_0)\), on pose \((a_1, b_1) = (m_0, b_0)\) et on continue la recherche de \(x\) dans \(\intervii{a_1}{b_1} \).
        \end{itemize}
        Dans ces deux derniers cas, on a : \(f (a_1) < y < f (b_1)\) et on passe à l’étape \(2\).
    \item Etape \(2\) : on pose\( m_1 = \frac{1}{2}(a_1 + b_1)\).
        \begin{itemize}
            \item si \(y = f (m_1)\) alors on a bien trouvé un réel \(x\) dans \(\intervii{a}{b} \) tel que \(y = f (x)\) : c’est terminé !
            \item si \(f (a_1) < y < f (m_1)\), on pose \((a_2, b_2) = (a_1, m_1)\) et on continue la recherche de \(x\) dans \(\intervii{a_2}{b_2} \).
            \item si \(f (m_1) < y < f (b_1)\), on pose \((a_2, b_2) = (m_1, b_1)\) et on continue la recherche de \(x\) dans \(\intervii{a_2}{b_2} \).
        \end{itemize}
        Dans ces deux derniers cas, on a : \(f (a_2) < y < f (b_2)\) et on passe à l’étape \(3...0\).
\end{itemize}
Dans ce processus, s’il existe un entier \(k_0\) tel que \(f (m_{k_0} ) = y\), c’est terminé ! Sinon, on a créé une suite croissante \((a_k)\) et une suite décroissante \((b_k)\) telles que la suite \(\paren{a_k-b_k} = \paren{\frac{b-a}{2^k}}\) a pour limite \(0\).\\~\\
Ces suites sont donc adjacentes. Par théorème, elles convergent vers une même limite réelle \(l\) qui vérifie \(\forall k \in \N, a_k \leq l \leq b_k\) donc, en particulier, \(a_0 \leq l \leq b_0 \) c’est-à-dire \(a \leq l \leq b\).\\~\\
Comme \(f\) est continue, on en déduit alors que les suites \((f (a_k))\) et \((f (b_k))\) convergent vers \(f (l)\).\\~\\
De plus, par construction des suites \((a_k)\) et \((b_k)\), on a : \(\forall k \in  \N, f (a_k) < y < f (b_k)\). Par passage à la limite, on trouve donc : \(f (l) \leq y \leq f (l)\) puis, par antisymétrie, \(f (l) = y\) et c’est terminé !\\~\\
\underline{Conclusion} : \(f\) atteint toute valeur intermédiaire entre \(f (a)\) et \(f (b)\).
\end{dem}


\begin{defprop}[Image d’un intervalle]
    L’image d’un intervalle de \(R\) par une fonction continue à valeurs réelles est un intervalle de \(R\).
\end{defprop}
\begin{dem}
    On suppose que \(f\) est une fonction définie, continue sur un intervalle \(I\) et à valeurs réelles.\\~\\
    \underline{Montrons que \(f (I)\) est un intervalle de \(\R\)} à l’aide de la caractérisation des intervalles vue dans le chapitre "Compléments sur les réels".\\~\\
    Soit \(\alpha\) et \(\beta\) deux réels quelconques de \(f (I)\) tels que \(\alpha < \beta\).\\~\\
    Alors il existe \(a\) et \(b\) deux réels de \(I\) tels que \(\alpha = f (a)\) et \(\beta = f (b)\).\\~\\
    Pour tout réel \(y\) de \(\intervii{\alpha}{\beta}\) , le théorème des valeurs intermédiaires assure alors l’existence d’un réel \(x\) compris entre \(a\) et \(b\) tel que \(y = f (x)\). Comme \(a\) et \(b\) sont des réels appartenant à l’intervalle \(I\), le réel \(x\) appartient aussi à l’intervalle \(I\) ce qui prouve que y appartient à \(f (I)\).\\~\\
    Ainsi : \(\forall (\alpha, \beta) \in (f (I))^2 , \alpha < \beta \imp \intervii{\alpha}{\beta} \subset f (I)\).\\~\\
    Par caractérisation des intervalles, on en déduit que \(f (I)\) est un intervalle.
\end{dem}

\begin{defprop}[Cas des fonctions continues strictement monotones]
    
Si \(f : I \to R\) est continue et strictement croissante sur \(I\), intervalle de bornes \(a\) et \(b\) avec \(a < b\), alors
\begin{itemize}
    \item  pour \(I = \intervii{a}{b}\), on a : \(f(I) = \intervii{f(a)}{f(b)}\)
    \item  pour \(I = \intervee{a}{b}\), on a : \(f(I) = \intervee{\lim_{x\to a^+}f(x)}{\lim_{x\to b^-}f(x)}\)
    \item  pour \(I = \intervie{a}{b}\), on a : \(f(I) = \intervie{f(a)}{\lim_{x\to b^-}f(x)}\)
    \item  pour \(I = \intervei{a}{b}\), on a : \(f(I) = \intervei{\lim_{x\to a^+}f(x)}{f(b)}\)
\end{itemize}
\end{defprop}
\subsection{Théorème des bornes atteintes et corollaire}
\begin{theo}[Théorème des bornes atteintes]
    Si \(f\) est une fonction continue sur un segment et à valeurs réelles alors \(f\) est bornée et atteint ses bornes.
\end{theo}
\begin{dem}
    On suppose les hypothèses réunies.\\~\\
    \(f\) étant continue sur l’intervalle \(\intervii{a}{b}\) et à valeurs réelles, l’image de \(\intervii{a}{b}\) par \(f\) est un intervalle \(J\). On note \(m\) la borne inférieure de \(J\) et \(M\) la borne supérieure de l’intervalle \(J\) considéré comme partie de la droite achevée \(\Rb\).

    Par propriété (vue dans le chapitre "Compléments sur les réels"), il existe une suite \((y_n)\) d’éléments de \(J\) de limite \(m\).\\~\\
    Comme \(J = f (\intervii{a}{b})\) , il existe alors une suite \((x_n)\) d’éléments de\( \intervii{a}{b}\) telle que \(\forall n \in \N, y_n = f (x_n)\).\\~\\
    La suite \((x_n)\) étant à valeurs dans \(\intervii{a}{b}\), elle est bornée. D’après le théorème de Bolzano-Weiestrass,elle admet donc une suite extraite convergente. On note \(x_{\phi(n)}\) une telle suite et \(l\) sa limite.\\~\\
    On a donc :
    \begin{itemize}
        \item  \(y_{\phi(n)} \to  m\) comme suite extraite d’une suite convergente de limite \(m\) ;
        \item \(x_{\phi(n)} \to l\)  ;
        \item \(f\) continue en \(l\) car \(f\) continue sur\( \intervii{a}{b}\) et \(l \in \intervii{a}{b}\), comme limite d’une suite à valeurs dans \(\intervii{a}{b}\).
    \end{itemize}
    On peut donc passer à la limite dans les égalités\\~\\
    \[\forall n \in \N, y_{\phi(n)} = f\paren{x_{\phi(n)}}\]
    Cela donne \(m = f (l)\) et prouve donc que \(m\) est un réel et que \(m\) est atteint par \(f\) .\\~\\
    On montre de même que \(M\) est un réel atteint par \(f\) .\\~\\
    \underline{Conclusion} : \(f\) est bornée et atteint ses bornes.
\end{dem}

\begin{defprop}[Image d’un segment]
    L’image d’un segment de \(\R\) par une fonction continue à valeurs réelles est un segment de \(\R\).
\end{defprop}

\subsection{Théorème de la bijection}
\begin{defprop}[Continuité et injectivité]
    Toute fonction continue sur un intervalle, à valeurs réelles et injective, est strictement monotone.\\~\\
\underline{Remarque}
    La réciproque est fausse ; en revanche, toute fonction strictement monotone sur un intervalle est injective.
\end{defprop}

\begin{dem}
    Soit \(I\) un intervalle de \(\R\), non vide et non réduit à un point, et \(f : I \to R\) continue et injective.\\~\\
    Raisonnons par l’absurde en supposant que \(f\) n’est ni strictement croissante, ni strictement décroissante. Alors,
    il existe \((a, b) \in I^2\) tel que \(a < b\) et \(f (a) \geq f (b)\) et il existe \((a', b') \in I^2\) tel que \(a' < b'\) et \(f (a') \leq f (b')\).\\~\\
    On note \(g : \intervii{0}{1} \to \R\) définie par : \(\forall t \in \intervii{0}{1}, g(t) = f ((1 - t)a' + ta) - f ((1 - t)b' + tb)\) . Par théorèmes généraux, \(g\) est continue sur \(\intervii{0}{1}\) avec \(g(0) = f (a') - f (b')\) et \(g(1) = f (a) - f (b)\) donc \(g(0) \leq 0 \) et \(g(1) \geq 0\). Par théorème des valeurs intermédiaires, il existe alors \(t_0 \in \intervii{0}{1}\) tel que \(g(t0) = 0\).\\~\\
    Ainsi \(f ((1 - t_0)a' + t_0a) = f ((1 - t_0)b' + t_0b)\) puis, par injectivité de \(f\) , \((1 - t_0)a' + t_0a = (1 - t_0)b' + t_0b\) ce qui donne \((1 - t_0)(b' - a') + t_0(b - a) = 0\). Comme les termes \((1 - t_0)(b' - a')\) et \(t_0(b - a)\) sont positifs, on en déduit que\( (1 - t_0)(b' - a') = t_0(b - a) = 0\) et enfin, comme \( b' - a'\) et \(b - a\) sont strictement positifs, on trouve \(1 - t_0 = 0\) et \( t0 = 0\) ce qui est absurde.\\~\\
    \underline{Conclusion} : \(f\) est strictement monotone.
\end{dem}
\begin{theo}[Théorème de la bijection]
Si \(f\) est une fonction à valeurs réelles définie, continue et strictement monotone sur un intervalle \(I\) alors \(f\) réalise une bijection de \(I\) sur\( J = f (I)\) dont la bijection réciproque \(f {^-1}\) est définie, continue et strictement monotone sur \(J\) avec même monotonie que \(f\) .
\end{theo}

\begin{dem}
On suppose les hypothèses réunies.\\~\\
\(f\) est injective (car strictement monotone) donc l’application \(\tilde{f} : I \mapsto f (I)\) définie par \(\forall x \in I, \tilde{f} (x) = f (x)\) est injective et surjective donc est une bijection : on dit que \(f\) réalise une bijection de \(I\) sur \(J = f (I)\). De plus, comme \(f\) est continue et à valeurs réelles, \(J = f (I)\) est un intervalle de \(R\), non vide (puisque \(I\) est non vide) et non réduit à un point de \(\R\) (puisque \(I\) n’est pas réduit à un point et que \(f\) est injective).\\~\\
La bijection réciproque \(\tilde{f^{-1}} : J \mapsto I\), notée plus simplement \(f ^{-1}\), est définie sur \(J\) et strictement monotone de même monotonie que \(f\) . En effet, si on suppose que \(f\) est strictement croissante (par ex), alors pour tout \((x, y) \in J^2\) tel que \(x < y\), on a \(f ^{-1}(x) < f ^{-1}(y)\) (sinon on aurait \(f ^{-1}(x) \geq f ^{-1}(y)\) puis par stricte croissance de \(f\) , \(x \geq y\) ce qui est faux) donc \(f ^{-1}\) est strictement croissante sur \(J\) par définition.\\~\\
Soit \(\lambda \in J\). Comme \(f ^{-1}\) est strictement monotone sur \(J\), le corollaire du théorème de limite monotone prouve (sous réserve que cela ait du sens) que \(l = lim _{\lambda^-} f ^{-1}\) existe, est finie et appartient à \(I\). Par continuité de \(f\) en \(l\), \(\lim_{x\to l}f (x) = f (l)\) puis par composition de limites, \(\lim_{ y\to\lambda^-} f \paren{f ^{-1}(y)} = f (l)\) ce qui donne \(f (l) = \lambda\) puis \(l = f ^{-1}(\lambda)\) et prouve que\( f ^{-1}\) est continue à gauche en \(\lambda\). On montre de même (sous réserve que cela ait du sens) la continuité à droite ce qui prouve la continuité de \(f ^{-1}\) en tout \(\lambda\) de \(J\).\\~\\
\underline{Conclusion} : \(f ^{-1}\) est définie, continue et strictement monotone sur \(J\) avec même monotonie que \(f\) .
\end{dem}


\section{Cas des fonctions à valeurs complexes}
\subsection{Ce qui s’étend aux fonctions complexes}
\begin{defprop}
    \begin{itemize}
        \item Limite finie :
        \begin{itemize}
            \item définition et caractérisations (cf infra) ;
            \item unicité ;
            \item opérations sur les limites finies ;
            \item lien entre existence d’une limite finie en un point et caractère borné au voisinage de ce point.
        \end{itemize}
        \item Continuité en un point et sur un intervalle.
    \end{itemize}
\end{defprop} 

\subsection{Ce qui ne s’étend pas aux fonctions à valeurs complexes}
\begin{defprop}
    \begin{itemize}
        \item Notion de limite infinie.
        \item Résultats utilisant la relation d’ordre dont les théorèmes d’existence de limite.
    \end{itemize}
\end{defprop}
\subsection{Limite d’une fonction à valeurs complexes}
\begin{defprop}
    Soit \(f\) une fonction définie sur \(I\) et à valeurs complexes, et \(l\) un nombre complexe.\\
    \underline{Définition} :\\~\\
    Soit \(a\) un point de \(I\) ou une extrémité de \(I\).\\
    On dit que \(f\) a pour limite \(l\) en \(a\) si la fonction à valeurs réelles \(\abs{f-l}\) a pour limite \(0\) en \(a\).
    \underline{Caractérisations} :\\~\\
    \begin{itemize}
        \item \(f\) admet pour limite \(l\) en \(a\) (point de \(I\) ou extrémité de \(I\)) si, et seulement si, \(\Reel{f}\) et \(\Ima{f}\) admettent respectivement pour limite \(\Reel{l}\) et \(\Ima{l}\) en \(a\).
        \item \(f\) est continue en \(a\) (point de \(I\)) si, et seulement si,  \(\Reel{f}\) et \(\Ima{f}\) le sont.
        \item \(f\) est continue sur \(I\) si, et seulement si, \(Re(f ) et Im(f )\) le sont.
    \end{itemize}
            
\end{defprop}
