\chapter{Nombres complexes}

\minitoc
\section{Généralité}
\begin{defi}[Propriété de \(\C\)]

	On ADMET l'existence d’un ensemble noté \(\C\), dont les éléments sont appelés nombres complexes, tel que :
	\begin{enumerate}
		\item \(\C\) contient\(\R \)
		\item \(\C\) est muni de deux opérations \(+\) et \(\times\) sur \(\C\) qui étendent les opérations \(+\) et \(\times\) connues sur \(\R\) et suivent les mêmes règles de calcul que celles-ci
		\item \(\C\) contient un élément noté \(\i\) vérifiant \(\i^2 = -1\)
		\item Tout élément \(z\) de \(\C\) s'écrit de manière uneique sous la forme \(z = a+\i b\) avec \(\paren{a,b} \in \R^2\)
	\end{enumerate}
\end{defi}
\begin{rem}
	\begin{itemize}
		\item La forme \(z = a+\i b\)  avec \(\paren{a,b} \in \R^2\) est dite forme algébrique du nombre complexe \(z\) \begin{itemize}
			      \item le réel \(a\) est dit partie réelle du nombre complexe \(z\) et noté \(a = \Reel{z}\)
			      \item le réel \(b\) est dit partie imaginaire du nombre complexe \(z\) et noté \(b = \Im{z}\)
		      \end{itemize}
		\item L'unicité d'écriture d'un nombre complexe sous forme algébrique se traduit par : \\
		      Pour tout réels \(a,b,a'\) et \(b'\), on a :
		      \[a+\i b = a'+\i b' \text{si, et seulement si, } a=a' \text{ et } b= b'\]
	\end{itemize}
\end{rem}

\begin{defprop}[Opériation sur \(\C\)]
	L’ensemble \(\C = \accol{a + \i b \tq \paren{a, b} \in \R^2}\) est muni deux opérations + et \(+\) et \(\times\) définies par, pour tout nombre complexe \(z\) de forme algébrique \(a + \i b\) et tout nombre complexe \(z'\) de forme algébrique \(a' + \i b'\) : \[\begin{cases}
			z+z' = (a+\i b) + (a'+\i b') = (a+a')+\i(b+b') \\
			z \times z' = (a+\i b) \times (a'+\i b') = (aa'-bb') + \i(ab'+a'b)
		\end{cases}\]
\end{defprop}

\begin{defprop}[Extension des résultat vus dans \(\R\)]
	\begin{enumerate}
		\item Pour tout \(n\) entier naturel et tout nombre complexe \(z\) différent de \(1\), on a :
		      \[\sum_{k=0}^n z^k = \frac{1-z^{k+1}}{1-z}\]
		\item Pour tout \(n\) entier naturel et tout couple \((z,z')\) nombres complexes , on a :
		      \[(z+z')^n = \sum_{k=0}^{n}\binom{n}{k}z^k(z')^{n-k} = \sum_{k=0}^{n}\binom{n}{k}z^{n-k}(z')^k\]
		\item Pour tout \(n\) entier naturel et tout couple \((z,z')\) nombres complexes , on a :
		      \[z^n+(z')^n = (z-z')\paren{z^{n-1}+z^{n-1}z'+\dots+z(z')^{n-2}+(z')^{n-1}} = (z-z')\sum_{k=0}^{n-1} z^{n-1-k}(z')^{k} =(z-z')\sum_{k=0}^{n-1} z^{k}(z')^{n-1-k} \]
	\end{enumerate}
\end{defprop}

\begin{defprop}[Plan complexe : affixe d’un point, d’un vecteur]
	Dans toute la suite, on considère le plan usuel muni d’un repère orthonormé direct.
	\begin{itemize}
		\item A tout complexe \(z\), on peut associer le point \(M\) de coordonnées \((\Reel{z}, \Ima{z})\) dit image de \(z\).
		\item A tout point \(M\) de coordonnées \((x, y)\), on peut associer le complexe \(z = x + \i y\) dit affixe de \(M\) .
	\end{itemize}
	On identifie donc \(\C\) au plan usuel muni d’un repère orthonormé direct et on parle de “plan complexe”. \\

	A tout complexe \(z\), on peut aussi associer le vecteur \(\vec{u}\) de coordonnées \((\Reel{z}, \Ima{z})\) dit image de z et à tout vecteur  \(\vec{u}\) de coordonnées \((x, y)\), on peut associer le complexe \( z = x + \i y\) dit affixe de  \(\vec{u}\) . Ainsi :
	\begin{itemize}
		\item Pour tout vecteur  \(\vec{u}\) d’affixe \(z\) et tout réel \(\alpha\), le vecteur \(\alpha \vec{u}\) a pour affixe \(\alpha z\). \\
		\item Pour tous vecteurs \(\vec{u}\) et \(\vec{u'}\) d’affixes respectives \(z\) et \(z'\), le vecteur \(\vec{u} + \vec{u'} \) a pour affixe \(z + z'\). \\
		\item Pour tous points \(M\) et \(M '\) d’affixes respectives \(z\) et \(z'\) , le vecteur \(\vec{MM'}\)a pour affixe \(z' - z\).
	\end{itemize}
\end{defprop}

\section{Conugué d'un nombre complexe}
\begin{defi}
	On appelle conjugué d’un nombre complexe \(z\) et on note \(\conj{z}\) le nombre complexe défini par : \[\conj{z} = \Reel{z} -\i \Ima{z}\]
	Pour tout nombre complexe \(z\), le point d’affixe \(\conj{z}\) et le point d’affixe \(z\) sont symétriques par rapport à l’axe des réels dans le plan complexe.
\end{defi}

\begin{defprop}
	Pour tous nombres complexes \(z\) et \(z'\), on a les propriétés suivantes :
	\begin{enumerate}
		\item \(z+\conj{z} = 2 \Reel{a}\)\\
		\item \(z-\conj{z} = -2 \Ima{z}\) \\
		\item \(\conj{\conj{z}} = z \)\\
		\item \(\conj{z+z'} = \conj{z}+\conj{z'}\)\\
		\item \(\conj{zz'} = \conj{z} \conj{z'} \)\\
		\item \(\conj{\frac{z}{z'}} = \frac{\conj{z}}{\conj{z'}}\)
	\end{enumerate}
\end{defprop}

\section{module d'un nombre complexe}
\begin{defprop}
	On appelle module d’un nombre complexe \(z\) et on note \(\abs{z}\) le nombre réel positif défini par : \[\abs{z} = \sqrt{ \paren{\Reel{z}}^2+\paren{\Ima{z}}^2}\]
\end{defprop}

\begin{defprop}[interprétation géometriques]
	\begin{itemize}
		\item Pour tout nombre complexe \(z\), le module \(\abs{z}\) est : \begin{itemize}
			      \item la distance entre le point d’affixe \(0\) et le point d’affixe \(z\) ;
			      \item la norme de tout vecteur d’affixe \(z\)
		      \end{itemize}
		\item Pour tous nombres complexes  \(z\) et \(z'\) le module \(\abs{z-z'}\) est :\begin{itemize}
			      \item la distance entre les points d’affixe \(z\) et \(z'\) ;
			      \item la norme du vecteur d’affixe \(z' - z\)
		      \end{itemize}
		\item Soit \(r\) un réel positif, \(z_0\) un nombre complexe et \(M_0\) le point d’affixe \(z_0\).
		      \begin{itemize}
			      \item Les points du plan dont l’affixe \(z\) vérifie \(\abs{z - z_0} = r\) forment le cercle de centre \(M_0\) et de rayon \(r\).
			      \item Les points du plan dont l’affixe \(z\) vérifie \(\abs{z - z_0} \leq r\) forment le disque de centre \(M_0\), de rayon \(r\)
		      \end{itemize}
	\end{itemize}
\end{defprop}

\begin{prop}
	Pour tous nombres complexes \(z\) et \(z'\), on a les propriétés suivantes :
	\begin{itemize}
		\item \(\abs{\Reel{z}} \leq \abs{z}\) et \(\abs{\Ima{z}} \leq \abs{z}\)
		\item \(\abs{z}^2 = z \conj{z}\)
		\item \(\abs{zz'} = \abs{z} \abs{z'}\)
		\item \(\abs{\frac{z}{z'}} = \frac{\abs{z}}{\abs{z'}}\) Dans le cas où \(z'\) est non nul
		\item \(\frac{z}{z'} = \frac{z\abs{z'}}{\abs{z'}^2}\)
		\item \(\abs{z+z'} \leq \abs{z}+\abs{z'}\) avec égalité si, et seulement si il existe un réel positif \(\alpha\) tel que \(z' = \alpha z\)
	\end{itemize}
\end{prop}

\section{Nombre complexe de module \(1\) et trigonométrie}
\begin{defi}[Cercle trigonométrique]
	On identifie le cercle trigonométrique et l’ensemble des nombres complexes de module \(1\) que l’on note : \[\U = \accol{z \in \C \tq \abs{z} = 1}\]
\end{defi}

\begin{defprop}
	Pour tout nombre réel \(t\), on appelle exponentielle imaginaire de \(t\) et on note \(e^{\i t}\) le nombre complexe défini par :
	\[e^{\i t} = \cos(t) + \i \sin(t) \]
	Pour tous nombres réels \(t\) et \(t'\), on a l’égalité : \[e^{\i(t+t')} = e^{\i t}e^{\i t'} \]
\end{defprop}


\begin{defprop}[Formule D'Euler]
	Pour tout nombre réel \(t\), on a les égalités suivantes dites formules d’Euler
	\[\cos(t) = \frac{e^{\i t}+e^{-\i t}}{2} \text{ et } \sin(t) = \frac{e^{\i t} - e^{-\i t}}{2}\]
\end{defprop}
\begin{prop}[Technique de l'angle moitié]

	La technique de l’angle moitié permet l’obtention de factorisations classiques à savoir retrouver :
	\begin{itemize}
		\item pour tout \(t\) réel, \(1+e^{\i t} = e^{\i \frac{t}{2}}\paren{e^{-\i \frac{t}{2}}+e^{\i \frac{t}{2}}} = 2 \cos\paren{-\frac{t}{2}}e^{\i \frac{t}{2}} = 2 \cos\paren{\frac{t}{2}}e^{\i \frac{t}{2}} \)
		\item pour tout \(t\) réel, \(1-e^{\i t} = e^{\i \frac{t}{2}}\paren{e^{-\i \frac{t}{2}}-e^{\i \frac{t}{2}}} = 2 \sin\paren{-\frac{t}{2}}e^{\i \frac{t}{2}} = -2 \sin\paren{\frac{t}{2}}e^{\i \frac{t}{2}} \)
		\item pour tout réel \(p\) et \(q\), \(e^{\i p}+e^{\i q} = e^{\i \frac{p+q}{2}}\paren{e^{\i \frac{p-q}{2}}+e^{-\i \frac{p-q}{2}}} = 2 \cos\paren{\frac{p-q}{2}}e^{\i \frac{p+q}{2}}  \)
		\item pour tout réel \(p\) et \(q\), \(e^{\i p}-e^{\i q} = e^{\i \frac{p+q}{2}}\paren{e^{\i \frac{p-q}{2}}-e^{-\i \frac{p-q}{2}}} = - 2 \sin\paren{\frac{p-q}{2}}e^{\i \frac{p+q}{2}}  \)
	\end{itemize}
	\underline{Remarque} : \\
	En écrivant la partie réelle et la partie imaginaire de \(e^{\i p} \pm e^{\i q}\) à partir des deux dernières factorisations, on trouve des formules de factorisation pour \(\cos (p) \pm \cos (q) \)et \(\sin (p) \pm \sin (q)\) \\ \\
	\underline{Linéarisation} \\
	A l’aide des formules d’Euler et du binôme de Newton, on peut transformer une expression du type
	\(cos(t)^n\) ou \(sin(t)^n\) avec \(t\) réel et \(n\) entier naturel en une combinaison linéaire de \(cos(pt)\) ou de \(sin(pt)\)
	avec \(p\) un entier naturel. Cela est notamment utile pour du \underline{calcul de primitives}.
\end{prop}

\begin{exoex}
	~\\Soit \(f(x) = \paren{\sin(x)}^3\) avec \(x \in \R\). Calculer la primitive de \(f\)
\end{exoex}

\begin{corr}
	\begin{align*}
		\paren{\sin(x)}^3 & = \paren{\frac{e^{\i x}-e^{-\i x}}{2 \i}}^3                                              \\
		                  & =\frac{1}{-8\i} \paren{e^{3\i x}+3\paren{e^{-\i x}}-3\paren{e^{\i x}} -e^{-3 \i x}}      \\
		                  & =\frac{1}{-4} \paren{\frac{e^{3\i x}-e^{-3\i x}}{2 \i}-3\frac{e^{\i x}-e^{_\i x}}{2 \i}} \\
		                  & = -\frac{1}{4}\sin(3x) +\frac{3}{4}\sin(x)
	\end{align*}
	Donc \(F_\lambda(x) = \frac{1}{12}\cos(3x)- \frac{3}{4}\cos(x) + \lambda \) pour \(\lambda \in \R \)
\end{corr}

\begin{defprop}[Formule de Moivre]
	Pour tout nombre réel \(t\) et tout entier relatif \(n\), on a \(e^{\i nt} = \paren{e^{\i t}}^n\), c’est-à-dire :
	\[\cos(nt)+\i\sin(nt) = \paren{cos(t)+\i \sin(t)}^n\]
\end{defprop}

\begin{dem}[Moivre par récurrence]
	Soit \(n \in \N\) et \(t\in\R\)
	Montrons que \(\quantifs{\forall (n,t) \in \N\times\R} e^{\i nt} = \paren{e^{\i t}}^n\) \\
	On note \(P(n)\) la Propriété \guillemets{\(e^{\i nt} = \paren{e^{\i t}}^n\)}
	\begin{itemize}
		\item \underline{Initialisation} :
		      \(P(0)\) est vrai car \(\begin{cases}
			      \paren{e^{\i t}}^0 & = 1 \\
			      e^{\i t 0}         & = 1
		      \end{cases}\)
		\item \underline{Hérédité}
		      Soit \(n \in \N\) tel que \(P(n)\) est vrai, Montrons que \(P(n+1)\) est vrai :
		      \begin{align*}
			      e^{\i (n+1) t} & = e^{\i(n+1)t}                      \\
			                     & = e^{\i n t} \times e^{\i t}        \\
			                     & =\paren{e^{\i t}}^n \times e^{\i t} \\
			                     & = \paren{e^{\i t}}^{n+1}            \\
		      \end{align*}
	\end{itemize}
	Donc \(P(n+1)\) Vrai.
\end{dem}

\begin{appl}[Applications usuelles importantes]
	~\\
	Soit \(C = \sum_{k=0}^n \cos(kt)\) et \(S = \sum_{k=0}^n \sin(kt)\) avec \(n \in \N\) et \(t \in \R\)\\
	On Obtient des expressions simplifiées des sommes \(C\) et \(S\) par le calcul annexe suivant
	\[C+\i S = \sum_{k=0}^n e^{\i kt} = \sum_{k=0}^n \paren{e^{\i t}}^k =
		\begin{cases}
			n+1                               & \text{si } t\equiv 0\croch{2 \pi} \\
			\frac{1-e^{\i(n+1)t}}{1-e^{\i t}} & \text{ sinon }
		\end{cases}
	\]
	qui donne \[C+\i S = \begin{cases}
			n+1                                                                       & \text{si } t\equiv 0\croch{2 \pi} \\
			\frac{\paren{1-e^{\i(n+1)t}}\paren{1-e^{\i t}}}{2\paren{1-\cos\paren{t}}} & \text{ sinon }
		\end{cases} \]
	On conclut alors sur les valeurs de \(C\) et \(S\) en exhibant les parties réelle et imaginaire de \(C + \i S\).
\end{appl}

\section{Forme trigonométrique pour les nombres complexes non nuls}

\begin{defprop}
	Tout nombre complexe non nul \(z\) peut s’écrire sous la forme \[ z = re^{\i \theta}\]
	avec \(r\) un réel strictement positif et \(\theta\) un réel. Cette écriture est dite forme trigonométrique de \(z\). \\
	\underline{Attention} \\
	Dans cette écriture de \(z\).
	\begin{itemize}
		\item le réel strictement positif \(r\) est \underline{unique} car il est nécessairement égal à \(\abs{z}\)
		\item le réel \(\theta\) n'est \underline{pas unique} car si le réel \(\theta\) convient alors les réels \(\theta ' \equiv \theta \croch{2 \pi}\) conviennent.
	\end{itemize}
\end{defprop}

\begin{dem}
	Soit \(z\in \Cs\), alors \(\abs{z} \neq 0 \) donc \(\frac{z}{\abs{z}}\) existe avec  \(\abs{\frac{z}{\abs{z}}} = \frac{\abs{z}}{\abs{\abs{z}}} = \frac{\abs{z}}{\abs{z}} = 1\) \\
	Donc \(\frac{z}{\abs{z}} \in \U\) donc il existe \(\theta \in \R \) tel que \(\frac{z}{\abs{z}} = e^{\i \theta} \iff z = \abs{z}e^{\i \theta}\) \\
	Ceci prouve l'existence de l'écriture. \\
	\(r\) est unique car : \(\begin{cases}
		z & = re^{\i \theta}  \\
		z & = r'e^{\i \theta}
	\end{cases} \imp \begin{cases}
		\abs{z} & = r  \\
		\abs{z} & = r'
	\end{cases} \imp r = r'\)
\end{dem}
\begin{defprop}[Arguments]
	Soit \(z\) un nombre complexe non nul. Tous les nombres réels \(\theta\) tels que \(z\) peut s'écrire \[z = re^{\i \theta}\] avec \(r\) réel strictement positif sont dits arguments de \(z\) \\
	\underline{Remarque}\\
	Si \(\theta\) est un argument de \(z\) complexe non nul, on peut écrire \(\arg(z) \equiv \theta\croch{2\pi}\)
\end{defprop}

\begin{prop}
	Pour tous nombres complexes non nuls \(z\) et \(z'\), on a :
	\begin{enumerate}
		\item \(\arg\paren{zz'} \equiv \arg\paren{z}+\arg\paren{z'}\croch{2 \pi}\) \\
		\item \( \arg\paren{\frac{z}{z'}} \equiv \arg\paren{z}-\arg\paren{z'}\croch{2 \pi}\)
	\end{enumerate}
\end{prop}

\begin{defprop}[Transformation de \(a\cos(t) + b\sin(t)\) en \(A\cos(t-\phi)\)]
	Soit \(a, b\) et \(t\) des nombres réels avec \((a, b)\neq (0, 0)\). On peut écrire
	\[a\cos(t)+b\sin(t) = \Reel{\paren{a-\i b}\paren{\cos(t)+\i \sin(t)}} = \Reel{(a-\i b)e^{\i t}} \]
	puis \(a-\i b = Ae^{-\i \phi} \) avec \(A\) réel strictement positif et \(\phi\) un réel ce qui donne :
	\[a\cos(t)+b\sin(t) = \Reel{(a-\i b)e^{\i t}} = \Reel{Ae^{\i(t-\phi)}} \]
	Donc \(a\cos(t)+b\sin(t) = A\cos(t-\phi)\)
\end{defprop}

\section{Fonctions d'une variable réelle à valeurs complexes}

\begin{defi}
	Une fonction de variable réelle à valeurs complexes notée \(f\) est un objet mathématique qui, tout élément \(x\) d’une partie non vide de \(\R\), associe un et un seul nombre complexes noté \(f (x)\).
\end{defi}

\begin{defprop}[Ce qui s’étend aux fonctions de variable réelle à valeurs complexes]
	\begin{itemize}
		\item Notation fonctionnelle
		\item Domaine de définition
		\item Image d’un réel, antécédent d’un complexe
		\item Parité, imparité, périodicité
		\item Somme, produit, quotient de fonctions et multiplication d’une fonction par un complexe
		\item Dérivation
	\end{itemize}
\end{defprop}


\begin{defprop}[Ce qui ne s’étend pas aux fonctions de variable réelle à valeurs complexes]
	\begin{itemize}
		\item Composition de fonctions
		\item Monotonie
		\item Fonction majorée, minorée ou bornée
		\item Fonction réciproque
	\end{itemize}
\end{defprop}

\begin{defprop}[Dérivation]
	Soit \(I\) un intervalle de \(\R\) non vide et non réduit à un point.
	Soit \(f\) une fonction définie sur \(I\) à valeurs complexe. \\
	On note \(\Reel{f} :I \to \R\) et \(\Ima{f}:I\to\R\) les fonctions d’une variable réelle à valeurs réelles définies par :
	\[\quantifs{\forall x \in I}\paren{\Reel{f}}(x) = \Reel{f(x)} \text{ et } \paren{\Ima{f}}(x) = \Ima{f(x)} \]
	On dit que : \begin{itemize}
		\item \(f\) est dérivable en \(x_0\) si les fonctions \(\Reel{f}\) et \( \Ima{f} \) sont dérivables en \(x_0\)
		\item \(f\) est dérivable sur \(I\) si les fonctions \(\Reel{f}\) et \( \Ima{f} \) sont dérivables sur \(I\)
	\end{itemize}
	Selon le cas de figure, on appelle :
	\begin{itemize}
		\item nombre dérvée de \(f\) en \(x_0\) et on note \(f'(x_0)\) le nombre complexe suivant : \[f'(x_0) = \paren{\Reel{f}'(x_0)} + \ \paren{\Ima{f}'(x_0)}\]
		\item fonction dérivée de \(f\) sur \(I\) et on note \(f'\) la fonction de variable réelle à valeurs complexes suivante :
		      \[ f' = \paren{\Reel{f}'} + \ \paren{\Ima{f}'}\]
	\end{itemize}
\end{defprop}

\begin{prop}
	\begin{enumerate}
		\item \underline{Combinaison linéaire}\\
		      Soit \(f\) et \(g\) deux fonctions définies sur \(I\) et à valeurs complexes et \((\alpha, \beta)\) un couple de complexes. Si \(f\) et \(g\) sont dérivables sur \(I\) alors \(\alpha f + \beta g \) est dérivable sur \(I\) et sa dérivée vérifie :
		      \[\paren{\alpha f + \beta g}' = \alpha f'+\beta g'\]
		\item \underline{Produit}\\
		      Soit \(f\) et \(g\) deux fonctions définies sur \(I\) et à valeurs complexes . Si \(f\) et \(g\) sont dérivables sur \(I\) alors \(fg\) est dérivable sur \(I\) et sa dérivée vérifie :
		      \[\paren{fg}' = f'g+fg'\]

		\item \underline{Quotient}\\
		      Soit \(f\) et \(g\) deux fonctions définies sur \(I\) et à valeurs complexes tel que \(g\) ne s’annule pas sur \(I\). Si \(f\) et \(g\) sont dérivables sur \(I\) alors \(\frac{f}{g}\) est dérivable sur \(I\) et sa dérivée vérifie :
		      \[\paren{\frac{f}{g}}' = \frac{f'g-g'f}{g^2}\]
	\end{enumerate}
\end{prop}

\begin{appl}[exemple important]
	Soit \(\phi\) une fonction définie sur \(I\) à valeurs complexes.
	On note \(f : I \to \C\) la fonction définie sur \(I\) par :
	\[\forall t \in I, f(t) = e^{\Reel{\phi(t)}}e^{\i\Ima{\phi(t)}}\]
	Si \(\phi\) est dérivable sur \(I\) alors \(f\) est dérivable sur \(I\) et sa dérivée vérifie :
	\[\forall t \in I,f'(t) = \phi '(t)f(t) \]
	\underline{Remarque} \\
	La fonction \(f\) sera aussi notée \(f = \exp(\phi)\) après étude de l’exponentielle complexe dans le chapitre \guillemets{Nombres complexes (\(2\))} ce qui permettra d’écrire \((\exp(\phi))' = \phi' \exp(\phi)\) et donc d’étendre une propriété déjà connue dans le cas où \(\phi\) est à valeurs réelles.
\end{appl}