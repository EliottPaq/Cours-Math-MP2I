\chapter{Arithmétique dans \(\Z\)}

\minitoc

\section{Division euclidienne}
Soit \((a, b, c, d) \in \Z^4\).
\subsection{Divisibilité dans \(\Z\)}
\begin{defi}
    S’il existe \(q\) dans \(\Z\) tel que \(a = bq\), on dit que \(b\) divise \(a\) (ou \(b\) est un diviseur de \(a\)) et on note \(b \divise a\).\\~\\
    Dans ce cas,\\~\\
    on dit aussi que \(a\) est divisible par \(b\) (ou \(a\) est un multiple de \(b\)).
\end{defi}

\begin{defprop}[Ensembles des diviseurs et des multiples]
    \begin{itemize}
        \item On note \(\mathcal{D}(a) = \accol{b \in \Z \tq \exists q \in \Z, a = bq}\) l’ensemble des diviseurs de \(a\).
        \begin{itemize}
            \item Si \(a = 0\) alors \(\mathcal{D}(a) = \Z\) donc \(\mathcal{D}(a)\) est infini.
            \item Si \(a \neq 0\) alors \(\mathcal{D}(a) \subset \interventierii{-\abs{a}}{\abs{a}} \) donc \(\mathcal{D}(a)\) est fini.
        \end{itemize}
        \item On note \(b\Z = \accol{bq \tq q \in \Z}\) l’ensemble des multiples de \(b\).
        \begin{itemize}
            \item Si \(b = 0\) alors \(b\Z = \accol{0}\) donc \(b\Z\) est fini.
            \item Si \(b \neq 0\) alors \(b\Z\) est infini.
        \end{itemize}
    \end{itemize}
\end{defprop}

\begin{defprop}[Caractérisation des couples d’entiers associés]
    Si l’une des propositions équivalentes suivantes est vérifiée, on dit que les entiers \(a\) et \(b\) sont associés.
    \begin{enumerate}
        \item \(a \divise b\) et \(b \divise a\).
        \item \(\abs{a} = \abs{b}\).
        \item \(a = b\) ou \(a = -b\).
    \end{enumerate}
\end{defprop}

\begin{defprop}[Propriétés immédiates]
    \begin{enumerate}
        \item \(a \divise a\).
        \item Si \(a \divise b\) et \(b \divise c\) alors \(a \divise c\).
        \item Si \(a \divise b\) et \(c \divise d\) alors \(ac \divise bd\).
        \item Si \(a \divise b\) alors, pour tout \(n \in \Ns\), \(an\divise bn\).
        \item Si \(c \divise a\) et \(c \divise b \) alors, pour tout \( (u, v) \in\Z^2, c \divise au + bv\).
        \item Si \(a = bc + d\) alors \(\mathcal{D}(a) \inter \mathcal{D}(b) = \mathcal{D}(b) \inter \mathcal{D}(d)\).
    \end{enumerate}
\end{defprop}

\subsection{Division euclidienne}
\begin{theo}[Théorème de la division euclidienne]
    Pour tout couple \((a, b)\) de \(\Z \times \Ns\), il existe un unique couple \((q, r)\) de \(\Z^2\) tel que :\\~\\
    \[a = bq + r \text{ et } 0 \leq r \leq b - 1.\]
    Dans la division euclidienne de \(a\) par \(b\), \(a\) est appelé dividende, \(b\) diviseur, \(q\) quotient et \(r\) reste.
\end{theo}

\begin{dem}
    Soit \((a,b)\in \Z \times \Ns\)\\~\\
    \existence posons \(q = \floor{\frac{a}{b}}\) et \(r = a-bq\)\\~\\
    alors \(q\leq \frac{a}{b}<b+1\)\\~\\
    donc \(bq\leq a < b(q+1) \iff 0\leq r<b\).\\~\\
    De plus \(q \in \Z\) donc \(r \in \Z\) ainsi avec ce qui précède on a : \(r \in \interventierii{0}{b-1}\)\\~\\
    \unicite On suppose qu'il existe \(\begin{cases}
        (q,r) &\in \Z \times \interventierii{0}{b-1}\\
        (q',r') &\in \Z \times \interventierii{0}{b-1}\\
    \end{cases}\) tel que \(\begin{cases}
        a &= bq+r\\
        a &= bq'+r'
    \end{cases}\)
    \\~\\
    Alors \(b(q-q') = r'-r\) Ainsi\\~\\
    Si \(q-q' \neq 0\) alors \(\abs{q-q'}\geq 1\) puis \(\abs{r'-r}\geq \abs{b}\) et donc \(\abs{r'-r}\geq b\) car \(b>0\).\\~\\
    or \(\abs{r'-r\leq b-1}\) car \(\begin{cases}
        0\leq r \leq b-1\\
        0\leq r' \leq b-1
    \end{cases}\) \\~\\
    Donc \(q-q' = 0 \imp q=q'\) et aussi \(r-r' = 0\imp r = r'\)
    \conclusion Le couple \(q,r\) existe et est unique 
\end{dem}


\begin{defprop}[Caractérisation de la divisibilité] 
    Soit \((a, b) \in \Z \times \Ns\).\\~\\
    \(b\) divise \(a\) si, et seulement si, le reste de la division euclidienne de \(a\) par \(b\) est nul.
\end{defprop}

\section{PGCD et PPCM}
\subsection{Cas de deux entiers naturels}
\begin{defi}[Définition du PGCD]
    Soit \((a, b) \in \N \times \Ns\).\\~\\
    Le plus grand élément (au sens de \(\leq\)) de l’ensemble des diviseurs communs à \(a\) et \(b\) est dit PGCD de \(a\) et \(b\) et noté \(a \wedge b\) :
    \[a \wedge b = \max (\mathcal{D}(a) \inter \mathcal{D}(b))\]
\end{defi}

\begin{prop}[Propriété importante]
    
    Soit \((a, b) \in \N \times \Ns\).\\~\\
    Si \(r\) est le reste de la division euclidienne de \(a\) par \(b\) alors \(\begin{cases}
        \mathcal{D}(a) \inter \mathcal{D}(b) &= \mathcal{D}(b) \inter \mathcal{D}(r) \\
        a\wedge b                            &= b\wedge r
    \end{cases}\)
\end{prop}

\begin{dem}
    Soit \((a, b) \in \N \times \Ns\), on note \(r \in \N\) le reste de la division euclidienne de \(a\) par \(b\).
    Montrons que \(\begin{cases}
        \mathcal{D}(a) \inter \mathcal{D}(b) &= \mathcal{D}(b) \inter \mathcal{D}(r) \\
        a\wedge b                            &= b\wedge r
    \end{cases}\) par double inclusion :
    \begin{itemize}
        \item Montrons que \(\mathcal{D}(a) \inter \mathcal{D}(b) \subset \mathcal{D}(b) \inter \mathcal{D}(r)\)\\~\\
            Soit \(d \in \mathcal{D}(a) \inter \mathcal{D}(b)\) alors : \\~\\
            \(d \divise b\) donc \(d \divise bq\) avec \(q \in \Zs\) et \(d \divise a\) donc \(d \divise a-bq\) or \(r = a-bq\) donc \(d \divise r\) et \(d \divise b\) donc \(d \in \mathcal{D}(b) \inter \mathcal{D}(r)\).
        \item De même Montrons que \(  \mathcal{D}(b) \inter \mathcal{D}(r) \subset \mathcal{D}(a) \inter \mathcal{D}(b)\)
         Soit \(d \in \mathcal{D}(b) \inter \mathcal{D}(r)\) alors :\\~\\
         alors \(d \divise b\) donc \(d \divise bq\) avec \(q \in \Zs\) et \(d \divise r\) donc \(d \divise bq+r\) or \(a = bq+r\) donc \(d \divise a\) et \(d \divise b\) donc \(d \in \mathcal{D}(b) \inter \mathcal{D}(a)\).
        
    \end{itemize}
    \conclusion Par double inclusion \(\mathcal{D}(a) \inter \mathcal{D}(b) = \mathcal{D}(b) \inter \mathcal{D}(r)\) et ainsi par définition du PGCD \(a\wedge b = b\wedge r\)
\end{dem}

\begin{defprop}[algorithme d'euclide]
    Soit \((a, b) \in \N \times \Ns\).\\~\\
    On pose \(r_0 = a\) et \(r_1 = b\) puis, pour tout \(i \in \Ns\) tel que \(r_i \neq 0\), on définit \(r_{i+1}\) comme suit :\\~\\
    \[r_{i+1}\text{ est le reste de la division euclidienne de } r_{i-1}\text{ par }r_i.\]
    Alors :
    \begin{itemize}
        \item il existe \(n \in N\) tel que
    \[r_{n+1} = 0\text{ et }r_n \neq 0\]
        \item pour tout \(i \in \interventierii{1}{n} \),
    \[r_{i-1} \wedge r_i = r_i \wedge r_{i+1}\]
    \end{itemize}
    En particulier \(r_0 \wedge r_1 = r_n \wedge r_{n+1}\) donc :
    \[a \wedge b = r_n\]
\end{defprop}

\begin{defprop}[Caractérisation du PGCD]
    Soit \((a, b) \in \N \times \Ns\).\\~\\
    L’ensemble des diviseurs communs à \(a\) et \(b\) est égal à l’ensemble des diviseurs de \(a \wedge b\) :
    \[\mathcal{D}(a) \inter \mathcal{D}(b) = \mathcal{D}(a \wedge b)\]
    Le PGCD de \(a\) et \(b\) est donc le plus grand élément (au sens de la divisibilité) de l’ensemble des diviseurs communs à \(a\) et \(b\), c’est-à-dire que :
    \begin{itemize}
        \item \(a \wedge b \divise a\) et \(a \wedge b \divise b\)
        \item\( \forall d \in \N, d \divise a\) et \( d \divise b \imp d \divise a \wedge b\).
    \end{itemize} 
\end{defprop}

\begin{defprop} [Propriété de factorisation du PGCD]
    Soit \((a, b) \in \N \times \Ns\).\\~\\
    Pour tout \(k \in \Ns\), le PGCD de \(ka\) et \(kb\) vérifie
    \[ka \wedge kb = k (a \wedge b)\]
\end{defprop}

\begin{dem}
    Soit \((a, b) \in \N \times \Ns\) et \(k \in \Ns\)
    \begin{itemize}
        \item \(\begin{cases}
            a\wedge b \divise a \\
            k \divise k
        \end{cases}\) 
        et 
        \(\begin{cases}
            a\wedge b \divise b \\
            k \divise k
        \end{cases}\)
         donc par propriété 
         \(\begin{cases}
            k\paren{a\wedge b} &\divise ka \\
            k\paren{a\wedge b} &\divise kb
         \end{cases}\) donc \(k\paren{a\wedge b} \divise ka\wedge kb\) car \(\mathcal{D}(ka)\inter \mathcal{D}(kb) = \mathcal{D}(ka\wedge kb)\)
         \item \(k \divise ka\) et \(k \divise kb\) donc \(k \divise ka \wedge kb\) donc \(\exists q \in \N\) tel que \(ka \wedge kb = kq\)\\~\\
         Ainsi \(kq\divise ka\) et \(kq\divise kb\) et donc \(q\divise a\) et \(q \divise b\)\\~\\
         ainsi \(q \divise a\wedge b\) puis \(kq \divise k\paren{a\wedge b}\) et donc enfin \(ka \wedge kb \divise k\paren{ a \wedge b}\)
    \end{itemize}
    Ainsi on a \(k\paren{a\wedge b} \divise ka\wedge kb\) et \(ka \wedge kb \divise k\paren{ a \wedge b}\) \\~\\
    donc \(k\paren{a\wedge b} \) et \(ka\wedge kb\) sont associés et donc égaux, car ce sont des entiers naturels non-nuls donc \(ka \wedge kb = k (a \wedge b)\)
\end{dem}

\subsection{Cas de deux entiers relatifs}

\begin{defi}
    Soit \((a, b) \in \Z^2\). \\~\\
    On appelle PGCD de \(a\) et \(b\) l’entier naturel noté \(a \wedge b\) défini par :
    \[a \wedge b = \begin{cases}
        \abs{a} \wedge \abs{b} &\text{ si } (a, b) \neq (0, 0) \\
        0 &\text{ si } (a, b) = (0, 0)
    \end{cases}
    \]
\end{defi}

\begin{defprop}[Extension des résultats vus pour les entiers naturels]
    \begin{enumerate}
        \item Soit \((a, b) \in \Z^2 \pd \accol{(0, 0)}\) .
        \begin{enumerate}
            \item \(a \wedge b\) est le plus grand élément (au sens de \(\leq\)) de l’ensemble des diviseurs communs à \(a\) et \(b\).
            \item \(a \wedge b\) est le plus grand élément (au sens de \(\divise\) ) de l’ensemble des diviseurs communs à \(a\) et \(b\).
        \end{enumerate}
        \item Soit \((a, b) \in \Z^2\).
        \begin{enumerate}
            \item \(\mathcal{D}(a) \inter \mathcal{D}(b) = \mathcal{D}(a \wedge b)\).
            \item Pour tout \(k \in \Z, ka \wedge kb = \abs{k} (a \wedge b) \)
        \end{enumerate}
    \end{enumerate}
\end{defprop}

\begin{defprop}[Relation de Bézout]
    Soit \((a, b) \in \Z^2\).\\~\\
    Il existe un couple d’entiers \((u, v) \in \Z^2\), dit couple de Bézout, tel que \(au + bv = a \wedge b\).
    \underline{Remarque}
    \begin{itemize}
    \item  Un tel couple n’est PAS UNIQUE.
    \item  Pour \((a, b) \neq (0, 0)\), on peut déterminer un tel couple par l’algorithme d’Euclide étendu.\\~\\
    on a : \((r_0, r_1) = (a, b), r_{i-1} = r_iq_i + r_{i+1}\) (division euclidienne de \(r_{i-1}\) par \(r_i\)) et \(n\) le plus petit entier tel que \(r_{n+1} = 0\). Ainsi, en posant
    \[\begin{cases}
    (u_0, v_0) &= (1, 0)
    (u_1, v_1) = (0, 1)\end{cases}\text{ et, pour tout }i \in \interventierii{1}{N} , (u_{i+1}, v_{i+1}) = (u_{i-1} - q_i u_i, v_{i-1} - q_iv_i)\] .
    on a : \(\forall i \in \interventierii{0}{n} , a u_i + b v_i = r_i\). En particulier, comme \(r_n\) est égal à \(a \wedge b\), on en déduit que :
    \[a \wedge b = au_n + bv_n\text{ avec }(u_n, v_n) \in \Z^2\]
    \item Il n’est pas nécessaire de connaître les relations de récurrence définissant les familles \( (u_i)_{0\leq i \leq n}\) et \((v_i)_{0\leq i \leq n}\).
    \end{itemize}
\end{defprop}

\subsection{PPCM}
\begin{defi}
    Soit \((a, b) \in \Z^2\). Le PPCM de \(a\) et \(b\) est l’entier naturel noté \(a \vee b\) défini par
    \[a \vee b = \begin{cases}
    \min \paren{\abs{a} \Ns \inter \abs{b} \Ns} &\text{ si } a\neq 0\text{ et }b \neq 0 \\
    0 &\text{ si } a = 0 \text{ ou } b = 0 
    \end{cases}
    \]
    \underline{Remarques}\\~\\
        \begin{itemize}
            \item Pour tout \(a \in \Z, a \vee a = a \vee 1 = \abs{1}\).
            \item Pour tout \((a, b) \in (\Zs)^2\), \(a \vee b\) est le plus petit entier naturel non nul, multiple commun de \(a\) et \(b\).
        \end{itemize}
\end{defi}

\begin{prop}
    Pour tout \((a, b) \in \Z^2\), on a :
    \[\abs{ab} = (a \wedge b) (a \vee b)\]  
\end{prop}

\begin{dem}
    \(\forall (a,b) \in \Z^2\)\\~\\
    \begin{itemize}
        \item si \((a,b)\neq(0,0)\) alors prenons \(k \in \N\) \\~\\
        alors : 
        \begin{align*}
            k \in \abs{a}\Ns\inter\abs{b}\Ns &\iff \abs{a}\divise k \text{ ou } \abs{b}\divise k \\
            &\iff \abs{ab}\divise \paren{k\abs{a}\wedge k \abs{b}} \\
            &\iff \abs{ab} \divise k \paren{\abs{a}\wedge \abs{b}}\\
            &\iff \exists q\in \N,k \paren{\abs{a}\wedge \abs{b}} = q\abs{ab} \\
            &\iff \exists q\in \N,k q\frac{\abs{ab}}{\paren{\abs{a}\wedge \abs{b}}}\\
            &\iff \frac{\abs{ab}}{\paren{\abs{a}\wedge \abs{b}}} \divise k\\
            &\iff k \in \frac{\abs{ab}}{\paren{\abs{a}\wedge \abs{b}}} \Ns \\
            &\iff \abs{a}\Ns\inter\abs{b}\Ns = \frac{\abs{ab}}{\paren{\abs{a}\wedge \abs{b}}} \Ns 
        \end{align*}
        Donc \(a\vee b = \frac{\abs{ab}}{\paren{\abs{a}\wedge \abs{b}}}\).\\~\\
        Ainsi\(\abs{ab} =\paren{a \vee b}\paren{\abs{a}\wedge \abs{b}} \imp \abs{ab} =\paren{a \vee b}\paren{a\wedge b}\)
        \item De plus si \((a,b) = (0,0)\) alors on a toujours \(\abs{ab} =\paren{a \vee b}\paren{a\wedge b}\) car \(\begin{cases}
            \abs{ab} &= 0 \\
            \paren{a \vee b}\paren{a\wedge b}&=0
        \end{cases}\)
    \end{itemize}
\end{dem}

\section{Entiers premiers entre eux}
\subsection{Cas de couples d’entiers}
Soit \((a, b, c, n) \in \Z^4\)

\begin{defi}
    Les entiers \(a\) et \(b\) sont dits premiers entre eux si leur PGCD est égal à \(1\).
    \underline{Remarque}\\~\\
    Autrement dit, \(a\) et \(b\) sont premiers entre eux si leurs seuls diviseurs communs sont \(-1\) et \(1\).
\end{defi}

\begin{theo}[Théorème de Bézout]
    \(a\) et \(b\) sont premiers entre eux si, et seulement si, il existe \((u, v) \in \Z^2\) tel que \(au + bv = 1\)
\end{theo}

\begin{dem}[Théorème de Bézout]
    Montrons le théorème par double inclusion\\~\\
    \begin{itemize}
        \item \impdir immédiat par relation de Bézout
        \item \imprec On suppose \(\exists (u,v)\in \Z^2,au+bv = 1\)\\~\\
        Soit \(d \in \N\) tq \(d \divise a\) et \(d \divise b\) alors \(d\divise au+bv\) donc \(d \divise 1\) donc \(d=1\) ainsi \(a\wedge b = 1\)

    \end{itemize}
\end{dem}

\begin{theo}[Lemme de Gauss]
    Si \(c\) divise \(ab\) et si \(a\) et \(c\) sont premiers entre eux alors \(c\) divise \(b\).\\~\\
    \underline{Remarque}\\~\\
        Tout nombre rationnel r non nul peut s’écrire sous la forme \(r = \frac{a}{b}\) avec \((a, b) \in \Zs \times \Ns \) et \(a \wedge b = 1\).\\~\\
        Cette écriture est unique et appelée forme irréductible de \(r\).
\end{theo}

\begin{prop}[Propriétés sur le produit]
    \begin{enumerate}
        \item Si \(a\) et \(b\) sont premiers entre eux et si \(a\) et \(b\) divisent \(n\) alors \(ab\) divise \(n\).
        \item Si \(a\) et \(n\) sont premiers entre eux et si \(b\) et \(n\) sont premiers entre eux alors \(ab\) et \(n\) sont premiers
        entre eux.
    \end{enumerate}
\end{prop}

\begin{dem}
    Soit \((a,b,n) \in \Z^3\)\\~\\
    Démontrons les deux propriétés précédentes
    \begin{enumerate}
        \item par hypothèse \(\begin{cases}
        \exists q \in \Z, n=aq\\
        \exists q' \in \Z, n=bq'
        \end{cases}\)
        donc \(aq = bq'\) avec \(a\wedge b = 1\) donc \(b\divise q\)\\~\\
        \(\exists q'' \in \Z\) tel que \(q = bq''\) ce qui donne \(n = abq''\) donc \(ab \divise n\)
        \item par hypothèse et d'après le théorème de bézout on a :
        \[\exists (u,v) \in \Z^2 \text{ tel que } au+nv=1\]
        \[\exists (u',v') \in \Z^2 \text{ tel que } bu'+nv'=1\]
        donc par multiplication membre à membre on a  : 
        \[ab(u'u) + n(bvu' + nvv'+auv') = 1\]
        donc d'après le théorème de bézout \(ab \wedge n =1\)
    \end{enumerate}
\end{dem}

\subsection{Cas de \(n\)-uplet d’entiers avec \( n \geq 2\) }
Soit \(n \in \N\) avec \(n \geq 2\) et \((a_1, \dots , a_n) \in \Z^n\).

\begin{defprop}[PGCD d’un nombre fini d’entiers]*
    On appelle PGCD des entiers \(\paren{a_1, \dots , a_n}\) l’entier naturel, noté \(a1 \wedge \dots \wedge a_n\), tel que
    \[\mathcal{D}\paren{a_1 \wedge \dots \wedge a_n} = \mathcal{D}(a_1) \inter \dots \inter \mathcal{D}(a_n)\]
\end{defprop}

\begin{defprop}[Relation de Bézout]
    Il existe un \(n\)-uplet d’entiers \((u_1, \dots , u_n) \in \Z^n\) tel que  \(a_1u_1 +\dots + a_nu_n = a_1 \wedge \dots \wedge a_n\)
\end{defprop}

\begin{defprop}[Entiers premiers entre eux]
    Les entiers \(a_1, \dots , a_n\) sont dits :
    \begin{itemize}
        \item premiers entre eux dans leur ensemble si \(a_1 \wedge \dots \wedge a_n = 1\).
        \item premiers entre eux deux à deux si \(\forall (i, j) \in \interventierii{1}{n} , i \neq j \imp a_i \wedge a_j = 1\).
    \end{itemize}
\end{defprop}
\begin{dem}[existence et unicité de la forme irréductible de tout rationnel non nul]
    Montrons l'existence et unicité de la forme irréductible de tout rationnel non nul autrement dit \(\forall r \in \Qs, \exists ! (a',b') \in \Zs \times \Ns, r = \frac{a'}{b'}\)
    \begin{itemize}
        \item \existence Soit \(r \in \Qs \) alors par définition \(\exists (a,b) \in \Zs \times \Ns, r = \frac{a'}{b'}\)\\~\\
        on note \(d = a\wedge b\) alors on note \(\begin{cases}
            a = da' &\text{ avec } a'\in \Zs\\
            b = db' &\text{ avec } b' \in \Zs
        \end{cases}\)\\~\\
        ce qui donne \(r = \frac{da'}{db'} = \frac{a'}{b'}\) avec \(a'\wedge b' = \frac{d\paren{a'\wedge b'}}{d} = \frac{da' \wedge db'}{d} = \frac{a \wedge b}{d} = 1\)
        \item \unicite Soit \(r \in \Qs\) tel que \(r = \frac{a'}{b'} = \frac{a''}{b''}\) avec \(\begin{cases}
            a'\wedge b' &= 1 \\
            a'' \wedge b'' &= 1
        \end{cases}\)\\~\\
        on en déduit que \(a'b'' = a'' b'\) ce qui donne \(b''\divise a''b'\) puis \(b'' \divise b'\) car \(a'\wedge b' =1\) \\~\\
        de même \(b'\divise b''\) donc \\~\\
        \(b'\) et \(b''\) sont associés et entier naturel et donc égaux ce qui donne \(a' = a''\) et \(b' = b''\) ce qui prouve l'unicité
    \end{itemize}
\end{dem}
\section{Nombres premiers}
\subsection{Généralités}
\begin{defi}
    Un nombre entier naturel non nul \(p\) est dit premier s’il admet uniquement deux diviseurs entiers naturels distincts (qui sont \(1\) et \(p\))
\end{defi}

\begin{defprop}
    Ensemble des nombres premiers
L’ensemble \(\mathcal{P}\) des nombres premiers est infini.
\end{defprop}
\begin{dem}
    Par l'absurde, supposons que \(\mathcal{P}\) est fini \cad \(\mathcal{P = \accol{p_1 \dots p_n}}\)\\~\\
    On pose \(N = \paren{\prod_{i = 1}^{n} p_i} + 1\)\\~\\
    alors \(\begin{cases}
        N \in \N \\
        N \geq 2
    \end{cases}\) donc \(N\) admet un diviseur premier \(i_0\) \\~\\
    \(\exists i_0 \in \interventierii{1}{n} , \begin{cases}
        p_{i_0} \divise N \\
        p_[i_0] \divise \prod_{i=1}^{n} p_i
    \end{cases}\) donc \(p_{i_0} \divise 1\)\\~\\
    d'où \(p_{i_0} = 1\) ce qui est faux car \(p_{i_0}\) est premier\\~\\
    \conclusion \(\mathcal{P}\) est infini    
\end{dem}
\subsection{Décomposition en produit de nombres premiers}

\begin{theo}
    Tout entier naturel \(n\) supérieur ou égal à \(2\) peut s’écrire de manière unique (à l’ordre près des facteurs) sous la forme
    \[n = \prod_{i=1}^{k} p_i^{\alpha_i}\]
    où \(k \in \Ns\) avec \(\forall i \in \interventierii{1}{k} , \alpha_i \in \Ns\) et \(p_i\) est un nombre premier.
\end{theo}

\begin{defprop}[Corrolaire]
    Tout entier naturel non nul \(n\) s’écrit de manière unique (à l’ordre près des facteurs) sous la forme
    \[n = \prod_{p \in \mathcal{P}}p^{\alpha_p}\]
    où \(\paren{\alpha_p}_{p \in \mathcal{P}}\) est une famille presque nulle d’entiers naturels, c’est-à-dire une famille dans laquelle tous les éléments sont nuls sauf un nombre fini d’entre eux.
\end{defprop}

\subsection{Valuation \(p\)-adique}
\begin{defprop}
    Soit \(p\) un nombre premier et \(n\) un entier naturel non nul.\\~\\
    L’entier \(\alpha_p\) qui apparaît dans la décomposition primaire de \(n\)
    \[n = \prod_{p \in \mathcal{p}} p ^{\alpha_p}\]
    est appelé valuation \(p\)-adique de \(n\) et noté \(v_p(n)\) .\\~\\
    \underline{Autrement dit}\\~\\
    \(v_p(n)\) est le plus grand entier naturel \(k\) tel que \(p^k\) divise \(n\)
\end{defprop}

\begin{defprop}[Valuation \(p\)-adique d’un produit]
    Pour tout nombre premier \(p\) et tous entiers naturels non nuls \(n\) et \(n'\), on a :
    \[v_p(nn') = v_p(n) + v_p(n')\]
\end{defprop}

\begin{defprop}[Caractérisation de la divisibilité]
    Soit \((a, b) \in (\Ns)^2\).\\~\\
    \(b\) divise \(a\) si, et seulement si, pour tout nombre premier \(p\), on a : \(v_p(b) \leq v_p(a)\)
\end{defprop}

\begin{dem}
    Soit \((a,b) \in \N^2\), procédons par double équivalence
    \begin{itemize}
        \item \impdir on suppose \(b\divise a\)\\~\\
        alors \(\exists q \in \N, a = bq\) donc \(\forall p \in \mathcal{P}, v_p(a) = v_p(b) + v_p(q) \imp v_p(a) \geq v_p(b)\)
        \item \imprec on suppose \(\forall p \in \mathcal{P}, v_p(a) \geq v_p(b)\)\\~\\
        alors \(p^{v_p(a)} = p^{v_p(a) - v_p(b)} \times p^{v_p(b)}\) donc \(p^{v_p(b)} \divise p^{v_p(a)}\) ainsi\\~\\
        \(\prod_{p \in \mathcal{P}}p^{v_p(b)} \divise \prod_{p \in \mathcal{P}}p^{v_p(a)}\) donc \(b \divise a\)
    \end{itemize}
\end{dem}

\begin{defprop}[PGCD et PPCM]
    Soit \((a, b) \in (\Ns)^2\).\\~\\
    Les PGCD et PPCM des entiers a et b vérifient :
    \[a\wedge b = \prod_{ p \in \mathcal{P}} p^{\min\paren{v_p(a),v_p(b)}}\]
    \[a\vee b = \prod_{p \in \mathcal{P}}p^{\max\paren{v_p(a),v_p(b)}}\]
\end{defprop}

\begin{dem}
    Soit \((a,b) \in \paren{\Ns}^2\)
    \begin{itemize}
        \item Montrons que \(a \wedge b = \prod_{ p \in \mathcal{P}} p^{\min\paren{v_p(a),v_p(b)}}\)\\~\\
         on note \(d = \prod_{ p \in \mathcal{P}} p^{\min\paren{v_p(a),v_p(b)}}\)\\~\\
         \(\forall p \in \mathcal{P},v_p(d) = \min\paren{v_p(a),v_p(b)}\) \\~\\
         donc \(\forall p \in \mathcal{P}, \begin{cases}
            v_p(d) &\leq v_p(a) \\
            v_p(d) &\leq v_p(b)
         \end{cases}\) d'où \(\begin{cases}
            d&\divise a \\
            d &\divise b
         \end{cases}\)\\~\\
         on note alors \(\begin{cases}
            a = da' &\text{ avec } a'\in \Ns\\
            b = db' &\text{ avec } b' \in \Ns
        \end{cases}\)
        Montrons alors que \(a'\wedge b' = 1\) ce qui donnerais alors \(a \wedge b = d\paren{a' \wedge b'} = d\)\\~\\
        Soit \(k\) un diviseur commun à \(a'\) et \(b'\) différent de \(1\) alors \(k \geq 2\) donc \(k\) admet un diviseur premier \(p'\)\\~\\
        donc \(\begin{cases}
            p' \divise a' \\
            p' \divise b'
        \end{cases}\) d'où \(\begin{cases}
            v_{p'}(p') \leq v_{p'}(a)\\
            v_{p'}(p') \leq v_{p'}(b)
        \end{cases}\) \ie \(\begin{cases}
            1\leq v_{p'}(a) & \qquad (1)\\
            1 \leq v_{p'}(b) & \qquad (2)
        \end{cases}\)
        Si \(v_{p'}(a) \leq v_{p'}(b)\) alors \(v_{p'}(d) = v_{p'}(a)\) et \(v_{p'}(a) = v_{p'}(d) + v_{p'}(a')\) d'où \(v_{p'}(a') = 0\) ce qui contredit \((1)\)\\~\\
        Si \(v_{p'}(b) < v_{p'}(a)\) alors \(v_{p'}(d) = v_{p'}(b)\) donc \(v_{p'}(b') = 0\)\\~\\
        Donc \(k=1\) d'où \(a'\wedge b' = 1\) d'où \(a \wedge b = d\)
        \item Montrons que \(a\vee b = \prod_{p \in \mathcal{P}}p^{\max\paren{v_p(a),v_p(b)}}\)
        on a :\begin{align*}
            a \vee b = \frac{ab}{a\wedge b} &= \frac{\paren{\prod_{p \in \mathcal{P}} p ^{v_p(a)}}\paren{\prod_{p \in \mathcal{P}} p ^{v_p(b)}}}{\prod_{ p \in \mathcal{P}} p^{\min\paren{v_p(a),v_p(b)}}} \\
            & = \prod_{p \in \mathcal{P}}p^{\max\paren{v_p(a),v_p(b)}}*
        \end{align*}
        \end{itemize}
\end{dem}

\subsection{Congruences}
Soit \((x, y, z, t) \in \Z^4\) et \(n \in \Ns\).

\begin{defi}
    \(x\) est dit congru à \(y\) modulo \(n\) s’il existe \(k \in \Z\) tel que \(x = y + nk\) autrement dit si \(x - y \in n\Z\).\\~\\
    \underline{Notation} : \( x \equiv y \croch{n}\)
\end{defi}

\subsection{Caractérisation}
\begin{defprop}
    \(x \equiv y \croch{n}\) si, et seulement si, les restes des divisions euclidiennes de \(x\) et \(y\) par \(n\) sont égaux.
\end{defprop}

\begin{dem}
    Soit \(\paren{x,y} \in \Z\) et \(n \in \Ns\) Montrons la caractérisation par double inclusion
    \begin{itemize}
        \item \impdir On suppose \(x \equiv y \croch{n}\) on écrit la division euclidienne de \(y\) par \(n\)\\~\\
        \(\exists (q,r) \in \Z \times \interventierii{0}{n-1}, y =nq+r\)\\~\\
        par hypothèse on a : \(\exists k \in \Z, x = y + nk\)\\~\\
        donc \(x = n(k+q)+r\) est la division euclidienne de \(x\) par \(n\) donc \(r\) est le reste de la division euclidienne de \(x\) et \(y\) par \(n\)
        \item \imprec On suppose que \(x\) et \(y\) ont le même reste dans la division euclidienne de \(x\) et \(y\) par \(n\) \\~\\
        Ainsi \(\exists(k,k') \in \Z^2, \exists r \in \interventierii{0}{n-1} \begin{cases}
            x &= nk+r \\
            y = nk' +r
        \end{cases}\)\\~\\
        donc \(x-y = n(k-k')\) \ie \(x-y \in n\Z\) \ie \(x \equiv y \croch{n}\)
    \end{itemize}
\end{dem}

\subsection{Propriétés}
\begin{defprop}
    \begin{enumerate}
        \item \(x \equiv x \croch{n}\) \hfill (réflexivité)
        \item si \(x \equiv y \croch{n}\) alors \(y \equiv x \croch{n}\) \hfill(symétrie)
        \item si \(x \equiv y \croch{n} \) et \(y \equiv z \croch{n}\) alors \(x \equiv z \croch{n}\) \hfill(transitivité)
    \end{enumerate}
\end{defprop}

\subsection{Opération}
\begin{defprop}
    \begin{enumerate}
        \item  Si \(x \equiv y \croch{n} \) et \(z \equiv t \croch{n} \) alors \(x+z \equiv y+t \croch{n} \)\hfill (compatibilité avec l’addition)
        \item  Si \(x \equiv y \croch{n} \) et \(z \equiv t \croch{n} \) alors \(xz \equiv yt \croch{n} \). \hfill(compatibilité avec la multiplication)
    \end{enumerate}
\end{defprop}

\subsection{Inverse modulo \(n\)}
\begin{defprop}
    \begin{itemize}
        \item Si \(x\) et \(n\) sont premiers entre eux, il existe un couple d’entiers \((u, v)\) tel que \(ux + vn = 1\).\\~\\
            On en déduit que
            \[ux \equiv 1 \croch{n}\]
            et on dit que \(u\) est un inverse de \(x\) modulo \(n\).
        \item Si \(u\) est un inverse de \(x\) modulo \(n\) alors il existe un couple d’entiers \((u, v)\) tel que \(ux + vn = 1\).
            On en déduit que \(x\) et \(n\) sont premiers entre eux.
    \end{itemize}
\end{defprop}

\subsection{Petit Théorème de Fermat}
\begin{theo}
    Si \(p\) est un nombre premier alors : 
    \begin{enumerate}
        \item \(\forall a \in \Z, ap \equiv a \croch{p}\).
        \item \(\forall a \in \Z, a \wedge p = 1 \imp a^{p-1} \equiv 1 \croch{p}\).
    \end{enumerate}
\end{theo}
