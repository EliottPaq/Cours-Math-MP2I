\chapter{Fraction Rationelles}

\minitoc

\section{Généralités}
La construction de l’ensemble des fractions rationnelles étant hors programme, la présentation faite ici est volontairement élémentaire.
\subsection{Le corps \(\K\paren{X}\)}
\begin{defi}
    \begin{itemize}
        \item On appelle fraction rationnelle à coefficients dans \(\K\) tout quotient (formel) du type 
        \[F = \frac{A}{B} \text{ avec }A \in \K\croch{X} \text{ et } B \in \K\croch{X} \pd \accol{0_{\K\croch{X}}}\]
        Si \(A\) et \(B\) sont premiers entre eux, on dit que la fraction rationnelle est irréductible.
        \item L’ensemble des fractions rationnelles à coefficients dans \(K\) est noté \(\K\paren{X}\).
    \end{itemize}
    \underline{Remarques} \\
    \begin{itemize}
        \item Les éléments de \(\K\paren{X}\) se manipulent comme les éléments de \(\Q\).
        \item Tout polynôme \(A\) est identifié à la fraction rationnelle \(\frac{A}{1_\K\croch{X}}\) donc \(\K \croch{X}\)  est inclus dans \(\K\paren{X}\).
        \item Les fractions rationnelles \(\frac{A_1}{B_1}\) et \(\frac{A_2}{B_2}\) sont égales si, et seulement si, \(A_1B_2\) et \(A_2B_1\) sont égaux.
    \end{itemize}
\end{defi}

\begin{defprop}[Opérations sur l’ensemble \(\K\paren{X}\)]
    ~\\
    Pour tous \(F_1 = \frac{A_1}{B_1}\) et \(F_2 = \frac{A_2}{B_2}\) éléments de \(\K\paren{X}\), on pose :
    \[F_1 + F_2 = \frac{A_1B_2 + A_2B_1}{B_1B_2} \text{ et }F_1 \times  F_2 = \frac{A_1A_2}{B_1B_2}\]
    \underline{Remarque} \\
Ces opérations, qui ne dépendent pas de l’écriture des fractions rationnelles \(F_1\) et \(F_2\), prolongent l’addition et la multiplication vues dans \(\K\croch{X}\).
\end{defprop}

\begin{defprop}[Structure de corps]
    L’ensemble \(\paren{\K\paren{X}, +, \times }\) est un corps :
    \begin{itemize}
        \item dont l’élément neutre pour l’addition est le polynôme \(0_{\K\croch{X}}\) ;
        \item dont l’élément neutre pour la multiplication est le polynôme \(1_\K\croch{X}\) ;
        \item qui contient l’anneau intègre \(\paren{\K\paren{X}, +, \times }\).
    \end{itemize}
    \underline{Remarque}\\
    Avec la loi externe \(.\) définie sur \(\K\paren{X}\) par
    \[\forall \lambda  \in K, \forall F = \frac{A}{B} \in \K\paren{X}, \lambda.F = \frac{\lambda A}{B}\]
    \(\paren{\K\paren{X}, +, . }\) est aussi un \(\K\)-espace vectoriel.
\end{defprop}

\subsection{Degré d’une fraction rationnelle}
\begin{defprop}
    ~\\
    Pour tout \(F = \frac{A}{B}\) de \(\K\paren{X}\), on définit le degré de la fraction rationnelle \(F\) , noté \(\deg(F )\), par :
    \[\deg(F ) = \deg(A) - \deg(B)\]
    \underline{Remarque} \\
    Le degré d’une fraction rationnelle, qui ne dépend pas de l’écriture de celle-ci, appartient à \(\Z \union \accol{\pinf}\) et prolonge la notion de degré vu pour les polynômes de \(\K \croch{X}\) 
\end{defprop}
\subsection{Partie entière d’une fraction rationnelle}
\begin{defprop}
    Toute fraction rationnelle \(F = \frac{A}{B}\) de \(\K\paren{X}\) s’écrit de manière unique sous la forme \(F = E + F_1\) avec :
    \begin{itemize}
        \item \(E\) un polynôme de \(\K\croch{X}\), appelé partie entière de la fraction rationnelle \(F\) ;
        \item \(F_1\) une fraction rationnelle de \(\K\paren{X}\) de degré strictement négatif.
    \end{itemize}
    Plus précisément,
    \begin{itemize}
        \item le polynôme \(E\) est le quotient de la division euclidienne de \(A\) par \(B\) ;
        \item la fraction rationnelle \(F_1\) est égale à \(\frac{\R}{B}\) où \(\R\) est le reste de la division euclidienne de \(A\) par \(B\).
    \end{itemize}
    
\end{defprop}
\subsection{}
\begin{defprop}[Zéros et pôles d’une fraction rationnelle]
    ~\\
    Soit \(F = \frac{A}{B}\) une fraction rationnelle irréductible de \(\K\paren{X}\) et \(m \in \N\).
    \begin{itemize}
        \item Toute racine du polynôme \(A\) de multiplicité \(m\) est dite zéro de \(F\) de multiplicité \(m\).
        \item Toute racine du polynôme \(B\) de multiplicité \(m\) est dite pôle de \(F\) de multiplicité \(m\).
    \end{itemize}
\end{defprop}
\subsection{}
\begin{defprop}[Fonction rationnelle]
    Soit \(F = \frac{A}{B}\) une fraction rationnelle irréductible de \(\K\paren{X}\).\\~\\
    La fonction \(\tilde{F} : x \mapsto \frac{A(x)}{B(x)}\) définie sur \(\K\) privé de l’ensemble (fini) des pôles de \(F\) et à valeurs dans \(\K\) est dite fonction rationnelle associée à la fraction rationnelle \(F\).
\end{defprop}
\section{Décomposition en éléments simples}
\subsection{Théorème de décomposition sur \(\C\) (preuve hors programme)}
\begin{defprop}
    Si \(F\) est une fraction rationnelle de \(\C\paren{X}\) de partie entière \(E\) et de pôles deux-à-deux distincts \(\lambda_1, \dots , \lambda_r\) de multiplicités \(m_1, \dots , m_r\) alors il existe une unique famille de complexes \((a_{k,p})_{(k,p)\in \interventierii{1}{r}\times \interventierii{1}{m_k}}\) telle que :
\[F = E + \sum^r_{k=1}\paren{\sum^{m_k}_{p=1} \frac{a_{k,p}}{\paren{X - \lambda_k}^p}}\]
autrement dit :
\[F = E + \underbrace{\paren{\frac{a_{1,1}}{\paren{X - \lambda_1}} + \frac{a_{1,2}}{\paren{X - \lambda_1}^2} + \dots + \frac{a_{1,m_1}}{\paren{X - \lambda_1}^{m_1}}}}_{\text{Partie polaire associée au pôle } \lambda_1} + \dots + \underbrace{\paren{\frac{a_{r,1}}{\paren{X - \lambda_r}} + \frac{a_{r,2}}{\paren{X - \lambda_r}^2} + \dots + \frac{a_{r,m_r}}{\paren{X - \lambda_r}^{m_r}}}}_{\text{Partie polaire associée au pôle } \lambda_r}\]
    \underline{Remarque}\\
Pour décomposer une fraction rationnelle de \(\C\paren{X}\) en éléments simples, il convient de l’écrire sous forme irréductible puis de décomposer son dénominateur en facteurs rréductibles de \(\C \croch{X}\), c’est-à-dire en produit de polynômes de degré \(1\).
\end{defprop}

\subsection{Théorème de décomposition sur \(\R\) (preuve hors programme)}
\begin{defprop}
    Si \(F = \frac{A}{B}\) est une fraction rationnelle irréductible de \(\R\paren{X}\) de partie entière \(E\) et que la décomposition du polynôme \(B\) en facteurs irréductibles de \(\R \croch{X}\) s’écrit 
    \[B = \beta \prod^r_{k=1} \paren{X - \lambda_k}^{m_k} \prod^s_{k=1} \paren{X^2 + b_kX + c_k}^{n_k}\]
    alors il existe d’uniques suites de réels \((a_{k,p})_{(k,p)\in \interventierii{1}{r}\times \interventierii{1}{m_k}}\), \((d_{k,p})_{(k,p)\in \interventierii{1}{s}\times \interventierii{1}{n_k}}\) et \((e_{k,p})_{(k,p)\in \interventierii{1}{s}\times \interventierii{1}{n_k}}\) telles que :
    \[F = E + \sum^r_{k=1}\paren{\sum^{m_k}_{p=1} \frac{a_{k,p}}{\paren{X - \lambda_k}^p}} + \sum^s_{k=1} \paren{\sum^{n_k}_{p=1}\frac{d_{k,p}X + e_{k,p}}{\paren{X^2 + b_kX + c_k}^p}}\]
    \underline{Remarque} \\
    Les irréductibles de \(\R \croch{X}\) sont les polynômes de degré \(1\) et les polynômes de degré \(2\) sans racine réelle. Dans la décomposition en facteurs irréductibles écrite pour \(B\) ci-dessus, les réels \(\lambda_1, \dots , \lambda_r\) sont les racines réelles deux-à-deux distinctes de \(B\) de mutiplicités respectives \(m_1, \dots , m_r\), les couples deux-à-deux distincts de réels \((b_k, c_k)\) vérifient \(b^2_k - 4c_k < 0\) et les \(n_k\) sont des entiers non nuls.
\end{defprop}
\subsection{Quelques applications des décompositions en éléments simples}
\begin{defprop}[Calcul de primitives]
    
\end{defprop}
\begin{defprop}[Calcul de dérivées successives]
    
\end{defprop}
\begin{defprop}[Calcul de sommes de séries numériques]
    
\end{defprop}

\subsection{Deux cas particuliers}
\begin{defprop}[Coefficients des éléments simples associés aux pôles simples]   
    Si \(F = \frac{A}{B}\) est une fraction rationnelle irréductible de \(\K\paren{X}\) et \(\lambda\) est un pôle simple de \(F\) alors, dans la décomposition en éléments simples de \(F\) , le coefficient de l’élément simple \(\frac{1}{X - \lambda}\)  est :
    \[a = \frac{A(\lambda )}{B'(\lambda )}\].
    \underline{Remarque} \\
    Plus généralement, lorsque \(\lambda\)  est un pôle de multiplicité \(m\) de la fraction irréductible \(F\) , l’évaluation en \(\lambda\)  de la fraction rationnelle obtenue après réduction de \(\paren{X - \lambda }^mF\) donne le coefficient de l’élément simple \(\frac{1}{\paren{X - \lambda }^m}\) dans la décomposition en éléments simples de \(F\) 
\end{defprop}

\begin{defprop}[Décomposition en éléments simples d’une fraction rationnelle du type \(\frac{B'}{B}\)]    
    Soit \(B \in \K \croch{X}  \pd \accol{0_{\R\croch{X}}}\) .
    \begin{itemize}
        \item On suppose que \(\K = \C\) et que la décomposition de \(B\) en facteurs irréductibles de \(\C \croch{X}\) s’écrit
            \[B = \beta \prod^r_{k=1}\paren{X - \lambda_k}^{m_k}\] 
            Alors, la décomposition en éléments simples de la fraction rationnelle \(F = \frac{B'}{B}\) sur \(\C\) est :
            \[\frac{B'}{B} = \sum^r_{k=1} \frac{m_k}{\paren{X - \lambda_k}}\]
        \item On suppose que \(\K = \R\) et que la décomposition de \(B\) en facteurs irréductibles de \(\R \croch{X}\) s’écrit
            \[B = \beta \prod^r_{k=1} \paren{X - \lambda_k}^{m_k} \prod^s_{k=1}\paren{X^2 + b_kX + c_k}^{n_k} \]
            Alors, la décomposition en éléments simples de la fraction rationnelle \(F = \frac{B'}{B}\) sur \(\R\) est :
            \[\frac{B'}{B} = \sum^r_{k=1}\frac{m_k}{X - \lambda_k} + \sum^s_{k=1} \frac{n_k \paren{2X + b_k}}{X^2 + b_kX + c_k}\]
    \end{itemize}
\end{defprop}