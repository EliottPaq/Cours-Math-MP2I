\chapter{Application Linéaire (1)}

\minitoc

Dans ce chapitre, \(\K\) désigne le corps \(\R\) ou \(\C\) et, sauf mention contraire \(E, F\) et \(G\)  sont des \(\K\)-espaces vectoriels.

\section{Généralités}
\subsection{Définitions - Notations}
\begin{defi}
    \begin{enumerate}
        \item Une application \(u\) de \(E\) dans \(F\) est dite application linéaire de \(E\) dans \(F\) si :
        \[\forall (\lambda , \mu ) \in  \K^2, \forall (x, y) \in  E^2, u(\lambda  x + \mu y) = \lambda  u(x) + \mu u(y).\]
        \item Les applications linéaires de \(E\) dans \(F\) constituent \(\cal{L}\paren{E,F}\).
        \item Les applications linéaires bijectives de \(E\) dans \(F\) sont dites \underline{isomorphismes} de \(E\) sur \(F\) .
        \item Les applications linéaires de \(E\) dans \(E\) sont dites \underline{endomorphismes} de \(E\) et constituent \(\cal{L}\paren{E}\).
        \item Les endomorphismes bijectifs de \(E\) sont dits \underline{automorphismes} de \(E\) et constituent \(\cal{GL}\paren{E}\).
    \end{enumerate}
    \underline{Remarque}\\
        Toute application \(u\) linéaire de \(E\) vers \(F\) est un morphisme de groupes additifs donc \(u (0_E ) = 0_F\) .
\end{defi}
\subsection{Opérations}
\begin{defprop}
    \begin{enumerate}
        \item La combinaison linéaire de deux applications linéaires est une application linéaire.
        \item La composée de deux applications linéaires est une application linéaire.
        \item La bijection réciproque d’un isomorphisme est un isomorphisme.
    \end{enumerate}
\end{defprop}
\subsection{Structures algébriques des ensembles d’applications linéaires}
\begin{defprop}
    \begin{enumerate}
        \item \(\paren{\cal{L}\paren{E,F}, +, .}\) est un \(\K\)-espace vectoriel car sous-espace vectoriel du \(\K\)-espace vectoriel \(\cal{F}\paren{E, F }\).
        \item \(\paren{\cal{L}\paren{E}, +, \circ }\) est un anneau dont l’élément neutre pour la loi \(\circ \) est \(\id{E}\) .
        \item \(\paren{\cal{GL}\paren{E}, \circ }\) est un groupe dit groupe linéaire.
    \end{enumerate}
\end{defprop}

\subsection{Image directe et image réciproque d’un sous-espace vectoriel}
\begin{defprop}
    Si \(u\) est une application linéaire de \(E\) dans \(F\) alors
    \begin{enumerate}
        \item l’image directe par \(u\) d’un sous-espace vectoriel de \(E\) est un sous-espace vectoriel de \(F\) ;
        \item l’image réciproque par \(u\) d’un sous-espace vectoriel de \(F\) est un sous-espace vectoriel de \(E\).
    \end{enumerate}
\end{defprop}
\subsection{Image et noyau d’une application linéaire}
\begin{defprop}[Structures algébriques]
    Si \(u\) est une application linéaire de \(E\) dans \(F\) alors :
    \begin{enumerate}
        \item \(\im u = u(E) = \accol{y \in  F \tq \exists x \in  E, y = u(x)}\) est un sous-espace vectoriel de \(F\) , appelé image de \(u\) ;
        \item \(\ker u = u^{-1} \accol{0_F } = \accol{x \in  E \tq u(x) = 0_F }\) est un sous-espace vectoriel de \(E\), appelé noyau de \(u\).
    \end{enumerate}
\end{defprop}
\begin{defprop}[Famille génératrice de l’image d’une application linéaire]
    Si \(u \in  \cal{L}\paren{E,F}\) et \(E = \Vect{(x_i)_{i\in I} } \) alors \(\im(u) = \Vect{(u(x_i))_{i\in I} }\).
\end{defprop}
\begin{defprop}[Caractérisation des applications linéaires injectives/surjectives]
  \begin{enumerate}  
        \item Une application linéaire \(u\) de \(E\) dans \(F\) est injective si, et seulement si, \(\ker u = \accol{0_E }\).
        \item Une application (linéaire) \(u\) de \(E\) dans \(F\) est surjective si, et seulement si, \(\im u = F \).
    \end{enumerate}

        \underline{Remarques}\\
        Sans la linéarité, l’équivalence \(1\) n’est pas conservée alors que l’équivalence \(2\) est conservée.
\end{defprop}

\subsection{Rang d’une application linéaire}
\begin{defi}[Définition du rang d’une application linéaire]
    Une application linéaire \(u\) de \(E\) dans \(F\) est dite de rang fini si son image \(\im(u)\) est de dimension finie.\\
    Dans ce cas, la dimension de \(\im (u)\) est appelée rang de u et notée \(\rg(u)\) :
    \[\rg(u) = \dim (\im (u))\]
\end{defi}

\begin{defprop}[Conditions suffisantes de finitude du rang]
    \begin{enumerate}
        \item Si \(u : E \to F\) est linéaire et \(E\) de dimension finie alors \(u\) est de rang fini avec \(\rg(u) \leq \dim E\).
        \item Si \(u : E \to F\) est linéaire et \(F\) de dimension finie alors \(u\) est de rang fini avec \(\rg(u) \leq \dim F\) .
    \end{enumerate}
    \end{defprop}

\begin{defprop}[Rang d’une composée]
    Soit \(u \in  \cal{L}\paren{E,F}\) et \(v \in  \cal{L}\paren{F, G}\).\\
    \begin{enumerate}
        \item Si \(u\) et \(v\) sont de rang fini alors \(v \circ u\) est de rang fini et vérifie \(\rg (v \circ u) \leq \min(\rg(u), \rg(v))\).
        \item Invariance du rang par composition par un isomorphisme
        \begin{enumerate}
            \item si \(u\) est un isomorphisme et \(v\) de rang fini alors \(v \circ u\) est de rang fini et \(\rg (v \circ u) = \rg (v)\).
            \item si \(v\) est un isomorphisme et \(u\) de rang fini alors \(v \circ u\) est de rang fini et \(\rg (v \circ u) = \rg (u)\).
        \end{enumerate}
    \end{enumerate}
\end{defprop}
\begin{dem}
    \begin{itemize}
        \item On suppose que \(u\) et \(v\) sont de rang fini.\\
        \begin{itemize}
            \item Comme \(v\) est de rang fini alors \(\im v\) est de dimension finie. L’inclusion naturelle \(\im v\circ u \subset \im v\) donne alors que \(\im v \circ u\) est de dimension finie (comme sous-espace vectoriel d’un espace de dimension finie), donc que \(v \circ u\) est de rang fini, avec \(\rg (v \circ u) \leq \rg (v)\) par définition du rang.
            \item D’autre part, \(\im v \circ u = \accol{v \circ u(x) \tq x \in  E} = \accol{v (u(x)) \tq x \in  E}\) donc on a :
                \[\im v \circ u = \accol{v(y) \tq y \in  \im u}\]
            On s’intéresse alors à \(\tilde{v} = v_{| \Im u}\) autrement dit à l’application \(\tilde{v} : \im u \to G\) définie par
                \[\forall y \in  \im u, \tilde{v}(y) = v(y)\]
            \(\tilde{v}\) est linéaire (car \(v\) l’est) d’espace de départ, \(\im u\), de dimension finie (car \(u\) est de rang fini) donc, par propriété, \(\tilde{v}\) est de rang fini avec \(\rg (\tilde{v}) \leq \dim (\im u)\) c’est-à-dire \(\dim (\im \tilde{v}) \leq \dim (\im u)\).\\~\\
            Par définition de \(\tilde{v}\), on a donc \(\im v \circ u = \im \tilde{v}\) ce qui permet de conclure que \(\dim (\im v \circ u) \leq \dim (\im u)\) autrement dit que \(\rg (v\circ u) \leq \rg (u) \).\\~\\
            Ainsi \(\rg (v \circ u) \leq \rg (v)\) et \(rg (v \circ u) \leq \rg (u)\) donc \(\rg (v \circ u) \leq \min (\rg (u) , \rg (v))\)\\~\\
            \conclusion si \(u\) et \(v\) sont de rang fini alors \(v \circ u\) est de rang fini et \(\rg (v \circ u) \leq \min(\rg(u), \rg(v))\).
        \end{itemize}
        \item Invariance du rang par composition par un isomorphisme
        \begin{enumerate}
            \item On suppose que \(u\) est un isomorphisme et que \(v\) de rang fini. \\
                On a vu au \(1\) que \(\im v \circ u = \accol{v(y) \tq y \in  \im u}\) autrement dit que \(\im v \circ u = v (\im u)\) par définition de l’image directe d’un ensemble.\\~\\
                Or \(u\) est un isomorphisme de \(E\) vers \(F\) donc est en particulier surjectif ce qui donne \(\im u = F\) .\\
                Ainsi, \(\im v \circ u = v (F )\) avec \(v (F ) = \im v\), par définition de l’image de \(v\), linéaire de \(F\) dans \(G\).\\~\\
                On en déduit que \(\im v \circ u = \im v\) ce qui prouve que \(v \circ u\) est de rang fini (car \(v\) l’est) et que \(\rg (v \circ u) = \rg (v)\) .\\~\\
                \conclusion pour \(u\) isomorphisme et \(v\) de rang fini, on a \(v \circ u\) de rang fini et \(\rg (v \circ u) =\rg (v)\).
            \item On suppose que \(v\) est un isomorphisme et \(u\) de rang fini.
                On reprend les notations et une partie des résultats de la preuve du \(1\)
                \begin{itemize}
                    \item On a \(\im v \circ u = \im\tilde{v}(\star)\) avec \(\tilde{v} = v_{|\im u}\) qui est de rang fini car elle est linéaire sur un espace de dimension finie \(\im u\).Par conséquent, \(\im \tilde{v}\) est de dimension finie et donc \(\im v \circ u\) aussi vu l’égalité écrite. Autrement dit, \(v \circ u\) est de rang fini.
                    \item Déterminons une base de \(\im \tilde{v}\) (donc de \(\im v \circ u\))\\~\\
                    Soit \((y_1, \dots , y_n)\) une base de \(\im u\) (espace de départ de \(\tilde{v}\)) avec \(n = \dim \im u\).\\~\\
                    Alors, par propriété, \(\im \tilde{v} = \Vect{\tilde{v}(y_1), \dots , \tilde{v}(y_n)} \ie (\tilde{v}(y_1), \dots , \tilde{v}(y_n))\) engendre \(\im \tilde{v}\).\\~\\
                    De plus, \((\tilde{v}(y_1), \dots , \tilde{v}(y_n))\) est une famille libre de \(G\).\\~\\
                    En effet, pour \((\lambda_1, \dots , \lambda_n) \in  \K^n\) tel que \(\lambda_1\tilde{v}(y_1) + \dots + \lambda_n\tilde{v}(y_n) = 0_G\), on a\\~\\
                    \(\lambda_1v(y_1) + \dots + \lambda_nv(y_n) = 0_G\) donc, par linéarité de \(v\),\( v (\lambda_1y_1 + \dots + \lambda_ny_n) = 0_G\) ce qui prouve que\( \lambda_1y_1 + \dots + \lambda_ny_n\) appartient au noyau de \(v\). Comme \(v\) est un isomorphisme, \(v\) est en particulier linéaire injective donc son noyau est réduit à \(\accol{0_F }\). On en déduit que\( \lambda_1y_1 + \dots + \lambda_ny_n = 0_F\) puis, par liberté de \((y_1, \dots , y_n)\) , que \(\lambda_1 = \dots = \lambda_n = 0\) ce qui prouve la liberté de \((\tilde{v}(y_1), \dots , \tilde{v}(y_n))\) .\\~\\
                    Ainsi,\( (\tilde{v}(y_1), \dots , \tilde{v}(y_n))\) est une base de \(\im \tilde{v}\) donc \(\dim \im \tilde{v} = n = \dim \im u\) puis. On obtient alors : \(\dim \im v \circ u = \dim \im u\) autrement dit \(\rg (v \circ u) = \rg (u)\) .\\
                    \conclusion pour \(v\) isomorphisme et \(u\) de rang fini, on a \(v\circ u\) de rang fini et \(\rg (v\circ u) = \rg (u)\).
                \end{itemize}
        \end{enumerate}
    \end{itemize}
\end{dem}
\section{Des applications linéaires usuelles}
\subsection{Homothéties}
\begin{defi}
   Soit\( \lambda  \in  \K\).\\
    L’application \(h_{\lambda}  = \lambda \id{E}\) est un endomorphisme de \(E\) appelé homothétie de rapport \(\lambda\) .
\end{defi}

\subsection{Projections/projecteurs}
    On suppose ici que \(E_1\) et \(E_2\) sont des sous-espaces vectoriels supplémentaires de \(E\). Alors :
    \[\forall x \in  E, \exists!(x_1, x_2) \in  E_1 \times E_2, x = x_1 + x_2\].
\begin{defi}[Définition géométrique]
    L’application \(p : E \to E\) définie par
    \[\forall x \in  E, p(x) = x_1\]
    est appelée projection (ou projecteur) sur \(E_1\) parallèlement à \(E_2\).
\end{defi}
\begin{prop}[Propriétés]
    Si \(p\) est la projection sur \(E_1\) parallèlement à \(E_2\) alors :
    \begin{enumerate}
        \item \(p\) est un endomorphisme de \(E\) qui vérifie \(p \circ p = p\) ;
        \item \(E_1 = \im p = \ker (p - \id{E} ) = \accol{x \in  E \tq p(x) = x}\) ;
        \item \(E_2 = \ker p\).
    \end{enumerate}
\end{prop}

\begin{defprop}[Caractérisation algébrique]
    Une application \(p : E \to E\) est un projecteur de \(E\) si, et seulement si, \(p\) est linéaire et \(p^2 = p\).\\
    Dans ce cas, on a :
    \begin{enumerate}
        \item \(E = \im p \oplus \ker p\) ;
        \item \(p\) est la projection sur \(\im p = \ker (p - \id{E} )\) parallèlement à \(\ker p\) ;
        \item \(\forall x \in  E, x = p(x) + (x - p(x)) \text{ avec} \begin{cases}
            p(x) &\in  \im p\\
            x - p(x) &\in  \ker p 
        \end{cases}\)
    \end{enumerate}
\end{defprop}
\begin{defprop}[Remarques]
    \begin{enumerate}
        \item Le seul projecteur de \(E\) bijectif est \(\id{E}\) .
        \item On peut avoir \(E = \im u \oplus \ker u\) sans que l’endomorphisme \(u\) de \(E\) ne soit un projecteur.
    \end{enumerate}
\end{defprop}
\subsection{Symétries}
    On suppose ici que \(E_1\) et \(E_2\) sont des sous-espaces vectoriels supplémentaires de \(E\). Alors :
    \[\forall x \in  E, \exists!(x_1, x_2) \in  E_1 \times E_2, x = x_1 + x_2\]
\begin{defi}[Définition géométrique]
    L’application \(s : E \to E\) définie par
    \[\forall x \in  E, s(x) = x_1 - x_2\]
    est appelée symétrie par rapport à \(E_1\) parallèlement à \(E_2\).\\
    \underline{Remarque}\\
    Cette application vérifie \(s = 2p - \id{E}\) où \(p\) est la projection sur \(E_1\) parallèlement à \(E_2\).
\end{defi}

\begin{prop}
    Si \(s\) est la symétrie par rapport à \(E_1\) parallèlement à \(E_2\) alors :
    \begin{enumerate}
        \item \(s\) est un endomorphisme de \(E\) qui vérifie \(s \circ s = \id{E}\) ;
        \item \(E_1 = \ker (s - \id{E} ) = \accol{x \in  E \tq s(x) = x}\) ;
        \item \(E_2 = \ker (s + \id{E} ) = \accol{x \in  E \tq s(x) = -x}\).
    \end{enumerate}
\end{prop}

\begin{defprop}[Caractérisation algébrique]
    Une application \(s : E \to E\) est une symétrie de \(E\) si, et seulement si, \(s\) est linéaire et \(s^2 = \id{E}\).\\
    Dans ce cas :
    \begin{enumerate}
        \item \(E = \ker (s - \id{E} ) \oplus \ker (s + \id{E} )\) ;
        \item \(s\) est la symétrie par rapport à \(\ker (s - \id{E} )\) parallèlement à \(\ker (s + \id{E} )\) ;
        \item \(\forall x \in  E, x = \frac{1}{2} (x + s(x)) + \frac{1}{2} (x - s(x))\text{ avec }\begin{cases}
        \frac{1}{2} (x + s(x)) &\in  \ker (s - \id{E} )\\[10pt]
        \frac{1}{2} (x - s(x)) &\in  \ker (s + \id{E} )
        \end{cases}\)
    \end{enumerate}
\end{defprop}

\begin{defprop}[Bijectivité des symétries]
    Toute symétrie \(s\) de \(E\) est un automorphisme de \(E\) dont la bijection réciproque est \(s\).
\end{defprop}

\section{Isomorphismes}
\subsection{Détermination d’une application linéaire}
\begin{defprop}[Action sur une base]
    Si \((e_i)_{i\in I}\) est une base de \(E\) et \((f_i)_{i\in I}\) une famille de \(F\) alors il existe un unique \(u \in  \cal{L}\paren{E,F}\) tel que :
    \[\forall i \in  I, u(e_i) = f_i.\]
    Autrement dit, une application linéaire est entièrement déterminée par la connaissance de son action sur les vecteurs d’une base de son espace vectoriel de départ.
\end{defprop}

\begin{defprop}[Recollement]
    Si \(E_1\) et \(E_2\) sont des sous-espaces vectoriels supplémentaires de \(E\) avec \(u_1 \in  L(E_1, F )\) et \(u_2 \in  L(E_2, F )\) alors il existe un unique \(u \in  \cal{L}\paren{E,F}\) tel que :
    \[u_1 = u_{|E_1} \text{ et }u_2 = u_{|E_2} \]
Autrement dit, une application linéaire est entièrement déterminée par la connaissance de son action sur les vecteurs de deux sous-espaces vectoriels supplémentaires de son espace vectoriel de départ.
\end{defprop}

\subsection{Caractérisations de l’injectivité, la surjectivité ou la bijectivité}
\begin{defprop}
    Soit \(u\) une application linéaire de \(E\) dans \(F\) et \((e_i)_{i\in I}\) une base de \(E\).
    \begin{enumerate}
        \item \(u\) est surjective si, et seulement si, \((u(e_i))_{i\in I}\) est une famille génératrice de \(F\) .
        \item \(u\) est injective si, et seulement si, \((u(e_i))_{i\in I}\) est une famille libre de \(F\) .
        \item \(u\) est bijective si, et seulement si, \((u(e_i))_{i\in I}\) est une base de \(F\) .
    \end{enumerate}
\end{defprop}

\subsection{Applications linéaires entre espaces de même dimension finie}
\begin{defprop}[Caractérisation des isomorphismes]    
    Si \(u\) est une application linéaire entre deux \(\K\) espaces vectoriels de même dimension finie alors les trois propositions suivantes sont deux à deux équivalentes.
    \begin{enumerate}
        \item \(u\) est injective
        \item \(u\) est surjective
        \item \(u\) est bijective.
    \end{enumerate}
\end{defprop}

\begin{defprop}[Caractérisation des automorphismes]
    
    Si \(u\) est un endomorphisme de \(E\) avec \(E\) de dimension finie alors les trois propositions suivantes sont deux à deux équivalentes.
    \begin{enumerate}
        \item \(u\) est bijective.
        \item \(u\) est inversible à droite (c’est-à-dire qu’il existe \(v \in  \cal{L}\paren{E}\) tel que\( u \circ v = \id{E}\)).
        \item \(u\) est inversible à gauche (c’est-à-dire qu’il existe \(w \in  \cal{L}\paren{E}\) tel que \(w \circ u = \id{E}\)).
    \end{enumerate}
\end{defprop}

\subsection{Espaces vectoriels isomorphes}
\begin{defi}
    Deux espaces vectoriels \(E\) et \(F\) sont dits isomorphes s’il existe un isomorphisme de \(E\) vers \(F\) .
\end{defi}

\begin{defprop}[Caractérisation par la dimension]
    Si \(E\) est de dimension finie alors \(F\) est isomorphe à \(E\) si, et seulement si, \(E\) et \(F\) ont même dimension.
\end{defprop}
\begin{defprop}[dimension de \(\cal{L}\paren{E,F}\)]
    Si \(E\) et \(F\) sont de dimension finie alors \(\cal{L}\paren{E,F}\) l’est aussi et\( \dim (\cal{L}\paren{E,F}) = \dim(E) \times \dim(F )\)
\end{defprop}
\section{Théorème du rang}
\subsection{Théorème du rang (version géométrique)}
\begin{theo}
    Si \(u\) est une application linéaire de \(E\) vers \(F\) et si \(S\) est un supplémentaire de \(\ker(u)\) dans \(E\) alors l’application \(\tilde{u} : S \to \Im u\) définie par
   \[ \forall x \in  S, \tilde{u}(x) = u(x)\]
    est un isomorphisme.\\
    \underline{Remarques}\\
    \begin{itemize}
        \item L’image d’une application linéaire est donc isomorphe à tout supplémentaire de son noyau.
        \item On dit aussi que \(u\) induit un isomorphisme de \(S\) sur \(\im(u)\).
        \item On note parfois \(\tilde{u}\) de la manière suivante \(\tilde{u} = u^{|\im(u)}_{|S}\) en parlant de bi-restriction de \(u\).
    \end{itemize}
\end{theo}

\begin{theo}[Théorème du rang (version dimension finie)]
    Si \(u\) est une application linéaire de \(E\) vers \(F\) avec \(E\) de dimension finie alors
    \begin{enumerate}
        \item \(u\) est de rang fini ;
        \item \(\dim E = \dim \ker (u) + \rg(u)\).
    \end{enumerate}
    \underline{Remarques}\\
    \begin{itemize}
        \item la dimension finie de l’espace de départ sur lequel est définie l’application linéaire suffit ici.
        \item \underline{ATTENTION}\\
            Le théorème du rang n’implique pas l’égalité \(E = \ker u \oplus \im u\) \((\star)\). En effet,
        \begin{itemize}
            \item \((\star)\) peut n’avoir aucun sens si \(E\) et \(F\) sont distincts (sens de \(x + y\) avec \(x \in  E\) et \(y \in  F\) ?) ;
            \item \((\star)\) peut être fausse si \(E\) et \(F\) sont égaux (cf \(u \in  \cal{L} (\R \croch{X})\) définie par \(\forall P \in  \R \croch{X} , u(P ) = P '\)).
        \end{itemize}
    \end{itemize}
\end{theo}

\section{Formes linéaires et hyperplans}
\subsection{Formes linéaires}
\begin{defprop}
    Toute application linéaire de \(E\) vers \(\K\) est dite forme linéaire sur \(E\).\\
    \underline{Exemple}\\
    Si \(\cal{B} = (e_i)_{i\in I}\) est une base de \(E\) alors tout \(x\) de \(E\) s’écrit de manière unique sous la forme
    \[x = \sum_{i\in I}\underbrace{e^{\star}_i (x)}_{\in \K} e_i\]
    Les fonctions \(e^{\star}_i : E \to \K\) sont des formes linéaires sur \(E\) dite formes coordonnées relativement à \(B\).
\end{defprop}
\subsection{Hyperplans}
\begin{defi}
    Un sous-espace vectoriel de \(E\) est dit hyperplan s’il est le noyau d’une forme linéaire non nulle sur \(E\).
\end{defi}

\begin{defprop}[Caractérisations des hyperplans comme supplémentaires de droites]
    Soit \(H\) un sous-espace vectoriel de \(E\).\\
    \(H\) est un hyperplan de \(E\) si, et seulement si, \(H\) est supplémentaire d’une droite de \(E\).\\
    \underline{Remarque}\\
    Dans le cas où \(H\) est un hyperplan, pour toute droite \(D\) de \(E\) non contenue dans \(H\), on a : \(E = H \oplus D\).
\end{defprop}

\begin{dem}
    \begin{itemize}
        \item On suppose que \(H\) est un hyperplan de \(E\).\\
        Alors, par définition, il existe \(\phi\) dans \(\cal{L} (E, \K)\) avec \(\phi\neq 0_{\cal{L}(E,\K)}\) tel que \(H = \ker\phi\).\\~\\
        Comme \(\phi\) n’est pas nulle, il existe un vecteur \(a\) de \(E\) tel que \(\phi(a)\neq 0\) autrement dit tel que \(a \in  E \pd H\).\\
        Le sous-espace vectoriel \(D = \Vect{a}\) est alors une droite puisque a est différent de \(0_E\) .\\~\\
        \underline{Montrons que \(E = H \oplus D\)}\\
        Soit \(x \in  E\).
        \begin{itemize}
            \item \analyse on suppose qu’il existe \((x_H , x_D) \in  H \times D\) tel que \(x = x_H + x_D\).\\~\\
            Alors : \(\exists\lambda  \in  \K, x_D = \lambda a\) et \(x_H = x - x_D\) avec \(\phi(x_H ) = 0\) par définition de \(H\).
            Par linéarité de \(\phi\), cela donne \(\phi(x) - \lambda\phi(a) = 0\) puis \(\lambda  = \frac{\phi(x)}{\phi(a)}\) car \(\phi(a)\neq 0\).\\~\\
            On en déduit donc que le seul couple \((x_H , x_D)\) possible est \(\paren{x - \frac{\phi(x)}{\phi(a)} a, \frac{\phi(x)}{\phi(a)} a}\).
            \item \synthese : on a bien \(x =\paren{x - \frac{\phi(x)}{\phi(a)} a}+ \frac{\phi(x)}{\phi(a)}a\) avec \(\frac{\phi(x)}{\phi(a)} a \in  D\) (par définition de \(D\) car \(\frac{\phi(x)}{\phi(a)} \in  \K\)) et \(x - \frac{\phi(x)}{\phi(a)} a \in  H\) (par définition de \(H\) et linéarité de \(\phi\) car \(\phi\paren{x - \frac{\phi(x)}{\phi(a)} a}= \phi(x) - \frac{\phi(x)}{\phi(a)} \phi(a) = 0\)).
        \end{itemize}
        On en déduit que tout vecteur de \(E\) s’écrit de manière unique comme somme d’un élément de \(H\) et d’un élément de \(D\). Ainsi \(E = H \oplus D\) ce qui prouve que \(H\) est supplémentaire d’une droite de \(E\).
        \item On suppose que \(H\) est supplémentaire d’une droite de \(E\) que l’on note \(D = \Vect{a}\) avec \(a\neq 0_E\) .\\~\\
        Tout \(x \in  E\) s’écrit donc de manière unique sous la forme
        \[x = x_H + \lambda_a \text{ avec } x_H \in  H \text{ et }\lambda  \in  \K\]
        On définit alors l’application linéaire \(\phi : E \to \K\) par recollement en posant :
        \[\forall x_H \in  H, \phi(x_H ) = 0 \text{ et }\forall \lambda  \in  \K, \phi(\lambda a) = \lambda \]
        Alors, \(\phi\) est une forme linéaire (par définition), non nulle car \(\phi(a) = 1(\neq 0)\) et de noyau \(H\). En effet,
        \[\phi(x) = 0 \iff \phi(x_H + \lambda a) = 0 \iff \phi(x_H ) + \phi(\lambda a) = 0 \iff \lambda  = 0 \iff x = x_H \iff x \in  H\]
        On en déduit donc, par définition, que \(H\) est un hyperplan de \(E\).
    \end{itemize}
    \conclusion : \(H\) hyperplan de \(E\) si, et seulement si, \(H\) est supplémentaire d’une droite de \(E\).
\end{dem}

\subsection{Hyperplans en dimension finie}
    Sauf mention contraire, dans cette partie, \(E\) est de dimension finie non nulle notée \(n\).
\begin{defprop}[Caractérisations des hyperplans avec la dimension]
    Un sous-espace vectoriel \(H\) de \(E\) est un hyperplan de \(E\) si, et seulement si, \(\dim(H) = \dim(E) - 1\).
\end{defprop}

\begin{defprop}[Equations d’un hyperplan en dimension finie]
    Soit \(H\) un sous-espace vectoriel de \(E\) et \(B = (e_1, \dots , e_n)\) une base de \(E\).\\
    \(H\) est un hyperplan de \(E\) si, et seulement si, il existe \((a_1, a_2, \dots , a_n) \in  \K^n \pd \accol{(0, 0, \dots , 0)}\) tel que :
        \[x \in  H \iff a_1x_1 + a_2x_2 + \dots a_nx_n = 0\]
    avec \((x_1, \dots , x_n)\) la famille des coordonnées du vecteur \(x\) de \(E\) dans la base \(\cal{B}\).\\
    L’équation \(a_1x_1 + a_2x_2 + \dots a_nx_n = 0\) est dite équation de \(H\) dans la base \(\cal{B}\)\\
    \underline{Remarque}\\
    Dans ce cas, \(H = \ker \phi\) où \(\phi\) est la forme linéaire sur \(E\) définie par
    \[\phi : x \mapsto a_1x_1 + a_2x_2 + \dots a_nx_n\].
\end{defprop}

\begin{defprop}[Comparaison des équations d’un hyperplan en dimension finie]
    Deux formes linéaires non nulles sur \(E\) de même noyau sont proportionnelles.\\
    \underline{Remarque}\\ : 
    les équations d’un même hyperplan en dimension finie diffèrent donc à une constante multiplicative non nulle près.
\end{defprop}

\begin{defprop}[Hyperplans en dimension \(2\) et \(3\)]
    \begin{itemize}
        \item Les hyperplans d’un espace vectoriel \(E\) de dimension finie égale à \(2\) sont les droites de \(E\). Leurs équations dans une base de \(E\) sont de la forme \(ax + by = 0\) avec \((a, b) \in  \K^2 \pd \accol{(0, 0)}\) en notant \(x\) et \(y\) les coordonnées d’un vecteur de \(E\) dans la base considérée.
        \item Les hyperplans d’un espace vectoriel \(E\) de dimension finie égale à \(3\) sont les plans de \(E\). Leurs équations dans une base de \(E\) sont de la forme \(ax + by + cz = 0\) avec \((a, b, c) \in  \K^3 \pd \accol{(0, 0, 0)}\) en notant \(x, y\) et \(z\) les coordonnées d’un vecteur de \(E\) dans la base considérée.
    \end{itemize}
    \underline{Remarque}\\
    On retrouve ainsi la forme des équations cartésiennes de droites vectorielles de \(\R^2\) et plans vectoriels de \(\R^3\) vues dans le chapitre “Espaces vectoriels”.
\end{defprop}
\subsection{Intersection d’hyperplans en dimension finie}
    On suppose ici que \(E\) est de dimension finie non nulle \(n\) et que \(m \in \interventierii{1}{n}\).

\begin{theo}
    \begin{enumerate}
        \item Si \(H_1, \dots , H_m\) sont des hyperplans de \(E\) alors \(\dim \biginter^{m}_{k=1}H_k \geq \dim E - m\).
        \item Si \(F\) est un sous-espace vectoriel de \(E\) de dimension \(\dim E - m\) alors il existe \(m\) hyperplans de \(E\) notés \(H_1, \dots , H_m\) tels que \(F =\biginter^m_{k=1}H_k\).
    \end{enumerate}
\end{theo}

\begin{dem}
    \begin{enumerate}
        \item On note \(F =\biginter^d{m}_{k=1}H_k\) avec \(H_1, \dots , H_m\) des hyperplans de \(E\), respectivement noyaux de \(\phi_1, \dots , \phi_m\) formes linéaires non nulles sur \(E\). On considère l’application \(u : E \to K^m\) définie par :
        \[\forall x \in  E, u(x) = (\phi_1(x), \dots , \phi_m(x)) \]
        \(u\) est linéaire sur \(E\) (car les \(\phi_i\) le sont) avec \(E\) de dimension finie. Le théorème du rang implique donc que \(\dim E = \dim \ker u + \dim \im u\).\\~\\
        Par ailleurs, \(F = \ker u\) (car \(F =\biginter^d{m}_{k=1}\phi_k\)) et \(\dim \im u \leq m\) (car \(\im u \subset K^m\) et \(\dim K^m = m\)).\\~\\
        Ainsi, \(\dim F = \dim E - \dim \im u\) donc \(\dim F \geq \dim E - m\).\\
        \conclusion : si \(H_1, \dots , H_m\) sont des hyperplans de \(E\) alors \(\dim\biginter^m_{k=1}H_k \geq \dim E - m\).
        \item Soit \(F\) un sous-espace vectoriel de \(E\) de dimension p\( = \dim E - m\).\\~\\
        Soit \(\cal{B}_F = (e_1, \dots , e_p)\) une base de \(F\) , complétée en \(\cal{B} = (e_1, \dots , e_p, e_{p+1}, \dots , e_n)\) base de \(E\).\\~\\
        Pour \(i \in \interventierii{1}{n}\) , on note \(e^{\star}_i\) la \(i_e\) forme coordonnée relativement à la base \(\cal{B}\) de \(E\).
        Soit \(x \in  E\).
        \begin{align*}
            x \in  F &\iff \forall i \in  \interventierii{p+1}{n}e^{\star}_i (x) = 0.\\
            &\iff x \in \biginter^{n}_{i=p+1} \ker e^{\star}_i \\
            &\iff x \in \biginter^{n-p}_{k=1}\ker e^{\star}_{n-k+1} \qquad (\text{ après le changement d’indice } i = n - k + 1)
        \end{align*}
        Ainsi \(F =\biginter^{n-p}_{k=1}H_k\) avec \(H_k = \ker e^{\star}_{n-k+1}\) hyperplan de \(E\) (comme noyau d’une forme linéaire non nulle sur \(E\)) et \(n - p = m\).
    \end{enumerate}
    \conclusion : si \(F\) est sous-espace vectoriel de \(E\) de dimension \(\dim E - m\) alors \(F\) est intersection de \(m\) hyperplans de \(E\).
\end{dem}

\begin{defprop}[Système d’équations d’un sous-espace vectoriel]
    Soit \(F\) un sous-espace vectoriel de \(E\) de dimension \(\dim E - m\) et \(H_1, \dots , H_m\) des hyperplans de \(E\) tels que
    \[F = \biginter^{m}_{k=1}H_k\]
    Le système d’équations obtenu en rassemblant des équations des \(m\) hyperplans \(H_i\) relativement à une base \(\cal{B}\) de \(E\) est appelé système d’équations du sous-espace vectoriel \(F\) relativement à \(B\).\\
    \underline{Exemple}
    Une droite vectorielle de \(\R^3\) a donc un système d’équations du type
    \[\begin{cases}
        ax + by + cz &= 0\\
        a'x + b'y + c'z &= 0
    \end{cases}\]
    avec \((a, b, c) \in  \R^3 \pd \accol{(0, 0, 0)}\) et \((a', b', c') \in  \R^3 \pd \accol{(0, 0, 0)}\).
\end{defprop}