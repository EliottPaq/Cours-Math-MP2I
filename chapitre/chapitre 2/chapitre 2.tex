\chapter{Inégalité et fonction (rappel et compléments)}

\minitoc

Dans ce chapitre, sont rassemblés des rappels ou compléments sur les inégalités ainsi que des fondamentaux sur les fonctions de variable réelle à valeurs réelles (sans preuve ni évocation de continuité).

\section{Inégalité}

\subsection{Relation d'ordre sur \(\R\)}

\begin{defi}
	On dit que la relation \(\leq\) est une relation d'équivalence sur \(\R\) car elle vérifie les propriétés suivantes :
	\begin{enumerate}
		\item Pour tout réel x, on a : \(x \leq x \). \hfill (réfléxivité)
		\item Pour tout couple de réels \(\paren{x,y}\) tel que \( x \leq y  \) et \(y \leq x\), on a :\( y = x  \) \hfill (antisymétrie)
		\item Pour tout triplet de réels \(\paren{x,y,z}\) tel que \(x \leq y  \) et \( y \leq z  \), on a : \( x \leq z  \) \hfill (transitivité)
	\end{enumerate}
\end{defi}

\begin{prop}[Compatibilité avec les opérations]
	Soit \(x,y,z,t\) et \(a\) des réels.
	\begin{enumerate}
		\item Si \(x\leq y\) et \(z\leq t\) alors \(x+z\leq y +t \)
		\item Si \(x\leq y \) et \( 0 \leq a\) alors \(a x \leq a y\)
		\item Si \(x\leq y \) et \( a \leq 0\) alors \(a y \leq a x\)
		\item Si \( 0 \leq x \leq y \) et \( 0\leq z \leq t \) alors \( 0 \leq xz \leq y t \)
	\end{enumerate}
\end{prop}

\begin{nota}[Intervalles de \(\R\)]
	Les partie \(I\) de \(\R\) pouvant s’écrire sous l’une des formes suivantes sont dites intervalles de \(\R\) :
	\begin{itemize}
		\item \(I = \emptyset\) \\
		\item \(I = \accol{x \in \R\tq a \leq x \leq b} \underset{\mathrm{notation}}{=} \intervii{a}{b}\) avec \(\paren{a,b} \in \R^2 \) et \(a\leq b \) \\
		\item \(I = \accol{x \in \R\tq a \leq x < b} \underset{\mathrm{notation}}{=} \intervie{a}{b}\) avec \(\paren{a,b} \in \R\times \paren{\R \union \accol{\pinf}} \) et \(a < b\) \\
		\item \(I = \accol{x \in \R\tq a < x \leq b} \underset{\mathrm{notation}}{=} \intervei{a}{b}\) avec \(\paren{a,b} \in \paren{\R \union \accol{\minf}}\times \R \) et \(a < b\) \\
		\item \(I = \accol{x \in \R\tq a < x \leq b} \underset{\mathrm{notation}}{=} \intervee{a}{b}\) avec \(\paren{a,b} \in \paren{\R \union \accol{\minf}}\times  \paren{\R \union \accol{\pinf}} \) et \(a < b\) \\

	\end{itemize}
\end{nota}

\begin{prop}
	\begin{enumerate}
		\item Passage à l'inverse dans une inégalité
		      \[\quantifs{\forall x \in \Rps ; \forall y \in \Rps} x\leq y \iff \frac{1}{y} \leq \frac{1}{x}\]
		      \[\quantifs{\forall x \in \Rms ; \forall y \in \Rms} x\leq y \iff \frac{1}{y} \leq \frac{1}{x}\] \\
		\item Passage au carré dans une inégalité
		      \[\quantifs{\forall x \in \Rps ; \forall y \in \Rps} x\leq y \iff x^2 \leq y^2\]
		      \[\quantifs{\forall x \in \Rms ; \forall y \in \Rms} x\leq y \iff y^2 \leq x^2\] \\
		\item Passage à la racine carrée dans une inégalité
		      \[\quantifs{\forall x \in \Rp ; \forall y \in \Rp} x\leq y \iff \sqrt{x}\leq \sqrt{y}\] \\
		\item Passage à l’exponentielle ou au logarithme népérien dans une inégalité
		      \[\quantifs{\forall x \in \R ; \forall y \in \R} x\leq y \iff \e{x}\leq \e{y}\]
		      \[\quantifs{\forall x \in \Rps ; \forall y \in \Rps} x\leq y \iff \ln{x}\leq \ln{y}\] \\
	\end{enumerate}
\end{prop}

\begin{exoex}
	Montrer \(\quantifs{\forall x \in \intervii{0}{1}} x(1-x) \leq \frac{1}{4}\).
\end{exoex}

\begin{corr}[2 Méthode]
	Soit \(x \in \intervii{0}{1} \)
	\begin{enumerate}
		\item Raisonnement par équivalence
		      \[\begin{aligned}
				      x(1-x) \leq \frac{1}{4} & \iff 0 \leq \frac{1}{4}-x(1-x)     \\
				                              & \iff 0\leq x^2 -x +  \frac{1}{4}   \\
				                              & \iff 0\leq\paren{x- \frac{1}{2}}^2
			      \end{aligned}
		      \]
		      Ceci étant vrai \(\quantifs{\forall x\in \intervii{0}{1}}\) car \(\Delta = 0\) et \(x_0 =  \frac{1}{2}\), on conclut \(\quantifs{\forall x \in \intervii{0}{1}} x(1-x) \leq \frac{1}{4}\).\\
		\item étude de la fonction \(\fonction{f}{\intervii{0}{1}}{\R}{x}{\frac{1}{4}-x(1-x)}\)\\
	\end{enumerate}
\end{corr}


\begin{exoex}
	Montrer \(\quantifs{\forall x \in \Rps} x+\frac{1}{x}\geq 2\).
\end{exoex}

\begin{corr}
	Soit \(x \in \Rps \)

	\[\begin{aligned}
			x+\frac{1}{x}\geq 2 & \iff \frac{x^2+1}{x}\geq 2 \\
			                    & \iff x^2-2x+1\geq    0     \\
			                    & \iff (x-1)^2 \geq 0
		\end{aligned}
	\]
	Ceci étant vrai \(\quantifs{\forall x\in \Rps}\) on conclut \(\quantifs{\forall x \in \Rps} x+\frac{1}{x}\geq 2\).
\end{corr}

\begin{exoex}
	Encadrer \(\frac{2x^2-x+1}{x^2+\sqrt{x+2}+3}\) pour \(x \in \intervii{-1}{1}\).
\end{exoex}

\begin{corr}
	Soit \(x \in \intervii{-1}{1} \)
	\begin{enumerate}
		\item \underline{numérateur} :
		      \[\begin{aligned}
				      -1 \leq x\leq 1 & \iff 0 \leq x^2 \leq 1      \\
				                      & \iff 0 \leq 2x^2 \leq 2     \\
				                      & \iff 0 \leq 2x^2-x+1 \leq 4
			      \end{aligned}
		      \]

		\item \underline{denominateur} : \[\begin{aligned}
				      -1 \leq x\leq 1 & \iff 0 \leq x^2 \leq 1                                                      \\
				                      & \iff 4 \leq x^2 +\sqrt{x+2}+3 \leq 4+\sqrt{3}                               \\
				                      & \iff \frac{1}{4+\sqrt{3}} \leq \frac{1}{x^2 +\sqrt{x+2}+3 }\leq \frac{1}{4} \\
			      \end{aligned}
		      \]
	\end{enumerate}
	Ainsi par produit des deux inégalités on as \(0\leq\frac{2x^2-x+1}{x^2+\sqrt{x+2}+3}\leq1\) pour \(x \in \intervii{-1}{1}\).
\end{corr}

\begin{exoex}
	Encadrer \(\frac{x-y^2+3}{x^2+y^2-y}\) pour \(\forall \paren{x,y} \in \intervii{1}{2}^2\).
\end{exoex}

\begin{corr}
	Soit \(x \in \intervii{-1}{1} \)
	\begin{enumerate}
		\item \underline{numérateur} :
		      \[\begin{aligned}
				      1-4+3\leq x-y^2+3 \leq 2-1+4 & \iff 0 \leq x-y^2+3 \leq 5
			      \end{aligned}
		      \]

		\item \underline{denominateur} : \[\begin{aligned}
				      0 \leq y-1\leq 1 & \iff 0 \leq y^2-y \leq y                        \\
				                       & \iff 0 \leq y^2-y \leq 2                        \\
				                       & \iff 1 \leq x^2+y^2-y\leq 6                     \\
				                       & \iff \frac{1}{6} \leq \frac{1}{x^2+y^2-y}\leq 1 \\
			      \end{aligned}
		      \]
	\end{enumerate}
	Ainsi par produit des deux inégalités on as \(0\leq \frac{x-y^2+3}{x^2+y^2-y} \leq 5\) pour \(\forall \paren{x,y} \in \intervii{1}{2}^2\).
\end{corr}

\begin{defi}[Parties majorées, majorants, maximum]
	Une partie \(A\) de \(\R\) est dite majorée s’il existe un réel \(M\) tel que, pour tout réel \(x\) de \(A\), on a : \(x \leq M\). \\
	Un tel réel \(M\) est alors dit :
	\begin{itemize}
		\item majorant de \(A\) dans le cas général. \\
		\item maximum de \(A\) dans le cas particulier où \(M\) appartient à \(A\).\\
	\end{itemize}

\end{defi}

\begin{defi}[Parties minorées, minorants, minimum]
	Une partie \(A\) de \(\R\) est dite minorée s’il existe un réel \(m\) tel que, pour tout réel \(x\) de \(A\), on a : \(m\leq x\). \\
	Un tel réel \(m\) est alors dit :
	\begin{itemize}
		\item minorant  de \(A\) dans le cas général. \\
		\item minimum  de \(A\) dans le cas particulier où \(m\) appartient à \(A\).\\
	\end{itemize}

\end{defi}

\begin{exoex}
	Que dire de \(B = \accol{\frac{n}{n^2+1} \tq n \in \N}\) ?
\end{exoex}

\begin{corr}
	\begin{itemize}

		\item \(B\) est minorée car \( \quantifs{\forall n \in \N} 0 \leq \frac{n}{n^2+1} \) par ailleurs \(0 \in B\) donc \(0\) est un minimum. \\
		\item \(B\) est majorée par \(\frac{1}{2}\). En effent en notant \(U_n = \frac{n}{n^2+1}\), On voit que \((U_n)\) est strictement décroissante
	\end{itemize}
\end{corr}

\begin{exoex}
	Que dire de \(C = \accol{\frac{\e{x}}{x} \tq x \in \Rps}\) ?
\end{exoex}

\begin{corr}
	\begin{itemize}

		\item \(C\) est minorée car \( \quantifs{\forall x \in \Rps} 0 \leq \frac{\e{x}}{x} \) donc \(0\) est un minorant mais pas un minimum  \\
		\item Supposons que \(C\) est majorée alors \(\quantifs{\exists M \in \R,\forall c \in C} c\leq M \) ainsi \(\quantifs{\forall x \in \Rps} \frac{\e{x}}{x} \leq M \) donc par passage à la limite en \(\pinf\) on trouve \(\pinf \leq M\) ce qui est absurde donc \(C\) n'est pas majorée.
	\end{itemize}
\end{corr}

\begin{defi}[Parties bornées]
	Une partie \(A\) de \(\R\) est dite bornée si elle est majorée et minorée autrement dit s’il existe deux réels \(m\) et \(M\) tel que, pour tout réel \(x\) de \(A\), on a : \(m\leq x \leq M\).
\end{defi}

\section{Valeur absolue d'un réel}
\begin{defi}
Pour tout \(x\) réel, la valeur absolue de \(x\), notée \(\abs{x}\), est définie par : \(\abs{x} = \begin{cases}
	-x & \text{si }  x < 0   \\
	x  & \text{si }  x\geq 0 \\
\end{cases}\)
\end{defi}

\begin{prop}
	\begin{enumerate}
		\item Pour tout \(x\) réel, on a : \(0\leq\abs{x}\) et \(x\leq\abs{x}\)
		\item Pour tout couple\((x,y)\) de réels, on a : \(\abs{xy} = \abs{x}\abs{y}\)
		\item Pour tout couple \((x,y)\) de réels tel que \(y\) est non nul, on a: \(\abs{\frac{x}{y}} = \frac{\abs{x}}{\abs{y}}\)
	\end{enumerate}
\end{prop}

\begin{defprop}[Deux inéquations élémentaires]
	Pour tout réel \(x\) et tout \underline{réel positif} \(\alpha\), on a:
	\begin{enumerate}
		\item \(\abs{x}\leq \alpha \iff -\alpha \leq x \leq \alpha \iff x \in \intervii{-\alpha}{\alpha}\)
		\item \(\abs{x}\geq \alpha \iff x \leq -\alpha\text{ ou } \alpha \leq x \iff x \in \intervei{\pinf}{-\alpha}\union\intervie{\alpha}{\pinf}\)
	\end{enumerate}
\end{defprop}

\begin{defprop}[Interprétation sur la droite des réels]
	Soit \(a\) un réel et \(b\) un \underline{réel positif}. \\
	L’ensemble des réels \(x\) vérifiant \(\abs{x-a}\leq b\) (resp. \(\abs{ x-a}\geq b \)) est l’ensemble des points de la droite des
	réels situés à une distance du point \(a\) inférieure ou égale (resp. supérieure ou égale) à \(b\).
\end{defprop}

\begin{prop}[Inégalité triangulaire]
	Pour tout couple \((x,y)\) de réels, on a :
	\[\abs{x+y}\leq \abs{x}+\abs{y}\]
\end{prop}

\begin{dem} [inégalité triangulaire]
	Soit \((x,y) \in \R^2\)
	\begin{align*}
		\abs{x+y}\leq \abs{x}+\abs{y} & \iff \abs{x+y}^2\leq (\abs{x}+\abs{y})^2 \\
        &\iff x^2+2xy+y^2 \leq x^2+y^2+2\abs{x}\abs{y} \\
        &\iff xy\leq \abs{xy}
	\end{align*}
    Ce qui est vrai donc l'inégalité est bien démontrer 
\end{dem}

\begin{exoex}
    Encadrer \(\frac{x\cos(x)+1}{\sin(x)+3}\) pour \(x\in\intervii{-\pi}{2\pi}\)
\end{exoex}

%exemple 2