\chapter{Dénombrement - Probabilité sur un univers fini}
\minitoc
\section{Dénombrement}
    Conformément au programme officiel de MP2I, on exclut ici toute formalisation excessive. Les propriétés intuitives sont donc admises et le recours systématique à des bijections pour dénombrer n’est pas attendu.
\subsection{Cardinal d’un ensemble fin}
\begin{defi}
    Soit \(A\) un ensemble.\\~\\
    \(A\) est dit fini s’il est vide ou s’il existe un entier naturel non nul \(n\) et une bijection de \(\interventierii{1}{n}\) vers \(A\).\\~\\
    Dans ce cas,
    \begin{itemize}
        \item si \(A\) est vide, on dit que \(A\) est de cardinal égal à \(0\).
        \item si \(A\) est non vide alors l’entier \(n\) ci-dessus est unique et appelé cardinal de \(A\).
    \end{itemize}
    Notations utilisées pour le cardinal : \(\Card{A}\) ou \(\abs{A}\).
\end{defi}
\begin{prop}
    \begin{itemize}
        \item Cardinal d’une partie\\
            Si \(B\) est une partie d’un ensemble fini \(A\) alors \(B\) est un ensemble fini avec \(\Card{B} \leq \Card{A}\) [et égalité des cardinaux si, et seulement si, \(B\) = \(A\)].
        \item Opérations sur les cardinaux
        \begin{enumerate}
            \item Si \(A\) et \(B\) sont deux ensembles finis alors :
                \begin{enumerate}
                    \item \(A \union B\) est un ensemble fini et \(\Card{A \union B} = \Card{A} + \Card{B} - \Card{A \inter B}\).
                    \item \(A \times B\) est un ensemble fini et\( \Card{A \times B} = \Card{A} \times \Card{B}\).
                \end{enumerate}
            \item Soit \(A\) et \(B\) des parties d’un ensemble fini \(E\).
                \begin{enumerate}
                    \item \(A\pd B\) est un ensemble fini et \(\Card{A\pd B} = \Card{A} - \Card{A \inter B}\)
                    \item \(A\) est un ensemble fini et \(\Card{A} = \Card{E} - \Card{A}\).
                \end{enumerate}
        \end{enumerate}
        \item Caractérisation des bijections entre ensembles finis de même cardinal\\
            Une application entre deux ensembles finis de même cardinal est bijective si, et seulement si, elle est injective, si, et seulement si, elle est surjective.
        \item Cardinal de l’ensemble des applications entre deux ensembles finis\\
            Si \(A\) et \(B\) sont des ensembles finis alors l’ensemble \(B^A\) des applications de \(A\) vers \(B\) est fini avec
            \[\Card{B^A} = \Card{B}^{\Card{A}}\] .
        \item Cardinal de l’ensemble des parties d’un ensemble fini\\
            Si \(A\) est un ensemble fini alors l’ensemble \(\cal{P}(A)\) des parties de \(A\) est fini avec
            \[\Card{\cal{P}(A)} = 2^{Card{A}}.\]
    \end{itemize}
\end{prop}
\subsection{Listes et combinaisons}
\begin{defi}
    Soit \(A\) un ensemble fini et \(p\) un entier naturel.
    \begin{enumerate}
        \item On appelle \(p-\)liste (ou \(p-\)uplet) d’éléments de \(A\) tout élément \((a_1, a_2, \dots , a_p)\) de \(A_p\).
        \item On appelle \(p-\)arrangement d’éléments de \(A\) toute \(p-\)liste d’éléments distincts de \(A\).
        \item On appelle \(p-\)combinaison d’éléments de \(A\) toute partie de \(A\) à \(p\) éléments.
    \end{enumerate}
\end{defi}
\begin{prop}
    Soit \(A\) un ensemble fini de cardinal \(n\) et \(p\) un entier naturel non nul inférieur ou égal à \(n\).
    \begin{enumerate}
        \item Le nombre de \(p-\)listes d’éléments de \(A\) est égal à \(n^p\).
        \item Le nombre de \(p-\)listes d’éléments distincts de \(A\) est égal à \(\prod^{p-1}_{k=0}(n - k)= \frac{n!}{(n - p)!}\).
        \item Le nombre de \(p-\)combinaisons d’éléments de \(A\) est égal à \(\frac{n!}{p!(n - p)!} =\binom{n}{p}\).
    \end{enumerate}
    \underline{Remarques}\\
    Le nombre d’applications
    \begin{itemize}
        \item injectives d’un ensemble fini de cardinal \(p\) dans un ensemble fini de cardinal \(n\) est égal à \(\frac{n!}{(n - p)!}\)
        \item bijectives d’un ensemble fini de cardinal \(n\) dans lui-même est égal à \(n!\).
    \end{itemize}
\end{prop}
\begin{defprop}[Retour sur les formules de Pascal et du binôme ]
    ~\\
    ~\\
    \begin{enumerate}
        \item \(\forall(n, p) \in \N^2, p \leq n \imp \binom{n}{p} + \binom{n}{p+1} = \binom{n+1}{p+1} \) \hfill (formule du triangle de Pascal)
        \item \(\forall n \in \N, \forall(a, b) \in \C^2, (a + b)^n =\sum^n_{p=0}\binom{n}{p}a^pb^{n-p}\) \hfill (formule du binôme de Newton)
    \end{enumerate}
\end{defprop}
\section{Probabilités sur un univers fini}
\subsection{Expérience aléatoire et univers}
\begin{defprop}[Expérience aléatoire]
    Une expérience qui, reproduite dans des conditions identiques, peut conduire à plusieurs résultats possibles dont on ne peut prévoir le résultat par avance est dite expérience aléatoire.
\end{defprop}
\begin{defprop}[Univers]
    L’ensemble des issues (résultats possibles ou réalisations) d’une expérience aléatoire est appelé univers (ou espace des états) et souvent noté \(\Omega\). \\~\\
    Conformément au programme de MP2I, on se limite en \(1^{\text{re}}\) année à des univers finis.
\end{defprop}
\begin{defprop}[Evénements]
    \begin{enumerate}
        \item Toute partie de l’univers \(\Omega\) est appelée événement.
        \item Parmi les parties de l’univers \(\Omega\), on trouve :
            \begin{itemize}
                \item la partie \(\Omega\), dite événement certain ;
                \item la partie \(\emptyset\), dite événement impossible ;
                \item les singletons de \(\Omega\), dits événements élémentaires souvent notés \(\accol{\omega}\) à pour \(\omega \in \Omega\).
            \end{itemize}
        \item Soient \(A\) et \(B\) deux parties de l’univers \(\Omega\) (autrement dit deux événements).
            \begin{itemize}
                \item La partie \(\conj{A} = \Omega \pd A\) est dite événement contraire de \(A\).
                \item La partie \(A \union B\) est dite événement \("A\) ou \(B"\).
                \item La partie \(A \inter B\) est dite événement \("A\) et \(B"\).
                \item Les parties \(A\) et \(B\) sont dites événements incompatibles (ou disjoints) si \(A \inter B = \emptyset\).
            \end{itemize}
        \item Soit   \( q \in \Ns\) et \(A_1, A_2, \dots , A_q\) des parties de \(\Omega\).
        L’ensemble  \(\accol{A_1, A_2, \dots , A_q}\) est dit système complet d’événements de \(\Omega\) si c’est une partition de \(\Omega\) autrement dit si :
        \begin{itemize}
            \item \(\bigunion_{i=1}^q A_i = \Omega \); \hfill (la réunion des événements du système est l’événement certain)
            \item \(\forall  i \in \interventierii{1}{q} , A_i\neq \emptyset\) ; \hfill(pas d’événenement impossible dans le système)
            \item \(\forall (i, j) \in \interventierii{1}{q}^2 , i\neq j \imp A_i \inter A_j = \emptyset.\) (événements du système incompatibles deux à deux)
        \end{itemize}
        \underline{Remarque}\\
        On note parfois \(\Omega = \bigsqcup^q_{i=1} A_i\) pour indiquer que les événements \(A_i\) sont deux à deux disjoints.
    \end{enumerate}
\end{defprop}

\subsection{Espaces probabilisés finis}
\begin{defi}
    Soit \(\Omega\) un univers fini.\\~\\
    On appelle probabilité sur \(\Omega\) toute application \(\mathbb{P}\) de \(\cal{P}(\Omega)\) dans \(\intervii{0}{1}\) telle que :
    \begin{enumerate}
        \item \(\proba{\Omega} = 1\) ;
        \item \(\forall (A, B) \in \cal{P} (\Omega) \times \cal{P} (\Omega), A \inter B = \emptyset \imp P(A \union B) = P(A) + P(B)\).
    \end{enumerate}
    Le couple \((\Omega, \mathbb{P})\) est alors appelé espace probabilisé fini.
\end{defi}
\begin{prop}
    Soit \((\Omega, \mathbb{P})\) un espace probabilisé fini.
    \begin{enumerate}
        \item \(\proba{\emptyset} = 0\).\hfill (événement impossible)
        \item \(\forall A \in \cal{P}(\Omega), \proba{\conj{A}} = 1 - \proba{A}\). \hfill (événement contraire)
        \item \(\forall(A, B) \in \cal{P}(\Omega) \times \cal{P}(\Omega), \proba{A \union B} = \proba{A} + \proba{B} - \proba{A \inter B}\). \hfill(réunion d’événements)
        \item \(\forall(A, B) \in \cal{P}(\Omega) \times \cal{P}(\Omega), \proba{A \pd B} = \proba{A} - \proba{A \inter B}\) \hfill (différence d’événements)
        \item \(\forall(A, B) \in \cal{P}(\Omega) \times \cal{P}(\Omega), A \subset B \imp \proba{A} \leq \proba{B}\) \hfill (croissance des probabilités)
    \end{enumerate}
\end{prop}
\begin{defprop}[Distribution de probabilités]
    On appelle distribution de probabilités sur un ensemble fini \(I\) toute famille de réels positifs indexée par \(I\) et de somme égale à \(1\) .
\end{defprop}
\begin{defprop}[Détermination d’une probabilité]
    Soit \(n \in \Ns\).\\~\\
    Si \(\Omega = \accol{\omega_1, \dots , \omega_n}\) est un univers fini et \(p_1, \dots , p_n\) des réels positifs tels que \(p_1 + \dots + p_n = 1 \) alors il existe une probabilité \(\mathbb{P}\) et une seule sur \(\Omega\) telle que \(\forall i \in \accol{1, 2, \dots , n} , \proba{\accol{\omega_i}} = p_i\).\\

    \underline{Remarque}\\
    Autrement dit, une probabilité sur un univers fini est entièrement déterminée :
    \begin{itemize}
        \item par sa valeur sur les singletons de l’univers.
        \item par la distribution de probabilités \(\paren{\proba{\accol{\omega}}}_{\omega\in\Omega}\) .
    \end{itemize}
\end{defprop}

\begin{defprop}[Probabilité uniforme]
    Soit \((\Omega, \mathbb{P})\) un espace probabilisé fini.\\~\\
    \(\mathbb{P}\) est dite probabilité uniforme sur \(\Omega\) s’il existe \(p \in \intervii{0}{1}\) tel que \(\forall\omega \in \Omega, \proba{\accol{\Omega}} = p\).\\~\\
    Dans ce cas,
    \[p = \frac{1}{\Card{\Omega}}\text{ et }\forall\omega \in \Omega, \proba{\accol{\omega}} = \frac{1}{\Card{\Omega}}\text{ et }\forall A \in \cal{P} (\Omega) , \proba{A} = \frac{\Card{A}}{\Card{\Omega}} \]
\end{defprop}
\subsection{Probabilités conditionnelles}
Soit \((\Omega, \mathbb{P})\) un espace probabilisé fini.
\begin{defi}
    Soient \((A, B) \in \cal{P}(\Omega) \times \cal{P}(\Omega)\) tel que \(\proba{B} > 0\).\\~\\
    On appelle probabilité conditionnelle de \(A\) sachant \(B\) le réel, noté \(\proba{A|B} \) ou \(\probacond{A}{B}\), défini par
    \[\proba{A|B} = \probacond{A}{B} = \frac{\proba{A \inter B}}{\proba{B} }\]
    \underline{Remarque}\\
    Pour tout \(B \in \cal{P} (\Omega)\) tel que \(\proba{B} > 0\), l’application \(\mathbb{P}_{B} : A \mapsto \frac{\proba{A \inter B}}{\proba{B}}\) est une probabilité sur \(\Omega\) et vérifie donc toutes les propriétés précédemment énoncées sur les probabilités.\\~\\
    Ainsi, on a notamment :
    \begin{itemize}
        \item \(\forall A \in \cal{P} (\Omega) , \mathbb{P}_{B} (\conj{A}) = 1 - \mathbb{P}_{B} (A).\)
        \item \(\forall (A, C) \in \cal{P}(\Omega) \times \cal{P}(\Omega), \mathbb{P}_{B} (A \union C) = \mathbb{P}_{B} (A) + \mathbb{P}_{B} (C) - \mathbb{P}_{B} (A \inter C).\)
    \end{itemize}
\end{defi}
\begin{defprop}[Formule des probabilités composées]
    Soit \(n \in N \)tel que \(n \geq 2\).
    Si \(A_1, \dots , A_n\) sont des événements tels que \(\proba{A_1 \inter \dots \inter A_{n-1}}\) est non nulle alors 
    \[\proba{A_1 \inter A_2 \inter \dots \inter A_n} = \proba{A_1} \probacond{A_2}{A_1} \probacond{A_3}{A_1 \inter A_2} \dots \probacond{A_n}{A_1 \inter \dots \inter A_{n-1}}\]
\end{defprop}

\begin{defprop}[Formule des probabilités totales]
    Si \(\accol{A_1, \dots , A_n}\) est un système complet d’événements et \(B\) un événement quelconque alors
    \[\proba{B} = \sum^n_{i=1} \probacond{B}{A_i} \proba{A_i}\]
    avec la convention usuelle suivante conservée dans la suite du chapitre : \(\probacond{B}{A_i} \mathbb{P} (A_i) = 0\) si \(\mathbb{P} (A_i) = 0\).
\end{defprop}

\begin{defprop}[Formules de Bayes]
    Soit \(B\) un événement de probabilité non nulle.
    \begin{enumerate}
        \item Si \(A\) est un événement alors
            \[\probacond{A}{B} = \frac{\probacond{B}{A} \proba{A}}{\proba{B}}\] .
        \item Si \(\accol{A_1, \dots , A_n}\) est un système complet d’événements alors
        \[\forall j \in \interventierii{1}{n} , \frac{\probacond{A_j}{B} = \probacond{B}{A_j}\mathbb{P}(A_j)}{\sum^n_{i=1} \probacond{B}{A_i}\proba{A_i}}\]
        (conséquence immédiate de la formule de Bayes précédente et de la formule des probabilités totales)
    \end{enumerate}
\end{defprop}

\subsection{Evénements indépendants}
Soit \((\Omega, \mathbb{P})\) un espace probabilisé fini.
\begin{defprop}[Couple d’événements indépendants]
    Deux événements \(A\) et \(B\) sont dits indépendants si \(\mathbb{P} (A \inter B)\) = \(\mathbb{P} (A) \mathbb{P} (B\)) .\\
    \underline{Remarques}\\
    \begin{itemize}
        \item Si \(B\) est de probabilité non nulle, l’indépendance de \(A\) et \(B\) se traduit par \(\probacond{A}{B}  = \mathbb{P} (A)\).
        \item La notion d’indépendance est une notion probabiliste (et pas ensembliste) à ne pas confondre avec la notion ensembliste d’événements disjoints/incompatibles
    \end{itemize}
\end{defprop}

\begin{defprop}[Famille finie d’événements indépendants]
    Soit \(n \in N\) tel que \(n \geq 2\).\\~\\
    Les événements \(A_1, \dots , A_n\) sont dits indépendants si, pour tout \(I \subset \interventierii{1}{n}\),
    \[\proba{\biginter_{i\in I}A_i}= \prod_{i\in I}\paren{\proba{A_i}}\]
    \underline{Remarques} \\
    \begin{itemize}
        \item On parle parfois de mutuelle indépendance au lieu d’indépendance.
        \item Pour \(n\) entier naturel supérieur ou égal à \(2\),
        \item si les événements \(A_1, \dots , A_n\) sont indépendants alors ils sont deux à deux indépendants.
        \item la réciproque de l’implication précédente est fausse pour \(n \geq 3\). Des événements peuvent être indépendants deux à deux sans être indépendants.
    \end{itemize}
\end{defprop}

\begin{defprop}[Indépendance et événements contraires]
    \begin{itemize}
        \item Si \(A\) et \(B\) sont deux événements indépendants alors \(A\) et \(\conj{B}\) sont indépendants.
        \item Plus généralement, pour \(n\) entier naturel supérieur ou égal à \(2\),\\~\\
        si les événements \(A_1, \dots , A_n\) sont indépendants alors les événements \( A'_1, \dots , A'_n\) tels que
        \[\forall i \in \interventierii{1}{n} , A'_i \in \accol{A_i, \conj{A_i}}\]
        le sont aussi.
    \end{itemize}
\end{defprop}