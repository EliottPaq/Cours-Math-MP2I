% Set up the document's format to A4 and the font's size to 12pt.
\documentclass[a4paper,12pt]{report}

% Set up the document's title, author and date.
\title{Maths -- MP2I}
\author{Eliott Paquet}
\date{\today}

% Set up the input's encoding to UTF-8, the document's font and language to T1 (adapted to french) and french (the grammar linter uses this parameter).
\usepackage[utf8]{inputenc}
\usepackage[T1]{fontenc}
\usepackage[french]{babel}

\usepackage[dvipsnames]{xcolor}

% Set up the document's margins.
\usepackage{geometry}
\geometry{hmargin=1.5cm,vmargin=1.5cm}

% The three main maths packages. They are used for a lot of things.
\usepackage{amssymb,amsmath}
\usepackage{mathtools}

% Useful to create nice and easy signs or variations tables.
\usepackage{tkz-tab}

% Useful to create any kind of visual representation (graph functions, illustrate geometry problems, etc)
\usepackage{tikz}
\usetikzlibrary{patterns,angles,quotes,arrows,arrows.meta,bending,matrix,calc}

% Allows to edit the itemize environment's default item document-wide.
\usepackage{enumitem}

% Allows to define \notfoo or \nfoo (not recommended) in order for \not\foo to work as wished.
\usepackage{newtxmath}
\DeclareSymbolFont{CMletters}{OML}{cmm}{m}{it}
\DeclareMathSymbol{\nu}{\mathord}{CMletters}{23}

% Makes the table of contents clickable and gives useful commands for links in general.
\usepackage[hypertexnames=false]{hyperref}
\hypersetup{colorlinks=false,linktoc=all}

% Gives the llbracket and rrbracket commands for integer intervals.
\usepackage{stmaryrd}

% Useful to insert nice-looking quotes.
\usepackage{epigraph}

% Allows to insert chapter-specific table of contents.
\usepackage{minitoc}
\mtcselectlanguage{french}
\setcounter{minitocdepth}{6}

% Useful when units are needed.
\usepackage{siunitx}
\sisetup{
locale=FR,
detect-all,
inter-unit-product=\ensuremath{\cdot},
list-final-separator={et},
list-pair-separator={et},
range-phrase={\ensuremath{\xleftrightarrow{}}},
exponent-product=\ensuremath{\cdot},
per-mode=power-positive-first
}

\usepackage[thmmarks,hyperref]{ntheorem}
\makeatletter
\let\old@thm\@thm
\usepackage[lowercase]{theoremref}
\def\@thm#1#2#3{\def\thmref@currname{#3}\old@thm{#1}{#2}{#3}}
\makeatother

% Allows whiteboard digits with \mathds
\usepackage{dsfont}

\usepackage{needspace}

% Useful for better-looking oneline fractions
\usepackage{nicefrac}

% Set up the horizontal space before the first line of a new paragraph to 2em and the vertical space between two paragraphs to 1em.
\setlength{\parindent}{0pt}
\setlength{\parskip}{1em}

% Adds 0.5em to the vertical space between two lines in an align environment. It looks better.
\addtolength{\jot}{0.5em}

% Allows align environment to break if it's too long to fit in the page where it began.
\allowdisplaybreaks[1]

% Trick to make semicolons considered like relation operators (such as =) and therefore being equidistantly spaced from the two numbers around it.
\mathcode`;=\numexpr\mathcode`;-"3000

% Commands for size-adaptative parentheses, brackets, curly brackets, absolute value and magnitude.
\newcommand{\paren}[1]{\left(#1\right)} % (x)
\newcommand{\croch}[1]{\left[#1\right]} % [x]
\newcommand{\accol}[1]{\left\lbrace#1\right\rbrace} % {x}
\newcommand{\abs}[1]{\left\lvert#1\right\rvert} % |x|
\newcommand{\norme}[1]{\left\|#1\right\|} % ||x||
\newcommand{\floor}[1]{\left\lfloor#1\right\rfloor} % ⌊x⌋
\newcommand{\ceil}[1]{\left\lceil#1\right\rceil} % ⌈x⌉

% Commands for size-adaptative intervals and integer intervals. The commands' roots are "interv" and "interventier" and the added e or i at the end mean "excluded" and "included" respectively.
\newcommand{\intervii}[2]{\left[#1;#2\right]} % [a;b]
\newcommand{\intervee}[2]{\left]#1;#2\right[} % ]a;b[
\newcommand{\intervie}[2]{\left[#1;#2\right[} % [a;b[
\newcommand{\intervei}[2]{\left]#1;#2\right]} % ]a;b]
\newcommand{\interventierii}[2]{\left\llbracket#1;#2\right\rrbracket} % non-ASCII characters needed
\newcommand{\interventieree}[2]{\left\rrbracket#1;#2\right\llbracket} % non-ASCII characters needed
\newcommand{\interventierie}[2]{\left\llbracket#1;#2\right\llbracket} % non-ASCII characters needed
\newcommand{\interventierei}[2]{\left\rrbracket#1;#2\right\rrbracket} % non-ASCII characters needed

% Commands for usually used sets.
\newcommand{\N}{\mathbb{N}} % natural integers
\newcommand{\Ns}{\mathbb{N}^*}

\newcommand{\Z}{\mathbb{Z}} % relative integers
\newcommand{\Zp}{\mathbb{Z}_+}
\newcommand{\Zs}{\mathbb{Z}^*}
\newcommand{\Zps}{\mathbb{Z}_+^*}
\newcommand{\Zm}{\mathbb{Z}_-}
\newcommand{\Zms}{\mathbb{Z}_-^*}

\newcommand{\D}{\mathbb{D}} % decimal numbers
\newcommand{\Dp}{\mathbb{D}_+}
\newcommand{\Dm}{\mathbb{D}_-}
\newcommand{\Ds}{\mathbb{D}^*}
\newcommand{\Dps}{\mathbb{D}_+^*}
\newcommand{\Dms}{\mathbb{D}_-^*}

\newcommand{\Q}{\mathbb{Q}} % rational numbers
\newcommand{\Qp}{\mathbb{Q}_+}
\newcommand{\Qm}{\mathbb{Q}_-}
\newcommand{\Qs}{\mathbb{Q}^*}
\newcommand{\Qps}{\mathbb{Q}_+^*}
\newcommand{\Qms}{\mathbb{Q}_-^*}

\newcommand{\R}{\mathbb{R}} % real numbers
\newcommand{\Rp}{\mathbb{R}_+}
\newcommand{\Rm}{\mathbb{R}_-}
\newcommand{\Rs}{\mathbb{R}^*}
\newcommand{\Rps}{\mathbb{R}_+^*}
\newcommand{\Rms}{\mathbb{R}_-^*}
\newcommand{\Rb}{\overline{\mathbb{R}}}

\newcommand{\C}{\mathbb{C}} % complex numbers
\newcommand{\Cs}{\mathbb{C}^*}

\newcommand{\K}{\mathbb{K}}
\newcommand{\Ks}{\mathbb{K}^*}

\newcommand{\A}{\mathbb{A}}
\renewcommand{\L}[2]{\mathscr{L}\paren{#1,#2}}
\newcommand{\Lendo}[1]{\mathscr{L}\paren{#1}}

\newcommand{\prem}{\mathbb{P}}

\newcommand{\U}{\mathbb{U}} % complex numbers whose modulus is 1

\renewcommand{\P}[1]{\mathscr{P}\paren{#1}} % subsets of a set
\newcommand{\Pf}[1]{\mathscr{P}_f\paren{#1}} % finite subsets of a set
\newcommand{\F}[2]{\mathscr{F}\paren{#1,#2}} % functions from 1 to 2
\newcommand{\V}[1]{\mathscr{V}\paren{#1}} % neighborhood of a number

% Redefines \Re and \Im to print Re and Im (the same way as ln or lim) instead of fraktur R and I which don't look nice and are less readable.
\newcommand{\Reel}[1]{\operatorname{Re}\paren{#1}}
\newcommand{\Ima}[1]{\operatorname{Im}\paren{#1}}
\newcommand{\Card}{\operatorname{Card}}

% Command to print an upright e for the exponential instead of a slanted e and put the exponent.
\newcommand{\e}[1]{\mathrm{e}^{#1}}

% Command to print the imaginary i with a little space on the right. This way, the exponents don't look confusing. \i normally prints a dotless i.
\renewcommand{\i}{i\mkern1mu}

% Redefines \vec such that the arrow covers the whole name of the vector.
%\renewcommand{\vec}[1]{\overrightarrow{#1}}

% Commands for 2D and 3D vectors' coordinates
\newcommand{\dcoords}[2]{\begin{pmatrix}#1\\#2\end{pmatrix}}
\newcommand{\tcoords}[3]{\begin{pmatrix}#1\\#2\\#3\end{pmatrix}}

% Redefines binom to print nicer parentheses around the numbers.
\renewcommand{\binom}[2]{\begin{pmatrix}#2\\#1\end{pmatrix}}

% Command for a QED black square. It automatically prints a whitespace before the square such that it looks nice.
\newcommand{\cqfd}{\text{ }\blacksquare}

%Acronym for "privé de" 
\newcommand{\pd}{\backslash}


% Commands with more explicit names for the best way to express divisibility (mid and nmid).
\newcommand{\divise}{\mid}
\newcommand{\notdivise}{\nmid}

% Commands that do the exact same thing but with explicit names for a complex number's conjugate and an event's negation in probability.
\newcommand{\conj}[1]{\overline{#1}}

% Command for a size-adaptative middle bar meaning "such that" (with spacing around it in order to look nice).
\newcommand{\tq}{\;\middle|\;}

% Command with an explicit name for the scalar product.
\newcommand{\scal}{\cdot}
\newcommand{\vecto}{\operatorname{_\wedge}}

% Shortcut for forcing displaystyle in inline mode.
\newcommand{\ds}{\displaystyle}

% Make the not version of implies, impliedby and iff look nice.
\newcommand{\notimp}{\centernot{\imp}}
\newcommand{\notimpr}{\centernot{\impr}}
\newcommand{\notssi}{\centernot{\ssi}}

\renewcommand{\subset}{\subseteq}
\renewcommand{\supset}{\supseteq}
\newcommand{\notsubset}{\centernot{\subset}}
\newcommand{\notsupset}{\centernot{\supset}}

% Shortcut for P(event).
\newcommand{\proba}[1]{\mathbb{P}\paren{#1}}
\newcommand{\probacond}[2]{\mathbb{P}_{#2}\paren{#1}}

% More explicit names for land (logical and) and lor (logical or).
\newcommand{\et}{\land}
\newcommand{\ou}{\lor}
\newcommand{\non}{\lnot}

% Explicitly named environment for tkz-tab tables. Automatically centers the table and handles the tikzpicture environment.
\newenvironment{tkz}
{
\begin{tikzpicture}
}
{
\end{tikzpicture}
}

% More explicitly named commands for the creation of tkz-tab tables.
\newcommand{\tableauinit}[2]{\tkzTabInit{#1}{#2}}
\newcommand{\tableausignes}[1]{\tkzTabLine{#1}}
\newcommand{\tableauvariations}[1]{\tkzTabVar{#1}}

% Shortcut for the curve and the domain of the given function.
\newcommand{\graphe}[1]{\Gamma_{#1}}
\newcommand{\ensembledef}[1]{\mathcal{D}_{#1}}

\renewcommand{\S}[1]{\mathfrak{S}_{#1}}
\newcommand{\frakA}[1]{\mathfrak{A}_{#1}}

\newcommand{\semihrule}{\rule{256.074815pt}{0.4pt}}

% Various environments that create boxes. Each one is one type of thing (example, proof, etc). Each type has its own automatic counter.
\theoremstyle{break}
\theorembodyfont{\upshape}
\theoremheaderfont{\itshape}
\theoremprework{\bigskip\needspace{\baselineskip}\color{green}\hrule\color{black}}
\theorempostwork{\bigskip}
\newtheorem{rem}{Remarque}[chapter]

\theoremstyle{break}
\theorembodyfont{\upshape}
\theoremheaderfont{\itshape}
\theoremprework{\bigskip\needspace{\baselineskip}\color{green}\hrule\color{black}}
\theorempostwork{\bigskip}
\newtheorem{ex}[rem]{Exemple}

\theoremstyle{break}
\theorembodyfont{\upshape}
\theoremheaderfont{\itshape}
\theoremprework{\bigskip\needspace{\baselineskip}\color{green}\hrule\color{black}}
\theorempostwork{\bigskip}
\newtheorem{rappel}[rem]{Rappel}

\theoremstyle{break}
\theorembodyfont{\upshape}
\theoremheaderfont{\itshape}
\theoremprework{\bigskip\needspace{\baselineskip}\color{brown}\hrule\color{black}}
\theorempostwork{\bigskip}
\newtheorem{oubli}[rem]{Oubli}

\theoremstyle{break}
\theorembodyfont{\itshape}
\theoremheaderfont{\normalfont\bfseries}
\theoremprework{\bigskip\needspace{\baselineskip}\color{orange}\hrule\color{black}}
\theorempostwork{\bigskip}
\newtheorem{formu}[rem]{Formule}

\theoremstyle{break}
\theorembodyfont{\upshape}
\theoremheaderfont{\normalfont\bfseries}
\theoremprework{\bigskip\needspace{\baselineskip}\color{blue}\hrule\color{black}}
\theorempostwork{\bigskip}
\newtheorem{defi}[rem]{Définition}

\theoremstyle{break}
\theorembodyfont{\upshape}
\theoremheaderfont{\normalfont\bfseries}
\theoremprework{\bigskip\needspace{\baselineskip}\color{blue}\hrule\color{black}}
\theorempostwork{\bigskip}
\newtheorem{reform}[rem]{Reformulation}

\theoremstyle{break}
\theorembodyfont{\upshape}
\theoremheaderfont{\normalfont\bfseries}
\theoremprework{\bigskip\needspace{\baselineskip}\color{magenta}\hrule\color{black}}
\theorempostwork{\bigskip}
\newtheorem{exo}[rem]{Exercice}

\theoremstyle{break}
\theorembodyfont{\upshape}
\theoremheaderfont{\normalfont\bfseries}
\theoremprework{\bigskip\needspace{\baselineskip}\color{magenta}\semihrule\color{green}\semihrule\color{black}}
\theorempostwork{\bigskip}
\newtheorem{exoex}[rem]{Exercice/Exemple}

\theoremstyle{break}
\theorembodyfont{\upshape}
\theoremheaderfont{\normalfont\bfseries}
\theoremprework{\bigskip\needspace{\baselineskip}\color{blue}\semihrule\color{red}\semihrule\color{black}}
\theorempostwork{\bigskip}
\newtheorem{defprop}[rem]{Définition/Propriétés}

\theoremstyle{break}
\theorembodyfont{\upshape}
\theoremheaderfont{\normalfont\bfseries}
\theoremprework{\bigskip\needspace{\baselineskip}\color{blue}\semihrule\color{red}\semihrule\color{black}}
\theorempostwork{\bigskip}
\newtheorem{deftheo}[rem]{Définition/Théorème}

\theoremstyle{break}
\theorembodyfont{\upshape}
\theoremheaderfont{\normalfont\bfseries}
\theoremprework{\bigskip\needspace{\baselineskip}\color{blue}\hrule\color{black}}
\theorempostwork{\bigskip}
\newtheorem{nota}[rem]{Notation}

\theoremstyle{break}
\theorembodyfont{\upshape}
\theoremheaderfont{\itshape}
\theoremprework{\bigskip\needspace{\baselineskip}\color{blue}\hrule}
\theorempostwork{\hrule\color{black}\needspace{\baselineskip}\bigskip}
\newtheorem*{brouill}{Brouillon}

\theoremstyle{break}
\theorembodyfont{\itshape}
\theoremheaderfont{\normalfont\bfseries}
\theoremprework{\bigskip\needspace{\baselineskip}\color{red}\hrule\color{black}}
\theorempostwork{\bigskip}
\newtheorem{theo}[rem]{Théorème}

\theoremstyle{break}
\theorembodyfont{\upshape}
\theoremheaderfont{\normalfont\bfseries}
\theoremprework{\bigskip\needspace{\baselineskip}\color{red}\hrule\color{black}}
\theorempostwork{\bigskip}
\newtheorem{prop}[rem]{Propriétés}

\theoremstyle{break}
\theorembodyfont{\itshape}
\theoremheaderfont{\normalfont\bfseries}
\theoremprework{\bigskip\needspace{\baselineskip}\color{red}\hrule\color{black}}
\theorempostwork{\bigskip}
\newtheorem{cor}[rem]{Corollaire}

\theoremstyle{break}
\theorembodyfont{\itshape}
\theoremheaderfont{\normalfont\bfseries}
\theoremprework{\bigskip\needspace{\baselineskip}\color{red}\hrule\color{black}}
\theorempostwork{\bigskip}
\newtheorem{lem}[rem]{Lemme}

\theoremstyle{break}
\theorembodyfont{\upshape}
\theoremheaderfont{\normalfont\bfseries}
\theoremprework{\bigskip\needspace{\baselineskip}\color{violet}\hrule\color{black}}
\theorempostwork{\bigskip}
\newtheorem{meth}[rem]{Méthode}

\theoremstyle{break}
\theorembodyfont{\upshape}
\theoremheaderfont{\normalfont\bfseries}
\theoremprework{\bigskip\needspace{\baselineskip}\color{violet}\hrule\color{black}}
\theorempostwork{\bigskip}
\newtheorem{appl}[rem]{Application}

\theoremstyle{break}
\theorembodyfont{\upshape}
\theoremheaderfont{\normalfont\bfseries}
\theoremprework{\bigskip\needspace{\baselineskip}\color{violet}\hrule\color{black}}
\theorempostwork{\bigskip}
\newtheorem{abus}[rem]{Abus}

\theoremstyle{break}
\theorembodyfont{\upshape}
\theoremheaderfont{\normalfont\bfseries}
\theoremprework{\bigskip\needspace{\baselineskip}\color{violet}\hrule\color{black}}
\theorempostwork{\bigskip}
\newtheorem{algo}[rem]{Algorithme}

\theoremstyle{break}
\theorembodyfont{\upshape}
\theoremheaderfont{\normalfont\bfseries}
\theoremprework{\bigskip\needspace{\baselineskip}\color{violet}\hrule\color{black}}
\theorempostwork{\bigskip}
\newtheorem{bilan}[rem]{Bilan}

\theoremstyle{break}
\theorembodyfont{\upshape}
\theoremheaderfont{\itshape}
\theoremprework{\bigskip\needspace{\baselineskip}\color{BurntOrange}\hrule\color{black}}
\theorempostwork{\bigskip}
\newtheorem{corr}[rem]{Correction}

\theoremstyle{break}
\theorembodyfont{\upshape}
\theoremheaderfont{\itshape}
\theoremsymbol{\ensuremath{\cqfd}}
\theoremprework{\bigskip\needspace{\baselineskip}\color{yellow}\hrule\color{black}}
\theorempostwork{\bigskip}
\newtheorem{dem}[rem]{Démonstration}



% Commands to make proofs easier to write
\newcommand{\impdir}{\fbox{\(\imp\)}~}
\newcommand{\imprec}{\fbox{\(\impr\)}~}
\newcommand{\incdir}{\fbox{\(\subset\)}~}
\newcommand{\increc}{\fbox{\(\supset\)}~}
\newcommand{\leqbox}{\fbox{\(\leq\)}~}
\newcommand{\geqbox}{\fbox{\(\geq\)}~}
\newcommand{\unicite}{\fbox{unicité}~}
\newcommand{\existence}{\fbox{existence}~}
\newcommand{\analyse}{\fbox{analyse}~}
\newcommand{\synthese}{\fbox{synthèse}~}
\newcommand{\conclusion}{\fbox{conclusion}~}

\renewcommand{\to}{\longrightarrow}
\renewcommand{\mapsto}{\longmapsto}

\newcommand{\fonction}[5]{\begin{array}[t]{cccc}#1 : & #2 & \to & #3 \\ & #4 & \mapsto & #5\end{array}}
\newcommand{\fonctionlambda}[4]{\begin{array}[t]{ccc}#1 & \to & #2 \\ #3 & \mapsto & #4\end{array}}

\renewcommand{\leq}{\leqslant}
\renewcommand{\geq}{\geqslant}

\newcommand{\pinf}{+\infty}
\newcommand{\minf}{-\infty}

\newcommand{\id}[1]{\mathrm{id}_{#1}}

\renewcommand{\phi}{\varphi}
\renewcommand{\epsilon}{\varepsilon}

\newcommand{\ind}[1]{\mathds{1}_{#1}}

\newcommand{\iR}{\i\R}

\newcommand{\tcheby}[2]{T_{#1}\paren{#2}}
\newcommand{\utcheby}[2]{U_{#1}\paren{#2}}

\mathcode`l="8000
\begingroup
\makeatletter
\lccode`\~=`\l
\DeclareMathSymbol{\lsb@l}{\mathalpha}{letters}{`l}
\lowercase{\gdef~{\ifnum\the\mathgroup=\m@ne \ell \else \lsb@l \fi}}%
\endgroup

\newcommand{\ensvide}{\varnothing}

\newcommand{\rond}{\circ}

\newcommand{\union}{\cup}
\newcommand{\inter}{\cap}
\newcommand{\bigunion}{\bigcup}
\newcommand{\biginter}{\bigcap}

\newcommand{\ssi}{\iff}
\newcommand{\imp}{\implies}
\newcommand{\impr}{\impliedby}

\newcommand{\excluant}{\setminus}

\newcommand{\littletaller}{\mathchoice{\vphantom{\big|}}{}{}{}}
\newcommand{\restr}[2]{{
\left.\kern-\nulldelimiterspace#1\littletaller\right|_{#2}
}}
\newcommand{\corestr}[2]{{
\left.\kern-\nulldelimiterspace#1\littletaller\right|^{#2}
}}
\newcommand{\restrbar}[1]{{
\left.\kern-\nulldelimiterspace#1\littletaller\right|
}}

\newcommand{\rel}{\mathscr{R}}

\newcommand{\classesdequiv}[1]{\nicefrac{#1}{\sim}}

\newcommand{\majo}[1]{\mathrm{majorants}\paren{#1}}
\newcommand{\mino}[1]{\mathrm{minorants}\paren{#1}}

\newcommand{\ensdiv}[1]{\operatorname{div}\paren{#1}}

\newcommand{\E}[1]{\times 10^{#1}}

\setcounter{secnumdepth}{3}

\newcommand{\guillemets}[1]{\og #1 \fg{}}

\newcommand{\prim}{^{\,\prime}}
\newcommand{\seconde}{^{\,\prime\prime}}

\newcommand{\note}[1]{\textbf{\(\star\star\) #1 \(\star\star\)}}
\newcommand{\cad}{c'est-à-dire }
\newcommand{\Cad}{C'est-à-dire }
\newcommand{\ie}{\textit{i.e.} }
\newcommand{\cf}{\textit{cf.} }
\newcommand{\Cf}{\textit{Cf.} }

\usepackage{xparse}

\NewDocumentCommand{\quantifs}{>{\SplitList{;}}m}{\ProcessList{#1}{\insertquantif}}
\newcommand{\insertquantif}[1]{#1,\;\:}

\DeclareDocumentCommand{\groupe}{m O{+}}{\paren{#1,#2}}
\DeclareDocumentCommand{\anneau}{m O{+} O{\times}}{\paren{#1,#2,#3}}
\DeclareDocumentCommand{\corps}{m O{+} O{\times}}{\paren{#1,#2,#3}}

\DeclareDocumentCommand{\poly}{O{\K} O{X}}{#1\croch{#2}}
\DeclareDocumentCommand{\polydeg}{O{\K} m O{X}}{#1_{#2}\croch{#3}}
\DeclareDocumentCommand{\fracrat}{O{\K} O{X}}{#1\paren{#2}}

\DeclareDocumentCommand{\M}{m O{\K}}{\mathscr{M}_{#1}\paren{#2}}
\DeclareDocumentCommand{\sym}{m O{\K}}{\mathscr{S}_{#1}\paren{#2}}
\DeclareDocumentCommand{\antisym}{m O{\K}}{\mathscr{A}_{#1}\paren{#2}}
\DeclareDocumentCommand{\GL}{m O{\K}}{\operatorname{GL}_{#1}\paren{#2}}
\DeclareDocumentCommand{\SL}{m O{\K}}{\operatorname{SL}_{#1}\paren{#2}}
\DeclareDocumentCommand{\Mat}{O{\fami{B}} m}{\operatorname{Mat}_{#1}\paren{#2}}
\newcommand{\pass}[2]{\mathscr{P}_{#1\to#2}}

\DeclareDocumentCommand{\contm}{O{\intervii{a}{b}} O{\K}}{\classe{0}_m\paren{#1,#2}}
\DeclareDocumentCommand{\Esc}{O{\intervii{a}{b}} O{\K}}{\operatorname{Esc}\paren{#1,#2}}

\usepackage{witharrows}

\newcommand{\croix}{^{\times}}

\usepackage{polynom}

\newcommand{\classe}[1]{\mathscr{C}^{#1}}
\newcommand{\ensclasse}[3]{\classe{#1}\paren{#2,#3}}

\newcommand{\deriv}[1]{^{\paren{#1}}}

\usepackage{derivative}
\derivset{\pdv}[delims-eval=.)]

\DeclareMathOperator{\Arctan}{Arctan}
\DeclareMathOperator{\Arcsin}{Arcsin}
\DeclareMathOperator{\Arccos}{Arccos}
\DeclareMathOperator{\cotan}{cotan}
\DeclareMathOperator{\sh}{sh}
\DeclareMathOperator{\ch}{ch}
\DeclareMathOperator{\tth}{th} %we can't use \th since it's a imbed latex command
\DeclareMathOperator{\sg}{sg}
\DeclareMathOperator{\supp}{supp}
\DeclareMathOperator{\Supp}{Supp}
\DeclareMathOperator{\rg}{rg}
\DeclareMathOperator{\tr}{tr}

\newcommand{\Hom}[2]{\operatorname{Hom}\paren{#1,#2}}
\newcommand{\Pol}[2]{\operatorname{Pol}\paren{#1,#2}}
\newcommand{\Aut}[1]{\operatorname{Aut}\paren{#1}}
\DeclareDocumentCommand{\Vect}{O{} m}{\operatorname{Vect}_{#1}\paren{#2}}

\newcommand{\diag}[1]{\operatorname{diag}\paren{#1}}

\usepackage{abstract}
\addto\captionsfrench{\renewcommand{\abstractname}{\Large Introduction}}

\newcommand{\inv}{^{-1}}
\newcommand{\etoile}{^{*}}

\newcounter{orcounter}

\newenvironment{orlist}
{
\begin{array}{|l}
\setcounter{orcounter}{0}
}
{
\end{array}
}

\newcommand{\oritem}[1]{%
\ifthenelse{\theorcounter<1}{}{\\ \text{ou} \\}#1\stepcounter{orcounter}
}

\NewDocumentCommand{\orenv}{>{\SplitList{\\}}m}{%
\begin{orlist}\ProcessList{#1}{\oritem}\end{orlist}}

\newcounter{permuitemcounter}

\newcommand{\permuitem}[1]{%
\ifthenelse{\thepermuitemcounter<1}{}{&}#1\stepcounter{permuitemcounter}}

\NewDocumentCommand{\permu}{>{\SplitList{;}}m >{\SplitList{;}}m}{%
\begin{pmatrix}\setcounter{permuitemcounter}{0}\ProcessList{#1}{\permuitem} \\ \setcounter{permuitemcounter}{0}\ProcessList{#2}{\permuitem}\end{pmatrix}}

\NewDocumentCommand{\cycle}{>{\SplitList{;}}m}{%
\begin{pmatrix}\setcounter{permuitemcounter}{0}\ProcessList{#1}{\permuitem}\end{pmatrix}}

\usepackage{pgfplots}

\DeclareDocumentCommand{\pgcd}{o o}{
\IfNoValueTF{#1}{\operatorname{pgcd}}{\operatorname{pgcd}\paren{#1,#2}}
}

\DeclareDocumentCommand{\bezout}{o o}{
\IfNoValueTF{#1}{\operatorname{bezout}}{\operatorname{bezout}\paren{#1,#2}}
}


\newcommand{\valp}[2]{v_{#1}\paren{#2}}

\newcommand{\fami}[1]{\mathscr{#1}}

\newcommand{\echange}{\leftrightarrow}

\newcommand{\trans}[1]{\prescript{t}{}{#1}}

\usepackage{mathdots}

\DeclareDocumentCommand{\detb}{O{\fami{B}}}{{\det}_{#1}}

\usepackage{cancel}

\usepackage{nicematrix}

\newcommand{\ps}[2]{\left\langle#1\tq#2\right\rangle}
\newcommand{\ortho}{^{\perp}}

\newcommand{\operp}{\mathrel{%
\begin{tikzpicture}[baseline=-0.25em]
\draw (0,0) circle (0.45em);
\draw (-0.38em,-0.25em) -- (0.38em,-0.25em);
\draw (0,-0.25em) -- (0,0.45em);
\end{tikzpicture}
}%
}

\usepackage{titletoc}
\dottedcontents{section}[5.5em]{}{3.2em}{1pc}

\newcommand{\bouleo}[2]{\mathbb{B}\paren{#1,#2}}
\newcommand{\boulef}[2]{\mathbb{B}\prim\paren{#1,#2}}
\newcommand{\sphere}[2]{\mathbb{S}\paren{#1,#2}}

\newcommand{\vdv}[2]{\operatorname{D}_{#1}#2}

\newcommand{\egqd}[1]{\underset{#1}{=}}
\newcommand{\simqd}[1]{\underset{#1}{\sim}}

\newcommand{\mediumrightarrow}{\,\begin{tikzpicture}\draw[->] (0, 0) -- (1, 0);\end{tikzpicture}\,}
\newcommand{\tendqd}[1]{\underset{#1}{\mediumrightarrow}}

\newcommand{\arr}[2]{A_{#2}^{#1}}
\newcommand{\comb}[2]{C_{#2}^{#1}}

\newcommand{\loiuniforme}[1]{\mathscr{U}\paren{#1}}
\newcommand{\loibernoulli}[1]{\mathscr{B}\paren{#1}}
\newcommand{\loibinomiale}[2]{\mathscr{B}\paren{#1,#2}}

\newcommand{\esp}[1]{\operatorname{E}\paren{#1}}
\newcommand{\vari}[1]{\operatorname{V}\paren{#1}}
\newcommand{\cov}[2]{\operatorname{Cov}\paren{#1,#2}}
\newcommand{\ecarttype}[1]{\sigma\paren{#1}}

\renewcommand{\O}[1]{\mathscr{O}\paren{#1}}
\renewcommand{\o}[1]{o\paren{#1}}

\setcounter{MaxMatrixCols}{200}

\newcommand{\Com}[1]{\operatorname{Com}#1}

\usepackage{microtype}

\newcommand{\sig}[1]{\epsilon\paren{#1}}

\ExplSyntaxOn
\RenewDocumentCommand{\v}{m}{
    \int_compare:nTF { \tl_count:n { #1 } > 1 }
    {
        \overrightarrow{#1}
    }
    {
        \vec{#1}
    }
}
\ExplSyntaxOff

\begin{document}
\renewcommand{\labelitemi}{\(\bullet\)}
\renewcommand{\labelenumi}{(\arabic{enumi})}

\everymath{\ds}

\maketitle

\begin{abstract}
	Ce document réunit l'ensemble de mes cours de Mathématiques de MP2I, ainsi que les TDs (travaux dirigés) les accompagnant. J'ai adapté certaines formulations me paraissant floues ou ne me plaisant pas mais le contenu pur des cours est strictement équivalent. Le document est organisé selon la hiérarchie suivante : chapitre, I), 1), a).

	Les éléments des tables des matières initiale et présentes au début de chaque chapitre sont cliquables (amenant directement à la partie cliquée). C'est également le cas des références à des éléments antérieurs de la forme, par exemple, \guillemets{Démonstration 5.22}.

	Dernier TD corrigé : aucun.
\end{abstract}

\dominitoc\tableofcontents

\part{Cours}

\chapter{trigonométrie (Rappels et compléments)}

\minitoc

Dans ce chapitre, on rappelle ce qui a été vu en trigonométrie au lycée et on complète avec les formules
d’addition et de duplication ainsi que l’étude de la fonction tangente.

\section{Cercle trigonométrique}

On se place dans le plan muni d'un repère orthonormé \(\paren{O,\vec{i},\vec{j}}\)

\begin{defi}[Cercle trigonométrique]

	On appelle cercle trigonométrique le cercle de centre \(O\) et de rayon \(1\)

\end{defi}

\begin{prop}[enroulement de la droite des réels sur le cercle trigonométrique]
	Soit \(M\) un point du plan. \\
	Le point \(M\) appartient au cercle trigonométrique si, et seulement si, il existe un réel \(t\) tel que les coordonnées de \(M\) dans le repère orthonormé \(\paren{O,\vec{i},\vec{j}}\) sont \(\paren{\cos t ; \sin t}\)
\end{prop}

\subsection{Relation de congruence modulo \(2\pi\) sur \(\R\)}

\begin{defi}
	Deux réels \(a\) et \(b\) sont dits congrus modulo \(2\pi\) s'il existe un entier relatif \(k\) tel que \(a-b = 2k\pi\)
	\underline{Notation} : \(a \equiv b \croch{2 \pi} \)
\end{defi}

\begin{defprop}
	On dit que la relation \(\equiv\) est une relation d'équivalence sur \(\R\) car elle vérifie les propriétés suivantes :
	\begin{enumerate}
		\item Pour tout réel x, on a : \(x \equiv x \croch{2 \pi}\). \hfill (réfléxivité)
		\item Pour tout couple de réels \(\paren{x,y}\) tel que \( x \equiv y \croch{2 \pi} \), on a :\( y \equiv x \croch{2 \pi} \) \hfill (symétrie)
		\item Pour tout triplet de réels \(\paren{x,y,z}\) tel que \(x \equiv y \croch{2 \pi} \) et \( y \equiv z \croch{2 \pi} \), on a : \( x \equiv z \croch{2 \pi} \) \hfill (transitivité)
	\end{enumerate}
\end{defprop}



\section{Cosinus et sinus}
\subsection{Formules et valeur remarquables}

\begin{formu}[Formule de base]
	Pour tout réel \(t\), on a :
	\begin{enumerate}
		\item \( \cos\paren{\pi - t} = -\cos t \) et \( \sin\paren{\pi - t} = \sin t \) \\
		\item \( \cos\paren{\pi + t} = -\cos t \) et \( \sin\paren{\pi + t} = -\sin t \) \\
		\item \( \cos\paren{\frac{\pi}{2} - t} = \sin t \) et \( \sin\paren{\frac{\pi}{2} - t} = \cos t \) \\
		\item \( \cos\paren{\frac{\pi}{2} + t} = -\sin t \) et \( \sin\paren{\frac{\pi}{2} + t} = \cos t \) \\
	\end{enumerate}
	\begin{tabular}{|c|c|c|c|c|c|}

		\hline
		\(t\)       & \(0\) & \(\frac{\pi}{6}\)      & \(\frac{\pi}{4}\)      & \(\frac{\pi}{3}\)       & \(\frac{\pi}{2}\) \\
		\hline
		\(\cos t \) & \(1\) & \(\frac{\sqrt{3}}{2}\) & \(\frac{\sqrt{2}}{2}\) & \(\frac{1}{2}\)        & \(0\)             \\
		\hline
		\(\sin t \) & \(0\) & \(\frac{1}{2}\)        & \(\frac{\sqrt{2}}{2}\) & \(\frac{\sqrt{3}}{2}\) & \(1\)             \\
		\hline
	\end{tabular}
\end{formu}

\begin{rem}
	Soient \(a\) et \(b\) des réels :
	\begin{itemize}
		\item
		      \(
		      \begin{aligned}
			      \cos a = \cos b
			      \iff\
			       & \left\{
			      \begin{aligned}
				      a & \equiv b \croch{2\pi}  \\
				        & \text{ou}              \\
				      a & \equiv -b \croch{2\pi}
			      \end{aligned}
			      \right.
			      \iff\
			       & \left\{
			      \begin{aligned}
				      \quantifs{\exists k \in \Z}  & a = b +2 k \pi   \\
				                                   & \text{ou}        \\
				      \quantifs{\exists k' \in \Z} & a = -b +2 k' \pi
			      \end{aligned}
			      \right.
		      \end{aligned}
		      \)\\
		      \item\(
		      \begin{aligned}
			      \sin a = \sin b
			      \iff\
			       & \left\{
			      \begin{aligned}
				      a & \equiv b \croch{2\pi}     \\
				        & \text{ou}                 \\
				      a & \equiv \pi-b \croch{2\pi}
			      \end{aligned}
			      \right.
			      \iff\
			       & \left\{
			      \begin{aligned}
				      \quantifs{\exists k \in \Z}  & a = b +2 k \pi      \\
				                                   & \text{ou}           \\
				      \quantifs{\exists k' \in \Z} & a = \pi-b +2 k' \pi
			      \end{aligned}
			      \right.
		      \end{aligned}
		      \)
	\end{itemize}

\end{rem}

\begin{formu} [Formule d'addition]
	Pour tout couple de réels \(\paren{a,b}\) on a :
	\begin{enumerate}
		\item \( \cos\paren{a+b} = \cos\paren{a} \cos\paren{b} - \sin\paren{a} \sin\paren{b} \) \\
		\item \( \cos\paren{a-b} = \cos\paren{a} \cos\paren{b} + \sin\paren{a} \sin\paren{b} \)\\
		\item \( \sin\paren{a+b} = \sin\paren{a} \cos\paren{b} + \cos\paren{a} \sin\paren{b} \) \\
		\item \( \sin\paren{a-b} = \sin\paren{a} \cos\paren{b} - \cos\paren{a} \sin\paren{b} \)\\
	\end{enumerate}
\end{formu}

\begin{formu}[Formule de simpson]
	Pour tout couple de réels \(\paren{a,b}\) on a :
	\begin{enumerate}
		\item \( \sin\paren{a+b} + \sin\paren{a-b} = 2\sin\paren{a} \cos\paren{b} \iff \frac{1}{2}\paren{\sin\paren{a+b} + \sin\paren{a-b}} = \sin\paren{a} \cos\paren{b}\) \\
		\item \( \cos\paren{a+b} + \cos\paren{a-b} = 2\cos\paren{a} \cos\paren{b} \iff \frac{1}{2}\paren{\cos\paren{a+b} + \cos\paren{a-b}} = \cos\paren{a} \cos\paren{b}\)

	\end{enumerate}

\end{formu}

\begin{appl}
	Calcul : \[\int_{0}^{\pi} \sin\paren{x} \cos\paren{3x} dx = \int{0}^{\pi} \frac{1}{2} \paren{\sin\paren{4x}+\sin(2x) dx} = 0\]
\end{appl}

\begin{formu}[Formule de duplication]
	Pour tout réel \(a\), on a :
	\begin{enumerate}
		\item \(\cos\paren{2a} = \cos^2\paren{a} - \sin^2\paren{a} = 2\cos^2(a)-1 = 1-\sin^2(a) \)
		\item \(\sin(2a) = 2\cos(a)\sin(a) \)
	\end{enumerate}
\end{formu}

\begin{prop}[Sinus et Cosinus]
	\begin{itemize}
		\item La fonction \(\cos\) est définie sur \(\R\), paire et périodique de période \(2\pi\). Elle est dérivable sur \(\R\) et sa dérivée vérifie \(\cos' = -\sin\)
		\item La fonction \(\sin\) est définie sur \(\R\), impaire et périodique de période \(2\pi\). Elle est dérivable sur \(\R\) et sa dérivée vérifie \(\sin' = \cos\)
	\end{itemize}
\end{prop}

\begin{prop}[Inégalité remarquable]
	Pour tout réel \(t\), on a : \(\abs{\sin(t)} \leq \abs{t}\)
\end{prop}

\section{La fonction tangente}
\begin{defi}
	La fonction \(\frac{\sin}{\cos} \) est appelée la fonction tangente et notée \(\tan\)
\end{defi}

\begin{prop}
	La fonction \(\tan\) est définie sur \(\R\backslash\accol{\frac{\pi}{2}+k\pi\tq k\in \Z}\), impaire et périodique de période \(\pi\). Elle est dérivable sur \(\R\)\(\R\backslash\accol{\frac{\pi}{2}+k\pi\tq k\in \Z}\) et sa dérivée vérifie \(\tan' = 1+\tan = \frac{1}{tan^2}\)
\end{prop}

\begin{formu}
	Pour tout réel \(t\), on a :
	\begin{enumerate}
        \item \(tan(\pi-t) = -\tan(t)\)
        \item \(tan(\pi+t) = \tan(t) \)
        \item \begin{tabular}{|c|c|c|c|c|c|}

		\hline
		\(t\)       & \(0\) & \(\frac{\pi}{6}\)      & \(\frac{\pi}{4}\)      & \(\frac{\pi}{3}\)       & \(\frac{\pi}{2}\) \\
		\hline
		\(\tan t \) & \(0\) & \(\frac{1}{\sqrt{3}}\) & \(1\) & \(\sqrt{3}\)       & NULL             \\
		\hline
	\end{tabular}
    \end{enumerate}
\end{formu}

\begin{formu}[addition et duplication]
    Pour tout couple de réels \(\paren{a,b}\) n'appartenant pas à l'ensemble \(\accol{\frac{\pi}{2}+k\pi\tq k\in \Z}\), on a :
    \begin{enumerate}
        \item Si \(a+b\) n'appartient pas à l'ensemble \(\accol{\frac{\pi}{2}+k\pi\tq k\in \Z}\) alors \(\tan(a+b) = \frac{\tan(a)+\tan(b)}{1-\tan(a) \tan(b)}\)
        \item Si \(a-b\) n'appartient pas à l'ensemble \(\accol{\frac{\pi}{2}+k\pi\tq k\in \Z}\) alors \(\tan(a-b) = \frac{\tan(a)-\tan(b)}{1+\tan(a) \tan(b)}\)
        \item Si \(2a\)  n'appartient pas à l'ensemble \(\accol{\frac{\pi}{2}+k\pi\tq k\in \Z}\) alors \(\tan(2a) = \frac{2\tan(a)}{1-\tan^2(a)} \)
    \end{enumerate}
\end{formu}

\begin{exoex}
Soit \(t\) réel n'appartenant pas à \(\accol{\frac{\pi}{4}+k\frac{\pi}{2}\tq k\in \Z}\) : 
    \begin{align*}
        \sin(t) &= 2\sin\paren{\frac{t}{2}}\cos\paren{\frac{t}{2}} \\
        &= \frac{2\sin\paren{\frac{t}{2}}}{\cos\paren{\frac{t}{2}}}\cos^2\paren{\frac{t}{2}}\\
        &= \frac{1}{1+\tan^2\paren{\frac{t}{2}}}\times 2 \tan\paren{\frac{t}{2}} \\
        &=\frac{2 \tan\paren{\frac{t}{2}}}{1+\tan^2\paren{\frac{t}{2}}} 
    \end{align*}
\end{exoex}

\chapter{Inégalité et fonction (rappel et compléments)}

\minitoc

Dans ce chapitre, sont rassemblés des rappels ou compléments sur les inégalités ainsi que des fondamentaux sur les fonctions de variable réelle à valeurs réelles (sans preuve ni évocation de continuité).

\section{Inégalité}

\subsection{Relation d'ordre sur \(\R\)}

\begin{defi}
	On dit que la relation \(\leq\) est une relation d'équivalence sur \(\R\) car elle vérifie les propriétés suivantes :
	\begin{enumerate}
		\item Pour tout réel x, on a : \(x \leq x \). \hfill (réfléxivité)
		\item Pour tout couple de réels \(\paren{x,y}\) tel que \( x \leq y  \) et \(y \leq x\), on a :\( y = x  \) \hfill (antisymétrie)
		\item Pour tout triplet de réels \(\paren{x,y,z}\) tel que \(x \leq y  \) et \( y \leq z  \), on a : \( x \leq z  \) \hfill (transitivité)
	\end{enumerate}
\end{defi}

\begin{prop}[Compatibilité avec les opérations]
	Soit \(x,y,z,t\) et \(a\) des réels.
	\begin{enumerate}
		\item Si \(x\leq y\) et \(z\leq t\) alors \(x+z\leq y +t \)
		\item Si \(x\leq y \) et \( 0 \leq a\) alors \(a x \leq a y\)
		\item Si \(x\leq y \) et \( a \leq 0\) alors \(a y \leq a x\)
		\item Si \( 0 \leq x \leq y \) et \( 0\leq z \leq t \) alors \( 0 \leq xz \leq y t \)
	\end{enumerate}
\end{prop}

\begin{nota}[Intervalles de \(\R\)]
	Les partie \(I\) de \(\R\) pouvant s’écrire sous l’une des formes suivantes sont dites intervalles de \(\R\) :
	\begin{itemize}
		\item \(I = \emptyset\) \\
		\item \(I = \accol{x \in \R\tq a \leq x \leq b} \underset{\mathrm{notation}}{=} \intervii{a}{b}\) avec \(\paren{a,b} \in \R^2 \) et \(a\leq b \) \\
		\item \(I = \accol{x \in \R\tq a \leq x < b} \underset{\mathrm{notation}}{=} \intervie{a}{b}\) avec \(\paren{a,b} \in \R\times \paren{\R \union \accol{\pinf}} \) et \(a < b\) \\
		\item \(I = \accol{x \in \R\tq a < x \leq b} \underset{\mathrm{notation}}{=} \intervei{a}{b}\) avec \(\paren{a,b} \in \paren{\R \union \accol{\minf}}\times \R \) et \(a < b\) \\
		\item \(I = \accol{x \in \R\tq a < x \leq b} \underset{\mathrm{notation}}{=} \intervee{a}{b}\) avec \(\paren{a,b} \in \paren{\R \union \accol{\minf}}\times  \paren{\R \union \accol{\pinf}} \) et \(a < b\) \\

	\end{itemize}
\end{nota}

\begin{prop}
	\begin{enumerate}
		\item Passage à l'inverse dans une inégalité
		      \[\quantifs{\forall x \in \Rps ; \forall y \in \Rps} x\leq y \iff \frac{1}{y} \leq \frac{1}{x}\]
		      \[\quantifs{\forall x \in \Rms ; \forall y \in \Rms} x\leq y \iff \frac{1}{y} \leq \frac{1}{x}\] \\
		\item Passage au carré dans une inégalité
		      \[\quantifs{\forall x \in \Rps ; \forall y \in \Rps} x\leq y \iff x^2 \leq y^2\]
		      \[\quantifs{\forall x \in \Rms ; \forall y \in \Rms} x\leq y \iff y^2 \leq x^2\] \\
		\item Passage à la racine carrée dans une inégalité
		      \[\quantifs{\forall x \in \Rp ; \forall y \in \Rp} x\leq y \iff \sqrt{x}\leq \sqrt{y}\] \\
		\item Passage à l’exponentielle ou au logarithme népérien dans une inégalité
		      \[\quantifs{\forall x \in \R ; \forall y \in \R} x\leq y \iff \e{x}\leq \e{y}\]
		      \[\quantifs{\forall x \in \Rps ; \forall y \in \Rps} x\leq y \iff \ln{x}\leq \ln{y}\] \\
	\end{enumerate}
\end{prop}

\begin{exoex}
	Montrer \(\quantifs{\forall x \in \intervii{0}{1}} x(1-x) \leq \frac{1}{4}\).
\end{exoex}

\begin{corr}[2 Méthode]
	Soit \(x \in \intervii{0}{1} \)
	\begin{enumerate}
		\item Raisonnement par équivalence
		      \[\begin{aligned}
				      x(1-x) \leq \frac{1}{4} & \iff 0 \leq \frac{1}{4}-x(1-x)     \\
				                              & \iff 0\leq x^2 -x +  \frac{1}{4}   \\
				                              & \iff 0\leq\paren{x- \frac{1}{2}}^2
			      \end{aligned}
		      \]
		      Ceci étant vrai \(\quantifs{\forall x\in \intervii{0}{1}}\) car \(\Delta = 0\) et \(x_0 =  \frac{1}{2}\), on conclut \(\quantifs{\forall x \in \intervii{0}{1}} x(1-x) \leq \frac{1}{4}\).\\
		\item étude de la fonction \(\fonction{f}{\intervii{0}{1}}{\R}{x}{\frac{1}{4}-x(1-x)}\)\\
	\end{enumerate}
\end{corr}


\begin{exoex}
    ~\\
	Montrer \(\quantifs{\forall x \in \Rps} x+\frac{1}{x}\geq 2\).
\end{exoex}

\begin{corr}
	Soit \(x \in \Rps \)

	\[\begin{aligned}
			x+\frac{1}{x}\geq 2 & \iff \frac{x^2+1}{x}\geq 2 \\
			                    & \iff x^2-2x+1\geq    0     \\
			                    & \iff (x-1)^2 \geq 0
		\end{aligned}
	\]
	Ceci étant vrai \(\quantifs{\forall x\in \Rps}\) on conclut \(\quantifs{\forall x \in \Rps} x+\frac{1}{x}\geq 2\).
\end{corr}

\begin{exoex}
    ~\\
	Encadrer \(\frac{2x^2-x+1}{x^2+\sqrt{x+2}+3}\) pour \(x \in \intervii{-1}{1}\).
\end{exoex}

\begin{corr}
	Soit \(x \in \intervii{-1}{1} \)
	\begin{enumerate}
		\item \underline{numérateur} :
		      \[\begin{aligned}
				      -1 \leq x\leq 1 & \iff 0 \leq x^2 \leq 1      \\
				                      & \iff 0 \leq 2x^2 \leq 2     \\
				                      & \iff 0 \leq 2x^2-x+1 \leq 4
			      \end{aligned}
		      \]

		\item \underline{denominateur} : \[\begin{aligned}
				      -1 \leq x\leq 1 & \iff 0 \leq x^2 \leq 1                                                      \\
				                      & \iff 4 \leq x^2 +\sqrt{x+2}+3 \leq 4+\sqrt{3}                               \\
				                      & \iff \frac{1}{4+\sqrt{3}} \leq \frac{1}{x^2 +\sqrt{x+2}+3 }\leq \frac{1}{4} \\
			      \end{aligned}
		      \]
	\end{enumerate}
	Ainsi par produit des deux inégalités on as \(0\leq\frac{2x^2-x+1}{x^2+\sqrt{x+2}+3}\leq1\) pour \(x \in \intervii{-1}{1}\).
\end{corr}

\begin{exoex}
    ~\\
	Encadrer \(\frac{x-y^2+3}{x^2+y^2-y}\) pour \(\forall \paren{x,y} \in \intervii{1}{2}^2\).
\end{exoex}

\begin{corr}
	Soit \(x \in \intervii{-1}{1} \)
	\begin{enumerate}
		\item \underline{numérateur} :
		      \[\begin{aligned}
				      1-4+3\leq x-y^2+3 \leq 2-1+4 & \iff 0 \leq x-y^2+3 \leq 5
			      \end{aligned}
		      \]

		\item \underline{denominateur} : \[\begin{aligned}
				      0 \leq y-1\leq 1 & \iff 0 \leq y^2-y \leq y                        \\
				                       & \iff 0 \leq y^2-y \leq 2                        \\
				                       & \iff 1 \leq x^2+y^2-y\leq 6                     \\
				                       & \iff \frac{1}{6} \leq \frac{1}{x^2+y^2-y}\leq 1 \\
			      \end{aligned}
		      \]
	\end{enumerate}
	Ainsi par produit des deux inégalités on as \(0\leq \frac{x-y^2+3}{x^2+y^2-y} \leq 5\) pour \(\forall \paren{x,y} \in \intervii{1}{2}^2\).
\end{corr}

\begin{defi}[Parties majorées, majorants, maximum]
	Une partie \(A\) de \(\R\) est dite majorée s’il existe un réel \(M\) tel que, pour tout réel \(x\) de \(A\), on a : \(x \leq M\). \\
	Un tel réel \(M\) est alors dit :
	\begin{itemize}
		\item majorant de \(A\) dans le cas général. \\
		\item maximum de \(A\) dans le cas particulier où \(M\) appartient à \(A\).\\
	\end{itemize}

\end{defi}

\begin{defi}[Parties minorées, minorants, minimum]
	Une partie \(A\) de \(\R\) est dite minorée s’il existe un réel \(m\) tel que, pour tout réel \(x\) de \(A\), on a : \(m\leq x\). \\
	Un tel réel \(m\) est alors dit :
	\begin{itemize}
		\item minorant  de \(A\) dans le cas général. \\
		\item minimum  de \(A\) dans le cas particulier où \(m\) appartient à \(A\).\\
	\end{itemize}

\end{defi}

\begin{exoex}
    ~\\
	Que dire de \(B = \accol{\frac{n}{n^2+1} \tq n \in \N}\) ?
\end{exoex}

\begin{corr}
	\begin{itemize}

		\item \(B\) est minorée car \( \quantifs{\forall n \in \N} 0 \leq \frac{n}{n^2+1} \) par ailleurs \(0 \in B\) donc \(0\) est un minimum. \\
		\item \(B\) est majorée par \(\frac{1}{2}\). En effent en notant \(U_n = \frac{n}{n^2+1}\), On voit que \((U_n)\) est strictement décroissante
	\end{itemize}
\end{corr}

\begin{exoex}
        ~\\
	Que dire de \(C = \accol{\frac{\e{x}}{x} \tq x \in \Rps}\) ?
\end{exoex}

\begin{corr}
	\begin{itemize}

		\item \(C\) est minorée car \( \quantifs{\forall x \in \Rps} 0 \leq \frac{\e{x}}{x} \) donc \(0\) est un minorant mais pas un minimum  \\
		\item Supposons que \(C\) est majorée alors \(\quantifs{\exists M \in \R;\forall c \in C} c\leq M \) ainsi \(\quantifs{\forall x \in \Rps} \frac{\e{x}}{x} \leq M \) donc par passage à la limite en \(\pinf\) on trouve \(\pinf \leq M\) ce qui est absurde donc \(C\) n'est pas majorée.
	\end{itemize}
\end{corr}

\begin{defi}[Parties bornées]
	Une partie \(A\) de \(\R\) est dite bornée si elle est majorée et minorée autrement dit s’il existe deux réels \(m\) et \(M\) tel que, pour tout réel \(x\) de \(A\), on a : \(m\leq x \leq M\).
\end{defi}

\section{Valeur absolue d'un réel}
\begin{defi}
	Pour tout \(x\) réel, la valeur absolue de \(x\), notée \(\abs{x}\), est définie par : \(\abs{x} = \begin{cases}
		-x & \text{si }  x < 0   \\
		x  & \text{si }  x\geq 0 \\
	\end{cases}\)
\end{defi}

\begin{prop}
	\begin{enumerate}
		\item Pour tout \(x\) réel, on a : \(0\leq\abs{x}\) et \(x\leq\abs{x}\)
		\item Pour tout couple\((x,y)\) de réels, on a : \(\abs{xy} = \abs{x}\abs{y}\)
		\item Pour tout couple \((x,y)\) de réels tel que \(y\) est non nul, on a: \(\abs{\frac{x}{y}} = \frac{\abs{x}}{\abs{y}}\)
	\end{enumerate}
\end{prop}

\begin{defprop}[Deux inéquations élémentaires]
	Pour tout réel \(x\) et tout \underline{réel positif} \(\alpha\), on a:
	\begin{enumerate}
		\item \(\abs{x}\leq \alpha \iff -\alpha \leq x \leq \alpha \iff x \in \intervii{-\alpha}{\alpha}\)
		\item \(\abs{x}\geq \alpha \iff x \leq -\alpha\text{ ou } \alpha \leq x \iff x \in \intervei{\pinf}{-\alpha}\union\intervie{\alpha}{\pinf}\)
	\end{enumerate}
\end{defprop}

\begin{defprop}[Interprétation sur la droite des réels]
	Soit \(a\) un réel et \(b\) un \underline{réel positif}. \\
	L’ensemble des réels \(x\) vérifiant \(\abs{x-a}\leq b\) (resp. \(\abs{ x-a}\geq b \)) est l’ensemble des points de la droite des
	réels situés à une distance du point \(a\) inférieure ou égale (resp. supérieure ou égale) à \(b\).
\end{defprop}

\begin{prop}[Inégalité triangulaire]
	Pour tout couple \((x,y)\) de réels, on a :
	\[\abs{x+y}\leq \abs{x}+\abs{y}\]
\end{prop}

\begin{dem} [inégalité triangulaire]
	Soit \((x,y) \in \R^2\)
	\begin{align*}
		\abs{x+y}\leq \abs{x}+\abs{y} & \iff \abs{x+y}^2\leq (\abs{x}+\abs{y})^2      \\
		                              & \iff x^2+2xy+y^2 \leq x^2+y^2+2\abs{x}\abs{y} \\
		                              & \iff xy\leq \abs{xy}
	\end{align*}
	Ce qui est vrai donc l'inégalité est bien démontrer
\end{dem}

\begin{exoex}   
     ~\\
	Encadrer \(\frac{x\cos(x)+1}{\sin(x)+3}\) pour \(x\in\intervii{-\pi}{2\pi}\)
\end{exoex}

\begin{corr}
	Soit \(x\in\intervii{-\pi}{2\pi}\)
	\begin{itemize}

		\item \underline{numérateur} : \(\abs{x\cos(x)+1}\leq \abs{x}\abs{\cos(x)}+1\leq 2\pi+1 = 2\pi+1\)
		\item \underline{dénominateur} : \(2\leq\abs{\sin(x)+3}\leq 4\)

	\end{itemize}
	Ainsi par produit des deux inégalités on as :\(0\leq\frac{\abs{x\cos(x)+1}}{\abs{\sin(x)+3}}\leq \frac{2\pi+1}{2} \)\\
	donc \(-\frac{2\pi+1}{2} \leq \frac{x\cos(x)+1}{\sin(x)+3} \leq \frac{2\pi+1}{2}\) pour \(x\in\intervii{-\pi}{2\pi}\).
\end{corr}

\begin{prop}
	Soit un couple \((x,y)\) de réels.
	\[\abs{\abs{x}-\abs{y}} \leq\abs{x-y}\]
\end{prop}

\begin{dem}
	Soit \((x,y) \in \R^2\)
	\(x =(x-y)+y\) donc \(\abs{x} \underset{\mathrm{\text{inég. triang.}}}{\leq} \abs{x-y}+\abs{y}\) d'où \(\abs{x} - \abs{y} \leq \abs{x-y}\) \\
	De même, \(y =(x-y)+x\) donc \(\abs{y} \underset{\mathrm{\text{inég. triang.}}}{\leq} \abs{x-y}+\abs{x}\) d'où \( -\abs{x-y} \leq\abs{x} - \abs{y}\)\\

	ainsi on a \(-\abs{x-y} \leq\abs{x} - \abs{y} \leq \abs{x-y}\) donc \(\abs{\abs{x}-\abs{y}} \leq\abs{x-y}\).
\end{dem}

\section{Partie entière d'un réel}
\begin{prop}
	Pour tout réel \(x\),il existe un unique entier \(n\) tel que :
	\[n\leq x < n+1\]
\end{prop}
\begin{defi}
	On appelle partie entière de \(x\), notée \(\lfloor x \rfloor\), l'unique entier \(n\) vérifiant la propriété précédente.
\end{defi}

\begin{ex}
	\(\lfloor 3.14 \rfloor = 3\), \(\lfloor -2.7 \rfloor = -3\) et \(\lfloor 5 \rfloor = 5\).
\end{ex}

\section{Généralité sur les fonctions}
\begin{defi} [Fonction]
	Une fonction de variable réelle à valeurs réelles notée \(f\) est un objet mathématique qui, à tout élément \(x\) d’une partie non vide de \(\R\), associe un et un seul nombre réel noté \(f(x)\). \\
	\underline{Notation Fonctionnelle} : \[\fonction{f}{A}{\R}{x}{f(x)}\]
\end{defi}

\begin{defi}
	Soit \(f\) une fonction de variable réelle à valeurs réelles.
	\begin{enumerate}
		\item L’ensemble des réels \(x\) pour lesquels \(f(x)\) existe est appelé ensemble/domaine de définition de \(f\) et souvent noté \(D_f = \accol{x \in \R \tq f(x) \text{existe}}\)
		\item Soit \(x \in D_f\)\\
		      La valeur réelle \(f(x)\) est appelée image de \(x\) par \(f\). \\
		\item soit \( y \in \R\) \\
		      S'il existe \(x\) dans \(D_f\) tel que \(f(x) = y\) alors \(x\) est dit antécédent de \(y\) par \(f\)
	\end{enumerate}
\end{defi}

\begin{defprop}[égalité entre fonction]
	Deux fonctions \(f\) et \(g\) de variable réelle à valeurs réelles sont dites égales si les deux conditions suivantes sont réunies :
	\begin{itemize}
		\item les fonctions \(f\) et \(g\) ont le même ensemble de définition \(D\) ;
		\item pour tout \(x\) de \(D\), \(f(x) = g(x)\).
	\end{itemize}
	dans ce cas, on note \(f = g\).
\end{defprop}

\begin{exoex}
	est-ce que les fonctions \(f\) et \(g\) définies par :
	\[f: x\mapsto\frac{1}{\sqrt{1+x}+1} \text{ et } g:  x\mapsto\frac{\sqrt{1+x}-1}{x}\]
	Sont égales ?
\end{exoex}
\begin{corr}
	Tout d'abord \(\quantifs{\forall x \in D_f\inter D_g} f(x) = g(x)\) car :
	\begin{align*} g(x) & = \frac{\sqrt{1+x}-1}{x}                                                 \\
                    & = \frac{\paren{\sqrt{1+x}-1}\paren{\sqrt{1+x}+1}}{x\paren{\sqrt{1+x}+1}} \\
                    & = \frac{1+x-1}{x\paren{\sqrt{1+x}+1}}                                    \\
                    & = \frac{x}{x\paren{\sqrt{1+x}+1}}                                        \\
                    & = \frac{1}{\sqrt{1+x}+1} = f(x)
	\end{align*}
	Donc \(f = g\) sur \(D_f\inter D_g\) mais
	\(D_f = \intervei{-1}{\pinf}\) or \(D_g = \intervie{-1}{\pinf}\pd\accol{0}\) donc \(D_f \neq D_g\) donc \(f \neq g\).
\end{corr}

\begin{defi}[représentation graphique d'une fonction]
	Dans le plan muni d’un repère orthonormé \((O, \vec{i}, \vec{j})\), l’ensemble de points \(\mathcal{C}_f\) défini par
	\[
		\mathcal{C}_f = \accol{ M(x; f(x)) \tq x \in D_f }
	\]
	est appelé représentation graphique de \(f\) (ou courbe représentative de \(f\)).
\end{defi}

\begin{defi}[Parité,imparité et périodicité d'une fonction]
	\begin{itemize}
		\item Une fonction \(f\) est dite paire si, pour tout \(x\) de son domaine de définition, on a : \(f(-x) = f(x)\).
		\item Une fonction \(f\) est dite impaire si, pour tout \(x\) de son domaine de définition, on a : \(f(-x) = -f(x)\).
		\item Une fonction \(f\) est dite périodique de période \(T\) si, pour tout \(x\) de son domaine de définition, on a : \(f(x+T) = f(x)\).
	\end{itemize}
\end{defi}
\begin{exo}
	Montrer que toute fonction de \(\R\) peut s'écrire de manière unique comme la somme d'une fonction paire et d'une fonction impaire.
\end{exo}

\begin{corr}[Analyse-synthèse]
	Soit \(f : \R \mapsto \R \) une fonction quelqu'onque
	\begin{itemize}
		\item \analyse :  Supposons qu'il existe \(\begin{cases}
			      p:\R \mapsto \R \text{ paire} \\
			      i:\R \mapsto \R \text{ impaire}
		      \end{cases}\) telles que \(f = p + i\) \\
		      Ainsi \(\forall x \in \R \begin{cases}
			      f(x) = p(x) + i(x) \hfill (1)                  \\
			      f(-x) = p(-x) + i(-x) = p(x) - i(x) \hfill (2) \\
		      \end{cases}\) \\
		      \begin{itemize}
			      \item\(\frac{1}{2}\paren{\text{(1)+(2)}}\) donne \(p:x\mapsto \frac{f(x)+f(-x)}{2}\) \\
			      \item \(\frac{1}{2}\paren{\text{(1)-(2)}}\) donne \(i:x\mapsto \frac{f(x)-f(-x)}{2}\) \\
		      \end{itemize}
		\item \synthese : vérifions que le seul couple trouvé convient :
		      \begin{itemize}
			      \item \(\forall x \in \R, f(x) = p(x)+i(x)\)\\
			      \item \(p(-x) = p(x) \text{ et } i(-x) = -i(x)\)\\
		      \end{itemize}
	\end{itemize}
	Ainsi \(f\) s'écrit de manière unique comme la somme d'une fonction paire et impaire
\end{corr}

\begin{defi} [opération et composition]
	Soit \(f\) et \(g\) deux fonctions de variable réelle à valeurs réelles de domaines de définition \(D_f\) et \(D_g\).
	\begin{itemize}
		\item La somme de \(f\) et \(g\) est la fonction, notée \(f + g\), définie par \(f + g : x \mapsto f(x) + g(x)\). \\
		      Son domaine de définition \(D_{f+g}\) vérifie : \(D_{f+g} = D_f \inter D_g\).
		\item  La multiplication de \(f\) par le réel \(\alpha\) est la fonction, notée \(\alpha f\), définie par \(\alpha f : x \mapsto \alpha f(x)\). \\
		      Son domaine de définition \(D_{\alpha f}\) vérifie : \(D_{\alpha f} = D_f\) si \(\alpha \neq 0\).
		\item Le produit de \(f\) et \(g\) est la fonction, notée \(f g\), définie par \(f g : x \mapsto f(x)g(x)\). \\
		      Son domaine de définition \(D_{fg}\) vérifie : \(D_{fg} = D_f \inter D_g\).
		\item Le quotient de \(f\) par \(g\) est la fonction , notée \(frac{f}{g}\), définie par \(frac{f}{g} : x \mapsto \frac{f(x)}{g(x)}\). \\
		      Son domaine de définition \(D_{frac{f}{g}}\) vérifie : \(D_{frac{f}{g}} = D_f \inter \accol{x \in D_g | g(x) \neq 0}\).
		\item La composée de \(g\) et \(f\) est la fonction, notée \(g \circ f\), définie par \(g \circ f : x \mapsto g(f(x))\). \\
		      Son domaine de définition \(D_{g \circ f}\) vérifie : \(D_{g \circ f} = \accol{x \in D_f | f(x) \in D_g}\).
	\end{itemize}
\end{defi}

\begin{exoex}
	Domaine de définition de : \(\fonction{f}{D_f}{\R}{x}{\sqrt{x-\frac{1}{x}}} \)
\end{exoex}

\begin{corr}
	Soit \(x \in D_f\) alors \(x-\frac{1}{x} \geq 0 \iff x\neq 0\) et \(\frac{x^2-1}{x} = \frac{(x-1)(x+1)}{x} \geq 0\)\\
	% Assurez-vous d'avoir \usepackage{tkz-tab} dans le préambule
	\begin{tikzpicture}
		\tkzTabInit{$x$/1,$(x-1)(x+1)$/1,$x$/1,$f$/1}{$-\infty$,$-1$,$0$,$1$,$+\infty$}
		\tkzTabLine{,+,0,-,t,-,0,+,}
		\tkzTabLine{,-,t,-,0,+,t,+,}
		\tkzTabLine{,-,0,+,d,-,0,+}


	\end{tikzpicture}\\
	ainsi on voit bien que \(D_f = \intervie{-1}{0}\union\intervie{1}{\pinf}\)
\end{corr}

\section{Fonction et relation d'ordre}
\begin{defi}[Monotonie]
	Soit \(f\) une fonction de variable réelle à valeurs réelles et \(D\) une partie de son domaine de définition \(D_f\).
	\begin{enumerate}
		\item \(f\) est dite \textbf{croissante} sur \(D\) si, pour tout \((x, y) \in D^2\) tel que \(x \leq y\), on a \(f(x) \leq f(y)\).
		\item \(f\) est dite \textbf{décroissante} sur \(D\) si, pour tout \((x, y) \in D^2\) tel que \(x \leq y\), on a \(f(x) \geq f(y)\).
		\item \(f\) est dite \textbf{strictement croissante} sur \(D\) si, pour tout \((x, y) \in D^2\) tel que \(x < y\), on a \(f(x) < f(y)\).
		\item \(f\) est dite \textbf{strictement décroissante} sur \(D\) si, pour tout \((x, y) \in D^2\) tel que \(x < y\), on a \(f(x) > f(y)\).
	\end{enumerate}
	\textbf{Remarque :} \(f\) est dite \textbf{monotone} (resp. \textbf{strictement monotone}) sur \(D\) si elle est croissante ou décroissante (resp. strictement croissante ou strictement décroissante) sur \(D\).
\end{defi}

\begin{rem}[Application de la définition]
	Sous réserve que cela ait du sens :
	\begin{itemize}
		\item La somme de deux fonctions croissantes(resp. décroissantes) est croissante(resp. décroissante).
		\item La composée de deux fonctions croissantes(resp. décroissantes) est croissante(resp. décroissante).
		\item La composée d'une fonction croissante et d'une fonction décroissante est décroissante
		\item Le produit de deux fonctions \underline{positives} croissantes (resp. décroissantes) est croissante(resp. décroissante).
	\end{itemize}
\end{rem}


\begin{defi}
	Soit \(f\) une fonction de variable réelle à valeurs réelles de domaine de définition \(D_f\). \\
	Soit \(D\) une partie non vide de \(D_f\).
	\begin{enumerate}
		\item \(f\) est dite \textbf{majorée} sur \(D\) si l'ensemble \(\accol{f(x) \tq x \in D}\) est majoré, c'est-à-dire s'il existe un réel \(M\) tel que, pour tout réel \(x\) de \(D\), on a : \(f(x) \leq M\).\\
		      Un tel réel \(M\) est alors dit :
		      \begin{itemize}
			      \item \textbf{majorant} de \(f\) sur \(D\) dans le cas général.
			      \item \textbf{maximum} de \(f\) sur \(D\) dans le cas particulier où il existe \(x_0\) dans \(D\) tel que \(M = f(x_0)\).
		      \end{itemize}
		\item \(f\) est dite \textbf{minoriée} sur \(D\) si l'ensemble \(\accol{f(x) \tq x \in D}\) est minoré, c'est-à-dire s'il existe un réel \(m\) tel que, pour tout réel \(x\) de \(D\), on a : \(m \leq f(x)\).\\
		      Un tel réel \(m\) est alors dit :
		      \begin{itemize}
			      \item \textbf{minorant} de \(f\) sur \(D\) dans le cas général.
			      \item \textbf{minimum} de \(f\) sur \(D\) dans le cas particulier où il existe \(x_0\) dans \(D\) tel que \(m = f(x_0)\).
		      \end{itemize}
		\item \(f\) est dite \textbf{bornée} sur \(D\) si \(f\) est majorée et minoriée sur \(D\), c'est-à-dire s'il existe deux réels \(m\) et \(M\) tels que, pour tout réel \(x\) de \(D\), on a : \(m \leq f(x) \leq M\).\end{enumerate}
\end{defi}

\begin{prop}
	Soit \(f\) une fonction de variable réelle à valeurs réelles de domaine de définition \(D_f\). \\
	Alors \(f\) est bornée sur \(D\) si, et seulement si, la fonction \(\abs{f}\) est majorée sur \(D\).
\end{prop}
\section{Dérivation des fonctions d'une variable réelle}

\begin{defi}[dérivée en un point]
	Soit \(f\) une fonction de variable réelle à valeurs réelles de domaine de définition \(D_f\) et \(x_0\) un point de \(D_f\). \\
	\(f\) est dite dérivable en \(x_0\) si la fonction \(x\mapsto \frac{f(x)-f(x_0)}{x-x_0}\) admet une limite finie en \(x_0\). \\
	Dans ce cas, on note \(f'(x_0)\) la valeur de cette limite et on l'appelle la dérivée de \(f\) en \(x_0\). \\
	Cela reient à déterminer si la fonction \(h\mapsto \frac{f(x_0+h)-f(x_0)}{h}\) admet une limite finie en \(0\).\\
\end{defi}

\begin{defi}{fonction dérivée}
	\(f\) est dite dérivable sur \(D_f\) si elle est dérivable en tout point de \(D_f\). \\
	Dans ce cas, la fonction \(x\mapsto f'(x)\) est appelée fonction dérivée de \(f\) et notée \(f'\). \\
\end{defi}

\begin{defprop}[équation de la tangente]
	On se place dans le plan muni d’un repère orthonormé \((O, \vec{i}, \vec{j})\). \\
	Soit \(f\) une fonction de variable réelle à valeurs réelles et \(C_f\) la courbe représentative de \(f\). \\
	Soit \(x_0\) un point de \(D_f\) .\\
	Si \(f\) est dérivable en \(x_0\), alors la tangente à la courbe \(C_f\) au point \(M(x_0, f(x_0))\) est la droite d’équation :
	\[y = f'(x_0)(x-x_0) + f(x_0)\]
\end{defprop}

\begin{defprop}[opération sur les fonctions dérivable]
	Soit \(I\) et \(J\) des intervalles de \(\R\) non vide et non réduits à un point. \\
	\begin{enumerate}
		\item \underline{Combinaison linéaire} : \\
		      Soit \(f\) et \(g\) deux fonctions définies sur \(I\) et à valeurs réelles et \((\alpha, \beta)\) deux réels. \\
		      Si \(f\) et \(g\) sont dérivables sur \(I\), alors \(\alpha f + \beta g\) est dérivable sur \(I\) et sa dérivée vérifie :
		      \[\alpha f + \beta g' = \alpha f' + \beta g'\]
		\item \underline{Produit} : \\
		      Soit \(f\) et \(g\) deux fonctions définies sur \(I\) et à valeurs réelles. \\
		      Si \(f\) et \(g\) sont dérivables sur \(I\), alors \(f g\) est dérivable sur \(I\) et sa dérivée vérifie :
		      \[(f g)' =f'g+fg'\]
		      \item\underline{quotient} :\\
		      Soit \(f\) et \(g\) deux fonctions définies sur \(I\) et à valeurs réelles tel que \(g\) est non nulle sur \(I\). \\
		      Si \(f\) et \(g\) sont dérivables sur \(I\), alors \(\frac{f}{g}\) est dérivable et sa dérivée vérifie :
		      \[\paren{\frac{f}{g}}' = \frac{f'g-fg'}{g^2}\]
		\item \underline{Composition} : \\
		      Soit \(f\) une fonction définie sur \(I\) et à valeurs réelle tel que, pour tout \(x\) de \(I\), \(f(x)\) appartient à \(J\)\\
		      Soit \(g\) une fonction définie sur \(J\) et à valeurs réelles. \\
		      Si \(f\) est dérivable sur \(I\) et \(g\) dérivable sur \(J\), alors la composée \(g \circ f\) est dérivable sur \(I\) et sa dérivée vérifie :
		      \[\paren{g \circ f}' = g' \circ f \times f'\]
	\end{enumerate}
\end{defprop}

\begin{defprop}[Caractérisation des fonctions constantes ou monotones]
	Soit \(f\) une fonction définie sur un intervalle \(I\) et à valeurs réelles. \\
	\begin{enumerate}
		\item \(f\) est constante sur \(I\) si, et seulement si, pour tout \(x\) de \(I\),\(f'(x)=0\).
		\item \(f\) est croissante sur \(I\) si, et seulement si, pour tout \(x\) de \(I\), \(f'(x) \geq 0\).
		\item \(f\) est décroissante sur \(I\) si, et seulement si, pour tout \(x\) de \(I\), \(f'(x) \leq 0\).
		\item \(f\) est strictement croissante sur \(I\) si, et seulement si, les deux conditions suivante sont réunies :
		      \begin{enumerate}
			      \item pour tout \(x\) de \(I\), \(f'(x) \geq 0\) ;
			      \item il n'existe pas de réels \(a\) et \(b\) dans \(I\) avec \(a < b\) tels que pour tout \(x\) de \(\intervii{a}{b}\), on a \(f'(x) = 0\).
		      \end{enumerate}
		\item \(f\) est strictement décroissante sur \(I\) si, et seulement si, les deux conditions suivante sont réunies :
		      \begin{enumerate}
			      \item pour tout \(x\) de \(I\), \(f'(x) \leq 0\) ;
			      \item il n'existe pas de réels \(a\) et \(b\) dans \(I\) avec \(a < b\) tels que pour tout \(x\) de \(\intervii{a}{b}\), on a \(f'(x) = 0\).
		      \end{enumerate}
	\end{enumerate}
\end{defprop}

\begin{defprop}[dérivées usuelles]
    ~\\
	\begin{tabular}{|c|c|c|}

		\hline
		\textbf{Fonction}                    & \textbf{Domaine de dérivabilitée}                 & \textbf{Fonction dérivée}                                             \\
		\hline
		\(x\mapsto a\) avec \(a \in \R\)     & \(\R\)                                            & \(x\mapsto 0\)                                                        \\
		\hline
		\(x\mapsto x^n\) avec \(n \in \Ns\)  & \(\R\)                                            & \(x\mapsto nx^{n-1}\)                                                 \\
		\hline
		\(x\mapsto x^-n\) avec \(n \in \Ns\) & \(\Rs\)                                           & \(x\mapsto -nx^{-n-1}\)                                               \\
		\hline
		\(x\mapsto \sqrt{x}\)                & \(\Rps\)                                          & \(x\mapsto \frac{1}{2\sqrt{x}}\)                                      \\
		\hline
		\(x\mapsto \e{x}\)                   & \(\R\)                                            & \(x\mapsto \e{x}\)                                                    \\
		\hline
		\(x\mapsto \ln(x)\)                  & \(\Rps\)                                          & \(x\mapsto \frac{1}{x}\)                                              \\
		\hline
		\(x\mapsto \sin(x)\)                 & \(\R\)                                            & \(x\mapsto \cos(x)\)                                                  \\
		\hline
		\(x\mapsto \cos(x)\)                 & \(\R\)                                            & \(x\mapsto -\sin(x)\)                                                 \\
		\hline
		\(x\mapsto \tan(x)\)                 & \(\R\pd\accol{\frac{\pi}{2}+2k\pi \tq k \in \Z}\) & \(x\mapsto \frac{1}{\cos^2(x)} \) ou \(x\mapsto \frac{1}{\cos^2(x)}\) \\
		\hline
	\end{tabular}



\end{defprop}

\begin{exoex}
    ~\\
	Calculer\(\int_{\frac{\pi}{4}}^{\frac{\pi}{3}} \frac{\sin^3(x)}{\cos^5(x)} dx\)
\end{exoex}

\begin{corr}
	\begin{align*}
		\int_{\frac{\pi}{4}}^{\frac{\pi}{3}} \frac{\sin^3(x)}{\cos^5(x)} dx & = \int_{\frac{\pi}{4}}^{\frac{\pi}{3}} \tan^3(x) \times \frac{1}{\cos^2(x)} dx \\
		                                                                    & = \int_{\frac{\pi}{4}}^{\frac{\pi}{3}} \tan^3(x) \times \paren{\tan^2(x)+1} dx \\
		                                                                    & = \croch{\frac{1}{4}\paren{\tan^4(x) }}_{\frac{\pi}{4}}^{\frac{\pi}{3}}        \\
		                                                                    & = \frac{1}{4}\paren{\tan^4\paren{\frac{\pi}{3}} - \tan^4\paren{\frac{\pi}{4}}} \\
		                                                                    & = \frac{1}{4}\paren{\paren{\sqrt{3}}^4 - 1^4}                                  \\
		                                                                    & = 2                                                                            \\
	\end{align*}
\end{corr}

\begin{defprop}[étude pratique d'une fonction]
	Le plan d'étude d'une fonction \(f\) est en général le suivant:
	\begin{itemize}
		\item Détermination du domaine de définition de \(f\)
		\item Réduction éventuelles du domaine d'étude selon les propriétés de \(f\) (parité, périodicité, etc.)
		\item Limites aux bornes du domaine d'étude
		\item Etude de la monotonie (le plus souvent,mais pas uniquement, après calcul de la dérivée de \(f\) et détermination du signe de celle-ci )
		\item Construction du tableau de variation de \(f\)(limites aux bornes, valeurs remarquables, variations)
		\item Tracé de la courbe représentative de \(f\)
	\end{itemize}
\end{defprop}
\begin{defprop}[dérivées d'odre supériéur]
	Soit \(f\) une fonction définie sur un intervalle \(I\) et à valeurs réelles. \\
	On note
	\[f^{(0)} = f\]
	puis, pour tout entier naturel \(k\) tel que la fonction\(f^{(k)}\) existe et est déribable sur \(I\), on pose :
	\[f^{(k+1)} = \paren{f^{(k)}}'\]
	Si \(n\) est un entier naturel, tel que la fonction \(f^{(n)}\) existe alors on dit que \(f\) est \(n\)-fois dérivable sur \(I\) et que \(f^{(n)}\) est la dérivée d'ordre \(n\) (ou dérivée \(n\)-ième) de \(f\).\\

\end{defprop}
\begin{defi}[Fonction réciproque]
	Soit \(f\) une fonction définie sur un intervalle \(I\) à valeurs dans \(J\)
	Si, pour tout y de \(J\), l’équation \(y = f(x)\) admet une unique solution \(x\) dans \(I\) notée \(x = f^{-1}(y)\) alors :
	\begin{itemize}
		\item la fonction \(f\) est dite bijection de \(I\) sur \(J\)
		\item la fonction \(f^{-1}\) ainsi définie sur \(J\) et à valeurs dans \(I\), est dite bijection réciproque de \(f\).
	\end{itemize}
	\underline{Exemples}:
	\begin{itemize}
		\item \(\sqrt{}\) est une bijection de \(\Rp\) sur \(\Rp\) de bijection réciproque \(f : \Rp \to \Rp\) définie par \(f(x) = x^2\).
		\item \(\exp\) est une bijection de \(\R\) sur \(\Rps\) de bijection réciproque la fonction \(\ln\)
	\end{itemize}
\end{defi}
\begin{prop}[Propriétés de la bijection réciproque]
	Si \(f\) est une bijection de \(I\) sur \(J\) de bijection réciproque notée \(f^{-1}\) alors on a :
	\begin{enumerate}
		\item pour tout \(x\) de \(I\), \(f(f^{-1}(x)) = x\) ;
		\item pour tout \(y\) de \(J\), \(f^{-1}(f(y)) = y\).
	\end{enumerate}

\end{prop}

\begin{defprop}[représentation graphique]
	on se place dans le plan muni d’un repère orthonormé \((O, \vec{i}, \vec{j})\). \\
	Si \(f\) est une bijection de \(I\) sur \(J\) alors la courbe représentative de \(f\) et de sa bijection réciproque \(f^{-1}\) sont symétriques par rapport à la droite d’équation \(y = x\).
\end{defprop}

\begin{defprop}[dérivée de la bijection réciproque]
	Soit \(f\) une bijection de \(I\) sur \(J\)  et si \(f\) est dérivable sur \(I\) alors sa bijection réciproque \(f^{-1}\) est dérivable en tout point y de \(J\) tel que \(f'(f^{-1}(y)) \neq 0\) avec, dasn ce cas : \[(f^{-1})'(y) = \frac{1}{f'(f^{-1}(y))}\]
\end{defprop}
\begin{dem}
	Soit \(f\) une bijection de \(I\) sur \(J\), soit \(y\) in \(J\) tel que \(f'(f^{-1}(y)) \neq 0\). \\
	on sait que \(f(f^{-1}(y)) = y\) donc en appliquant la définition de la dérivée de fonction composée on a :
	\[(f(f^{-1}(y)))' = (y)' \iff f'(f^{-1}(y))\times (f^{-1}(y))' = 1 \iff (f^{-1}(y))' = \frac{1}{f'(f^{-1}(y))}\]
\end{dem}
\begin{defprop}[Trois fonction usuelles trigonométriques]
	\begin{itemize}
		\item \underline{Fonction \(\Arccos\)} : \\
		      La fonction \(\Arccos\) est la réciproque de la fonction \(\fonction{c}{\intervii{0}{\pi}}{\intervii{-1}{1}}{x}{\cos(x)}\) et est donc définie sur \(\intervii{-1}{1}\) à valeurs dans \(\intervii{0}{\pi}\) et dérivable sur \(\intervee{-1}{1}\) de dérivée: \[\arccos':x\mapsto \frac{-1}{\sqrt{1-x^2}}\]\\
		\item \underline{Fonction \(\Arcsin\)} : \\
		      La fonction \(\Arccos\) est la réciproque de la fonction \(\fonctionlambda{\intervii{-\frac{\pi}{2}}{\frac{\pi}{2}}}{\intervii{-1}{1}}{x}{\sin(x)}\) et est donc définie sur \(\intervii{-1}{1}\) à valeurs dans \(\intervii{-\frac{\pi}{2}}{\frac{\pi}{2}}\) et dérivable sur \(\intervee{-1}{1}\) de dérivée: \[\arcsin':x\mapsto \frac{1}{\sqrt{1-x^2}}\]\\
		\item \underline{Fonction \(\Arctan\)} : \\
		      La fonction \(\Arccos\) est la réciproque de la fonction \(\fonctionlambda{\intervee{-\frac{\pi}{2}}{\frac{\pi}{2}}}{\R}{x}{\tan(x)}\) et est donc définie sur \(\R\) à valeurs dans \(\intervee{-\frac{\pi}{2}}{\frac{\pi}{2}}\) et dérivable sur \(\R\) de dérivée: \[\arctan':x\mapsto \frac{1}{1+x^2}\]\\
	\end{itemize}
\end{defprop}


\begin{dem}[démonstration de la dérivée de la fonction \(\Arccos\)]
	Soit \(y\in\intervii{-1}{1}\), on note \(\fonction{c}{\intervii{0}{\pi}}{\intervii{-1}{1}}{x}{\cos(x)}\)
	\begin{align*}
		c'(c^{-1}(y)) & = -\sin(c^{-1}(y))                                                                                                \\
		              & = -\sqrt{\sin^2(c^{-1}(y))} \qquad  \text{car }c^{-1}(y)\in \intervii{0}{\pi} \text{ donc } \sin(c^{-1}(y))\geq 0 \\
		              & = -\sqrt{1-\cos^2(c^{-1}(y))}                                                                                     \\
		              & = -\sqrt{1-y^2}                                                                                                   \\
	\end{align*}
	Ainsi d'après la définition de la dérivée de la bijection réciproque on a : \(\Arccos'(y) = \frac{-1}{\sqrt{1-y^2}}\)
\end{dem}

\begin{rem}[démonstration d'une relation intéressante entre \(\Arctan(x)\) et \(\Arctan\paren{{\frac{1}{x}}}\)]
	Soit \(f:x\mapsto \Arctan{\paren{\frac{1}{x}}}\), on as \(D_f = \R\pd\accol{0}\) et \(f\) dérivable sur \(D_f\)
	\begin{align*}
		f'(x) & = \Arctan'\paren{\frac{1}{x}} \times \paren{\frac{1}{x}}'         \\
		      & = \frac{1}{1+\paren{\frac{1}{x}}^2} \times \paren{\frac{-1}{x^2}} \\
		      & = \frac{-1}{x^2+1}
	\end{align*}
	On remarque que \(\quantifs{\forall x \in \Rs}f'(x) = -\Arctan'\paren{x}\) ainsi \(\quantifs{\forall x \in \Rps}f'(x) +\Arctan'\paren{x} = 0\) donc \(\quantifs{\forall x \in \Rs}\paren{f(x)+\Arctan\paren{x}}'=0\) \\
	Ainsi il existe \(c\) un réel tel que \(\quantifs{\forall x \in \Rps}f(x) + \Arctan\paren{x} = c\)
	\begin{align*}
		\text{Pour } x = 1, f(1) + \Arctan(1) & = c \\
		f(1) + \frac{\pi}{4}                  & = c \\
		c = \frac{\pi}{2}
	\end{align*}
	Ainsi \(\quantifs{\forall x \in \Rps} \Arctan\paren{\frac{1}{x}} + \Arctan\paren{x} = \frac{\pi}{2}\) \\
	De manière analogue on trouve  \(\quantifs{\forall x \in \Rms} \Arctan\paren{\frac{1}{x}} + \Arctan\paren{x} = -\frac{\pi}{2}\) \\
\end{rem}

\chapter{Calcul algébrique (rappels et compléments)}

\minitoc
\section{Sommes et produit finis}
\begin{nota}
	Soit \(\paren{a_i}_{i\in I}\) une famille de réels indexée par un ensemble \(I\) fini. \\
	La somme (resp. le produit) de tous les réels de la famille est notée \(\sum_{i\in I} a_i\) (resp. \(\prod_{i\in I} a_i\)). \\
	\begin{itemize}
		\item Si \(I\) est l'ensemble vide, on convient que : \(\sum_{i\in I} a_i = 0\) et \(\prod_{i\in I} a_i = 1\).
		\item Si \(I = \accol{1,2,\ldots,n}\) avec \(n\) un entier naturel non nul, on note \(\sum_{i=1}^n a_i\)  ou \(\sum_{1\leq i \leq n} a_i\) au lieu de \(\sum_{i\in I} a_i\) (resp. \(prod_{i=1}^n a_i\) ou \(\prod_{1\leq i \leq n} a_i\) au lieu de \(\prod_{i\in I} a_i\)).
	\end{itemize}
\end{nota}
\begin{prop}[opération et calcul par paquets]
	\begin{itemize}
		\item Pour toutes familles \(\paren{a_i}_{i\in I}\) et \(\paren{b_i}_{i\in I}\) de réels indexées par \(I\) et pour tout couple\((\alpha,\beta)\) de réels, on a :
		      \[ \sum_{i\in I} \paren{\alpha a_i + \beta b_i} = \alpha \sum_{i\in I} a_i + \beta \sum_{i\in I} b_i \qquad \text{ et } \qquad \prod_{i\in I} \paren{a_i b_i} = \paren{\prod_{i\in I} a_i}\paren{\prod_{b_i}}\]
		\item Pour toute famille \(\paren{a_i}_{i\in I}\) de réels indexée par \(I\) avec \(I=I_1 \union I_2\) et \(I_1\cap I_2 = \emptyset\), on a :
		      \[ \sum_{i\in I} a_i = \sum_{i\in I_1} a_i + \sum_{i\in I_2} a_i \qquad \text{ et } \qquad  \prod_{i\in I} a_i = \prod_{i\in I_1} a_i \prod_{i\in I_2} a_i \]
	\end{itemize}
\end{prop}

\begin{exoex}
	~\\
	Calculer : \(\sum_{k=1}^{2n} (-1)^k k \) avec \(n \in \N\)
\end{exoex}

\begin{corr}
	\begin{align*}
		\sum_{k=1}^{2n} (-1)^k k & = \sum_{k=0}^{n-1} (-1)^{2k+1} (2k-+) + \sum_{k=1}^{n} (-1)^{2k} (2k) \\
		                         & = -\sum_{k=0}^{n-1} (2k+1) + \sum_{k=1}^{n} 2k                        \\
		                         & = -\paren{2\sum_{k=0}^{n-1} k + n} + 2\sum_{k=1}^{n} k                \\
		                         & = -\paren{2\frac{(n-1)n}{2} + n} + 2\frac{n(n+1)}{2}                  \\
		                         & = n\paren{n+1-n+1-1}                                                  \\
		                         & = n                                                                   \\
	\end{align*}
\end{corr}
\begin{defprop}[téléscopage]
	Soit \(\paren{b_i}_{1 \leq i \leq n}\) une famille \underline{finie} de réels avec \(n\) supérieur ou égal à \(2\).
	\begin{enumerate}
		\item La somme \(\sum_{i=1}^n b_{i+1}-b_i\) est dire somme télescopique et vaut \(b_{n+1}-b_1\).
		\item Si tous les \(b_i\) sont non nuls, le produit \(\prod_{i=1}^n \frac{b_{i+1}}{b_i}\) est dit produit télescopique et vaut \(\frac{b_{n+1}}{b_1}\).
	\end{enumerate}
\end{defprop}

\begin{defprop}[Somme usuelles]
	Pour tout entier naturel \(n\) et tout réel \(x\) différent de \(1\), on a :
	\[\sum_{k=0}^n k = \frac{n(n+1)}{2} \qquad \sum_{k=0}^n k^2 = \frac{n(n+1)(2n+1)}{6} \qquad \sum_{k=0}^n x^k = \frac{x^{n+1}-1}{x-1}\]
\end{defprop}

\begin{defprop}[Factorisation de \(a^n-b^n\) ]
	Pour tout \(n\) entier naturel non nul et tout couple \((a,b)\) de réels, on a :
	\begin{align*}
		a^n-b^n & = (a-b)\paren{a^{n-1} + a^{n-2}b + \ldots + ab^{n-2} + b^{n-1}} \\
		        & = (a-b)\sum_{k=0}^{n-1} a^{n-1-k}b^k                            \\
		        & = (a-b)\sum_{k=0}^{n-1} a^kb^{n-1-k}                            \\
	\end{align*}
\end{defprop}

\begin{defprop}[coefficients binomiaux]
	Soit \(n\) un entier naturel non et \(k\) entière relatif, on a:
	\begin{enumerate}
		\item \(\binom{n}{k} = \binom{n}{n-k} \hfill \text{(symétrie)}\)
		\item \(binom{n}{k} +\binom{n}{k+1} = \binom{n+1}{k+1} \hfill \text{(relation de Pascal)}\)
		\item \(\binom{n}{k} \) est un entier naturel
	\end{enumerate}
\end{defprop}

\begin{defprop}[Formule du binôme de Newton]
	Pour tout couple \((a,b)\) de réels et tout entier naturel \(n\), on a :
	\[\paren{a+b}^n = \sum_{k=0}^n \binom{n}{k} a^{n-k}b^k = \sum_{k=0}^n \binom{n}{k} a^{k}b^{n-k}\]
\end{defprop}


\section{Cas des sommes doubles finies}
\begin{defi}
	Soit \(A\) un ensemble fini de couples et \((a_{i,j})_{(i,j)\in A}\) une famille de réels indexée par \(A\). La somme de tous les réels de la famille \((a_{i,j})_{(i,j)\in A}\) est notée \(\sum_{(i,j)\in A} a_{i,j}\) et appelée somme double. \\
	\underline{Remarque} : Si \(A\) est l'ensemble vide, on convient que \(\sum_{(i,j)\in A} a_{i,j} = 0\)
\end{defi}
\begin{defprop}[Sommes double rectangulaires]
	Dans le cas où \(A = \accol{1,2,\ldots,n}\times \accol{1,2,\ldots,m}\) avec \(n\) et \(m\) des entiers naturels non nuls,
	\begin{itemize}
		\item la somme double \(\sum_{(i,j)\in A} a_{i,j}\) est rectangulaire
		\item le somme double \(\sum_{(i,j)\in A} a_{i,j}\) s'écrit aussi \(\sum_{\substack{1 \leq i \leq n \\ 1 \leq j \leq m}} a_{i,j}\)
		\item la somme double \(\sum_{(i,j)\in A} a_{i,j}\) vaut  :
		      \[ sum_{(i,j)\in A} a_{i,j} = \sum_{\substack{1 \leq i \leq n \\ 1 \leq j \leq m}} a_{i,j} = \sum_{i=1}^n \paren{\sum_{j=1}^m a_{i,j}} = \sum_{j=1}^m \paren{\sum_{i=1}^n a_{i,j}} \]
		\item si \((b_i)_{1\leq i \leq n}\) et \((c_j)_{1\leq j \leq m}\) sont des familles finies de réels, alors : \[\paren{\sum_{i=1}^n b_i}\paren{\sum_{j=1}^m c_j} = \sum_{\substack{1 \leq i \leq n \\ 1 \leq j \leq m}} b_i c_j\]
	\end{itemize}
\end{defprop}

\begin{defprop}[somme double triangulaire]
	Dans le cas où \(A = \accol{(i,j) \in \N^2 | 1 \leq i \leq j \leq n}\) avec \(n\) un entier naturel non nul,
	\begin{itemize}
		\item La somme double \(\sum_{(i,j)\in A} a_{i,j}\) est dite triangulaire.
		\item La somme double \(\sum_{(i,j)\in A} a_{i,j}\) s'écrit aussi \(\sum_{1\leq i \leq j \leq n} a_{i,j}\) et vaut:
		      \[\sum_{(i,j)\in A} a_{i,i} = \sum_{1\leq i \leq j \leq n} a_{i,j} = \sum_{i=1}^n \paren{\sum_{j=i}^n a_{i,j}} = \sum_{j=1}^n \paren{\sum_{i=1}^j a_{i,j}}\]
	\end{itemize}
\end{defprop}


\section{Système linéaire de deux équations à deux inconnues}
\begin{defprop}[rappel de première]
	Dans le plan \(\R^2\) muni d’un repère orthonormé \((O,\vec{i},\vec{j})\), toute droite \(D\) admet une équation de la forme \[ax + by = c\]
	où \(a\), \(b\) et \(c\) sont des réels tels que \((a,b)\neq (0,0)\). \\
	Avec ces notations,
	\begin{itemize}
		\item le vecteur \(\vec{n}\) de coordonnées \((a,b)\) est un vecteur normal à \(D\) ;
		\item le vecteur \(\vec{u}\) de coordonnées \((-b,a)\) est un vecteur directeur de \(D\).
	\end{itemize}
\end{defprop}

\begin{defprop}[Système linéaire de deux équations à deux inconnues]
	Soit \(a\), \(b\), \(c\), \(a'\), \(b'\) et \(c'\) des réels. Le système d’équations
	\[
		(S) :
		\begin{cases}
			ax + by = c \\
			a'x + b'y = c'
		\end{cases}
	\]
	d’inconnues les réels \(x\) et \(y\) est dit système linéaire de deux équations à deux inconnues.
\end{defprop}

\begin{defprop}[Interprétation géométrique]
	Dans le cas où \((a,b)\neq (0,0)\) et \((a',b')\neq (0,0)\), résoudre le système \((S)\) revient à  déterminer l’intersection entre deux droites \(D\) et \(D'\) du plan.
	Trois cas se présentent :
	\begin{itemize}
		\item Les droites sont confondues donc \((S)\) a une infinité de solutions qui forment une droite ;
		\item Les droites sont sécantes donc \((S)\) a une unique solution ;
		\item Les droites sont parallèles non confondues donc \((S)\) n’a pas de solutions.
	\end{itemize}
\end{defprop}


\section{Système linéaire de trois équations à trois inconnues}

\begin{defprop}[rappel de terminale]
	Dans l'espace \(\R^3\) muni d’un repère orthonormé \((O,\vec{i},\vec{j},\vec{k})\), tout plan \(P\) admet une équation de la forme
	\[ax + by + cz = d\]
	où \(a\), \(b\), \(c\) et \(d\) sont des réels tels que \((a,b,c)\neq (0,0,0)\)
	\begin{itemize}
		\item le vecteur \(\vec{n}\) de coordonnées \((a,b,c)\) est un vecteur normal à \(P\) ;
		\item deux vecteurs non colinéaires pris parmi les vecteurs de coordonnées \((-b,a,0)\), \((0,-c,b)\) et \((-c,0,a)\) donnent la direction de \(P\).
	\end{itemize}
\end{defprop}

\begin{defprop}[Système linéaire de deux équations à trois inconnues]
	Soit \(a\), \(b\), \(c\), \(d\), \(a'\), \(b'\), \(c'\) et \(d'\) des réels. Le système d’équations
	\[
		(S) :
		\begin{cases}
			ax + by + cz = d \\
			a'x + b'y + c'z = d'
		\end{cases}
	\]
	d’inconnues les réels \(x\), \(y\) et \(z\) est dit système linéaire de deux équations à trois inconnues.
\end{defprop}

\begin{defprop}[Interprétation géométrique]
	Dans le cas où \((a,b,c)\neq (0,0,0)\) et \((a',b',c')\neq (0,0,0)\), résoudre le système \((S)\) revient à déterminer l’intersection entre deux plans \(P\) et \(P'\) de l’espace.
	Trois cas se présentent :
	\begin{itemize}
		\item Les plans sont confondus donc \((S)\) a une infinité de solutions qui forment un plan ;
		\item Les plans sont sécants donc \((S)\) a une infinité de solutions qui forment une droite ;
		\item Les plans sont parallèles non confondus donc \((S)\) n’a pas de solutions.
	\end{itemize}
\end{defprop}
\begin{defprop}[Système linéaire de trois équations à trois inconnues]
	Soit \(a\), \(b\), \(c\), \(d\), \(a'\), \(b'\), \(c'\), \(d'\), \(a''\), \(b''\), \(c''\) et \(d''\) des réels. Le système d’équations
	\[ (S) : \begin{cases}
			ax + by + cz = d     \\
			a'x + b'y + c'z = d' \\
			a''x + b''y + c''z = d''
		\end{cases} \]
	d’inconnues les réels \(x\), \(y\) et \(z\) est dit système linéaire de trois équations à trois inconnues.
\end{defprop}

\begin{defprop}[Interprétation géométrique]
	Dans le cas où \((a,b,c)\neq (0,0,0)\), \((a',b',c')\neq (0,0,0)\) et \((a'',b'',c'')\neq (0,0,0)\), résoudre le système \((S)\) revient à déterminer l’intersection entre trois plans \(P\), \(P'\) et \(P''\) de l’espace.
	Cela conduit à distinguer huit cas de figures qui donnent quatre types d’ensemble-solution pour \((S)\) :
	\begin{itemize}
		\item Le système \((S)\) a une infinité de solutions qui forment un plan ;
		\item Le système \((S)\) a une infinité de solutions qui forment une droite ;
		\item Le système \((S)\) a une unique solution ;
		\item Le système \((S)\) n’a pas de solutions.
	\end{itemize}
\end{defprop}

\section{Algorithme du Pivot}
\begin{rem} [Remarque préliminaire]
	En cycle terminal, de petits systèmes linéaires ont été rencontrés et résolus dans des cas simples, le plus souvent par “substitution”. \\
	En MP2I, nous utiliserons en priorité la méthode de résolution par “pivot”. Plus efficace et élégante, cette technique sera reprise au semestre 2 dans le chapitre “Matrices” pour résoudre plus généralement des systèmes linéaires de \(n\) équations à \(p\) inconnues.
\end{rem}

\begin{defprop}[Opérations élémentaires]
	On reprend les notations des paragraphes III. et IV. et on note \(L_i\) la \(i\)-ème ligne du système \((S)\).\\
	On appelle opérations élémentaires sur les lignes du système linéaire \((S)\) :
	\begin{enumerate}
		\item l’échange de deux lignes distinctes : \(L_i \leftrightarrow L_j\) avec \(i\neq j\) ;
		\item la multiplication d'une ligne par un réel non nul : \(L_i \leftarrow \lambda L_i\) avec \(\lambda\neq 0\) ;
		\item l'addition à une ligne du produit d'une autre ligne par un réel non nul : \(L_i \leftarrow L_i + \lambda L_j\) avec \(i\neq j\) et \(\lambda\neq 0\).
	\end{enumerate}
\end{defprop}

\begin{prop}[Propriété importante]
	Toute opération élémentaire sur les lignes d’un système linéaire le transforme en un système linéaire équivalent c’est-à-dire un système ayant le même ensemble de solutions.
\end{prop}

\begin{defprop}[résolution d'un système linéaire par la méthode du pivot]
	La résolution d’un système linéaire par la méthode du pivot se déroule en deux phases :
	\begin{itemize}
		\item \underline{phase de descente} : en effectuant des opérations élémentaires sur les lignes du système, on transforme le système en un système de forme “triangulaire” ou “trapézoïdale” comme, par exemple,
		      \[(S1) : \begin{cases} a_1x+b_1y = c_1 \\ b'_1y = c'_1 \end{cases}\]
		      \[(S2) : \begin{cases} a_1x+b_1y+c_1z = d_1 \\ b'_1y+c'_1z = d'_1 \end{cases}\]
		      \[(S3) : \begin{cases} a_1x+b_1y+c_1z = d_1 \\ b'_1y+c'_1z = d'_1 \\ c''_1z = d''_1 \end{cases}\]
		\item \underline{phase de remontée} : Le système obtenu est équivalent au système initial ; il est facile à résoudre ce qui permet d’obtenir l’ensemble des solutions du système initial. Dans cette phase de remontée, on peut au choix :
		      \begin{itemize}
			      \item effectuer des substitutions successives (moins élégant) ;
			      \item utiliser à nouveau des opérations élémentaires sur les lignes pour réduire le système sous forme “diagonale” (plus élégant et facile à coder).
		      \end{itemize}
	\end{itemize}
\end{defprop}
\begin{rem}
	Les opérations élémentaires effectuées lors de la résolution d’un système linéaire par la méthode du pivot (phases de descente et de remontée) doivent systématiquement être indiquées en marge du système étudié pour faciliter la lecture des correcteurs et permettre de retrouver les éventuelles erreurs de calcul.
\end{rem}

\begin{rem}[Pour aller plus loin (pour ceux qui ont suivi l’option maths expertes)]
	\begin{itemize}
		\item Les petits systèmes linéaires décrits au III. et IV. peuvent se traduire matriciellement par une équation matricielle du type \(AX = B\) avec \(A\) et \(B\) des matrices à préciser et \(X\) une matrice colonne inconnue.
		\item L’effet des opérations élémentaires sur les lignes de ces systèmes peut se traduire matriciellement par des multiplications de la matrice \(A\) à gauche par des matrices inversibles bien
	\end{itemize}
\end{rem}


%finir de taper les notes de cours 

\chapter{Nombres complexes}

\minitoc
\section{Généralité}
\begin{defi}[Propriété de \(\C\)]

	On ADMET l'existence d’un ensemble noté \(\C\), dont les éléments sont appelés nombres complexes, tel que :
	\begin{enumerate}
		\item \(\C\) contient\(\R \)
		\item \(\C\) est muni de deux opérations \(+\) et \(\times\) sur \(\C\) qui étendent les opérations \(+\) et \(\times\) connues sur \(\R\) et suivent les mêmes règles de calcul que celles-ci
		\item \(\C\) contient un élément noté \(\i\) vérifiant \(\i^2 = -1\)
		\item Tout élément \(z\) de \(\C\) s'écrit de manière uneique sous la forme \(z = a+\i b\) avec \(\paren{a,b} \in \R^2\)
	\end{enumerate}
\end{defi}
\begin{rem}
	\begin{itemize}
		\item La forme \(z = a+\i b\)  avec \(\paren{a,b} \in \R^2\) est dite forme algébrique du nombre complexe \(z\) \begin{itemize}
			      \item le réel \(a\) est dit partie réelle du nombre complexe \(z\) et noté \(a = \Reel{z}\)
			      \item le réel \(b\) est dit partie imaginaire du nombre complexe \(z\) et noté \(b = \Im{z}\)
		      \end{itemize}
		\item L'unicité d'écriture d'un nombre complexe sous forme algébrique se traduit par : \\
		      Pour tout réels \(a,b,a'\) et \(b'\), on a :
		      \[a+\i b = a'+\i b' \text{si, et seulement si, } a=a' \text{ et } b= b'\]
	\end{itemize}
\end{rem}

\begin{defprop}[Opériation sur \(\C\)]
	L’ensemble \(\C = \accol{a + \i b \tq \paren{a, b} \in \R^2}\) est muni deux opérations + et \(+\) et \(\times\) définies par, pour tout nombre complexe \(z\) de forme algébrique \(a + \i b\) et tout nombre complexe \(z'\) de forme algébrique \(a' + \i b'\) : \[\begin{cases}
			z+z' = (a+\i b) + (a'+\i b') = (a+a')+\i(b+b') \\
			z \times z' = (a+\i b) \times (a'+\i b') = (aa'-bb') + \i(ab'+a'b)
		\end{cases}\]
\end{defprop}

\begin{defprop}[Extension des résultat vus dans \(\R\)]
	\begin{enumerate}
		\item Pour tout \(n\) entier naturel et tout nombre complexe \(z\) différent de \(1\), on a :
		      \[\sum_{k=0}^n z^k = \frac{1-z^{k+1}}{1-z}\]
		\item Pour tout \(n\) entier naturel et tout couple \((z,z')\) nombres complexes , on a :
		      \[(z+z')^n = \sum_{k=0}^{n}\binom{n}{k}z^k(z')^{n-k} = \sum_{k=0}^{n}\binom{n}{k}z^{n-k}(z')^k\]
		\item Pour tout \(n\) entier naturel et tout couple \((z,z')\) nombres complexes , on a :
		      \[z^n+(z')^n = (z-z')\paren{z^{n-1}+z^{n-1}z'+\dots+z(z')^{n-2}+(z')^{n-1}} = (z-z')\sum_{k=0}^{n-1} z^{n-1-k}(z')^{k} =(z-z')\sum_{k=0}^{n-1} z^{k}(z')^{n-1-k} \]
	\end{enumerate}
\end{defprop}

\begin{defprop}[Plan complexe : affixe d’un point, d’un vecteur]
	Dans toute la suite, on considère le plan usuel muni d’un repère orthonormé direct.
	\begin{itemize}
		\item A tout complexe \(z\), on peut associer le point \(M\) de coordonnées \((\Reel{z}, \Ima{z})\) dit image de \(z\).
		\item A tout point \(M\) de coordonnées \((x, y)\), on peut associer le complexe \(z = x + \i y\) dit affixe de \(M\) .
	\end{itemize}
	On identifie donc \(\C\) au plan usuel muni d’un repère orthonormé direct et on parle de “plan complexe”. \\

	A tout complexe \(z\), on peut aussi associer le vecteur \(\vec{u}\) de coordonnées \((\Reel{z}, \Ima{z})\) dit image de z et à tout vecteur  \(\vec{u}\) de coordonnées \((x, y)\), on peut associer le complexe \( z = x + \i y\) dit affixe de  \(\vec{u}\) . Ainsi :
	\begin{itemize}
		\item Pour tout vecteur  \(\vec{u}\) d’affixe \(z\) et tout réel \(\alpha\), le vecteur \(\alpha \vec{u}\) a pour affixe \(\alpha z\). \\
		\item Pour tous vecteurs \(\vec{u}\) et \(\vec{u'}\) d’affixes respectives \(z\) et \(z'\), le vecteur \(\vec{u} + \vec{u'} \) a pour affixe \(z + z'\). \\
		\item Pour tous points \(M\) et \(M '\) d’affixes respectives \(z\) et \(z'\) , le vecteur \(\vec{MM'}\)a pour affixe \(z' - z\).
	\end{itemize}
\end{defprop}

\section{Conugué d'un nombre complexe}
\begin{defi}
	On appelle conjugué d’un nombre complexe \(z\) et on note \(\conj{z}\) le nombre complexe défini par : \[\conj{z} = \Reel{z} -\i \Ima{z}\]
	Pour tout nombre complexe \(z\), le point d’affixe \(\conj{z}\) et le point d’affixe \(z\) sont symétriques par rapport à l’axe des réels dans le plan complexe.
\end{defi}

\begin{defprop}
	Pour tous nombres complexes \(z\) et \(z'\), on a les propriétés suivantes :
	\begin{enumerate}
		\item \(z+\conj{z} = 2 \Reel{a}\)\\
		\item \(z-\conj{z} = -2 \Ima{z}\) \\
		\item \(\conj{\conj{z}} = z \)\\
		\item \(\conj{z+z'} = \conj{z}+\conj{z'}\)\\
		\item \(\conj{zz'} = \conj{z} \conj{z'} \)\\
		\item \(\conj{\frac{z}{z'}} = \frac{\conj{z}}{\conj{z'}}\)
	\end{enumerate}
\end{defprop}

\section{module d'un nombre complexe}
\begin{defprop}
	On appelle module d’un nombre complexe \(z\) et on note \(\abs{z}\) le nombre réel positif défini par : \[\abs{z} = \sqrt{ \paren{\Reel{z}}^2+\paren{\Ima{z}}^2}\]
\end{defprop}

\begin{defprop}[interprétation géometriques]
	\begin{itemize}
		\item Pour tout nombre complexe \(z\), le module \(\abs{z}\) est : \begin{itemize}
			      \item la distance entre le point d’affixe \(0\) et le point d’affixe \(z\) ;
			      \item la norme de tout vecteur d’affixe \(z\)
		      \end{itemize}
		\item Pour tous nombres complexes  \(z\) et \(z'\) le module \(\abs{z-z'}\) est :\begin{itemize}
			      \item la distance entre les points d’affixe \(z\) et \(z'\) ;
			      \item la norme du vecteur d’affixe \(z' - z\)
		      \end{itemize}
		\item Soit \(r\) un réel positif, \(z_0\) un nombre complexe et \(M_0\) le point d’affixe \(z_0\).
		      \begin{itemize}
			      \item Les points du plan dont l’affixe \(z\) vérifie \(\abs{z - z_0} = r\) forment le cercle de centre \(M_0\) et de rayon \(r\).
			      \item Les points du plan dont l’affixe \(z\) vérifie \(\abs{z - z_0} \leq r\) forment le disque de centre \(M_0\), de rayon \(r\)
		      \end{itemize}
	\end{itemize}
\end{defprop}

\begin{prop}
	Pour tous nombres complexes \(z\) et \(z'\), on a les propriétés suivantes :
	\begin{itemize}
		\item \(\abs{\Reel{z}} \leq \abs{z}\) et \(\abs{\Ima{z}} \leq \abs{z}\)
		\item \(\abs{z}^2 = z \conj{z}\)
		\item \(\abs{zz'} = \abs{z} \abs{z'}\)
		\item \(\abs{\frac{z}{z'}} = \frac{\abs{z}}{\abs{z'}}\) Dans le cas où \(z'\) est non nul
		\item \(\frac{z}{z'} = \frac{z\abs{z'}}{\abs{z'}^2}\)
		\item \(\abs{z+z'} \leq \abs{z}+\abs{z'}\) avec égalité si, et seulement si il existe un réel positif \(\alpha\) tel que \(z' = \alpha z\)
	\end{itemize}
\end{prop}

\section{Nombre complexe de module \(1\) et trigonométrie}
\begin{defi}[Cercle trigonométrique]
	On identifie le cercle trigonométrique et l’ensemble des nombres complexes de module \(1\) que l’on note : \[\U = \accol{z \in \C \tq \abs{z} = 1}\]
\end{defi}

\begin{defprop}
	Pour tout nombre réel \(t\), on appelle exponentielle imaginaire de \(t\) et on note \(e^{\i t}\) le nombre complexe défini par :
	\[e^{\i t} = \cos(t) + \i \sin(t) \]
	Pour tous nombres réels \(t\) et \(t'\), on a l’égalité : \[e^{\i(t+t')} = e^{\i t}e^{\i t'}+ \]
\end{defprop}


\begin{defprop}[Formule D'Euler]
	Pour tout nombre réel \(t\), on a les égalités suivantes dites formules d’Euler
	\[\cos(t) = \frac{e^{\i t}+e^{-\i t}}{2} \text{ et } \sin(t) = \frac{e^{\i t} - e^{-\i t}}{2}\]
\end{defprop}
\begin{prop}[Technique de l'angle moitié]

	La technique de l’angle moitié permet l’obtention de factorisations classiques à savoir retrouver :
	\begin{itemize}
		\item pour tout \(t\) réel, \(1+e^{\i t} = e^{\i \frac{t}{2}}\paren{e^{-\i \frac{t}{2}}+e^{\i \frac{t}{2}}} = 2 \cos\paren{-\frac{t}{2}}e^{\i \frac{t}{2}} = 2 \cos\paren{\frac{t}{2}}e^{\i \frac{t}{2}} \)
		\item pour tout \(t\) réel, \(1-e^{\i t} = e^{\i \frac{t}{2}}\paren{e^{-\i \frac{t}{2}}-e^{\i \frac{t}{2}}} = 2 \sin\paren{-\frac{t}{2}}e^{\i \frac{t}{2}} = -2 \sin\paren{\frac{t}{2}}e^{\i \frac{t}{2}} \)
		\item pour tout réel \(p\) et \(q\), \(e^{\i p}+e^{\i q} = e^{\i \frac{p+q}{2}}\paren{e^{\i \frac{p-q}{2}}+e^{-\i \frac{p-q}{2}}} = 2 \cos\paren{\frac{p-q}{2}}e^{\i \frac{p+q}{2}}  \)
		\item pour tout réel \(p\) et \(q\), \(e^{\i p}-e^{\i q} = e^{\i \frac{p+q}{2}}\paren{e^{\i \frac{p-q}{2}}-e^{-\i \frac{p-q}{2}}} = - 2 \sin\paren{\frac{p-q}{2}}e^{\i \frac{p+q}{2}}  \)
	\end{itemize}
	\underline{Remarque} : \\
	En écrivant la partie réelle et la partie imaginaire de \(e^{\i p} \pm e^{\i q}\) à partir des deux dernières factorisations, on trouve des formules de factorisation pour \(\cos (p) \pm \cos (q) \)et \(\sin (p) \pm \sin (q)\) \\ \\
	\underline{Linéarisation} \\
	A l’aide des formules d’Euler et du binôme de Newton, on peut transformer une expression du type
	\(cos(t)^n\) ou \(sin(t)^n\) avec \(t\) réel et \(n\) entier naturel en une combinaison linéaire de \(cos(pt)\) ou de \(sin(pt)\)
	avec \(p\) un entier naturel. Cela est notamment utile pour du \underline{calcul de primitives}.
\end{prop}

\begin{exoex}
	~\\Soit \(f(x) = \paren{\sin(x)}^3\) avec \(x \in \R\). Calculer la primitive de \(f\)
\end{exoex}

\begin{corr}
	\begin{align*}
		\paren{\sin(x)}^3 & = \paren{\frac{e^{\i x}-e^{-\i x}}{2 \i}}^3                                              \\
		                  & =\frac{1}{-8\i} \paren{e^{3\i x}+3\paren{e^{-\i x}}-3\paren{e^{\i x}} -e^{-3 \i x}}      \\
		                  & =\frac{1}{-4} \paren{\frac{e^{3\i x}-e^{-3\i x}}{2 \i}-3\frac{e^{\i x}-e^{_\i x}}{2 \i}} \\
		                  & = -\frac{1}{4}\sin(3x) +\frac{3}{4}\sin(x)
	\end{align*}
	Donc \(F_\lambda(x) = \frac{1}{12}\cos(3x)- \frac{3}{4}\cos(x) + \lambda \) pour \(\lambda \in \R \)
\end{corr}

\begin{defprop}[Formule de Moivre]
	Pour tout nombre réel \(t\) et tout entier relatif \(n\), on a \(e^{\i nt} = \paren{e^{\i t}}^n\), c’est-à-dire :
	\[\cos(nt)+\i\sin(nt) = \paren{cos(t)+\i \sin(t)}^n\]
\end{defprop}

\begin{dem}[Moivre par récurrence]
	Soit \(n \in \N\) et \(t\in\R\)
	Montrons que \(\quantifs{\forall (n,t) \in \N\times\R} e^{\i nt} = \paren{e^{\i t}}^n\) \\
	On note \(P(n)\) la Propriété \guillemets{\(e^{\i nt} = \paren{e^{\i t}}^n\)}
	\begin{itemize}
		\item \underline{Initialisation} :
		      \(P(0)\) est vrai car \(\begin{cases}
			      \paren{e^{\i t}}^0 & = 1 \\
			      e^{\i t 0}         & = 1
		      \end{cases}\)
		\item \underline{Hérédité}
		      Soit \(n \in \N\) tel que \(P(n)\) est vrai, Montrons que \(P(n+1)\) est vrai :
		      \begin{align*}
			      e^{\i (n+1) t} & = e^{\i(n+1)t}                      \\
			                     & = e^{\i n t} \times e^{\i t}        \\
			                     & =\paren{e^{\i t}}^n \times e^{\i t} \\
			                     & = \paren{e^{\i t}}^{n+1}            \\
		      \end{align*}
	\end{itemize}
	Donc \(P(n+1)\) Vrai.
\end{dem}

\begin{appl}[Applications usuelles importantes]
	~\\
	Soit \(C = \sum_{k=0}^n \cos(kt)\) et \(S = \sum_{k=0}^n \sin(kt)\) avec \(n \in \N\) et \(t \in \R\)\\
	On Obtient des expressions simplifiées des sommes \(C\) et \(S\) par le calcul annexe suivant
	\[C+\i S = \sum_{k=0}^n e^{\i kt} = \sum_{k=0}^n \paren{e^{\i t}}^k =
		\begin{cases}
			n+1                               & \text{si } t\equiv 0\croch{2 \pi} \\
			\frac{1-e^{\i(n+1)t}}{1-e^{\i t}} & \text{ sinon }
		\end{cases}
	\]
	qui donne \[C+\i S = \begin{cases}
			n+1                                                                       & \text{si } t\equiv 0\croch{2 \pi} \\
			\frac{\paren{1-e^{\i(n+1)t}}\paren{1-e^{\i t}}}{2\paren{1-\cos\paren{t}}} & \text{ sinon }
		\end{cases} \]
	On conclut alors sur les valeurs de \(C\) et \(S\) en exhibant les parties réelle et imaginaire de \(C + \i S\).
\end{appl}

\section{Forme trigonométrique pour les nombres complexes non nuls}

\begin{defprop}
	Tout nombre complexe non nul \(z\) peut s’écrire sous la forme \[ z = re^{\i \theta}\]
	avec \(r\) un réel strictement positif et \(\theta\) un réel. Cette écriture est dite forme trigonométrique de \(z\). \\
	\underline{Attention} \\
	Dans cette écriture de \(z\).
	\begin{itemize}
		\item le réel strictement positif \(r\) est \underline{unique} car il est nécessairement égal à \(\abs{z}\)
		\item le réel \(\theta\) n'est \underline{pas unique} car si le réel \(\theta\) convient alors les réels \(\theta ' \equiv \theta \croch{2 \pi}\) conviennent.
	\end{itemize}
\end{defprop}

\begin{dem}
	Soit \(z\in \Cs\), alors \(\abs{z} \neq 0 \) donc \(\frac{z}{\abs{z}}\) existe avec  \(\abs{\frac{z}{\abs{z}}} = \frac{\abs{z}}{\abs{\abs{z}}} = \frac{\abs{z}}{\abs{z}} = 1\) \\
	Donc \(\frac{z}{\abs{z}} \in \U\) donc il existe \(\theta \in \R \) tel que \(\frac{z}{\abs{z}} = e^{\i \theta} \iff z = \abs{z}e^{\i \theta}\) \\
	Ceci prouve l'existence de l'écriture. \\
	\(r\) est unique car : \(\begin{cases}
		z & = re^{\i \theta}  \\
		z & = r'e^{\i \theta}
	\end{cases} \imp \begin{cases}
		\abs{z} & = r  \\
		\abs{z} & = r'
	\end{cases} \imp r = r'\)
\end{dem}
\begin{defprop}[Arguments]
	Soit \(z\) un nombre complexe non nul. Tous les nombres réels \(\theta\) tels que \(z\) peut s'écrire \[z = re^{\i \theta}\] avec \(r\) réel strictement positif sont dits arguments de \(z\) \\
	\underline{Remarque}\\
	Si \(\theta\) est un argument de \(z\) complexe non nul, on peut écrire \(\arg(z) \equiv \theta\croch{2\pi}\)
\end{defprop}

\begin{prop}
	Pour tous nombres complexes non nuls \(z\) et \(z'\), on a :
	\begin{enumerate}
		\item \(\arg\paren{zz'} \equiv \arg\paren{z}+\arg\paren{z'}\croch{2 \pi}\) \\
		\item \( \arg\paren{\frac{z}{z'}} \equiv \arg\paren{z}-\arg\paren{z'}\croch{2 \pi}\)
	\end{enumerate}
\end{prop}

\begin{defprop}[Transformation de \(a\cos(t) + b\sin(t)\) en \(A\cos(t-\phi)\)]
	Soit \(a, b\) et \(t\) des nombres réels avec \((a, b)\neq (0, 0)\). On peut écrire
	\[a\cos(t)+b\sin(t) = \Reel{\paren{a-\i b}\paren{\cos(t)+\i \sin(t)}} = \Reel{(a-\i b)e^{\i t}} \]
	puis \(a-\i b = Ae^{-\i \phi} \) avec \(A\) réel strictement positif et \(\phi\) un réel ce qui donne :
	\[a\cos(t)+b\sin(t) = \Reel{(a-\i b)e^{\i t}} = \Reel{Ae^{\i(t-\phi)}} \]
	Donc \(a\cos(t)+b\sin(t) = A\cos(t-\phi)\)
\end{defprop}

\section{Fonctions d'une variable réelle à valeurs complexes}

\begin{defi}
	Une fonction de variable réelle à valeurs complexes notée \(f\) est un objet mathématique qui, tout élément \(x\) d’une partie non vide de \(\R\), associe un et un seul nombre complexes noté \(f (x)\).
\end{defi}

\begin{defprop}[Ce qui s’étend aux fonctions de variable réelle à valeurs complexes]
	\begin{itemize}
		\item Notation fonctionnelle
		\item Domaine de définition
		\item Image d’un réel, antécédent d’un complexe
		\item Parité, imparité, périodicité
		\item Somme, produit, quotient de fonctions et multiplication d’une fonction par un complexe
		\item Dérivation
	\end{itemize}
\end{defprop}


\begin{defprop}[Ce qui ne s’étend pas aux fonctions de variable réelle à valeurs complexes]
	\begin{itemize}
		\item Composition de fonctions
		\item Monotonie
		\item Fonction majorée, minorée ou bornée
		\item Fonction réciproque
	\end{itemize}
\end{defprop}

\begin{defprop}[Dérivation]
	Soit \(I\) un intervalle de \(\R\) non vide et non réduit à un point.
	Soit \(f\) une fonction définie sur \(I\) à valeurs complexe. \\
	On note \(\Reel{f} :I \to \R\) et \(\Ima{f}:I\to\R\) les fonctions d’une variable réelle à valeurs réelles définies par :
	\[\quantifs{\forall x \in I}\paren{\Reel{f}}(x) = \Reel{f(x)} \text{ et } \paren{\Ima{f}}(x) = \Ima{f(x)} \]
	On dit que : \begin{itemize}
		\item \(f\) est dérivable en \(x_0\) si les fonctions \(\Reel{f}\) et \( \Ima{f} \) sont dérivables en \(x_0\)
		\item \(f\) est dérivable sur \(I\) si les fonctions \(\Reel{f}\) et \( \Ima{f} \) sont dérivables sur \(I\)
	\end{itemize}
	Selon le cas de figure, on appelle :
	\begin{itemize}
		\item nombre dérvée de \(f\) en \(x_0\) et on note \(f'(x_0)\) le nombre complexe suivant : \[f'(x_0) = \paren{\Reel{f}'(x_0)} + \ \paren{\Ima{f}'(x_0)}\]
		\item fonction dérivée de \(f\) sur \(I\) et on note \(f'\) la fonction de variable réelle à valeurs complexes suivante :
		      \[ f' = \paren{\Reel{f}'} + \ \paren{\Ima{f}'}\]
	\end{itemize}
\end{defprop}

\begin{prop}
	\begin{enumerate}
		\item \underline{Combinaison linéaire}\\
		      Soit \(f\) et \(g\) deux fonctions définies sur \(I\) et à valeurs complexes et \((\alpha, \beta)\) un couple de complexes. Si \(f\) et \(g\) sont dérivables sur \(I\) alors \(\alpha f + \beta g \) est dérivable sur \(I\) et sa dérivée vérifie :
		      \[\paren{\alpha f + \beta g}' = \alpha f'+\beta g'\]
		\item \underline{Produit}\\
		      Soit \(f\) et \(g\) deux fonctions définies sur \(I\) et à valeurs complexes . Si \(f\) et \(g\) sont dérivables sur \(I\) alors \(fg\) est dérivable sur \(I\) et sa dérivée vérifie :
		      \[\paren{fg}' = f'g+fg'\]

		\item \underline{Quotient}\\
		      Soit \(f\) et \(g\) deux fonctions définies sur \(I\) et à valeurs complexes tel que \(g\) ne s’annule pas sur \(I\). Si \(f\) et \(g\) sont dérivables sur \(I\) alors \(\frac{f}{g}\) est dérivable sur \(I\) et sa dérivée vérifie :
		      \[\paren{\frac{f}{g}}' = \frac{f'g-g'f}{g^2}\]
	\end{enumerate}
\end{prop}

\begin{appl}[exemple important]
	Soit \(\phi\) une fonction définie sur \(I\) à valeurs complexes.
	On note \(f : I \to \C\) la fonction définie sur \(I\) par :
	\[\forall t \in I, f(t) = e^{\Reel{\phi(t)}}e^{\i\Ima{\phi(t)}}\]
	Si \(\phi\) est dérivable sur \(I\) alors \(f\) est dérivable sur \(I\) et sa dérivée vérifie :
	\[\forall t \in I,f'(t) = \phi '(t)f(t) \]
	\underline{Remarque} \\
	La fonction \(f\) sera aussi notée \(f = \exp(\phi)\) après étude de l’exponentielle complexe dans le chapitre \guillemets{Nombres complexes (\(2\))} ce qui permettra d’écrire \((\exp(\phi))' = \phi' \exp(\phi)\) et donc d’étendre une propriété déjà connue dans le cas où \(\phi\) est à valeurs réelles.
\end{appl}

\chapter{Fonctions usuelles : Rappel et complément}

\minitoc
\section{Fonction exponentielle}
\begin{defprop}
	Il existe une unique fonction \(f\) définie sur \(\R\), dérivable sur \(\R\) à valeurs réelles vérifiant \(f' = f\) et \(f(0) = 1\) \\
	Cette fonction, appelée fonction exponentielle et notée \(x\mapsto \exp(x)\) ou \(x\mapsto e^x\) vérifie :
	\begin{itemize}
		\item pour tout \(x\) et \(y\) des réels , \(e^{x+y}  =e^xe^y\)
		\item pour tout x réel, \(e^{-x} = \frac{1}{e^x} \)
		\item pour tout \(x\) réel et tout \(n\) entier relatif, \(e^{nx} = \paren{e^x}^n\)
		\item pour tout \(x\) réel, \(e^x >0\)
		\item la fonction \(\exp\) est définie et dérivable sur \(\R\).
		\item la dérivée de \(\exp\) sur \(\R\) est \(\exp\).
		\item la fonction \(\exp\) est strictement croissante sur \(\R\).
		\item \(\lim_{x\to\minf}e^x =0\)
		\item \(\lim_{x\to\pinf} e^x = \pinf \)
		\item \(\lim_{x\to 0} \frac{e^x-1}{x} = 1\)
		\item pour tout réel \(x\), \(e^x\geq 1+x\)
	\end{itemize}
\end{defprop}

\section{Fonction logarithmes}
\begin{defprop}
    La fonction réciproque de la fonction exponentielle est appelée fonction logarithme népérien et notée \(\ln\) . \\
Elle vérifie : \begin{itemize}
    \item  pour tous \(x\) et \(y\) réels strictement positifs, \(ln(xy) = ln(x) + ln(y)\) 
    \item pour tout \(x\) réel strictement positif, \(\ln\paren{\frac{1}{x}} = -\ln(x) \)
    \item \(\ln(1) = 0\)
    \item pour tout \(x\) réel strictement positif et tout \(n\) entier relatif, \(\ln(x^n) = n\ln(x)\)
    \item la fonction \(\ln\) est définie et dérivable sur \(\Rps\).
	\item la dérivée de \(\ln\) sur \(\Rps\) est \(x\mapsto\frac{1}{x}\).
	\item la fonction \(\ln\) est strictement croissante sur \(\Rps\).
	\item \(\lim_{x\to 0}\ln(x) =\pinf\)
	\item \(\lim_{x\to\pinf} e^x = \pinf \)
	\item \(\lim_{x\to 0} \frac{\ln(x+1)}{x} = 1\)
	\item pour tout réel \(x>-1\), \(\ln(1+x)\geq x\)
\end{itemize}
\end{defprop}

\begin{defprop}[logarithme en base \(2\) et en base \(10\)]
    Les fonctions logarithme en base \(2\), notée \(\log_2\), et logarithme en base \(10\) notée \(\log_{10}\) sont définie sur \(\Rps\) par, pour tout réel \(x\) strictement positif : 
    \[\log_2(x) = \frac{\ln(x)}{\ln(2)}\text{ et }\log_{10}(x) =\frac{\ln(x)}{\ln(10)} \]
    On as aussi : \begin{itemize}
        \item \(\log_2(2) = 1\) et \(\log_{10}(10) = 1\)
        \item pour tout \(x\) entier relatif, \(\log_2(2^n) = n \) et \(\log_{10}(10^n) =n\)
        \item \(\log_2\) et \(\log_{10}\) ont même monotonie et même limites aux bornes de \(\Rps\) que la fonction \(\ln\)
    \end{itemize}
\end{defprop}

\section{Fonctions hyperboliques}
\begin{defprop}
    \begin{enumerate}
        \item On appelle cosinus hyperbolique la fonction, notée \(\ch\) définie \(\R\) par, pour tout \(x\) réel, 
        \[\ch(x) = \frac{e^x+e^{-x}}{2}\]
        \item On appelle sinus hyperbolique la fonction, notée \(\sh\) définie \(\R\) par, pour tout \(x\) réel, 
        \[\sh(x) = \frac{e^x-e^{-x}}{2}\]
    \end{enumerate}
\end{defprop}

\begin{defprop}[Relation fondamentale de la trigonométrie hyperbolique]
    Pour tout réel \(x\),on a : \[\ch^2(x)-\sh^2(x) = 1\]
\end{defprop}

\begin{dem}
\[\quantifs{\forall x \in \R} \ch^2(x)-\sh^2(x) = \paren{\ch(x)+\sh(x)}\paren{\ch(x)-\sh(x)} = \paren{e^x}\paren{e^{-x}} = e^0 = 1 \]
\end{dem}

\begin{defprop}[étude de la fonction \(\ch\)]
    \begin{enumerate}
        \item La fonction \(\ch\) est définie et dérivable sur \(\R\)
        \item la dérivée de \(\ch\) sur \(\R\) est la fonction \(\sh\)
        \item la fonction \(\ch\) est paire avec \(\ch(0) =1\)
        \item la fonction \(\ch\) est : 
        \begin{enumerate}
            \item strictement décroissante sur \(\Rms\)
            \item strictement croissante sur \(\Rps\)
        \end{enumerate}
        \item \(\lim_{x\to\minf} \ch(x) = \pinf \)
        \item \(\lim_{x\to\pinf} \ch(x) = \pinf \)
    \end{enumerate}    
\end{defprop}

\begin{defprop}[étude de la fonction \(\sh\)]
    \begin{enumerate}
        \item La fonction \(\sh\) est définie et dérivable sur \(\R\)
        \item la dérivée de \(\sh\) sur \(\R\) est la fonction \(\ch\)
        \item la fonction \(\sh\) est impaire avec \(\sh(0) =0\)
        \item la fonction \(\sh\) est strictement croissante sur \(\R\)
        \item \(\lim_{x\to\minf} \sh(x) = \minf \)
        \item \(\lim_{x\to\pinf} \sh(x) = \pinf \)
    \end{enumerate}    
\end{defprop}

\section{Tangente hyperbolique}
\begin{defprop}
    On appelle tangente hyperbolique la fonction, notée, \(\tth\), définie sur \(\R\) par, pour tout \(x\) réel \[\tth(x) = \frac{\ch(x)}{\sh(x)} = \frac{e^x-e^{-x}}{e^x+e^{-x}}\].
\end{defprop}
\begin{defprop}[étude de la fonction \(\tth\)]
    \begin{enumerate}
        \item La fonction \(\tth\) est définie et dérivable sur \(\R\)
        \item la dérivée de \(\tth\) sur \(\R\) est la fonction \(1-\tth^2=\frac{1}{\ch^2}\)
        \item la fonction \(\tth\) est impaire avec donc \(\tth(0) =0\)
        \item la fonction \(\tth\) est strictement croissante sur \(\R\)
        \item \(\lim_{x\to\minf} \tth(x) = -1 \)
        \item \(\lim_{x\to\pinf} \tth(x) = 1 \)
    \end{enumerate}    
\end{defprop}

\begin{defprop}[formule d'addition et de duplication]
    Pour tout couple de réel \((a,b)\), on a:
    \begin{enumerate}
        \item \(\ch(a+b) = \ch(a)\ch(b)+\sh(a)\sh(b)\)
        \item \(\ch(a-b) = \ch(a)\ch(b)-\sh(a)\sh(b)\)
        \item \(\sh(a+b) = \ch(a)\sh(b)+\sh(a)\ch(b)\)
        \item \(\sh(a-b) = \ch(a)\sh(b)-\sh(a)\ch(b)\)
        \item \(\tth(a+b) = \frac{\tth(a)+\tth(b)}{1+\tth(a)\tth(b)}\)
        \item \(\tth(a-b) = \frac{\tth(a)-\tth(b)}{1-\tth(a)\tth(b)}\)
        \item \(\ch(2a) = \ch^2(a)-\sh^2(a) = 2\ch^2(a)-1 = 2\sh^2(a) +1\)
        \item \(\sh(2a) = 2\sh(a)\ch(a)\)
        \item \(\tth(2a) = \frac{2\tth(a)}{1+\tth^2(a)}\)
    \end{enumerate}
\end{defprop}

\section{Arccos}
\begin{defprop}
    La fonction \(c : \intervii{0}{\pi} \to \intervii{-1}{1}\) définie par :

    \[\text{Pour tout } x \text{ dans },c(x) = \cos(x)\] 
    
    est une bijection de \(\intervii{0}{\pi}\) sur \(\intervii{-1}{1}\) de bijection réciproque \(c^{-1} : \intervii{-1}{1} \to \intervii{0}{\pi}\) notée \(\Arccos\)
    \\ Autrement dit : 
    \begin{itemize}
        \item pour tout réel \(y\) dans \(\intervii{-1}{1}\), l'équation \(y=\cos(x)\) admet une unique solution dans \(\intervii{0}{\pi}\)
        \item pour tout réel \(y\) dans \(\intervii{-1}{1}\), \(\Arccos (y)\) est l'unique réel de \(\intervii{0}{\pi}\) donc le cosinus est égal à \(y\)
    \end{itemize}
    Par ailleurs la fonction \(\Arccos\) possède ces propriétés : 
    \begin{enumerate}
        \item la fonction \(\Arccos\) est définie sur \(\intervii{-1}{1}\) et dérivable sur \(\intervee{-1}{1}\)
        \item la dérivée de \(\Arccos\) sur \(\intervee{-1}{1}\) est la fonction \(\Arccos' : x\mapsto \frac{-1}{\sqrt{1-x^2}} \)
        \item la fonction \(\Arccos\) est strictement décroissante sur \(\intervii{-1}{1}\)
    \end{enumerate}
\end{defprop}

\section{Arcsin}
\begin{defprop}
    La fonction \(s : \intervii{-\frac{\pi}{2}}{\frac{\pi}{2}} \to \intervii{-1}{1}\) définie par :

    \[\text{Pour tout } x \text{ dans },s(x) = \sin(x)\] 
    
    est une bijection de \(\intervii{-\frac{\pi}{2}}{\frac{\pi}{2}}\) sur \(\intervii{-1}{1}\) de bijection réciproque \(s^{-1} : \intervii{-1}{1} \to \intervii{-\frac{\pi}{2}}{\frac{\pi}{2}}\) notée \(\Arcsin\)
    \\ Autrement dit : 
    \begin{itemize}
        \item pour tout réel \(y\) dans \(\intervii{-1}{1}\), l'équation \(y=\sin(x)\) admet une unique solution dans \(\intervii{-\frac{\pi}{2}}{\frac{\pi}{2}}\)
        \item pour tout réel \(y\) dans \(\intervii{-1}{1}\), \(\Arcsin (y)\) est l'unique réel de \(\intervii{-\frac{\pi}{2}}{\frac{\pi}{2}}\) donc le sinus est égal à \(y\)
    \end{itemize}
    Par ailleurs la fonction \(\Arcsin\) possède ces propriétés : 
    \begin{enumerate}
        \item la fonction \(\Arcsin\) est définie sur \(\intervii{-1}{1}\) et dérivable sur \(\intervee{-1}{1}\)
        \item la dérivée de \(\Arcsin\) sur \(\intervee{-1}{1}\) est la fonction \(\Arcsin' : x\mapsto \frac{1}{\sqrt{1-x^2}} \)
        \item la fonction \(\Arcsin\) est impaire sur \(\intervee{-1}{1}\)
        \item la fonction \(\Arcsin\) est strictement croissante sur \(\intervii{-1}{1}\)
    \end{enumerate}
\end{defprop}

\section{Arctan}
\begin{defprop}
    ~\\
    La fonction \(t : \intervee{-\frac{\pi}{2}}{\frac{\pi}{2}} \to \R\) définie par :

    \[\text{Pour tout } x \text{ dans },t(x) = \tan(x)\] 
    
    est une bijection de \(\intervee{-\frac{\pi}{2}}{\frac{\pi}{2}}\) sur \(\R\) de bijection réciproque \(t^{-1} : \R \to \intervee{-\frac{\pi}{2}}{\frac{\pi}{2}}\) notée \(\Arctan\)
    \\ Autrement dit : 
    \begin{itemize}
        \item pour tout réel \(y\) dans \(\R\), l'équation \(y=\tan(x)\) admet une unique solution dans \(\intervee{-\frac{\pi}{2}}{\frac{\pi}{2}}\)
        \item pour tout réel \(y\) dans \(\R\), \(\Arctan (y)\) est l'unique réel de \(\intervee{-\frac{\pi}{2}}{\frac{\pi}{2}}\) donc la tangente est égal à \(y\)
    \end{itemize}
    Par ailleurs la fonction \(\Arctan\) possède ces propriétés : 
    \begin{enumerate}
        \item la fonction \(\Arctan\) est définie et dérivable sur \(\R\)
        \item la dérivée de \(\Arctan\) sur \(\R\) est la fonction \(\Arctan' : x\mapsto \frac{1}{1+x^2} \)
        \item la fonction \(\Arctan\) est impaire sur \(\R\)
        \item la fonction \(\Arctan\) est strictement croissante sur \(\R\)
        \item \(\lim_{x\to\minf} \Arctan(x) = -\frac{\pi}{2}\)
        \item \(\lim_{x\to\pinf} \Arctan(x) = \frac{\pi}{2}\)
    \end{enumerate}
\end{defprop}

\section{Fonction puissances réelles}
\begin{defi}
    Soit \(\alpha\) un réel.

    La fonction \(f_{\alpha}\) définie sur \(\Rps\) par 

    \[\quantifs{\forall x \in \Rps} f_{\alpha}(x) = e^{\alpha \ln(x)} \]
    est notée \(f_{\alpha}: x \mapsto x^{\alpha}\) et appelée fonction puissances (réelle). Elle respecte ces propriétés : 
    \begin{itemize}
        \item la fonction \(x\mapsto x^{\alpha}\) est définie et dérivable sur \(\Rps\)
        \item la dérivée de \(x\mapsto x^{\alpha}\) sur \(\Rps\) est \(x\mapsto \alpha x^{\alpha-1}\)
        \item la fonction \(x\mapsto x^{\alpha}\) est : 
        \begin{itemize}
            \item strictement croissante sur \(\Rps\) pour \(\alpha > 0 \) 
            \item strictement décroissante  sur \(\Rps\) pour \(\alpha < 0 \) 
        \end{itemize}
        \item  \(\lim_{x\to 0} x^{\alpha} = \begin{cases}
            0 &\text{ pour } \alpha >0 \\
            \pinf &\text{ pour } \alpha <0
        \end{cases}\)
        \item  
        \(\lim_{x\to \pinf} x^{\alpha} = 
        \begin{cases}
            \pinf &\text{ pour } \alpha >0 \\
             0 &\text{ pour } \alpha <0
        \end{cases}
        \)
    \end{itemize}
\end{defi}

\begin{prop}
    Pour tout couple de réels \(\alpha,\beta\) et tout couple de réels strictement positifs \((x,y)\), on a:
    \[\ln(x^{\alpha}) = \alpha \ln(x) \qquad (xy)^{\alpha} = x^{\alpha}y^{\alpha} \qquad x^{\alpha+\beta} = x^{\alpha}x^{\beta} \qquad \paren{x^{\alpha}}^{\beta} = x^{\alpha \beta}\]
\end{prop}

\begin{defprop}[cas particulier des puissances entières]
    Les fonctions vues ci-dessus étendent les notions de puissances entières déjà connues sur \(\R\) ou \(\Rs\) :
    \begin{itemize}
        \item pour tout entier naturel \(n\), la fonction \(f_n : x\mapsto \prod_{k=1}^{n}x\) est notée \(x\mapsto x^n\)\\
        elle est définie sur \(\R\), dérivable sur \(\R\) et de dérivée \(x\mapsto nx^{n-1}\)
        \item pour tout entier relatif strictement négatif \(n\), la fonction \(f_n : x\mapsto \prod_{k=1}^{-n}x^{-1}\) est notée \(x\mapsto x^n\)\\
        elle est définie sur \(\Rs\), dérivable sur \(\Rs\) et de dérivée \(x\mapsto nx^{n-1}\)
    \end{itemize}
\end{defprop}

\section{croissance comparées}
\begin{defprop}[Cas des fonctions \(x \mapsto \ln (x), x \mapsto x^{\alpha}\) et \(x \mapsto e^x\) avec \(\alpha >0\) ]
    Pour tout \(\alpha\) réel strictement positif, lems croissances comparées des fonctions \(x \mapsto \ln (x), x \mapsto x^{\alpha}\) et \(x \mapsto e^x\) se résument à : 
    \[\lim_{x\to\pinf} \frac{\ln(x)}{x^{\alpha}} = 0 \qquad \lim_{x\to\pinf} \frac{x^{\alpha}}{e^x} = 0 \qquad \lim_{x\to 0} x^{\alpha} \ln(x) = 0\]
    \underline{Remarques}:
    On en déduit les croissances comparées en \(\pinf\) des fonctions précédentes prises deux à deux :
    \begin{itemize}
        \item comparaison du logarithme népérien avec les puissances réelles ou l’exponentielle en \(\pinf\) :
        \[\lim_{x\to\pinf} \frac{\ln(x)}{x^{\alpha}} = 0 \qquad \lim_{x\to\pinf} \frac{\ln(x)}{e^x} = 0\]
        \item comparaison des puissances réelles avec le logarithme népérien ou l’exponentielle en \(\pinf\)
        \[\lim_{x\to\pinf} \frac{x^{\alpha}}{\ln(x)} = \pinf \qquad \lim_{x\to\pinf} \frac{x^{\alpha}}{e^x} = 0\]
        \item comparaison de l’exponentielle avec le logarithme népérien ou les puissances réelles en \(\pinf\)
        \[\lim_{x\to\pinf} \frac{e^x}{\ln(x)} = \pinf \qquad \lim_{x\to\pinf} \frac{e^x}{x^{\alpha}} = \alpha\]
    
    \end{itemize}
\end{defprop}

\begin{defprop}[Cas des fonctions \(x \mapsto \abs{\ln (x)}^{\beta} , x \mapsto x^{\alpha} \)et \(x \mapsto e^{\gamma x}\)]
    Pour tous réels strictement positifs \(\alpha , \beta \) et \(\gamma \), les croissances comparées des fonctions  \(x \mapsto \abs{\ln (x)}^{\beta} , x \mapsto x^{\alpha} \)et \(x \mapsto e^{\gamma x}\) se résument à :
    \[\lim_{x\to\pinf} \frac{\abs{\ln (x)}^{\beta}}{x^{\alpha}} = 0 \qquad \lim_{x\to\pinf} \frac{x^{\alpha}}{e^{\gamma x}} = 0 \qquad \lim_{x\to 0} x^{\alpha}\abs{\ln (x)}^{\beta} = 0 \] 

\end{defprop}

\chapter{Nombres complexes (\(2\))}

\minitoc

\section{Équations algébriques}
\subsection{Préliminaires}
\begin{defi}[Définition d'une fonction polynomiale]
    Une fonction \(P:\C\to\C\) est dite fonction polynomiale à coefficients complexes s'il existe un entier naturel \(n\) et un \(n+1\)-uplet de nombres complexes \((b_0,b_1,\dots,b_n)\) tel que pour tout \(z\) de \(\C\), 
    \[P(z) = b_0+b_1z+\dots+b_nz^n = \sum_{k=0}^n b_k z^k\]
\end{defi}
\begin{prop}[Propriétés de factorisation]
    Soit \(P\) une fonction polynomiale à coefficients complexes et \(a\) un nombre complexe.\\
    Si \(a\) est une racine de \(P\) , autrement dit si \(P (a) = 0\), alors il existe une fonction polynomiale à coefficients complexes \(Q\) tel que, pour tout \(z\) de \(\C\), on a :
    \[P(z) = (z-a)Q(z)\]
\end{prop}

\subsection{Résolution des équations du second degré dans \(\C\)}
\begin{defprop} [cas particulier des équations du type \(z^2 = z_0\)]
    Soit \(z_0\) et \(z\) des nombres complexes de formes algébriques respectives \(x_0 + \i y_0\) et \(x+\i y\) 
    \[z^2 = z_0 \text{ six et seulement si ,} 
    \begin{cases}
        x^2-y^2 &= x_0 \\
        x^2+y^2 &= \sqrt{x_0^2 + y_0^2}\\
        2xy &= y_0 \\
    \end{cases}
    \]
\end{defprop}

\begin{defprop}[Cas général]
    soit \(a,b\) et \(c\) des nombres complexes avec \(a\) non nul. \\
    \begin{itemize}
        \item \underline{Racines} \\Les solutions de l'équation polynomiale \(az^2+bz+c=0\) d'inconnue le nombre complexe \(z\) sont : 
        \[z_1 = \frac{-b-\delta}{2a} \text{ et } z_2 = \frac{-b+\delta}{2a}\]
        où \(\delta\) est une "racine carré" de \(\Delta = b^2 -4ac\), autrement dit où \(\delta\) est un nombre complexe vérifiant : 
        \[\delta^2 = \Delta\]
        \item \underline{Somme et produit des racines (formules de Viète)} \\
        Les racines \(z_1\) et \(z_2\) de la fonction polynomiale \(P:z\mapsto az^2 + bz +c \) vérifient :
        \[z_1+z_2 = -\frac{b}{a} \text{ et } z_1z_2 = \frac{c}{a}\]

    \end{itemize}
\end{defprop}

\begin{dem}[Formule des solutions du cas général]
    soit \(a,b\) et \(c\) des nombres complexes avec \(a\) non nul. \\
    Soit \(z \in \C\)
    \begin{align*}
        az^2+bz+c &= a\paren{z^2+\frac{b}{a}z+\frac{c}{a}} \\
        &=a\paren{\paren{z+\frac{b}{2a}}^2+\frac{c}{a} - \frac{b^2}{4a^2}} \\
        &=a\paren{\paren{z+\frac{b}{2a}}^2 - \frac{b^2-4ac}{4a^2}} \\
        &=a\paren{\paren{z+\frac{b}{2a}}^2 - \frac{\Delta}{\paren{2a}^2}}  \tag*{ on pose \(\Delta = b^2-4ac\)}\\
        &=a\paren{\paren{z+\frac{b}{2a}}^2 - \paren{\frac{\delta}{2a}}^2} \tag*{ on pose \(\delta\)  comme étant la "racine carré" de \(\Delta\)}\\
        &=a\paren{z+\frac{b}{2a}-\frac{\delta}{2a}}\paren{z+\frac{b}{2a}+\frac{\delta}{2a}} \\
        &=a\paren{z-z_1}\paren{z-z_2} \text{ avec } 
        \begin{cases}
            z_1 &= \frac{-b-\delta}{2a} \\\\
            z_2 &= \frac{-b+\delta}{2a}
        \end{cases}
    \end{align*}
\end{dem}

\begin{dem}[Formule de viète]
    soit \(a,b\) et \(c\) des nombres complexes avec \(a\) non nul. \\
    Soit \(P:z\mapsto az^2+bz+c\) 
    \[P(z) = az^2+bz+c = a(z-z_1)(z-z_2) = a(z^2-(z_1+z_2)z+z_1z_2)\]
    donc par identification : 
    \[
    \begin{cases}
        b &=-a(z_1+z_2) \\
        c &= az_1z_2 
    \end{cases} \iff\begin{cases}
        -\frac{b}{a} &=z_1+z_2 \\
        \frac{c}{a} &= z_1z_2 
    \end{cases} 
    \]
\end{dem}

\subsection{Résolution des équations du type \(z^n = z_0\) dans \(\C\) avec \(n\in \Ns\)}

\begin{defi}
    Soit \(n\) un entier naturel non nul et \(z_0\) un nombre complexe. \\
    On appelle racine \(n\)- ième de \(z_0\) tout nombre complexe tel que \(z^n = z_0\)
\end{defi}

\begin{defprop}[Cas particulier où \(z_0 = 1\)]
    \begin{itemize}
        \item \underline{Racines} \\
        Il y a \(n\) racine \(n\)-ième de l'unité qui sont les nombres complexes suivants : 
        \[\omega_k = e^{\i\frac{2k\pi}{n}} \text{ avec } k \in \interventierii{0}{n-1}\] 
        \item \underline{L'ensemble des racines} \\
        \begin{itemize}
            \item L'ensemble des racines \(n\)-ièmes de l'unité est noté 
            \[\U_n = \accol{z\in\R \tq z^n = 1}\]
            \item Les points dont les affixes sont les racines \(n\)-ièmes de l’unité sont les sommets d’un polygone régulier à \(n\) côtés, de centre \(O\) et inscrit dans \(\U\).
        \end{itemize}
    \end{itemize}
\end{defprop}
\begin{dem}
    Soit \(z \in\C\) tel que \(z^n = 1\) \\
    \(z=0\) n'est pas solution donc \(\quantifs{\exists (r,\theta) \in \Rps\times\R}z = re^{\i \theta}\) 
    \begin{align*}
        z^n=1 &\iff r^ne^{\i \theta n }= 1e^{\i \times 0} \\
        &\iff  
            \begin{cases}
                r^n &= 1 \\
                n \theta &\equiv 0 [2\pi]
            \end{cases}\\
        &\iff \begin{cases}
                r &= 1 \\
                \theta \equiv 0 \croch{\frac{2\pi}{n}}
            \end{cases}\\
    \end{align*}
    Ainsi \(S = \U_n =\accol{e^{\i \frac{k2\pi}{n}}\tq k \in \Z}\) \\
    On note \(\fonction{f}{\Z}{\C}{k}{e^{\i \frac{k2\pi}{n}}}\) alors on sait que \(f\) est \(n\) périodique car \(\forall  k \in \Z, \begin{cases}
        k+n &\in \Z \\
        k-n &\in \Z
    \end{cases} \)
    et 
    \begin{align*}
        f(k+n) &= e^{\i \frac{2(k+n)\pi}{n}} \\
               &= e^{\i \frac{2k\pi}{n}}\times e^{\i \frac{2n\pi}{n}} \\
               &= e^{\i \frac{2k\pi}{n}}\times 1 \\
               &= f(k)
    \end{align*}
    Donc \(S = \U_n =\accol{e^{\i \frac{k2\pi}{n}}\tq k \in \interventierii{0}{n-1}}\). \\
    Montrons que \(\U_n\) contient \(n\) élément autrement dit que: 
    \[\quantifs{\forall (k,k') \in\interventierii{0}{n-1}^2;k<k'} \imp e^{\i \frac{k2\pi}{n}} \neq e^{\i \frac{k'2\pi}{n}}\]
    \underline{Par l'absurde :}\\
    Soit \(k\) et \(k'\) dans \(\interventierii{0}{n-1}\) avec \(k<k'\), supposons que \(e^{\i \frac{k2\pi}{n}} = e^{\i \frac{k'2\pi}{n}}\) \\
    alors \(\frac{k2\pi}{n} \equiv \frac{k'2\pi}{n} \croch{2\pi}\)\\
    donc il existe \(k'' \in \Ns\) tel que \(\frac{k2\pi}{n} -\frac{k'2\pi}{n} = 2 k'' \pi\) car \(k'-k >0\)\\
    Ainsi \(k'-k = nk''\) avec \(\begin{cases}
        k'-k \in \interventierii{1}{n-1} &\text{ car } 0\leq k<k'\leq n-1 \\
        nk'' \in \interventierie{n}{\pinf} & \text{ car } k''\in \Ns
    \end{cases}\) \\
    Ce qui est absurde et prouve que \(e^{\i \frac{k2\pi}{n}} \neq e^{\i \frac{k'2\pi}{n}}\)\\
    \conclusion \\
    Il y a exactement \(n\) racine \(n\)-ièmes de l'unité qui sont les \( \omega_k = e^{\i \frac{k2\pi}{n}}\) pour \(k\in \interventierii{0}{n-1}\)
\end{dem}


\begin{defprop}[Cas général]
    Il y a \(n\) racines \(n\)- ièmes pour le nombre complexe non nul \(z_0\) de forme trigonométrique \(z_0 = r_0e^{\i\theta_0}\) qui sont les nombres complexes suivants :
    \[\sqrt[n]{r_0}e^{\i \paren{\frac{\theta_0}{n}+\frac{2k\pi}{n}}} \text{ avec }k \in \interventierii{0}{n-1}\]
\end{defprop}

\begin{ex}
\[\U_3 = \accol{1,\exp\paren{\frac{2\i \pi}{3}},\exp \paren{\frac{4 \i \pi}{3}}}\]

\[\U_4 = \accol{1,\exp\paren{\frac{2\i \pi}{4}},\exp \paren{\frac{4 \i \pi}{4}},\exp \paren{\frac{6 \i \pi}{4}}} = \accol{1,\i,-1,-\i}\]

\[\U_4 = \accol{1,\exp\paren{\frac{2\i \pi}{5}},\exp \paren{\frac{4 \i \pi}{5}},\exp \paren{\frac{6 \i \pi}{5}},\exp \paren{\frac{8\i \pi}{5}}} \]

\end{ex}

\section{Exponentielle complexe}
\begin{defi}
Pour tout nombre complexe \(z\), on appelle exponentielle de \(z\) le nombre complexe noté \(e^z\) le nombre complexe \(e^z\) défini par : 
\[e^z = e^{\Reel{z}}e^{\i \Ima{z}}\]
dont le module est \(\abs{e^z} = e^{\Reel{z}}\) et les arguments vérifient \(\arg(e^z)\equiv \Ima{z} [2\pi]\)
\end{defi}

\begin{prop}
    Soit un couple de nombres complexes \((z,z')\)
    \begin{itemize}
        \item on a l'égalité suivante : 
        \[e^{z+z'} = e^ze^{z'}\]
        on en déduit les propriétés suivantes :
        \begin{itemize}
            \item \(\frac{1}{e^z} = e^{-z}\)
            \item pour tout entier relatif \(n\), on a: \(e^{nz} = \paren{e^z}^n\)
        \end{itemize}
        \item \(e^z = e^{z'}\) si et seulement si, \(z-z' \in 2\i\pi \Z\) en notant \(2\i \pi \Z =\accol{2\i k \pi \tq k\in \Z}\)
    \end{itemize}
\end{prop}
\begin{defprop}[Résolution de l'équations \(e^z = a\) avec \(a\) un nombre complexe]
    Soit \(a\) un nombre complexe. \\
    \begin{itemize}
        \item Si \(a\) est nul alors l'équation \(e^z = a\) n'a pas de solution dans \(\C\)
        \item Si \(a\) est non nul alors l'équation \(e^z = a\) possède une infinité de solutions dans \(\C\) qui sont les nombres complexes 
        \[z= \ln(z)+\i \theta\]
        avec \(r\) le module de \(a\) et \(\theta\) un argument de \(a\).
    \end{itemize}
\end{defprop}

\section{Interprétations géométriques}
\begin{defprop}[Module et arguments de \(\frac{z'-\omega}{z-\omega}\)]
    Soit \(\omega,z \) et \(z'\) des nombres complexes tel que \(\omega \neq z\) et \(\omega \neq z'\) de points images notés \(\Omega,M\) et \(M'\). \\
    Alors : 
    \[\abs{\frac{z'-\omega}{z-\omega}} = \frac{\Omega M'}{\Omega M} \text{ et } \arg\paren{\frac{z'-\omega}{z-\omega}} = \paren{\overrightarrow{\Omega M},\overrightarrow{\Omega M'}}[2\pi]\]

\end{defprop}
\begin{defprop}[Traduction de l’alignement et l’orthogonalité]
     Soit \(\Omega,M \) et \(M'\) trois points du plan tels que \(\Omega \neq M\) et \(\Omega \neq M'\) d'affixes respectivement notées \(\omega,z\) et \(z'\)
     \begin{itemize}
        \item Les points  \(\Omega,M \) et \(M'\) sont alignés si, et seulement si,  \(\frac{z'-\omega}{z-\omega}\) est un réel
        \item Les droites \(\Omega M\) et \(\Omega M'\) sont orthogonales si, et seulement si, \(\frac{z'-\omega}{z-\omega}\) est un imaginaire pur.
     \end{itemize}
\end{defprop}
\begin{defprop}[Ecriture complexe de transformations du plan vues au collège]
    Dans ce paragraphe, \(M\) et \(M'\) sont deux points du plan complexe d’affixes respectives \(z\) et \(z'\).
    \begin{itemize}
        \item \underline{Translation} \\
        Soit \(b\) un nombre complexe. \\
        \(M'\) est l'image par \(M\) par la translation de vecteur d'affixe \(b\) si, et seulement si \[z' = z+b\]
        \item \underline{Homothétie} \\
        Soit \(\alpha\) un nombre réel et \(\Omega\) un point du plan d'affixe \(\omega\). \\
        \(M'\) est l'image par \(M\) par l'Homothétie de centre \(\Omega\) et de rapport \(\alpha\)  si, et seulement si \[z'-\omega = \alpha(z-\omega)\]
        \item \underline{Rotation} \\
        Soit \(\theta\) un nombre réel et \(\Omega\) un point du plan d'affixe \(\omega\). \\
        \(M'\) est l'image par \(M\) par la rotation de centre \(\Omega\) et d'angle \(\theta\)  si, et seulement si \[z'-\omega = e^{\i \theta}(z-\omega)\]
    \end{itemize}
\end{defprop}

\begin{defprop}[Applicaitons \(z\mapsto az+b\) avec \((a,b) \in \Cs\times\C\)]
    Soit  \((a,b) \in \Cs\times\C\).L'application \(f\) de \(\C\) dans \(\C\) définie par 
    \[f(z) = az+b\]
    est dite similitude directe. \\
    \underline{Interprétation géométrique :}
    Pour tout \(z\in \C\), on note \(M\) le point d'affixe \(z\) et \(M'\) le point d'affixe \(z' = f(z)\)
    \begin{itemize}
        \item \underline{Cas où \(a=1\)} \\
        On a alors l'équivalence suivante : \(z' = f(z)\) so et seulement si, \(z'-z = b\) \\
        L'application \(f\) est donc la translation de vecteur d'affixe \(b\).
        \item\underline{Cas où \(a\neq1\)}\\ 
        \(f\) admet alors un point fixe \(\omega\) donné par \(\omega = \frac{b}{1-a} \) dont le point image est noté \(\Omega\) \\
        On en déduit les équivalences suivantes :  \\
        \begin{align*}
            z' = f(z) &\text{ si, et seulement si, } z'-\omega = a'(z-\omega) \\
                      &\text{ si, et seulement si, } z'-\omega  = \abs{a}\paren{e^{\i \arg(a)}(z-\omega)} \\
                      &\text{ si, et seulement si, } z'-\omega  = e^{\i \arg(a)}\paren{\abs{a}(z-\omega)} \\
        \end{align*}
    \end{itemize}
    L'application \(f\) est donc la composée commutative : 
    \begin{itemize}
        \item de l'Homothétie de centre \(\Omega\) et de rapport \(\abs{a}\)
        \item de la rotation de centre \(\Omega\) et d'angle \(\arg(a)\)
    \end{itemize}
\end{defprop}

\begin{defprop}[Applications \(z\mapsto a\conj{z}+b\) avec \((a,b) \in \Cs\times\C\)]
    Soit  \((a,b) \in \Cs\times\C\). \\
    L'application \(g\) de \(\C\) dans \(\C\) définie par 
    \[g(z) = a\conj{z}+b\]
    est dite similitude indirect. Elle peut s'écrire sous la forme de la composée non commutative. 
    \[g = f \circ s\]
    avec :
    \begin{itemize}
        \item \(s:z\mapsto \conj{z}\) qui est la symétrie axiale d'axe de la droite des réels
        \item \(f:z\mapsto az+b\) qui est une similitude directe.
    \end{itemize}
\end{defprop}

\chapter{Calcul de primitives}

\minitoc
\begin{nota}
    \begin{itemize}
        \item \(I\) et \(J\) désignent des intervalles de \(\R\), non vides et non réduits à un point
        \item \(\K\) désigne l'ensemble \(\R\) ou \(\C\)
    \end{itemize}
\end{nota}

\section{Primitives}

\begin{defprop}
    Soit \(f:I\to\K\) une fonction quelconque. \\
    On dit qu'une fonction \(F:I\to\K\) est une primitive de \(f\) sur \(I\) si \(F\) est dérivable sur \(I\) de dérivée \(f\) \\
    Si \(f\) admet une primitive \(F\) sur \(I\) alors l'ensemble des primitives de \(f\) sur \(I\) est \(\accol{x\mapsto F(x)+ \lambda \tq \lambda \in \K}\)
\end{defprop}

\begin{theo}[Théorème fondamental de l'analyse]
    Si \(f\) \textbf{CONTINUE} sur \(I\) alors : 
    \begin{itemize}
        \item pour tout \(x_0\) réel, la fonction \(F:\int_{x_0}^{x} f(t) dt\) est une primitive de \(f\) sur \(I\)
        \item la fonction \(f\) admet des primitives sur \(I\)
    \end{itemize}
\end{theo}

\begin{defprop}[Application au calcul d'intégrales sur un segment]
    Si \(f\) est \textbf{CONTINUE} sur \(I\) et \(F\) une primitive de \(f\) sur \(I\) alors, pour tous réels \(a\) et \(b\) dans \(I\), on a :
    \[\int_{a}^b f(t) dt = F(b)-F(a) \underset{\mathrm{notation}}{=} \croch{F}^a_b \]
\end{defprop}

\section{Primitives usuelles}
\begin{defprop}[Puissances entières ou réelles]
    ~\\
    \renewcommand{\arraystretch}{2.75}
	\begin{tabular}{|l|l|c|}

		\hline
		Si la fonction \(f\) est \(\dots\) & alors une primitive de \(f\) est \(\dots\) & sur tout intervalle \(I\) inclus dans \(\dots\)\\
        \hline
        \(x\mapsto x^n\) avec \(n \in \N\) & \(x\mapsto \frac{1}{n+1} x^{n+1}\) & \(\R\) \\
        \(x\mapsto x^n\) avec \(n \in \Zm \pd\accol{-1}\) & \(x\mapsto \frac{1}{n+1} x^{n+1}\) & \(\Rs\) \\
        \(x\mapsto \frac{1}{x}\) & \(x\mapsto \ln\paren{\abs{x}}\) & \(\Rs\) \\
        \(x\mapsto \frac{1}{2\sqrt{x}}\) & \(x\mapsto \sqrt{x}\) & \(\Rps\) \\
        \(x\mapsto x^{\alpha}\) avec \(\alpha \in \R \pd\Z\)& \(x\mapsto \frac{1}{\alpha + 1}x^{\alpha+1}\) & \(\Rps\) \\

        \hline
	\end{tabular}
\end{defprop}

\begin{defprop}[Exponentielle à valeurs réelles ou complexes et logarithme népérien]
    ~\\
    \renewcommand{\arraystretch}{2.75}
	\begin{tabular}{|l|l|c|}

		\hline
		Si la fonction \(f\) est \(\dots\) & alors une primitive de \(f\) est \(\dots\) & sur tout intervalle \(I\) inclus dans \(\dots\)\\
        \hline
        \(x\mapsto e^{\lambda x}\) avec \(\lambda \in \Ks\) & \(x\mapsto \frac{1}{\lambda} e^{\lambda x}\) & \(\R\) \\
        \(x\mapsto e^x\) & \(x\mapsto e^x\) & \(\R\) \\
        \(x\mapsto \ln(x)\) & \(x\mapsto x\ln(x)-x\) & \(\Rps\) \\
        \hline
	\end{tabular}
\end{defprop}
\vspace{10cm} 
\begin{defprop}[Fonctions hyperboliques]
    ~\\
    \renewcommand{\arraystretch}{2.75}
	\begin{tabular}{|l|l|c|}

		\hline
		Si la fonction \(f\) est \(\dots\) & alors une primitive de \(f\) est \(\dots\) & sur tout intervalle \(I\) inclus dans \(\dots\)\\
        \hline
        \(x\mapsto \ch(x)\) & \(x\mapsto \sh(x) \) & \(\R\) \\
        \(x\mapsto \sh(x)\) & \(x\mapsto \ch(x) \) & \(\R\) \\
        \(x\mapsto 1-\tth^2(x)\) & \(x\mapsto \tth(x) \) & \(\R\) \\
        \(x\mapsto \frac{1}{\ch^2(x)}\) & \(x\mapsto \tth(x) \) & \(\R\) \\        
        \hline
	\end{tabular}
\end{defprop}
\begin{defprop}[Fonctions circulaires et fonctions circulaires réciproques]
    ~\\
    \renewcommand{\arraystretch}{2.75}
	\begin{tabular}{|l|l|c|}

		\hline
		Si la fonction \(f\) est \(\dots\) & alors une primitive de \(f\) est \(\dots\) & sur tout intervalle \(I\) inclus dans \(\dots\)\\
        \hline
        \(x\mapsto \cos(x)\) & \(x\mapsto \sin(x) \) & \(\R\) \\
        \(x\mapsto \sin(x)\) & \(x\mapsto -\cos(x) \) & \(\R\) \\
        \(x\mapsto 1+\tan^2(x)\) & \(x\mapsto \tan(x) \) & \(\R\pd\accol{\frac{\pi}{2}+k\pi\tq k\in\Z}\) \\
        \(x\mapsto \frac{1}{\cos^2(x)}\) & \(x\mapsto \tan(x) \) & \(\R\pd\accol{\frac{\pi}{2}+k\pi\tq k\in\Z}\) \\
        \(x\mapsto \frac{-1}{\sqrt{1-x^2}}\) & \(x\mapsto \Arccos(x) \) & \(\intervee{-1}{1}\) \\
        \(x\mapsto \frac{1}{\sqrt{1-x^2}}\) & \(x\mapsto \Arcsin(x) \) & \(\intervee{-1}{1}\) \\
        \(x\mapsto \frac{1}{1+x^2}\) & \(x\mapsto \Arctan(x) \) & \(\R\) \\

        \hline
	\end{tabular}
\end{defprop}

\section{Calculs de primitives}
\begin{defprop}
    \begin{itemize}
        \item \underline{Primitives d’une combinaison linéaire de fonctions}\\
        Si \(f : I \mapsto \K\) et \(g : I \mapsto \K\) sont des fonctions qui admettent des primitives sur \(I\) notées \(F\) et \(G\) alors, pour tous \(\alpha\) et \(\beta\) dans \(\K\), la fonction\( \alpha f + \beta g : I \mapsto \K\) admet pour primitive sur \(I\) la fonction \( \alpha F + \beta G\)
        \item \underline{Primitives d’une fonction dérivée de fonctions composées} \\
        Si \(u : I \mapsto \R\) est une fonction dérivable sur \(I\) tel que pour tout \(x\) de \(I\), \(u(x)\) appartient à \(J\) et si \(g : J \mapsto \K \) est une fonction dérivable sur \(I\) alors une primitive de la fonction \(f : x  \mapsto u'(x)g'(u(x))\) sur \(I\) est la fonction \(F : x  \mapsto g (u(x))\).\\
        Dans le tableau ci-dessous (à savoir retrouver à partir des primitives usuelles), \(I\) désigne un intervalle sur lequel \(u\) est dérivable et tel que, pour tout \(x\) de \(I\), \(u(x)\) appartient au domaine de dérivabilité de \(F\) .
    \end{itemize}
    \renewcommand{\arraystretch}{2.5}
    \begin{center}
        
	    \begin{tabular}{|l|l|}
	    	\hline
	    	Si la fonction \(f\) est \(\dots\) & alors une primitive de \(f\) est \(\dots\) \\
            \hline
            \(x\mapsto u'(x)\paren{u(x)}^{\alpha}\) avec \(\alpha \in \R\pd\accol{-1}\)  & \(x\mapsto \frac{1}{\alpha+1}\paren{u(x)}^{\alpha+1} \) \\
            \(x\mapsto \frac{u'(x)}{u(x)}\)  & \(x\mapsto \ln(\abs{u(x)}) \) \\
            \hline
            \(x\mapsto u'(x)e^{\lambda u(x)}\) avec \(\lambda \in \Ks\) & \(x\mapsto \frac{1}{\lambda}e^{\lambda u(x)} \) \\
            \(x\mapsto u'(x)\ln\paren{u(x)}\) & \(x\mapsto u(x)\ln(u(x))-u(x) \) \\
            \hline
            \(x\mapsto u'(x)\ch\paren{u(x)}\) & \(x\mapsto \sh(u(x)) \) \\
            \(x\mapsto u'(x)\sh\paren{u(x)}\) & \(x\mapsto \ch(u(x)) \) \\
            \(x\mapsto u'(x)\paren{1+\tth^2\paren{u(x)}}\) & \(x\mapsto \tth(u(x)) \) \\
            \hline

            \(x\mapsto u'(x)\cos\paren{u(x)}\) & \(x\mapsto \sin(u(x)) \) \\
            \(x\mapsto u'(x)\sin\paren{u(x)}\) & \(x\mapsto -\cos(u(x)) \) \\
            \(x\mapsto u'(x)\paren{1+\tan^2\paren{u(x)}}\) & \(x\mapsto \tan(u(x)) \) \\
            \hline
            \(x\mapsto \frac{-u'(x)}{\sqrt{1-u^2(x)}}\) & \(x\mapsto \Arccos(u(x)) \)\\
            \(x\mapsto \frac{u'(x)}{\sqrt{1-u^2(x)}}\) & \(x\mapsto \Arcsin(u(x)) \)  \\
            \(x\mapsto \frac{u'(x)}{1+u^2(x)}\) & \(x\mapsto \Arctan(u(x)) \) \\
            \hline
	    \end{tabular}
    \end{center}
\end{defprop}

\subsection{Deux théorèmes importants}
\begin{defi}[préliminaire]
    Une fonction \(f : I \mapsto \K\) est dite de classe \(\classe{1}\) sur \(I\) si \(f\) est dérivable sur \(I\) et de dérivée continue sur \(I\)
\end{defi}

\begin{theo}[Intégration par parties]
    Si \(u\) et \(v\) sont deux fonctions de classe \(\classe{1}\) sur \(I\) alors, pour tous réels \(a\) et \(b\) dans \(I\), on a :
    \[\int_{a}^{b}u'(t)v(t)dt = \croch{u(t)v(t)}^b_a - \int_{a}^{b}u(t)v'(t)dt\]
\end{theo}

\begin{dem}
    Soit \(u\) et \(v\) deux applications de \(\ensclasse{1}{I}{\R}\) alors \(\forall (a,b) \in I^2\) : 
    \[
    \begin{aligned}
        \int_{a}^{b}(uv)'(t)dt &= \int_{a}^{b}\paren{u'v+uv'}(t)dt \\
         \croch{uv}^b_a &= \int_{a}^{b}(u'v)(t)dt + \int_{a}^{b}(uv')(t)dt \\
          \int_{a}^{b}u'(t)v(t)dt &= \croch{uv}^b_a- \int_{a}^{b}(uv')(t)dt 
    \end{aligned}
    \]
\end{dem}
\begin{theo}[Changement de variable]
    Si \(\phi  :J \mapsto \R\) est fonction de classe \(\classe{1}\) sur \(J\) tel que, pour tout \(t\) de \(J\), \(\phi(t)\) appartient à \(I\) \\
    et\\
    Si \(f  :I\mapsto \K\) est fonction continue sur \(I\) tel que, pour touts \(\alpha\) et \(\beta\) dans \(J\), on a: 
    \[\int_{\alpha}^{\beta} f(\phi(t))\phi'(t) dt = \int_{\phi(\alpha)}^{\phi(\beta)}f(x)dx\]
\end{theo}

\begin{dem}
    Soit \(\phi  :J \mapsto \R\) une fonction de classe \(\classe{1}\) sur \(J\) tel que, pour tout \(t\) de \(J\), \(\phi(t)\) appartient à \(I\) et \(f  :I\mapsto \K\) une fonction continue sur \(I\) tel que, pour tous \(\alpha\) et \(\beta\) dans \(J\), alors : \\

    \(f\) possède une primitive sur \(I\) (car \(f\) est continue \(I\) ) que l'on note \(F\). \\
    On note aussi \(G:t\mapsto F(\phi(t))\) qui est dérivable sur \(J\) par composition ainsi \(G':t\mapsto F'(\phi(t))\times \phi'(t)\), alors :
    \[
    \begin{aligned}
        \int_{\alpha}^{\beta} f(\phi(t))\phi'(t)dt &= \int_{min}^{max} G'(t)dt \\
        &=\croch{G(t)}_{\alpha}^{\beta} \\
        &=F(\phi(\beta)) - F(\phi(\alpha)) \\
        &= \croch{F}_{\phi(\alpha)}^{\phi(\beta)} \\
        &= \int_{\phi(\alpha)}^{\phi(\beta)} f(x)dx
    \end{aligned}
    \]
\end{dem}
\subsection{Primitives de \(x \mapsto e^{ax} \cos(bx)\) ou \(x \mapsto e^{ax} \sin(bx)\)}
\begin{defprop}[]
    \begin{itemize}
        \item \underline{Préliminaire} \\
        Soit \(f\) et \(F\) des fonctions définies sur un intervalle \(I\) à valeurs complexes.
        \begin{enumerate}
            \item \(f\) admet des primitives sur \(I\) si, et seulement si, \(\Reel{f}\) et \(\Ima{f}\) admettent des primitives sur \(I\).
            \item \(F\) est une primitive de \(f\) sur \(I\) si, et seulement si, 
            \(\begin{cases*}
                \Reel{F}\text{ est une primitive de } \Reel{f}\text{ sur  } I \\
                \Ima(F) \text{ est une primitive de } \Ima{f} \text{ sur } I
            \end{cases*}\).
        \end{enumerate}
        \item \underline{Une application usuelle du résultat précédent} \\
        Soit \(a\) et \(b\) des réels tels que \((a,b) \neq (0,0)\). \\
        On note \(\lambda = a+\i b\) et \(f_{\lambda}\) la fonction définie sur \(\R\) par, pour tout \(x\) réel 
            \[f_{\lambda}(x) = e^{ax}\cos(bx) + \i e^{ax}\sin(bx) = e^{ax}e^{bx} \underset{\mathrm{def}}{=} e^{(a+\i b)x} = e^{\lambda x}\]
        La fonction \(F_{\lambda}:x\mapsto \frac{1}{\lambda} e^{\lambda x}\) est une primitive de \(f_{\lambda}\) sur \(\R\) donc :
        \begin{itemize}
            \item la fonction \(\Reel{F_{\lambda}}\) est une primitive de la fonction \(\Reel{F_{\lambda}}:x\mapsto e^{ax}\cos(bx)\) sur \(\R\)
            \item la fonction \(\Ima{F_{\lambda}}\) est une primitive de la fonction \(\Ima{F_{\lambda}}: x\mapsto e^{ax}\sin(bx)\) sur \(\R\)
        \end{itemize}
    \end{itemize}
\end{defprop}

\subsection{Primitives de \(x\mapsto \frac{1}{ax^2+bx+c}\) avec \(a,b\) et \(c\) des réels et \(a\) non nul}
\begin{appl}
    Soit \(a,b\) et \(c\) des réels avec \(a\) non nul et \(g\) la fonction \(g:\R \mapsto \R\) définie par \(g(x) = ax^2+bx+c\) \\
    Trois cas se présentent : 
    \begin{enumerate}
        \item Si \(g\) admet deux racines réelles distinctes \(r_1\) et \(r_2\) alors il existe deux réels \(\alpha_1\) et \(\alpha_2\) tel que :
        \[\forall x \in R\pd\accol{r_1,r_2}, \frac{1}{ax^2+bx+c} = \frac{\alpha_1}{x-r_1}+\frac{\alpha_2}{x-r_2}\]
        Dans ce cas, \\
        une primitive de \(x\mapsto \frac{1}{ax^2+bx+c}\) sur tout intervalle \(I\) inclus dans \(R\pd\accol{r_1,r_2}\) est :
        \[x\mapsto \alpha_1 \ln\abs{x-r_1} + \alpha_2\ln\abs{x-r_2}\]


        \item si \(g\) admet une racine réelle double \(r\) alors il existe un réel \(\alpha\) tel que :
        \[\forall x \in R\pd\accol{r}, \frac{1}{ax^2+bx+c} = \frac{\alpha}{(x-r)^2}\]
        Dans ce cas,\\
        une primitive de \(x\mapsto \frac{1}{ax^2+bx+c}\) sur tout intervalle \(I\) inclus dans \(\R\pd\accol{r}\) est :
        \[x\mapsto \frac{-\alpha}{x-r}\]


        \item Si \(g\) n'admet pas de racines réelles alors, en écrivant \(g\) sous forme canonique, on peut trouver trois réels \(\alpha,\beta \) et \(\gamma\) tel que :
        \[\forall \in \R, \frac{1}{ax^2+bx+c} = \frac{\alpha}{\paren{\frac{x+\beta}{\gamma}}^2+1}\]
        Dans ce cas,\\
        une primitive de \(x\mapsto \frac{1}{ax^2+bx+c}\) sur tout intervalle \(I\) inclus dans \(\R\) est :
        \[x\mapsto \alpha \gamma \arctan\paren{\frac{x+\beta}{\gamma}}\]
    \end{enumerate}
\end{appl}


\chapter{Compléments sur les nombres réels}

\minitoc
\section{Parties denses de \(\R\)}

\begin{defprop}[Généralité]
    Une partie \(X\) de \(R\) est dite dense dans \(\R\) si elle rencontre tout intervalle ouvert non vide de \(\R\). \\
    ~\\
    \underline{En pratique:} \\
    Pour établir qu'une partie \(X\) de \(R\) est dense dans \(R\) à l'aide de cette définition, on montre que tout intervalle du type \(\intervee{a}{b}\) avec \(a\) et \(b\) des réels tel que \(a<b\), contient au moins un élément de \(X\).
\end{defprop}

\begin{ex}
\begin{itemize}
    \item Les ensembles \(\N\) et \(\Z\) sont des parties de \(\R\) qui ne sont pas denses dans \(\R\)
    \item Les ensemble \(\Q\) et \(\R\pd\Q\) sont des parties de \(\R\) qui sont denses dans \(\R\)
\end{itemize}
\end{ex}

\begin{dem}[Preuve de \(Q\) dense dans \(\R\)]
    Soit \(a\) et \(b\) des réels avec \(a<b\).\\
    Montrons que \(\intervee{a}{b}\) contient un élément de \(\Q\), c'est à dire \(\exists (p,q) \in \Z\times\Ns\) tel que \(a<\frac{p}{q}<b\)
    autrement dit \(qa<p<qb\) \\
    Ainsi pour que \(p\) existe il faut que : \\
    \begin{align*}
    &qa - qb > 1       \tag*{car \( p \in \Z \)} \\
    &q(a - b) > 1      \\
    &q > \frac{1}{b-a} \tag*{car \( b > a \)}\\
    \text{Prenons } &q=\floor{\frac{1}{b-a}} +1 \tag*{car \(\frac{1}{b-a}>\floor{\frac{1}{b-a}}+1\)}  
\end{align*}
Prenons \(p=\floor{qa}+1\), donc  \( p-1 \leq qa<p\) \\
or \(p<qb\) car \(q>\frac{1}{b-a} \iff qb-qa>1 \iff qb>qa+1\geq \floor{qa}+1=p\) \\
Ainsi \(qa<p<qb\imp a<\frac{p}{q}<b<b\) avec \(q=\floor{\frac{1}{b-a}} +1\) et \(p=\floor{qa}+1\).
\\\\
\conclusion \\
Tout intervalle réel de type \(\intervee{a}{b}\) avec \(a<b\) contient un rationnel donc par définition, \(\Q\) est dense dans \(\R\).
\end{dem}

\begin{dem}[preuve que \(\R\pd\Q\) est dense dans \(\R\)] ~\\
    \begin{itemize}
        \item \underline{Préliminaire} : Démonstration que \(\sqrt{2}\) est irrationnel\\
        On suppose qu'il existe \((p,q) \in \Z\times\Ns \) avec \(p\) et \(q\) premier entre eux tel que \(\frac{p}{q} = \sqrt{2}\) alors : 
        \begin{align*}
            \frac{p}{q} = \sqrt{2} &\iff \sqrt{2}q = p \\
            &\imp 2q^2 = p^2 \qquad \text{donc } p^2 \text{ est pair ce qui explique } p \text{ pair}\\
            &\imp 2q^2 = (2k)^2 \qquad \text{en posant } p =2k \text{ avec } k \in \Z \\
            &\imp 2q^2 = 4k^2 \\
            &\imp 2k^2 = q^2 \qquad \text{donc } q^2 \text{ est pair et donc } q \text{ aussi} 
        \end{align*}
        Ce qui est absurde car \(p\) et \(q\) sont premier entre eux donc ils ne peuvent pas être tous les deux pair.
        \conclusion \(\sqrt{2}\) est irrationnel.
        \item \underline{Preuve que \(\R\pd\Q\) est dense dans \(\R\)} \\~\\
        Soit \(a\) et \(b\) des réels avec \(a<b\).\\
        Montrons que \(\intervee{a}{b}\) contient un irrationnel :\\
        Par densité de \(\Q\) dans \(\R\), \(\intervee{\frac{a}{\sqrt{2}}}{\frac{b}{\sqrt{2}}}\) contient un rationnel \(r\)\\
        on a donc \(\frac{a}{\sqrt{2}}<r<\frac{b}{\sqrt{2}} \imp a<\sqrt{2}r<b\)
        \begin{itemize}
            \item \underline{Si \(r\neq 0\)}\\
            \(\sqrt{2}r \in \intervee{a}{b}\) et \(\sqrt{2}r\) est irrationnel car sinon \(\sqrt{2}r\) serait rationnel et alors \(\underset{\in \Q}{\sqrt{2}r} \times \underset{\in \Q}{\frac{1}{r}} = \sqrt{2}\) donc \(\sqrt{2} \in \Q\) ce qui est faux.\\
            Donc \(\intervee{a}{b}\) contient un irrationnel.
            \item \underline{Si \(r = 0\)}\\
            On raisonne de même manière mais sur avec un intervalle \(\intervee{0}{b}\) et \(\intervee{0}{\frac{b}{\sqrt{2}}}\)\\
            Ainsi on trouve \(r'\in \intervee{0}{\frac{b}{\sqrt{2}}}\inter \Q\) puis \(r'\sqrt{2} \in \intervee{0}{b}\inter\paren{\R\pd\Q}\)\\
            Donc \(\intervee{a}{b}\) contient un irrationnel.
        \end{itemize}
    \end{itemize}
    \conclusion Tout intervalle réel de type \(\intervee{a}{b}\) avec \(a<b\) contient un irrationnel donc par définition, \(\R\pd\Q\) est dense dans \(\R\).
\end{dem}

\begin{theo}[Caractérisation séquentiel des parties denses dans \(\R\)]
    Une partie \(X\) de \(\R\) est dense dans \(\R\) si, et seulement si, tout réel est limite d'une suite d'élément de \(X\)
\end{theo}
\begin{dem}
    Soit \(X\) une partie de \(\R\)
    On procède par double implication.
    \begin{itemize}
    \item[\impdir]
    On suppose que \(X\) est dense dans \(\R\), soit \(x\) un réel et \(n\in\N\) \\
    alors \(\intervee{x-\frac{1}{n+1}}{x}\) contient un élément de \((u_n)\) de \(X\) par densité de \(X\) dans \(\R\) \\
    Donc \(\forall n \in \N,x-\frac{1}{n+1}<u_n<x\) or \(x-\frac{1}{n+1} \underset{n\to\pinf}{\to}x\) et \(x\underset{n\to\pinf}{\to}x\) donc par théorème d'encadrement \(u_n \underset{n\to\pinf}{\to}x\) \\
    \conclusion \\
    tour réel x est limite d'une suite \((u_n)\) d'élement de \(X\)
    \item[\imprec] On suppose que tout réel est limite d'une suite d'élement de \(X\) \\
    Soit \((a,b)\in\R^2\) avec \(a<b\) et \(l\in \intervee{a}{b}\)\\
    par hypothèse, il existe une suite \((u_n)\) telle que \(\forall n \in \N, u_n \in X\) et \(u_n\underset{n\to\pinf}{\to}l\)\\
    par définition de la limite, \(\intervee{a}{b}\) qui contient \(l\) contient aussi tous les termes de la suite \((u_n)\) à partir d'un certain rang d'où l'existence de 
    \(\begin{cases}
        u_{n_0} &\in X \\
        u_{n_0} &\in \intervee{a}{b}
    \end{cases}\)\\
    \conclusion \\
    \(X\) est dense car pour tout \(\intervee{a}{b}\) avec \(a<b\) il existe un élément (ici \(u_{n_0}\)) de \(X\) dans \(\intervee{a}{b}\)
    \end{itemize} 
    \conclusion \\
    Par double implication le théorème est vérifié
\end{dem}

\section{Approximation décimale d'un réel}
\begin{defprop}[rappel]
    L'ensemble des nombres décimaux est notée \(\D\) et définie par \(\D = \accol{\frac{p}{10^n}\tq (p,n)\in\Z\times\N}\)
\end{defprop}
\begin{prop}[Approximation décimales d'un réel]
    Soit \(x\) un réel et \(n\) un entier naturel. Il existe un unique nombre décimal \(d_n\) tel que :
    \[10^nd_n \in\Z \text{ et } d_n \leq x \leq d_n+10^{-n}\]
    Par ailleurs pour tout réel \(x\) les suites de nombres décimaux \((d_n)\) et \((d_n+10^{-n})\) définie ci-dessus sont convergentes de limite égal à\(x\) donc, par caractérisation séquentielle, l'ensemble \(\D\) est dense dans \(\R\)
\end{prop}

\begin{defprop}[Dévellopement décimal d'un réel]
    Soit \(x\) un réel et \((d_n)\) la suite des valeurs décimales approchées de \(x\) à \(10^{-n}\) près par défaut. \\
    Alors :
    \begin{itemize}
        \item Pour tout \(k\) dans \(Ns\), il existe un unique entier \(a_k\) dans \(\interventierii{0}{9} \) tel que \(d_k-d_{k-1} = \frac{a_k}{10^k}\) \\
        \item Pour tout \(n\) dans \(\N\), \(d_n = \sum_{k=0}^{n} \frac{a_k}{10^k}\) avec \(a_0 = \floor{x}\)
    \end{itemize}
    Puisque la suite \((d_n)\) converge vers \(x\), on peut donc écrire que :   
    \[x = \lim_{n\to\pinf} \paren{\sum_{k=0}^{n} \frac{a_k}{10^k}} \underset{Notation}{=} \sum_{k=0}^{\pinf} \frac{a_k}{10^k} = a_0,a_1a_2\dots\]
    ce qu'on appelle un "dévellopement décimal illimié de \(x\)". \\
    \underline{Par ailleurs} : \\
    L’existence et l’unicité d’un tel \(a_k\) résulte du fait que : \(\forall k \in \Ns, 10^k (d_k - d_{k-1}) \in \interventierii{0}{9}\). L’expression de \(d_n\) sous forme de somme finie s’obtient alors par sommation des égalités \(d_k - d_{k-1} =\frac{a_k}{10^k} \) et télescopage
\end{defprop}

\section{Borne inférieure et supérieure d'une partie de \(\R\)}

\begin{defi}
    Soit \(X\) une partie de \(\R\). S'il existe :
    \begin{itemize}
        \item le plus petit des majorants de \(X\) est appelé borne supérieure de \(X\) et noté \(\sup X\)
        \item le plus grand des minorants de \(X\) est appelé borne inférieure de \(X\) et noté \(\inf X\)
    \end{itemize}
    \underline{Remarques} : \\
    \begin{itemize}
        \item les bornes supérieure ou inférieure de \(X\) ne sont pas nécessairement dans \(X\).
        \item En revanche,
        \begin{itemize}
            \item si \(X\) admet un maximum alors \(X\) admet une borne supérieure, égale au maximum de \(X\) ;
            \item si \(X\) admet un minimum alors \(X\) admet une borne inférieure, égale au minimum de \(X\).
        \end{itemize}

    \end{itemize}
\end{defi}

\begin{prop}[Propriété dite de la borne supérieure/inférieur]
    \begin{itemize}
        \item toute partie non vide et majorée de \(\R\) admet une borne supérieure.
        \item Toute partie non vide et minorée de \(\R\) admet une borne inférieure.
    \end{itemize}
\end{prop}

\begin{defprop}[ Traduction séquentielle de la borne supérieure/inférieure]
Soit \(X\) une partie de \(\R\).
    \begin{itemize}
        \item Si \(X\) est non vide et minorée alors il existe une suite d’éléments de \(X\)  de limite \(\inf X\).
        \item Si \(X\) est non vide et majorée alors il existe une suite d’éléments de \(X\)  de limite \(\sup X\).
        \item Si \(X\) est non vide et non minorée alors il existe une suite d’éléments de \(X\)  de limite \(\minf\) .
        \item Si \(X\) est non vide et non majorée alors il existe une suite d’éléments de \(X\)  de limite \(\pinf\).
    \end{itemize}
\end{defprop}

\begin{defprop}[Droite achevée \(\Rb\)]
    On appelle droite achevée l'ensemble noté \(\Rb\) défini par :
    \[\Rb = \R \union \accol{\minf,\pinf}\]
    On y étend la relation d’ordre \(\leq\), l’addition et la multiplication connues sur \(\R\) avec les conventions :
    \begin{enumerate}
        \item \(\forall x\in \R,\minf < x\pinf\)
        \item \((\minf)+(\minf) = \minf\)
        \item \((\pinf)+(\pinf) = \pinf\)
        \item \(\forall x \in \R,x+(\minf) = (\minf)+x = \minf\)
        \item \(\forall x \in \R,x+(\pinf) = (\pinf)+x = \pinf\)
        \item \(\forall x \in \Rb \pd \accol{0}, x \times (\minf) = (\minf)\times x = \begin{cases}
            \pinf & \text{ si } x<0 \\
            \minf & \text{ si } x>0
        \end{cases}\)
        \item \(\forall x \in \Rb \pd \accol{0}, x \times (\pinf) = (\pinf)\times x = \begin{cases}
            \minf & \text{ si } x<0 \\
            \minf & \text{ si } x>0
        \end{cases}\)
    \end{enumerate}
\end{defprop}

\begin{defprop}[Caractérisation des intervalles de \(\R\)]
    Une partie \(X\) de \(\R\) est un intervalle de \(\R\) si, et seulement si, pour tous réels \(a\) et \(b\) dans \(X\) tels que \(a\leq b\) le segment \(\intervii{a}{b}\) est inclus dans \(X\)
\end{defprop}

\begin{dem}
    On rappelle que \(I\) est un intervalle de \(\R\) si \(I\) est de l'une des formes suivantes : 
    \begin{itemize}
		\item \(I = \emptyset\) \\
		\item \(I = \accol{x \in \R\tq a \leq x \leq b} \underset{\mathrm{notation}}{=} \intervii{a}{b}\) avec \(\paren{a,b} \in \R^2 \) et \(a\leq b \) \\
		\item \(I = \accol{x \in \R\tq a \leq x < b} \underset{\mathrm{notation}}{=} \intervie{a}{b}\) avec \(\paren{a,b} \in \R\times \paren{\R \union \accol{\pinf}} \) et \(a < b\) \\
		\item \(I = \accol{x \in \R\tq a < x \leq b} \underset{\mathrm{notation}}{=} \intervei{a}{b}\) avec \(\paren{a,b} \in \paren{\R \union \accol{\minf}}\times \R \) et \(a < b\) \\
		\item \(I = \accol{x \in \R\tq a < x \leq b} \underset{\mathrm{notation}}{=} \intervee{a}{b}\) avec \(\paren{a,b} \in \paren{\R \union \accol{\minf}}\times  \paren{\R \union \accol{\pinf}} \) et \(a < b\) \\
	\end{itemize}

    Soit \(X\) une partie de \(\R\).
    Dans le cas où \(X\) est l'ensemble vide, l'équivalence attendue est immédiate. On se place donc, dans la suite, dans le cas où \(X\) est une partie non vide de \(\R\) et on raisonne par double implication
    \begin{itemize}
        \item[\impdir] On suppose que \(X\) est un intervalle de \(\R\)\\
        \(X\) est alors d'une des formes \( 2, 3, 4\) ou \(5\) indiquées ci-dessus. Ainsi, pour tous réels \(\alpha\) et \(\beta\) dans \(X\) tels que \(\alpha \leq \beta\), on a bien \(\intervii{\alpha}{\beta} \subset X\)
        \item[\imprec] On suppose que : \(\forall (\alpha,\beta) \in X^2, \alpha \leq \beta \imp \intervii{\alpha}{\beta} \subset X \) \\
        En considérant \(X\) comme partie de la droite achevée \(\Rb\), on peut noter \(m = \inf X\) et \(M = \sup X\) \\
        Montrons que \(\intervee{m}{M} \subset X \subset \intervii{m}{M}\)
        \begin{itemize}
            \item Soit \(t\in \intervee{m}{M}\) \\
            Alors le réel \(t\) n'est pas un majorant de \(X\) (car \(t\) est strictement inférieur à \(M\) qui est le plus petit des majorants de \(X\)) et le réel \(t\) n'est pas un minorant de \(X\)(car \(t\) est strictement supérieur à \(m\) qui est le plus grand es minorants de \(X\)). \\
            ~\\
            Il existe donc\((\alpha, \beta) \in X^2\) tel que \(\alpha <t<\beta\) ce qui prouve que \(t\) appartint à l'intervalle \(\intervee{\alpha}{\beta}\) donc au segment \(\intervii{\alpha}{\beta}\). Comme les réels \(\alpha\) et \(\beta\) appartiennent à \(X\), l'hypothèse faite sur \(x\) donne \(\intervii{\alpha}{\beta} \subset X\) ce qui prouve, en particulier, que \(t\) appartient à \(X\)\\
            \conclusion \(\intervee{m}{M} \subset X\)
            \item Soit \(t\in X\)\\
            Alors, par définition de \(m\) et \(X\), on a : \(m\leq t\leq M\) c'est à dire \(t\in \intervii{m}{M}\) \\
            \conclusion \(X \subset \intervii{m}{M} \)
        \end{itemize} 
        On a donc montré que \(\intervee{m}{M} \subset X\subset \intervii{m}{M}\). Cela implique que \(X\), vue comme partie de \(\Rb\) est égale à l'une des parties suivantes \(\intervee{m}{M},\intervei{m}{M},\intervie{m}{M}\) ou \(\intervii{m}{M}\).\\
        Comme \(X\) est une partie de \(\R\), on en déduit que \(X\) est bien de l'une des formes \( 2, 3, 4\) ou \(5\) indiquées ci-dessus donc que \(X\) est un intervalle de \(\R\)\\
    \end{itemize}
    \conclusion \(X\) est un intervalle de \(\R\) si, et seulement si, \(\forall(\alpha,\beta) \in X^2, \alpha \leq \beta \imp \intervii{\alpha}{\beta} \subset X\)
\end{dem}

\chapter{Ensemble, application et relation}

\minitoc

\section{Ensemble}
\subsection{Généralité}
\begin{defi}
    \begin{itemize}
        \item Un ensemble est une collection d’objets, sans répétition et non ordonnée.
        \item Les objets de l’ensemble sont appelés les éléments de l’ensemble.
        \begin{itemize}
            \item Si \(x\) est un élément de l’ensemble \(E\), on dit que \(x\) appartient à \(E\) et on note \(x \in E\) .
            \item Dans le cas contraire, on dit que \(x\) n’appartient pas à \(E\) et on note \(x \notin E\).
        \end{itemize}
        \item L’ensemble sans élément est appelé l’ensemble vide et noté \(\emptyset\).
        \item Les ensembles avec un seul élément sont appelés des singletons.
        \item Les ensembles avec deux éléments sont appelés des paires.
    \end{itemize}
\end{defi}
\begin{defprop}[Modes de définition d’un ensemble]
    Un ensemble \(E\) peut être défini :\\
    \begin{itemize}
        \item en extension, c’est-à-dire en explicitant tous les éléments de l’ensemble \(E\), dans le cas où
        il compte un nombre fini d’éléments appelé cardinal de l’ensemble. Les éléments de l’ensemble
        sont ainsi tous cités entre accolades.
        Par exemple :
        \begin{itemize}
            \item \(E = \accol{\i}\) singleton contenant le nombre complexe \(\i\) ;
            \item \(E = \accol{\cos, \sin} \)paire contenant les fonctions cosinus et sinus ;
            \item \(E = \accol{2, 3, 5, 7}\) ensemble des nombres premiers inférieurs à \(10\) ;
            \item \(E = \accol{3, 4, . . . , 10}\) ensemble des entiers compris entre \(3\) et \(10\) au sens large (noté aussi \(\interventierii{3}{10}\)).
        \end{itemize}
        \item en compréhension, c’est-à-dire en donnant des propriétés vérifiées par les éléments de
        l’ensemble et eux seuls. Là encore, on utilise des accolades.
        Par exemple :\\
        \begin{itemize}
            \item \(E = \accol{x \in \R \tq x \equiv 0 \croch{2\pi}}\) ensemble des réels congrus à \(0\) modulo \(2\pi\) ;
            \item \(E = \accol{f : \R \to \R \tq \forall x \in \R, f (-x) = f (x)}\) ensemble des fonctions paires de \(\R\) dans \(\R\) ;
            \item \(E = \accol{z \in \C \tq \exists k \in \Z, z = e^{\frac{2\i k \pi}{5}}}\) ensemble des racines \(5\)-ièmes de l’unité.
            \item \(E = \accol{\alpha e \tq \alpha \in \R}\) ensemble des fonctions de la forme \(x \mapsto \alpha e^x\) lorsque \(\alpha\) parcourt \(\R\).
        \end{itemize}
    \end{itemize}
\end{defprop}

\subsection{Inclusion entre ensembles et parties}
\begin{defprop}
    Soit E un ensemble.
    \begin{itemize}
        \item \underline{Inclusion}\\ On dit qu’un ensemble \(F\) est inclus dans \(E\) et on note \(F \subset E\), si tous les éléments de \(F\) appartiennent à \(E\), c’est-à-dire : \(\forall x, \paren{x \in F \imp x \in E}\) .
        \item \underline{Parties}\\ On dit qu’un ensemble \(F\) est une partie ou un sous-ensemble de \(E\) si \(F\) est inclus dans \(E\).
        \item \underline{Ensemble des parties}\\ On note \(\P{E}\) l’ensemble des parties de \(E\), c’est-à-dire \(\P{E} = \accol{A \tq A \subset E}\) .
    \end{itemize}
\end{defprop}
\subsection{Egalité entre ensembles}
\begin{defprop}
    
\begin{itemize}
    \item  \underline{Définition}\\ On dit que deux ensembles \(E\) et \(F\) sont égaux, et on note \(E = F\) , s’ils ont les mêmes éléments, c’est-à-dire : \(\forall x, \paren{x \in E \iff x \in F }\) .
\item  \underline{Caractérisation de l’égalité par double inclusion}\\ Deux ensembles \(E\) et \(F\) sont égaux si, et seulement si, \(E \subset F et F \subset E\).
\end{itemize}
\end{defprop}

\subsection{Opérations sur les parties d’un ensemble}
\begin{defprop}
Soit \(E\) un ensemble et, \(A\) et \(B\) deux parties de \(E\).\\
Soit \(I\) un ensemble et \(\accol{A_i \tq i \in I}\) un ensemble de parties de \(E\).
\begin{itemize}
    \item  \underline{Réunion}\\ 
        On appelle réunion de \(A\) et \(B\), et on note \(A\union B\), la partie de \(E\) définie par \\\(A\union B = \accol{x \in E \tq x \in A \text{ ou } x \in B}\).\\
        Plus généralement, on définit la réunion de parties \(A_i\) de \(E\), avec \(i\) qui varie dans un ensemble \(I\) : \[ \bigunion_{i\in I}A_i = \accol{x \in E \tq \exists i_0 \in I, x \in A_{i_0} }\] .
    \item \underline{ Intersection}\\ 
        On appelle intersection de \(A\) et \(B\), et on note \(A\inter B\), la partie de \(E\) définie par \(A \inter B = \accol{x \in E \tq x \in A \text{ et } x \in B}\).\\
        Plus généralement, on définit l’intersection de parties \(A_i\) de\( E\), avec \(i\) qui varie dans un ensemble \(I\) :\[ \biginter_{i\in I}A_i = \accol{x \in E \tq \forall i \in I, x \in A{i} }\].
    \item \underline{ Différence}\\ On appelle différence de \(B\) dans \(A\), et on note \(A \pd B\), la partie de \(E\) définie par \(A\pd B = \accol{x \in E \tq x \in A \text{ et } X\notin B}\) .
    \item \underline{Complémentaire}\\ 
    On appelle complémentaire de \(A\) dans \(E\) la partie \(E \pd A = \accol{x \in E \tq x \notin A}\) qui est encore notée \(\conj{A}\) ou \(A^c\) (en l’absence d’ambiguité sur l’ensemble dans lequel le complémentaire est considéré).
    \item \underline{ Quelques règles de calcul ou loi de Morgan}\\
    \begin{itemize}
        \item \(\paren{\bigunion_{i\in I}A_i}\inter B = \bigunion_{i \in I} (A_i \inter B)\) et \(  \paren{\biginter_{i\in I} A_i} \union B = \biginter_{i\in I}(A_i \union B)\)
        \item \(\conj{\biginter_{i \in I}A_i} = \bigunion_{i \in I} \conj{A_i}\) et \(\conj{\bigunion_{i \in I}A_i} = \biginter_{i \in I}\conj{A_i}\)
    \end{itemize}
    \item \underline{Recouvrement disjoint et partition d’un ensemble}\\
    L’ensemble \(\accol{A_i \tq i \in I}\) de parties de \(E\) est dit partition de \(E\) si les conditions suivantes sont réunies :\\
    \begin{itemize}
        \item \(E = \bigunion_{i\in I}A_i\)
        \item \(\forall i \in I,A_i \neq \emptyset\)
        \item \(\forall i \in I,\forall j \in I, i\neq j \imp A_i\inter A_j = \emptyset\)
    \end{itemize}
\end{itemize}
\end{defprop}

\begin{dem}[Loi de Morgan]
    Soit \(E\) un ensemble et \(A_j\) des parties de \(E\) où \(i \in I\) et \(B\) une partie de \(E\).
    \begin{itemize}
        \item \underline{Distributivité de l'intersection sur l'union} :\\
        \begin{align*}
            x\in \paren{\bigunion_{i\in I}A_i}\inter B &\iff \paren{x \in \bigunion_{i\in I}A_i} \text{ et } \paren{x \in B} \\
            &\iff \paren{\exists i_0 \in I, x\in A_{i_0}} \text{ et } \paren{x \in B}\\
            &\iff \exists i_0 \in I, x\in A_{i_0}\inter B \\
            &\iff x\in \bigunion_{i\in I}\paren{A_i\inter B}
        \end{align*}
        \item \(\conj{\biginter_{i\in I}A_i} = \bigunion_{i\in I}\conj{A_i}\) :\\
        \begin{align*}
            x \in \conj{\biginter_{i\in I}A_i} &\iff x\notin \biginter_{i \in I} A_i\\
            &\iff \exists A_{i_0}, x\notin A_{i_0}\\
            &\iff x\in \conj{A_{i_0}} \\
            &\iff x \in \bigunion_{i \in I} \conj{A_i}
        \end{align*}
    \end{itemize}
\end{dem}

\subsection{Produit cartésien d’un nombre fini d’ensembles}
\begin{defprop}
    Soit \(E_1, \dots, E_n\) des ensembles.\\
    On appelle produit cartésien de \(E_1, \dots, E_n\) l’ensemble noté \(E_1 \times \dots \times E_n\) défini par :
    \[E_1 \times \dots \times E_n = \accol{\paren{x_1,\dots,x_n} \tq \forall i \interventierii{1}{n}, x_i \in E_i} \]
\end{defprop}


\section{Application}

\subsection{définition de base}
\begin{defprop}
    Une application \(f\) de \(E\) (ensemble de départ) dans \(F\) (ensemble d’arrivée) est un objet mathématique qui, à tout élément \(x\) de \(E\), associe un unique élément de \(F\) noté \(f (x)\) \\
    \underline{Notation fonctionnelle} : \[\fonction{f}{E}{F}{x}{f(x)}\]
\end{defprop}


\begin{defprop}[Image et antécédent]
    Soit \(f : E \mapsto F\) une application.
    \begin{itemize}
        \item Pour tout \(x\) élément de \(E\),\( f (x)\) est un élément de \(F\) appelé l’image de \(x\) par \(f\) .
        \item Soit \(y \in F\) . S’il existe \(x\) dans \(E\) tel que \(y = f (x)\) alors \(x\) est dit un antécédent de \(y\) par \(f\) .
    \end{itemize}
\end{defprop}
\begin{defprop}[Ensemble des applications]
    L’ensemble des applications de \(E\) dans \(F\) est noté \(\ensclasse{\mathcal{F}}{E}{F}\) ou \(F^E\).
\end{defprop}
\begin{defprop}[Egalité entre applications]
    On dit que deux applications \(f\) et \(g\) sont égales, et on note \(f = g\), si les conditions suivantes sont réunies :
    \begin{itemize}
        \item \(f\) et \(g\) ont le même ensemble de départ \(E\) et le même ensemble d’arrivée \(F\) ;
        \item pour tout \(x\) de \(E\),\( f (x) = g(x)\).
    \end{itemize}
\end{defprop}

\begin{defprop}  [Graphe]
Soit \(f : E \mapsto F\) une application. \\
On appelle graphe de \(f\) la partie \(G\) de \(E \times F\) définie par :
\[ G = \accol{\paren{x; f (x)}\tq x \in E} \]
\end{defprop}

\subsection{Fonctions particulières}
\begin{defprop}
    \begin{itemize}
        \item  \underline{Fonction indicatrice d’une partie} \\
        Soit \(A\) une partie de \(E\). L’application \(f\) de \(E\) dans \(\accol{0, 1}\) définie par :
        \[\forall x \in E, f(x) = \begin{cases}
            1 &\text{ si } x\in A \\
            0 &\text{ si } x\notin A
        \end{cases}\]
        est dite fonction indicatrice de \(A\) et notée \(\ind{A}\).
        \item \underline{Restriction} \\
        Soit\( f : E \mapsto F\) une application et \(A\) une partie de \(E\). \\
        L’application \(g : A \mapsto F\) définie par \( \forall x \in A, g(x) = f (x)\) est dite restriction de \(f\) à \(A\) et notée \(\restr{f}{A}\).
        \item \underline{Prolongement} \\
        Soit \(A\) une partie de \(E\) et \(h : A \mapsto F\) une application. \\
        Toute application \(f : E \mapsto F \) telle que \(\restr{f}{A} = h\) est dite prolongement de \(h\) à \(E\).
    \end{itemize}
\end{defprop}

\subsection{Image directe et image réciproque}

\begin{defprop}
    Soit \(f : E \mapsto F\) une application.
    \begin{itemize}
        \item \underline{Image} : \\
        Soit \(A\) une partie de \(E\). On appelle image directe de \(A\) par \(f\) la partie de \(F\) définie par :
        \[f(A) = \accol{y \in F \tq \exists x \in A,y = f(x)} = \accol{f(x)\tq x \in A}\]
        C’est l’ensemble des images par \(f\) des éléments de \(A\).
        \item \underline{Image réciproque} :
        Soit \(B\) une partie de \(F\) . On appelle image réciproque de \(B\) par \(f\) la partie de \(E\) définie par :
        \[f^{-1}(B) = \accol{x\in E\tq f(x) \in B}\]
        C’est l’ensemble des antécédents par \(f\) des éléments de \(B\).
    \end{itemize}
\end{defprop}   

\subsection{Composition d’applications}
\begin{defprop}  
Soit \(f : E \mapsto F\) et \(g : F \mapsto G\) deux applications. L’application \(h : E \mapsto G\) définie par :
\[\forall x \in E, h(x) = g (f (x))\]
est dite composée des applications \(f\) et \(g\) et notée \(h = g \circ f\) .
\end{defprop}

\subsection{Injection, surjection}
\begin{defprop}
    Une application \(f : E \mapsto F\) est dite :
    \begin{itemize}
        \item \underline{Définitions} :\\
        \begin{itemize}
            \item  \underline{injection} si tout élément de \(F\) a au plus un antécédent par \(f\) .
            \item \underline{surjection} si tout élément de \(F\) a au moins un antécédent par \(f\) .
        \end{itemize} 
        \item \underline{Caractérisations pratiques} : \\
        \begin{itemize}
        \item \(f\) est une injection si, et seulement si :\( \forall(x, x') \in E^2, f (x) = f (x') \imp x = x'\).
        \item \(f\) est une surjection si, et seulement si : \(\forall y \in F, \exists x \in E, y = f (x)\).
        \end{itemize}
        \item \underline{Composition} :\\
            La composée de deux injections (resp. surjections) est une injection (resp. surjection).
    \end{itemize}
\end{defprop}

\begin{dem}[Composition]
    \begin{itemize}
        \item \underline{injection} :\\
        Soit \(f : E\mapsto F\) et \(g:F\mapsto G\) deux fonctions injective \\
        \(\forall (x,x')\in E^2 \) tel que \( g(f(x)) = g(f(x'))\) \\
        On a \(f(x) = f(x')\) car \(g\) est une injection \\
        et donc \(x=x'\) car \(f\) est une injection \\
        \conclusion \(\forall (x,x')\in E^2, g(f(x)) = g(f(x')) \imp x=x'\) donc \(g\circ f\) injective\\

        \item\underline{surjection} : \\
        Soit \(f : E\mapsto F\) et \(g:F\mapsto G\) deux fonctions surjectives \\
        Soit\(z \in G\) alors \(\exists y \in F,z=g(y)\) car \(g\) surjective \\
        Soit\(y \in F\) alors \(\exists x \in E,y=f(x)\) car \(f\) surjective \\
        \conclusion \(\forall z \in G,\exists x \in E \) tel que \(z = g(f(x))\) donc \(g\circ f\) surjective
    \end{itemize}
\end{dem}

\subsection{Bijection}
\begin{defprop}
    \begin{itemize}
        \item \underline{Définitions} : \\
        Une application \(f : E \mapsto F\) est dite bijection si tout élément de \(F\) a un unique antécédent par \(f\).\\
        Dans ce cas, l’application \(f ^{-1} : F \mapsto E\) définie par :
        \[\forall y \in F, f^{-1}(y) = x \text{ avec } x \text{ l’unique élément de }E \text{ tel que } y = f (x)\]
        est dite bijection réciproque de \(f\) et vérifie :
        \[f \circ f^{-1} = \id{F} \text{ et } f^{-1} \circ f = \id{F}\]
        \item \underline{ Caractérisation pratique} :\\
        Une application \(f : E \mapsto F\) est une bijection si, et seulement si, \(f\) est une injection et une surjection.
        \item \underline{Composition} :
        \begin{itemize}
            \item La composée de deux bijections est une bijection.
            \item La bijection réciproque de la composée \(g \circ f\) où \(f\) et \(g\) sont des bijections est l’application
            \[(g \circ f )^{-1} = f ^{-1} \circ g^{-1}\]
        \end{itemize}
    \end{itemize}
\end{defprop}

\section{Relation Binaire sur un ensemble}

\subsection{Généralité}
\begin{defprop}
    \begin{itemize}
    \item \underline{Définitions} : \\
    On appelle relation binaire sur un ensemble \(E\) toute partie \(\mathcal{R}\) de \(E \times E\).\\
    Pour tout \((x, y) \in \mathcal{R}\) :
    \begin{itemize}
        \item on dit que \(x\) est en relation avec \(y\) par la relation \(\mathcal{R}\) ;
        \item on note usuellement \(x\mathcal{R}y\)
    \end{itemize}
    \item \underline{Propriétés} : \\
    On dit qu’une relation binaire \(\mathcal{R}\) sur un ensemble \(E\) est :
    \begin{itemize}
        \item réflexive si : \(\forall x \in E, x\mathcal{R}x\) ;
        \item transitive si :\(\forall (x, y, z) \in E^3, (x\mathcal{R}y \text{ et } y\mathcal{R}z)\imp x\mathcal{R}z\) ;
        \item symétrique si : \(\forall (x, y) \in E^2, x\mathcal{R}y \imp y\mathcal{R}x\) ;
        \item antisymétrique si : \(\forall (x, y) \in E^2, (x\mathcal{R}y et y\mathcal{R}x) \imp x = y\).
    \end{itemize}
    \item \underline{ Quelques exemples déjà rencontrés} : 
        \begin{enumerate}
            \item Sur un ensemble \(E\) : la relation d’égalité.
            \item Sur l’ensemble \(\P{E}\) des parties d’un ensemble \(E\) : la relation d’inclusion.
            \item Sur l’ensemble \(\R\) : les relations \(\leq, <\) et la relation de congruence modulo un réel non nul.
            \item Sur l’ensemble \(\ensclasse{\mathcal{F}}{D}{\R} = \R^D\) des applications d’une partie \(D\) de \(\R\) dans \(\R\) : la relation \(\leq\).
            \item Sur l’ensemble Z : les relations de divisibilité \(\divise\) et de congruence modulo un entier non nul.
        \end{enumerate}
    \end{itemize}
\end{defprop}

\subsection{Relations d'équivalence}
\begin{defprop}
    \begin{itemize}
        \item \underline{Définitions} : \\
            Toute relation binaire sur un ensemble \(E\) qui est réflexive, transitive et symétrique est dite relation d’équivalence sur \(E\). Les relations d’équivalence sont souvent notées \(\sim,\simeq \) ou \(equiv\).
        \item \underline{Théorème} : \\
        Soit \(\sim\) une relation d’équivalence sur un ensemble \(E\).\\
        Alors la famille d’ensembles \(\paren{\accol{y \in E \tq x \sim y}}_{x\in E}\) est une partition de \(E\).
        \item \underline{Exemples des relations de congruence}
        \begin{itemize}
            \item La relation de congruence modulo \(2\pi\) est une relation d’équivalence sur \(\R\).\\
                Les classes d’équivalence sont les ensembles \(x + 2\pi\Z = \accol{x + 2n\pi \tq n \in Z}\) avec \(x\) qui décrit \(\intervie{0}{2\pi}\).
            \item La relation de congruence modulo \(n \in \Ns\) est une relation d’équivalence sur \( \Z\).
                Les classes d’équivalence sont les ensembles \(r + n\Z = \accol{r + nq \tq q \in \Z}\) avec \(r\) qui décrit \(\interventierii{0}{n-1}\).
        \end{itemize}
    \end{itemize}
\end{defprop}

\subsection{Relation d'ordre}

\begin{defprop}
    \begin{itemize}
        \item \underline{Définitions} : \\
            Toute relation binaire sur un ensemble \(E\) qui est réflexive, transitive et antisymétrique est dite relation d’ordre sur \(E\). Les relations d’ordre sont souvent notées \(\leq, \precsim , \lesssim \) ou \(\preceq \).
        \item \underline{Ordre partiel et ordre total} : \\ 
            Une relation d’ordre \(\preceq\) sur un ensemble \(E\) est dite totale si :
            \[\forall (x, y) \in E^2, x \preceq y ou y \preceq x\]
            Dans le cas contraire, la relation d’ordre \(\preceq\) est dite partielle.
        \item \underline{ Minorant, majorant, maximum, minimum, etc} : \\
            Les notions de partie minorée, majorée ou bornée ainsi que celles de minorant, majorant, minimum, maximum, borne inférieure ou borne supérieure vues pour les parties de \(\R\) peuvent être étendues aux parties d’un ensemble muni d’une relation d’ordre.\\
            Par exemple, pour \(E\) un ensemble muni d’une relation d’ordre \(\preceq\) et \(A\) une partie de \(E\) :
            \begin{itemize}
                \item \(A\) est dite majorée pour \(\preceq\) s’il existe \(M\) dans \(E\) tel que, pour tout élément \(x\) de \(A\), on a :\( x \preceq M\).\\
                Dans ce cas, on dit que \(M\) est un majorant de \(A\) pour \(\preceq\).
                \item si \(A\) admet un majorant \(M\) pour \(\preceq\) qui appartient à \(A\) alors celui-ci est unique et est appelé le maximum de \(A\) ou le plus grand élément de \(A\) pour \(\preceq\).
            \end{itemize}
    \end{itemize}
\end{defprop}

\end{document}