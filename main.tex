% Set up the document's format to A4 and the font's size to 12pt.
\documentclass[a4paper,12pt]{report}

% Set up the document's title, author and date.
\title{Maths -- MP2I}
\author{Eliott Paquet}
\date{\today}

% Set up the input's encoding to UTF-8, the document's font and language to T1 (adapted to french) and french (the grammar linter uses this parameter).
\usepackage[utf8]{inputenc}
\usepackage[T1]{fontenc}
\usepackage[french]{babel}

\usepackage[dvipsnames]{xcolor}

% Set up the document's margins.
\usepackage{geometry}
\geometry{hmargin=1.5cm,vmargin=1.5cm}

% The three main maths packages. They are used for a lot of things.
\usepackage{amssymb,amsmath}
\usepackage{mathtools}

% Useful to create nice and easy signs or variations tables.
\usepackage{tkz-tab}

% Useful to create any kind of visual representation (graph functions, illustrate geometry problems, etc)
\usepackage{tikz}
\usetikzlibrary{patterns,angles,quotes,arrows,arrows.meta,bending,matrix,calc}

% Allows to edit the itemize environment's default item document-wide.
\usepackage{enumitem}

% Allows to define \notfoo or \nfoo (not recommended) in order for \not\foo to work as wished.
\usepackage{newtxmath}
\DeclareSymbolFont{CMletters}{OML}{cmm}{m}{it}
\DeclareMathSymbol{\nu}{\mathord}{CMletters}{23}

% Makes the table of contents clickable and gives useful commands for links in general.
\usepackage[hypertexnames=false]{hyperref}
\hypersetup{colorlinks=false,linktoc=all}

% Gives the llbracket and rrbracket commands for integer intervals.
\usepackage{stmaryrd}

% Useful to insert nice-looking quotes.
\usepackage{epigraph}

% Allows to insert chapter-specific table of contents.
\usepackage{minitoc}
\mtcselectlanguage{french}
\setcounter{minitocdepth}{6}
% --- Chiffres romains uniquement dans les minitoc
\makeatletter
\newcommand{\minitocromain}{%
  {%
    % Sauvegarde
    \let\oldthesection\thesection
    \let\oldthesubsection\thesubsection
    \let\oldthesubsubsection\thesubsubsection
    % Numérotation "Chapitre en Romain" dans le minitoc
    \renewcommand{\thesection}{\Roman{chapter}.\arabic{section}}
    \renewcommand{\thesubsection}{\Roman{chapter}.\arabic{section}.\arabic{subsection}}
    \renewcommand{\thesubsubsection}{\Roman{chapter}.\arabic{section}.\arabic{subsection}.\arabic{subsubsection}}
    % Génération du minitoc
    \minitoc
  }%
}
\makeatother

% Useful when units are needed.
\usepackage{siunitx}
\sisetup{
locale=FR,
detect-all,
inter-unit-product=\ensuremath{\cdot},
list-final-separator={et},
list-pair-separator={et},
range-phrase={\ensuremath{\xleftrightarrow{}}},
exponent-product=\ensuremath{\cdot},
per-mode=power-positive-first
}

\usepackage[thmmarks,hyperref]{ntheorem}
\makeatletter
\let\old@thm\@thm
\usepackage[lowercase]{theoremref}
\def\@thm#1#2#3{\def\thmref@currname{#3}\old@thm{#1}{#2}{#3}}
\makeatother

% Allows whiteboard digits with \mathds
\usepackage{dsfont}

\usepackage{needspace}

% Useful for better-looking oneline fractions
\usepackage{nicefrac}

% Set up the horizontal space before the first line of a new paragraph to 2em and the vertical space between two paragraphs to 1em.
\setlength{\parindent}{0pt}
\setlength{\parskip}{1em}

% Adds 0.5em to the vertical space between two lines in an align environment. It looks better.
\addtolength{\jot}{0.5em}

% Allows align environment to break if it's too long to fit in the page where it began.
\allowdisplaybreaks[1]

% Trick to make semicolons considered like relation operators (such as =) and therefore being equidistantly spaced from the two numbers around it.
\mathcode`;=\numexpr\mathcode`;-"3000

% Commands for size-adaptative parentheses, brackets, curly brackets, absolute value and magnitude.
\newcommand{\paren}[1]{\left(#1\right)} % (x)
\newcommand{\croch}[1]{\left[#1\right]} % [x]
\newcommand{\accol}[1]{\left\lbrace#1\right\rbrace} % {x}
\newcommand{\abs}[1]{\left\lvert#1\right\rvert} % |x|
\newcommand{\norme}[1]{\left\|#1\right\|} % ||x||
\newcommand{\floor}[1]{\left\lfloor#1\right\rfloor} % ⌊x⌋
\newcommand{\ceil}[1]{\left\lceil#1\right\rceil} % ⌈x⌉

% Commands for size-adaptative intervals and integer intervals. The commands' roots are "interv" and "interventier" and the added e or i at the end mean "excluded" and "included" respectively.
\newcommand{\intervii}[2]{\left[#1;#2\right]} % [a;b]
\newcommand{\intervee}[2]{\left]#1;#2\right[} % ]a;b[
\newcommand{\intervie}[2]{\left[#1;#2\right[} % [a;b[
\newcommand{\intervei}[2]{\left]#1;#2\right]} % ]a;b]
\newcommand{\interventierii}[2]{\left\llbracket#1;#2\right\rrbracket} % non-ASCII characters needed
\newcommand{\interventieree}[2]{\left\rrbracket#1;#2\right\llbracket} % non-ASCII characters needed
\newcommand{\interventierie}[2]{\left\llbracket#1;#2\right\llbracket} % non-ASCII characters needed
\newcommand{\interventierei}[2]{\left\rrbracket#1;#2\right\rrbracket} % non-ASCII characters needed

% Commands for usually used sets.
\newcommand{\N}{\mathbb{N}} % natural integers
\newcommand{\Ns}{\mathbb{N}^*}

\newcommand{\Z}{\mathbb{Z}} % relative integers
\newcommand{\Zp}{\mathbb{Z}_+}
\newcommand{\Zs}{\mathbb{Z}^*}
\newcommand{\Zps}{\mathbb{Z}_+^*}
\newcommand{\Zm}{\mathbb{Z}_-}
\newcommand{\Zms}{\mathbb{Z}_-^*}

\newcommand{\D}{\mathbb{D}} % decimal numbers
\newcommand{\Dp}{\mathbb{D}_+}
\newcommand{\Dm}{\mathbb{D}_-}
\newcommand{\Ds}{\mathbb{D}^*}
\newcommand{\Dps}{\mathbb{D}_+^*}
\newcommand{\Dms}{\mathbb{D}_-^*}

\newcommand{\Q}{\mathbb{Q}} % rational numbers
\newcommand{\Qp}{\mathbb{Q}_+}
\newcommand{\Qm}{\mathbb{Q}_-}
\newcommand{\Qs}{\mathbb{Q}^*}
\newcommand{\Qps}{\mathbb{Q}_+^*}
\newcommand{\Qms}{\mathbb{Q}_-^*}

\newcommand{\R}{\mathbb{R}} % real numbers
\newcommand{\Rp}{\mathbb{R}_+}
\newcommand{\Rm}{\mathbb{R}_-}
\newcommand{\Rs}{\mathbb{R}^*}
\newcommand{\Rps}{\mathbb{R}_+^*}
\newcommand{\Rms}{\mathbb{R}_-^*}
\newcommand{\Rb}{\overline{\mathbb{R}}}

\newcommand{\C}{\mathbb{C}} % complex numbers
\newcommand{\Cs}{\mathbb{C}^*}

\newcommand{\K}{\mathbb{K}}
\newcommand{\Ks}{\mathbb{K}^*}

\newcommand{\A}{\mathbb{A}}
\renewcommand{\L}[2]{\mathscr{L}\paren{#1,#2}}
\newcommand{\Lendo}[1]{\mathscr{L}\paren{#1}}

\newcommand{\prem}{\mathbb{P}}

\newcommand{\U}{\mathbb{U}} % complex numbers whose modulus is 1

\renewcommand{\P}[1]{\mathscr{P}\paren{#1}} % subsets of a set
\newcommand{\Pf}[1]{\mathscr{P}_f\paren{#1}} % finite subsets of a set
\newcommand{\F}[2]{\mathscr{F}\paren{#1,#2}} % functions from 1 to 2
\newcommand{\V}[1]{\mathscr{V}\paren{#1}} % neighborhood of a number

% Redefines \Re and \Im to print Re and Im (the same way as ln or lim) instead of fraktur R and I which don't look nice and are less readable.
\newcommand{\Reel}[1]{\operatorname{Re}\paren{#1}}
\newcommand{\Ima}[1]{\operatorname{Im}\paren{#1}}
\newcommand{\Card}[1]{\operatorname{Card}\paren{#1}}

\newcommand{\im}{\operatorname{Im}}

% Command to print an upright e for the exponential instead of a slanted e and put the exponent.
\newcommand{\e}[1]{\mathrm{e}^{#1}}

% Command to print the imaginary i with a little space on the right. This way, the exponents don't look confusing. \i normally prints a dotless i.
\renewcommand{\i}{i\mkern1mu}

%make the \mathcal command shorter
\renewcommand{\cal}[1]{\mathcal{#1}}

% Redefines \vec such that the arrow covers the whole name of the vector.
%\renewcommand{\vec}[1]{\overrightarrow{#1}}

% Commands for 2D and 3D vectors' coordinates
\newcommand{\dcoords}[2]{\begin{pmatrix}#1\\#2\end{pmatrix}}
\newcommand{\tcoords}[3]{\begin{pmatrix}#1\\#2\\#3\end{pmatrix}}

% Redefines binom to print nicer parentheses around the numbers.
\renewcommand{\binom}[2]{\begin{pmatrix}#2\\#1\end{pmatrix}}

% Command for a QED black square. It automatically prints a whitespace before the square such that it looks nice.
\newcommand{\cqfd}{\text{ }\blacksquare}

%Acronym for "privé de" 
\newcommand{\pd}{\backslash}


% Commands with more explicit names for the best way to express divisibility (mid and nmid).
\newcommand{\divise}{\mid}
\newcommand{\notdivise}{\nmid}

% Commands that do the exact same thing but with explicit names for a complex number's conjugate and an event's negation in probability.
\newcommand{\conj}[1]{\overline{#1}}

% Command for a size-adaptative middle bar meaning "such that" (with spacing around it in order to look nice).
\newcommand{\tq}{\;\middle|\;}

% Command with an explicit name for the scalar product.
\newcommand{\scal}{\cdot}
\newcommand{\vecto}{\operatorname{_\wedge}}

% Shortcut for forcing displaystyle in inline mode.
\newcommand{\ds}{\displaystyle}

% Make the not version of implies, impliedby and iff look nice.
\newcommand{\notimp}{\centernot{\imp}}
\newcommand{\notimpr}{\centernot{\impr}}
\newcommand{\notiff}{\centernot{\iff}}

\newcommand{\notsubset}{\centernot{\subset}}
\newcommand{\notsupset}{\centernot{\supset}}

% Shortcut for P(event).
\newcommand{\proba}[1]{\mathbb{P}\paren{#1}}
\newcommand{\probacond}[2]{\mathbb{P}_{#2}\paren{#1}}

% More explicit names for land (logical and) and lor (logical or).
\newcommand{\et}{\land}
\newcommand{\ou}{\lor}
\newcommand{\non}{\lnot}

% Explicitly named environment for tkz-tab tables. Automatically centers the table and handles the tikzpicture environment.
\newenvironment{tkz}
{
\begin{tikzpicture}
}
{
\end{tikzpicture}
}

% More explicitly named commands for the creation of tkz-tab tables.
\newcommand{\tableauinit}[2]{\tkzTabInit{#1}{#2}}
\newcommand{\tableausignes}[1]{\tkzTabLine{#1}}
\newcommand{\tableauvariations}[1]{\tkzTabVar{#1}}

% Shortcut for the curve and the domain of the given function.
\newcommand{\graphe}[1]{\Gamma_{#1}}
\newcommand{\ensembledef}[1]{\mathcal{D}_{#1}}

\renewcommand{\S}[1]{\mathfrak{S}_{#1}}
\newcommand{\frakA}[1]{\mathfrak{A}_{#1}}

\newcommand{\semihrule}{\rule{256.074815pt}{0.4pt}}

% Various environments that create boxes. Each one is one type of thing (example, proof, etc). Each type has its own automatic counter.
\theoremstyle{break}
\theorembodyfont{\upshape}
\theoremheaderfont{\itshape}
\theoremprework{\bigskip\needspace{\baselineskip}\color{green}\hrule\color{black}}
\theorempostwork{\bigskip}
\newtheorem{rem}{Remarque}[chapter]

\theoremstyle{break}
\theorembodyfont{\upshape}
\theoremheaderfont{\itshape}
\theoremprework{\bigskip\needspace{\baselineskip}\color{green}\hrule\color{black}}
\theorempostwork{\bigskip}
\newtheorem{ex}[rem]{Exemple}

\theoremstyle{break}
\theorembodyfont{\upshape}
\theoremheaderfont{\itshape}
\theoremprework{\bigskip\needspace{\baselineskip}\color{green}\hrule\color{black}}
\theorempostwork{\bigskip}
\newtheorem{rappel}[rem]{Rappel}

\theoremstyle{break}
\theorembodyfont{\upshape}
\theoremheaderfont{\itshape}
\theoremprework{\bigskip\needspace{\baselineskip}\color{brown}\hrule\color{black}}
\theorempostwork{\bigskip}
\newtheorem{oubli}[rem]{Oubli}

\theoremstyle{break}
\theorembodyfont{\itshape}
\theoremheaderfont{\normalfont\bfseries}
\theoremprework{\bigskip\needspace{\baselineskip}\color{orange}\hrule\color{black}}
\theorempostwork{\bigskip}
\newtheorem{formu}[rem]{Formule}

\theoremstyle{break}
\theorembodyfont{\upshape}
\theoremheaderfont{\normalfont\bfseries}
\theoremprework{\bigskip\needspace{\baselineskip}\color{blue}\hrule\color{black}}
\theorempostwork{\bigskip}
\newtheorem{defi}[rem]{Définition}

\theoremstyle{break}
\theorembodyfont{\upshape}
\theoremheaderfont{\normalfont\bfseries}
\theoremprework{\bigskip\needspace{\baselineskip}\color{blue}\hrule\color{black}}
\theorempostwork{\bigskip}
\newtheorem{reform}[rem]{Reformulation}

\theoremstyle{break}
\theorembodyfont{\upshape}
\theoremheaderfont{\normalfont\bfseries}
\theoremprework{\bigskip\needspace{\baselineskip}\color{magenta}\hrule\color{black}}
\theorempostwork{\bigskip}
\newtheorem{exo}[rem]{Exercice}

\theoremstyle{break}
\theorembodyfont{\upshape}
\theoremheaderfont{\normalfont\bfseries}
\theoremprework{\bigskip\needspace{\baselineskip}\color{magenta}\semihrule\color{green}\semihrule\color{black}}
\theorempostwork{\bigskip}
\newtheorem{exoex}[rem]{Exercice/Exemple}

\theoremstyle{break}
\theorembodyfont{\upshape}
\theoremheaderfont{\normalfont\bfseries}
\theoremprework{\bigskip\needspace{\baselineskip}\color{blue}\semihrule\color{red}\semihrule\color{black}}
\theorempostwork{\bigskip}
\newtheorem{defprop}[rem]{Définition/Propriétés}

\theoremstyle{break}
\theorembodyfont{\upshape}
\theoremheaderfont{\normalfont\bfseries}
\theoremprework{\bigskip\needspace{\baselineskip}\color{blue}\semihrule\color{red}\semihrule\color{black}}
\theorempostwork{\bigskip}
\newtheorem{deftheo}[rem]{Définition/Théorème}

\theoremstyle{break}
\theorembodyfont{\upshape}
\theoremheaderfont{\normalfont\bfseries}
\theoremprework{\bigskip\needspace{\baselineskip}\color{blue}\hrule\color{black}}
\theorempostwork{\bigskip}
\newtheorem{nota}[rem]{Notation}

\theoremstyle{break}
\theorembodyfont{\upshape}
\theoremheaderfont{\itshape}
\theoremprework{\bigskip\needspace{\baselineskip}\color{blue}\hrule}
\theorempostwork{\hrule\color{black}\needspace{\baselineskip}\bigskip}
\newtheorem*{brouill}{Brouillon}

\theoremstyle{break}
\theorembodyfont{\itshape}
\theoremheaderfont{\normalfont\bfseries}
\theoremprework{\bigskip\needspace{\baselineskip}\color{red}\hrule\color{black}}
\theorempostwork{\bigskip}
\newtheorem{theo}[rem]{Théorème}

\theoremstyle{break}
\theorembodyfont{\upshape}
\theoremheaderfont{\normalfont\bfseries}
\theoremprework{\bigskip\needspace{\baselineskip}\color{red}\hrule\color{black}}
\theorempostwork{\bigskip}
\newtheorem{prop}[rem]{Propriétés}

\theoremstyle{break}
\theorembodyfont{\itshape}
\theoremheaderfont{\normalfont\bfseries}
\theoremprework{\bigskip\needspace{\baselineskip}\color{red}\hrule\color{black}}
\theorempostwork{\bigskip}
\newtheorem{cor}[rem]{Corollaire}

\theoremstyle{break}
\theorembodyfont{\itshape}
\theoremheaderfont{\normalfont\bfseries}
\theoremprework{\bigskip\needspace{\baselineskip}\color{red}\hrule\color{black}}
\theorempostwork{\bigskip}
\newtheorem{lem}[rem]{Lemme}

\theoremstyle{break}
\theorembodyfont{\upshape}
\theoremheaderfont{\normalfont\bfseries}
\theoremprework{\bigskip\needspace{\baselineskip}\color{violet}\hrule\color{black}}
\theorempostwork{\bigskip}
\newtheorem{meth}[rem]{Méthode}

\theoremstyle{break}
\theorembodyfont{\upshape}
\theoremheaderfont{\normalfont\bfseries}
\theoremprework{\bigskip\needspace{\baselineskip}\color{violet}\hrule\color{black}}
\theorempostwork{\bigskip}
\newtheorem{appl}[rem]{Application}

\theoremstyle{break}
\theorembodyfont{\upshape}
\theoremheaderfont{\normalfont\bfseries}
\theoremprework{\bigskip\needspace{\baselineskip}\color{violet}\hrule\color{black}}
\theorempostwork{\bigskip}
\newtheorem{abus}[rem]{Abus}

\theoremstyle{break}
\theorembodyfont{\upshape}
\theoremheaderfont{\normalfont\bfseries}
\theoremprework{\bigskip\needspace{\baselineskip}\color{violet}\hrule\color{black}}
\theorempostwork{\bigskip}
\newtheorem{algo}[rem]{Algorithme}

\theoremstyle{break}
\theorembodyfont{\upshape}
\theoremheaderfont{\normalfont\bfseries}
\theoremprework{\bigskip\needspace{\baselineskip}\color{violet}\hrule\color{black}}
\theorempostwork{\bigskip}
\newtheorem{bilan}[rem]{Bilan}

\theoremstyle{break}
\theorembodyfont{\upshape}
\theoremheaderfont{\itshape}
\theoremprework{\bigskip\needspace{\baselineskip}\color{BurntOrange}\hrule\color{black}}
\theorempostwork{\bigskip}
\newtheorem{corr}[rem]{Correction}

\theoremstyle{break}
\theorembodyfont{\upshape}
\theoremheaderfont{\itshape}
\theoremsymbol{\ensuremath{\cqfd}}
\theoremprework{\bigskip\needspace{\baselineskip}\color{yellow}\hrule\color{black}}
\theorempostwork{\bigskip}
\newtheorem*{dem}{Démonstration}

% Numérotation des environnements par sous-section
\numberwithin{rem}{subsection}
\renewcommand{\thesection}{\Roman{section}} % section en romain majuscule
\renewcommand{\thesubsection}{\thesection.\arabic{subsection}}
% Affichage du compteur : chapitre.section.soussection.num
\renewcommand{\therem}{\Roman{section}.\arabic{subsection}.\arabic{rem}}

% Commands to make proofs easier to write
\newcommand{\impdir}{\fbox{\(\imp\)}~}
\newcommand{\imprec}{\fbox{\(\impr\)}~}
\newcommand{\incdir}{\fbox{\(\subset\)}~}
\newcommand{\increc}{\fbox{\(\supset\)}~}
\newcommand{\leqbox}{\fbox{\(\leq\)}~}
\newcommand{\geqbox}{\fbox{\(\geq\)}~}
\newcommand{\unicite}{\fbox{unicité}~}
\newcommand{\existence}{\fbox{existence}~}
\newcommand{\analyse}{\fbox{analyse}~}
\newcommand{\synthese}{\fbox{synthèse}~}
\newcommand{\conclusion}{\fbox{conclusion}~}
\newcommand{\wt}[1]{\widetilde{#1}}

\renewcommand{\to}{\longrightarrow}
\renewcommand{\mapsto}{\longmapsto}

\newcommand{\fonction}[5]{\begin{array}[t]{cccc}#1 : & #2 & \to & #3 \\ & #4 & \mapsto & #5\end{array}}
\newcommand{\fonctionlambda}[4]{\begin{array}[t]{ccc}#1 & \to & #2 \\ #3 & \mapsto & #4\end{array}}

\renewcommand{\leq}{\leqslant}
\renewcommand{\geq}{\geqslant}

\newcommand{\pinf}{+\infty}
\newcommand{\minf}{-\infty}

\newcommand{\id}[1]{\mathrm{Id}_{#1}}

\renewcommand{\phi}{\varphi}
\renewcommand{\epsilon}{\varepsilon}

\newcommand{\ind}[1]{\mathds{1}_{#1}}

\newcommand{\iR}{\i\R}

\newcommand{\tcheby}[2]{T_{#1}\paren{#2}}
\newcommand{\utcheby}[2]{U_{#1}\paren{#2}}

\mathcode`l="8000
\begingroup
\makeatletter
\lccode`\~=`\l
\DeclareMathSymbol{\lsb@l}{\mathalpha}{letters}{`l}
\lowercase{\gdef~{\ifnum\the\mathgroup=\m@ne \ell \else \lsb@l \fi}}%
\endgroup

\newcommand{\ensvide}{\varnothing}

\newcommand{\rond}{\circ}

\newcommand{\union}{\cup}
\newcommand{\inter}{\cap}
\newcommand{\bigunion}{\bigcup}
\newcommand{\biginter}{\bigcap}

\newcommand{\imp}{\implies}
\newcommand{\impr}{\impliedby}

\newcommand{\excluant}{\setminus}

\newcommand{\littletaller}{\mathchoice{\vphantom{\big|}}{}{}{}}
\newcommand{\restr}[2]{{
\left.\kern-\nulldelimiterspace#1\littletaller\right|_{#2}
}}
\newcommand{\corestr}[2]{{
\left.\kern-\nulldelimiterspace#1\littletaller\right|^{#2}
}}
\newcommand{\restrbar}[1]{{
\left.\kern-\nulldelimiterspace#1\littletaller\right|
}}

\newcommand{\rel}{\mathscr{R}}

\newcommand{\classesdequiv}[1]{\nicefrac{#1}{\sim}}

\newcommand{\majo}[1]{\mathrm{majorants}\paren{#1}}
\newcommand{\mino}[1]{\mathrm{minorants}\paren{#1}}

\newcommand{\ensdiv}[1]{\operatorname{div}\paren{#1}}

\newcommand{\E}[1]{\times 10^{#1}}

\setcounter{secnumdepth}{3}

\newcommand{\guillemets}[1]{\og #1 \fg{}}

\newcommand{\prim}{^{\,\prime}}
\newcommand{\seconde}{^{\,\prime\prime}}

\newcommand{\note}[1]{\textbf{\(\star\star\) #1 \(\star\star\)}}
\newcommand{\cad}{c'est-à-dire }
\newcommand{\Cad}{C'est-à-dire }
\newcommand{\ssi}{si, et seulement si, }
\newcommand{\Ssi}{Si, et seulement si, }
\newcommand{\ie}{\textit{i.e.} }
\newcommand{\cf}{\textit{cf.} }
\newcommand{\Cf}{\textit{Cf.} }

\usepackage{xparse}

\NewDocumentCommand{\quantifs}{>{\SplitList{;}}m}{\ProcessList{#1}{\insertquantif}}
\newcommand{\insertquantif}[1]{#1,\;\:}

\DeclareDocumentCommand{\groupe}{m O{+}}{\paren{#1,#2}}
\DeclareDocumentCommand{\anneau}{m O{+} O{\times}}{\paren{#1,#2,#3}}
\DeclareDocumentCommand{\corps}{m O{+} O{\times}}{\paren{#1,#2,#3}}

\DeclareDocumentCommand{\poly}{O{\K} O{X}}{#1\croch{#2}}
\DeclareDocumentCommand{\polydeg}{O{\K} m O{X}}{#1_{#2}\croch{#3}}
\DeclareDocumentCommand{\fracrat}{O{\K} O{X}}{#1\paren{#2}}

\DeclareDocumentCommand{\M}{m O{\K}}{\mathcal{M}_{#1}\paren{#2}}
\DeclareDocumentCommand{\sym}{m O{\K}}{\mathcal{S}_{#1}\paren{#2}}
\DeclareDocumentCommand{\antisym}{m O{\K}}{\mathcal{A}_{#1}\paren{#2}}
\DeclareDocumentCommand{\GL}{m O{\K}}{\operatorname{\cal{GL}}_{#1}\paren{#2}}
\DeclareDocumentCommand{\SL}{m O{\K}}{\operatorname{SL}_{#1}\paren{#2}}
\DeclareDocumentCommand{\Mat}{m m}{\underset{#1}{\operatorname{Mat}}\paren{#2}}
\newcommand{\pass}[2]{\operatorname{Pass}\paren{#1\to#2}}

\DeclareDocumentCommand{\contm}{O{\intervii{a}{b}} O{\K}}{\classe{0}_m\paren{#1,#2}}
\DeclareDocumentCommand{\Esc}{O{\intervii{a}{b}} O{\K}}{\operatorname{Esc}\paren{#1,#2}}

\usepackage{witharrows}

\newcommand{\croix}{^{\times}}

\usepackage{polynom}

\newcommand{\classe}[1]{\mathscr{C}^{#1}}
\newcommand{\ensclasse}[3]{\classe{#1}\paren{#2,#3}}

\newcommand{\deriv}[1]{^{\paren{#1}}}

\usepackage{derivative}
\derivset{\pdv}[delims-eval=.)]

\DeclareMathOperator{\Arctan}{Arctan}
\DeclareMathOperator{\Arcsin}{Arcsin}
\DeclareMathOperator{\Arccos}{Arccos}
\DeclareMathOperator{\cotan}{cotan}
\DeclareMathOperator{\sh}{sh}
\DeclareMathOperator{\ch}{ch}
\DeclareMathOperator{\tth}{th} %we can't use \th since it's a imbed latex command
\DeclareMathOperator{\sg}{sg}
\DeclareMathOperator{\supp}{supp}
\DeclareMathOperator{\Supp}{Supp}
\DeclareMathOperator{\rg}{rg}
\DeclareMathOperator{\tr}{tr}

\newcommand{\Hom}[2]{\operatorname{Hom}\paren{#1,#2}}
\newcommand{\Pol}[2]{\operatorname{Pol}\paren{#1,#2}}
\newcommand{\Aut}[1]{\operatorname{Aut}\paren{#1}}
\DeclareDocumentCommand{\Vect}{O{} m}{\operatorname{Vect}_{#1}\paren{#2}}

\newcommand{\diag}[1]{\operatorname{diag}\paren{#1}}

\usepackage{abstract}
\addto\captionsfrench{\renewcommand{\abstractname}{\Large Introduction}}

\newcommand{\inv}{^{-1}}
\newcommand{\etoile}{^{*}}

\newcounter{orcounter}

\newenvironment{orlist}
{
\begin{array}{|l}
\setcounter{orcounter}{0}
}
{
\end{array}
}

\newcommand{\oritem}[1]{%
\ifthenelse{\theorcounter<1}{}{\\ \text{ou} \\}#1\stepcounter{orcounter}
}

\NewDocumentCommand{\orenv}{>{\SplitList{\\}}m}{%
\begin{orlist}\ProcessList{#1}{\oritem}\end{orlist}}

\newcounter{permuitemcounter}

\newcommand{\permuitem}[1]{%
\ifthenelse{\thepermuitemcounter<1}{}{&}#1\stepcounter{permuitemcounter}}

\NewDocumentCommand{\permu}{>{\SplitList{;}}m >{\SplitList{;}}m}{%
\begin{pmatrix}\setcounter{permuitemcounter}{0}\ProcessList{#1}{\permuitem} \\ \setcounter{permuitemcounter}{0}\ProcessList{#2}{\permuitem}\end{pmatrix}}

\NewDocumentCommand{\cycle}{>{\SplitList{;}}m}{%
\begin{pmatrix}\setcounter{permuitemcounter}{0}\ProcessList{#1}{\permuitem}\end{pmatrix}}

\usepackage{pgfplots}

\DeclareDocumentCommand{\pgcd}{o o}{
\IfNoValueTF{#1}{\operatorname{pgcd}}{\operatorname{pgcd}\paren{#1,#2}}
}

\DeclareDocumentCommand{\bezout}{o o}{
\IfNoValueTF{#1}{\operatorname{bezout}}{\operatorname{bezout}\paren{#1,#2}}
}


\newcommand{\valp}[2]{v_{#1}\paren{#2}}

\newcommand{\fami}[1]{\mathscr{#1}}

\newcommand{\echange}{\leftrightarrow}

\newcommand{\trans}[1]{#1^{\top}}

\usepackage{mathdots}

\DeclareDocumentCommand{\detb}{O{\fami{B}}}{{\det}_{#1}}

\usepackage{cancel}

\usepackage{nicematrix}

\newcommand{\ps}[2]{\left\langle#1\tq#2\right\rangle}
\newcommand{\ortho}{^{\perp}}

\newcommand{\operp}{\mathrel{%
\begin{tikzpicture}[baseline=-0.25em]
\draw (0,0) circle (0.45em);
\draw (-0.38em,-0.25em) -- (0.38em,-0.25em);
\draw (0,-0.25em) -- (0,0.45em);
\end{tikzpicture}
}%
}

\usepackage{titletoc}
\dottedcontents{section}[5.5em]{}{3.2em}{1pc}

\newcommand{\bouleo}[2]{\mathbb{B}\paren{#1,#2}}
\newcommand{\boulef}[2]{\mathbb{B}\prim\paren{#1,#2}}
\newcommand{\sphere}[2]{\mathbb{S}\paren{#1,#2}}

\newcommand{\vdv}[2]{\operatorname{D}_{#1}#2}

\newcommand{\egqd}[1]{\underset{#1}{=}}
\newcommand{\simqd}[1]{\underset{#1}{\sim}}

\newcommand{\mediumrightarrow}{\,\begin{tikzpicture}\draw[->] (0, 0) -- (1, 0);\end{tikzpicture}\,}
\newcommand{\tendqd}[1]{\underset{#1}{\mediumrightarrow}}

\newcommand{\arr}[2]{A_{#2}^{#1}}
\newcommand{\comb}[2]{C_{#2}^{#1}}

\newcommand{\loiuniforme}[1]{\mathscr{U}\paren{#1}}
\newcommand{\loibernoulli}[1]{\mathscr{B}\paren{#1}}
\newcommand{\loibinomiale}[2]{\mathscr{B}\paren{#1,#2}}

\newcommand{\esp}[1]{\operatorname{E}\paren{#1}}
\newcommand{\vari}[1]{\operatorname{V}\paren{#1}}
\newcommand{\cov}[2]{\operatorname{Cov}\paren{#1,#2}}
\newcommand{\ecarttype}[1]{\sigma\paren{#1}}

\renewcommand{\O}[1]{\mathscr{O}\paren{#1}}
\renewcommand{\o}[1]{o\paren{#1}}

\setcounter{MaxMatrixCols}{200}

\newcommand{\Com}[1]{\operatorname{Com}#1}

\usepackage{microtype}

\newcommand{\sig}[1]{\epsilon\paren{#1}}

\ExplSyntaxOn
\RenewDocumentCommand{\v}{m}{
    \int_compare:nTF { \tl_count:n { #1 } > 1 }
    {
        \overrightarrow{#1}
    }
    {
        \vec{#1}
    }
}
\ExplSyntaxOff

\newcommand{\nombredeligne}{12062}


\begin{document}
\renewcommand{\labelitemi}{\(\bullet\)}
\renewcommand{\labelenumi}{(\arabic{enumi})}

\everymath{\ds}

\maketitle

\begin{abstract}
	Ce document réunit l'ensemble de mes cours de Mathématiques de MP2I, ainsi que les TDs (travaux dirigés) les accompagnant. J'ai adapté certaines formulations me paraissant floues ou ne me plaisant pas mais le contenu pur des cours est strictement équivalent. Le document est organisé selon la hiérarchie suivante : chapitre, I), 1), a).

	Les éléments des tables des matières initiale et présentes au début de chaque chapitre sont cliquables (amenant directement à la partie cliquée). C'est également le cas des références à des éléments antérieurs de la forme, par exemple, \guillemets{Démonstration 5.22}.

	Dernier TD corrigé : aucun.

    Le nombre total de lignes de latex utilisé pour générer tout ce document est : \nombredeligne.
    % se génère en lançant : echo "\\newcommand{\\nombredeligne}{$(wc -l chapitre/chapitre*/chapitre*.tex | awk '{total += $1} END {print total}')}" > nblignes.tex
    % dans son terminal

\end{abstract}

\dominitoc\tableofcontents

\part{Cours}

\chapter{trigonométrie (Rappels et compléments)}

\minitoc

Dans ce chapitre, on rappelle ce qui a été vu en trigonométrie au lycée et on complète avec les formules
d’addition et de duplication ainsi que l’étude de la fonction tangente.

\section{Cercle trigonométrique}

On se place dans le plan muni d'un repère orthonormé \(\paren{O,\vec{i},\vec{j}}\)

\begin{defi}[Cercle trigonométrique]

	On appelle cercle trigonométrique le cercle de centre \(O\) et de rayon \(1\)

\end{defi}

\begin{prop}[enroulement de la droite des réels sur le cercle trigonométrique]
	Soit \(M\) un point du plan. \\
	Le point \(M\) appartient au cercle trigonométrique si, et seulement si, il existe un réel \(t\) tel que les coordonnées de \(M\) dans le repère orthonormé \(\paren{O,\vec{i},\vec{j}}\) sont \(\paren{\cos t ; \sin t}\)
\end{prop}

\subsection{Relation de congruence modulo \(2\pi\) sur \(\R\)}

\begin{defi}
	Deux réels \(a\) et \(b\) sont dits congrus modulo \(2\pi\) s'il existe un entier relatif \(k\) tel que \(a-b = 2k\pi\)
	\underline{Notation} : \(a \equiv b \croch{2 \pi} \)
\end{defi}

\begin{defprop}
	On dit que la relation \(\equiv\) est une relation d'équivalence sur \(\R\) car elle vérifie les propriétés suivantes :
	\begin{enumerate}
		\item Pour tout réel x, on a : \(x \equiv x \croch{2 \pi}\). \hfill (réfléxivité)
		\item Pour tout couple de réels \(\paren{x,y}\) tel que \( x \equiv y \croch{2 \pi} \), on a :\( y \equiv x \croch{2 \pi} \) \hfill (symétrie)
		\item Pour tout triplet de réels \(\paren{x,y,z}\) tel que \(x \equiv y \croch{2 \pi} \) et \( y \equiv z \croch{2 \pi} \), on a : \( x \equiv z \croch{2 \pi} \) \hfill (transitivité)
	\end{enumerate}
\end{defprop}



\section{Cosinus et sinus}
\subsection{Formules et valeur remarquables}

\begin{formu}[Formule de base]
	Pour tout réel \(t\), on a :
	\begin{enumerate}
		\item \( \cos\paren{\pi - t} = -\cos t \) et \( \sin\paren{\pi - t} = \sin t \) \\
		\item \( \cos\paren{\pi + t} = -\cos t \) et \( \sin\paren{\pi + t} = -\sin t \) \\
		\item \( \cos\paren{\frac{\pi}{2} - t} = \sin t \) et \( \sin\paren{\frac{\pi}{2} - t} = \cos t \) \\
		\item \( \cos\paren{\frac{\pi}{2} + t} = -\sin t \) et \( \sin\paren{\frac{\pi}{2} + t} = \cos t \) \\
	\end{enumerate}
	\begin{tabular}{|c|c|c|c|c|c|}

		\hline
		\(t\)       & \(0\) & \(\frac{\pi}{6}\)      & \(\frac{\pi}{4}\)      & \(\frac{\pi}{3}\)       & \(\frac{\pi}{2}\) \\
		\hline
		\(\cos t \) & \(1\) & \(\frac{\sqrt{3}}{2}\) & \(\frac{\sqrt{2}}{2}\) & \(\frac{1}{2}\)        & \(0\)             \\
		\hline
		\(\sin t \) & \(0\) & \(\frac{1}{2}\)        & \(\frac{\sqrt{2}}{2}\) & \(\frac{\sqrt{3}}{2}\) & \(1\)             \\
		\hline
	\end{tabular}
\end{formu}

\begin{rem}
	Soient \(a\) et \(b\) des réels :
	\begin{itemize}
		\item
		      \(
		      \begin{aligned}
			      \cos a = \cos b
			      \iff\
			       & \left\{
			      \begin{aligned}
				      a & \equiv b \croch{2\pi}  \\
				        & \text{ou}              \\
				      a & \equiv -b \croch{2\pi}
			      \end{aligned}
			      \right.
			      \iff\
			       & \left\{
			      \begin{aligned}
				      \quantifs{\exists k \in \Z}  & a = b +2 k \pi   \\
				                                   & \text{ou}        \\
				      \quantifs{\exists k' \in \Z} & a = -b +2 k' \pi
			      \end{aligned}
			      \right.
		      \end{aligned}
		      \)\\
		      \item\(
		      \begin{aligned}
			      \sin a = \sin b
			      \iff\
			       & \left\{
			      \begin{aligned}
				      a & \equiv b \croch{2\pi}     \\
				        & \text{ou}                 \\
				      a & \equiv \pi-b \croch{2\pi}
			      \end{aligned}
			      \right.
			      \iff\
			       & \left\{
			      \begin{aligned}
				      \quantifs{\exists k \in \Z}  & a = b +2 k \pi      \\
				                                   & \text{ou}           \\
				      \quantifs{\exists k' \in \Z} & a = \pi-b +2 k' \pi
			      \end{aligned}
			      \right.
		      \end{aligned}
		      \)
	\end{itemize}

\end{rem}

\begin{formu} [Formule d'addition]
	Pour tout couple de réels \(\paren{a,b}\) on a :
	\begin{enumerate}
		\item \( \cos\paren{a+b} = \cos\paren{a} \cos\paren{b} - \sin\paren{a} \sin\paren{b} \) \\
		\item \( \cos\paren{a-b} = \cos\paren{a} \cos\paren{b} + \sin\paren{a} \sin\paren{b} \)\\
		\item \( \sin\paren{a+b} = \sin\paren{a} \cos\paren{b} + \cos\paren{a} \sin\paren{b} \) \\
		\item \( \sin\paren{a-b} = \sin\paren{a} \cos\paren{b} - \cos\paren{a} \sin\paren{b} \)\\
	\end{enumerate}
\end{formu}

\begin{formu}[Formule de simpson]
	Pour tout couple de réels \(\paren{a,b}\) on a :
	\begin{enumerate}
		\item \( \sin\paren{a+b} + \sin\paren{a-b} = 2\sin\paren{a} \cos\paren{b} \iff \frac{1}{2}\paren{\sin\paren{a+b} + \sin\paren{a-b}} = \sin\paren{a} \cos\paren{b}\) \\
		\item \( \cos\paren{a+b} + \cos\paren{a-b} = 2\cos\paren{a} \cos\paren{b} \iff \frac{1}{2}\paren{\cos\paren{a+b} + \cos\paren{a-b}} = \cos\paren{a} \cos\paren{b}\)

	\end{enumerate}

\end{formu}

\begin{appl}
	Calcul : \[\int_{0}^{\pi} \sin\paren{x} \cos\paren{3x} dx = \int{0}^{\pi} \frac{1}{2} \paren{\sin\paren{4x}+\sin(2x) dx} = 0\]
\end{appl}

\begin{formu}[Formule de duplication]
	Pour tout réel \(a\), on a :
	\begin{enumerate}
		\item \(\cos\paren{2a} = \cos^2\paren{a} - \sin^2\paren{a} = 2\cos^2(a)-1 = 1-\sin^2(a) \)
		\item \(\sin(2a) = 2\cos(a)\sin(a) \)
	\end{enumerate}
\end{formu}

\begin{prop}[Sinus et Cosinus]
	\begin{itemize}
		\item La fonction \(\cos\) est définie sur \(\R\), paire et périodique de période \(2\pi\). Elle est dérivable sur \(\R\) et sa dérivée vérifie \(\cos' = -\sin\)
		\item La fonction \(\sin\) est définie sur \(\R\), impaire et périodique de période \(2\pi\). Elle est dérivable sur \(\R\) et sa dérivée vérifie \(\sin' = \cos\)
	\end{itemize}
\end{prop}

\begin{prop}[Inégalité remarquable]
	Pour tout réel \(t\), on a : \(\abs{\sin(t)} \leq \abs{t}\)
\end{prop}

\section{La fonction tangente}
\begin{defi}
	La fonction \(\frac{\sin}{\cos} \) est appelée la fonction tangente et notée \(\tan\)
\end{defi}

\begin{prop}
	La fonction \(\tan\) est définie sur \(\R\backslash\accol{\frac{\pi}{2}+k\pi\tq k\in \Z}\), impaire et périodique de période \(\pi\). Elle est dérivable sur \(\R\)\(\R\backslash\accol{\frac{\pi}{2}+k\pi\tq k\in \Z}\) et sa dérivée vérifie \(\tan' = 1+\tan = \frac{1}{tan^2}\)
\end{prop}

\begin{formu}
	Pour tout réel \(t\), on a :
	\begin{enumerate}
        \item \(tan(\pi-t) = -\tan(t)\)
        \item \(tan(\pi+t) = \tan(t) \)
        \item \begin{tabular}{|c|c|c|c|c|c|}

		\hline
		\(t\)       & \(0\) & \(\frac{\pi}{6}\)      & \(\frac{\pi}{4}\)      & \(\frac{\pi}{3}\)       & \(\frac{\pi}{2}\) \\
		\hline
		\(\tan t \) & \(0\) & \(\frac{1}{\sqrt{3}}\) & \(1\) & \(\sqrt{3}\)       & NULL             \\
		\hline
	\end{tabular}
    \end{enumerate}
\end{formu}

\begin{formu}[addition et duplication]
    Pour tout couple de réels \(\paren{a,b}\) n'appartenant pas à l'ensemble \(\accol{\frac{\pi}{2}+k\pi\tq k\in \Z}\), on a :
    \begin{enumerate}
        \item Si \(a+b\) n'appartient pas à l'ensemble \(\accol{\frac{\pi}{2}+k\pi\tq k\in \Z}\) alors \(\tan(a+b) = \frac{\tan(a)+\tan(b)}{1-\tan(a) \tan(b)}\)
        \item Si \(a-b\) n'appartient pas à l'ensemble \(\accol{\frac{\pi}{2}+k\pi\tq k\in \Z}\) alors \(\tan(a-b) = \frac{\tan(a)-\tan(b)}{1+\tan(a) \tan(b)}\)
        \item Si \(2a\)  n'appartient pas à l'ensemble \(\accol{\frac{\pi}{2}+k\pi\tq k\in \Z}\) alors \(\tan(2a) = \frac{2\tan(a)}{1-\tan^2(a)} \)
    \end{enumerate}
\end{formu}

\begin{exoex}
Soit \(t\) réel n'appartenant pas à \(\accol{\frac{\pi}{4}+k\frac{\pi}{2}\tq k\in \Z}\) : 
    \begin{align*}
        \sin(t) &= 2\sin\paren{\frac{t}{2}}\cos\paren{\frac{t}{2}} \\
        &= \frac{2\sin\paren{\frac{t}{2}}}{\cos\paren{\frac{t}{2}}}\cos^2\paren{\frac{t}{2}}\\
        &= \frac{1}{1+\tan^2\paren{\frac{t}{2}}}\times 2 \tan\paren{\frac{t}{2}} \\
        &=\frac{2 \tan\paren{\frac{t}{2}}}{1+\tan^2\paren{\frac{t}{2}}} 
    \end{align*}
\end{exoex}

\chapter{Inégalité et fonction (rappel et compléments)}

\minitoc

Dans ce chapitre, sont rassemblés des rappels ou compléments sur les inégalités ainsi que des fondamentaux sur les fonctions de variable réelle à valeurs réelles (sans preuve ni évocation de continuité).

\section{Inégalité}

\subsection{Relation d'ordre sur \(\R\)}

\begin{defi}
	On dit que la relation \(\leq\) est une relation d'équivalence sur \(\R\) car elle vérifie les propriétés suivantes :
	\begin{enumerate}
		\item Pour tout réel x, on a : \(x \leq x \). \hfill (réfléxivité)
		\item Pour tout couple de réels \(\paren{x,y}\) tel que \( x \leq y  \) et \(y \leq x\), on a :\( y = x  \) \hfill (antisymétrie)
		\item Pour tout triplet de réels \(\paren{x,y,z}\) tel que \(x \leq y  \) et \( y \leq z  \), on a : \( x \leq z  \) \hfill (transitivité)
	\end{enumerate}
\end{defi}

\begin{prop}[Compatibilité avec les opérations]
	Soit \(x,y,z,t\) et \(a\) des réels.
	\begin{enumerate}
		\item Si \(x\leq y\) et \(z\leq t\) alors \(x+z\leq y +t \)
		\item Si \(x\leq y \) et \( 0 \leq a\) alors \(a x \leq a y\)
		\item Si \(x\leq y \) et \( a \leq 0\) alors \(a y \leq a x\)
		\item Si \( 0 \leq x \leq y \) et \( 0\leq z \leq t \) alors \( 0 \leq xz \leq y t \)
	\end{enumerate}
\end{prop}

\begin{nota}[Intervalles de \(\R\)]
	Les partie \(I\) de \(\R\) pouvant s’écrire sous l’une des formes suivantes sont dites intervalles de \(\R\) :
	\begin{itemize}
		\item \(I = \emptyset\) \\
		\item \(I = \accol{x \in \R\tq a \leq x \leq b} \underset{\mathrm{notation}}{=} \intervii{a}{b}\) avec \(\paren{a,b} \in \R^2 \) et \(a\leq b \) \\
		\item \(I = \accol{x \in \R\tq a \leq x < b} \underset{\mathrm{notation}}{=} \intervie{a}{b}\) avec \(\paren{a,b} \in \R\times \paren{\R \union \accol{\pinf}} \) et \(a < b\) \\
		\item \(I = \accol{x \in \R\tq a < x \leq b} \underset{\mathrm{notation}}{=} \intervei{a}{b}\) avec \(\paren{a,b} \in \paren{\R \union \accol{\minf}}\times \R \) et \(a < b\) \\
		\item \(I = \accol{x \in \R\tq a < x \leq b} \underset{\mathrm{notation}}{=} \intervee{a}{b}\) avec \(\paren{a,b} \in \paren{\R \union \accol{\minf}}\times  \paren{\R \union \accol{\pinf}} \) et \(a < b\) \\

	\end{itemize}
\end{nota}

\begin{prop}
	\begin{enumerate}
		\item Passage à l'inverse dans une inégalité
		      \[\quantifs{\forall x \in \Rps ; \forall y \in \Rps} x\leq y \iff \frac{1}{y} \leq \frac{1}{x}\]
		      \[\quantifs{\forall x \in \Rms ; \forall y \in \Rms} x\leq y \iff \frac{1}{y} \leq \frac{1}{x}\] \\
		\item Passage au carré dans une inégalité
		      \[\quantifs{\forall x \in \Rps ; \forall y \in \Rps} x\leq y \iff x^2 \leq y^2\]
		      \[\quantifs{\forall x \in \Rms ; \forall y \in \Rms} x\leq y \iff y^2 \leq x^2\] \\
		\item Passage à la racine carrée dans une inégalité
		      \[\quantifs{\forall x \in \Rp ; \forall y \in \Rp} x\leq y \iff \sqrt{x}\leq \sqrt{y}\] \\
		\item Passage à l’exponentielle ou au logarithme népérien dans une inégalité
		      \[\quantifs{\forall x \in \R ; \forall y \in \R} x\leq y \iff \e{x}\leq \e{y}\]
		      \[\quantifs{\forall x \in \Rps ; \forall y \in \Rps} x\leq y \iff \ln{x}\leq \ln{y}\] \\
	\end{enumerate}
\end{prop}

\begin{exoex}
	Montrer \(\quantifs{\forall x \in \intervii{0}{1}} x(1-x) \leq \frac{1}{4}\).
\end{exoex}

\begin{corr}[2 Méthode]
	Soit \(x \in \intervii{0}{1} \)
	\begin{enumerate}
		\item Raisonnement par équivalence
		      \[\begin{aligned}
				      x(1-x) \leq \frac{1}{4} & \iff 0 \leq \frac{1}{4}-x(1-x)     \\
				                              & \iff 0\leq x^2 -x +  \frac{1}{4}   \\
				                              & \iff 0\leq\paren{x- \frac{1}{2}}^2
			      \end{aligned}
		      \]
		      Ceci étant vrai \(\quantifs{\forall x\in \intervii{0}{1}}\) car \(\Delta = 0\) et \(x_0 =  \frac{1}{2}\), on conclut \(\quantifs{\forall x \in \intervii{0}{1}} x(1-x) \leq \frac{1}{4}\).\\
		\item étude de la fonction \(\fonction{f}{\intervii{0}{1}}{\R}{x}{\frac{1}{4}-x(1-x)}\)\\
	\end{enumerate}
\end{corr}


\begin{exoex}
    ~\\
	Montrer \(\quantifs{\forall x \in \Rps} x+\frac{1}{x}\geq 2\).
\end{exoex}

\begin{corr}
	Soit \(x \in \Rps \)

	\[\begin{aligned}
			x+\frac{1}{x}\geq 2 & \iff \frac{x^2+1}{x}\geq 2 \\
			                    & \iff x^2-2x+1\geq    0     \\
			                    & \iff (x-1)^2 \geq 0
		\end{aligned}
	\]
	Ceci étant vrai \(\quantifs{\forall x\in \Rps}\) on conclut \(\quantifs{\forall x \in \Rps} x+\frac{1}{x}\geq 2\).
\end{corr}

\begin{exoex}
    ~\\
	Encadrer \(\frac{2x^2-x+1}{x^2+\sqrt{x+2}+3}\) pour \(x \in \intervii{-1}{1}\).
\end{exoex}

\begin{corr}
	Soit \(x \in \intervii{-1}{1} \)
	\begin{enumerate}
		\item \underline{numérateur} :
		      \[\begin{aligned}
				      -1 \leq x\leq 1 & \iff 0 \leq x^2 \leq 1      \\
				                      & \iff 0 \leq 2x^2 \leq 2     \\
				                      & \iff 0 \leq 2x^2-x+1 \leq 4
			      \end{aligned}
		      \]

		\item \underline{denominateur} : \[\begin{aligned}
				      -1 \leq x\leq 1 & \iff 0 \leq x^2 \leq 1                                                      \\
				                      & \iff 4 \leq x^2 +\sqrt{x+2}+3 \leq 4+\sqrt{3}                               \\
				                      & \iff \frac{1}{4+\sqrt{3}} \leq \frac{1}{x^2 +\sqrt{x+2}+3 }\leq \frac{1}{4} \\
			      \end{aligned}
		      \]
	\end{enumerate}
	Ainsi par produit des deux inégalités on as \(0\leq\frac{2x^2-x+1}{x^2+\sqrt{x+2}+3}\leq1\) pour \(x \in \intervii{-1}{1}\).
\end{corr}

\begin{exoex}
    ~\\
	Encadrer \(\frac{x-y^2+3}{x^2+y^2-y}\) pour \(\forall \paren{x,y} \in \intervii{1}{2}^2\).
\end{exoex}

\begin{corr}
	Soit \(x \in \intervii{-1}{1} \)
	\begin{enumerate}
		\item \underline{numérateur} :
		      \[\begin{aligned}
				      1-4+3\leq x-y^2+3 \leq 2-1+4 & \iff 0 \leq x-y^2+3 \leq 5
			      \end{aligned}
		      \]

		\item \underline{denominateur} : \[\begin{aligned}
				      0 \leq y-1\leq 1 & \iff 0 \leq y^2-y \leq y                        \\
				                       & \iff 0 \leq y^2-y \leq 2                        \\
				                       & \iff 1 \leq x^2+y^2-y\leq 6                     \\
				                       & \iff \frac{1}{6} \leq \frac{1}{x^2+y^2-y}\leq 1 \\
			      \end{aligned}
		      \]
	\end{enumerate}
	Ainsi par produit des deux inégalités on as \(0\leq \frac{x-y^2+3}{x^2+y^2-y} \leq 5\) pour \(\forall \paren{x,y} \in \intervii{1}{2}^2\).
\end{corr}

\begin{defi}[Parties majorées, majorants, maximum]
	Une partie \(A\) de \(\R\) est dite majorée s’il existe un réel \(M\) tel que, pour tout réel \(x\) de \(A\), on a : \(x \leq M\). \\
	Un tel réel \(M\) est alors dit :
	\begin{itemize}
		\item majorant de \(A\) dans le cas général. \\
		\item maximum de \(A\) dans le cas particulier où \(M\) appartient à \(A\).\\
	\end{itemize}

\end{defi}

\begin{defi}[Parties minorées, minorants, minimum]
	Une partie \(A\) de \(\R\) est dite minorée s’il existe un réel \(m\) tel que, pour tout réel \(x\) de \(A\), on a : \(m\leq x\). \\
	Un tel réel \(m\) est alors dit :
	\begin{itemize}
		\item minorant  de \(A\) dans le cas général. \\
		\item minimum  de \(A\) dans le cas particulier où \(m\) appartient à \(A\).\\
	\end{itemize}

\end{defi}

\begin{exoex}
    ~\\
	Que dire de \(B = \accol{\frac{n}{n^2+1} \tq n \in \N}\) ?
\end{exoex}

\begin{corr}
	\begin{itemize}

		\item \(B\) est minorée car \( \quantifs{\forall n \in \N} 0 \leq \frac{n}{n^2+1} \) par ailleurs \(0 \in B\) donc \(0\) est un minimum. \\
		\item \(B\) est majorée par \(\frac{1}{2}\). En effent en notant \(U_n = \frac{n}{n^2+1}\), On voit que \((U_n)\) est strictement décroissante
	\end{itemize}
\end{corr}

\begin{exoex}
        ~\\
	Que dire de \(C = \accol{\frac{\e{x}}{x} \tq x \in \Rps}\) ?
\end{exoex}

\begin{corr}
	\begin{itemize}

		\item \(C\) est minorée car \( \quantifs{\forall x \in \Rps} 0 \leq \frac{\e{x}}{x} \) donc \(0\) est un minorant mais pas un minimum  \\
		\item Supposons que \(C\) est majorée alors \(\quantifs{\exists M \in \R;\forall c \in C} c\leq M \) ainsi \(\quantifs{\forall x \in \Rps} \frac{\e{x}}{x} \leq M \) donc par passage à la limite en \(\pinf\) on trouve \(\pinf \leq M\) ce qui est absurde donc \(C\) n'est pas majorée.
	\end{itemize}
\end{corr}

\begin{defi}[Parties bornées]
	Une partie \(A\) de \(\R\) est dite bornée si elle est majorée et minorée autrement dit s’il existe deux réels \(m\) et \(M\) tel que, pour tout réel \(x\) de \(A\), on a : \(m\leq x \leq M\).
\end{defi}

\section{Valeur absolue d'un réel}
\begin{defi}
	Pour tout \(x\) réel, la valeur absolue de \(x\), notée \(\abs{x}\), est définie par : \(\abs{x} = \begin{cases}
		-x & \text{si }  x < 0   \\
		x  & \text{si }  x\geq 0 \\
	\end{cases}\)
\end{defi}

\begin{prop}
	\begin{enumerate}
		\item Pour tout \(x\) réel, on a : \(0\leq\abs{x}\) et \(x\leq\abs{x}\)
		\item Pour tout couple\((x,y)\) de réels, on a : \(\abs{xy} = \abs{x}\abs{y}\)
		\item Pour tout couple \((x,y)\) de réels tel que \(y\) est non nul, on a: \(\abs{\frac{x}{y}} = \frac{\abs{x}}{\abs{y}}\)
	\end{enumerate}
\end{prop}

\begin{defprop}[Deux inéquations élémentaires]
	Pour tout réel \(x\) et tout \underline{réel positif} \(\alpha\), on a:
	\begin{enumerate}
		\item \(\abs{x}\leq \alpha \iff -\alpha \leq x \leq \alpha \iff x \in \intervii{-\alpha}{\alpha}\)
		\item \(\abs{x}\geq \alpha \iff x \leq -\alpha\text{ ou } \alpha \leq x \iff x \in \intervei{\pinf}{-\alpha}\union\intervie{\alpha}{\pinf}\)
	\end{enumerate}
\end{defprop}

\begin{defprop}[Interprétation sur la droite des réels]
	Soit \(a\) un réel et \(b\) un \underline{réel positif}. \\
	L’ensemble des réels \(x\) vérifiant \(\abs{x-a}\leq b\) (resp. \(\abs{ x-a}\geq b \)) est l’ensemble des points de la droite des
	réels situés à une distance du point \(a\) inférieure ou égale (resp. supérieure ou égale) à \(b\).
\end{defprop}

\begin{prop}[Inégalité triangulaire]
	Pour tout couple \((x,y)\) de réels, on a :
	\[\abs{x+y}\leq \abs{x}+\abs{y}\]
\end{prop}

\begin{dem} [inégalité triangulaire]
	Soit \((x,y) \in \R^2\)
	\begin{align*}
		\abs{x+y}\leq \abs{x}+\abs{y} & \iff \abs{x+y}^2\leq (\abs{x}+\abs{y})^2      \\
		                              & \iff x^2+2xy+y^2 \leq x^2+y^2+2\abs{x}\abs{y} \\
		                              & \iff xy\leq \abs{xy}
	\end{align*}
	Ce qui est vrai donc l'inégalité est bien démontrer
\end{dem}

\begin{exoex}   
     ~\\
	Encadrer \(\frac{x\cos(x)+1}{\sin(x)+3}\) pour \(x\in\intervii{-\pi}{2\pi}\)
\end{exoex}

\begin{corr}
	Soit \(x\in\intervii{-\pi}{2\pi}\)
	\begin{itemize}

		\item \underline{numérateur} : \(\abs{x\cos(x)+1}\leq \abs{x}\abs{\cos(x)}+1\leq 2\pi+1 = 2\pi+1\)
		\item \underline{dénominateur} : \(2\leq\abs{\sin(x)+3}\leq 4\)

	\end{itemize}
	Ainsi par produit des deux inégalités on as :\(0\leq\frac{\abs{x\cos(x)+1}}{\abs{\sin(x)+3}}\leq \frac{2\pi+1}{2} \)\\
	donc \(-\frac{2\pi+1}{2} \leq \frac{x\cos(x)+1}{\sin(x)+3} \leq \frac{2\pi+1}{2}\) pour \(x\in\intervii{-\pi}{2\pi}\).
\end{corr}

\begin{prop}
	Soit un couple \((x,y)\) de réels.
	\[\abs{\abs{x}-\abs{y}} \leq\abs{x-y}\]
\end{prop}

\begin{dem}
	Soit \((x,y) \in \R^2\)
	\(x =(x-y)+y\) donc \(\abs{x} \underset{\mathrm{\text{inég. triang.}}}{\leq} \abs{x-y}+\abs{y}\) d'où \(\abs{x} - \abs{y} \leq \abs{x-y}\) \\
	De même, \(y =(x-y)+x\) donc \(\abs{y} \underset{\mathrm{\text{inég. triang.}}}{\leq} \abs{x-y}+\abs{x}\) d'où \( -\abs{x-y} \leq\abs{x} - \abs{y}\)\\

	ainsi on a \(-\abs{x-y} \leq\abs{x} - \abs{y} \leq \abs{x-y}\) donc \(\abs{\abs{x}-\abs{y}} \leq\abs{x-y}\).
\end{dem}

\section{Partie entière d'un réel}
\begin{prop}
	Pour tout réel \(x\),il existe un unique entier \(n\) tel que :
	\[n\leq x < n+1\]
\end{prop}
\begin{defi}
	On appelle partie entière de \(x\), notée \(\lfloor x \rfloor\), l'unique entier \(n\) vérifiant la propriété précédente.
\end{defi}

\begin{ex}
	\(\lfloor 3.14 \rfloor = 3\), \(\lfloor -2.7 \rfloor = -3\) et \(\lfloor 5 \rfloor = 5\).
\end{ex}

\section{Généralité sur les fonctions}
\begin{defi} [Fonction]
	Une fonction de variable réelle à valeurs réelles notée \(f\) est un objet mathématique qui, à tout élément \(x\) d’une partie non vide de \(\R\), associe un et un seul nombre réel noté \(f(x)\). \\
	\underline{Notation Fonctionnelle} : \[\fonction{f}{A}{\R}{x}{f(x)}\]
\end{defi}

\begin{defi}
	Soit \(f\) une fonction de variable réelle à valeurs réelles.
	\begin{enumerate}
		\item L’ensemble des réels \(x\) pour lesquels \(f(x)\) existe est appelé ensemble/domaine de définition de \(f\) et souvent noté \(D_f = \accol{x \in \R \tq f(x) \text{existe}}\)
		\item Soit \(x \in D_f\)\\
		      La valeur réelle \(f(x)\) est appelée image de \(x\) par \(f\). \\
		\item soit \( y \in \R\) \\
		      S'il existe \(x\) dans \(D_f\) tel que \(f(x) = y\) alors \(x\) est dit antécédent de \(y\) par \(f\)
	\end{enumerate}
\end{defi}

\begin{defprop}[égalité entre fonction]
	Deux fonctions \(f\) et \(g\) de variable réelle à valeurs réelles sont dites égales si les deux conditions suivantes sont réunies :
	\begin{itemize}
		\item les fonctions \(f\) et \(g\) ont le même ensemble de définition \(D\) ;
		\item pour tout \(x\) de \(D\), \(f(x) = g(x)\).
	\end{itemize}
	dans ce cas, on note \(f = g\).
\end{defprop}

\begin{exoex}
	est-ce que les fonctions \(f\) et \(g\) définies par :
	\[f: x\mapsto\frac{1}{\sqrt{1+x}+1} \text{ et } g:  x\mapsto\frac{\sqrt{1+x}-1}{x}\]
	Sont égales ?
\end{exoex}
\begin{corr}
	Tout d'abord \(\quantifs{\forall x \in D_f\inter D_g} f(x) = g(x)\) car :
	\begin{align*} g(x) & = \frac{\sqrt{1+x}-1}{x}                                                 \\
                    & = \frac{\paren{\sqrt{1+x}-1}\paren{\sqrt{1+x}+1}}{x\paren{\sqrt{1+x}+1}} \\
                    & = \frac{1+x-1}{x\paren{\sqrt{1+x}+1}}                                    \\
                    & = \frac{x}{x\paren{\sqrt{1+x}+1}}                                        \\
                    & = \frac{1}{\sqrt{1+x}+1} = f(x)
	\end{align*}
	Donc \(f = g\) sur \(D_f\inter D_g\) mais
	\(D_f = \intervei{-1}{\pinf}\) or \(D_g = \intervie{-1}{\pinf}\pd\accol{0}\) donc \(D_f \neq D_g\) donc \(f \neq g\).
\end{corr}

\begin{defi}[représentation graphique d'une fonction]
	Dans le plan muni d’un repère orthonormé \((O, \vec{i}, \vec{j})\), l’ensemble de points \(\mathcal{C}_f\) défini par
	\[
		\mathcal{C}_f = \accol{ M(x; f(x)) \tq x \in D_f }
	\]
	est appelé représentation graphique de \(f\) (ou courbe représentative de \(f\)).
\end{defi}

\begin{defi}[Parité,imparité et périodicité d'une fonction]
	\begin{itemize}
		\item Une fonction \(f\) est dite paire si, pour tout \(x\) de son domaine de définition, on a : \(f(-x) = f(x)\).
		\item Une fonction \(f\) est dite impaire si, pour tout \(x\) de son domaine de définition, on a : \(f(-x) = -f(x)\).
		\item Une fonction \(f\) est dite périodique de période \(T\) si, pour tout \(x\) de son domaine de définition, on a : \(f(x+T) = f(x)\).
	\end{itemize}
\end{defi}
\begin{exo}
	Montrer que toute fonction de \(\R\) peut s'écrire de manière unique comme la somme d'une fonction paire et d'une fonction impaire.
\end{exo}

\begin{corr}[Analyse-synthèse]
	Soit \(f : \R \mapsto \R \) une fonction quelqu'onque
	\begin{itemize}
		\item \analyse :  Supposons qu'il existe \(\begin{cases}
			      p:\R \mapsto \R \text{ paire} \\
			      i:\R \mapsto \R \text{ impaire}
		      \end{cases}\) telles que \(f = p + i\) \\
		      Ainsi \(\forall x \in \R \begin{cases}
			      f(x) = p(x) + i(x) \hfill (1)                  \\
			      f(-x) = p(-x) + i(-x) = p(x) - i(x) \hfill (2) \\
		      \end{cases}\) \\
		      \begin{itemize}
			      \item\(\frac{1}{2}\paren{\text{(1)+(2)}}\) donne \(p:x\mapsto \frac{f(x)+f(-x)}{2}\) \\
			      \item \(\frac{1}{2}\paren{\text{(1)-(2)}}\) donne \(i:x\mapsto \frac{f(x)-f(-x)}{2}\) \\
		      \end{itemize}
		\item \synthese : vérifions que le seul couple trouvé convient :
		      \begin{itemize}
			      \item \(\forall x \in \R, f(x) = p(x)+i(x)\)\\
			      \item \(p(-x) = p(x) \text{ et } i(-x) = -i(x)\)\\
		      \end{itemize}
	\end{itemize}
	Ainsi \(f\) s'écrit de manière unique comme la somme d'une fonction paire et impaire
\end{corr}

\begin{defi} [opération et composition]
	Soit \(f\) et \(g\) deux fonctions de variable réelle à valeurs réelles de domaines de définition \(D_f\) et \(D_g\).
	\begin{itemize}
		\item La somme de \(f\) et \(g\) est la fonction, notée \(f + g\), définie par \(f + g : x \mapsto f(x) + g(x)\). \\
		      Son domaine de définition \(D_{f+g}\) vérifie : \(D_{f+g} = D_f \inter D_g\).
		\item  La multiplication de \(f\) par le réel \(\alpha\) est la fonction, notée \(\alpha f\), définie par \(\alpha f : x \mapsto \alpha f(x)\). \\
		      Son domaine de définition \(D_{\alpha f}\) vérifie : \(D_{\alpha f} = D_f\) si \(\alpha \neq 0\).
		\item Le produit de \(f\) et \(g\) est la fonction, notée \(f g\), définie par \(f g : x \mapsto f(x)g(x)\). \\
		      Son domaine de définition \(D_{fg}\) vérifie : \(D_{fg} = D_f \inter D_g\).
		\item Le quotient de \(f\) par \(g\) est la fonction , notée \(frac{f}{g}\), définie par \(frac{f}{g} : x \mapsto \frac{f(x)}{g(x)}\). \\
		      Son domaine de définition \(D_{frac{f}{g}}\) vérifie : \(D_{frac{f}{g}} = D_f \inter \accol{x \in D_g | g(x) \neq 0}\).
		\item La composée de \(g\) et \(f\) est la fonction, notée \(g \circ f\), définie par \(g \circ f : x \mapsto g(f(x))\). \\
		      Son domaine de définition \(D_{g \circ f}\) vérifie : \(D_{g \circ f} = \accol{x \in D_f | f(x) \in D_g}\).
	\end{itemize}
\end{defi}

\begin{exoex}
	Domaine de définition de : \(\fonction{f}{D_f}{\R}{x}{\sqrt{x-\frac{1}{x}}} \)
\end{exoex}

\begin{corr}
	Soit \(x \in D_f\) alors \(x-\frac{1}{x} \geq 0 \iff x\neq 0\) et \(\frac{x^2-1}{x} = \frac{(x-1)(x+1)}{x} \geq 0\)\\
	% Assurez-vous d'avoir \usepackage{tkz-tab} dans le préambule
	\begin{tikzpicture}
		\tkzTabInit{$x$/1,$(x-1)(x+1)$/1,$x$/1,$f$/1}{$-\infty$,$-1$,$0$,$1$,$+\infty$}
		\tkzTabLine{,+,0,-,t,-,0,+,}
		\tkzTabLine{,-,t,-,0,+,t,+,}
		\tkzTabLine{,-,0,+,d,-,0,+}


	\end{tikzpicture}\\
	ainsi on voit bien que \(D_f = \intervie{-1}{0}\union\intervie{1}{\pinf}\)
\end{corr}

\section{Fonction et relation d'ordre}
\begin{defi}[Monotonie]
	Soit \(f\) une fonction de variable réelle à valeurs réelles et \(D\) une partie de son domaine de définition \(D_f\).
	\begin{enumerate}
		\item \(f\) est dite \textbf{croissante} sur \(D\) si, pour tout \((x, y) \in D^2\) tel que \(x \leq y\), on a \(f(x) \leq f(y)\).
		\item \(f\) est dite \textbf{décroissante} sur \(D\) si, pour tout \((x, y) \in D^2\) tel que \(x \leq y\), on a \(f(x) \geq f(y)\).
		\item \(f\) est dite \textbf{strictement croissante} sur \(D\) si, pour tout \((x, y) \in D^2\) tel que \(x < y\), on a \(f(x) < f(y)\).
		\item \(f\) est dite \textbf{strictement décroissante} sur \(D\) si, pour tout \((x, y) \in D^2\) tel que \(x < y\), on a \(f(x) > f(y)\).
	\end{enumerate}
	\textbf{Remarque :} \(f\) est dite \textbf{monotone} (resp. \textbf{strictement monotone}) sur \(D\) si elle est croissante ou décroissante (resp. strictement croissante ou strictement décroissante) sur \(D\).
\end{defi}

\begin{rem}[Application de la définition]
	Sous réserve que cela ait du sens :
	\begin{itemize}
		\item La somme de deux fonctions croissantes(resp. décroissantes) est croissante(resp. décroissante).
		\item La composée de deux fonctions croissantes(resp. décroissantes) est croissante(resp. décroissante).
		\item La composée d'une fonction croissante et d'une fonction décroissante est décroissante
		\item Le produit de deux fonctions \underline{positives} croissantes (resp. décroissantes) est croissante(resp. décroissante).
	\end{itemize}
\end{rem}


\begin{defi}
	Soit \(f\) une fonction de variable réelle à valeurs réelles de domaine de définition \(D_f\). \\
	Soit \(D\) une partie non vide de \(D_f\).
	\begin{enumerate}
		\item \(f\) est dite \textbf{majorée} sur \(D\) si l'ensemble \(\accol{f(x) \tq x \in D}\) est majoré, c'est-à-dire s'il existe un réel \(M\) tel que, pour tout réel \(x\) de \(D\), on a : \(f(x) \leq M\).\\
		      Un tel réel \(M\) est alors dit :
		      \begin{itemize}
			      \item \textbf{majorant} de \(f\) sur \(D\) dans le cas général.
			      \item \textbf{maximum} de \(f\) sur \(D\) dans le cas particulier où il existe \(x_0\) dans \(D\) tel que \(M = f(x_0)\).
		      \end{itemize}
		\item \(f\) est dite \textbf{minoriée} sur \(D\) si l'ensemble \(\accol{f(x) \tq x \in D}\) est minoré, c'est-à-dire s'il existe un réel \(m\) tel que, pour tout réel \(x\) de \(D\), on a : \(m \leq f(x)\).\\
		      Un tel réel \(m\) est alors dit :
		      \begin{itemize}
			      \item \textbf{minorant} de \(f\) sur \(D\) dans le cas général.
			      \item \textbf{minimum} de \(f\) sur \(D\) dans le cas particulier où il existe \(x_0\) dans \(D\) tel que \(m = f(x_0)\).
		      \end{itemize}
		\item \(f\) est dite \textbf{bornée} sur \(D\) si \(f\) est majorée et minoriée sur \(D\), c'est-à-dire s'il existe deux réels \(m\) et \(M\) tels que, pour tout réel \(x\) de \(D\), on a : \(m \leq f(x) \leq M\).\end{enumerate}
\end{defi}

\begin{prop}
	Soit \(f\) une fonction de variable réelle à valeurs réelles de domaine de définition \(D_f\). \\
	Alors \(f\) est bornée sur \(D\) si, et seulement si, la fonction \(\abs{f}\) est majorée sur \(D\).
\end{prop}
\section{Dérivation des fonctions d'une variable réelle}

\begin{defi}[dérivée en un point]
	Soit \(f\) une fonction de variable réelle à valeurs réelles de domaine de définition \(D_f\) et \(x_0\) un point de \(D_f\). \\
	\(f\) est dite dérivable en \(x_0\) si la fonction \(x\mapsto \frac{f(x)-f(x_0)}{x-x_0}\) admet une limite finie en \(x_0\). \\
	Dans ce cas, on note \(f'(x_0)\) la valeur de cette limite et on l'appelle la dérivée de \(f\) en \(x_0\). \\
	Cela reient à déterminer si la fonction \(h\mapsto \frac{f(x_0+h)-f(x_0)}{h}\) admet une limite finie en \(0\).\\
\end{defi}

\begin{defi}{fonction dérivée}
	\(f\) est dite dérivable sur \(D_f\) si elle est dérivable en tout point de \(D_f\). \\
	Dans ce cas, la fonction \(x\mapsto f'(x)\) est appelée fonction dérivée de \(f\) et notée \(f'\). \\
\end{defi}

\begin{defprop}[équation de la tangente]
	On se place dans le plan muni d’un repère orthonormé \((O, \vec{i}, \vec{j})\). \\
	Soit \(f\) une fonction de variable réelle à valeurs réelles et \(C_f\) la courbe représentative de \(f\). \\
	Soit \(x_0\) un point de \(D_f\) .\\
	Si \(f\) est dérivable en \(x_0\), alors la tangente à la courbe \(C_f\) au point \(M(x_0, f(x_0))\) est la droite d’équation :
	\[y = f'(x_0)(x-x_0) + f(x_0)\]
\end{defprop}

\begin{defprop}[opération sur les fonctions dérivable]
	Soit \(I\) et \(J\) des intervalles de \(\R\) non vide et non réduits à un point. \\
	\begin{enumerate}
		\item \underline{Combinaison linéaire} : \\
		      Soit \(f\) et \(g\) deux fonctions définies sur \(I\) et à valeurs réelles et \((\alpha, \beta)\) deux réels. \\
		      Si \(f\) et \(g\) sont dérivables sur \(I\), alors \(\alpha f + \beta g\) est dérivable sur \(I\) et sa dérivée vérifie :
		      \[\alpha f + \beta g' = \alpha f' + \beta g'\]
		\item \underline{Produit} : \\
		      Soit \(f\) et \(g\) deux fonctions définies sur \(I\) et à valeurs réelles. \\
		      Si \(f\) et \(g\) sont dérivables sur \(I\), alors \(f g\) est dérivable sur \(I\) et sa dérivée vérifie :
		      \[(f g)' =f'g+fg'\]
		      \item\underline{quotient} :\\
		      Soit \(f\) et \(g\) deux fonctions définies sur \(I\) et à valeurs réelles tel que \(g\) est non nulle sur \(I\). \\
		      Si \(f\) et \(g\) sont dérivables sur \(I\), alors \(\frac{f}{g}\) est dérivable et sa dérivée vérifie :
		      \[\paren{\frac{f}{g}}' = \frac{f'g-fg'}{g^2}\]
		\item \underline{Composition} : \\
		      Soit \(f\) une fonction définie sur \(I\) et à valeurs réelle tel que, pour tout \(x\) de \(I\), \(f(x)\) appartient à \(J\)\\
		      Soit \(g\) une fonction définie sur \(J\) et à valeurs réelles. \\
		      Si \(f\) est dérivable sur \(I\) et \(g\) dérivable sur \(J\), alors la composée \(g \circ f\) est dérivable sur \(I\) et sa dérivée vérifie :
		      \[\paren{g \circ f}' = g' \circ f \times f'\]
	\end{enumerate}
\end{defprop}

\begin{defprop}[Caractérisation des fonctions constantes ou monotones]
	Soit \(f\) une fonction définie sur un intervalle \(I\) et à valeurs réelles. \\
	\begin{enumerate}
		\item \(f\) est constante sur \(I\) si, et seulement si, pour tout \(x\) de \(I\),\(f'(x)=0\).
		\item \(f\) est croissante sur \(I\) si, et seulement si, pour tout \(x\) de \(I\), \(f'(x) \geq 0\).
		\item \(f\) est décroissante sur \(I\) si, et seulement si, pour tout \(x\) de \(I\), \(f'(x) \leq 0\).
		\item \(f\) est strictement croissante sur \(I\) si, et seulement si, les deux conditions suivante sont réunies :
		      \begin{enumerate}
			      \item pour tout \(x\) de \(I\), \(f'(x) \geq 0\) ;
			      \item il n'existe pas de réels \(a\) et \(b\) dans \(I\) avec \(a < b\) tels que pour tout \(x\) de \(\intervii{a}{b}\), on a \(f'(x) = 0\).
		      \end{enumerate}
		\item \(f\) est strictement décroissante sur \(I\) si, et seulement si, les deux conditions suivante sont réunies :
		      \begin{enumerate}
			      \item pour tout \(x\) de \(I\), \(f'(x) \leq 0\) ;
			      \item il n'existe pas de réels \(a\) et \(b\) dans \(I\) avec \(a < b\) tels que pour tout \(x\) de \(\intervii{a}{b}\), on a \(f'(x) = 0\).
		      \end{enumerate}
	\end{enumerate}
\end{defprop}

\begin{defprop}[dérivées usuelles]
    ~\\
	\begin{tabular}{|c|c|c|}

		\hline
		\textbf{Fonction}                    & \textbf{Domaine de dérivabilitée}                 & \textbf{Fonction dérivée}                                             \\
		\hline
		\(x\mapsto a\) avec \(a \in \R\)     & \(\R\)                                            & \(x\mapsto 0\)                                                        \\
		\hline
		\(x\mapsto x^n\) avec \(n \in \Ns\)  & \(\R\)                                            & \(x\mapsto nx^{n-1}\)                                                 \\
		\hline
		\(x\mapsto x^-n\) avec \(n \in \Ns\) & \(\Rs\)                                           & \(x\mapsto -nx^{-n-1}\)                                               \\
		\hline
		\(x\mapsto \sqrt{x}\)                & \(\Rps\)                                          & \(x\mapsto \frac{1}{2\sqrt{x}}\)                                      \\
		\hline
		\(x\mapsto \e{x}\)                   & \(\R\)                                            & \(x\mapsto \e{x}\)                                                    \\
		\hline
		\(x\mapsto \ln(x)\)                  & \(\Rps\)                                          & \(x\mapsto \frac{1}{x}\)                                              \\
		\hline
		\(x\mapsto \sin(x)\)                 & \(\R\)                                            & \(x\mapsto \cos(x)\)                                                  \\
		\hline
		\(x\mapsto \cos(x)\)                 & \(\R\)                                            & \(x\mapsto -\sin(x)\)                                                 \\
		\hline
		\(x\mapsto \tan(x)\)                 & \(\R\pd\accol{\frac{\pi}{2}+2k\pi \tq k \in \Z}\) & \(x\mapsto \frac{1}{\cos^2(x)} \) ou \(x\mapsto \frac{1}{\cos^2(x)}\) \\
		\hline
	\end{tabular}



\end{defprop}

\begin{exoex}
    ~\\
	Calculer\(\int_{\frac{\pi}{4}}^{\frac{\pi}{3}} \frac{\sin^3(x)}{\cos^5(x)} dx\)
\end{exoex}

\begin{corr}
	\begin{align*}
		\int_{\frac{\pi}{4}}^{\frac{\pi}{3}} \frac{\sin^3(x)}{\cos^5(x)} dx & = \int_{\frac{\pi}{4}}^{\frac{\pi}{3}} \tan^3(x) \times \frac{1}{\cos^2(x)} dx \\
		                                                                    & = \int_{\frac{\pi}{4}}^{\frac{\pi}{3}} \tan^3(x) \times \paren{\tan^2(x)+1} dx \\
		                                                                    & = \croch{\frac{1}{4}\paren{\tan^4(x) }}_{\frac{\pi}{4}}^{\frac{\pi}{3}}        \\
		                                                                    & = \frac{1}{4}\paren{\tan^4\paren{\frac{\pi}{3}} - \tan^4\paren{\frac{\pi}{4}}} \\
		                                                                    & = \frac{1}{4}\paren{\paren{\sqrt{3}}^4 - 1^4}                                  \\
		                                                                    & = 2                                                                            \\
	\end{align*}
\end{corr}

\begin{defprop}[étude pratique d'une fonction]
	Le plan d'étude d'une fonction \(f\) est en général le suivant:
	\begin{itemize}
		\item Détermination du domaine de définition de \(f\)
		\item Réduction éventuelles du domaine d'étude selon les propriétés de \(f\) (parité, périodicité, etc.)
		\item Limites aux bornes du domaine d'étude
		\item Etude de la monotonie (le plus souvent,mais pas uniquement, après calcul de la dérivée de \(f\) et détermination du signe de celle-ci )
		\item Construction du tableau de variation de \(f\)(limites aux bornes, valeurs remarquables, variations)
		\item Tracé de la courbe représentative de \(f\)
	\end{itemize}
\end{defprop}
\begin{defprop}[dérivées d'odre supériéur]
	Soit \(f\) une fonction définie sur un intervalle \(I\) et à valeurs réelles. \\
	On note
	\[f^{(0)} = f\]
	puis, pour tout entier naturel \(k\) tel que la fonction\(f^{(k)}\) existe et est déribable sur \(I\), on pose :
	\[f^{(k+1)} = \paren{f^{(k)}}'\]
	Si \(n\) est un entier naturel, tel que la fonction \(f^{(n)}\) existe alors on dit que \(f\) est \(n\)-fois dérivable sur \(I\) et que \(f^{(n)}\) est la dérivée d'ordre \(n\) (ou dérivée \(n\)-ième) de \(f\).\\

\end{defprop}
\begin{defi}[Fonction réciproque]
	Soit \(f\) une fonction définie sur un intervalle \(I\) à valeurs dans \(J\)
	Si, pour tout y de \(J\), l’équation \(y = f(x)\) admet une unique solution \(x\) dans \(I\) notée \(x = f^{-1}(y)\) alors :
	\begin{itemize}
		\item la fonction \(f\) est dite bijection de \(I\) sur \(J\)
		\item la fonction \(f^{-1}\) ainsi définie sur \(J\) et à valeurs dans \(I\), est dite bijection réciproque de \(f\).
	\end{itemize}
	\underline{Exemples}:
	\begin{itemize}
		\item \(\sqrt{}\) est une bijection de \(\Rp\) sur \(\Rp\) de bijection réciproque \(f : \Rp \to \Rp\) définie par \(f(x) = x^2\).
		\item \(\exp\) est une bijection de \(\R\) sur \(\Rps\) de bijection réciproque la fonction \(\ln\)
	\end{itemize}
\end{defi}
\begin{prop}[Propriétés de la bijection réciproque]
	Si \(f\) est une bijection de \(I\) sur \(J\) de bijection réciproque notée \(f^{-1}\) alors on a :
	\begin{enumerate}
		\item pour tout \(x\) de \(I\), \(f(f^{-1}(x)) = x\) ;
		\item pour tout \(y\) de \(J\), \(f^{-1}(f(y)) = y\).
	\end{enumerate}

\end{prop}

\begin{defprop}[représentation graphique]
	on se place dans le plan muni d’un repère orthonormé \((O, \vec{i}, \vec{j})\). \\
	Si \(f\) est une bijection de \(I\) sur \(J\) alors la courbe représentative de \(f\) et de sa bijection réciproque \(f^{-1}\) sont symétriques par rapport à la droite d’équation \(y = x\).
\end{defprop}

\begin{defprop}[dérivée de la bijection réciproque]
	Soit \(f\) une bijection de \(I\) sur \(J\)  et si \(f\) est dérivable sur \(I\) alors sa bijection réciproque \(f^{-1}\) est dérivable en tout point y de \(J\) tel que \(f'(f^{-1}(y)) \neq 0\) avec, dasn ce cas : \[(f^{-1})'(y) = \frac{1}{f'(f^{-1}(y))}\]
\end{defprop}
\begin{dem}
	Soit \(f\) une bijection de \(I\) sur \(J\), soit \(y\) in \(J\) tel que \(f'(f^{-1}(y)) \neq 0\). \\
	on sait que \(f(f^{-1}(y)) = y\) donc en appliquant la définition de la dérivée de fonction composée on a :
	\[(f(f^{-1}(y)))' = (y)' \iff f'(f^{-1}(y))\times (f^{-1}(y))' = 1 \iff (f^{-1}(y))' = \frac{1}{f'(f^{-1}(y))}\]
\end{dem}
\begin{defprop}[Trois fonction usuelles trigonométriques]
	\begin{itemize}
		\item \underline{Fonction \(\Arccos\)} : \\
		      La fonction \(\Arccos\) est la réciproque de la fonction \(\fonction{c}{\intervii{0}{\pi}}{\intervii{-1}{1}}{x}{\cos(x)}\) et est donc définie sur \(\intervii{-1}{1}\) à valeurs dans \(\intervii{0}{\pi}\) et dérivable sur \(\intervee{-1}{1}\) de dérivée: \[\arccos':x\mapsto \frac{-1}{\sqrt{1-x^2}}\]\\
		\item \underline{Fonction \(\Arcsin\)} : \\
		      La fonction \(\Arccos\) est la réciproque de la fonction \(\fonctionlambda{\intervii{-\frac{\pi}{2}}{\frac{\pi}{2}}}{\intervii{-1}{1}}{x}{\sin(x)}\) et est donc définie sur \(\intervii{-1}{1}\) à valeurs dans \(\intervii{-\frac{\pi}{2}}{\frac{\pi}{2}}\) et dérivable sur \(\intervee{-1}{1}\) de dérivée: \[\arcsin':x\mapsto \frac{1}{\sqrt{1-x^2}}\]\\
		\item \underline{Fonction \(\Arctan\)} : \\
		      La fonction \(\Arccos\) est la réciproque de la fonction \(\fonctionlambda{\intervee{-\frac{\pi}{2}}{\frac{\pi}{2}}}{\R}{x}{\tan(x)}\) et est donc définie sur \(\R\) à valeurs dans \(\intervee{-\frac{\pi}{2}}{\frac{\pi}{2}}\) et dérivable sur \(\R\) de dérivée: \[\arctan':x\mapsto \frac{1}{1+x^2}\]\\
	\end{itemize}
\end{defprop}


\begin{dem}[démonstration de la dérivée de la fonction \(\Arccos\)]
	Soit \(y\in\intervii{-1}{1}\), on note \(\fonction{c}{\intervii{0}{\pi}}{\intervii{-1}{1}}{x}{\cos(x)}\)
	\begin{align*}
		c'(c^{-1}(y)) & = -\sin(c^{-1}(y))                                                                                                \\
		              & = -\sqrt{\sin^2(c^{-1}(y))} \qquad  \text{car }c^{-1}(y)\in \intervii{0}{\pi} \text{ donc } \sin(c^{-1}(y))\geq 0 \\
		              & = -\sqrt{1-\cos^2(c^{-1}(y))}                                                                                     \\
		              & = -\sqrt{1-y^2}                                                                                                   \\
	\end{align*}
	Ainsi d'après la définition de la dérivée de la bijection réciproque on a : \(\Arccos'(y) = \frac{-1}{\sqrt{1-y^2}}\)
\end{dem}

\begin{rem}[démonstration d'une relation intéressante entre \(\Arctan(x)\) et \(\Arctan\paren{{\frac{1}{x}}}\)]
	Soit \(f:x\mapsto \Arctan{\paren{\frac{1}{x}}}\), on as \(D_f = \R\pd\accol{0}\) et \(f\) dérivable sur \(D_f\)
	\begin{align*}
		f'(x) & = \Arctan'\paren{\frac{1}{x}} \times \paren{\frac{1}{x}}'         \\
		      & = \frac{1}{1+\paren{\frac{1}{x}}^2} \times \paren{\frac{-1}{x^2}} \\
		      & = \frac{-1}{x^2+1}
	\end{align*}
	On remarque que \(\quantifs{\forall x \in \Rs}f'(x) = -\Arctan'\paren{x}\) ainsi \(\quantifs{\forall x \in \Rps}f'(x) +\Arctan'\paren{x} = 0\) donc \(\quantifs{\forall x \in \Rs}\paren{f(x)+\Arctan\paren{x}}'=0\) \\
	Ainsi il existe \(c\) un réel tel que \(\quantifs{\forall x \in \Rps}f(x) + \Arctan\paren{x} = c\)
	\begin{align*}
		\text{Pour } x = 1, f(1) + \Arctan(1) & = c \\
		f(1) + \frac{\pi}{4}                  & = c \\
		c = \frac{\pi}{2}
	\end{align*}
	Ainsi \(\quantifs{\forall x \in \Rps} \Arctan\paren{\frac{1}{x}} + \Arctan\paren{x} = \frac{\pi}{2}\) \\
	De manière analogue on trouve  \(\quantifs{\forall x \in \Rms} \Arctan\paren{\frac{1}{x}} + \Arctan\paren{x} = -\frac{\pi}{2}\) \\
\end{rem}

\chapter{Calcul algébrique (rappels et compléments)}

\minitoc
\section{Sommes et produit finis}
\begin{nota}
	Soit \(\paren{a_i}_{i\in I}\) une famille de réels indexée par un ensemble \(I\) fini. \\
	La somme (resp. le produit) de tous les réels de la famille est notée \(\sum_{i\in I} a_i\) (resp. \(\prod_{i\in I} a_i\)). \\
	\begin{itemize}
		\item Si \(I\) est l'ensemble vide, on convient que : \(\sum_{i\in I} a_i = 0\) et \(\prod_{i\in I} a_i = 1\).
		\item Si \(I = \accol{1,2,\ldots,n}\) avec \(n\) un entier naturel non nul, on note \(\sum_{i=1}^n a_i\)  ou \(\sum_{1\leq i \leq n} a_i\) au lieu de \(\sum_{i\in I} a_i\) (resp. \(prod_{i=1}^n a_i\) ou \(\prod_{1\leq i \leq n} a_i\) au lieu de \(\prod_{i\in I} a_i\)).
	\end{itemize}
\end{nota}
\begin{prop}[opération et calcul par paquets]
	\begin{itemize}
		\item Pour toutes familles \(\paren{a_i}_{i\in I}\) et \(\paren{b_i}_{i\in I}\) de réels indexées par \(I\) et pour tout couple\((\alpha,\beta)\) de réels, on a :
		      \[ \sum_{i\in I} \paren{\alpha a_i + \beta b_i} = \alpha \sum_{i\in I} a_i + \beta \sum_{i\in I} b_i \qquad \text{ et } \qquad \prod_{i\in I} \paren{a_i b_i} = \paren{\prod_{i\in I} a_i}\paren{\prod_{b_i}}\]
		\item Pour toute famille \(\paren{a_i}_{i\in I}\) de réels indexée par \(I\) avec \(I=I_1 \union I_2\) et \(I_1\cap I_2 = \emptyset\), on a :
		      \[ \sum_{i\in I} a_i = \sum_{i\in I_1} a_i + \sum_{i\in I_2} a_i \qquad \text{ et } \qquad  \prod_{i\in I} a_i = \prod_{i\in I_1} a_i \prod_{i\in I_2} a_i \]
	\end{itemize}
\end{prop}

\begin{exoex}
	~\\
	Calculer : \(\sum_{k=1}^{2n} (-1)^k k \) avec \(n \in \N\)
\end{exoex}

\begin{corr}
	\begin{align*}
		\sum_{k=1}^{2n} (-1)^k k & = \sum_{k=0}^{n-1} (-1)^{2k+1} (2k-+) + \sum_{k=1}^{n} (-1)^{2k} (2k) \\
		                         & = -\sum_{k=0}^{n-1} (2k+1) + \sum_{k=1}^{n} 2k                        \\
		                         & = -\paren{2\sum_{k=0}^{n-1} k + n} + 2\sum_{k=1}^{n} k                \\
		                         & = -\paren{2\frac{(n-1)n}{2} + n} + 2\frac{n(n+1)}{2}                  \\
		                         & = n\paren{n+1-n+1-1}                                                  \\
		                         & = n                                                                   \\
	\end{align*}
\end{corr}
\begin{defprop}[téléscopage]
	Soit \(\paren{b_i}_{1 \leq i \leq n}\) une famille \underline{finie} de réels avec \(n\) supérieur ou égal à \(2\).
	\begin{enumerate}
		\item La somme \(\sum_{i=1}^n b_{i+1}-b_i\) est dire somme télescopique et vaut \(b_{n+1}-b_1\).
		\item Si tous les \(b_i\) sont non nuls, le produit \(\prod_{i=1}^n \frac{b_{i+1}}{b_i}\) est dit produit télescopique et vaut \(\frac{b_{n+1}}{b_1}\).
	\end{enumerate}
\end{defprop}

\begin{defprop}[Somme usuelles]
	Pour tout entier naturel \(n\) et tout réel \(x\) différent de \(1\), on a :
	\[\sum_{k=0}^n k = \frac{n(n+1)}{2} \qquad \sum_{k=0}^n k^2 = \frac{n(n+1)(2n+1)}{6} \qquad \sum_{k=0}^n x^k = \frac{x^{n+1}-1}{x-1}\]
\end{defprop}

\begin{defprop}[Factorisation de \(a^n-b^n\) ]
	Pour tout \(n\) entier naturel non nul et tout couple \((a,b)\) de réels, on a :
	\begin{align*}
		a^n-b^n & = (a-b)\paren{a^{n-1} + a^{n-2}b + \ldots + ab^{n-2} + b^{n-1}} \\
		        & = (a-b)\sum_{k=0}^{n-1} a^{n-1-k}b^k                            \\
		        & = (a-b)\sum_{k=0}^{n-1} a^kb^{n-1-k}                            \\
	\end{align*}
\end{defprop}

\begin{defprop}[coefficients binomiaux]
	Soit \(n\) un entier naturel non et \(k\) entière relatif, on a:
	\begin{enumerate}
		\item \(\binom{n}{k} = \binom{n}{n-k} \hfill \text{(symétrie)}\)
		\item \(binom{n}{k} +\binom{n}{k+1} = \binom{n+1}{k+1} \hfill \text{(relation de Pascal)}\)
		\item \(\binom{n}{k} \) est un entier naturel
	\end{enumerate}
\end{defprop}

\begin{defprop}[Formule du binôme de Newton]
	Pour tout couple \((a,b)\) de réels et tout entier naturel \(n\), on a :
	\[\paren{a+b}^n = \sum_{k=0}^n \binom{n}{k} a^{n-k}b^k = \sum_{k=0}^n \binom{n}{k} a^{k}b^{n-k}\]
\end{defprop}


\section{Cas des sommes doubles finies}
\begin{defi}
	Soit \(A\) un ensemble fini de couples et \((a_{i,j})_{(i,j)\in A}\) une famille de réels indexée par \(A\). La somme de tous les réels de la famille \((a_{i,j})_{(i,j)\in A}\) est notée \(\sum_{(i,j)\in A} a_{i,j}\) et appelée somme double. \\
	\underline{Remarque} : Si \(A\) est l'ensemble vide, on convient que \(\sum_{(i,j)\in A} a_{i,j} = 0\)
\end{defi}
\begin{defprop}[Sommes double rectangulaires]
	Dans le cas où \(A = \accol{1,2,\ldots,n}\times \accol{1,2,\ldots,m}\) avec \(n\) et \(m\) des entiers naturels non nuls,
	\begin{itemize}
		\item la somme double \(\sum_{(i,j)\in A} a_{i,j}\) est rectangulaire
		\item le somme double \(\sum_{(i,j)\in A} a_{i,j}\) s'écrit aussi \(\sum_{\substack{1 \leq i \leq n \\ 1 \leq j \leq m}} a_{i,j}\)
		\item la somme double \(\sum_{(i,j)\in A} a_{i,j}\) vaut  :
		      \[ sum_{(i,j)\in A} a_{i,j} = \sum_{\substack{1 \leq i \leq n \\ 1 \leq j \leq m}} a_{i,j} = \sum_{i=1}^n \paren{\sum_{j=1}^m a_{i,j}} = \sum_{j=1}^m \paren{\sum_{i=1}^n a_{i,j}} \]
		\item si \((b_i)_{1\leq i \leq n}\) et \((c_j)_{1\leq j \leq m}\) sont des familles finies de réels, alors : \[\paren{\sum_{i=1}^n b_i}\paren{\sum_{j=1}^m c_j} = \sum_{\substack{1 \leq i \leq n \\ 1 \leq j \leq m}} b_i c_j\]
	\end{itemize}
\end{defprop}

\begin{defprop}[somme double triangulaire]
	Dans le cas où \(A = \accol{(i,j) \in \N^2 | 1 \leq i \leq j \leq n}\) avec \(n\) un entier naturel non nul,
	\begin{itemize}
		\item La somme double \(\sum_{(i,j)\in A} a_{i,j}\) est dite triangulaire.
		\item La somme double \(\sum_{(i,j)\in A} a_{i,j}\) s'écrit aussi \(\sum_{1\leq i \leq j \leq n} a_{i,j}\) et vaut:
		      \[\sum_{(i,j)\in A} a_{i,i} = \sum_{1\leq i \leq j \leq n} a_{i,j} = \sum_{i=1}^n \paren{\sum_{j=i}^n a_{i,j}} = \sum_{j=1}^n \paren{\sum_{i=1}^j a_{i,j}}\]
	\end{itemize}
\end{defprop}


\section{Système linéaire de deux équations à deux inconnues}
\begin{defprop}[rappel de première]
	Dans le plan \(\R^2\) muni d’un repère orthonormé \((O,\vec{i},\vec{j})\), toute droite \(D\) admet une équation de la forme \[ax + by = c\]
	où \(a\), \(b\) et \(c\) sont des réels tels que \((a,b)\neq (0,0)\). \\
	Avec ces notations,
	\begin{itemize}
		\item le vecteur \(\vec{n}\) de coordonnées \((a,b)\) est un vecteur normal à \(D\) ;
		\item le vecteur \(\vec{u}\) de coordonnées \((-b,a)\) est un vecteur directeur de \(D\).
	\end{itemize}
\end{defprop}

\begin{defprop}[Système linéaire de deux équations à deux inconnues]
	Soit \(a\), \(b\), \(c\), \(a'\), \(b'\) et \(c'\) des réels. Le système d’équations
	\[
		(S) :
		\begin{cases}
			ax + by = c \\
			a'x + b'y = c'
		\end{cases}
	\]
	d’inconnues les réels \(x\) et \(y\) est dit système linéaire de deux équations à deux inconnues.
\end{defprop}

\begin{defprop}[Interprétation géométrique]
	Dans le cas où \((a,b)\neq (0,0)\) et \((a',b')\neq (0,0)\), résoudre le système \((S)\) revient à  déterminer l’intersection entre deux droites \(D\) et \(D'\) du plan.
	Trois cas se présentent :
	\begin{itemize}
		\item Les droites sont confondues donc \((S)\) a une infinité de solutions qui forment une droite ;
		\item Les droites sont sécantes donc \((S)\) a une unique solution ;
		\item Les droites sont parallèles non confondues donc \((S)\) n’a pas de solutions.
	\end{itemize}
\end{defprop}


\section{Système linéaire de trois équations à trois inconnues}

\begin{defprop}[rappel de terminale]
	Dans l'espace \(\R^3\) muni d’un repère orthonormé \((O,\vec{i},\vec{j},\vec{k})\), tout plan \(P\) admet une équation de la forme
	\[ax + by + cz = d\]
	où \(a\), \(b\), \(c\) et \(d\) sont des réels tels que \((a,b,c)\neq (0,0,0)\)
	\begin{itemize}
		\item le vecteur \(\vec{n}\) de coordonnées \((a,b,c)\) est un vecteur normal à \(P\) ;
		\item deux vecteurs non colinéaires pris parmi les vecteurs de coordonnées \((-b,a,0)\), \((0,-c,b)\) et \((-c,0,a)\) donnent la direction de \(P\).
	\end{itemize}
\end{defprop}

\begin{defprop}[Système linéaire de deux équations à trois inconnues]
	Soit \(a\), \(b\), \(c\), \(d\), \(a'\), \(b'\), \(c'\) et \(d'\) des réels. Le système d’équations
	\[
		(S) :
		\begin{cases}
			ax + by + cz = d \\
			a'x + b'y + c'z = d'
		\end{cases}
	\]
	d’inconnues les réels \(x\), \(y\) et \(z\) est dit système linéaire de deux équations à trois inconnues.
\end{defprop}

\begin{defprop}[Interprétation géométrique]
	Dans le cas où \((a,b,c)\neq (0,0,0)\) et \((a',b',c')\neq (0,0,0)\), résoudre le système \((S)\) revient à déterminer l’intersection entre deux plans \(P\) et \(P'\) de l’espace.
	Trois cas se présentent :
	\begin{itemize}
		\item Les plans sont confondus donc \((S)\) a une infinité de solutions qui forment un plan ;
		\item Les plans sont sécants donc \((S)\) a une infinité de solutions qui forment une droite ;
		\item Les plans sont parallèles non confondus donc \((S)\) n’a pas de solutions.
	\end{itemize}
\end{defprop}
\begin{defprop}[Système linéaire de trois équations à trois inconnues]
	Soit \(a\), \(b\), \(c\), \(d\), \(a'\), \(b'\), \(c'\), \(d'\), \(a''\), \(b''\), \(c''\) et \(d''\) des réels. Le système d’équations
	\[ (S) : \begin{cases}
			ax + by + cz = d     \\
			a'x + b'y + c'z = d' \\
			a''x + b''y + c''z = d''
		\end{cases} \]
	d’inconnues les réels \(x\), \(y\) et \(z\) est dit système linéaire de trois équations à trois inconnues.
\end{defprop}

\begin{defprop}[Interprétation géométrique]
	Dans le cas où \((a,b,c)\neq (0,0,0)\), \((a',b',c')\neq (0,0,0)\) et \((a'',b'',c'')\neq (0,0,0)\), résoudre le système \((S)\) revient à déterminer l’intersection entre trois plans \(P\), \(P'\) et \(P''\) de l’espace.
	Cela conduit à distinguer huit cas de figures qui donnent quatre types d’ensemble-solution pour \((S)\) :
	\begin{itemize}
		\item Le système \((S)\) a une infinité de solutions qui forment un plan ;
		\item Le système \((S)\) a une infinité de solutions qui forment une droite ;
		\item Le système \((S)\) a une unique solution ;
		\item Le système \((S)\) n’a pas de solutions.
	\end{itemize}
\end{defprop}

\section{Algorithme du Pivot}
\begin{rem} [Remarque préliminaire]
	En cycle terminal, de petits systèmes linéaires ont été rencontrés et résolus dans des cas simples, le plus souvent par “substitution”. \\
	En MP2I, nous utiliserons en priorité la méthode de résolution par “pivot”. Plus efficace et élégante, cette technique sera reprise au semestre 2 dans le chapitre “Matrices” pour résoudre plus généralement des systèmes linéaires de \(n\) équations à \(p\) inconnues.
\end{rem}

\begin{defprop}[Opérations élémentaires]
	On reprend les notations des paragraphes III. et IV. et on note \(L_i\) la \(i\)-ème ligne du système \((S)\).\\
	On appelle opérations élémentaires sur les lignes du système linéaire \((S)\) :
	\begin{enumerate}
		\item l’échange de deux lignes distinctes : \(L_i \leftrightarrow L_j\) avec \(i\neq j\) ;
		\item la multiplication d'une ligne par un réel non nul : \(L_i \leftarrow \lambda L_i\) avec \(\lambda\neq 0\) ;
		\item l'addition à une ligne du produit d'une autre ligne par un réel non nul : \(L_i \leftarrow L_i + \lambda L_j\) avec \(i\neq j\) et \(\lambda\neq 0\).
	\end{enumerate}
\end{defprop}

\begin{prop}[Propriété importante]
	Toute opération élémentaire sur les lignes d’un système linéaire le transforme en un système linéaire équivalent c’est-à-dire un système ayant le même ensemble de solutions.
\end{prop}

\begin{defprop}[résolution d'un système linéaire par la méthode du pivot]
	La résolution d’un système linéaire par la méthode du pivot se déroule en deux phases :
	\begin{itemize}
		\item \underline{phase de descente} : en effectuant des opérations élémentaires sur les lignes du système, on transforme le système en un système de forme “triangulaire” ou “trapézoïdale” comme, par exemple,
		      \[(S1) : \begin{cases} a_1x+b_1y = c_1 \\ b'_1y = c'_1 \end{cases}\]
		      \[(S2) : \begin{cases} a_1x+b_1y+c_1z = d_1 \\ b'_1y+c'_1z = d'_1 \end{cases}\]
		      \[(S3) : \begin{cases} a_1x+b_1y+c_1z = d_1 \\ b'_1y+c'_1z = d'_1 \\ c''_1z = d''_1 \end{cases}\]
		\item \underline{phase de remontée} : Le système obtenu est équivalent au système initial ; il est facile à résoudre ce qui permet d’obtenir l’ensemble des solutions du système initial. Dans cette phase de remontée, on peut au choix :
		      \begin{itemize}
			      \item effectuer des substitutions successives (moins élégant) ;
			      \item utiliser à nouveau des opérations élémentaires sur les lignes pour réduire le système sous forme “diagonale” (plus élégant et facile à coder).
		      \end{itemize}
	\end{itemize}
\end{defprop}
\begin{rem}
	Les opérations élémentaires effectuées lors de la résolution d’un système linéaire par la méthode du pivot (phases de descente et de remontée) doivent systématiquement être indiquées en marge du système étudié pour faciliter la lecture des correcteurs et permettre de retrouver les éventuelles erreurs de calcul.
\end{rem}

\begin{rem}[Pour aller plus loin (pour ceux qui ont suivi l’option maths expertes)]
	\begin{itemize}
		\item Les petits systèmes linéaires décrits au III. et IV. peuvent se traduire matriciellement par une équation matricielle du type \(AX = B\) avec \(A\) et \(B\) des matrices à préciser et \(X\) une matrice colonne inconnue.
		\item L’effet des opérations élémentaires sur les lignes de ces systèmes peut se traduire matriciellement par des multiplications de la matrice \(A\) à gauche par des matrices inversibles bien
	\end{itemize}
\end{rem}


%finir de taper les notes de cours 

\chapter{Nombres complexes}

\minitoc
\section{Généralité}
\begin{defi}[Propriété de \(\C\)]

	On ADMET l'existence d’un ensemble noté \(\C\), dont les éléments sont appelés nombres complexes, tel que :
	\begin{enumerate}
		\item \(\C\) contient\(\R \)
		\item \(\C\) est muni de deux opérations \(+\) et \(\times\) sur \(\C\) qui étendent les opérations \(+\) et \(\times\) connues sur \(\R\) et suivent les mêmes règles de calcul que celles-ci
		\item \(\C\) contient un élément noté \(\i\) vérifiant \(\i^2 = -1\)
		\item Tout élément \(z\) de \(\C\) s'écrit de manière uneique sous la forme \(z = a+\i b\) avec \(\paren{a,b} \in \R^2\)
	\end{enumerate}
\end{defi}
\begin{rem}
	\begin{itemize}
		\item La forme \(z = a+\i b\)  avec \(\paren{a,b} \in \R^2\) est dite forme algébrique du nombre complexe \(z\) \begin{itemize}
			      \item le réel \(a\) est dit partie réelle du nombre complexe \(z\) et noté \(a = \Reel{z}\)
			      \item le réel \(b\) est dit partie imaginaire du nombre complexe \(z\) et noté \(b = \Im{z}\)
		      \end{itemize}
		\item L'unicité d'écriture d'un nombre complexe sous forme algébrique se traduit par : \\
		      Pour tout réels \(a,b,a'\) et \(b'\), on a :
		      \[a+\i b = a'+\i b' \text{si, et seulement si, } a=a' \text{ et } b= b'\]
	\end{itemize}
\end{rem}

\begin{defprop}[Opériation sur \(\C\)]
	L’ensemble \(\C = \accol{a + \i b \tq \paren{a, b} \in \R^2}\) est muni deux opérations + et \(+\) et \(\times\) définies par, pour tout nombre complexe \(z\) de forme algébrique \(a + \i b\) et tout nombre complexe \(z'\) de forme algébrique \(a' + \i b'\) : \[\begin{cases}
			z+z' = (a+\i b) + (a'+\i b') = (a+a')+\i(b+b') \\
			z \times z' = (a+\i b) \times (a'+\i b') = (aa'-bb') + \i(ab'+a'b)
		\end{cases}\]
\end{defprop}

\begin{defprop}[Extension des résultat vus dans \(\R\)]
	\begin{enumerate}
		\item Pour tout \(n\) entier naturel et tout nombre complexe \(z\) différent de \(1\), on a :
		      \[\sum_{k=0}^n z^k = \frac{1-z^{k+1}}{1-z}\]
		\item Pour tout \(n\) entier naturel et tout couple \((z,z')\) nombres complexes , on a :
		      \[(z+z')^n = \sum_{k=0}^{n}\binom{n}{k}z^k(z')^{n-k} = \sum_{k=0}^{n}\binom{n}{k}z^{n-k}(z')^k\]
		\item Pour tout \(n\) entier naturel et tout couple \((z,z')\) nombres complexes , on a :
		      \[z^n+(z')^n = (z-z')\paren{z^{n-1}+z^{n-1}z'+\dots+z(z')^{n-2}+(z')^{n-1}} = (z-z')\sum_{k=0}^{n-1} z^{n-1-k}(z')^{k} =(z-z')\sum_{k=0}^{n-1} z^{k}(z')^{n-1-k} \]
	\end{enumerate}
\end{defprop}

\begin{defprop}[Plan complexe : affixe d’un point, d’un vecteur]
	Dans toute la suite, on considère le plan usuel muni d’un repère orthonormé direct.
	\begin{itemize}
		\item A tout complexe \(z\), on peut associer le point \(M\) de coordonnées \((\Reel{z}, \Ima{z})\) dit image de \(z\).
		\item A tout point \(M\) de coordonnées \((x, y)\), on peut associer le complexe \(z = x + \i y\) dit affixe de \(M\) .
	\end{itemize}
	On identifie donc \(\C\) au plan usuel muni d’un repère orthonormé direct et on parle de “plan complexe”. \\

	A tout complexe \(z\), on peut aussi associer le vecteur \(\vec{u}\) de coordonnées \((\Reel{z}, \Ima{z})\) dit image de z et à tout vecteur  \(\vec{u}\) de coordonnées \((x, y)\), on peut associer le complexe \( z = x + \i y\) dit affixe de  \(\vec{u}\) . Ainsi :
	\begin{itemize}
		\item Pour tout vecteur  \(\vec{u}\) d’affixe \(z\) et tout réel \(\alpha\), le vecteur \(\alpha \vec{u}\) a pour affixe \(\alpha z\). \\
		\item Pour tous vecteurs \(\vec{u}\) et \(\vec{u'}\) d’affixes respectives \(z\) et \(z'\), le vecteur \(\vec{u} + \vec{u'} \) a pour affixe \(z + z'\). \\
		\item Pour tous points \(M\) et \(M '\) d’affixes respectives \(z\) et \(z'\) , le vecteur \(\vec{MM'}\)a pour affixe \(z' - z\).
	\end{itemize}
\end{defprop}

\section{Conugué d'un nombre complexe}
\begin{defi}
	On appelle conjugué d’un nombre complexe \(z\) et on note \(\conj{z}\) le nombre complexe défini par : \[\conj{z} = \Reel{z} -\i \Ima{z}\]
	Pour tout nombre complexe \(z\), le point d’affixe \(\conj{z}\) et le point d’affixe \(z\) sont symétriques par rapport à l’axe des réels dans le plan complexe.
\end{defi}

\begin{defprop}
	Pour tous nombres complexes \(z\) et \(z'\), on a les propriétés suivantes :
	\begin{enumerate}
		\item \(z+\conj{z} = 2 \Reel{a}\)\\
		\item \(z-\conj{z} = -2 \Ima{z}\) \\
		\item \(\conj{\conj{z}} = z \)\\
		\item \(\conj{z+z'} = \conj{z}+\conj{z'}\)\\
		\item \(\conj{zz'} = \conj{z} \conj{z'} \)\\
		\item \(\conj{\frac{z}{z'}} = \frac{\conj{z}}{\conj{z'}}\)
	\end{enumerate}
\end{defprop}

\section{module d'un nombre complexe}
\begin{defprop}
	On appelle module d’un nombre complexe \(z\) et on note \(\abs{z}\) le nombre réel positif défini par : \[\abs{z} = \sqrt{ \paren{\Reel{z}}^2+\paren{\Ima{z}}^2}\]
\end{defprop}

\begin{defprop}[interprétation géometriques]
	\begin{itemize}
		\item Pour tout nombre complexe \(z\), le module \(\abs{z}\) est : \begin{itemize}
			      \item la distance entre le point d’affixe \(0\) et le point d’affixe \(z\) ;
			      \item la norme de tout vecteur d’affixe \(z\)
		      \end{itemize}
		\item Pour tous nombres complexes  \(z\) et \(z'\) le module \(\abs{z-z'}\) est :\begin{itemize}
			      \item la distance entre les points d’affixe \(z\) et \(z'\) ;
			      \item la norme du vecteur d’affixe \(z' - z\)
		      \end{itemize}
		\item Soit \(r\) un réel positif, \(z_0\) un nombre complexe et \(M_0\) le point d’affixe \(z_0\).
		      \begin{itemize}
			      \item Les points du plan dont l’affixe \(z\) vérifie \(\abs{z - z_0} = r\) forment le cercle de centre \(M_0\) et de rayon \(r\).
			      \item Les points du plan dont l’affixe \(z\) vérifie \(\abs{z - z_0} \leq r\) forment le disque de centre \(M_0\), de rayon \(r\)
		      \end{itemize}
	\end{itemize}
\end{defprop}

\begin{prop}
	Pour tous nombres complexes \(z\) et \(z'\), on a les propriétés suivantes :
	\begin{itemize}
		\item \(\abs{\Reel{z}} \leq \abs{z}\) et \(\abs{\Ima{z}} \leq \abs{z}\)
		\item \(\abs{z}^2 = z \conj{z}\)
		\item \(\abs{zz'} = \abs{z} \abs{z'}\)
		\item \(\abs{\frac{z}{z'}} = \frac{\abs{z}}{\abs{z'}}\) Dans le cas où \(z'\) est non nul
		\item \(\frac{z}{z'} = \frac{z\abs{z'}}{\abs{z'}^2}\)
		\item \(\abs{z+z'} \leq \abs{z}+\abs{z'}\) avec égalité si, et seulement si il existe un réel positif \(\alpha\) tel que \(z' = \alpha z\)
	\end{itemize}
\end{prop}

\section{Nombre complexe de module \(1\) et trigonométrie}
\begin{defi}[Cercle trigonométrique]
	On identifie le cercle trigonométrique et l’ensemble des nombres complexes de module \(1\) que l’on note : \[\U = \accol{z \in \C \tq \abs{z} = 1}\]
\end{defi}

\begin{defprop}
	Pour tout nombre réel \(t\), on appelle exponentielle imaginaire de \(t\) et on note \(e^{\i t}\) le nombre complexe défini par :
	\[e^{\i t} = \cos(t) + \i \sin(t) \]
	Pour tous nombres réels \(t\) et \(t'\), on a l’égalité : \[e^{\i(t+t')} = e^{\i t}e^{\i t'}+ \]
\end{defprop}


\begin{defprop}[Formule D'Euler]
	Pour tout nombre réel \(t\), on a les égalités suivantes dites formules d’Euler
	\[\cos(t) = \frac{e^{\i t}+e^{-\i t}}{2} \text{ et } \sin(t) = \frac{e^{\i t} - e^{-\i t}}{2}\]
\end{defprop}
\begin{prop}[Technique de l'angle moitié]

	La technique de l’angle moitié permet l’obtention de factorisations classiques à savoir retrouver :
	\begin{itemize}
		\item pour tout \(t\) réel, \(1+e^{\i t} = e^{\i \frac{t}{2}}\paren{e^{-\i \frac{t}{2}}+e^{\i \frac{t}{2}}} = 2 \cos\paren{-\frac{t}{2}}e^{\i \frac{t}{2}} = 2 \cos\paren{\frac{t}{2}}e^{\i \frac{t}{2}} \)
		\item pour tout \(t\) réel, \(1-e^{\i t} = e^{\i \frac{t}{2}}\paren{e^{-\i \frac{t}{2}}-e^{\i \frac{t}{2}}} = 2 \sin\paren{-\frac{t}{2}}e^{\i \frac{t}{2}} = -2 \sin\paren{\frac{t}{2}}e^{\i \frac{t}{2}} \)
		\item pour tout réel \(p\) et \(q\), \(e^{\i p}+e^{\i q} = e^{\i \frac{p+q}{2}}\paren{e^{\i \frac{p-q}{2}}+e^{-\i \frac{p-q}{2}}} = 2 \cos\paren{\frac{p-q}{2}}e^{\i \frac{p+q}{2}}  \)
		\item pour tout réel \(p\) et \(q\), \(e^{\i p}-e^{\i q} = e^{\i \frac{p+q}{2}}\paren{e^{\i \frac{p-q}{2}}-e^{-\i \frac{p-q}{2}}} = - 2 \sin\paren{\frac{p-q}{2}}e^{\i \frac{p+q}{2}}  \)
	\end{itemize}
	\underline{Remarque} : \\
	En écrivant la partie réelle et la partie imaginaire de \(e^{\i p} \pm e^{\i q}\) à partir des deux dernières factorisations, on trouve des formules de factorisation pour \(\cos (p) \pm \cos (q) \)et \(\sin (p) \pm \sin (q)\) \\ \\
	\underline{Linéarisation} \\
	A l’aide des formules d’Euler et du binôme de Newton, on peut transformer une expression du type
	\(cos(t)^n\) ou \(sin(t)^n\) avec \(t\) réel et \(n\) entier naturel en une combinaison linéaire de \(cos(pt)\) ou de \(sin(pt)\)
	avec \(p\) un entier naturel. Cela est notamment utile pour du \underline{calcul de primitives}.
\end{prop}

\begin{exoex}
	~\\Soit \(f(x) = \paren{\sin(x)}^3\) avec \(x \in \R\). Calculer la primitive de \(f\)
\end{exoex}

\begin{corr}
	\begin{align*}
		\paren{\sin(x)}^3 & = \paren{\frac{e^{\i x}-e^{-\i x}}{2 \i}}^3                                              \\
		                  & =\frac{1}{-8\i} \paren{e^{3\i x}+3\paren{e^{-\i x}}-3\paren{e^{\i x}} -e^{-3 \i x}}      \\
		                  & =\frac{1}{-4} \paren{\frac{e^{3\i x}-e^{-3\i x}}{2 \i}-3\frac{e^{\i x}-e^{_\i x}}{2 \i}} \\
		                  & = -\frac{1}{4}\sin(3x) +\frac{3}{4}\sin(x)
	\end{align*}
	Donc \(F_\lambda(x) = \frac{1}{12}\cos(3x)- \frac{3}{4}\cos(x) + \lambda \) pour \(\lambda \in \R \)
\end{corr}

\begin{defprop}[Formule de Moivre]
	Pour tout nombre réel \(t\) et tout entier relatif \(n\), on a \(e^{\i nt} = \paren{e^{\i t}}^n\), c’est-à-dire :
	\[\cos(nt)+\i\sin(nt) = \paren{cos(t)+\i \sin(t)}^n\]
\end{defprop}

\begin{dem}[Moivre par récurrence]
	Soit \(n \in \N\) et \(t\in\R\)
	Montrons que \(\quantifs{\forall (n,t) \in \N\times\R} e^{\i nt} = \paren{e^{\i t}}^n\) \\
	On note \(P(n)\) la Propriété \guillemets{\(e^{\i nt} = \paren{e^{\i t}}^n\)}
	\begin{itemize}
		\item \underline{Initialisation} :
		      \(P(0)\) est vrai car \(\begin{cases}
			      \paren{e^{\i t}}^0 & = 1 \\
			      e^{\i t 0}         & = 1
		      \end{cases}\)
		\item \underline{Hérédité}
		      Soit \(n \in \N\) tel que \(P(n)\) est vrai, Montrons que \(P(n+1)\) est vrai :
		      \begin{align*}
			      e^{\i (n+1) t} & = e^{\i(n+1)t}                      \\
			                     & = e^{\i n t} \times e^{\i t}        \\
			                     & =\paren{e^{\i t}}^n \times e^{\i t} \\
			                     & = \paren{e^{\i t}}^{n+1}            \\
		      \end{align*}
	\end{itemize}
	Donc \(P(n+1)\) Vrai.
\end{dem}

\begin{appl}[Applications usuelles importantes]
	~\\
	Soit \(C = \sum_{k=0}^n \cos(kt)\) et \(S = \sum_{k=0}^n \sin(kt)\) avec \(n \in \N\) et \(t \in \R\)\\
	On Obtient des expressions simplifiées des sommes \(C\) et \(S\) par le calcul annexe suivant
	\[C+\i S = \sum_{k=0}^n e^{\i kt} = \sum_{k=0}^n \paren{e^{\i t}}^k =
		\begin{cases}
			n+1                               & \text{si } t\equiv 0\croch{2 \pi} \\
			\frac{1-e^{\i(n+1)t}}{1-e^{\i t}} & \text{ sinon }
		\end{cases}
	\]
	qui donne \[C+\i S = \begin{cases}
			n+1                                                                       & \text{si } t\equiv 0\croch{2 \pi} \\
			\frac{\paren{1-e^{\i(n+1)t}}\paren{1-e^{\i t}}}{2\paren{1-\cos\paren{t}}} & \text{ sinon }
		\end{cases} \]
	On conclut alors sur les valeurs de \(C\) et \(S\) en exhibant les parties réelle et imaginaire de \(C + \i S\).
\end{appl}

\section{Forme trigonométrique pour les nombres complexes non nuls}

\begin{defprop}
	Tout nombre complexe non nul \(z\) peut s’écrire sous la forme \[ z = re^{\i \theta}\]
	avec \(r\) un réel strictement positif et \(\theta\) un réel. Cette écriture est dite forme trigonométrique de \(z\). \\
	\underline{Attention} \\
	Dans cette écriture de \(z\).
	\begin{itemize}
		\item le réel strictement positif \(r\) est \underline{unique} car il est nécessairement égal à \(\abs{z}\)
		\item le réel \(\theta\) n'est \underline{pas unique} car si le réel \(\theta\) convient alors les réels \(\theta ' \equiv \theta \croch{2 \pi}\) conviennent.
	\end{itemize}
\end{defprop}

\begin{dem}
	Soit \(z\in \Cs\), alors \(\abs{z} \neq 0 \) donc \(\frac{z}{\abs{z}}\) existe avec  \(\abs{\frac{z}{\abs{z}}} = \frac{\abs{z}}{\abs{\abs{z}}} = \frac{\abs{z}}{\abs{z}} = 1\) \\
	Donc \(\frac{z}{\abs{z}} \in \U\) donc il existe \(\theta \in \R \) tel que \(\frac{z}{\abs{z}} = e^{\i \theta} \iff z = \abs{z}e^{\i \theta}\) \\
	Ceci prouve l'existence de l'écriture. \\
	\(r\) est unique car : \(\begin{cases}
		z & = re^{\i \theta}  \\
		z & = r'e^{\i \theta}
	\end{cases} \imp \begin{cases}
		\abs{z} & = r  \\
		\abs{z} & = r'
	\end{cases} \imp r = r'\)
\end{dem}
\begin{defprop}[Arguments]
	Soit \(z\) un nombre complexe non nul. Tous les nombres réels \(\theta\) tels que \(z\) peut s'écrire \[z = re^{\i \theta}\] avec \(r\) réel strictement positif sont dits arguments de \(z\) \\
	\underline{Remarque}\\
	Si \(\theta\) est un argument de \(z\) complexe non nul, on peut écrire \(\arg(z) \equiv \theta\croch{2\pi}\)
\end{defprop}

\begin{prop}
	Pour tous nombres complexes non nuls \(z\) et \(z'\), on a :
	\begin{enumerate}
		\item \(\arg\paren{zz'} \equiv \arg\paren{z}+\arg\paren{z'}\croch{2 \pi}\) \\
		\item \( \arg\paren{\frac{z}{z'}} \equiv \arg\paren{z}-\arg\paren{z'}\croch{2 \pi}\)
	\end{enumerate}
\end{prop}

\begin{defprop}[Transformation de \(a\cos(t) + b\sin(t)\) en \(A\cos(t-\phi)\)]
	Soit \(a, b\) et \(t\) des nombres réels avec \((a, b)\neq (0, 0)\). On peut écrire
	\[a\cos(t)+b\sin(t) = \Reel{\paren{a-\i b}\paren{\cos(t)+\i \sin(t)}} = \Reel{(a-\i b)e^{\i t}} \]
	puis \(a-\i b = Ae^{-\i \phi} \) avec \(A\) réel strictement positif et \(\phi\) un réel ce qui donne :
	\[a\cos(t)+b\sin(t) = \Reel{(a-\i b)e^{\i t}} = \Reel{Ae^{\i(t-\phi)}} \]
	Donc \(a\cos(t)+b\sin(t) = A\cos(t-\phi)\)
\end{defprop}

\section{Fonctions d'une variable réelle à valeurs complexes}

\begin{defi}
	Une fonction de variable réelle à valeurs complexes notée \(f\) est un objet mathématique qui, tout élément \(x\) d’une partie non vide de \(\R\), associe un et un seul nombre complexes noté \(f (x)\).
\end{defi}

\begin{defprop}[Ce qui s’étend aux fonctions de variable réelle à valeurs complexes]
	\begin{itemize}
		\item Notation fonctionnelle
		\item Domaine de définition
		\item Image d’un réel, antécédent d’un complexe
		\item Parité, imparité, périodicité
		\item Somme, produit, quotient de fonctions et multiplication d’une fonction par un complexe
		\item Dérivation
	\end{itemize}
\end{defprop}


\begin{defprop}[Ce qui ne s’étend pas aux fonctions de variable réelle à valeurs complexes]
	\begin{itemize}
		\item Composition de fonctions
		\item Monotonie
		\item Fonction majorée, minorée ou bornée
		\item Fonction réciproque
	\end{itemize}
\end{defprop}

\begin{defprop}[Dérivation]
	Soit \(I\) un intervalle de \(\R\) non vide et non réduit à un point.
	Soit \(f\) une fonction définie sur \(I\) à valeurs complexe. \\
	On note \(\Reel{f} :I \to \R\) et \(\Ima{f}:I\to\R\) les fonctions d’une variable réelle à valeurs réelles définies par :
	\[\quantifs{\forall x \in I}\paren{\Reel{f}}(x) = \Reel{f(x)} \text{ et } \paren{\Ima{f}}(x) = \Ima{f(x)} \]
	On dit que : \begin{itemize}
		\item \(f\) est dérivable en \(x_0\) si les fonctions \(\Reel{f}\) et \( \Ima{f} \) sont dérivables en \(x_0\)
		\item \(f\) est dérivable sur \(I\) si les fonctions \(\Reel{f}\) et \( \Ima{f} \) sont dérivables sur \(I\)
	\end{itemize}
	Selon le cas de figure, on appelle :
	\begin{itemize}
		\item nombre dérvée de \(f\) en \(x_0\) et on note \(f'(x_0)\) le nombre complexe suivant : \[f'(x_0) = \paren{\Reel{f}'(x_0)} + \ \paren{\Ima{f}'(x_0)}\]
		\item fonction dérivée de \(f\) sur \(I\) et on note \(f'\) la fonction de variable réelle à valeurs complexes suivante :
		      \[ f' = \paren{\Reel{f}'} + \ \paren{\Ima{f}'}\]
	\end{itemize}
\end{defprop}

\begin{prop}
	\begin{enumerate}
		\item \underline{Combinaison linéaire}\\
		      Soit \(f\) et \(g\) deux fonctions définies sur \(I\) et à valeurs complexes et \((\alpha, \beta)\) un couple de complexes. Si \(f\) et \(g\) sont dérivables sur \(I\) alors \(\alpha f + \beta g \) est dérivable sur \(I\) et sa dérivée vérifie :
		      \[\paren{\alpha f + \beta g}' = \alpha f'+\beta g'\]
		\item \underline{Produit}\\
		      Soit \(f\) et \(g\) deux fonctions définies sur \(I\) et à valeurs complexes . Si \(f\) et \(g\) sont dérivables sur \(I\) alors \(fg\) est dérivable sur \(I\) et sa dérivée vérifie :
		      \[\paren{fg}' = f'g+fg'\]

		\item \underline{Quotient}\\
		      Soit \(f\) et \(g\) deux fonctions définies sur \(I\) et à valeurs complexes tel que \(g\) ne s’annule pas sur \(I\). Si \(f\) et \(g\) sont dérivables sur \(I\) alors \(\frac{f}{g}\) est dérivable sur \(I\) et sa dérivée vérifie :
		      \[\paren{\frac{f}{g}}' = \frac{f'g-g'f}{g^2}\]
	\end{enumerate}
\end{prop}

\begin{appl}[exemple important]
	Soit \(\phi\) une fonction définie sur \(I\) à valeurs complexes.
	On note \(f : I \to \C\) la fonction définie sur \(I\) par :
	\[\forall t \in I, f(t) = e^{\Reel{\phi(t)}}e^{\i\Ima{\phi(t)}}\]
	Si \(\phi\) est dérivable sur \(I\) alors \(f\) est dérivable sur \(I\) et sa dérivée vérifie :
	\[\forall t \in I,f'(t) = \phi '(t)f(t) \]
	\underline{Remarque} \\
	La fonction \(f\) sera aussi notée \(f = \exp(\phi)\) après étude de l’exponentielle complexe dans le chapitre \guillemets{Nombres complexes (\(2\))} ce qui permettra d’écrire \((\exp(\phi))' = \phi' \exp(\phi)\) et donc d’étendre une propriété déjà connue dans le cas où \(\phi\) est à valeurs réelles.
\end{appl}

\chapter{Fonctions usuelles : Rappel et complément}

\minitoc
\section{Fonction exponentielle}
\begin{defprop}
	Il existe une unique fonction \(f\) définie sur \(\R\), dérivable sur \(\R\) à valeurs réelles vérifiant \(f' = f\) et \(f(0) = 1\) \\
	Cette fonction, appelée fonction exponentielle et notée \(x\mapsto \exp(x)\) ou \(x\mapsto e^x\) vérifie :
	\begin{itemize}
		\item pour tout \(x\) et \(y\) des réels , \(e^{x+y}  =e^xe^y\)
		\item pour tout x réel, \(e^{-x} = \frac{1}{e^x} \)
		\item pour tout \(x\) réel et tout \(n\) entier relatif, \(e^{nx} = \paren{e^x}^n\)
		\item pour tout \(x\) réel, \(e^x >0\)
		\item la fonction \(\exp\) est définie et dérivable sur \(\R\).
		\item la dérivée de \(\exp\) sur \(\R\) est \(\exp\).
		\item la fonction \(\exp\) est strictement croissante sur \(\R\).
		\item \(\lim_{x\to\minf}e^x =0\)
		\item \(\lim_{x\to\pinf} e^x = \pinf \)
		\item \(\lim_{x\to 0} \frac{e^x-1}{x} = 1\)
		\item pour tout réel \(x\), \(e^x\geq 1+x\)
	\end{itemize}
\end{defprop}

\section{Fonction logarithmes}
\begin{defprop}
    La fonction réciproque de la fonction exponentielle est appelée fonction logarithme népérien et notée \(\ln\) . \\
Elle vérifie : \begin{itemize}
    \item  pour tous \(x\) et \(y\) réels strictement positifs, \(ln(xy) = ln(x) + ln(y)\) 
    \item pour tout \(x\) réel strictement positif, \(\ln\paren{\frac{1}{x}} = -\ln(x) \)
    \item \(\ln(1) = 0\)
    \item pour tout \(x\) réel strictement positif et tout \(n\) entier relatif, \(\ln(x^n) = n\ln(x)\)
    \item la fonction \(\ln\) est définie et dérivable sur \(\Rps\).
	\item la dérivée de \(\ln\) sur \(\Rps\) est \(x\mapsto\frac{1}{x}\).
	\item la fonction \(\ln\) est strictement croissante sur \(\Rps\).
	\item \(\lim_{x\to 0}\ln(x) =\pinf\)
	\item \(\lim_{x\to\pinf} e^x = \pinf \)
	\item \(\lim_{x\to 0} \frac{\ln(x+1)}{x} = 1\)
	\item pour tout réel \(x>-1\), \(\ln(1+x)\geq x\)
\end{itemize}
\end{defprop}

\begin{defprop}[logarithme en base \(2\) et en base \(10\)]
    Les fonctions logarithme en base \(2\), notée \(\log_2\), et logarithme en base \(10\) notée \(\log_{10}\) sont définie sur \(\Rps\) par, pour tout réel \(x\) strictement positif : 
    \[\log_2(x) = \frac{\ln(x)}{\ln(2)}\text{ et }\log_{10}(x) =\frac{\ln(x)}{\ln(10)} \]
    On as aussi : \begin{itemize}
        \item \(\log_2(2) = 1\) et \(\log_{10}(10) = 1\)
        \item pour tout \(x\) entier relatif, \(\log_2(2^n) = n \) et \(\log_{10}(10^n) =n\)
        \item \(\log_2\) et \(\log_{10}\) ont même monotonie et même limites aux bornes de \(\Rps\) que la fonction \(\ln\)
    \end{itemize}
\end{defprop}

\section{Fonctions hyperboliques}
\begin{defprop}
    \begin{enumerate}
        \item On appelle cosinus hyperbolique la fonction, notée \(\ch\) définie \(\R\) par, pour tout \(x\) réel, 
        \[\ch(x) = \frac{e^x+e^{-x}}{2}\]
        \item On appelle sinus hyperbolique la fonction, notée \(\sh\) définie \(\R\) par, pour tout \(x\) réel, 
        \[\sh(x) = \frac{e^x-e^{-x}}{2}\]
    \end{enumerate}
\end{defprop}

\begin{defprop}[Relation fondamentale de la trigonométrie hyperbolique]
    Pour tout réel \(x\),on a : \[\ch^2(x)-\sh^2(x) = 1\]
\end{defprop}

\begin{dem}
\[\quantifs{\forall x \in \R} \ch^2(x)-\sh^2(x) = \paren{\ch(x)+\sh(x)}\paren{\ch(x)-\sh(x)} = \paren{e^x}\paren{e^{-x}} = e^0 = 1 \]
\end{dem}

\begin{defprop}[étude de la fonction \(\ch\)]
    \begin{enumerate}
        \item La fonction \(\ch\) est définie et dérivable sur \(\R\)
        \item la dérivée de \(\ch\) sur \(\R\) est la fonction \(\sh\)
        \item la fonction \(\ch\) est paire avec \(\ch(0) =1\)
        \item la fonction \(\ch\) est : 
        \begin{enumerate}
            \item strictement décroissante sur \(\Rms\)
            \item strictement croissante sur \(\Rps\)
        \end{enumerate}
        \item \(\lim_{x\to\minf} \ch(x) = \pinf \)
        \item \(\lim_{x\to\pinf} \ch(x) = \pinf \)
    \end{enumerate}    
\end{defprop}

\begin{defprop}[étude de la fonction \(\sh\)]
    \begin{enumerate}
        \item La fonction \(\sh\) est définie et dérivable sur \(\R\)
        \item la dérivée de \(\sh\) sur \(\R\) est la fonction \(\ch\)
        \item la fonction \(\sh\) est impaire avec \(\sh(0) =0\)
        \item la fonction \(\sh\) est strictement croissante sur \(\R\)
        \item \(\lim_{x\to\minf} \sh(x) = \minf \)
        \item \(\lim_{x\to\pinf} \sh(x) = \pinf \)
    \end{enumerate}    
\end{defprop}

\section{Tangente hyperbolique}
\begin{defprop}
    On appelle tangente hyperbolique la fonction, notée, \(\tth\), définie sur \(\R\) par, pour tout \(x\) réel \[\tth(x) = \frac{\ch(x)}{\sh(x)} = \frac{e^x-e^{-x}}{e^x+e^{-x}}\].
\end{defprop}
\begin{defprop}[étude de la fonction \(\tth\)]
    \begin{enumerate}
        \item La fonction \(\tth\) est définie et dérivable sur \(\R\)
        \item la dérivée de \(\tth\) sur \(\R\) est la fonction \(1-\tth^2=\frac{1}{\ch^2}\)
        \item la fonction \(\tth\) est impaire avec donc \(\tth(0) =0\)
        \item la fonction \(\tth\) est strictement croissante sur \(\R\)
        \item \(\lim_{x\to\minf} \tth(x) = -1 \)
        \item \(\lim_{x\to\pinf} \tth(x) = 1 \)
    \end{enumerate}    
\end{defprop}

\begin{defprop}[formule d'addition et de duplication]
    Pour tout couple de réel \((a,b)\), on a:
    \begin{enumerate}
        \item \(\ch(a+b) = \ch(a)\ch(b)+\sh(a)\sh(b)\)
        \item \(\ch(a-b) = \ch(a)\ch(b)-\sh(a)\sh(b)\)
        \item \(\sh(a+b) = \ch(a)\sh(b)+\sh(a)\ch(b)\)
        \item \(\sh(a-b) = \ch(a)\sh(b)-\sh(a)\ch(b)\)
        \item \(\tth(a+b) = \frac{\tth(a)+\tth(b)}{1+\tth(a)\tth(b)}\)
        \item \(\tth(a-b) = \frac{\tth(a)-\tth(b)}{1-\tth(a)\tth(b)}\)
        \item \(\ch(2a) = \ch^2(a)-\sh^2(a) = 2\ch^2(a)-1 = 2\sh^2(a) +1\)
        \item \(\sh(2a) = 2\sh(a)\ch(a)\)
        \item \(\tth(2a) = \frac{2\tth(a)}{1+\tth^2(a)}\)
    \end{enumerate}
\end{defprop}

\section{Arccos}
\begin{defprop}
    La fonction \(c : \intervii{0}{\pi} \to \intervii{-1}{1}\) définie par :

    \[\text{Pour tout } x \text{ dans },c(x) = \cos(x)\] 
    
    est une bijection de \(\intervii{0}{\pi}\) sur \(\intervii{-1}{1}\) de bijection réciproque \(c^{-1} : \intervii{-1}{1} \to \intervii{0}{\pi}\) notée \(\Arccos\)
    \\ Autrement dit : 
    \begin{itemize}
        \item pour tout réel \(y\) dans \(\intervii{-1}{1}\), l'équation \(y=\cos(x)\) admet une unique solution dans \(\intervii{0}{\pi}\)
        \item pour tout réel \(y\) dans \(\intervii{-1}{1}\), \(\Arccos (y)\) est l'unique réel de \(\intervii{0}{\pi}\) donc le cosinus est égal à \(y\)
    \end{itemize}
    Par ailleurs la fonction \(\Arccos\) possède ces propriétés : 
    \begin{enumerate}
        \item la fonction \(\Arccos\) est définie sur \(\intervii{-1}{1}\) et dérivable sur \(\intervee{-1}{1}\)
        \item la dérivée de \(\Arccos\) sur \(\intervee{-1}{1}\) est la fonction \(\Arccos' : x\mapsto \frac{-1}{\sqrt{1-x^2}} \)
        \item la fonction \(\Arccos\) est strictement décroissante sur \(\intervii{-1}{1}\)
    \end{enumerate}
\end{defprop}

\section{Arcsin}
\begin{defprop}
    La fonction \(s : \intervii{-\frac{\pi}{2}}{\frac{\pi}{2}} \to \intervii{-1}{1}\) définie par :

    \[\text{Pour tout } x \text{ dans },s(x) = \sin(x)\] 
    
    est une bijection de \(\intervii{-\frac{\pi}{2}}{\frac{\pi}{2}}\) sur \(\intervii{-1}{1}\) de bijection réciproque \(s^{-1} : \intervii{-1}{1} \to \intervii{-\frac{\pi}{2}}{\frac{\pi}{2}}\) notée \(\Arcsin\)
    \\ Autrement dit : 
    \begin{itemize}
        \item pour tout réel \(y\) dans \(\intervii{-1}{1}\), l'équation \(y=\sin(x)\) admet une unique solution dans \(\intervii{-\frac{\pi}{2}}{\frac{\pi}{2}}\)
        \item pour tout réel \(y\) dans \(\intervii{-1}{1}\), \(\Arcsin (y)\) est l'unique réel de \(\intervii{-\frac{\pi}{2}}{\frac{\pi}{2}}\) donc le sinus est égal à \(y\)
    \end{itemize}
    Par ailleurs la fonction \(\Arcsin\) possède ces propriétés : 
    \begin{enumerate}
        \item la fonction \(\Arcsin\) est définie sur \(\intervii{-1}{1}\) et dérivable sur \(\intervee{-1}{1}\)
        \item la dérivée de \(\Arcsin\) sur \(\intervee{-1}{1}\) est la fonction \(\Arcsin' : x\mapsto \frac{1}{\sqrt{1-x^2}} \)
        \item la fonction \(\Arcsin\) est impaire sur \(\intervee{-1}{1}\)
        \item la fonction \(\Arcsin\) est strictement croissante sur \(\intervii{-1}{1}\)
    \end{enumerate}
\end{defprop}

\section{Arctan}
\begin{defprop}
    ~\\
    La fonction \(t : \intervee{-\frac{\pi}{2}}{\frac{\pi}{2}} \to \R\) définie par :

    \[\text{Pour tout } x \text{ dans },t(x) = \tan(x)\] 
    
    est une bijection de \(\intervee{-\frac{\pi}{2}}{\frac{\pi}{2}}\) sur \(\R\) de bijection réciproque \(t^{-1} : \R \to \intervee{-\frac{\pi}{2}}{\frac{\pi}{2}}\) notée \(\Arctan\)
    \\ Autrement dit : 
    \begin{itemize}
        \item pour tout réel \(y\) dans \(\R\), l'équation \(y=\tan(x)\) admet une unique solution dans \(\intervee{-\frac{\pi}{2}}{\frac{\pi}{2}}\)
        \item pour tout réel \(y\) dans \(\R\), \(\Arctan (y)\) est l'unique réel de \(\intervee{-\frac{\pi}{2}}{\frac{\pi}{2}}\) donc la tangente est égal à \(y\)
    \end{itemize}
    Par ailleurs la fonction \(\Arctan\) possède ces propriétés : 
    \begin{enumerate}
        \item la fonction \(\Arctan\) est définie et dérivable sur \(\R\)
        \item la dérivée de \(\Arctan\) sur \(\R\) est la fonction \(\Arctan' : x\mapsto \frac{1}{1+x^2} \)
        \item la fonction \(\Arctan\) est impaire sur \(\R\)
        \item la fonction \(\Arctan\) est strictement croissante sur \(\R\)
        \item \(\lim_{x\to\minf} \Arctan(x) = -\frac{\pi}{2}\)
        \item \(\lim_{x\to\pinf} \Arctan(x) = \frac{\pi}{2}\)
    \end{enumerate}
\end{defprop}

\section{Fonction puissances réelles}
\begin{defi}
    Soit \(\alpha\) un réel.

    La fonction \(f_{\alpha}\) définie sur \(\Rps\) par 

    \[\quantifs{\forall x \in \Rps} f_{\alpha}(x) = e^{\alpha \ln(x)} \]
    est notée \(f_{\alpha}: x \mapsto x^{\alpha}\) et appelée fonction puissances (réelle). Elle respecte ces propriétés : 
    \begin{itemize}
        \item la fonction \(x\mapsto x^{\alpha}\) est définie et dérivable sur \(\Rps\)
        \item la dérivée de \(x\mapsto x^{\alpha}\) sur \(\Rps\) est \(x\mapsto \alpha x^{\alpha-1}\)
        \item la fonction \(x\mapsto x^{\alpha}\) est : 
        \begin{itemize}
            \item strictement croissante sur \(\Rps\) pour \(\alpha > 0 \) 
            \item strictement décroissante  sur \(\Rps\) pour \(\alpha < 0 \) 
        \end{itemize}
        \item  \(\lim_{x\to 0} x^{\alpha} = \begin{cases}
            0 &\text{ pour } \alpha >0 \\
            \pinf &\text{ pour } \alpha <0
        \end{cases}\)
        \item  
        \(\lim_{x\to \pinf} x^{\alpha} = 
        \begin{cases}
            \pinf &\text{ pour } \alpha >0 \\
             0 &\text{ pour } \alpha <0
        \end{cases}
        \)
    \end{itemize}
\end{defi}

\begin{prop}
    Pour tout couple de réels \(\alpha,\beta\) et tout couple de réels strictement positifs \((x,y)\), on a:
    \[\ln(x^{\alpha}) = \alpha \ln(x) \qquad (xy)^{\alpha} = x^{\alpha}y^{\alpha} \qquad x^{\alpha+\beta} = x^{\alpha}x^{\beta} \qquad \paren{x^{\alpha}}^{\beta} = x^{\alpha \beta}\]
\end{prop}

\begin{defprop}[cas particulier des puissances entières]
    Les fonctions vues ci-dessus étendent les notions de puissances entières déjà connues sur \(\R\) ou \(\Rs\) :
    \begin{itemize}
        \item pour tout entier naturel \(n\), la fonction \(f_n : x\mapsto \prod_{k=1}^{n}x\) est notée \(x\mapsto x^n\)\\
        elle est définie sur \(\R\), dérivable sur \(\R\) et de dérivée \(x\mapsto nx^{n-1}\)
        \item pour tout entier relatif strictement négatif \(n\), la fonction \(f_n : x\mapsto \prod_{k=1}^{-n}x^{-1}\) est notée \(x\mapsto x^n\)\\
        elle est définie sur \(\Rs\), dérivable sur \(\Rs\) et de dérivée \(x\mapsto nx^{n-1}\)
    \end{itemize}
\end{defprop}

\section{croissance comparées}
\begin{defprop}[Cas des fonctions \(x \mapsto \ln (x), x \mapsto x^{\alpha}\) et \(x \mapsto e^x\) avec \(\alpha >0\) ]
    Pour tout \(\alpha\) réel strictement positif, lems croissances comparées des fonctions \(x \mapsto \ln (x), x \mapsto x^{\alpha}\) et \(x \mapsto e^x\) se résument à : 
    \[\lim_{x\to\pinf} \frac{\ln(x)}{x^{\alpha}} = 0 \qquad \lim_{x\to\pinf} \frac{x^{\alpha}}{e^x} = 0 \qquad \lim_{x\to 0} x^{\alpha} \ln(x) = 0\]
    \underline{Remarques}:
    On en déduit les croissances comparées en \(\pinf\) des fonctions précédentes prises deux à deux :
    \begin{itemize}
        \item comparaison du logarithme népérien avec les puissances réelles ou l’exponentielle en \(\pinf\) :
        \[\lim_{x\to\pinf} \frac{\ln(x)}{x^{\alpha}} = 0 \qquad \lim_{x\to\pinf} \frac{\ln(x)}{e^x} = 0\]
        \item comparaison des puissances réelles avec le logarithme népérien ou l’exponentielle en \(\pinf\)
        \[\lim_{x\to\pinf} \frac{x^{\alpha}}{\ln(x)} = \pinf \qquad \lim_{x\to\pinf} \frac{x^{\alpha}}{e^x} = 0\]
        \item comparaison de l’exponentielle avec le logarithme népérien ou les puissances réelles en \(\pinf\)
        \[\lim_{x\to\pinf} \frac{e^x}{\ln(x)} = \pinf \qquad \lim_{x\to\pinf} \frac{e^x}{x^{\alpha}} = \alpha\]
    
    \end{itemize}
\end{defprop}

\begin{defprop}[Cas des fonctions \(x \mapsto \abs{\ln (x)}^{\beta} , x \mapsto x^{\alpha} \)et \(x \mapsto e^{\gamma x}\)]
    Pour tous réels strictement positifs \(\alpha , \beta \) et \(\gamma \), les croissances comparées des fonctions  \(x \mapsto \abs{\ln (x)}^{\beta} , x \mapsto x^{\alpha} \)et \(x \mapsto e^{\gamma x}\) se résument à :
    \[\lim_{x\to\pinf} \frac{\abs{\ln (x)}^{\beta}}{x^{\alpha}} = 0 \qquad \lim_{x\to\pinf} \frac{x^{\alpha}}{e^{\gamma x}} = 0 \qquad \lim_{x\to 0} x^{\alpha}\abs{\ln (x)}^{\beta} = 0 \] 

\end{defprop}

\chapter{Nombres complexes (\(2\))}

\minitoc

\section{Équations algébriques}
\subsection{Préliminaires}
\begin{defi}[Définition d'une fonction polynomiale]
    Une fonction \(P:\C\to\C\) est dite fonction polynomiale à coefficients complexes s'il existe un entier naturel \(n\) et un \(n+1\)-uplet de nombres complexes \((b_0,b_1,\dots,b_n)\) tel que pour tout \(z\) de \(\C\), 
    \[P(z) = b_0+b_1z+\dots+b_nz^n = \sum_{k=0}^n b_k z^k\]
\end{defi}
\begin{prop}[Propriétés de factorisation]
    Soit \(P\) une fonction polynomiale à coefficients complexes et \(a\) un nombre complexe.\\
    Si \(a\) est une racine de \(P\) , autrement dit si \(P (a) = 0\), alors il existe une fonction polynomiale à coefficients complexes \(Q\) tel que, pour tout \(z\) de \(\C\), on a :
    \[P(z) = (z-a)Q(z)\]
\end{prop}

\subsection{Résolution des équations du second degré dans \(\C\)}
\begin{defprop} [cas particulier des équations du type \(z^2 = z_0\)]
    Soit \(z_0\) et \(z\) des nombres complexes de formes algébriques respectives \(x_0 + \i y_0\) et \(x+\i y\) 
    \[z^2 = z_0 \text{ six et seulement si ,} 
    \begin{cases}
        x^2-y^2 &= x_0 \\
        x^2+y^2 &= \sqrt{x_0^2 + y_0^2}\\
        2xy &= y_0 \\
    \end{cases}
    \]
\end{defprop}

\begin{defprop}[Cas général]
    soit \(a,b\) et \(c\) des nombres complexes avec \(a\) non nul. \\
    \begin{itemize}
        \item \underline{Racines} \\Les solutions de l'équation polynomiale \(az^2+bz+c=0\) d'inconnue le nombre complexe \(z\) sont : 
        \[z_1 = \frac{-b-\delta}{2a} \text{ et } z_2 = \frac{-b+\delta}{2a}\]
        où \(\delta\) est une "racine carré" de \(\Delta = b^2 -4ac\), autrement dit où \(\delta\) est un nombre complexe vérifiant : 
        \[\delta^2 = \Delta\]
        \item \underline{Somme et produit des racines (formules de Viète)} \\
        Les racines \(z_1\) et \(z_2\) de la fonction polynomiale \(P:z\mapsto az^2 + bz +c \) vérifient :
        \[z_1+z_2 = -\frac{b}{a} \text{ et } z_1z_2 = \frac{c}{a}\]

    \end{itemize}
\end{defprop}

\begin{dem}[Formule des solutions du cas général]
    soit \(a,b\) et \(c\) des nombres complexes avec \(a\) non nul. \\
    Soit \(z \in \C\)
    \begin{align*}
        az^2+bz+c &= a\paren{z^2+\frac{b}{a}z+\frac{c}{a}} \\
        &=a\paren{\paren{z+\frac{b}{2a}}^2+\frac{c}{a} - \frac{b^2}{4a^2}} \\
        &=a\paren{\paren{z+\frac{b}{2a}}^2 - \frac{b^2-4ac}{4a^2}} \\
        &=a\paren{\paren{z+\frac{b}{2a}}^2 - \frac{\Delta}{\paren{2a}^2}}  \tag*{ on pose \(\Delta = b^2-4ac\)}\\
        &=a\paren{\paren{z+\frac{b}{2a}}^2 - \paren{\frac{\delta}{2a}}^2} \tag*{ on pose \(\delta\)  comme étant la "racine carré" de \(\Delta\)}\\
        &=a\paren{z+\frac{b}{2a}-\frac{\delta}{2a}}\paren{z+\frac{b}{2a}+\frac{\delta}{2a}} \\
        &=a\paren{z-z_1}\paren{z-z_2} \text{ avec } 
        \begin{cases}
            z_1 &= \frac{-b-\delta}{2a} \\\\
            z_2 &= \frac{-b+\delta}{2a}
        \end{cases}
    \end{align*}
\end{dem}

\begin{dem}[Formule de viète]
    soit \(a,b\) et \(c\) des nombres complexes avec \(a\) non nul. \\
    Soit \(P:z\mapsto az^2+bz+c\) 
    \[P(z) = az^2+bz+c = a(z-z_1)(z-z_2) = a(z^2-(z_1+z_2)z+z_1z_2)\]
    donc par identification : 
    \[
    \begin{cases}
        b &=-a(z_1+z_2) \\
        c &= az_1z_2 
    \end{cases} \iff\begin{cases}
        -\frac{b}{a} &=z_1+z_2 \\
        \frac{c}{a} &= z_1z_2 
    \end{cases} 
    \]
\end{dem}

\subsection{Résolution des équations du type \(z^n = z_0\) dans \(\C\) avec \(n\in \Ns\)}

\begin{defi}
    Soit \(n\) un entier naturel non nul et \(z_0\) un nombre complexe. \\
    On appelle racine \(n\)- ième de \(z_0\) tout nombre complexe tel que \(z^n = z_0\)
\end{defi}

\begin{defprop}[Cas particulier où \(z_0 = 1\)]
    \begin{itemize}
        \item \underline{Racines} \\
        Il y a \(n\) racine \(n\)-ième de l'unité qui sont les nombres complexes suivants : 
        \[\omega_k = e^{\i\frac{2k\pi}{n}} \text{ avec } k \in \interventierii{0}{n-1}\] 
        \item \underline{L'ensemble des racines} \\
        \begin{itemize}
            \item L'ensemble des racines \(n\)-ièmes de l'unité est noté 
            \[\U_n = \accol{z\in\R \tq z^n = 1}\]
            \item Les points dont les affixes sont les racines \(n\)-ièmes de l’unité sont les sommets d’un polygone régulier à \(n\) côtés, de centre \(O\) et inscrit dans \(\U\).
        \end{itemize}
    \end{itemize}
\end{defprop}
\begin{dem}
    Soit \(z \in\C\) tel que \(z^n = 1\) \\
    \(z=0\) n'est pas solution donc \(\quantifs{\exists (r,\theta) \in \Rps\times\R}z = re^{\i \theta}\) 
    \begin{align*}
        z^n=1 &\iff r^ne^{\i \theta n }= 1e^{\i \times 0} \\
        &\iff  
            \begin{cases}
                r^n &= 1 \\
                n \theta &\equiv 0 [2\pi]
            \end{cases}\\
        &\iff \begin{cases}
                r &= 1 \\
                \theta \equiv 0 \croch{\frac{2\pi}{n}}
            \end{cases}\\
    \end{align*}
    Ainsi \(S = \U_n =\accol{e^{\i \frac{k2\pi}{n}}\tq k \in \Z}\) \\
    On note \(\fonction{f}{\Z}{\C}{k}{e^{\i \frac{k2\pi}{n}}}\) alors on sait que \(f\) est \(n\) périodique car \(\forall  k \in \Z, \begin{cases}
        k+n &\in \Z \\
        k-n &\in \Z
    \end{cases} \)
    et 
    \begin{align*}
        f(k+n) &= e^{\i \frac{2(k+n)\pi}{n}} \\
               &= e^{\i \frac{2k\pi}{n}}\times e^{\i \frac{2n\pi}{n}} \\
               &= e^{\i \frac{2k\pi}{n}}\times 1 \\
               &= f(k)
    \end{align*}
    Donc \(S = \U_n =\accol{e^{\i \frac{k2\pi}{n}}\tq k \in \interventierii{0}{n-1}}\). \\
    Montrons que \(\U_n\) contient \(n\) élément autrement dit que: 
    \[\quantifs{\forall (k,k') \in\interventierii{0}{n-1}^2;k<k'} \imp e^{\i \frac{k2\pi}{n}} \neq e^{\i \frac{k'2\pi}{n}}\]
    \underline{Par l'absurde :}\\
    Soit \(k\) et \(k'\) dans \(\interventierii{0}{n-1}\) avec \(k<k'\), supposons que \(e^{\i \frac{k2\pi}{n}} = e^{\i \frac{k'2\pi}{n}}\) \\
    alors \(\frac{k2\pi}{n} \equiv \frac{k'2\pi}{n} \croch{2\pi}\)\\
    donc il existe \(k'' \in \Ns\) tel que \(\frac{k2\pi}{n} -\frac{k'2\pi}{n} = 2 k'' \pi\) car \(k'-k >0\)\\
    Ainsi \(k'-k = nk''\) avec \(\begin{cases}
        k'-k \in \interventierii{1}{n-1} &\text{ car } 0\leq k<k'\leq n-1 \\
        nk'' \in \interventierie{n}{\pinf} & \text{ car } k''\in \Ns
    \end{cases}\) \\
    Ce qui est absurde et prouve que \(e^{\i \frac{k2\pi}{n}} \neq e^{\i \frac{k'2\pi}{n}}\)\\
    \conclusion \\
    Il y a exactement \(n\) racine \(n\)-ièmes de l'unité qui sont les \( \omega_k = e^{\i \frac{k2\pi}{n}}\) pour \(k\in \interventierii{0}{n-1}\)
\end{dem}


\begin{defprop}[Cas général]
    Il y a \(n\) racines \(n\)- ièmes pour le nombre complexe non nul \(z_0\) de forme trigonométrique \(z_0 = r_0e^{\i\theta_0}\) qui sont les nombres complexes suivants :
    \[\sqrt[n]{r_0}e^{\i \paren{\frac{\theta_0}{n}+\frac{2k\pi}{n}}} \text{ avec }k \in \interventierii{0}{n-1}\]
\end{defprop}

\begin{ex}
\[\U_3 = \accol{1,\exp\paren{\frac{2\i \pi}{3}},\exp \paren{\frac{4 \i \pi}{3}}}\]

\[\U_4 = \accol{1,\exp\paren{\frac{2\i \pi}{4}},\exp \paren{\frac{4 \i \pi}{4}},\exp \paren{\frac{6 \i \pi}{4}}} = \accol{1,\i,-1,-\i}\]

\[\U_4 = \accol{1,\exp\paren{\frac{2\i \pi}{5}},\exp \paren{\frac{4 \i \pi}{5}},\exp \paren{\frac{6 \i \pi}{5}},\exp \paren{\frac{8\i \pi}{5}}} \]

\end{ex}

\section{Exponentielle complexe}
\begin{defi}
Pour tout nombre complexe \(z\), on appelle exponentielle de \(z\) le nombre complexe noté \(e^z\) le nombre complexe \(e^z\) défini par : 
\[e^z = e^{\Reel{z}}e^{\i \Ima{z}}\]
dont le module est \(\abs{e^z} = e^{\Reel{z}}\) et les arguments vérifient \(\arg(e^z)\equiv \Ima{z} [2\pi]\)
\end{defi}

\begin{prop}
    Soit un couple de nombres complexes \((z,z')\)
    \begin{itemize}
        \item on a l'égalité suivante : 
        \[e^{z+z'} = e^ze^{z'}\]
        on en déduit les propriétés suivantes :
        \begin{itemize}
            \item \(\frac{1}{e^z} = e^{-z}\)
            \item pour tout entier relatif \(n\), on a: \(e^{nz} = \paren{e^z}^n\)
        \end{itemize}
        \item \(e^z = e^{z'}\) si et seulement si, \(z-z' \in 2\i\pi \Z\) en notant \(2\i \pi \Z =\accol{2\i k \pi \tq k\in \Z}\)
    \end{itemize}
\end{prop}
\begin{defprop}[Résolution de l'équations \(e^z = a\) avec \(a\) un nombre complexe]
    Soit \(a\) un nombre complexe. \\
    \begin{itemize}
        \item Si \(a\) est nul alors l'équation \(e^z = a\) n'a pas de solution dans \(\C\)
        \item Si \(a\) est non nul alors l'équation \(e^z = a\) possède une infinité de solutions dans \(\C\) qui sont les nombres complexes 
        \[z= \ln(z)+\i \theta\]
        avec \(r\) le module de \(a\) et \(\theta\) un argument de \(a\).
    \end{itemize}
\end{defprop}

\section{Interprétations géométriques}
\begin{defprop}[Module et arguments de \(\frac{z'-\omega}{z-\omega}\)]
    Soit \(\omega,z \) et \(z'\) des nombres complexes tel que \(\omega \neq z\) et \(\omega \neq z'\) de points images notés \(\Omega,M\) et \(M'\). \\
    Alors : 
    \[\abs{\frac{z'-\omega}{z-\omega}} = \frac{\Omega M'}{\Omega M} \text{ et } \arg\paren{\frac{z'-\omega}{z-\omega}} = \paren{\overrightarrow{\Omega M},\overrightarrow{\Omega M'}}[2\pi]\]

\end{defprop}
\begin{defprop}[Traduction de l’alignement et l’orthogonalité]
     Soit \(\Omega,M \) et \(M'\) trois points du plan tels que \(\Omega \neq M\) et \(\Omega \neq M'\) d'affixes respectivement notées \(\omega,z\) et \(z'\)
     \begin{itemize}
        \item Les points  \(\Omega,M \) et \(M'\) sont alignés si, et seulement si,  \(\frac{z'-\omega}{z-\omega}\) est un réel
        \item Les droites \(\Omega M\) et \(\Omega M'\) sont orthogonales si, et seulement si, \(\frac{z'-\omega}{z-\omega}\) est un imaginaire pur.
     \end{itemize}
\end{defprop}
\begin{defprop}[Ecriture complexe de transformations du plan vues au collège]
    Dans ce paragraphe, \(M\) et \(M'\) sont deux points du plan complexe d’affixes respectives \(z\) et \(z'\).
    \begin{itemize}
        \item \underline{Translation} \\
        Soit \(b\) un nombre complexe. \\
        \(M'\) est l'image par \(M\) par la translation de vecteur d'affixe \(b\) si, et seulement si \[z' = z+b\]
        \item \underline{Homothétie} \\
        Soit \(\alpha\) un nombre réel et \(\Omega\) un point du plan d'affixe \(\omega\). \\
        \(M'\) est l'image par \(M\) par l'Homothétie de centre \(\Omega\) et de rapport \(\alpha\)  si, et seulement si \[z'-\omega = \alpha(z-\omega)\]
        \item \underline{Rotation} \\
        Soit \(\theta\) un nombre réel et \(\Omega\) un point du plan d'affixe \(\omega\). \\
        \(M'\) est l'image par \(M\) par la rotation de centre \(\Omega\) et d'angle \(\theta\)  si, et seulement si \[z'-\omega = e^{\i \theta}(z-\omega)\]
    \end{itemize}
\end{defprop}

\begin{defprop}[Applicaitons \(z\mapsto az+b\) avec \((a,b) \in \Cs\times\C\)]
    Soit  \((a,b) \in \Cs\times\C\).L'application \(f\) de \(\C\) dans \(\C\) définie par 
    \[f(z) = az+b\]
    est dite similitude directe. \\
    \underline{Interprétation géométrique :}
    Pour tout \(z\in \C\), on note \(M\) le point d'affixe \(z\) et \(M'\) le point d'affixe \(z' = f(z)\)
    \begin{itemize}
        \item \underline{Cas où \(a=1\)} \\
        On a alors l'équivalence suivante : \(z' = f(z)\) so et seulement si, \(z'-z = b\) \\
        L'application \(f\) est donc la translation de vecteur d'affixe \(b\).
        \item\underline{Cas où \(a\neq1\)}\\ 
        \(f\) admet alors un point fixe \(\omega\) donné par \(\omega = \frac{b}{1-a} \) dont le point image est noté \(\Omega\) \\
        On en déduit les équivalences suivantes :  \\
        \begin{align*}
            z' = f(z) &\text{ si, et seulement si, } z'-\omega = a'(z-\omega) \\
                      &\text{ si, et seulement si, } z'-\omega  = \abs{a}\paren{e^{\i \arg(a)}(z-\omega)} \\
                      &\text{ si, et seulement si, } z'-\omega  = e^{\i \arg(a)}\paren{\abs{a}(z-\omega)} \\
        \end{align*}
    \end{itemize}
    L'application \(f\) est donc la composée commutative : 
    \begin{itemize}
        \item de l'Homothétie de centre \(\Omega\) et de rapport \(\abs{a}\)
        \item de la rotation de centre \(\Omega\) et d'angle \(\arg(a)\)
    \end{itemize}
\end{defprop}

\begin{defprop}[Applications \(z\mapsto a\conj{z}+b\) avec \((a,b) \in \Cs\times\C\)]
    Soit  \((a,b) \in \Cs\times\C\). \\
    L'application \(g\) de \(\C\) dans \(\C\) définie par 
    \[g(z) = a\conj{z}+b\]
    est dite similitude indirect. Elle peut s'écrire sous la forme de la composée non commutative. 
    \[g = f \circ s\]
    avec :
    \begin{itemize}
        \item \(s:z\mapsto \conj{z}\) qui est la symétrie axiale d'axe de la droite des réels
        \item \(f:z\mapsto az+b\) qui est une similitude directe.
    \end{itemize}
\end{defprop}

\chapter{Calcul de primitives}

\minitoc
\begin{nota}
    \begin{itemize}
        \item \(I\) et \(J\) désignent des intervalles de \(\R\), non vides et non réduits à un point
        \item \(\K\) désigne l'ensemble \(\R\) ou \(\C\)
    \end{itemize}
\end{nota}

\section{Primitives}

\begin{defprop}
    Soit \(f:I\to\K\) une fonction quelconque. \\
    On dit qu'une fonction \(F:I\to\K\) est une primitive de \(f\) sur \(I\) si \(F\) est dérivable sur \(I\) de dérivée \(f\) \\
    Si \(f\) admet une primitive \(F\) sur \(I\) alors l'ensemble des primitives de \(f\) sur \(I\) est \(\accol{x\mapsto F(x)+ \lambda \tq \lambda \in \K}\)
\end{defprop}

\begin{theo}[Théorème fondamental de l'analyse]
    Si \(f\) \textbf{CONTINUE} sur \(I\) alors : 
    \begin{itemize}
        \item pour tout \(x_0\) réel, la fonction \(F:\int_{x_0}^{x} f(t) dt\) est une primitive de \(f\) sur \(I\)
        \item la fonction \(f\) admet des primitives sur \(I\)
    \end{itemize}
\end{theo}

\begin{defprop}[Application au calcul d'intégrales sur un segment]
    Si \(f\) est \textbf{CONTINUE} sur \(I\) et \(F\) une primitive de \(f\) sur \(I\) alors, pour tous réels \(a\) et \(b\) dans \(I\), on a :
    \[\int_{a}^b f(t) dt = F(b)-F(a) \underset{\mathrm{notation}}{=} \croch{F}^a_b \]
\end{defprop}

\section{Primitives usuelles}
\begin{defprop}[Puissances entières ou réelles]
    ~\\
    \renewcommand{\arraystretch}{2.75}
	\begin{tabular}{|l|l|c|}

		\hline
		Si la fonction \(f\) est \(\dots\) & alors une primitive de \(f\) est \(\dots\) & sur tout intervalle \(I\) inclus dans \(\dots\)\\
        \hline
        \(x\mapsto x^n\) avec \(n \in \N\) & \(x\mapsto \frac{1}{n+1} x^{n+1}\) & \(\R\) \\
        \(x\mapsto x^n\) avec \(n \in \Zm \pd\accol{-1}\) & \(x\mapsto \frac{1}{n+1} x^{n+1}\) & \(\Rs\) \\
        \(x\mapsto \frac{1}{x}\) & \(x\mapsto \ln\paren{\abs{x}}\) & \(\Rs\) \\
        \(x\mapsto \frac{1}{2\sqrt{x}}\) & \(x\mapsto \sqrt{x}\) & \(\Rps\) \\
        \(x\mapsto x^{\alpha}\) avec \(\alpha \in \R \pd\Z\)& \(x\mapsto \frac{1}{\alpha + 1}x^{\alpha+1}\) & \(\Rps\) \\

        \hline
	\end{tabular}
\end{defprop}

\begin{defprop}[Exponentielle à valeurs réelles ou complexes et logarithme népérien]
    ~\\
    \renewcommand{\arraystretch}{2.75}
	\begin{tabular}{|l|l|c|}

		\hline
		Si la fonction \(f\) est \(\dots\) & alors une primitive de \(f\) est \(\dots\) & sur tout intervalle \(I\) inclus dans \(\dots\)\\
        \hline
        \(x\mapsto e^{\lambda x}\) avec \(\lambda \in \Ks\) & \(x\mapsto \frac{1}{\lambda} e^{\lambda x}\) & \(\R\) \\
        \(x\mapsto e^x\) & \(x\mapsto e^x\) & \(\R\) \\
        \(x\mapsto \ln(x)\) & \(x\mapsto x\ln(x)-x\) & \(\Rps\) \\
        \hline
	\end{tabular}
\end{defprop}
\vspace{10cm} 
\begin{defprop}[Fonctions hyperboliques]
    ~\\
    \renewcommand{\arraystretch}{2.75}
	\begin{tabular}{|l|l|c|}

		\hline
		Si la fonction \(f\) est \(\dots\) & alors une primitive de \(f\) est \(\dots\) & sur tout intervalle \(I\) inclus dans \(\dots\)\\
        \hline
        \(x\mapsto \ch(x)\) & \(x\mapsto \sh(x) \) & \(\R\) \\
        \(x\mapsto \sh(x)\) & \(x\mapsto \ch(x) \) & \(\R\) \\
        \(x\mapsto 1-\tth^2(x)\) & \(x\mapsto \tth(x) \) & \(\R\) \\
        \(x\mapsto \frac{1}{\ch^2(x)}\) & \(x\mapsto \tth(x) \) & \(\R\) \\        
        \hline
	\end{tabular}
\end{defprop}
\begin{defprop}[Fonctions circulaires et fonctions circulaires réciproques]
    ~\\
    \renewcommand{\arraystretch}{2.75}
	\begin{tabular}{|l|l|c|}

		\hline
		Si la fonction \(f\) est \(\dots\) & alors une primitive de \(f\) est \(\dots\) & sur tout intervalle \(I\) inclus dans \(\dots\)\\
        \hline
        \(x\mapsto \cos(x)\) & \(x\mapsto \sin(x) \) & \(\R\) \\
        \(x\mapsto \sin(x)\) & \(x\mapsto -\cos(x) \) & \(\R\) \\
        \(x\mapsto 1+\tan^2(x)\) & \(x\mapsto \tan(x) \) & \(\R\pd\accol{\frac{\pi}{2}+k\pi\tq k\in\Z}\) \\
        \(x\mapsto \frac{1}{\cos^2(x)}\) & \(x\mapsto \tan(x) \) & \(\R\pd\accol{\frac{\pi}{2}+k\pi\tq k\in\Z}\) \\
        \(x\mapsto \frac{-1}{\sqrt{1-x^2}}\) & \(x\mapsto \Arccos(x) \) & \(\intervee{-1}{1}\) \\
        \(x\mapsto \frac{1}{\sqrt{1-x^2}}\) & \(x\mapsto \Arcsin(x) \) & \(\intervee{-1}{1}\) \\
        \(x\mapsto \frac{1}{1+x^2}\) & \(x\mapsto \Arctan(x) \) & \(\R\) \\

        \hline
	\end{tabular}
\end{defprop}

\section{Calculs de primitives}
\begin{defprop}
    \begin{itemize}
        \item \underline{Primitives d’une combinaison linéaire de fonctions}\\
        Si \(f : I \mapsto \K\) et \(g : I \mapsto \K\) sont des fonctions qui admettent des primitives sur \(I\) notées \(F\) et \(G\) alors, pour tous \(\alpha\) et \(\beta\) dans \(\K\), la fonction\( \alpha f + \beta g : I \mapsto \K\) admet pour primitive sur \(I\) la fonction \( \alpha F + \beta G\)
        \item \underline{Primitives d’une fonction dérivée de fonctions composées} \\
        Si \(u : I \mapsto \R\) est une fonction dérivable sur \(I\) tel que pour tout \(x\) de \(I\), \(u(x)\) appartient à \(J\) et si \(g : J \mapsto \K \) est une fonction dérivable sur \(I\) alors une primitive de la fonction \(f : x  \mapsto u'(x)g'(u(x))\) sur \(I\) est la fonction \(F : x  \mapsto g (u(x))\).\\
        Dans le tableau ci-dessous (à savoir retrouver à partir des primitives usuelles), \(I\) désigne un intervalle sur lequel \(u\) est dérivable et tel que, pour tout \(x\) de \(I\), \(u(x)\) appartient au domaine de dérivabilité de \(F\) .
    \end{itemize}
    \renewcommand{\arraystretch}{2.5}
    \begin{center}
        
	    \begin{tabular}{|l|l|}
	    	\hline
	    	Si la fonction \(f\) est \(\dots\) & alors une primitive de \(f\) est \(\dots\) \\
            \hline
            \(x\mapsto u'(x)\paren{u(x)}^{\alpha}\) avec \(\alpha \in \R\pd\accol{-1}\)  & \(x\mapsto \frac{1}{\alpha+1}\paren{u(x)}^{\alpha+1} \) \\
            \(x\mapsto \frac{u'(x)}{u(x)}\)  & \(x\mapsto \ln(\abs{u(x)}) \) \\
            \hline
            \(x\mapsto u'(x)e^{\lambda u(x)}\) avec \(\lambda \in \Ks\) & \(x\mapsto \frac{1}{\lambda}e^{\lambda u(x)} \) \\
            \(x\mapsto u'(x)\ln\paren{u(x)}\) & \(x\mapsto u(x)\ln(u(x))-u(x) \) \\
            \hline
            \(x\mapsto u'(x)\ch\paren{u(x)}\) & \(x\mapsto \sh(u(x)) \) \\
            \(x\mapsto u'(x)\sh\paren{u(x)}\) & \(x\mapsto \ch(u(x)) \) \\
            \(x\mapsto u'(x)\paren{1+\tth^2\paren{u(x)}}\) & \(x\mapsto \tth(u(x)) \) \\
            \hline

            \(x\mapsto u'(x)\cos\paren{u(x)}\) & \(x\mapsto \sin(u(x)) \) \\
            \(x\mapsto u'(x)\sin\paren{u(x)}\) & \(x\mapsto -\cos(u(x)) \) \\
            \(x\mapsto u'(x)\paren{1+\tan^2\paren{u(x)}}\) & \(x\mapsto \tan(u(x)) \) \\
            \hline
            \(x\mapsto \frac{-u'(x)}{\sqrt{1-u^2(x)}}\) & \(x\mapsto \Arccos(u(x)) \)\\
            \(x\mapsto \frac{u'(x)}{\sqrt{1-u^2(x)}}\) & \(x\mapsto \Arcsin(u(x)) \)  \\
            \(x\mapsto \frac{u'(x)}{1+u^2(x)}\) & \(x\mapsto \Arctan(u(x)) \) \\
            \hline
	    \end{tabular}
    \end{center}
\end{defprop}

\subsection{Deux théorèmes importants}
\begin{defi}[préliminaire]
    Une fonction \(f : I \mapsto \K\) est dite de classe \(\classe{1}\) sur \(I\) si \(f\) est dérivable sur \(I\) et de dérivée continue sur \(I\)
\end{defi}

\begin{theo}[Intégration par parties]
    Si \(u\) et \(v\) sont deux fonctions de classe \(\classe{1}\) sur \(I\) alors, pour tous réels \(a\) et \(b\) dans \(I\), on a :
    \[\int_{a}^{b}u'(t)v(t)dt = \croch{u(t)v(t)}^b_a - \int_{a}^{b}u(t)v'(t)dt\]
\end{theo}

\begin{dem}
    Soit \(u\) et \(v\) deux applications de \(\ensclasse{1}{I}{\R}\) alors \(\forall (a,b) \in I^2\) : 
    \[
    \begin{aligned}
        \int_{a}^{b}(uv)'(t)dt &= \int_{a}^{b}\paren{u'v+uv'}(t)dt \\
         \croch{uv}^b_a &= \int_{a}^{b}(u'v)(t)dt + \int_{a}^{b}(uv')(t)dt \\
          \int_{a}^{b}u'(t)v(t)dt &= \croch{uv}^b_a- \int_{a}^{b}(uv')(t)dt 
    \end{aligned}
    \]
\end{dem}
\begin{theo}[Changement de variable]
    Si \(\phi  :J \mapsto \R\) est fonction de classe \(\classe{1}\) sur \(J\) tel que, pour tout \(t\) de \(J\), \(\phi(t)\) appartient à \(I\) \\
    et\\
    Si \(f  :I\mapsto \K\) est fonction continue sur \(I\) tel que, pour touts \(\alpha\) et \(\beta\) dans \(J\), on a: 
    \[\int_{\alpha}^{\beta} f(\phi(t))\phi'(t) dt = \int_{\phi(\alpha)}^{\phi(\beta)}f(x)dx\]
\end{theo}

\begin{dem}
    Soit \(\phi  :J \mapsto \R\) une fonction de classe \(\classe{1}\) sur \(J\) tel que, pour tout \(t\) de \(J\), \(\phi(t)\) appartient à \(I\) et \(f  :I\mapsto \K\) une fonction continue sur \(I\) tel que, pour tous \(\alpha\) et \(\beta\) dans \(J\), alors : \\

    \(f\) possède une primitive sur \(I\) (car \(f\) est continue \(I\) ) que l'on note \(F\). \\
    On note aussi \(G:t\mapsto F(\phi(t))\) qui est dérivable sur \(J\) par composition ainsi \(G':t\mapsto F'(\phi(t))\times \phi'(t)\), alors :
    \[
    \begin{aligned}
        \int_{\alpha}^{\beta} f(\phi(t))\phi'(t)dt &= \int_{min}^{max} G'(t)dt \\
        &=\croch{G(t)}_{\alpha}^{\beta} \\
        &=F(\phi(\beta)) - F(\phi(\alpha)) \\
        &= \croch{F}_{\phi(\alpha)}^{\phi(\beta)} \\
        &= \int_{\phi(\alpha)}^{\phi(\beta)} f(x)dx
    \end{aligned}
    \]
\end{dem}
\subsection{Primitives de \(x \mapsto e^{ax} \cos(bx)\) ou \(x \mapsto e^{ax} \sin(bx)\)}
\begin{defprop}[]
    \begin{itemize}
        \item \underline{Préliminaire} \\
        Soit \(f\) et \(F\) des fonctions définies sur un intervalle \(I\) à valeurs complexes.
        \begin{enumerate}
            \item \(f\) admet des primitives sur \(I\) si, et seulement si, \(\Reel{f}\) et \(\Ima{f}\) admettent des primitives sur \(I\).
            \item \(F\) est une primitive de \(f\) sur \(I\) si, et seulement si, 
            \(\begin{cases*}
                \Reel{F}\text{ est une primitive de } \Reel{f}\text{ sur  } I \\
                \Ima(F) \text{ est une primitive de } \Ima{f} \text{ sur } I
            \end{cases*}\).
        \end{enumerate}
        \item \underline{Une application usuelle du résultat précédent} \\
        Soit \(a\) et \(b\) des réels tels que \((a,b) \neq (0,0)\). \\
        On note \(\lambda = a+\i b\) et \(f_{\lambda}\) la fonction définie sur \(\R\) par, pour tout \(x\) réel 
            \[f_{\lambda}(x) = e^{ax}\cos(bx) + \i e^{ax}\sin(bx) = e^{ax}e^{bx} \underset{\mathrm{def}}{=} e^{(a+\i b)x} = e^{\lambda x}\]
        La fonction \(F_{\lambda}:x\mapsto \frac{1}{\lambda} e^{\lambda x}\) est une primitive de \(f_{\lambda}\) sur \(\R\) donc :
        \begin{itemize}
            \item la fonction \(\Reel{F_{\lambda}}\) est une primitive de la fonction \(\Reel{F_{\lambda}}:x\mapsto e^{ax}\cos(bx)\) sur \(\R\)
            \item la fonction \(\Ima{F_{\lambda}}\) est une primitive de la fonction \(\Ima{F_{\lambda}}: x\mapsto e^{ax}\sin(bx)\) sur \(\R\)
        \end{itemize}
    \end{itemize}
\end{defprop}

\subsection{Primitives de \(x\mapsto \frac{1}{ax^2+bx+c}\) avec \(a,b\) et \(c\) des réels et \(a\) non nul}
\begin{appl}
    Soit \(a,b\) et \(c\) des réels avec \(a\) non nul et \(g\) la fonction \(g:\R \mapsto \R\) définie par \(g(x) = ax^2+bx+c\) \\
    Trois cas se présentent : 
    \begin{enumerate}
        \item Si \(g\) admet deux racines réelles distinctes \(r_1\) et \(r_2\) alors il existe deux réels \(\alpha_1\) et \(\alpha_2\) tel que :
        \[\forall x \in R\pd\accol{r_1,r_2}, \frac{1}{ax^2+bx+c} = \frac{\alpha_1}{x-r_1}+\frac{\alpha_2}{x-r_2}\]
        Dans ce cas, \\
        une primitive de \(x\mapsto \frac{1}{ax^2+bx+c}\) sur tout intervalle \(I\) inclus dans \(R\pd\accol{r_1,r_2}\) est :
        \[x\mapsto \alpha_1 \ln\abs{x-r_1} + \alpha_2\ln\abs{x-r_2}\]


        \item si \(g\) admet une racine réelle double \(r\) alors il existe un réel \(\alpha\) tel que :
        \[\forall x \in R\pd\accol{r}, \frac{1}{ax^2+bx+c} = \frac{\alpha}{(x-r)^2}\]
        Dans ce cas,\\
        une primitive de \(x\mapsto \frac{1}{ax^2+bx+c}\) sur tout intervalle \(I\) inclus dans \(\R\pd\accol{r}\) est :
        \[x\mapsto \frac{-\alpha}{x-r}\]


        \item Si \(g\) n'admet pas de racines réelles alors, en écrivant \(g\) sous forme canonique, on peut trouver trois réels \(\alpha,\beta \) et \(\gamma\) tel que :
        \[\forall \in \R, \frac{1}{ax^2+bx+c} = \frac{\alpha}{\paren{\frac{x+\beta}{\gamma}}^2+1}\]
        Dans ce cas,\\
        une primitive de \(x\mapsto \frac{1}{ax^2+bx+c}\) sur tout intervalle \(I\) inclus dans \(\R\) est :
        \[x\mapsto \alpha \gamma \arctan\paren{\frac{x+\beta}{\gamma}}\]
    \end{enumerate}
\end{appl}


\chapter{Compléments sur les nombres réels}

\minitoc
\section{Parties denses de \(\R\)}

\begin{defprop}[Généralité]
    Une partie \(X\) de \(R\) est dite dense dans \(\R\) si elle rencontre tout intervalle ouvert non vide de \(\R\). \\
    ~\\
    \underline{En pratique:} \\
    Pour établir qu'une partie \(X\) de \(R\) est dense dans \(R\) à l'aide de cette définition, on montre que tout intervalle du type \(\intervee{a}{b}\) avec \(a\) et \(b\) des réels tel que \(a<b\), contient au moins un élément de \(X\).
\end{defprop}

\begin{ex}
\begin{itemize}
    \item Les ensembles \(\N\) et \(\Z\) sont des parties de \(\R\) qui ne sont pas denses dans \(\R\)
    \item Les ensemble \(\Q\) et \(\R\pd\Q\) sont des parties de \(\R\) qui sont denses dans \(\R\)
\end{itemize}
\end{ex}

\begin{dem}[Preuve de \(Q\) dense dans \(\R\)]
    Soit \(a\) et \(b\) des réels avec \(a<b\).\\
    Montrons que \(\intervee{a}{b}\) contient un élément de \(\Q\), c'est à dire \(\exists (p,q) \in \Z\times\Ns\) tel que \(a<\frac{p}{q}<b\)
    autrement dit \(qa<p<qb\) \\
    Ainsi pour que \(p\) existe il faut que : \\
    \begin{align*}
    &qa - qb > 1       \tag*{car \( p \in \Z \)} \\
    &q(a - b) > 1      \\
    &q > \frac{1}{b-a} \tag*{car \( b > a \)}\\
    \text{Prenons } &q=\floor{\frac{1}{b-a}} +1 \tag*{car \(\frac{1}{b-a}>\floor{\frac{1}{b-a}}+1\)}  
\end{align*}
Prenons \(p=\floor{qa}+1\), donc  \( p-1 \leq qa<p\) \\
or \(p<qb\) car \(q>\frac{1}{b-a} \iff qb-qa>1 \iff qb>qa+1\geq \floor{qa}+1=p\) \\
Ainsi \(qa<p<qb\imp a<\frac{p}{q}<b<b\) avec \(q=\floor{\frac{1}{b-a}} +1\) et \(p=\floor{qa}+1\).
\\\\
\conclusion \\
Tout intervalle réel de type \(\intervee{a}{b}\) avec \(a<b\) contient un rationnel donc par définition, \(\Q\) est dense dans \(\R\).
\end{dem}

\begin{dem}[preuve que \(\R\pd\Q\) est dense dans \(\R\)] ~\\
    \begin{itemize}
        \item \underline{Préliminaire} : Démonstration que \(\sqrt{2}\) est irrationnel\\
        On suppose qu'il existe \((p,q) \in \Z\times\Ns \) avec \(p\) et \(q\) premier entre eux tel que \(\frac{p}{q} = \sqrt{2}\) alors : 
        \begin{align*}
            \frac{p}{q} = \sqrt{2} &\iff \sqrt{2}q = p \\
            &\imp 2q^2 = p^2 \qquad \text{donc } p^2 \text{ est pair ce qui explique } p \text{ pair}\\
            &\imp 2q^2 = (2k)^2 \qquad \text{en posant } p =2k \text{ avec } k \in \Z \\
            &\imp 2q^2 = 4k^2 \\
            &\imp 2k^2 = q^2 \qquad \text{donc } q^2 \text{ est pair et donc } q \text{ aussi} 
        \end{align*}
        Ce qui est absurde car \(p\) et \(q\) sont premier entre eux donc ils ne peuvent pas être tous les deux pair.
        \conclusion \(\sqrt{2}\) est irrationnel.
        \item \underline{Preuve que \(\R\pd\Q\) est dense dans \(\R\)} \\~\\
        Soit \(a\) et \(b\) des réels avec \(a<b\).\\
        Montrons que \(\intervee{a}{b}\) contient un irrationnel :\\
        Par densité de \(\Q\) dans \(\R\), \(\intervee{\frac{a}{\sqrt{2}}}{\frac{b}{\sqrt{2}}}\) contient un rationnel \(r\)\\
        on a donc \(\frac{a}{\sqrt{2}}<r<\frac{b}{\sqrt{2}} \imp a<\sqrt{2}r<b\)
        \begin{itemize}
            \item \underline{Si \(r\neq 0\)}\\
            \(\sqrt{2}r \in \intervee{a}{b}\) et \(\sqrt{2}r\) est irrationnel car sinon \(\sqrt{2}r\) serait rationnel et alors \(\underset{\in \Q}{\sqrt{2}r} \times \underset{\in \Q}{\frac{1}{r}} = \sqrt{2}\) donc \(\sqrt{2} \in \Q\) ce qui est faux.\\
            Donc \(\intervee{a}{b}\) contient un irrationnel.
            \item \underline{Si \(r = 0\)}\\
            On raisonne de même manière mais sur avec un intervalle \(\intervee{0}{b}\) et \(\intervee{0}{\frac{b}{\sqrt{2}}}\)\\
            Ainsi on trouve \(r'\in \intervee{0}{\frac{b}{\sqrt{2}}}\inter \Q\) puis \(r'\sqrt{2} \in \intervee{0}{b}\inter\paren{\R\pd\Q}\)\\
            Donc \(\intervee{a}{b}\) contient un irrationnel.
        \end{itemize}
    \end{itemize}
    \conclusion Tout intervalle réel de type \(\intervee{a}{b}\) avec \(a<b\) contient un irrationnel donc par définition, \(\R\pd\Q\) est dense dans \(\R\).
\end{dem}

\begin{theo}[Caractérisation séquentiel des parties denses dans \(\R\)]
    Une partie \(X\) de \(\R\) est dense dans \(\R\) si, et seulement si, tout réel est limite d'une suite d'élément de \(X\)
\end{theo}
\begin{dem}
    Soit \(X\) une partie de \(\R\)
    On procède par double implication.
    \begin{itemize}
    \item[\impdir]
    On suppose que \(X\) est dense dans \(\R\), soit \(x\) un réel et \(n\in\N\) \\
    alors \(\intervee{x-\frac{1}{n+1}}{x}\) contient un élément de \((u_n)\) de \(X\) par densité de \(X\) dans \(\R\) \\
    Donc \(\forall n \in \N,x-\frac{1}{n+1}<u_n<x\) or \(x-\frac{1}{n+1} \underset{n\to\pinf}{\to}x\) et \(x\underset{n\to\pinf}{\to}x\) donc par théorème d'encadrement \(u_n \underset{n\to\pinf}{\to}x\) \\
    \conclusion \\
    tour réel x est limite d'une suite \((u_n)\) d'élement de \(X\)
    \item[\imprec] On suppose que tout réel est limite d'une suite d'élement de \(X\) \\
    Soit \((a,b)\in\R^2\) avec \(a<b\) et \(l\in \intervee{a}{b}\)\\
    par hypothèse, il existe une suite \((u_n)\) telle que \(\forall n \in \N, u_n \in X\) et \(u_n\underset{n\to\pinf}{\to}l\)\\
    par définition de la limite, \(\intervee{a}{b}\) qui contient \(l\) contient aussi tous les termes de la suite \((u_n)\) à partir d'un certain rang d'où l'existence de 
    \(\begin{cases}
        u_{n_0} &\in X \\
        u_{n_0} &\in \intervee{a}{b}
    \end{cases}\)\\
    \conclusion \\
    \(X\) est dense car pour tout \(\intervee{a}{b}\) avec \(a<b\) il existe un élément (ici \(u_{n_0}\)) de \(X\) dans \(\intervee{a}{b}\)
    \end{itemize} 
    \conclusion \\
    Par double implication le théorème est vérifié
\end{dem}

\section{Approximation décimale d'un réel}
\begin{defprop}[rappel]
    L'ensemble des nombres décimaux est notée \(\D\) et définie par \(\D = \accol{\frac{p}{10^n}\tq (p,n)\in\Z\times\N}\)
\end{defprop}
\begin{prop}[Approximation décimales d'un réel]
    Soit \(x\) un réel et \(n\) un entier naturel. Il existe un unique nombre décimal \(d_n\) tel que :
    \[10^nd_n \in\Z \text{ et } d_n \leq x \leq d_n+10^{-n}\]
    Par ailleurs pour tout réel \(x\) les suites de nombres décimaux \((d_n)\) et \((d_n+10^{-n})\) définie ci-dessus sont convergentes de limite égal à\(x\) donc, par caractérisation séquentielle, l'ensemble \(\D\) est dense dans \(\R\)
\end{prop}

\begin{defprop}[Dévellopement décimal d'un réel]
    Soit \(x\) un réel et \((d_n)\) la suite des valeurs décimales approchées de \(x\) à \(10^{-n}\) près par défaut. \\
    Alors :
    \begin{itemize}
        \item Pour tout \(k\) dans \(Ns\), il existe un unique entier \(a_k\) dans \(\interventierii{0}{9} \) tel que \(d_k-d_{k-1} = \frac{a_k}{10^k}\) \\
        \item Pour tout \(n\) dans \(\N\), \(d_n = \sum_{k=0}^{n} \frac{a_k}{10^k}\) avec \(a_0 = \floor{x}\)
    \end{itemize}
    Puisque la suite \((d_n)\) converge vers \(x\), on peut donc écrire que :   
    \[x = \lim_{n\to\pinf} \paren{\sum_{k=0}^{n} \frac{a_k}{10^k}} \underset{Notation}{=} \sum_{k=0}^{\pinf} \frac{a_k}{10^k} = a_0,a_1a_2\dots\]
    ce qu'on appelle un "dévellopement décimal illimié de \(x\)". \\
    \underline{Par ailleurs} : \\
    L’existence et l’unicité d’un tel \(a_k\) résulte du fait que : \(\forall k \in \Ns, 10^k (d_k - d_{k-1}) \in \interventierii{0}{9}\). L’expression de \(d_n\) sous forme de somme finie s’obtient alors par sommation des égalités \(d_k - d_{k-1} =\frac{a_k}{10^k} \) et télescopage
\end{defprop}

\section{Borne inférieure et supérieure d'une partie de \(\R\)}

\begin{defi}
    Soit \(X\) une partie de \(\R\). S'il existe :
    \begin{itemize}
        \item le plus petit des majorants de \(X\) est appelé borne supérieure de \(X\) et noté \(\sup X\)
        \item le plus grand des minorants de \(X\) est appelé borne inférieure de \(X\) et noté \(\inf X\)
    \end{itemize}
    \underline{Remarques} : \\
    \begin{itemize}
        \item les bornes supérieure ou inférieure de \(X\) ne sont pas nécessairement dans \(X\).
        \item En revanche,
        \begin{itemize}
            \item si \(X\) admet un maximum alors \(X\) admet une borne supérieure, égale au maximum de \(X\) ;
            \item si \(X\) admet un minimum alors \(X\) admet une borne inférieure, égale au minimum de \(X\).
        \end{itemize}

    \end{itemize}
\end{defi}

\begin{prop}[Propriété dite de la borne supérieure/inférieur]
    \begin{itemize}
        \item toute partie non vide et majorée de \(\R\) admet une borne supérieure.
        \item Toute partie non vide et minorée de \(\R\) admet une borne inférieure.
    \end{itemize}
\end{prop}

\begin{defprop}[ Traduction séquentielle de la borne supérieure/inférieure]
Soit \(X\) une partie de \(\R\).
    \begin{itemize}
        \item Si \(X\) est non vide et minorée alors il existe une suite d’éléments de \(X\)  de limite \(\inf X\).
        \item Si \(X\) est non vide et majorée alors il existe une suite d’éléments de \(X\)  de limite \(\sup X\).
        \item Si \(X\) est non vide et non minorée alors il existe une suite d’éléments de \(X\)  de limite \(\minf\) .
        \item Si \(X\) est non vide et non majorée alors il existe une suite d’éléments de \(X\)  de limite \(\pinf\).
    \end{itemize}
\end{defprop}

\begin{defprop}[Droite achevée \(\Rb\)]
    On appelle droite achevée l'ensemble noté \(\Rb\) défini par :
    \[\Rb = \R \union \accol{\minf,\pinf}\]
    On y étend la relation d’ordre \(\leq\), l’addition et la multiplication connues sur \(\R\) avec les conventions :
    \begin{enumerate}
        \item \(\forall x\in \R,\minf < x\pinf\)
        \item \((\minf)+(\minf) = \minf\)
        \item \((\pinf)+(\pinf) = \pinf\)
        \item \(\forall x \in \R,x+(\minf) = (\minf)+x = \minf\)
        \item \(\forall x \in \R,x+(\pinf) = (\pinf)+x = \pinf\)
        \item \(\forall x \in \Rb \pd \accol{0}, x \times (\minf) = (\minf)\times x = \begin{cases}
            \pinf & \text{ si } x<0 \\
            \minf & \text{ si } x>0
        \end{cases}\)
        \item \(\forall x \in \Rb \pd \accol{0}, x \times (\pinf) = (\pinf)\times x = \begin{cases}
            \minf & \text{ si } x<0 \\
            \minf & \text{ si } x>0
        \end{cases}\)
    \end{enumerate}
\end{defprop}

\begin{defprop}[Caractérisation des intervalles de \(\R\)]
    Une partie \(X\) de \(\R\) est un intervalle de \(\R\) si, et seulement si, pour tous réels \(a\) et \(b\) dans \(X\) tels que \(a\leq b\) le segment \(\intervii{a}{b}\) est inclus dans \(X\)
\end{defprop}

\begin{dem}
    On rappelle que \(I\) est un intervalle de \(\R\) si \(I\) est de l'une des formes suivantes : 
    \begin{itemize}
		\item \(I = \emptyset\) \\
		\item \(I = \accol{x \in \R\tq a \leq x \leq b} \underset{\mathrm{notation}}{=} \intervii{a}{b}\) avec \(\paren{a,b} \in \R^2 \) et \(a\leq b \) \\
		\item \(I = \accol{x \in \R\tq a \leq x < b} \underset{\mathrm{notation}}{=} \intervie{a}{b}\) avec \(\paren{a,b} \in \R\times \paren{\R \union \accol{\pinf}} \) et \(a < b\) \\
		\item \(I = \accol{x \in \R\tq a < x \leq b} \underset{\mathrm{notation}}{=} \intervei{a}{b}\) avec \(\paren{a,b} \in \paren{\R \union \accol{\minf}}\times \R \) et \(a < b\) \\
		\item \(I = \accol{x \in \R\tq a < x \leq b} \underset{\mathrm{notation}}{=} \intervee{a}{b}\) avec \(\paren{a,b} \in \paren{\R \union \accol{\minf}}\times  \paren{\R \union \accol{\pinf}} \) et \(a < b\) \\
	\end{itemize}

    Soit \(X\) une partie de \(\R\).
    Dans le cas où \(X\) est l'ensemble vide, l'équivalence attendue est immédiate. On se place donc, dans la suite, dans le cas où \(X\) est une partie non vide de \(\R\) et on raisonne par double implication
    \begin{itemize}
        \item[\impdir] On suppose que \(X\) est un intervalle de \(\R\)\\
        \(X\) est alors d'une des formes \( 2, 3, 4\) ou \(5\) indiquées ci-dessus. Ainsi, pour tous réels \(\alpha\) et \(\beta\) dans \(X\) tels que \(\alpha \leq \beta\), on a bien \(\intervii{\alpha}{\beta} \subset X\)
        \item[\imprec] On suppose que : \(\forall (\alpha,\beta) \in X^2, \alpha \leq \beta \imp \intervii{\alpha}{\beta} \subset X \) \\
        En considérant \(X\) comme partie de la droite achevée \(\Rb\), on peut noter \(m = \inf X\) et \(M = \sup X\) \\
        Montrons que \(\intervee{m}{M} \subset X \subset \intervii{m}{M}\)
        \begin{itemize}
            \item Soit \(t\in \intervee{m}{M}\) \\
            Alors le réel \(t\) n'est pas un majorant de \(X\) (car \(t\) est strictement inférieur à \(M\) qui est le plus petit des majorants de \(X\)) et le réel \(t\) n'est pas un minorant de \(X\)(car \(t\) est strictement supérieur à \(m\) qui est le plus grand es minorants de \(X\)). \\
            ~\\
            Il existe donc\((\alpha, \beta) \in X^2\) tel que \(\alpha <t<\beta\) ce qui prouve que \(t\) appartint à l'intervalle \(\intervee{\alpha}{\beta}\) donc au segment \(\intervii{\alpha}{\beta}\). Comme les réels \(\alpha\) et \(\beta\) appartiennent à \(X\), l'hypothèse faite sur \(x\) donne \(\intervii{\alpha}{\beta} \subset X\) ce qui prouve, en particulier, que \(t\) appartient à \(X\)\\
            \conclusion \(\intervee{m}{M} \subset X\)
            \item Soit \(t\in X\)\\
            Alors, par définition de \(m\) et \(X\), on a : \(m\leq t\leq M\) c'est à dire \(t\in \intervii{m}{M}\) \\
            \conclusion \(X \subset \intervii{m}{M} \)
        \end{itemize} 
        On a donc montré que \(\intervee{m}{M} \subset X\subset \intervii{m}{M}\). Cela implique que \(X\), vue comme partie de \(\Rb\) est égale à l'une des parties suivantes \(\intervee{m}{M},\intervei{m}{M},\intervie{m}{M}\) ou \(\intervii{m}{M}\).\\
        Comme \(X\) est une partie de \(\R\), on en déduit que \(X\) est bien de l'une des formes \( 2, 3, 4\) ou \(5\) indiquées ci-dessus donc que \(X\) est un intervalle de \(\R\)\\
    \end{itemize}
    \conclusion \(X\) est un intervalle de \(\R\) si, et seulement si, \(\forall(\alpha,\beta) \in X^2, \alpha \leq \beta \imp \intervii{\alpha}{\beta} \subset X\)
\end{dem}

\chapter{Ensemble, application et relation}

\minitoc

\section{Ensemble}
\subsection{Généralité}
\begin{defi}
    \begin{itemize}
        \item Un ensemble est une collection d’objets, sans répétition et non ordonnée.
        \item Les objets de l’ensemble sont appelés les éléments de l’ensemble.
        \begin{itemize}
            \item Si \(x\) est un élément de l’ensemble \(E\), on dit que \(x\) appartient à \(E\) et on note \(x \in E\) .
            \item Dans le cas contraire, on dit que \(x\) n’appartient pas à \(E\) et on note \(x \notin E\).
        \end{itemize}
        \item L’ensemble sans élément est appelé l’ensemble vide et noté \(\emptyset\).
        \item Les ensembles avec un seul élément sont appelés des singletons.
        \item Les ensembles avec deux éléments sont appelés des paires.
    \end{itemize}
\end{defi}
\begin{defprop}[Modes de définition d’un ensemble]
    Un ensemble \(E\) peut être défini :\\
    \begin{itemize}
        \item en extension, c’est-à-dire en explicitant tous les éléments de l’ensemble \(E\), dans le cas où
        il compte un nombre fini d’éléments appelé cardinal de l’ensemble. Les éléments de l’ensemble
        sont ainsi tous cités entre accolades.
        Par exemple :
        \begin{itemize}
            \item \(E = \accol{\i}\) singleton contenant le nombre complexe \(\i\) ;
            \item \(E = \accol{\cos, \sin} \)paire contenant les fonctions cosinus et sinus ;
            \item \(E = \accol{2, 3, 5, 7}\) ensemble des nombres premiers inférieurs à \(10\) ;
            \item \(E = \accol{3, 4, . . . , 10}\) ensemble des entiers compris entre \(3\) et \(10\) au sens large (noté aussi \(\interventierii{3}{10}\)).
        \end{itemize}
        \item en compréhension, c’est-à-dire en donnant des propriétés vérifiées par les éléments de
        l’ensemble et eux seuls. Là encore, on utilise des accolades.
        Par exemple :\\
        \begin{itemize}
            \item \(E = \accol{x \in \R \tq x \equiv 0 \croch{2\pi}}\) ensemble des réels congrus à \(0\) modulo \(2\pi\) ;
            \item \(E = \accol{f : \R \to \R \tq \forall x \in \R, f (-x) = f (x)}\) ensemble des fonctions paires de \(\R\) dans \(\R\) ;
            \item \(E = \accol{z \in \C \tq \exists k \in \Z, z = e^{\frac{2\i k \pi}{5}}}\) ensemble des racines \(5\)-ièmes de l’unité.
            \item \(E = \accol{\alpha e \tq \alpha \in \R}\) ensemble des fonctions de la forme \(x \mapsto \alpha e^x\) lorsque \(\alpha\) parcourt \(\R\).
        \end{itemize}
    \end{itemize}
\end{defprop}

\subsection{Inclusion entre ensembles et parties}
\begin{defprop}
    Soit E un ensemble.
    \begin{itemize}
        \item \underline{Inclusion}\\ On dit qu’un ensemble \(F\) est inclus dans \(E\) et on note \(F \subset E\), si tous les éléments de \(F\) appartiennent à \(E\), c’est-à-dire : \(\forall x, \paren{x \in F \imp x \in E}\) .
        \item \underline{Parties}\\ On dit qu’un ensemble \(F\) est une partie ou un sous-ensemble de \(E\) si \(F\) est inclus dans \(E\).
        \item \underline{Ensemble des parties}\\ On note \(\P{E}\) l’ensemble des parties de \(E\), c’est-à-dire \(\P{E} = \accol{A \tq A \subset E}\) .
    \end{itemize}
\end{defprop}
\subsection{Egalité entre ensembles}
\begin{defprop}
    
\begin{itemize}
    \item  \underline{Définition}\\ On dit que deux ensembles \(E\) et \(F\) sont égaux, et on note \(E = F\) , s’ils ont les mêmes éléments, c’est-à-dire : \(\forall x, \paren{x \in E \iff x \in F }\) .
\item  \underline{Caractérisation de l’égalité par double inclusion}\\ Deux ensembles \(E\) et \(F\) sont égaux si, et seulement si, \(E \subset F et F \subset E\).
\end{itemize}
\end{defprop}

\subsection{Opérations sur les parties d’un ensemble}
\begin{defprop}
Soit \(E\) un ensemble et, \(A\) et \(B\) deux parties de \(E\).\\
Soit \(I\) un ensemble et \(\accol{A_i \tq i \in I}\) un ensemble de parties de \(E\).
\begin{itemize}
    \item  \underline{Réunion}\\ 
        On appelle réunion de \(A\) et \(B\), et on note \(A\union B\), la partie de \(E\) définie par \\\(A\union B = \accol{x \in E \tq x \in A \text{ ou } x \in B}\).\\
        Plus généralement, on définit la réunion de parties \(A_i\) de \(E\), avec \(i\) qui varie dans un ensemble \(I\) : \[ \bigunion_{i\in I}A_i = \accol{x \in E \tq \exists i_0 \in I, x \in A_{i_0} }\] .
    \item \underline{ Intersection}\\ 
        On appelle intersection de \(A\) et \(B\), et on note \(A\inter B\), la partie de \(E\) définie par \(A \inter B = \accol{x \in E \tq x \in A \text{ et } x \in B}\).\\
        Plus généralement, on définit l’intersection de parties \(A_i\) de\( E\), avec \(i\) qui varie dans un ensemble \(I\) :\[ \biginter_{i\in I}A_i = \accol{x \in E \tq \forall i \in I, x \in A{i} }\].
    \item \underline{ Différence}\\ On appelle différence de \(B\) dans \(A\), et on note \(A \pd B\), la partie de \(E\) définie par \(A\pd B = \accol{x \in E \tq x \in A \text{ et } X\notin B}\) .
    \item \underline{Complémentaire}\\ 
    On appelle complémentaire de \(A\) dans \(E\) la partie \(E \pd A = \accol{x \in E \tq x \notin A}\) qui est encore notée \(\conj{A}\) ou \(A^c\) (en l’absence d’ambiguité sur l’ensemble dans lequel le complémentaire est considéré).
    \item \underline{ Quelques règles de calcul ou loi de Morgan}\\
    \begin{itemize}
        \item \(\paren{\bigunion_{i\in I}A_i}\inter B = \bigunion_{i \in I} (A_i \inter B)\) et \(  \paren{\biginter_{i\in I} A_i} \union B = \biginter_{i\in I}(A_i \union B)\)
        \item \(\conj{\biginter_{i \in I}A_i} = \bigunion_{i \in I} \conj{A_i}\) et \(\conj{\bigunion_{i \in I}A_i} = \biginter_{i \in I}\conj{A_i}\)
    \end{itemize}
    \item \underline{Recouvrement disjoint et partition d’un ensemble}\\
    L’ensemble \(\accol{A_i \tq i \in I}\) de parties de \(E\) est dit partition de \(E\) si les conditions suivantes sont réunies :\\
    \begin{itemize}
        \item \(E = \bigunion_{i\in I}A_i\)
        \item \(\forall i \in I,A_i \neq \emptyset\)
        \item \(\forall i \in I,\forall j \in I, i\neq j \imp A_i\inter A_j = \emptyset\)
    \end{itemize}
\end{itemize}
\end{defprop}

\begin{dem}[Loi de Morgan]
    Soit \(E\) un ensemble et \(A_j\) des parties de \(E\) où \(i \in I\) et \(B\) une partie de \(E\).
    \begin{itemize}
        \item \underline{Distributivité de l'intersection sur l'union} :\\
        \begin{align*}
            x\in \paren{\bigunion_{i\in I}A_i}\inter B &\iff \paren{x \in \bigunion_{i\in I}A_i} \text{ et } \paren{x \in B} \\
            &\iff \paren{\exists i_0 \in I, x\in A_{i_0}} \text{ et } \paren{x \in B}\\
            &\iff \exists i_0 \in I, x\in A_{i_0}\inter B \\
            &\iff x\in \bigunion_{i\in I}\paren{A_i\inter B}
        \end{align*}
        \item \(\conj{\biginter_{i\in I}A_i} = \bigunion_{i\in I}\conj{A_i}\) :\\
        \begin{align*}
            x \in \conj{\biginter_{i\in I}A_i} &\iff x\notin \biginter_{i \in I} A_i\\
            &\iff \exists A_{i_0}, x\notin A_{i_0}\\
            &\iff x\in \conj{A_{i_0}} \\
            &\iff x \in \bigunion_{i \in I} \conj{A_i}
        \end{align*}
    \end{itemize}
\end{dem}

\subsection{Produit cartésien d’un nombre fini d’ensembles}
\begin{defprop}
    Soit \(E_1, \dots, E_n\) des ensembles.\\
    On appelle produit cartésien de \(E_1, \dots, E_n\) l’ensemble noté \(E_1 \times \dots \times E_n\) défini par :
    \[E_1 \times \dots \times E_n = \accol{\paren{x_1,\dots,x_n} \tq \forall i \interventierii{1}{n}, x_i \in E_i} \]
\end{defprop}


\section{Application}

\subsection{définition de base}
\begin{defprop}
    Une application \(f\) de \(E\) (ensemble de départ) dans \(F\) (ensemble d’arrivée) est un objet mathématique qui, à tout élément \(x\) de \(E\), associe un unique élément de \(F\) noté \(f (x)\) \\
    \underline{Notation fonctionnelle} : \[\fonction{f}{E}{F}{x}{f(x)}\]
\end{defprop}


\begin{defprop}[Image et antécédent]
    Soit \(f : E \mapsto F\) une application.
    \begin{itemize}
        \item Pour tout \(x\) élément de \(E\),\( f (x)\) est un élément de \(F\) appelé l’image de \(x\) par \(f\) .
        \item Soit \(y \in F\) . S’il existe \(x\) dans \(E\) tel que \(y = f (x)\) alors \(x\) est dit un antécédent de \(y\) par \(f\) .
    \end{itemize}
\end{defprop}
\begin{defprop}[Ensemble des applications]
    L’ensemble des applications de \(E\) dans \(F\) est noté \(\ensclasse{\mathcal{F}}{E}{F}\) ou \(F^E\).
\end{defprop}
\begin{defprop}[Egalité entre applications]
    On dit que deux applications \(f\) et \(g\) sont égales, et on note \(f = g\), si les conditions suivantes sont réunies :
    \begin{itemize}
        \item \(f\) et \(g\) ont le même ensemble de départ \(E\) et le même ensemble d’arrivée \(F\) ;
        \item pour tout \(x\) de \(E\),\( f (x) = g(x)\).
    \end{itemize}
\end{defprop}

\begin{defprop}  [Graphe]
Soit \(f : E \mapsto F\) une application. \\
On appelle graphe de \(f\) la partie \(G\) de \(E \times F\) définie par :
\[ G = \accol{\paren{x; f (x)}\tq x \in E} \]
\end{defprop}

\subsection{Fonctions particulières}
\begin{defprop}
    \begin{itemize}
        \item  \underline{Fonction indicatrice d’une partie} \\
        Soit \(A\) une partie de \(E\). L’application \(f\) de \(E\) dans \(\accol{0, 1}\) définie par :
        \[\forall x \in E, f(x) = \begin{cases}
            1 &\text{ si } x\in A \\
            0 &\text{ si } x\notin A
        \end{cases}\]
        est dite fonction indicatrice de \(A\) et notée \(\ind{A}\).
        \item \underline{Restriction} \\
        Soit\( f : E \mapsto F\) une application et \(A\) une partie de \(E\). \\
        L’application \(g : A \mapsto F\) définie par \( \forall x \in A, g(x) = f (x)\) est dite restriction de \(f\) à \(A\) et notée \(\restr{f}{A}\).
        \item \underline{Prolongement} \\
        Soit \(A\) une partie de \(E\) et \(h : A \mapsto F\) une application. \\
        Toute application \(f : E \mapsto F \) telle que \(\restr{f}{A} = h\) est dite prolongement de \(h\) à \(E\).
    \end{itemize}
\end{defprop}

\subsection{Image directe et image réciproque}

\begin{defprop}
    Soit \(f : E \mapsto F\) une application.
    \begin{itemize}
        \item \underline{Image} : \\
        Soit \(A\) une partie de \(E\). On appelle image directe de \(A\) par \(f\) la partie de \(F\) définie par :
        \[f(A) = \accol{y \in F \tq \exists x \in A,y = f(x)} = \accol{f(x)\tq x \in A}\]
        C’est l’ensemble des images par \(f\) des éléments de \(A\).
        \item \underline{Image réciproque} :
        Soit \(B\) une partie de \(F\) . On appelle image réciproque de \(B\) par \(f\) la partie de \(E\) définie par :
        \[f^{-1}(B) = \accol{x\in E\tq f(x) \in B}\]
        C’est l’ensemble des antécédents par \(f\) des éléments de \(B\).
    \end{itemize}
\end{defprop}   

\subsection{Composition d’applications}
\begin{defprop}  
Soit \(f : E \mapsto F\) et \(g : F \mapsto G\) deux applications. L’application \(h : E \mapsto G\) définie par :
\[\forall x \in E, h(x) = g (f (x))\]
est dite composée des applications \(f\) et \(g\) et notée \(h = g \circ f\) .
\end{defprop}

\subsection{Injection, surjection}
\begin{defprop}
    Une application \(f : E \mapsto F\) est dite :
    \begin{itemize}
        \item \underline{Définitions} :\\
        \begin{itemize}
            \item  \underline{injection} si tout élément de \(F\) a au plus un antécédent par \(f\) .
            \item \underline{surjection} si tout élément de \(F\) a au moins un antécédent par \(f\) .
        \end{itemize} 
        \item \underline{Caractérisations pratiques} : \\
        \begin{itemize}
        \item \(f\) est une injection si, et seulement si :\( \forall(x, x') \in E^2, f (x) = f (x') \imp x = x'\).
        \item \(f\) est une surjection si, et seulement si : \(\forall y \in F, \exists x \in E, y = f (x)\).
        \end{itemize}
        \item \underline{Composition} :\\
            La composée de deux injections (resp. surjections) est une injection (resp. surjection).
    \end{itemize}
\end{defprop}

\begin{dem}[Composition]
    \begin{itemize}
        \item \underline{injection} :\\
        Soit \(f : E\mapsto F\) et \(g:F\mapsto G\) deux fonctions injective \\
        \(\forall (x,x')\in E^2 \) tel que \( g(f(x)) = g(f(x'))\) \\
        On a \(f(x) = f(x')\) car \(g\) est une injection \\
        et donc \(x=x'\) car \(f\) est une injection \\
        \conclusion \(\forall (x,x')\in E^2, g(f(x)) = g(f(x')) \imp x=x'\) donc \(g\circ f\) injective\\

        \item\underline{surjection} : \\
        Soit \(f : E\mapsto F\) et \(g:F\mapsto G\) deux fonctions surjectives \\
        Soit\(z \in G\) alors \(\exists y \in F,z=g(y)\) car \(g\) surjective \\
        Soit\(y \in F\) alors \(\exists x \in E,y=f(x)\) car \(f\) surjective \\
        \conclusion \(\forall z \in G,\exists x \in E \) tel que \(z = g(f(x))\) donc \(g\circ f\) surjective
    \end{itemize}
\end{dem}

\subsection{Bijection}
\begin{defprop}
    \begin{itemize}
        \item \underline{Définitions} : \\
        Une application \(f : E \mapsto F\) est dite bijection si tout élément de \(F\) a un unique antécédent par \(f\).\\
        Dans ce cas, l’application \(f ^{-1} : F \mapsto E\) définie par :
        \[\forall y \in F, f^{-1}(y) = x \text{ avec } x \text{ l’unique élément de }E \text{ tel que } y = f (x)\]
        est dite bijection réciproque de \(f\) et vérifie :
        \[f \circ f^{-1} = \id{F} \text{ et } f^{-1} \circ f = \id{F}\]
        \item \underline{ Caractérisation pratique} :\\
        Une application \(f : E \mapsto F\) est une bijection si, et seulement si, \(f\) est une injection et une surjection.
        \item \underline{Composition} :
        \begin{itemize}
            \item La composée de deux bijections est une bijection.
            \item La bijection réciproque de la composée \(g \circ f\) où \(f\) et \(g\) sont des bijections est l’application
            \[(g \circ f )^{-1} = f ^{-1} \circ g^{-1}\]
        \end{itemize}
    \end{itemize}
\end{defprop}

\section{Relation Binaire sur un ensemble}

\subsection{Généralité}
\begin{defprop}
    \begin{itemize}
    \item \underline{Définitions} : \\
    On appelle relation binaire sur un ensemble \(E\) toute partie \(\mathcal{R}\) de \(E \times E\).\\
    Pour tout \((x, y) \in \mathcal{R}\) :
    \begin{itemize}
        \item on dit que \(x\) est en relation avec \(y\) par la relation \(\mathcal{R}\) ;
        \item on note usuellement \(x\mathcal{R}y\)
    \end{itemize}
    \item \underline{Propriétés} : \\
    On dit qu’une relation binaire \(\mathcal{R}\) sur un ensemble \(E\) est :
    \begin{itemize}
        \item réflexive si : \(\forall x \in E, x\mathcal{R}x\) ;
        \item transitive si :\(\forall (x, y, z) \in E^3, (x\mathcal{R}y \text{ et } y\mathcal{R}z)\imp x\mathcal{R}z\) ;
        \item symétrique si : \(\forall (x, y) \in E^2, x\mathcal{R}y \imp y\mathcal{R}x\) ;
        \item antisymétrique si : \(\forall (x, y) \in E^2, (x\mathcal{R}y et y\mathcal{R}x) \imp x = y\).
    \end{itemize}
    \item \underline{ Quelques exemples déjà rencontrés} : 
        \begin{enumerate}
            \item Sur un ensemble \(E\) : la relation d’égalité.
            \item Sur l’ensemble \(\P{E}\) des parties d’un ensemble \(E\) : la relation d’inclusion.
            \item Sur l’ensemble \(\R\) : les relations \(\leq, <\) et la relation de congruence modulo un réel non nul.
            \item Sur l’ensemble \(\ensclasse{\mathcal{F}}{D}{\R} = \R^D\) des applications d’une partie \(D\) de \(\R\) dans \(\R\) : la relation \(\leq\).
            \item Sur l’ensemble Z : les relations de divisibilité \(\divise\) et de congruence modulo un entier non nul.
        \end{enumerate}
    \end{itemize}
\end{defprop}

\subsection{Relations d'équivalence}
\begin{defprop}
    \begin{itemize}
        \item \underline{Définitions} : \\
            Toute relation binaire sur un ensemble \(E\) qui est réflexive, transitive et symétrique est dite relation d’équivalence sur \(E\). Les relations d’équivalence sont souvent notées \(\sim,\simeq \) ou \(equiv\).
        \item \underline{Théorème} : \\
        Soit \(\sim\) une relation d’équivalence sur un ensemble \(E\).\\
        Alors la famille d’ensembles \(\paren{\accol{y \in E \tq x \sim y}}_{x\in E}\) est une partition de \(E\).
        \item \underline{Exemples des relations de congruence}
        \begin{itemize}
            \item La relation de congruence modulo \(2\pi\) est une relation d’équivalence sur \(\R\).\\
                Les classes d’équivalence sont les ensembles \(x + 2\pi\Z = \accol{x + 2n\pi \tq n \in Z}\) avec \(x\) qui décrit \(\intervie{0}{2\pi}\).
            \item La relation de congruence modulo \(n \in \Ns\) est une relation d’équivalence sur \( \Z\).
                Les classes d’équivalence sont les ensembles \(r + n\Z = \accol{r + nq \tq q \in \Z}\) avec \(r\) qui décrit \(\interventierii{0}{n-1}\).
        \end{itemize}
    \end{itemize}
\end{defprop}

\subsection{Relation d'ordre}

\begin{defprop}
    \begin{itemize}
        \item \underline{Définitions} : \\
            Toute relation binaire sur un ensemble \(E\) qui est réflexive, transitive et antisymétrique est dite relation d’ordre sur \(E\). Les relations d’ordre sont souvent notées \(\leq, \precsim , \lesssim \) ou \(\preceq \).
        \item \underline{Ordre partiel et ordre total} : \\ 
            Une relation d’ordre \(\preceq\) sur un ensemble \(E\) est dite totale si :
            \[\forall (x, y) \in E^2, x \preceq y ou y \preceq x\]
            Dans le cas contraire, la relation d’ordre \(\preceq\) est dite partielle.
        \item \underline{ Minorant, majorant, maximum, minimum, etc} : \\
            Les notions de partie minorée, majorée ou bornée ainsi que celles de minorant, majorant, minimum, maximum, borne inférieure ou borne supérieure vues pour les parties de \(\R\) peuvent être étendues aux parties d’un ensemble muni d’une relation d’ordre.\\
            Par exemple, pour \(E\) un ensemble muni d’une relation d’ordre \(\preceq\) et \(A\) une partie de \(E\) :
            \begin{itemize}
                \item \(A\) est dite majorée pour \(\preceq\) s’il existe \(M\) dans \(E\) tel que, pour tout élément \(x\) de \(A\), on a :\( x \preceq M\).\\
                Dans ce cas, on dit que \(M\) est un majorant de \(A\) pour \(\preceq\).
                \item si \(A\) admet un majorant \(M\) pour \(\preceq\) qui appartient à \(A\) alors celui-ci est unique et est appelé le maximum de \(A\) ou le plus grand élément de \(A\) pour \(\preceq\).
            \end{itemize}
    \end{itemize}
\end{defprop}

\chapter{Suites numériques particulières}

\minitoc

\section{Suite arithmétique}
\begin{defi}
    Soit \((u_n)\) une suite réelle (resp. complexe).\\
    La suite \((u_n)\) est dite arithmétique s’il existe un réel (resp. complexe) \(r\) tel que :
    \[\forall n \in \N, u_{n+1} = u_n + r\]
    Le nombre \(r\) est unique et appelé raison de la suite \((u_n)\).
\end{defi}

\begin{defprop}[Expression du terme général]
    Si \((u_n)\) est une suite arithmétique réelle (resp. complexe) de raison \(r\) alors :
    \[\forall p \in \N, \forall n \in \N, n \geq p \imp u_n = u_p + \paren{n - p}r\]
\end{defprop}
\begin{defprop}[Limite]
    Soit \((u_n)\) une suite arithmétique réelle (resp. complexe) de raison \(r\).
    \begin{itemize}
        \item Si \(r\) = 0 alors \((u_n)\) converge vers \(u_0\).
        \item Si \(r \neq 0\) alors \((u_n)\) diverge avec, dans le cas où la suite est réelle, \(u_n \to \begin{cases}
            \pinf & \text{ si } r>0\\
            \minf& \text{ si } r<0
        \end{cases}\)
    \end{itemize}
\end{defprop}

\begin{defprop}[Somme finie de termes consécutifs]
   Si \((u_n)\) est une suite arithmétique réelle (resp. complexe) de raison \(r\) alors
    \[\forall p \in \N, \forall n \in \N, n \geq p \imp \sum_{k=p}^n u_k = \frac{\paren{u_p + u_n}\paren{n-p+1}}{2}\]
\end{defprop}

\section{Suites géométriques}

\begin{defi}
    Soit \((u_n)\) une suite réelle (resp. complexe).\\
    La suite \((u_n)\) est dite géométrique s’il existe un réel (resp. complexe) \(q\) tel que :
    \[\forall n \in \N, u_{n+1} = q\times u_n \]
    Le nombre \(q\) est unique et appelé raison de la suite \((u_n)\).
\end{defi}

\begin{defprop}[Expression du terme général]
    Si \((u_n)\) est une suite géométrique réelle (resp. complexe) de raison \(q\) alors :
    \[\forall p \in \N, \forall n \in \N, n \geq p \imp u_n = q^{n-p}\times u_p \]
\end{defprop}
\begin{defprop}[Limite]
    Soit \((u_n)\) une suite géométrique réelle (resp. complexe) de raison \(q\).
    \begin{itemize}
        \item Si \(\abs{q}<1\) ou \(u_0 =0\) alors \((u_n)\) converge vers \(0\).
        \item Si \(\abs{q} =1\) et \(u_0\neq 0\) alors \((u_n)\) diverge sauf dans le cas particulier \(q=1\) où elle converge vers \(u_0\).
        \item Si \(\abs{q}>1\) et \(u_0\neq 0\) alors \((u_n)\) diverge avec, dans le cas où la suite est réelle et q > 1, \[ u_n \to \begin{cases}
            \pinf & \text{ si } u_0>0\\
            \minf& \text{ si } u_0<0
        \end{cases}\]

    \end{itemize}
\end{defprop}

\begin{defprop}[Somme finie de termes consécutifs]
   Si \((u_n)\) est une suite géométrique réelle (resp. complexe) de raison \(q\) alors
    \[\forall p \in \N, \forall n \in \N, n \geq p \imp \sum_{k=p}^n u_k = \begin{cases}
        u_p \times \frac{1-q^{n-p+1}}{1-q} & \text{ si } q\neq 1\\
        u_p \times (n-p+1) & \text{ si } q=1
    \end{cases}\]
\end{defprop}

\section{Suites arithmético-géométriques}

\begin{defi}
    Soit \((u_n)\) une suite réelle (resp. complexe).\\
    La suite \((u_n)\) est dite arithmético-géométrique s’il existe des réels (resp. complexes) \(a\) et \(b\) tels que
    \[\forall n \in \N, u_{n+1} = a\times u_n + b\]
    \underline{Remarques} :\\
    \begin{itemize}
        \item Si \(a = 1\), on retrouve les suites arithmétiques de raison \(b\).
        \item Si \(b = 0\), on retrouve les suites géométriques de raison \(a\).
    \end{itemize}
\end{defi}

\begin{defprop}[Expression du terme général]
    Soit \((u_n)\) une suite arithmético-géométrique définie par la donnée de \(u_0\) réel (resp. complexe) et par :
    \[\forall n \in \N, u_{n+1} = a\times u_n + b\]
    avec \(a\) et \(b\) des réels (resp. complexes) tel que \(a\neq 1\)\\
    \underline{Méthode d’obtention du terme général }\\
    On montre que : \\
    \begin{itemize}
        \item La seule suite \((v_n)\) constante qui vérifie \(\forall n \in \N, v_{n+1} = a\times v_n + b\) est donnée par : 
        \[\forall n \in \N,v_n = \frac{b}{1-a}\]
        \item La suite \(w_n\) définie par \(\forall n \in \N , w_n = u_n-v_n\) est alors une suite géométrique de raison \(a\) donc : 
        \[\forall n \in \N, w_n = w_0\times a^n\]
        on en déduit que :
        \[\forall n \in \N, u_n = \frac{b}{1-a} + \paren{u_0 + \frac{b}{1-a}}a^n\]
    \end{itemize}
\end{defprop}

\begin{defprop}[Limite]
    Soit \((u_n)\) une suite arithmético-géométrique définie par la donnée de son premier terme \((u_0)\) et par :
    \[\forall n \in \N, u_{n+1} = a\times u_n + b\]
    avec \(a\) et \(b\) des réels (resp. complexes) tels que \(a\neq 1\)
    \begin{itemize}
        \item Si \(\abs{a}<1\) ou \(u_0 = \frac{b}{1-a}\)
        \item Si \(\abs{a} \geq 1\) et \(u_0 \neq \frac{b}{1-a}\) alors \((u_n)\) diverge avec, dans le cas où la suite est réelle et \(a > 1\),
        \[ u_n \to \begin{cases}
            \pinf & \text{ si } u_0>\frac{b}{1-a}\\
            \minf& \text{ si } u_0<\frac{b}{1-a}
        \end{cases}\]
    \end{itemize}
\end{defprop}

\section{Suites récurrentes linéaires d’ordre \(2\) à coefficients constants}

\begin{defi}
    Soit \((u_n)\) une suite réelle (resp. complexe).\\
    La suite \((u_n)\) est dite récurrente linéaire homogène d’ordre \(2\) à coefficients constants s’il existe des réels (resp. complexes) \(a\) et \(b\) tel que
    \[\forall n \in \N,u_{n+2} +a u_{n+1} + b u_n = 0\]
\end{defi}

\begin{defprop}[Equation caractéristique associée]
    Soit \(a\) et \(b\) deux réels (resp. complexes). \\
    La recherche de suites géométriques non nulles de raison \(q\) vérifiant la relation de récurrence
    \[\paren{E} : \forall n \in \N, u_{n+2} + a u_{n+1} + bu_n = 0\]
    conduit à l’équation dite “équation caractéristique” suivante :
   \[\paren{EC} : q^2 + aq + b = 0.\]
\end{defprop}

\begin{defprop}[Expression du terme général]
    \begin{enumerate}
        \item Cas où \((u_n)\) est COMPLEXE et vérifie \(\forall n \in \N, u_{n+2} + a u_{n+1} + b u_n = 0\) avec \(\paren{a,b} \in \C^2\).
        \begin{itemize}
            \item Si \(EC\) a deux racines distinctes \(q_1\) et \(q_2\) alors il existe des complexes \(\lambda_1\) et \(\lambda_2\) tel que 
            \[\forall n \in \N,u_n = \lambda_1 q_1^n + \lambda_2q_2^n\]
            \item Si \(EC\) a une racine double \(q\) alors il existe des complexes \(\lambda_1\) et \(\lambda_2\) tel que 
            \[\forall n \in \N,u_n = \paren{\lambda_1 + \lambda_2 n}q^n\]
        \end{itemize}
        \item Cas où \((u_n)\) est RÉELLE  et vérifie \(\forall n \in \N, u_{n+2} + a u_{n+1} + b u_n = 0\) avec \(\paren{a,b} \in \R^2\).
        \begin{itemize}
            \item Si \(EC\) a deux racines distinctes \(q_1\) et \(q_2\) alors il existe des réels \(\lambda_1\) et \(\lambda_2\) tel que 
            \[\forall n \in \N,u_n = \lambda_1 q_1^n + \lambda_2q_2^n\]
            \item Si \(EC\) a une racine double \(q\) alors il existe des réels \(\lambda_1\) et \(\lambda_2\) tel que 
            \[\forall n \in \N,u_n = \paren{\lambda_1 + \lambda_2 n}q^n\]
            \item Si \(EC\) a deux racines complexes non réelles \(q\) et \(\conj{q}\) alors il existe des réels \(\lambda_1\) et \(\lambda_2\) tel que 
            \[\forall n \in \N,u_n = \paren{\lambda_1\cos\paren{\theta n} + \lambda_2 \sin \paren{n \theta}}r^n\]
            avec \(re^{\i \theta}\) forme trigonométrique de \(q\).
        \end{itemize}
    \end{enumerate}
\end{defprop}

\begin{dem}[Suite complexes récurrentes linéaire d'ordre \(2\) à coefficients constants]
    Soit \(a\) et \(b\) des complexes avec \(b\neq 0\)\\~\\
    On cherche à expliciter l'ensemble \(\mathcal{E}_{a,b}\) des suites \((u_n)_{n\in \N}\) de complexes qui vérifient : 
    \[\forall n \in \N, u_{n+2} + a u_{n+1} b u_n = 0\] 
    \underline{Préliminaire} : 
    \begin{enumerate}
        \item Combinaison linéaire d'éléments de \(\mathcal{E}_{a,b}\) : \\~\\
            Si \((u_n)_{n\in \N}\) et \((v_n)_{n\in \N}\) sont deux suites appartenant à \(\mathcal{E}_{a,b}\) \\~\\
            alors pour tout couple \(\paren{\lambda_1,\lambda_2}\) de complexes la suite \(\paren{\lambda_1 u_n + \lambda_2 v_n}_{n\in \N}\) appartient à \(\mathcal{E}_{a,b}\)\\~\\
            autrement dit, \(\mathcal{E}_{a,b}\) est stable par combinaison linéaire, \\~\\
            \underline{Démonstration :}
            On suppose les hypothèses réunies, en notant \(\paren{w_n}_{n \in \N} = \paren{\lambda_1 u_n + \lambda_2 v_n}_{n \in \N}\) on a :
            \begin{align*}
                \forall n \in \N, w_{n+2} + a w_{n+1} +b w_n &= \paren{\lambda_1 u_{n+2} + \lambda_2 v_{n+2}} + a \paren{\lambda_1 u_{n+1} + \lambda_2 v_{n+1}} + b\paren{\lambda_1 u_n + \lambda_2 v_n}\\
                &=\lambda_1\paren{u_{n+2} + au_{n+1} + bu_{n}} + \lambda_2 \paren{v_{n+2} + a v_{n+1} + bv_{n}} \\
                &= \lambda_1 \paren{ 0 } + \lambda_2 \paren{0} \text{ car } (u_n)_{n\in \N} \text{ et }(v_n)_{n\in \N} \text{ appartiennent à } \mathcal{E}_{a,b} \\
                &= 0
            \end{align*}
            Par conséquent, \((w_n)_{n\in \N}\) appartient à \(\mathcal{E}_{a,b}\).~\\
        \item Recherche de suites géométriques dans \(\mathcal{E}_{a,b}\) : \\~\\
            soit \(q\) un complexe non nul. \\~\\
            La suite \(\paren{q^n}_{n\in \N} \) appartient à \(\mathcal{E}_{a,b}\) si, et seulement si, \(q\) est racine de l'équation suivante.
            \[\paren{EC} : q^2 + aq+b=0\]
            \(\paren{EC}\) est dite équation caractéristique associée à \(\mathcal{E}_{a,b}\)\\~\\
            \underline{Démonstration :}
            \begin{align*}
            \text{La suite }\paren{q^n}_{n \in \N}\text{ appartient à }\mathcal{E}_{a,b} &\text{ si, et seulement si : } \forall n \in \N, q^{n+2} + a q^{n+1} b q^n =0 \\
            &\text{si, et seulement si : } \forall n \in \N,q^n\paren{q^2 + aq +b} =0 \\
            &\text{si, et seulement si : } \forall n \in \N, q^2 +aq+b =0 \text{ car } \forall n \in \N,q^n \neq 0\\
            &\text{si, et seulement si : } q^2 + aq +b =0
            \end{align*}
    \end{enumerate}
    ~\\
    \underline{Détermination des éléments de \(\mathcal{E}_{a,b}\)} : \\
    \begin{itemize}
        \item Cas où l'équation \(\paren{EC}\) a deux racines complexes distinctes \(q_1\) et \(q_2\).\\~\\
        Dans ce cas, \(q_1\) et \(q_2\) sont tous deux non nuls car \(q_1 q_2 = b\) (Formule de Viète) et \(b\neq 1\)\\~\\
        Pour tout complexes \(\lambda_1\) et \(\lambda_2\), la suite \(\paren{\lambda_1 q_1^n + \lambda_2q_2^n}_{n \in \N}\) appartient alors à \(\mathcal{E}_{a,b}\) par combinaison linéaire d'élément de \(\mathcal{E}_{a,b}\) \\~\\
        Montrons qu'il n'y a pas d'autres suites que celles trouvées ci-dessus dans \(\mathcal{E}_{a,b}\) : \\~\\
        Soit \(\paren{u_n}_{n \in \N}\) une suite appartenant à \(\mathcal{E}_{a,b}\)\\
        \analyse 
        on suppose qu'il existe \(\lambda_1\) et \(\lambda_2\) des complexes tel que \(\forall n \in \N, u_n = \lambda_1 q_1^n + \lambda_2 q_2^n\). \\
        on a alors en particulier, \(\begin{cases}
            u_0&= \lambda_1 + \lambda_2 \\
            u_1 &= \lambda_1q_1 + \lambda_2q_2
        \end{cases} \)\\
        Avec les opérations sur les lignes suivantes \(q_1 L_1 - L_2\) et \(q_2L_1 - L_2\), on en déduit que
        \[q_1 u_0-u_1 = \lambda_2(q_1-q_2) \qquad q_2u_0-u_1 = \lambda_1(q_2-q_1) \]
        Comme \(q_1\) et \(q_2\) sont distincts, on obtient finalement : 
        \[\lambda_1 = \frac{u_0q_2-u_1}{q_2 - q_1} \qquad \lambda_2 = \frac{u_1-u_0q_1}{q_2-q_1}\]
        ~\\
        \synthese pour tout \(n \in \N\), on note \(w_n = u_n-\lambda_1q_1^n-\lambda_2q_2^n\)avec les nombres complexes \(\lambda_1\) et \(\lambda_2\) trouvées dans l'analyse \\
        ~\\ Un calcul simple donne alors \[w_0 =w_1 =0 \qquad (1) \]
        Par ailleurs la suite \(\paren{w_n}_{n \in \N}\) appartient à \(\mathcal{E}_{a,b}\) comme combinaison linéaire d'éléments de \(\mathcal{E}_{a,b}\) donc
        \[\forall n \in \N, w_{n+2} + a w_{n+1} +b w_n = 0 \qquad (2)\]
        Par récurrence immédiate en utilisant \((1)\) et \((2)\), on trouve que \(\paren{w_n}_{n \in \N}\) est la suite nulle ce qui prouve que 
        \[\forall n \in \N, u_n = \lambda_1q_1^n + \lambda_2q_2^n\]
        Ainsi si \(\paren{u_n}_{n \in \N}\) est une suite de \(\mathcal{E}_{a,b}\) alors il existe des complexes \(\lambda_1\) et \(\lambda_2\) tel que \(\paren{u_n}_{n \in \N} = \paren{\lambda_1q_1^n + \lambda_2q_2^n}_{n \in \N}\)\\ ~\\
        \conclusion si l'équation caractéristique \(\paren{EC}\) a deux racine complexes distinctes \(q_1\) \(q_2\) alors 
        \[\mathcal{E}_{a,b} = \accol{\paren{\lambda_1q_1^n + \lambda_2q_2^n}_{n \in \N} \tq \paren{\lambda_1,\lambda_2} \in \C^2}\]
        ~\\
        \item cas où l'équation caractéristique \(\paren{EC}\) a une racine complexe double \(q\) \\~\\
        Le discriminant de \(\paren{EC}\) est alors nul (donc \(a^2 = 4b\)) et \(q = -\frac{1}{2}a\) ce qui implique que \(q\) est non nul sinon on aurait \(a =b = 0\) ce qui est exclu par hypothèse sur b. \\~\\
        Pour tout complexes, \(\lambda_1\) et \(\lambda_2\), la suite \(\paren{\lambda_1 q^n + \lambda_2 n q^n}_{n \in \N}\) appartient alors à \(\mathcal{E}_{a,b}\) par combinaison linéaire d'élément de \(\mathcal{E}_{a,b}\)\\~\\
        En, effet \(\paren{q^n}_{n\in \N}\) appartient à \(\mathcal{E}_{a,b}\) (d'après le Préliminaire \(2\)) et \(\paren{nq^n}_{n \in \N}\) appartient à \(\mathcal{E}_{a,b}\) car 
        \begin{align*}
            \forall n \in \N, \paren{n+2}q^{n+2}+a\paren{n+1}q^{n+1} + bnq^n &= nq^n\paren{q^2+aq+b} + q^n\paren{2q^2+aq} \\
            &= nq^n(0)+q^n(0) \\
            &=0
        \end{align*}
        Montrons qu'il n'y a pas d'autres suites que celles trouvées ci-dessus dans \(\mathcal{E}_{a,b}\) \\~\\
        Soit \(\paren{u_n}_{n \in \N}\) une suite appartenant à \(\mathcal{E}_{a,b}\)\\~\\
        \analyse on suppose qu'il existe \(\lambda_1\) et \(\lambda_2\)des complexes tel que \(\forall n \in \N, u_n = \lambda_1q^n + \lambda_2nq^n\)\\~\\
        On a alors, en particulier, \(\begin{cases}
            u_0 &= \lambda_1 \\
            u_1 &= \lambda_1q+\lambda_2q
        \end{cases}\)\\~\\
        Comme \(q\) est non nul, on trouve :
        \[\lambda_1 = u_0 \qquad \lambda_2 = \frac{u_1-u_0q}{q}\]
        \synthese pour tout \(n \in \N\), on note \(w_n = u_n-\lambda_1 q^n - \lambda_2 n q^n\) avec les nombres complexes \(\lambda_1\) et \(\lambda_2\) trouvées dans l'analyse. \\~\\
        Un calcul simple donne alors :\[w_0 = w_1 =0 \qquad (1)\]
        Par ailleurs, la suite \(\paren{w_n}_{n\in \N}\) appartient à \(\mathcal{E}_{a,b}\) comme combinaison linéaire d'élément de \(\mathcal{E}_{a,b}\) donc 
        \[\forall n \in \N,w_{n+2}+a w_{n+1}+bw_n = 0 \qquad (2)\]
        Par récurrence immédiate en utilisant \((1)\) et \((2)\), on trouve que \(\paren{w_n}_{n\in \N}\) est la suite nulle ce qui provoque que 
        \[\forall n \in \N, u_n = \lambda_1q^n + \lambda_2 n q^n\]
        Ainsi si \(\paren{v_n}_{n\in \N}\) est une suite de \(\mathcal{E}_{a,b}\) alors il existe des complexes \(\lambda_1\) et \(\lambda_2\) tel que \(\paren{u_n}_{n \in \N} = \paren{ \lambda_1 q^n + \lambda_2 n q^n}_{n \in \N}\)\\~\\
        \conclusion si l'équation caractéristique \(\paren{EC}\) a une racine complexe double \(q\) alors 
        \[\mathcal{E}_{a,b} = \accol{\paren{\lambda_1q^n + \lambda_2 n q^n}_{n \in \N}\tq \paren{\lambda_1,\lambda_2}\in \C^2}\]
    \end{itemize}
\end{dem}

\begin{dem}[Suite réelles récurrentes linéaire d'ordre \(2\) à coefficients constants]
     Soit \(a\) et \(b\) des réels avec \(b\neq 0\)\\~\\
    On cherche à expliciter l'ensemble \(\mathcal{E}_{a,b}\) des suites \((u_n)_{n\in \N}\) de réels qui vérifient : 
    \[\forall n \in \N, u_{n+2} + a u_{n+1} b u_n = 0\]
    on appelle toujours \underline{équation caractéristique} associée à \(\mathcal{E}_{a,b}\) l'équation \(\paren{EC} : q^2 +aq+b=0\)\\~\\
    Les deux cas suivants se traitent de la même manière que pour les suites compelxes 
    \begin{itemize}
        \item Cas où l'équation \(\paren{EC}\) a deux racines réelles distinctes \(q_1\) et \(q_2\).\\~\\
        \conclusion si l'équation caractéristique \(\paren{EC}\) a deux racine réelles distinctes \(q_1\) \(q_2\) alors 
        \[\mathcal{E}_{a,b} = \accol{\paren{\lambda_1q_1^n + \lambda_2q_2^n}_{n \in \N} \tq \paren{\lambda_1,\lambda_2} \in \R^2}\]
        \item cas où l'équation caractéristique \(\paren{EC}\) a une racine réelle double \(q\) \\~\\
        \conclusion si l'équation caractéristique \(\paren{EC}\) a une racine réelle double \(q\) alors 
        \[\mathcal{E}_{a,b} = \accol{\paren{\lambda_1q^n + \lambda_2 n q^n}_{n \in \N}\tq \paren{\lambda_1,\lambda_2}\in \R^2}\]
        \item Cas où l'équation \(\paren{EC}\) a deux racines complexes conjuguées non réelles \(q\) et \(\conj{q}\).\\~\\
        Comme \(q\) et \(\conj{q}\) sont distincts (car q n'est pas réel), on sait que les suites complexes vérifiant 
        \[\forall n \in \N,u_{n+2} + au_{n+1}+bu_n = 0\]
        sont les suites \(\paren{\lambda_1 q^n + \lambda_2\conj{q}^n}_{n \in \N}\) avec \(\paren{\lambda_1,\lambda_2}\in C^2\) \\~\\
        Determinons parmi ces suites \underline{celles qui sont à valeurs réelles} en utilisant les propriété de la conjugaison. \\~\\
        \begin{align*}
        \paren{\lambda_1q^n + \lambda_2 \conj{q}^n}_{n \in \N} \text{ est à valeurs réelles } &\text{si, et seulement si, } \forall n \in \N,\lambda_1 q^n + \lambda_2 \conj{q}^n =\conj{\lambda_1 q^n + \lambda_2 \conj{q}^n} \\
        &\text{si, et seulement si, } \forall n \in \N,\lambda_1 q^n + \lambda_2 \conj{q}^n =\conj{\conj{\lambda_1} \conj{q}^n + \conj{\lambda_2} q^n}\\
        &\text{si, et seulement si, } \forall n \in \N,\paren{\lambda_1 -\conj{\lambda_2}}q^ - \paren{\conj{\lambda_1}-\lambda_2}\conj{q}^n = 0 \\
        &\text{si, et seulement si, } \forall n \in \N,\paren{\lambda_1 -\conj{\lambda_2}}q^ - \conj{\paren{\lambda_1-\conj{\lambda_2}}q^n} = 0 \\
        &\text{si, et seulement si, } \forall n \in \N,2 \Ima{\paren{\lambda_1-\conj{\lambda_2}}q^n} = 0 
        \end{align*}
        \begin{itemize}
            \item Si \(\paren{\lambda_1 q^n + \lambda_2\conj{q}^n}_{n \in \N}\)  est à valeurs réelles, on a donc \(\Ima{\paren{\lambda_1-\conj{\lambda_2}}q^0} = 0 \) et \(\Ima{\paren{\lambda_1-\conj{\lambda_2}}q} = 0 \)\\
            La première égalité donne \(\lambda_1 - \conj{\lambda_2} \in \R\). La seconde égalité implique alors que \(\paren{\lambda_1-\conj{\lambda_2}}\Ima{q} =0\) puis que \(\paren{\lambda_1-\conj{\lambda_2}} = 0\) (car \(q\) n'est pas réel donc sa partie imaginaire est non nulle). Ainsi \(\lambda_1 = \lambda_2\).
            \item Réciproquement, si \(\lambda_1 = \conj{\lambda_2}\) alors, pour tout \(n\) entier naturel, on a \(2i \Ima{\paren{\lambda_1-\conj{\lambda_2}q^n}} = 0\) donc, avec les équivalences précédentes \(\paren{\lambda_1q^n + \lambda_2 \conj{q}^n}_{n \in \N}\) est à valeurs réelles. 
        \end{itemize}
        En résumé : les suites de \(\mathcal{E}_{a,b}\) sont donc les suites \(\paren{\lambda_1 q^n + \conj{\lambda_1}\conj{q}^n}_{n \in \N} \) avec \(\lambda_1\) complexe quelconque. \\~\\
        Pour faire apparaître une forme de terme général plus explicite (sans nombres complexes), on écrit \(q\) sous forme trigonométrique \(q = re^{\i \theta}\)(\(r>0\) et \(\theta\) réel) et \(\lambda_1\) sous forme algébrique \(\lambda_1 = \alpha_1 + \i \beta_1\)(\(\alpha_1\) et \(\beta_1\) réels)\\~\\
        On a alors : \[\lambda_1 q^n + \conj{\lambda_1}\conj{q}^n = 2\Reel{\lambda_1q^n} = 2\Reel{r^n\paren{\alpha_1+\i \beta_1}e^{\i n \theta}} = 2r^n\paren{\alpha_1\cos\paren{n\theta}-\beta \sin\paren{n\theta}}\]
        ce qui peut encore s'écrire sous la forme 
        \[u_n = r^n\paren{\mu_1\cos\paren{n\theta} + \mu_2\sin\paren{n\theta}}\]
        avec \(\paren{\mu_1,\mu_2} \in \R^2\)
        \conclusion : Si l'équation \(\paren{EC}\) a deux racines complexes conjuguées non réelles \(q\) et \(\conj{q}\) alors 
        \[\mathcal{E}_{a,b} = \accol{r^n\paren{\mu_1\cos(n \theta)+\mu_2 \sin(n \theta)}\tq \paren{\mu_1,\mu_2}\in \R^2}\]
        où \(r= \abs{q}\) et \(\theta\) est un argmument de \(q\).
    \end{itemize}
\end{dem}

\section{Cas simples de suites récurrentes du type \(u_{n+1} = f (u_n)\)}

\begin{defi}
    On s’intéresse à la suite réelle \((u_n)\) définie par récurrence par la donnée de :\\
    \[u_0 \in I \text{ et } \forall n \in \N, u_{n+1} = f (u_n)\]
    avec \(I\) un intervalle de \(\R\), non vide et non réduit à un point et \(f : I \to I \) une fonction.
\end{defi}
\begin{defprop}[Limite éventuelle]
    
    Si \((u_n)\) converge vers un réel \(l \in I\) en lequel \(f\) est continue alors \(f (l) = l\).\\
    \underline{Attention} : \\
    \begin{itemize}
        \item La réciproque de la propriété précédente est FAUSSE.
        \item La recherche des réels \(l \in I\) tel que \(f (l) = l\) fournit uniquement les limites éventuelles de \((u_n)\).
        \item Une étude complémentaire permet de conclure si \((u_n)\) converge vers une des valeurs trouvées.
    \end{itemize}
    Dans certains cas, l’étude de la fonction \(g : x \mapsto f (x) - x\) peut être utile pour montrer l’existence de racines pour \(g\) qui sont les limites éventuelles de \((u_n)\).
\end{defprop}

\begin{defprop}[Monotonie éventuelle]
    Pour montrer une monotonie éventuelle de \((u_n)\), on regarde si le signe de
    \[u_{n+1} - un = \begin{cases}
        f(u_n)-u_n &(1)\\
        f(u_n)-f(u_{n_1}) &(2)
    \end{cases}\]
    est fixe lorsque \(n\) varie dans \(\Ns\) ou à partir d’un certain rang.
    \begin{itemize}
        \item Dans certains cas, l’étude de la fonction \(g : x \mapsto f (x) - x\) peut aider à déterminer le signe de \((1)\).
        \item Dans le cas où \(f\) est CROISSANTE sur \(I\),
        \begin{itemize}
            \item une récurrence simple avec \((2)\) montre que, pour tout \(n \in \N, u_{n+1} - u_n\) est du signe de\( u_1 - u_0\) :
            \[\begin{cases} 
                \text{Si }u_0 < u_1 \text{ alors } (u_n) \text{est croissante}\\
                \text{Si }u_0 > u_1 \text{ alors } (u_n) \text{est décroissante}
            \end{cases}\]
            \item l’étude de la fonction \(g : x \mapsto f (x)-x\) peut être utile pour déterminer le signe \(u_1 -u_0 = f (u_0)-u_0\).
        \end{itemize} 
    \end{itemize}
\end{defprop}

\chapter{Suites numériques}

\minitoc

\section{Généralité sur les suites réelles}
\subsection{Définition}
\begin{defprop}
    Toute fonction \(u\) définie sur \(\N\) et à valeurs dans \(\R\) est dite suite réelle.\\
    \underline{Notations usuelles}\\
    \begin{itemize}
        \item Pour tout \(n \in \N\) est noté \(u_n\) (terme général de la suite)
        \item La fonction \(u\) est notée \(\paren{u_n}_{n \in \N}\) ou \(\paren{u_n}_{n \geq 0}\) ou encore \((u_n)\)
    \end{itemize}
    \underline{Remarque}\\
    Plus généralement, on appelle suite réelle et on note \(\paren{u_n}_{n\geq p}\) toutes fonctions \(u\) définie sur 
    \[
    \interventierie{p}{\pinf} = \accol{n \in \N \tq n \geq p}
    \]
    et à valeurs dans \(\R\) avec \(p\) un entier fixé.
\end{defprop}

\begin{defprop}[Modes de définition d’une suite]
    Une suite réelle \((u_n)\) peut être définie : 
    \begin{enumerate}
        \item explicitement par la donnée, pour tout entier naturel \(n\), de l'expression de \(u_n\) en fonctions de \(n\)\\
        \item implicitement par la donnée d'une propriété vérifiée par les termes de la suite \\
        \item par récurrence
    \end{enumerate}
\end{defprop}

\subsection{Suites majorées, minorées, bornées}
\begin{defprop}
    Soit \((u_n)\) une suite réelle et \(A = \accol{u_n \tq n \in \N}\) la partie de \(\R\) contenant tous les termes de la suite. 
    \begin{itemize}
        \item La suite \((u_n)\) est dite \underline{majorée} si \(A\) est majorée\\
        \cad s'il existe un réel \(M\) tel que, pour tout entier naturel \(n\), on a \(u_n \leq M\)
        \item La suite \((u_n)\) est dite \underline{minorée} si \(A\) est minorée\\
        \cad s'il existe un réel \(m\) tel que, pour tout entier naturel \(n\), on a \(m \leq u_n\)
        \item La suite \((u_n)\) est dite \underline{bornée} si \(A\) est bornée\\
        \cad s'il existe des réels \(M\) et \(m\) tel que, pour tout entier naturel \(n\), on a \(m \leq u_n \leq M\)
    \end{itemize}
\end{defprop}
\begin{defprop}[Caractérisation du caractère borné]
    Une suite réelle \((u_n)\) est bornée si, et seulement si, la suite \((\abs{u_n})\) est majorée par un réel strictement positif.
\end{defprop}

\subsection{Suites stationnaires, monotones, strictement monotones}
\begin{defprop}
    Une suite réelle \((u_n)\) est dite :
    \begin{itemize}
        \item \underline{stationnaire} s'il existe un entier naturel \(p\) tel que, pour tout entier \(n\) supérieur à \(p\), on a \(u_n = u_p\)\\~\\
        \item \underline{croissante} si, pour tout entier naturel \(n\), on a \(u_n \leq u_{n+1}\)
        \item \underline{décroissante} si, pour tout entier naturel \(n\), on a \(u_{n+1}\leq u_n\)\\~\\
        \item \underline{strictement croissante} si, pour tout entier naturel \(n\), on a \(u_n < u_{n+1}\)
        \item \underline{strictement décroissante} si, pour tout entier naturel \(n\), on a \(u_{n+1}< u_n\)\\~\\
        \item \underline{monotone} si elle est croissante ou décroissante.
        \item \underline{strictement décroissante} si elle est strictement croissante ou strictement décroissante
    \end{itemize} 
\end{defprop}

\section{ Limite d'une suite réelle}
\subsection{Généralités sur les limites}
\begin{defprop}[Définition d'une limite finie]
    Soit \((u_n)\) une suite réelle et \(l\) un réel. \\
    On dit que la suite \((u_n)\) a pour limite \(l\) si tout segment centrée en \(l\) contient tous les termes de la suite \((u_n)\) à partir d'un certain rang, ce qui se traduit par
    \[\quantifs{\forall x \in \Rs;\exists n_0 \in \N; \forall n \in \N;n\geq n_0} \imp \abs{u_n - l}\leq \epsilon\]
\end{defprop}
\begin{defprop}[Définition d'une limite infinie]
    Soit \((u_n)\) une suite réelle. \\
    \begin{itemize}
        \item On dit que la suite \((u_n)\) a pour limite \(\pinf\) si tout intervalle du type \(\interventierie{A}{\pinf}\) contient tous les termes de la suite \((u_n)\) à partir d'un certain rang, ce qui se traduit par :
        \[\quantifs{\forall A \in \Rs;\exists n_0 \in \N; \forall n \in \N; n \geq n_0} \imp u_n \geq A\]
        \item On dit que la suite \((u_n)\) a pour limite \(\minf\) si tout intervalle du type \(\interventierei{\minf}{A}\) contient tous les termes de la suite \((u_n)\) à partir d'un certain rang, ce qui se traduit par :
        \[\quantifs{\forall A \in \Rs;\exists n_0 \in \N; \forall n \in \N; n \geq n_0} \imp u_n \leq A\]
    \end{itemize}
\end{defprop}

\begin{prop}[Unicité de la limte d'une suite]
    Si \((u_n)\) est une suite réelle de limite \(l\) alors \(l\) est unique et notée \(l = \lim u_n \) ou \(u_n \to l\)
\end{prop}

\subsection{Cas particulier des limites finies : retour en \(0\)}
\begin{defprop}
    Soit \((u_n)\) une suite réelle et \(l\) un réel. \\~\\
    Pour tout \( n \in \N, \abs{u_n -l} = \abs{\paren{u_n - l}-0} = \abs{\abs{u_n - l}-0}\) donc : \\
    \begin{itemize}
        \item la suite \((u_n)\) a pour limite \((l)\) si, et seulement si, la suite \((u_n - l)\) converge vers \(0\) \\
        \item la suite \((u_n)\) a pour limite \((l)\) si, et seulement si, la suite \(\abs{u_n - l}\) converge vers \(0\)
    \end{itemize}
\end{defprop}

\subsection{Suites convergentes et divergentes}

\begin{defi}
    Une suite réelle \((u_n)\) est dite : 
    \begin{itemize}
        \item \underline{convergente} si elle admet une limite réelle \(l\) et, dans ce cas, on dit que \((u_n)\) converge vers \(l\)
        \item \underline{divergente} sinon.
    \end{itemize}
\end{defi}

\begin{prop}
    \begin{enumerate}
        \item Toute suite réelle convergente est bornée. \\
        \item Toute suite réelle non bornée est divergente.
    \end{enumerate}
\end{prop}

\subsection{Opérations sur les limites}
    Soit \((u_n)\) et \(u'_n\) deux suites réelles et \(\alpha\) un réel.
\begin{defprop}
    \begin{enumerate}
        \item \underline{Addition} \\~\\
        \begin{enumerate}
            \item Si \(u_n \to l\) avec \(l \in \R\) et \(u'_n \to l'\) avec \(l' \in \R\) alors \(u_n + u'_n \to l + l'\)
            \item Si \(u_n \to \pinf\)  et \(u'_n\to l'\) avec \(l' \in \R \union \accol{\pinf}\) alors \(u_n + u'_n \to \pinf\)
            \item     Si \(u_n \to \minf\)  et \(u'_n\to l'\) avec \(l' \in \R \union \accol{\minf}\) alors \(u_n + u'_n \to \minf\)
        \end{enumerate}
        \item \underline{Multiplication par un réel}.\\~\\
            \begin{enumerate}
                \item Si \(u_n \to l \) avec \(l \in \R\) alors \(\alpha u_n \to \alpha l\)
                \item Si \(u_n \to \pinf\) alors \(\alpha u_n \to \begin{cases}
                    \pinf &\text{ si } \alpha>0 \\
                    0 &\text{ si } \alpha=0 \\
                    \minf &\text{ si } \alpha<0 \\
                \end{cases}\)
                \item Si \(u_n \to \minf\) alors \(\alpha u_n \to \begin{cases}
                    \pinf &\text{ si } \alpha<0 \\
                    0 &\text{ si } \alpha=0 \\
                    \minf &\text{ si } \alpha>0 \\
                \end{cases}\)
            \end{enumerate}
        \item \underline{Produit} 
            \begin{enumerate}
                \item Si \(u_n \to l\) avec \(l \in \R\) et \(u'_n \to l'\) avec \(l' \in \R\) alors \(u_n u'_n \to ll'\)
                \item Si \(u_n \to \pinf\) et \(u'_n \to l'\) avec \(l' \in \Rb \pd \accol{ 0}\) alors \(u_n u'_n \to \begin{cases}
                    \pinf &\text{ si } l' > 0 \\
                    \minf &\text{ si } l'<0
                \end{cases}\)
                \item Si \(u_n \to \minf\) et \(u'_n \to l'\) avec \(l' \in \Rb \pd \accol{0}\) alors \(u_n u'_n \to \begin{cases}
                    \minf &\text{ si } l'> 0 \\
                    \pinf &\text{ si } l' < 0 
                \end{cases}\)
            \end{enumerate}
        \item \underline{Inverse}
            \begin{enumerate}
                \item Si \(u_n \to l\) avec \( l \in Rs\) alors \(\frac{1}{u_n} \to \frac{1}{l}\)
                \item Si \(u_n \to l\) avec \( l\in \accol{\pinf, \minf}\) alors \(\frac{1}{u_n} \to 0\)
                \item Si \(u_n \to 0\) avec les termes \(u_n\) strictement positifs à partir d'un certain rang alors \(\frac{1}{u_n}\to \pinf\)
                \item Si \(u_n \to 0\) avec les termes \(u_n\) strictement négatifs à partir d'un certain rang alors \(\frac{1}{u_n} \to \minf\)
            \end{enumerate}
    \end{enumerate}
\end{defprop}


\subsection{Limite et relation d'ordre}
\begin{defprop}[Passage à la limite d'une inégalité large]
    Soit \((u_n)\) et \((u'_n)\) deux suites réelles convergentes respectivement vers des réels \(l\) et \(l'\)\\~\\
    S'il existe un entier \(n_0\) tel que \(\quantifs{\forall n \in N} n \leq n_0 \imp u_n \leq u'_n\) alors \(l \leq l'\)
\end{defprop}
\begin{defprop}[Signes des termes d'une suite et signe de la limite]
    Soit \((u_n)\) une suite réelle de limite \(l\) appartenant \(\Rb\).
    \begin{itemize}
        \item Si \(l>0\) alors il existe un rang à partir duquel tous les termes \(u_n\) sont strictement positif
        \item  Si \(l<0\) alors il existe un rang à partir duquel tous les termes \(u_n\) sont strictement négatif
    \end{itemize}
\end{defprop}

\subsection{Existence d'une limite finie}

\begin{theo}[Théorème d'encadrement]
    Soit \((u_n),(v_n)\) et \((w_n)\) trois suites réelles et \(l\) un réel. \\~\\
    S'il existe un entier \(n_0\) tel que  \(\quantifs{\forall n \in \N}, n\geq n_0 \imp v_n \leq u_n \leq w_n\) et si \((v_n)\) et \((w_n)\) convergent vers \(l\) alors \((u_n)\) converge vers \(l\).
\end{theo}

\begin{prop}[pratique]
    Soit \((u_n)\) et \((v_n)\) deux suites réelles et \(l\) un réel. \\~\\
    S'il existe un rang à partir duquel on a 
    \[\abs{u_n -l} \leq v_n \text{ avec } (v_n) \text{ convergente vers } 0\]
    alors \((u_n)\) converge vers \(l\).
\end{prop}

\begin{defprop}[Conséquence]
    Soit \((u_n)\) et \((v_n)\) deux suites réelles. \\~\\
    \begin{enumerate}
        \item Si \((u_n)\) converge vers un réel \(l\) alors \((\abs{u_n})\) converge vers \(\abs{l}\).
        \item Si \((u_n)\) converge vers un réel \(0\) et \(v_n\) est bornée alors \((u_n v_n)\) converge vers \(0\)
    \end{enumerate}
\end{defprop}

\subsection{Existence d'une limite infinie}
\begin{theo}[Théorème de minoration]
    Soit \((u_n)\) et \((v_n)\) deux suites réelles.\\~\\
    S'il existe un entier \(n_0\) tel que \(\forall n \in \N, n \leq n_0 \imp v_n\leq u_n\) et si \((v_n)\) a pour limite \(\pinf\) alors \((u_n)\) a pour limite \(\pinf\)
\end{theo}

\begin{theo}[Théorème de majoration]
    Soit \((u_n)\) et \((v_n)\) deux suites réelles.\\~\\
    S'il existe un entier \(n_0\) tel que \(\forall n \in \N, n \leq n_0 \imp v_n\geq u_n\) et si \((v_n)\) a pour limite \(\minf\) alors \((u_n)\) a pour limite \(\minf\)
\end{theo}

\subsection{Cas des suites monotones}
\begin{theo}[Théorèmes de la limite monotone]
    \begin{itemize}
        \item Si \((u_n)\) est une suite réelle croissante et majorée alors \((u_n)\) converge vers \(l = \sup \accol{u_n \tq n \in \N}\)
        \item Si \((u_n)\) est une suite réelle croissante et non majorée alors \((u_n)\) a pour limite \(\pinf\)\\~\\
        
        \item Si \((u_n)\) est une suite réelle décroissante et minorée alors \((u_n)\) converge vers \(l = \inf \accol{u_n \tq n \in \N}\)
        \item Si \((u_n)\) est une suite réelle décroissante et non minorée alors \((u_n)\) a pour limite \(\minf\)
    \end{itemize}
\end{theo}

\begin{theo}[Théorème des suites adjacentes]
    Soit \((u_n)\) et \((v_n)\) deux suites réelles.\\~\\
    Si \((u_n)\) est croissante, \((v_n)\) est décroissante et \((v_n -u_n)\) converge vers \(0\) alors \((u_n)\) et \((v_n)\) convergent vers une même limite réelle \(l\) qui vérifie \(\forall n \in \N, u_n \leq l \leq v_n\)
\end{theo}
\begin{dem}[Théorème des suites adjacentes]
    On suppose les hypothèses réunies.\\~\\
    \begin{itemize}
        \item Montrons tout d'abord que  : \(\forall n \in \N, u_v \leq v_n\)\\~\\
        Raisonnons par l'absurde en supposant qu'il existe un entier naturel \(n_0\) tel que \(v_{n_0} < u_{n_0}\). Par monotonie des suites \((u_n)\) et \((v_n)\), on en déduit : 
        \[\forall n \in \N, n \geq n_0 \imp v_n \leq v_{n_0} < u_{n_0}\leq u_n\]
        ce qui donne 
        \[\forall n \in \N,u_{n_0}-v_{n_0}\leq u_n -v_n\]
        La suite \((u_n - v_n)\) étant convergente de limite nulle, par passage à la limite dans une inégalité large, on obtient alors : \(u_{n_0}-v_{n_0}\leq 0\) ce qui contredit l'hypothèse fait que \(v_{n_0}<u_{n_0}\)\\~\\
        \underline{conclusion} : \(\forall n \in \N, u_n \leq v_n\)
        \item Montrons alors que les suites \((u_n)\) et \((v_n)\) convergent.\\~\\
        Par décroissance de la suite \((v_n)\) et le résultat trouvé ci-dessus, on a  : \(\forall n \in \N, u_n \leq v_0\). La suite \((u_n)\) est donc croissante et majorée. Par théorème de la limite monotone, on en déduit que la suite \((u_n)\) converge\\~\\
        De même, la suite \((v_n)\) est décroissante et minorée (par \(u_0\)) donc elle converge. \\~\\
        On note \(l = \lim u_n\) et \(l' = \lim v_n\). Par opération algébrique sur les limites, la suite \((u_n - v_n)\) converge vers \(l-l'\). Par unicité de la limite, l'hypothèse faite sur la suite \((u_n - v_n)\) donne alors \((l-l' = 0)\) donc \(l = l'\)\\~\\
        \underline{conclusion} : les suites \((u_n)\) et \((v_n)\) convergent vers une même limite \(l\).
        \item Montrons que \(\forall n \in \N, u_n \leq  l \leq v_n\)
        Par théorème de la limite monotone, 
        \begin{itemize}
            \item comme \(u_n\) est croissante et convergente vers \(l\), on a \(l  = \sup_{n \in \N} u_n\)
            \item comme \(v_n\) est décroissante et convergente vers \(l\), on a \(l  = \inf_{n \in \N} u_n\)
        \end{itemize}
        \underline{conclusion} : \(\forall n \in \N, u_n \leq l \leq v_n\)
    \end{itemize}
    
\end{dem}

\section{Suites extraites}
\subsection{Définition}
\begin{defi}
    Soit \((u_n)\) une suite réelle.\\~\\
    On appelle suite extraite de \((u_n)\) toute suite \((v_k)\) telle que \(\forall k\in \N, v_k = u_{\phi(k)}\) avec \(\phi\) une fonction strictement croissante définie sur \(\N\) et à valeurs dans \(\N\).
\end{defi}
\subsection{Suites extraites et limites}
\begin{prop}
Si \(u_n\) est une suite réelle de limite \(l \in \Rb\) alors toutes les suites extraites de \((u_n)\) ont la même limite \(l\).
\end{prop}

\begin{dem}[Suites extraites et limites]
    \underline{Résultat préliminaire}\\~\\
    Soit \(\phi : \N \to \N\) une fonction strictement croissante.\\~\\
    On a \(\phi(0)\geq 0\). Soit \(k \in \N\) tel que \(\phi(k)\geq k\) alors par stricte croissance de \(\phi\), \(\phi(k+1)>\phi(k)\) donc, puisque \(\phi\) est à valeurs dans \(\N\), on a \(\phi(k+1)\geq \phi(k)+1\) et enfin \(\phi(k +1) \geq k+1\).\\~\\
    Par principe de récurrence, on a donc : \[\forall k \in \N, \phi(k)\geq k\]
    \begin{itemize}
        \item On suppose que \(u\) est une suite réelle de limite réelle \(l\) et \(\phi : \N \to \N\) une fonction strictement croissante.\\~\\
        Soit \(\epsilon\in \Rps\). Par hypothèse  sur la suite \(u\) il existe un entier naturel \(n_0\) tel que pour tout entier naturel \(n\) supérieur ou égal à \(n_0\), on a \(\abs{u_n - l \leq \epsilon}\)\\~\\
        Soit \(k \in \N\) tel que \(k \geq n_0\). Alors par stricte croissance de \(\phi\) et avec le résultat préliminaire, on a \(\phi(k)\geq \phi(n_0)\geq n_0\) ce qui permet d'obtenir, avec ce qui précède, \(\abs{u_{\phi(k)}-l}\leq l\)\\~\\
        Autrement dit, la suite \((u_{\phi(k)})\) a pour limite \(l\).
        \item On suppose que \(u\) est une suite réelle de limite \(\pinf\) et \(\phi : \N \to \N\) une fonction strictement croissante.\\~\\
        Soit \(A \in \Rps\). Par hypothèse sur la suite \(u\), il existe un entier naturel \(n_0\) tel que pour tout entier naturel \(n\) supérieur ou égal à \(n_0\), on a \(u_n \geq A\)\\~\\
        Soit \(k \in \N\) tel que \(k \geq n_0\). Comme ci-dessus obtient \(u_{\phi(k)}\geq A\).\\~\\
        En résumé : \(\forall A \in Rps, \exists n_0 \in \N, k \geq n_0 \imp u_{\phi(k)}\geq A\)
        \item Le cas où \(u\) est une suite réelle de limite \(\minf\) se traite de la même façon.
    \end{itemize}
    \underline{Conclusion} : Si \(u\) est une suite réelle de limite \(k \in \Rb\) alors toute suite extraite de \(u\) a pour limite \(l\).
\end{dem}

\begin{defprop}[Utilisation de suites extraites pour prouver une divergence]
    Soit \((u_n)\) une suite réelle.
    \begin{itemize}
        \item S'il existe une suite extraite de \((u_n)\) qui diverge alors la suite \((u_n)\) diverge
        \item S'il existe deux suites extraites de \((u_n)\) de limites réelles différentes alors la suite \((u_n)\) diverge
    \end{itemize} 
\end{defprop}

\begin{defprop}[Utilisation des suites extraites pour prouver une convergence]
    Soit \((u_n)\) une suite réelle.\\~\\
    Si les suites \(u_{2n}\) et \((u_{2n+1})\) ont pour limite \(l\) avec \(l\) appartenant à \(\Rb\) alors \((u_n)\) a pour limite \(l\)
\end{defprop}

\begin{theo}[Théorème de Bolzano-Weierstrass]
    Toute suite réelle bornée admet une suite extraite convergente.
\end{theo}
\begin{dem}[Théorème de Bolzano-Weierstrass]
    \begin{itemize}
        \item Montrons le résultat annoncé dans le cas des suites réelles\\~\\
        On suppose que \((u_n)\) est une suite réelle bornée.\\~\\
        \((u_n)\) admet donc une borne inférieure et une borne supérieure ; on note \(m = \inf_{n \in \N} u_n\) et \(M = \sup_{n \in \N}u_n\).
        \begin{itemize}
            \item \underline{Construction d'une suite de segments par dichotomie}\\~\\
            \begin{enumerate}
                \item On note \(I_0\) le segment \(\intervii{m}{M}\) : \(I_0\) est de longueur de \(M-m\) et contient tous les termes de la suite \(u_n\).
                \item L'un des deux segments \(\intervii{m}{\frac{m+M}{2}}\) ou \(\intervii{\frac{m+M}{2}}{M}\) contient nécessairement une infinité de termes de la suite \((u_n)\) ; on le note \(I_1 :I_1\) est inclus dans \(I_0\), est de longueur \(\frac{M-m}{2}\) et contient une infinité de termes de la suite \((u_n)\)
                \item à partir de \(I_1\), on construit un segment noté \(I_2\) inclus dans \(I_1\), de longueur \(\frac{M-m}{2^2}\) et qui contient une infinité de termes de la suite \((u_n)\)
            \end{enumerate}
            En répétant l'opération on construit ainsi une suite de segments \((I_n)\) telle que : \\~\\
            \begin{enumerate}
                \item \(\forall n \in \N,I_{n+1}\subset I_n\)
                \item pour tout \(n \in \N, I_n\) est de longueur \(\frac{M-m}{2^n}\)
                \item pour tout \(n \in \N,I_n\) contient une infinité de termes de la suite.
            \end{enumerate}
            Dans chaque segment \(I_n\), il y a une infinité de termes de la suite \((u_n)\). Il existe donc une application \(\phi : \N \to \N\) strictement croissante telle que 
            \[\forall n \in \N, u_{\phi(n)} \in I_n\]
            \item Montrons que la suite \((u_{\phi(n)})\) ainsi construite, qui est extraite de \((u_n)\), est une suite convergente.\\~\\
            Pour tout  \(n \in \N\), on note \(I_n = \intervii{\alpha_n}{\beta_n}\). Par décroissance de la suite \((I_n)\) pour l'inclusion, la suite \(\alpha_n\) est croissante et la suite \((\beta_n)\) est décroissante. Par ailleurs, la suite \(\paren{\beta_n - \alpha_n}\) est égale à la suite \(\frac{(M-m)}{2^n}\) donc elle converge vers \(0\).\\~\\
            Par théorème des suites adjacentes, on en déduit que les suites \((\alpha_n)\) et \((\beta_n)\) convergent vers une même limite \(l\). Le théorème d'encadrement utilisé avec les inégalités \(\forall n \in \N,\alpha_n \leq u_{\phi(n)}\leq \beta_n\) permet alors de conclure que la suite \((u_{\phi(n)})\) converge vers \(l\).
        \end{itemize}
    \item Montrons le résultat annoncé dans le cas des suites complexes
    Soit \(u_n\) une suite bornée de \(\C\).\\~\\
    Alors \((x_n) = (\Reel{u_n})\) et \((y_n) = (\Ima{u_n})\) sont deux suites bornées de \(\R\)\\~\\
    On peut donc extraire de \((x_n)\) une suite convergente \(x_{\phi_1(n)}\) notée \((a_n)\)\\~\\
    La suite \((y_{\phi_1(n)})\), notée \((\beta_n)\), est alors une suite bornée de \(\R\), car elle est extraite de la suite bornée \((y_n)\) de \(\R\). On peut donc extraire de \((b_n)\) une suite convergente \((\beta_{\phi_2(n)})\) notée \((\beta_n)\)\\~\\
    La suite \((a_{\phi_2(n)})\), notée \((\alpha_n)\), est alors convergente puisqu'elle est extraite de la suite convergente \((a_n)\)\\~\\
    On en déduit que la suite \((\alpha_n + \i \beta_n)\) est une suite extraite de \((u_n)\) qui converge.
    \underline{Conclusion} de toute suite bornée de complexes, on peut extraire une suite convergente
    \end{itemize}
\end{dem}

\section{Suite complexes}
\begin{defi}
    Toute fonction \(u\) définie sur \(\N\) et à valeurs dans \(\C\) est dite suite complexe.
\end{defi}

\begin{defprop}[Ce qui s’étend aux suites complexes]
    \begin{itemize}
        \item Notation séquentielle, modes de définition d’une suite, suite stationnaire
        \item Limite \underline{finie} : définition et caractérisation (cf. infra), unicité, opérations sur les limites \underline{finies}
        \item Convergence et divergence
        \item Suite bornée : définition (cf. infra), lien avec la convergence
        \item Suites extraites : définitions, propriétés, théorème de Bolzano-Weierstrass
    \end{itemize}
\end{defprop}

\begin{defprop}[Ce qui ne s’étend pas aux suites complexes]
    \begin{itemize}
        \item Notation de limite \underline{infinie}
        \item Résultats utilisant la relation d’ordre dont les théorèmes d’existence de limite.
    \end{itemize}
\end{defprop}

\subsection{Suite complexe bornée et limite d’une suite complexe}

\begin{defi}
    Une suite complexe \((u_n)\) est dite bornée s'il existe un réel strictement positif \(M\) tel que, pour tout entier naturel\(n, \abs{u_n}\leq M\)
\end{defi}
\begin{defi}[Limite d'une suite complexe]
    Soit \((u_n)\) une suite complexe et \(l\) un complexe.\\~\\
    On dit que la suite \((u_n)\) a pour limite \(l\) si tout disque fermé centré en \(l\) contient tous les termes de la suite \((u_n)\) à partir d'un certain rang, ce qui se traduit par 
    \[\quantifs{\forall \epsilon \in\Rps;\exists n_0 \in \N; \forall n \in \N}  n\geq n_0\imp \abs{u_n - l}\leq \epsilon\]
\end{defi}
\begin{defprop}[Caractérisation de la limite d'une suite complexe]
    Soit \((u_n)\) une suite complexe et \(l\) un complexe.\\~\\
    La suite complexe \((u_n)\) a pour limite \(l\) si et seulement si, les suites réelles \((\Reel{u_n})\) et \(\Ima{u_n}\) ont respectivement pour limites \(\Reel{l}\) et \(\Ima{l}\)
\end{defprop}

\chapter{Limite et continuité}

\minitoc
\begin{nota}
    Dans ce chapitre, \(I\) et \(J\) désignent des intervalles de \(\R\), non vides et non réduits à un point.
\end{nota}
\section{étude locale des fonctions à valeurs réelles}
\subsection{Limite en un point \(a\) de \(\Rb\) appartenant à \(I\) ou extrémité de \(I\)}

\begin{defi}
    Soit \(f\) une fonction définie sur \(I\) à  valeur dans \(\R\)
    \begin{itemize}
        \item Cas où \(a\) est un réel, appartenant à \(I\) ou extrémité de \(I\).\\
        On dit que \(f\) admet pour limite \(l\) en \(a\) si : \(\quantifs{\forall \epsilon \in \Rps;\exists\delta \in \Rps;\forall x \in I} \abs{x-a}\leq \delta \imp \abs{f(x)-l}\leq \epsilon\)
        \item cas où \(a = \pinf\) est extrémité de \(I\)\\
            On dit que \(f\) admet pour limite \(l\) en \(\pinf\) si : \(\quantifs{\forall \epsilon \in \Rps;\exists B \in \Rps;\forall x \in I} x\geq B \imp \abs{f(x)-l}\leq \epsilon\)
        \item cas où \(a = \minf\) est extrémité de \(I\)\\
         On dit que \(f\) admet pour limite \(l\) en \(\minf\) si : \(\quantifs{\forall \epsilon \in \Rps;\exists B \in \Rms;\forall x \in I} x\leq B \imp \abs{f(x)-l}\leq \epsilon\)
    \end{itemize}
\end{defi}

\begin{defi}[Définitions d’une limite infinie]
    \begin{itemize}
        \item cas où \(a\) est un réel, appartenant à \(I\) ou extrémité de \(I\).\\~\\
            On dit que \(f\) admet pour limite \(\pinf \) en \(a\) si : \(\quantifs{\forall A \in \Rps;\exists\delta \in \Rps;\forall x \in I} \abs{x-a}\leq \delta \imp f(x)\geq A\)\\
            On dit que \(f\) admet pour limite \(\minf \) en \(a\) si : \(\quantifs{\forall A \in \Rms;\exists\delta \in \Rps;\forall x \in I} \abs{x-a}\leq \delta \imp f(x)\leq A\)
        \item cas où \(a = \pinf\) est extrémité de \(I\)\\~\\
            On dit que \(f\) admet pour limite \(\pinf \) en \(\pinf\) si : \(\quantifs{\forall A \in \Rps;\exists B \in \Rps;\forall x \in I} x\geq B \imp f(x)\geq A\)\\
            On dit que \(f\) admet pour limite \(\minf \) en \(\pinf\) si : \(\quantifs{\forall A \in \Rms;\exists B \in \Rps;\forall x \in I} x\geq B \imp f(x)\leq A\)
        \item cas où \(a = \minf\) est extrémité de \(I\)\\~\\
            On dit que \(f\) admet pour limite \(\pinf \) en \(\minf\) si : \(\quantifs{\forall A \in \Rps;\exists B \in \Rms;\forall x \in I} x\leq B \imp f(x)\geq A\)\\
            On dit que \(f\) admet pour limite \(\minf \) en \(\minf\) si : \(\quantifs{\forall A \in \Rms;\exists B \in \Rms;\forall x \in I} x\leq B \imp f(x)\leq A\)
    \end{itemize}
\end{defi}

\begin{defprop}[Unicité]
    Si \(f\) admet une limite \(l\) en \(a\) alors celle-ci est unique et on note \(f(x)\underset{x\to a}{\to} l\) ou \(\lim_{x\to a} f(x) = l\).
\end{defprop}
\begin{defprop}[Existence d’une limite en un point où la fonction est définie]
    Si \(f\) est définie en \(a\) et possède une limite en \(a\) alors \(\lim_{x\to a} f(x) = f(a)\).
\end{defprop}

\begin{defprop}[condition nécessaire d'existence de limite]
    Si \(f\) possède une limite finie en \(a\) alors \(f\) est bornée au voisinage de \(a\).
\end{defprop}

\subsection{Limite à gauche et à droite en un réel appartenant à \(I\) ou extrémité de \(I\).}

\begin{nota}
    Soit \(f\) une fonction définie sur \(I\), à valeurs dans \(\R\).
\end{nota}
\begin{defi}
    Soit \(a\) un point de \(\R\), appartenant à \(I\) ou extrémité de \(I\).
    \begin{enumerate}
        \item On dit que \(f\) admet une limite à gauche en \(a\) si la restriction \(f_{| I \inter \intervee{\minf}{a}}\) admet une limite en \(a\)\\
        Dans ce cas, on note \(\lim_{x\to a^-}f(x)\)  ou \(\lim_{\substack{x \to a \\ x < a}} f(x)\) la limite obtenue.
        \item On dit que \(f\) admet une limite à droite en \(a\) si la restriction \(f_{| I \inter \intervee{a}{\pinf}}\) admet une limite en \(a\)\\
        Dans ce cas, on note \(\lim_{x\to a^+}f(x)\)  ou \(\lim_{\substack{x \to a \\ x > a}} f(x)\) la limite obtenue.
    \end{enumerate}
\end{defi}

\begin{defprop}[Condition nécessaire et suffisante d’existence de limite]
    Soit \(a\) un point de \(\R\) appartenant à \(I\) mais pas extrémité de \(I\)\\~\\
    \(f\) admet une limite en \(a\) \ssi les trois conditions suivantes sont réunies :
    \begin{enumerate}
        \item \(f\) a une limite à gauche en \(a\).
        \item \(f\) a une limite à droite en \(a\)
        \item \(\lim_{x\to a^-} f(x) = \lim_{x \to a^+} f(x) = f(a)\)
    \end{enumerate}
\end{defprop}
\subsection{Caractérisation séquentielle de la limite}
\begin{theo}
    Soit \(f\) une fonction définie sur \(I\), à valeurs dans \(\R\)\\
    Soit \(a\) un point de \(\Rb\), appartenant à \(I\) ou extrémité de \(I\), et \(l\) un point de \(\Rb\)\\~\\
    \(f\) admet une limite \(l\) en \(a\) \ssi pour toute suite \((x_n)\) d'éléments de \(I\) qui admet pour limite \(a\), la suite réelle \((f(x_n))\) admet pour limite \(l\)
\end{theo}

\subsection{Opérations sur les limites}
\begin{defprop}
    Soit \(a\) un point de \(\Rb\), appartenant à \(I\) ou extrémité de \(I\).
    Soit \(f\) et \(g\) deux fonctions définies sur \(I\) et à valeurs réelles et \(\lambda\) un réel
    \begin{enumerate}
        \item \underline{Addition} \\~\\
        \begin{enumerate}
            \item Si \(f(x) \underset{x\to a}{\to} l\) avec \(l \in \R\) et \(g \underset{x\to a}{\to} l'\) avec \(l' \in \R\) alors \((f+g)(x) \underset{x\to a}{\to} l + l'\)
            \item Si \(f(x) \underset{x\to a}{\to} \pinf\)  et \(g(x)\underset{x\to a}{\to} l'\) avec \(l' \in \R \union \accol{\pinf}\) alors \((f+g)(x) \underset{x\to a}{\to} \pinf\)
            \item     Si \(f(x) \underset{x\to a}{\to}\minf\)  et \(g(x)\underset{x\to a}{\to} l'\) avec \(l' \in \R \union \accol{\minf}\) alors \((f+g)(x)\underset{x\to a}{\to} \minf\)
        \end{enumerate}
        \item \underline{Multiplication par un réel}.\\~\\
            \begin{enumerate}
                \item Si \(f(x) \underset{x\to a}{\to} l \) avec \(l \in \R\) alors \(\lambda f(x) \underset{x\to a}{\to} \lambda l\)
                \item Si \(f(x) \underset{x\to a}{\to}\pinf\) alors \(\lambda f(x)\underset{x\to a}{\to}\begin{cases}
                    \pinf &\text{ si } \lambda>0 \\
                    0 &\text{ si } \lambda=0 \\
                    \minf &\text{ si } \lambda<0 \\
                \end{cases}\)
                \item Si \(f(x) \underset{x\to a}{\to} \minf\) alors \(\lambda f(x) \underset{x\to a}{\to} \begin{cases}
                    \pinf &\text{ si } \lambda<0 \\
                    0 &\text{ si } \lambda=0 \\
                    \minf &\text{ si } \lambda>0 \\
                \end{cases}\)
            \end{enumerate}
        \item \underline{Produit} 
            \begin{enumerate}
                \item Si \(f(x) \underset{x\to a}{\to} l\) avec \(l \in \R\) et \(g(x) \underset{x\to a}{\to} l'\) avec \(l' \in \R\) alors \((fg)(x)\underset{x\to a}{\to}ll'\)
                \item Si \(f(x) \underset{x\to a}{\to} \pinf\) et \(g(x) \underset{x\to a}{\to} l'\) avec \(l' \in \Rb \pd \accol{0}\) alors \((fg)(x) \underset{x\to a}{\to} \begin{cases}
                    \pinf &\text{ si } l' > 0 \\
                    \minf &\text{ si } l'<0
                \end{cases}\)
                \item Si \(f(x) \underset{x\to a}{\to} \minf\) et \(g(x) \underset{x\to a}{\to} l'\) avec \(l' \in \Rb \pd \accol{0}\) alors \((fg)(x) \underset{x\to a}{\to} \begin{cases}
                    \minf &\text{ si } l'> 0 \\
                    \pinf &\text{ si } l' < 0 
                \end{cases}\)
            \end{enumerate}
        \item \underline{Inverse}\\
            On suppose que \(f\) ne s'annule pas sur un voisinage de \(a\) sauf éventuellement en \(a\).
            \begin{enumerate}
                \item Si \(f(x) \underset{x\to a}{\to} l\) avec \( l \in Rs\) alors \(\frac{1}{f(x)} \underset{x\to a}{\to} \frac{1}{l}\)
                \item Si \(f(x) \underset{x\to a}{\to} l\) avec \( l\in \accol{\pinf, \minf}\) alors \(\frac{1}{f(x)} \underset{x\to a}{\to} 0\)
                \item Si \(f(x) \underset{x\to a}{\to} 0\) avec les termes \(f(x)\) strictement positifs au voisinage de \(a\) alors \(\frac{1}{f(x)}\underset{x\to a}{\to} \pinf\)
                \item Si \(f(x) \underset{x\to a}{\to} 0\) avec les termes \(f(x)\) strictement négatifs au voisinage de \(a\) alors \(\frac{1}{f(x)} \underset{x\to a}{\to} \minf\)
            \end{enumerate}
        \item \underline{Composition}\\
            Soit \(f\) une fonction définie sur \(I\) et à valeurs réelles telle que \(f (I) \subset J\).\\
            Soit \(g\) une fonction définie sur \(J\) et à valeurs réelles.\\
            Soit \(a\) un point de \(\Rb\), appartenant à \(I\) ou extrémité de \(I\).\\
            Soit \(b\) un point de \(\Rb\), appartenant à \(J\) ou extrémité de \(J\).\\
            Soit \(l\) un point de \(\Rb\).\\
            Si \(f\) admet pour limite \(b\) en \(a\) et si \(g\) admet pour limite \(l\) en \(b\) alors \(g \circ f\) admet pour limite \(l\) en \(a\). Autrement dit,
            \[f(x) \underset{x\to a}{\to} b \text{ et } g(y) \underset{y\to b}{\to} l \imp g \circ f (x)\underset{x\to a}{\to} l\]
    \end{enumerate}
\end{defprop}

\subsection{Limites et relation d’ordre}
Soit \(a\) un point de \(\Rb\), appartenant à \(I\) ou extrémité de \(I\).
\begin{defprop}[Passage à la limite d’une inégalité large]
    Soit \((l,l')\in \Rb \times \Rb\)\\~\\
    Si \(f\) et \(g\) sont deux fonctions définies sur \(I\), à valeurs réelles telles que \(f\leq g\) au voisinage  \(a\) avec \(f\) de limite \(l\) en \(a\) et \(g\) de limite \(l'\) en \(a\) alors \(l \leq l'\)
\end{defprop}

\begin{defprop}[Signe de la fonction et signe de la limite]
    Soit \(f\) une fonction définie sur \(I\), à valeurs réelles, de limite \(l \in \R\) en \(a\).
    \begin{itemize}
        \item Si \(l > 0\) alors \(f\) est strictement positive au voisinage de \(a\).
        \item Si \(l < 0\) alors \(f\) est strictement négative au voisinage de \(a\).
    \end{itemize}
\end{defprop}
\subsection{Existence d’une limite finie}
Soit \(a\) un point de \(\R\), appartenant à \(I\) ou extrémité de \(I\).
\begin{theo}[Théorème d’encadrement]
    Soit \(f\) une fonction définie sur \(I\) et à valeurs réelles, et \(l\) un nombre réel. S’il existe deux fonctions \(g\) et \(h\) définies sur \(I\), à valeurs réelles telles que \(g \leq f \leq h\) au voisinage de \(a\) avec \(g\) et \(h\) de même limite finie \(l\) en \(a\) alors \(f\) admet pour limite \(l\) en \(a\).
\end{theo}

\begin{defprop}[Propriété pratique]
    Soit \(f\) et \(g\) deux fonctions définies sur \(I\), à valeurs réelles, et \(l\) un nombre réel. S’il existe un voisinage de \(a\) sur lequel on \(a\) pour tout \(x\), \(\abs{f(x)-l}\leq g(x)\) avec g de limite \(0\) en \(a\) alors \(f\) a pour limite \(l\) en \(a\).
\end{defprop}

\begin{defprop}[Corollaires de la propriété pratique]
Soit \(f\) et \(g\) deux fonctions définies sur \(I\), à valeurs réelles.
    \begin{itemize}
        \item Si \(f\) a pour limite le réel \(l\) en \(a\) alors \(\abs{f}\) a pour limite \(\abs{l}\) en \(a\).
        \item Si \(f\) a pour limite \(0\) en \(a\) et si \(g\) est bornée au voisinage de \(a\) alors \(f g\) a pour limite \(0\) en a
    \end{itemize}
\end{defprop}

\subsection{Existence d’une limite infinie}
Soit \(f\) une fonction définie sur \(I\) et à valeurs réelles.\\~\\
Soit \(a\) un point de \(\R\), appartenant à \(I\) ou extrémité de \(I\).

\begin{theo}[Théorème de minoration]
    S’il existe une fonction \(g\) définie sur \(I\), à valeurs réelles, telle que \(g \leq f\) au voisinage de \(a\) avec \(g\) de limite \(\pinf\) en \(a\) alors \(f\) admet pour limite \(\pinf\) en \(a\).
\end{theo} 

\begin{theo} [Théorème de majoration]
S’il existe une fonction \(h\) définie sur \(I\), à valeurs réelles telle que \(f \leq h\) au voisinage de \(a\) avec \(h\) de limite \(\minf\) en \(a\) alors \(f\) admet pour limite \(\minf\) en \(a\).
\end{theo}

\subsection{Théorèmes de limite monotone}
\begin{theo}
    Soit \((a, b) \in \R \times \R\) avec \(a < b\).\\~\\
    \begin{itemize}
        \item Cas où la fonction f\( : \intervee{a}{b} \to \R\) définie sur \(\intervee{a}{b}\) est \underline{CROISSANTE}\\
        \begin{itemize}
            \item Si \(f\) est croissante et majorée alors \(f\) admet une limite finie en \(b\) et \( \lim_{x \to b^-} f(x) = \sup_{x \in \intervee{a}{b}} (f(x))\)
            \item Si \(f\) est croissante et non majorée alors \(f\) admet pour limite \(\pinf \) en \(b\).\\
            \item Si \(f\) est croissante et minorée alors \(f\) admet une limite finie en \(a\) et \( \lim_{x \to a^+} f(x) = \inf_{x \in \intervee{a}{b}} (f(x))\)
            \item Si \(f\) est croissante et non minorée alors \(f\) admet pour limite \(\minf\) en \(a\).
        \end{itemize}
        \item Cas où la fonction f\( : \intervee{a}{b} \to \R\) définie sur \(\intervee{a}{b}\) est \underline{DECROISSANTE}\\
        \begin{itemize}
            \item Si \(f\) est décroissante et minorée alors \(f\) admet une limite finie en \(b\) et \( \lim_{x \to b^-} f(x) = \inf_{x \in \intervee{a}{b}} (f(x))\)
            \item Si \(f\) est décroissante et non minorée alors \(f\) admet pour limite \(\minf \) en \(b\).\\
            \item Si \(f\) est décroissante et majorée alors \(f\) admet une limite finie en \(a\) et \( \lim_{x \to a^+} f(x) = \sup_{x \in \intervee{a}{b}} (f(x))\)
            \item Si \(f\) est décroissante et non majorée alors \(f\) admet pour limite \(\pinf\) en \(a\).
        \end{itemize}
\end{itemize}
\end{theo}

\section{Continuité des fonctions à valeurs réelles en un point}
Soit \(f\) une fonction définie sur \(I\), à valeurs dans \(\R\) et \(a\) un réel appartenant à \(I\).
\subsection{Définition}
\begin{defi}
    \begin{enumerate}
        \item \(f\) est dite continue en \(a\) si \(f\) admet pour limite \(f (a)\) en \(a\).
        \item \(f\) est dite continue à gauche en \(a\) si la restriction \(f_{|I\inter\intervee{\minf}{a}}\) est continue en \(a\) c’est-à-dire si \(\lim_{ x\to a^-}f(x)\) existe et vaut \(f (a)\).
        \item \(f\) est dite continue à droite en \(a\) si la restriction \(f_{|I\inter\intervee{a}{\pinf}}\) est continue en \(a\) c’est-à-dire si \(\lim_{ x\to a^+}f(x)\) existe et vaut \(f (a)\).
    \end{enumerate}
\end{defi}
\subsection{Condition nécessaire et suffisante de continuité en un point}
\begin{defprop}
     \(f\) est continue en \(a\) si, et seulement si, elle est continue à gauche et à droite en \(a\).
\end{defprop}
\subsection{Caractérisation séquentielle de la continuité en un point}
\begin{defprop}
     \(f\) est continue en \(a\) si, et seulement si, pour toute suite \((x_n)\) d’éléments de \(I\) qui admet pour limite \(a\), la suite réelle \((f (x_n))\) admet pour limite \(f (a)\).
\end{defprop}

\subsection{Opérations sur les fonctions continues en un point}
\begin{defprop}
    Soit \(f\) et \(g\) deux fonctions définies sur \(I\), à valeurs réelles.
    \begin{enumerate}
        \item \underline{Combinaison linéaire}\\~\\
            Si \(f\) et \(g\) sont continues en \(a\) et \((\lambda, \mu)\) est un couple de réels alors \(\lambda f + \mu g\) est continue en \(a\).
        \item \underline{Produit}\\~\\
            Si \(f\) et \(g\) sont continues en \(a\) alors \(f g\) est continue en \(a\).
        \item \underline{Quotient}\\~\\
            Si \(f\) et \(g\) sont continues en \(a\) et si \(g\) ne s’annule pas au voisinage de \(a\) alors \(f g\) est continue en \(a\).
    \end{enumerate}
\end{defprop}

\subsection{Composition de fonctions continues en un point}
    \begin{defprop}
        Soit \(f\) une fonction définie sur \(I\) et à valeurs réelles tel que, pour tout \(x\) de \(I\), \(f (x)\) appartient à \(J\).\\
        Soit \(g\) une fonction définie sur \(J\) et à valeurs réelles.\\
        Soit \(a\) un réel de \(I\).\\
        Si \(f\) est continue en \(a\) et si \(g\) est continue en\( f (a)\) alors \(g \circ f\) est continue en \(a\).\\
\end{defprop}


\subsection{Prolongement par continuité} 
\begin{defprop} 
    Soit \(b\) un réel n’appartenant pas à \(I\) mais extrémité de \(I\). \\
    Si \(f\) admet une limite finie \(l\) en \(b\) alors le prolongement de \(f\) à \(I \union \accol{b}\) noté \(\tilde{f} : I \union {b} \to \R\) défini par \(\forall x \in  I, \tilde{f} (x) = f (x)\) et \(\tilde{f} (b) = l \) est continu en \(b\) et appelé prolongement par continuité de \(f\) en \(b\).
\end{defprop}

\section{Continuité des fonctions sur un intervalle}
\subsection{Définition}
\begin{defi}
    Une fonction définie sur \(I\), à valeurs dans \(R\) est dite continue sur \(I\) si elle est continue en tout \(a\) de \(I\). \\
    L’ensemble des fonctions continues sur \(I\) à valeurs dans \(R\) est souvent noté \(\mathscr{C}\paren{I,\R} \) ou \(\classe{I}\)
\end{defi}
\subsection{Théorèmes généraux : combinaison linéaire, produit, quotient, composée}
\begin{theo}
    \begin{itemize}
        \item \(\forall \paren{f,g} \in \paren{\mathscr{C}\paren{I,\R}}^2,\forall \paren{\alpha,\beta}\in\R^2,\alpha f + \beta g \in \mathscr{C}\paren{I,\R}\)
        \item \(\forall \paren{f,g} \in \paren{\mathscr{C}\paren{I,\R}}^2,fg \in \mathscr{C}\paren{I,\R}\)
        \item \(\forall \paren{f,g} \in \paren{\mathscr{C}\paren{I,\R}}^2,g(I)\subset \Rs,\frac{f}{g}\in \mathscr{C}\paren{I,\R}\)
        \item \(\forall f \in \mathscr{C}\paren{I,\R},\forall g \in \mathscr{C}\paren{J,\R},f(I)\subset J \imp g \circ f \in \mathscr{C}\paren{I,\R}\)
    \end{itemize}
\end{theo}

\subsection{Théorème des valeurs intermédiaires et corollaires}
\begin{theo} [Théorème des valeurs intermédiaires]
Soit \(f\) une fonction définie sur \(I\) à valeurs dans \(\R\) et, \(a\) et \(b\) deux points de \(I\).\\
Si \(f\) est continue sur \(I\) avec \(f (a) \leq f (b)\) alors \(f\)\(\) atteint toute valeur intermédiaire entre \(f (a)\) et \(f (b)\)
\end{theo}
\begin{dem}
    On suppose les hypothèses réunies. Dans le cas \(a = b\), le résultat attendu est immédiat. On se place donc dans le cas \(a < b\) (sans perte de généralité) avec \(f (a) < f (b)\) (car le cas \(f (a) = f (b)\) est immédiat).\\~\\
Soit \(y\) un réel de l’intervalle \(\intervee{f(a)}{f(b)}\).\\~\\
\underline{Montrons, en suivant le principe de dichotomie, qu’il existe un réel \(x\) dans \(\intervii{a}{b} \) tel que \(y = f (x)\).
}
\begin{itemize}
    \item On note \(a_0 = a, b_0 = b\) ; on a alors \( f (a_0) < y < f (b_0)\).\\
    \item Etape \(1\) : on pose \( m_0 = \frac{1}{2}(a_0 + b_0)\).
        \begin{itemize}
            \item si \(y = f (m_0)\) alors on a bien trouvé un réel \(x\) dans \(\intervii{a}{b} \) tel que \(y = f (x)\) : c’est terminé !
            \item si \(f (a_0) < y < f (m_0)\), on pose \((a_1, b_1) = (a_0, m_0)\) et on continue la recherche de \(x\) dans \(\intervii{a_1}{b_1} \).
            \item si \(f (m_0) < y < f (b_0)\), on pose \((a_1, b_1) = (m_0, b_0)\) et on continue la recherche de \(x\) dans \(\intervii{a_1}{b_1} \).
        \end{itemize}
        Dans ces deux derniers cas, on a : \(f (a_1) < y < f (b_1)\) et on passe à l’étape \(2\).
    \item Etape \(2\) : on pose\( m_1 = \frac{1}{2}(a_1 + b_1)\).
        \begin{itemize}
            \item si \(y = f (m_1)\) alors on a bien trouvé un réel \(x\) dans \(\intervii{a}{b} \) tel que \(y = f (x)\) : c’est terminé !
            \item si \(f (a_1) < y < f (m_1)\), on pose \((a_2, b_2) = (a_1, m_1)\) et on continue la recherche de \(x\) dans \(\intervii{a_2}{b_2} \).
            \item si \(f (m_1) < y < f (b_1)\), on pose \((a_2, b_2) = (m_1, b_1)\) et on continue la recherche de \(x\) dans \(\intervii{a_2}{b_2} \).
        \end{itemize}
        Dans ces deux derniers cas, on a : \(f (a_2) < y < f (b_2)\) et on passe à l’étape \(3...0\).
\end{itemize}
Dans ce processus, s’il existe un entier \(k_0\) tel que \(f (m_{k_0} ) = y\), c’est terminé ! Sinon, on a créé une suite croissante \((a_k)\) et une suite décroissante \((b_k)\) telles que la suite \(\paren{a_k-b_k} = \paren{\frac{b-a}{2^k}}\) a pour limite \(0\).\\~\\
Ces suites sont donc adjacentes. Par théorème, elles convergent vers une même limite réelle \(l\) qui vérifie \(\forall k \in \N, a_k \leq l \leq b_k\) donc, en particulier, \(a_0 \leq l \leq b_0 \) c’est-à-dire \(a \leq l \leq b\).\\~\\
Comme \(f\) est continue, on en déduit alors que les suites \((f (a_k))\) et \((f (b_k))\) convergent vers \(f (l)\).\\~\\
De plus, par construction des suites \((a_k)\) et \((b_k)\), on a : \(\forall k \in  \N, f (a_k) < y < f (b_k)\). Par passage à la limite, on trouve donc : \(f (l) \leq y \leq f (l)\) puis, par antisymétrie, \(f (l) = y\) et c’est terminé !\\~\\
\underline{Conclusion} : \(f\) atteint toute valeur intermédiaire entre \(f (a)\) et \(f (b)\).
\end{dem}


\begin{defprop}[Image d’un intervalle]
    L’image d’un intervalle de \(R\) par une fonction continue à valeurs réelles est un intervalle de \(R\).
\end{defprop}
\begin{dem}
    On suppose que \(f\) est une fonction définie, continue sur un intervalle \(I\) et à valeurs réelles.\\~\\
    \underline{Montrons que \(f (I)\) est un intervalle de \(\R\)} à l’aide de la caractérisation des intervalles vue dans le chapitre “Compléments sur les réels”.\\~\\
    Soit \(\alpha\) et \(\beta\) deux réels quelconques de \(f (I)\) tels que \(\alpha < \beta\).\\~\\
    Alors il existe \(a\) et \(b\) deux réels de \(I\) tels que \(\alpha = f (a)\) et \(\beta = f (b)\).\\~\\
    Pour tout réel \(y\) de \(\intervii{\alpha}{\beta}\) , le théorème des valeurs intermédiaires assure alors l’existence d’un réel \(x\) compris entre \(a\) et \(b\) tel que \(y = f (x)\). Comme \(a\) et \(b\) sont des réels appartenant à l’intervalle \(I\), le réel \(x\) appartient aussi à l’intervalle \(I\) ce qui prouve que y appartient à \(f (I)\).\\~\\
    Ainsi : \(\forall (\alpha, \beta) \in (f (I))^2 , \alpha < \beta \imp \intervii{\alpha}{\beta} \subset f (I)\).\\~\\
    Par caractérisation des intervalles, on en déduit que \(f (I)\) est un intervalle.
\end{dem}

\begin{defprop}[Cas des fonctions continues strictement monotones]
    
Si \(f : I \to R\) est continue et strictement croissante sur \(I\), intervalle de bornes \(a\) et \(b\) avec \(a < b\), alors
\begin{itemize}
    \item  pour \(I = \intervii{a}{b}\), on a : \(f(I) = \intervii{f(a)}{f(b)}\)
    \item  pour \(I = \intervee{a}{b}\), on a : \(f(I) = \intervee{\lim_{x\to a^+}f(x)}{\lim_{x\to b^-}f(x)}\)
    \item  pour \(I = \intervie{a}{b}\), on a : \(f(I) = \intervie{f(a)}{\lim_{x\to b^-}f(x)}\)
    \item  pour \(I = \intervei{a}{b}\), on a : \(f(I) = \intervei{\lim_{x\to a^+}f(x)}{f(b)}\)
\end{itemize}
\end{defprop}
\subsection{Théorème des bornes atteintes et corollaire}
\begin{theo}[Théorème des bornes atteintes]
    Si \(f\) est une fonction continue sur un segment et à valeurs réelles alors \(f\) est bornée et atteint ses bornes.
\end{theo}
\begin{dem}
    On suppose les hypothèses réunies.\\~\\
    \(f\) étant continue sur l’intervalle \(\intervii{a}{b}\) et à valeurs réelles, l’image de \(\intervii{a}{b}\) par \(f\) est un intervalle \(J\). On note \(m\) la borne inférieure de \(J\) et \(M\) la borne supérieure de l’intervalle \(J\) considéré comme partie de la droite achevée \(\Rb\).

    Par propriété (vue dans le chapitre “Compléments sur les réels”), il existe une suite \((y_n)\) d’éléments de \(J\) de limite \(m\).\\~\\
    Comme \(J = f (\intervii{a}{b})\) , il existe alors une suite \((x_n)\) d’éléments de\( \intervii{a}{b}\) telle que \(\forall n \in \N, y_n = f (x_n)\).\\~\\
    La suite \((x_n)\) étant à valeurs dans \(\intervii{a}{b}\), elle est bornée. D’après le théorème de Bolzano-Weiestrass,elle admet donc une suite extraite convergente. On note \(x_{\phi(n)}\) une telle suite et \(l\) sa limite.\\~\\
    On a donc :
    \begin{itemize}
        \item  \(y_{\phi(n)} \to  m\) comme suite extraite d’une suite convergente de limite \(m\) ;
        \item \(x_{\phi(n)} \to l\)  ;
        \item \(f\) continue en \(l\) car \(f\) continue sur\( \intervii{a}{b}\) et \(l \in \intervii{a}{b}\), comme limite d’une suite à valeurs dans \(\intervii{a}{b}\).
    \end{itemize}
    On peut donc passer à la limite dans les égalités\\~\\
    \[\forall n \in \N, y_{\phi(n)} = f\paren{x_{\phi(n)}}\]
    Cela donne \(m = f (l)\) et prouve donc que \(m\) est un réel et que \(m\) est atteint par \(f\) .\\~\\
    On montre de même que \(M\) est un réel atteint par \(f\) .\\~\\
    \underline{Conclusion} : \(f\) est bornée et atteint ses bornes.
\end{dem}

\begin{defprop}[Image d’un segment]
    L’image d’un segment de \(\R\) par une fonction continue à valeurs réelles est un segment de \(\R\).
\end{defprop}

\subsection{Théorème de la bijection}
\begin{defprop}[Continuité et injectivité]
    Toute fonction continue sur un intervalle, à valeurs réelles et injective, est strictement monotone.\\~\\
\underline{Remarque}
    La réciproque est fausse ; en revanche, toute fonction strictement monotone sur un intervalle est injective.
\end{defprop}

\begin{dem}
    Soit \(I\) un intervalle de \(\R\), non vide et non réduit à un point, et \(f : I \to R\) continue et injective.\\~\\
    Raisonnons par l’absurde en supposant que \(f\) n’est ni strictement croissante, ni strictement décroissante. Alors,
    il existe \((a, b) \in I^2\) tel que \(a < b\) et \(f (a) \geq f (b)\) et il existe \((a', b') \in I^2\) tel que \(a' < b'\) et \(f (a') \leq f (b')\).\\~\\
    On note \(g : \intervii{0}{1} \to \R\) définie par : \(\forall t \in \intervii{0}{1}, g(t) = f ((1 - t)a' + ta) - f ((1 - t)b' + tb)\) . Par théorèmes généraux, \(g\) est continue sur \(\intervii{0}{1}\) avec \(g(0) = f (a') - f (b')\) et \(g(1) = f (a) - f (b)\) donc \(g(0) \leq 0 \) et \(g(1) \geq 0\). Par théorème des valeurs intermédiaires, il existe alors \(t_0 \in \intervii{0}{1}\) tel que \(g(t0) = 0\).\\~\\
    Ainsi \(f ((1 - t_0)a' + t_0a) = f ((1 - t_0)b' + t_0b)\) puis, par injectivité de \(f\) , \((1 - t_0)a' + t_0a = (1 - t_0)b' + t_0b\) ce qui donne \((1 - t_0)(b' - a') + t_0(b - a) = 0\). Comme les termes \((1 - t_0)(b' - a')\) et \(t_0(b - a)\) sont positifs, on en déduit que\( (1 - t_0)(b' - a') = t_0(b - a) = 0\) et enfin, comme \( b' - a'\) et \(b - a\) sont strictement positifs, on trouve \(1 - t_0 = 0\) et \( t0 = 0\) ce qui est absurde.\\~\\
    \underline{Conclusion} : \(f\) est strictement monotone.
\end{dem}
\begin{theo}[Théorème de la bijection]
Si \(f\) est une fonction à valeurs réelles définie, continue et strictement monotone sur un intervalle \(I\) alors \(f\) réalise une bijection de \(I\) sur\( J = f (I)\) dont la bijection réciproque \(f {^-1}\) est définie, continue et strictement monotone sur \(J\) avec même monotonie que \(f\) .
\end{theo}

\begin{dem}
On suppose les hypothèses réunies.\\~\\
\(f\) est injective (car strictement monotone) donc l’application \(\tilde{f} : I \mapsto f (I)\) définie par \(\forall x \in I, \tilde{f} (x) = f (x)\) est injective et surjective donc est une bijection : on dit que \(f\) réalise une bijection de \(I\) sur \(J = f (I)\). De plus, comme \(f\) est continue et à valeurs réelles, \(J = f (I)\) est un intervalle de \(R\), non vide (puisque \(I\) est non vide) et non réduit à un point de \(\R\) (puisque \(I\) n’est pas réduit à un point et que \(f\) est injective).\\~\\
La bijection réciproque \(\tilde{f^{-1}} : J \mapsto I\), notée plus simplement \(f ^{-1}\), est définie sur \(J\) et strictement monotone de même monotonie que \(f\) . En effet, si on suppose que \(f\) est strictement croissante (par ex), alors pour tout \((x, y) \in J^2\) tel que \(x < y\), on a \(f ^{-1}(x) < f ^{-1}(y)\) (sinon on aurait \(f ^{-1}(x) \geq f ^{-1}(y)\) puis par stricte croissance de \(f\) , \(x \geq y\) ce qui est faux) donc \(f ^{-1}\) est strictement croissante sur \(J\) par définition.\\~\\
Soit \(\lambda \in J\). Comme \(f ^{-1}\) est strictement monotone sur \(J\), le corollaire du théorème de limite monotone prouve (sous réserve que cela ait du sens) que \(l = lim _{\lambda^-} f ^{-1}\) existe, est finie et appartient à \(I\). Par continuité de \(f\) en \(l\), \(\lim_{x\to l}f (x) = f (l)\) puis par composition de limites, \(\lim_{ y\to\lambda^-} f \paren{f ^{-1}(y)} = f (l)\) ce qui donne \(f (l) = \lambda\) puis \(l = f ^{-1}(\lambda)\) et prouve que\( f ^{-1}\) est continue à gauche en \(\lambda\). On montre de même (sous réserve que cela ait du sens) la continuité à droite ce qui prouve la continuité de \(f ^{-1}\) en tout \(\lambda\) de \(J\).\\~\\
\underline{Conclusion} : \(f ^{-1}\) est définie, continue et strictement monotone sur \(J\) avec même monotonie que \(f\) .
\end{dem}


\section{Cas des fonctions à valeurs complexes}
\subsection{Ce qui s’étend aux fonctions complexes}
\begin{defprop}
    \begin{itemize}
        \item Limite finie :
        \begin{itemize}
            \item définition et caractérisations (cf infra) ;
            \item unicité ;
            \item opérations sur les limites finies ;
            \item lien entre existence d’une limite finie en un point et caractère borné au voisinage de ce point.
        \end{itemize}
        \item Continuité en un point et sur un intervalle.
    \end{itemize}
\end{defprop} 

\subsection{Ce qui ne s’étend pas aux fonctions à valeurs complexes}
\begin{defprop}
    \begin{itemize}
        \item Notion de limite infinie.
        \item Résultats utilisant la relation d’ordre dont les théorèmes d’existence de limite.
    \end{itemize}
\end{defprop}
\subsection{Limite d’une fonction à valeurs complexes}
\begin{defprop}
    Soit \(f\) une fonction définie sur \(I\) et à valeurs complexes, et \(l\) un nombre complexe.\\
    \underline{Définition} :\\~\\
    Soit \(a\) un point de \(I\) ou une extrémité de \(I\).\\
    On dit que \(f\) a pour limite \(l\) en \(a\) si la fonction à valeurs réelles \(\abs{f-l}\) a pour limite \(0\) en \(a\).
    \underline{Caractérisations} :\\~\\
    \begin{itemize}
        \item \(f\) admet pour limite \(l\) en \(a\) (point de \(I\) ou extrémité de \(I\)) si, et seulement si, \(\Reel{f}\) et \(\Ima{f}\) admettent respectivement pour limite \(\Reel{l}\) et \(\Ima{l}\) en \(a\).
        \item \(f\) est continue en \(a\) (point de \(I\)) si, et seulement si,  \(\Reel{f}\) et \(\Ima{f}\) le sont.
        \item \(f\) est continue sur \(I\) si, et seulement si, \(Re(f ) et Im(f )\) le sont.
    \end{itemize}
            
\end{defprop}


\chapter{Calcul matriciel et systèmes linéaires}

\minitoc

Dans ce chapitre \(\K = \R\) ou \(\C\)
\section{Matrices rectangles}
Soit \((m,n,p,q,r,s)\in \N^6\)
\subsection{Généralités}
\begin{defi}
    Toute application\( A :\intervii{1}{n}\times \intervii{1}{p} \to \K\) est appelée matrice de taille \((n, p)\) à coefficients dans \(K\).\\~\\
    \underline{Notations et représentation}
    \begin{itemize}
        \item Pour tout \((i, j) \in \intervii{1}{n}\times \intervii{1}{p} \), on pose \(a_{ij} = A(i, j)\) et on note usuellement
            \[A = \paren{a_{ij}}_{\substack {1\leq i \leq n \\ 1\leq j \leq p}}\]
        \item On représente \(A = {\paren{a_{ij}}}_{\substack{1\leq i \leq n \\ 1\leq j \leq p}}\) sous forme d’un tableau, à \(n\) lignes et \(p\) colonnes, dont l’élément situé en ligne \(i\) et colonne \(j\) est le nombre \(a_{ij}\).
    \end{itemize}
\end{defi}

\begin{defprop}[L’ensemble des matrices rectangles]
   L’ensemble des matrices de taille \((n, p)\) à coefficients dans \(\K\) est noté \(\mathcal{M}_{n,p}\paren{\K}\). 
\end{defprop}

\begin{defprop}[Opérations sur les matrices rectangles]
    On munit l'ensemble \(\mathcal{M}_{n,p}\paren{\K}\) de deux lois :
    \begin{itemize}
        \item une loi interne (addition entre matrices) notée \(+\) définie par :
        \[\forall (A,B) \in \paren{\mathcal{M}_{n,p}\paren{\K}}^2, A+B = {\paren{a_{ij} + b_{ij}}}_{\substack {1\leq i \leq n \\ 1\leq j \leq p}}\]
        \item une loi externe (Multiplication par un scalaire \ie un élement de \(\K\)) noté \(.\) définie par : 
        \[\forall \lambda \in \K, \forall A \in \mathcal{M}_{n,p}\paren{\K}, \lambda . A  = \paren{\lambda a_{ij}}_{\substack {1\leq i \leq n \\ 1\leq j \leq p}}\]
    \end{itemize}
\end{defprop}

\begin{defprop}[Matrices élémentaires]
    Toute matrice \(A\) de \(\mathcal{M}_{n,p}\paren{\K}\) peut s’écrire \(A = \sum_{\substack {1\leq i \leq n \\ 1\leq j \leq p}}a_{i,j}E_{ij}\) avec \(E_{ij}\) la matrice de\(\mathcal{M}_{n,p}\paren{\K}\) à coefficients tous nuls, sauf celui de la \(i^e\) ligne et \(j^e\) colonne qui vaut \(1\).
\end{defprop}


\subsection{Produit}
\begin{defi}
    On définit le produit de deux matrices rectangles de taille \((n, p)\) et \((p, q)\) de la manière suivante :
    \[\forall A \in \mathcal{M}_{n,p}\paren{\K},\forall B \in \mathcal{M}_{p,q}\paren{\K}, A \times B = \paren{c_{ik}}_{{\substack {1\leq i \leq n \\ 1\leq j \leq p}} \in \mathcal{M}_{n,q}\paren{\K}}\]
    avec 
    \[c_{ij} = a_{i1}b_{1j} + a_{i2} b_{2j} + \dots +a_{ip} b_{pj} = \sum_{k=1}^p a_{ik}b_{kj}\]
\end{defi}

\begin{prop}
    \begin{enumerate}
        \item Le produit matriciel est bilinéaire, c’est-à-dire :
         \[\forall (A,B) \in \paren{ \mathcal{M}_{n,p}\paren{\K}}^2, \forall C \in  \mathcal{M}_{p,q}\paren{\K}, \forall (\alpha,\beta) \in \K^2, \paren{\alpha A + \beta B}C = \alpha A C + \beta B C\]
         \[\forall (A,B) \in \paren{ \mathcal{M}_{n,p}\paren{\K}}^2, \forall C \in  \mathcal{M}_{p,q}\paren{\K}, \forall (\alpha,\beta) \in C\K^2, \paren{\alpha A + \beta B} = \alpha C A  + \beta C B\]
        \item Le produit matriciel est associatif, \cad : 
        \[\forall A \in  \mathcal{M}_{n,p}\paren{\K}, \forall B \in  \mathcal{M}_{p,q}\paren{\K},\forall C \in  \mathcal{M}_{q,r}\paren{\K} , \paren{AB}C =   A \paren{BC}\]
    \end{enumerate}
\end{prop}

\begin{dem}[Preuve de l'associativité du produit matriciel]
    Soit \(A \in  \mathcal{M}_{n,p}\paren{\K}, B \in  \mathcal{M}_{p,q}\paren{\K}, C \in  \mathcal{M}_{q,r}\paren{\K} \).
    On pose :
    \begin{itemize}
        \item \(A = \paren{a_{i,j}}_{\substack {1\leq i \leq n \\ 1\leq j \leq n}}\)
        \item \(B = \paren{b_{i,j}}_{\substack {1\leq i \leq p \\ 1\leq j \leq q}}\)
        \item \(C = \paren{c_{i,j}}_{\substack {1\leq i \leq q \\ 1\leq j \leq r}}\)
        \item \(BC = \paren{d_{i,j}}_{\substack {1\leq i \leq p \\ 1\leq j \leq r}} \)
        \item \(A\paren{BC} = \paren{e_{i,j}}_{\substack {1\leq i \leq n \\ 1\leq j \leq r}}\)
        \item \(AB = \paren{d'_{i,j}}_{\substack {1\leq i \leq n \\ 1\leq j \leq q}}\)
        \item \(\paren{AB}C = \paren{e'_{i,j}}_{\substack {1\leq i \leq n \\ 1\leq j \leq r}}\)
    \end{itemize}
    Soit \(\paren{i,j} \in \intervii{1}{n}\times \intervii{1}{r}\)
    \begin{align*}
        A\paren{BC} = e_{i,j} &= \sum_{k=1}^p a_{i,k}d_{k,j} \\
                              &= \sum_{k=1}^p \paren{a_{i,k} \paren{\sum_{s=1}^q b_{k,s}c_{s,j}}}\\
                              &= \sum_{k=1}^p \paren{\sum_{s=1}^q a_{i,k}b_{k,s}c_{s,j} }\\
                              &= \sum_{s=1}^q \paren{\sum_{k=1}^p a_{i,k}b_{k,s}c_{s,j} } \\
                              &= \sum_{s=1}^q \paren{\sum_{k=1}^p a_{i,k}b_{k,s}}c_{s,j}  \\
                              &= \sum_{s=1}^q d'_{i,s}c_{s,j} \\
                              &= \paren{AB}C\\
    \end{align*}
\end{dem}

\begin{defprop}[Produits remarquables]
    \begin{enumerate}
        \item Le produit d'une matrice \(A\) de \(\mathcal{M}_{n,p}\paren{\K}\) et d'une matrice colonne \(X\) de \(\mathcal{M}_{p,1}\paren{\K}\) est une combinaison linéaire des colonnes de \(A\).
        \item Le produit des matrices élémentaires \(E_{xy}\) de \(\mathcal{M}_{n,p}\paren{\K}\) et \(E_{zt}\) de \(\mathcal{M}_{p,q}\paren{\K}\) est la matrice \(\mathcal{M}_{n,q}\paren{\K}\) 
        \[E_{xy}E_{zt} = \delta_{y,z} E_{xt}\text{ avec } \delta_{y,z} = \begin{cases}
            1 & \text{ si } y = z \\
            0 & \text{ si } y \neq z
        \end{cases}\]
        \underline{Remarque}\\
        Pour \(\paren{i,j} \in \N^2 , \delta_{i,j}\) est appelé "sybome de Kronecker"
    \end{enumerate}
\end{defprop}
\subsection{Transposition}
\begin{defi}
    La transposée de \( A = \paren{a_{ij}}_{\substack {1\leq i \leq n \\ 1\leq j \leq p}}\) de \(\mathcal{M}_{n,p}\paren{\K}\) est la matrice de \(\mathcal{M}_{n,p}\paren{\K}\) notée \(\trans{A}\) définie par : 
    \[\trans{A} = \paren{b_{ij}}_{\substack {1\leq i \leq n \\ 1\leq j \leq p}} \in \mathcal{M}_{n,p}\paren{\K} \text{ avec } b_{ij} = a_{ji}\]
\end{defi}
\begin{defprop}[Linéarité de la transposition]
    \[\forall \paren{A,B} \in \mathcal{M}_{n,p}\paren{\K} \times \mathcal{M}_{n,p}\paren{\K}, \forall (\lambda,\mu) \in \K^2 , \trans{\paren{\lambda A + \mu B}} = \lambda \trans{A} + \mu \trans{B}\]
\end{defprop}

\begin{defprop}[transposée d'un produit]
    \[\forall \paren{A,B} \in \mathcal{M}_{n,p}\paren{\K} \times \mathcal{M}_{p,q},\trans{\paren{AB}} = \trans{B}\trans{A}\]
\end{defprop}

\section{Opérations élémentaires, systèmes linéaires}
\subsection{Définitions}
\begin{defi}
    On appelle opération élémentaire sur les lignes \(L_1, \dots , L_n\) d’une matrice de \(\mathcal{M}_{n,p}\paren{\K}\) l’une des opérations suivantes :
    \begin{enumerate}
        \item  Echange de deux lignes distinctes :
        \[L_r \leftrightarrow L_s\]
        avec \(r \neq s\)
        \item Multiplication d’une ligne par un scalaire non nul :    
        \[L_r \leftarrow \lambda L_r\]
        avec \(\lambda \neq 0 \).       
        \item Addition à une ligne du produit d’une autre ligne par un scalaire non nul :     
        \[L_r \leftarrow L_r + \lambda L_s\] 
        avec \(r\neq s\) et \( \lambda \neq 0\).
    \end{enumerate}
\end{defi}

\subsection{Traduction en termes de produit matriciel}
\begin{defprop}[Matrice identité]
    ~\\
    \begin{enumerate}
        \item La matrice de \(\mathcal{M}_{n,p}\paren{\K}\) définie par \(I_n = \begin{pmatrix}
1 & 0 & \cdots & 0 \\
0 & 1 & \ddots & \vdots \\
\vdots & \ddots & \ddots & 0 \\
0 & \cdots & 0 & 1
\end{pmatrix} \) est dite matrice identité
        \item \(\forall A \in \mathcal{M}_{n,p}\paren{\K}, I_nA = A\)
        \item \(\forall A \in \mathcal{M}_{n,p}\paren{\K}, A I_p = A\)
    \end{enumerate}
\end{defprop}

\begin{defprop}[Opérations élémentaires et produits matriciels]
    \begin{itemize}
        \item L’opération \(L_r \leftrightarrow L_s\) sur \(A \in \mathcal{M}_{n,p}\paren{\K}\) équivaut à la multiplication \(P_{r,s} \times A\) avec \(P_{r,s} \in \mathcal{M}_{n,p}\paren{\K}\) définie par :
        \[P_{r,s} = I_n +(E_{rs} + E_{rs} - E_{rr}-E_{ss})\]
        \(P_{r,s}\) est dite matrice de permutation
        \item L'opération \(L_r \leftarrow \lambda L_r\) sur \(A \in \mathcal{M}_{n,p}\paren{\K}\) équivaut à la multiplication \(D_{r,\lambda} \times A\) avec \(D_{r,\lambda} \times A\) avec \(D_{r,\lambda} \in \mathcal{M}_n\paren{\K}\) définie par : 
        \[D_{r,s} = I_n + \paren{\lambda - 1}E_{rr}\]
        \(D_{r,\lambda}\) est dite matrice de dilatation
        \item L'opération \(L_r \leftarrow L_r + \lambda L_s\) sur \(A\in \mathcal{M}_{n,p}\paren{\K}\) équivaut à la multiplication \(T_{r,s,\lambda} \times A \) avec \(T_{r,s,\lambda} \in \mathcal{M}_n\paren{\K}\) définie par :
        \[T_{r,s,\lambda} = I_n + \lambda E_{rs}\]
        \(T_{r,s,\lambda} \) est dite matrice de transvection.
    \end{itemize}
\end{defprop}

\subsection{Système d'équation linéaires}
\begin{defprop}
    Soit \(A = \paren{a_{ij}}_{\substack {1\leq i \leq n \\ 1\leq j \leq p}} \in \mathcal{M}_{n,p}\paren{\K}\) et \(B = \begin{pmatrix}
        b_1 \\ \vdots \\b_n 
    \end{pmatrix} \in \mathcal{M}_{n,1}\paren{\K}\)\\~\\
    Le Système linéaire \(\mathcal{S} : 
        \left\{
        \begin{array}{rcl}
        a_{11}x_1 + \dots + a_{1p}x_p & = & b_1 \\
        \vdots & & \vdots \\
        a_{n1}x_1 + \dots + a_{np}x_p & = & b_n
        \end{array}
        \right.
        \) d'inconnue \(\paren{x_1,\dots,x_n} \in \K^p\) se traduit matriciellement par l'équation \(AX = B\) d'inconnue \(X = \begin{pmatrix}
            x_1 \\\vdots\\x_n
        \end{pmatrix} \in \mathcal{M}_{p,1}\paren{\K}\) que l'on appelle encore système

    \begin{itemize}
        \item \underline{Compatibilité du système}\\~\\
        On dit que le système \(AX = B\) est compatible si \(B\) est combinaison linéaire des colonnes de \(A\) (ce qui assure l’existence de solutions au système).
        \item \underline{Ensemble-solution de \(\mathcal{S}\)}
        Si le système \(AX = B\) est compatible alors ses solutions sont les matrices \(X_0 + Y\) avec :
        \begin{enumerate}
            \item \(X_0 \in \mathcal{M}_{p,1}\paren{\K}\) une solution particulière de \(AX = B\) ;
            \item \(Y \in \mathcal{M}_{p,1}\paren{\K}\) solution quelconque du système homogène \(AX = 0\) associé.
        \end{enumerate}
        \item Résolution effective de \(\mathcal{S}\)
        Par opérations élémentaires sur les lignes du système \(\mathcal{S}\), on peut obtenir un système \(\mathcal{S}'\), dit équivalent à \(\mathcal{S}\) (car il a les mêmes solutions que \(\mathcal{S}\)) de forme trapézoïdale
        \[S' : \left\{
        \begin{array}{rccccccl}
        a'_{11}x_1 +&            &            &\cdots&      & + a'_{1p}x_p&=      & b'_1     \\
                    &a'_{22}x_2 +&            &\cdots&      & + a'_{2p}x_p&=      & b'_2     \\
                    &  \ddots    &            &      &      & \vdots      & \vdots&\vdots    \\
                    &            &a'_{qq}x_q  &+     &\cdots& + a'_{qp}x_p&=      & b'_q     \\
                    &            &            &      &      &0            &=      & b'_{q+1} \\
                    &            &            &      &      & \vdots      &\vdots& \vdots    \\
                    &            &            &      &      & 0           &=      & b'_n
        \end{array}
        \right.\] 
        qui peut se traduire matriciellement par :
        \[A'X = B'\]
        avec \(A'\) matrice  \(\mathcal{M}_{n,p}\paren{\K}\) telle que :
        \begin{itemize}
            \item  les lignes de \(1\) à \(q\) contiennent chacune au moins un coefficient non nul ;
            \item dans chaque ligne de \(2\) à \(q\), le premier coefficient non nul à partir de la gauche est situé à droite du premier coefficient non nul de la ligne précédente ;
            \item les lignes numérotées de q + 1 à n sont nulles.
        \end{itemize}
        Les \((n - q)\) dernières équations de \(\mathcal{S}'\) donnent les conditions de compatibilité du système. Ces conditions étant réunies, le nombre de paramètres pour la résolution est \((p - q)\).
    \end{itemize}
\end{defprop}

\section{Matrices carrées}
\subsection{Ensemble des matrices carrées}
\begin{defi}
    L’ensemble \(\mathcal{M}_{n,n}\paren{\K}\) est souvent noté plus simplement \(\mathcal{M}_n\paren{\K}\).
\end{defi}

\subsection{Matrices carrées de formes particulières}
\begin{defprop}
    Soit \(A =  \paren{a_{ij}}_{1\leq i,j \leq ,}\) une matrice de \(\mathcal{M}_n\paren{\K}\).
    \begin{enumerate}
        \item \underline{Matrices diagonales ou triangulaires}
        \begin{enumerate}
            \item \(A\) est dite scalaire s’il existe \( \lambda \in \K\) tel que \(A = \lambda I_n\).
            \item \(A\) est dite diagonale si \(\forall(i, j) \intervii{1}{n}^2 , i\neq j \imp a_{ij} = 0\).
            \item \(A\) est dite triangulaire supérieure si \(\forall(i, j) \intervii{1}{n}^2 , i > j \imp a_{ij} = 0\).
            \item \(A\) est dite triangulaire inférieure si \(\forall(i, j) \intervii{1}{n}^2 , i < j \imp a_{ij} = 0\).
        \end{enumerate}
        \item \underline{Matrices symétriques ou antisymétriques}
        \begin{enumerate}
            \item \(A\) est dite symétrique si \( \trans{A} = A\). \\
            On note \(\mathcal{S}_n \paren{\K}\) l’ensemble des matrices symétriques de \(\mathcal{M}_n\paren{\K}\).
            \item \(A\) est dite antisymétrique si \(\trans{A} = -A\). \\
            On note \(\mathcal{A}_n \paren{\K}\) l’ensemble des matrices antisymétriques de \(\mathcal{M}_n\paren{\K}\). 
        \end{enumerate}
    \end{enumerate}
\end{defprop}

\subsection{Deux formules usuelles}
Soit \(\paren{A,B} \in \paren{\mathcal{M}_n\paren{\K}}^2\)
\begin{itemize}
    \item \underline{Formule du binôme}\\~\\
        Si \(AB = BA\) alors, pour tout \(p \in \N\)
        \[\paren{A +B }^n = \sum_{k=0}^p \binom{p}{k}A^k B^{p-k}=\sum_{k=0}^p \binom{p}{k}A^{p-k} B^{k}\]
    \item \underline{Une formule de factorisation}\\~\\
        Si \(AB = BA\) alors, pour tout \(p \in \N\)
        \[A^p - B^p = \paren{A-B}\sum_{k=0}^{p-1} A^k B^{p-1-k} = \paren{A-B}\sum_{k=0}^{p-1} A^{p-1-k} B^{k}\]
\end{itemize}

\subsection{Matrices inversibles}
\begin{defprop}
    \begin{itemize}
        \item Une matrice \(A \in \mathcal{M}_n\paren{\K}\) est dite inversible s’il existe \(B \in \mathcal{M}_n\paren{\K}\) telle que
            \[AB = BA = I_n\]
        Dans ce cas, \\
        la matrice \(B\) est unique, notée \(B = A^{-1}\), et appelée matrice inverse de \(A\).
        \item L’ensemble des matrices inversibles de \(\mathcal{M}_n\paren{\K}\) est noté \(\mathcal{GL}_n\paren{\K}\) et appelé groupe linéaire.
    \end{itemize}
\end{defprop}
\begin{prop}
\begin{itemize}
    \item Si \(A\) et \(B\) sont deux matrices inversibles de \(\mathcal{M}_n\paren{\K}\) alors \(AB\) est inversible d’inverse \(A^{-1}B^{-1}\), autrement dit :
    \[\forall \paren{A, B} \in \paren{\mathcal{GL}_n\paren{\K}}^2 , AB \in \mathcal{GL}_n\paren{\K}\text{ et } \paren{AB}^{-1} = B^{-1}A^{-1}.\]
    \item Si \(A\) est une matrice inversible de \(\mathcal{M}_n\paren{\K}\) alors \(\trans{A}\) est inversible d’inverse \(\trans{\paren{A^{-1}}}\), autrement dit :
    \[\forall A \in \mathcal{GL}_n\paren{\K}, \trans{A} \in \mathcal{GL}_n\paren{\K} \text{ et } \paren{\trans{A}}^{-1} = \trans{\paren{A^{-1}}}\]
\end{itemize}
\end{prop}

\begin{defprop}[Trois caractérisations des matrices inversibles]
    Soit \(A \in \mathcal{M}_n\paren{\K}\).
    \begin{enumerate}
        \item \(A\) est inversible si, et seulement si, il existe \(B \in  \mathcal{M}_n\paren{\K}\) telle que \(AB = I_n\).\\~\\
        Dans ce cas,\( B = A^{-1}\).
        \item \(A\) est inversible si, et seulement si, il existe \(C \in  \mathcal{M}_n\paren{\K}\) telle que \(CA = I_n\).\\~\\
        Dans ce cas, \(C = A^{-1}\).
        \item \(A\) est inversible si, et seulement si, pour toute matrice colonne \(Y \in  \mathcal{M}_{n,1}\paren{\K}\), le système \(AX = Y \) d’inconnue \(X \in \mathcal{M}_{n,1}\paren{\K}\) a une unique solution.
    \end{enumerate}
\end{defprop}

\subsection{Calculs de matrices inverses en pratique}
\begin{defprop}[Calcul de l’inverse par résolution d’un système]
    La résolution du système \(AX = Y\) avec \(A \in  \mathcal{M}_n\paren{\K}\) et \(\paren{X, Y } \in  \paren{\mathcal{M}_n\paren{\K}}^2\) permet de déterminer si la matrice \(A\) est inversible et d’obtenir son inverse si celle-ci existe.\\~\\
    En effet,
    \begin{itemize}
        \item si \(A\) est inversible alors le système \(AX = Y\) a une unique solution \(X = A^{-1}Y\) . Dans ce cas, l’expression de \(X\) en fonction de \(Y\) obtenue après résolution permet d’expliciter \(A^{-1}\).
        \item Si le système \(AX = Y\) n’a pas de solution unique (pas de solution ou plusieurs solutions) alors \(A\) n’est pas inversible.
    \end{itemize}
\end{defprop}

\begin{defprop}[Préservation de l’inversibilité par les opérations élémentaires]
    Si \(A\) est une matrice carrée inversible alors la matrice obtenue à partir de \(A\) après des opérations élémentaires sur les lignes ou colonnes de \(A\) est inversible.\\~\\
    \underline{Remarques}
    \begin{itemize}
        \item Cela résulte de l’inversibilité des matrices de permutation, de dilatation et de transvection \(P_{r,s}, D_{r,\lambda}\) et \(T_{r,s,\lambda}\) et de la stabilité de \(\mathcal{GL}_n\paren{\K}\) par produit.
        \item Par contraposition, si la matrice obtenue à partir de \(A\) après des opérations élémentaires sur les lignes ou colonnes n’est pas inversible alors la matrice \(A\) n’est pas inversible.
    \end{itemize}

\end{defprop}

\begin{defprop}[Calcul de l’inverse par opérations élémentaires ]
    En réalisant en parallèle les mêmes opérations élémentaires sur les lignes (ou les colonnes) d’une matrice \(A\) de \(\mathcal{M}_n\paren{\K}\) et de la matrice identité \(I_n\), on peut déterminer si la matrice \(A\) est inversible et obtenir son inverse si celle-ci existe. En effet, en essayant de retransformer \(A\) en la matrice identité \(I_n\) et en reproduisant simultanément les même opérations sur la matrice identité \(I_n\) alors à la fin de la transformation, la matrice obtenue de la matrice identité est \(A^{-1}\). (méthode du pivot de Gauss-Jordan).\\~\\
    \underline{Remarque}\\~\\
    Dans cette méthode, il est impératif de ne pas mélanger les opérations sur les lignes et colonnes : autrement dit, on agit uniquement sur les lignes ou uniquement sur les colonnes. On pourra lui préférer la méthode de résolution du système dans laquelle la confusion ne peut se faire.
\end{defprop}

\subsection{Cas particulier}
\begin{defprop}[Matrices diagonales]
    Une matrice diagonale est inversible si, et seulement si, ses coefficients diagonaux sont tous non-nuls.\\~\\
    Dans ce cas, sa matrice inverse est diagonale.
\end{defprop}

\begin{defprop}[Matrices triangulaires]
    Une matrice triangulaire est inversible si, et seulement si, ses coefficients diagonaux sont tous non-nuls.\\~\\
    Dans ce cas, sa matrice inverse est triangulaire.
\end{defprop}

\chapter{Équations différentielles  linéaires}

\minitoc

Dans ce chapitre, \(I\) désigne un intervalle de \(\R\) non vide réduit à un point de \(\K\) l'ensemble \(\R\) ou \(\R\)

\section{Équations différentielles linéaires d’ordre \(1\)}
\subsection{Définition}
\begin{defi}
    Soit \(a\) et \(b\) deux fonctions continues sur \(I\), à valeurs dans \(\K\).\\~\\
    La fonction \(f : I \to \K\) est dite solution de l’équation différentielle linéaire du premier ordre
    \[(E) : y' + a(t)y = b(t)\]
    si \(f\) est dérivable sur \(I\) et vérifie :
    \[\forall t \in I, f'(t) + a(t)f (t) = b(t).\]
\end{defi}

\subsection{Forme générale des solutions}
\begin{defprop}
    Soit \(a\) et \(b\) deux fonctions continues sur \(I\), à valeurs dans \(\K\).\\~\\
    Les solutions de l’équation différentielle linéaire du premier ordre \((E) : y' + a(t)y = b(t)\) s’obtiennent en additionnant :
    \begin{itemize}
        \item UNE solution particulière de \((E)\) ;
        \item LES solutions de l’équation différentielle homogène associée \((H) : y' + a(t)y = 0\).
    \end{itemize}
    
\end{defprop}

\begin{dem}
    Soit \(a \) et \(b\) deux fonctions continues sur \(I\), à valeurs dans \(\K\).\\~\\
    on pose \((E) : y' + a(t)y = b(t)\)\\~\\
    Supposons que \(y_0\) est solution de \(E\)\\~\\
    Soit \(y : I \mapsto \K\) dérivable\\~\\
    \begin{align*}
        y \text{ solution de } (E) & \iff \forall t \in I,y'(t) + a(t) y(t) = y_0'(t) + a(t) y_0(t) \\
        &\iff \forall t \in I,\paren{y-y_0}'(t) + a(t)\paren{y-y_0}(t) = 0\\
        &\iff y-y_0 \text{ solution de } (H):z' + a(t)z = 0 \\
        &\iff y_0 \text{ s'écrit } y = y_0 +z \text{ où } z \text{ est une solution quelconque de } (H) \text{ sur } I
    \end{align*}
    Ainsi : 
    \[\mathcal{S}_{E,I} = y_0+\mathcal{S}_{H,I}\]
\end{dem}

\subsection{Solutions de l’équation différentielle homogène \(y' + a(t)y = 0\).}
\begin{defprop}
    Soit \(a\) une fonction continue sur \(I\), à valeurs dans \(\K\).\\~\\
    L’ensemble des solutions de l’équation différentielle linéaire homogène \((H) : y' + a(t)y = 0\) sur \(I\) est
    \[\mathcal{S}_H = \accol{t\mapsto \lambda e^{-A(t)} \tq \lambda \in \K}\]
    où \(A\) désigne une primitive de la fonction \(a\) sur \(I\).
\end{defprop}
\begin{dem}
    Résolution de \((H) : y' + a(t)y = 0\) sur \(I\)\\~\\
    On note \(A\) une primitive de \(a\) sur \(I\)\\~\\
    Soit \(y : I \mapsto \K\) dérivable sur I\\~\\
    \begin{align*}
        y \text{ solution de } (H) & \iff \forall t \in I, y'(t) + A'(t)y(t) =0 \\
        &\iff \forall t \in I, y'(t)e^{A(t)} + A'(t) e^{A(t)} y(t) \text{ car } \forall t \in I,e^{A(t)} \neq 0 \\
        &\iff \forall t \in I, g'(t) = 0 \text{ avec } g(t) = y(t)e^{A(t)}\\
        &\iff \exists \lambda \in \K, \forall t \in I, g(t) = \lambda \\
        &\iff \exists \lambda \in \K, \forall t \in y(t) = \lambda e^{-A(t)}
    \end{align*}
    \underline{Conclusion} : 
    \[\mathcal{S}_H = \accol{t\mapsto \lambda e^{-A(t)} \tq \lambda \in \K}\]
\end{dem}

\subsection{Solution particulière de l’équation différentielle \( y' + a(t)y = b(t)\).}
\begin{defprop}[Principe de superposition de solutions]
    Soit \(a, b_1\) et \(b_2\) des fonctions continues sur \(I\), à valeurs dans \(\K\).
    \[\text{Si }\begin{cases}
        f_1 : I \to K \text{ est solution de l’équation différentielle linéaire } y' + a(t)y = b_1(t) \text{ sur } I\\
        f_2 : I \to K \text{ est solution de l’équation différentielle linéaire } y' + a(t)y = b_2(t) \text{ sur } I\\
    \end{cases}\]

    alors, \( f_1 + f_2 : I \to K\) est solution sur \(I\) de l’équation différentielle linéaire \(y' + a(t)y = b_1(t) + b_2(t)\).
\end{defprop}

\begin{defprop}[Détermination d’une solution particulière \(y_0\)]

    Soit \(a\) et \(b\) deux fonctions continues sur \(I\), à valeurs dans \(\K\).\\~\\
    S’il n’y a pas de solution particulière évidente/connue pour \((E) : y' + a(t)y = b(t)\) ou si le principe de superposition des solutions n’est pas applicable pour en déterminer une alors on pourra chercher une solution particulière de \((E)\) selon la méthode dite de "variation de la constante" c’est-à-dire sous la forme
    \[y_0 : t \mapsto \lambda(t)e^{-A(t)}\]
    avec \(A\) une primitive de \(a\) sur \(I\) et \(\lambda\) une fonction inconnue dérivable sur \(I\) à valeurs dans \(\K\).
\end{defprop}

\begin{dem}[Démonstration de la méthode de la variation de la constante]
    Soit \(a\) et \(b\) deux fonctions continues sur \(I\), à valeurs dans \(\K\).\\~\\
    Résolution de \((E) : y' + a(t)y = b(t)\) sur \(I\)\\~\\
    On pose \(y_0(t) = \lambda(t)e^{-A(t)}\) avec \(\lambda\) une fonction dérivable sur \(I\) et à valeur dans \(\K\) et \(A\) une primitive de \(a\)
    \begin{align*}
        y_0 \text{ solution de } (E) &\iff \forall t \in I,y_0'(t) + a(t)y_0(t) = b(t) \\
        &\iff \forall t \in I,\lambda'(t)e^{-A(t)} + \lambda(t)\paren{-a(t)e^{-A(t)} + a(t)e^{-A(t)}} = b(t) \\
        &\iff \forall t \in I,\lambda'(t)e^{-A(t)} = b(t) \\
        &\iff \forall t \in I, \lambda'(t) = b(t)e^{A(t)}
    \end{align*}
    Ainsi en primitivant \(b(t)e^{A(t)}\) (qui existe car \(b\) et \(A\) sont continue) on trouve une solution particulière
\end{dem}

\subsection{Théorème de Cauchy : existence et unicité}
\begin{theo}
    Soit \(a\) et \(b\) deux fonctions continues sur \(I\), à valeurs dans \(\K\).\\~\\
    Pout tout \(t_0 \in I\) et tout \(\alpha_0 \in \K \), il existe une unique solution \(f\) sur \(I\) de l’équation différentielle linéaire du premier ordre \(y' + a(t)y = b(t)\) telle que \(f(t_0) = \alpha_0\)
\end{theo}

\section{Equations différentielles linéaires d’ordre \(2\) à coefficients constants}
\subsection{Définition}
\begin{defi}
    Soit \(a\) et \(b\) deux éléments de \(\K\) et \(g\) une application continue sur \(I\), à valeurs dans \(\K\).\\~\\
    La fonction\( f : I \to \K\) est dite solution de l’équation différentielle linéaire d’ordre \(2\) à coefficients constants
    \[(E) : y''+ ay' + by = g(t)\]
    si \(f\) est deux fois dérivable sur \(I\) et vérifie :
    \(\forall t \in I, f''(t) + af'(t) + bf (t) = g(t).\)
\end{defi}

\subsection{Forme générale des solutions}
\begin{defprop}
    Soit \(a\) et \(b\) deux éléments de \(\K\) et \(g\) une application continue sur \(I\), à valeurs dans \(\K\).\\~\\
    Les solutions de l’équation différentielle linéaire du second ordre\( (E) : y'' + ay' + by = g(t)\) s’obtiennent en additionnant :
    \begin{itemize}
        \item une solution particulière de \((E)\) ;
        \item les solutions de l’équation différentielle homogène associée \((H) : y'' + ay' + by = 0\)
    \end{itemize}
    
\end{defprop}



\subsection{Solutions de l’équation différentielle linéaire homogène \(y'' + ay' + by = 0\)}

\begin{defprop}[Equation caractéristique]
    Soit \(a\) et \(b\) deux éléments de \(\K\).\\~\\
    La recherche de solutions de l’équation différentielle linéaire homogène à coefficients constants
    \[(H) : y'' + ay' + by = 0\]
    sous la forme \(t \mapsto e^{rt}\) avec \(r \in \K\) conduit à l’équation
    \[(EC) : r^2 + ar + b = 0\]
    dite équation caractéristique associée à \((H)\).
\end{defprop}

\begin{defprop}[Ensemble des solutions dans le cas où \(\K = \C\)]
    Soit \(a\) et \(b\) deux éléments de \(\C\)\\~\\
    On note \(\mathcal{S}_H\) l’ensemble des solutions sur \(I\) de l’équation différentielle \((H) : y'' + ay' + by = 0\)
    \begin{itemize}
        \item Si l'équation caractéristique \((EC)\) a deux racines distinctes \(r_1\) et \(r_2\) alors : 
        \[\mathcal{S}_H = \accol{t \mapsto \lambda_1 e^{r_1 t} + \lambda_2 e^{r_2 t} \tq \paren{\lambda_1,\lambda_2} \in \C^2}\]
        \item Si l'équation caractéristique \((EC)\) a une racine double \(r\) alors 
        \[\mathcal{S}_H = \accol{t \mapsto \paren{\lambda_1 + \lambda_2  t }e^{r t} \tq \paren{\lambda_1,\lambda_2} \in \C^2}\]
    \end{itemize}

\end{defprop}
\begin{defprop}[Ensemble des solutions dans le cas où \(\K = \R\)]
    Soit \(a\) et \(b\) deux éléments de \(\R\)\\~\\
    On note \(\mathcal{S}_H\) l’ensemble des solutions sur \(I\) de l’équation différentielle \((H) : y'' + ay' + by = 0\)
    \begin{itemize}
        \item \item Si l'équation caractéristique \((EC)\) a deux racines distinctes \(r_1\) et \(r_2\) alors : 
        \[\mathcal{S}_H = \accol{t \mapsto \lambda_1 e^{r_1 t} + \lambda_2 e^{r_2 t} \tq \paren{\lambda_1,\lambda_2} \in \R^2}\]
        \item Si l'équation caractéristique \((EC)\) a une racine double \(r\) alors 
        \[\mathcal{S}_H = \accol{t \mapsto \paren{\lambda_1 + \lambda_2  t }e^{r t} \tq \paren{\lambda_1,\lambda_2} \in \R^2}\]
        \item Si l'équation caractéristique \((EC)\) a deux racines complexes conjuguées \(r\) et \(\conj{r}\) non réelles alors
        \[\mathcal{S}_H = \accol{t \mapsto e^{\alpha t}\paren{\lambda_1 \cos (\beta t) + \lambda_2 \sin (\beta t)}\tq \paren{\lambda_1,\lambda_2} \in \R^2}\]
        \[\text{avec } \alpha = \Reel{r} \text{ et } \beta = \Ima{r}\]

    \end{itemize}
\end{defprop}

\begin{defprop}[Structure de l’ensemble des solutions]
    Soit \(a\) et \(b\) deux éléments de \(\K\).\\~\\
    Les deux points précédents permettent de mettre en évidence le résultat suivant.\\~\\
    L’ensemble des solutions sur \(I\) de l’équation différentielle linéaire \((H) : y'' + ay' + by = 0\) est donc :
    \[\mathcal{S}_H = \accol{t \mapsto \lambda_1 y_1 (t) + \lambda_2 y_2 (t) \tq \paren{\lambda_1,\lambda_2} \in \K ^2 }\]
    où \((y_1, y_2)\) un couple de fonctions non colinéaires solutions de \((H)\) sur \(I\).
\end{defprop}

\subsection{Solution particulière de l’équation différentielle \(y'' + ay' + by = g(t)\).}

\begin{defprop}[Principe de superposition de solutions]
    Soit \(a, b\) deux éléments de \(\K\), \(g_1\) et \(g_2\) des fonctions continues sur \(I\), à valeurs dans \(\K\).
    \[\text{Si }\begin{cases}
        f_1 : I \to K \text{ est solution de l’équation différentielle linéaire } y'' + ay' + by= g_1(t) \text{ sur } I\\
        f_2 : I \to K \text{ est solution de l’équation différentielle linéaire } y'' + ay' + by= g_2(t) \text{ sur } I\\
    \end{cases}\]

    alors, \( f_1 + f_2 : I \to K\) est solution sur \(I\) de l’équation différentielle linéaire \(y'' + ay' + by = b_1(t) + b_2(t)\).
\end{defprop}

\begin{defprop}[Détermination d’une solution particulière y0]
    Soit \(a\) et \(b\) deux éléments de \(\K\) et \(g\) une application continue sur \(I\), à valeurs dans \(\K\).\\~\\
    Selon le programme de MP2I,\\~\\
    s’il n’y a pas de solution particulière \(y_0\) évidente/connue pour \((E) : y'' + ay' + by = g(t)\) ou si le principe de superposition ne s’applique pas pour en déterminer une, les étudiants doivent savoir en trouver une dans les trois cas suivants selon le type du second membre.
    \begin{itemize}
        \item Cas où \(g\) est une fonction polynomiale de degré \(n\)\\~\\
        On pourra chercher \(y_0\) sous la forme d’une fonction polynomiale de degré \(n\) si \(b\) est différent de\( 0\) ou de degré \(n + 1\) si \(b\) est égal à \(0\).
        \item Cas où \(g : t \mapsto A e^{\lambda t}\) avec \(A\) et \(\lambda\) deux éléments de \(\K\).
        On pourra chercher \(y_0\) sous l’une des formes suivantes selon la valeur de \(\lambda\) :
        \[y_0 : t \mapsto \begin{cases}
            \alpha e^{\lambda t} &\text{ si } \lambda \text{ n'est pas racine de } (EC)\\
            \alpha t e^{\lambda t} &\text{ si } \lambda \text{ n'est pas racine simple de } (EC) \text{ avec } \alpha \in \K\\
            \alpha t^2 e^{\lambda t} &\text{ si } \lambda \text{ n'est pas racine double de } (EC)\\
        \end{cases}\]
        \item  Cas où \(\K = \R\) et \(g : t \mapsto B \cos (\omega t)\) [ou \(g : t \mapsto B \sin (\omega t)\)] avec \(B\) et \(\omega\) deux éléments de \(\R\)\\~\\
            On pourra, à l’aide de la méthode décrite ci-dessus, déterminer une solution particulière \(z_0\) de l’équation
           \[ y'' + ay' + by = B e^{i\omega t}\]
            et conclure que \(y_0 = \Reel{z_0}\) [ou \(y_0 = \Ima{z_0}\) selon le cas étudié] convient.
    \end{itemize}
\end{defprop}

\subsection{Théorème de Cauchy : existence et unicité (preuve hors programme)}
\begin{defprop}
    Soit \(a\) et \(b\) deux éléments de \(\K\) et \(g\) une application continue sur \(I\), à valeurs dans \(\K\).\\~\\

    Pout tout \(t_0 \in I\) et tout \((\alpha_0, \beta_0) \in \K^2\), il existe une unique solution \(f\) sur \(I\) de l’équation différentielle linéaire du second ordre à coefficients constants \(y'' + ay' + by = g(t)\) telle que \(f (t_0) = \alpha_0\) et \(f'(t_0) = \beta_0\).
\end{defprop}

\chapter{Arithmétique dans \(\Z\)}

\minitoc

\section{Division euclidienne}
Soit \((a, b, c, d) \in \Z^4\).
\subsection{Divisibilité dans \(\Z\)}
\begin{defi}
	S’il existe \(q\) dans \(\Z\) tel que \(a = bq\), on dit que \(b\) divise \(a\) (ou \(b\) est un diviseur de \(a\)) et on note \(b \divise a\).\\~\\
	Dans ce cas,\\~\\
	on dit aussi que \(a\) est divisible par \(b\) (ou \(a\) est un multiple de \(b\)).
\end{defi}

\begin{defprop}[Ensembles des diviseurs et des multiples]
	\begin{itemize}
		\item On note \(\mathcal{D}(a) = \accol{b \in \Z \tq \exists q \in \Z, a = bq}\) l’ensemble des diviseurs de \(a\).
		      \begin{itemize}
			      \item Si \(a = 0\) alors \(\mathcal{D}(a) = \Z\) donc \(\mathcal{D}(a)\) est infini.
			      \item Si \(a \neq 0\) alors \(\mathcal{D}(a) \subset \interventierii{-\abs{a}}{\abs{a}} \) donc \(\mathcal{D}(a)\) est fini.
		      \end{itemize}
		\item On note \(b\Z = \accol{bq \tq q \in \Z}\) l’ensemble des multiples de \(b\).
		      \begin{itemize}
			      \item Si \(b = 0\) alors \(b\Z = \accol{0}\) donc \(b\Z\) est fini.
			      \item Si \(b \neq 0\) alors \(b\Z\) est infini.
		      \end{itemize}
	\end{itemize}
\end{defprop}

\begin{defprop}[Caractérisation des couples d’entiers associés]
	Si l’une des propositions équivalentes suivantes est vérifiée, on dit que les entiers \(a\) et \(b\) sont associés.
	\begin{enumerate}
		\item \(a \divise b\) et \(b \divise a\).
		\item \(\abs{a} = \abs{b}\).
		\item \(a = b\) ou \(a = -b\).
	\end{enumerate}
\end{defprop}

\begin{defprop}[Propriétés immédiates]
	\begin{enumerate}
		\item \(a \divise a\).
		\item Si \(a \divise b\) et \(b \divise c\) alors \(a \divise c\).
		\item Si \(a \divise b\) et \(c \divise d\) alors \(ac \divise bd\).
		\item Si \(a \divise b\) alors, pour tout \(n \in \Ns\), \(an\divise bn\).
		\item Si \(c \divise a\) et \(c \divise b \) alors, pour tout \( (u, v) \in\Z^2, c \divise au + bv\).
		\item Si \(a = bc + d\) alors \(\mathcal{D}(a) \inter \mathcal{D}(b) = \mathcal{D}(b) \inter \mathcal{D}(d)\).
	\end{enumerate}
\end{defprop}

\subsection{Division euclidienne}
\begin{theo}[Théorème de la division euclidienne]
	Pour tout couple \((a, b)\) de \(\Z \times \Ns\), il existe un unique couple \((q, r)\) de \(\Z^2\) tel que :\\~\\
	\[a = bq + r \text{ et } 0 \leq r \leq b - 1.\]
	Dans la division euclidienne de \(a\) par \(b\), \(a\) est appelé dividende, \(b\) diviseur, \(q\) quotient et \(r\) reste.
\end{theo}

\begin{dem}
	Soit \((a,b)\in \Z \times \Ns\)\\~\\
	\existence posons \(q = \floor{\frac{a}{b}}\) et \(r = a-bq\)\\~\\
	alors \(q\leq \frac{a}{b}<b+1\)\\~\\
	donc \(bq\leq a < b(q+1) \iff 0\leq r<b\).\\~\\
	De plus \(q \in \Z\) donc \(r \in \Z\) ainsi avec ce qui précède on a : \(r \in \interventierii{0}{b-1}\)\\~\\
	\unicite On suppose qu'il existe \(\begin{cases}
		(q,r)   & \in \Z \times \interventierii{0}{b-1} \\
		(q',r') & \in \Z \times \interventierii{0}{b-1} \\
	\end{cases}\) tel que \(\begin{cases}
		a & = bq+r   \\
		a & = bq'+r'
	\end{cases}\)
	\\~\\
	Alors \(b(q-q') = r'-r\) Ainsi\\~\\
	Si \(q-q' \neq 0\) alors \(\abs{q-q'}\geq 1\) puis \(\abs{r'-r}\geq \abs{b}\) et donc \(\abs{r'-r}\geq b\) car \(b>0\).\\~\\
	or \(\abs{r'-r\leq b-1}\) car \(\begin{cases}
		0\leq r \leq b-1 \\
		0\leq r' \leq b-1
	\end{cases}\) \\~\\
	Donc \(q-q' = 0 \imp q=q'\) et aussi \(r-r' = 0\imp r = r'\)
	\conclusion Le couple \(q,r\) existe et est unique
\end{dem}


\begin{defprop}[Caractérisation de la divisibilité]
	Soit \((a, b) \in \Z \times \Ns\).\\~\\
	\(b\) divise \(a\) si, et seulement si, le reste de la division euclidienne de \(a\) par \(b\) est nul.
\end{defprop}

\section{PGCD et PPCM}
\subsection{Cas de deux entiers naturels}
\begin{defi}[Définition du PGCD]
	Soit \((a, b) \in \N \times \Ns\).\\~\\
	Le plus grand élément (au sens de \(\leq\)) de l’ensemble des diviseurs communs à \(a\) et \(b\) est dit PGCD de \(a\) et \(b\) et noté \(a \wedge b\) :
	\[a \wedge b = \max (\mathcal{D}(a) \inter \mathcal{D}(b))\]
\end{defi}

\begin{prop}[Propriété importante]

	Soit \((a, b) \in \N \times \Ns\).\\~\\
	Si \(r\) est le reste de la division euclidienne de \(a\) par \(b\) alors \(\begin{cases}
		\mathcal{D}(a) \inter \mathcal{D}(b) & = \mathcal{D}(b) \inter \mathcal{D}(r) \\
		a\wedge b                            & = b\wedge r
	\end{cases}\)
\end{prop}

\begin{dem}
	Soit \((a, b) \in \N \times \Ns\), on note \(r \in \N\) le reste de la division euclidienne de \(a\) par \(b\).
	Montrons que \(\begin{cases}
		\mathcal{D}(a) \inter \mathcal{D}(b) & = \mathcal{D}(b) \inter \mathcal{D}(r) \\
		a\wedge b                            & = b\wedge r
	\end{cases}\) par double inclusion :
	\begin{itemize}
		\item Montrons que \(\mathcal{D}(a) \inter \mathcal{D}(b) \subset \mathcal{D}(b) \inter \mathcal{D}(r)\)\\~\\
		      Soit \(d \in \mathcal{D}(a) \inter \mathcal{D}(b)\) alors : \\~\\
		      \(d \divise b\) donc \(d \divise bq\) avec \(q \in \Zs\) et \(d \divise a\) donc \(d \divise a-bq\) or \(r = a-bq\) donc \(d \divise r\) et \(d \divise b\) donc \(d \in \mathcal{D}(b) \inter \mathcal{D}(r)\).
		\item De même Montrons que \(  \mathcal{D}(b) \inter \mathcal{D}(r) \subset \mathcal{D}(a) \inter \mathcal{D}(b)\)
		      Soit \(d \in \mathcal{D}(b) \inter \mathcal{D}(r)\) alors :\\~\\
		      alors \(d \divise b\) donc \(d \divise bq\) avec \(q \in \Zs\) et \(d \divise r\) donc \(d \divise bq+r\) or \(a = bq+r\) donc \(d \divise a\) et \(d \divise b\) donc \(d \in \mathcal{D}(b) \inter \mathcal{D}(a)\).

	\end{itemize}
	\conclusion Par double inclusion \(\mathcal{D}(a) \inter \mathcal{D}(b) = \mathcal{D}(b) \inter \mathcal{D}(r)\) et ainsi par définition du PGCD \(a\wedge b = b\wedge r\)
\end{dem}

\begin{defprop}[algorithme d'euclide]
	Soit \((a, b) \in \N \times \Ns\).\\~\\
	On pose \(r_0 = a\) et \(r_1 = b\) puis, pour tout \(i \in \Ns\) tel que \(r_i \neq 0\), on définit \(r_{i+1}\) comme suit :\\~\\
	\[r_{i+1}\text{ est le reste de la division euclidienne de } r_{i-1}\text{ par }r_i.\]
	Alors :
	\begin{itemize}
		\item il existe \(n \in N\) tel que
		      \[r_{n+1} = 0\text{ et }r_n \neq 0\]
		\item pour tout \(i \in \interventierii{1}{n} \),
		      \[r_{i-1} \wedge r_i = r_i \wedge r_{i+1}\]
	\end{itemize}
	En particulier \(r_0 \wedge r_1 = r_n \wedge r_{n+1}\) donc :
	\[a \wedge b = r_n\]
\end{defprop}

\begin{defprop}[Caractérisation du PGCD]
	Soit \((a, b) \in \N \times \Ns\).\\~\\
	L’ensemble des diviseurs communs à \(a\) et \(b\) est égal à l’ensemble des diviseurs de \(a \wedge b\) :
	\[\mathcal{D}(a) \inter \mathcal{D}(b) = \mathcal{D}(a \wedge b)\]
	Le PGCD de \(a\) et \(b\) est donc le plus grand élément (au sens de la divisibilité) de l’ensemble des diviseurs communs à \(a\) et \(b\), c’est-à-dire que :
	\begin{itemize}
		\item \(a \wedge b \divise a\) et \(a \wedge b \divise b\)
		      \item\( \forall d \in \N, d \divise a\) et \( d \divise b \imp d \divise a \wedge b\).
	\end{itemize}
\end{defprop}

\begin{defprop} [Propriété de factorisation du PGCD]
	Soit \((a, b) \in \N \times \Ns\).\\~\\
	Pour tout \(k \in \Ns\), le PGCD de \(ka\) et \(kb\) vérifie
	\[ka \wedge kb = k (a \wedge b)\]
\end{defprop}

\begin{dem}
	Soit \((a, b) \in \N \times \Ns\) et \(k \in \Ns\)
	\begin{itemize}
		\item \(\begin{cases}
			      a\wedge b \divise a \\
			      k \divise k
		      \end{cases}\)
		      et
		      \(\begin{cases}
			      a\wedge b \divise b \\
			      k \divise k
		      \end{cases}\)
		      donc par propriété
		      \(\begin{cases}
			      k\paren{a\wedge b} & \divise ka \\
			      k\paren{a\wedge b} & \divise kb
		      \end{cases}\) donc \(k\paren{a\wedge b} \divise ka\wedge kb\) car \(\mathcal{D}(ka)\inter \mathcal{D}(kb) = \mathcal{D}(ka\wedge kb)\)
		\item \(k \divise ka\) et \(k \divise kb\) donc \(k \divise ka \wedge kb\) donc \(\exists q \in \N\) tel que \(ka \wedge kb = kq\)\\~\\
		      Ainsi \(kq\divise ka\) et \(kq\divise kb\) et donc \(q\divise a\) et \(q \divise b\)\\~\\
		      ainsi \(q \divise a\wedge b\) puis \(kq \divise k\paren{a\wedge b}\) et donc enfin \(ka \wedge kb \divise k\paren{ a \wedge b}\)
	\end{itemize}
	Ainsi on a \(k\paren{a\wedge b} \divise ka\wedge kb\) et \(ka \wedge kb \divise k\paren{ a \wedge b}\) \\~\\
	donc \(k\paren{a\wedge b} \) et \(ka\wedge kb\) sont associés et donc égaux, car ce sont des entiers naturels non-nuls donc \(ka \wedge kb = k (a \wedge b)\)
\end{dem}

\subsection{Cas de deux entiers relatifs}

\begin{defi}
	Soit \((a, b) \in \Z^2\). \\~\\
	On appelle PGCD de \(a\) et \(b\) l’entier naturel noté \(a \wedge b\) défini par :
	\[a \wedge b = \begin{cases}
			\abs{a} \wedge \abs{b} & \text{ si } (a, b) \neq (0, 0) \\
			0                      & \text{ si } (a, b) = (0, 0)
		\end{cases}
	\]
\end{defi}

\begin{defprop}[Extension des résultats vus pour les entiers naturels]
	\begin{enumerate}
		\item Soit \((a, b) \in \Z^2 \pd \accol{(0, 0)}\) .
		      \begin{enumerate}
			      \item \(a \wedge b\) est le plus grand élément (au sens de \(\leq\)) de l’ensemble des diviseurs communs à \(a\) et \(b\).
			      \item \(a \wedge b\) est le plus grand élément (au sens de \(\divise\) ) de l’ensemble des diviseurs communs à \(a\) et \(b\).
		      \end{enumerate}
		\item Soit \((a, b) \in \Z^2\).
		      \begin{enumerate}
			      \item \(\mathcal{D}(a) \inter \mathcal{D}(b) = \mathcal{D}(a \wedge b)\).
			      \item Pour tout \(k \in \Z, ka \wedge kb = \abs{k} (a \wedge b) \)
		      \end{enumerate}
	\end{enumerate}
\end{defprop}

\begin{defprop}[Relation de Bézout]
	Soit \((a, b) \in \Z^2\).\\~\\
	Il existe un couple d’entiers \((u, v) \in \Z^2\), dit couple de Bézout, tel que \(au + bv = a \wedge b\).
	\underline{Remarque}
	\begin{itemize}
		\item  Un tel couple n’est PAS UNIQUE.
		\item  Pour \((a, b) \neq (0, 0)\), on peut déterminer un tel couple par l’algorithme d’Euclide étendu.\\~\\
		      on a : \((r_0, r_1) = (a, b), r_{i-1} = r_iq_i + r_{i+1}\) (division euclidienne de \(r_{i-1}\) par \(r_i\)) et \(n\) le plus petit entier tel que \(r_{n+1} = 0\). Ainsi, en posant
		      \[\begin{cases}
				      (u_0, v_0) & = (1, 0)
				      (u_1, v_1) = (0, 1)\end{cases}\text{ et, pour tout }i \in \interventierii{1}{N} , (u_{i+1}, v_{i+1}) = (u_{i-1} - q_i u_i, v_{i-1} - q_iv_i)\] .
		      on a : \(\forall i \in \interventierii{0}{n} , a u_i + b v_i = r_i\). En particulier, comme \(r_n\) est égal à \(a \wedge b\), on en déduit que :
		      \[a \wedge b = au_n + bv_n\text{ avec }(u_n, v_n) \in \Z^2\]
		\item Il n’est pas nécessaire de connaître les relations de récurrence définissant les familles \( (u_i)_{0\leq i \leq n}\) et \((v_i)_{0\leq i \leq n}\).
	\end{itemize}
\end{defprop}

\subsection{PPCM}
\begin{defi}
	Soit \((a, b) \in \Z^2\). Le PPCM de \(a\) et \(b\) est l’entier naturel noté \(a \vee b\) défini par
	\[a \vee b = \begin{cases}
			\min \paren{\abs{a} \Ns \inter \abs{b} \Ns} & \text{ si } a\neq 0\text{ et }b \neq 0 \\
			0                                           & \text{ si } a = 0 \text{ ou } b = 0
		\end{cases}
	\]
	\underline{Remarques}\\~\\
	\begin{itemize}
		\item Pour tout \(a \in \Z, a \vee a = a \vee 1 = \abs{1}\).
		\item Pour tout \((a, b) \in (\Zs)^2\), \(a \vee b\) est le plus petit entier naturel non nul, multiple commun de \(a\) et \(b\).
	\end{itemize}
\end{defi}

\begin{prop}
	Pour tout \((a, b) \in \Z^2\), on a :
	\[\abs{ab} = (a \wedge b) (a \vee b)\]
\end{prop}

\begin{dem}
	\(\forall (a,b) \in \Z^2\)\\~\\
	\begin{itemize}
		\item si \((a,b)\neq(0,0)\) alors prenons \(k \in \N\) \\~\\
		      alors :
		      \begin{align*}
			      k \in \abs{a}\Ns\inter\abs{b}\Ns & \iff \abs{a}\divise k \text{ ou } \abs{b}\divise k                                   \\
			                                       & \iff \abs{ab}\divise \paren{k\abs{a}\wedge k \abs{b}}                                \\
			                                       & \iff \abs{ab} \divise k \paren{\abs{a}\wedge \abs{b}}                                \\
			                                       & \iff \exists q\in \N,k \paren{\abs{a}\wedge \abs{b}} = q\abs{ab}                     \\
			                                       & \iff \exists q\in \N,k q\frac{\abs{ab}}{\paren{\abs{a}\wedge \abs{b}}}               \\
			                                       & \iff \frac{\abs{ab}}{\paren{\abs{a}\wedge \abs{b}}} \divise k                        \\
			                                       & \iff k \in \frac{\abs{ab}}{\paren{\abs{a}\wedge \abs{b}}} \Ns                        \\
			                                       & \iff \abs{a}\Ns\inter\abs{b}\Ns = \frac{\abs{ab}}{\paren{\abs{a}\wedge \abs{b}}} \Ns
		      \end{align*}
		      Donc \(a\vee b = \frac{\abs{ab}}{\paren{\abs{a}\wedge \abs{b}}}\).\\~\\
		      Ainsi\(\abs{ab} =\paren{a \vee b}\paren{\abs{a}\wedge \abs{b}} \imp \abs{ab} =\paren{a \vee b}\paren{a\wedge b}\)
		\item De plus si \((a,b) = (0,0)\) alors on a toujours \(\abs{ab} =\paren{a \vee b}\paren{a\wedge b}\) car \(\begin{cases}
			      \abs{ab}                          & = 0 \\
			      \paren{a \vee b}\paren{a\wedge b} & =0
		      \end{cases}\)
	\end{itemize}
\end{dem}

\section{Entiers premiers entre eux}
\subsection{Cas de couples d’entiers}
Soit \((a, b, c, n) \in \Z^4\)

\begin{defi}
	Les entiers \(a\) et \(b\) sont dits premiers entre eux si leur PGCD est égal à \(1\).
	\underline{Remarque}\\~\\
	Autrement dit, \(a\) et \(b\) sont premiers entre eux si leurs seuls diviseurs communs sont \(-1\) et \(1\).
\end{defi}

\begin{theo}[Théorème de Bézout]
	\(a\) et \(b\) sont premiers entre eux si, et seulement si, il existe \((u, v) \in \Z^2\) tel que \(au + bv = 1\)
\end{theo}

\begin{dem}[Théorème de Bézout]
	Montrons le théorème par double inclusion\\~\\
	\begin{itemize}
		\item \impdir immédiat par relation de Bézout
		\item \imprec On suppose \(\exists (u,v)\in \Z^2,au+bv = 1\)\\~\\
		      Soit \(d \in \N\) tq \(d \divise a\) et \(d \divise b\) alors \(d\divise au+bv\) donc \(d \divise 1\) donc \(d=1\) ainsi \(a\wedge b = 1\)

	\end{itemize}
\end{dem}

\begin{theo}[Lemme de Gauss]
	Si \(c\) divise \(ab\) et si \(a\) et \(c\) sont premiers entre eux alors \(c\) divise \(b\).\\~\\
	\underline{Remarque}\\~\\
	Tout nombre rationnel r non nul peut s’écrire sous la forme \(r = \frac{a}{b}\) avec \((a, b) \in \Zs \times \Ns \) et \(a \wedge b = 1\).\\~\\
	Cette écriture est unique et appelée forme irréductible de \(r\).
\end{theo}

\begin{prop}[Propriétés sur le produit]
	\begin{enumerate}
		\item Si \(a\) et \(b\) sont premiers entre eux et si \(a\) et \(b\) divisent \(n\) alors \(ab\) divise \(n\).
		\item Si \(a\) et \(n\) sont premiers entre eux et si \(b\) et \(n\) sont premiers entre eux alors \(ab\) et \(n\) sont premiers
		      entre eux.
	\end{enumerate}
\end{prop}

\begin{dem}
	Soit \((a,b,n) \in \Z^3\)\\~\\
	Démontrons les deux propriétés précédentes
	\begin{enumerate}
		\item par hypothèse \(\begin{cases}
			      \exists q \in \Z, n=aq \\
			      \exists q' \in \Z, n=bq'
		      \end{cases}\)
		      donc \(aq = bq'\) avec \(a\wedge b = 1\) donc \(b\divise q\)\\~\\
		      \(\exists q'' \in \Z\) tel que \(q = bq''\) ce qui donne \(n = abq''\) donc \(ab \divise n\)
		\item par hypothèse et d'après le théorème de bézout on a :
		      \[\exists (u,v) \in \Z^2 \text{ tel que } au+nv=1\]
		      \[\exists (u',v') \in \Z^2 \text{ tel que } bu'+nv'=1\]
		      donc par multiplication membre à membre on a  :
		      \[ab(u'u) + n(bvu' + nvv'+auv') = 1\]
		      donc d'après le théorème de bézout \(ab \wedge n =1\)
	\end{enumerate}
\end{dem}

\subsection{Cas de \(n\)-uplet d’entiers avec \( n \geq 2\) }
Soit \(n \in \N\) avec \(n \geq 2\) et \((a_1, \dots , a_n) \in \Z^n\).

\begin{defprop}[PGCD d’un nombre fini d’entiers]*
	On appelle PGCD des entiers \(\paren{a_1, \dots , a_n}\) l’entier naturel, noté \(a1 \wedge \dots \wedge a_n\), tel que
	\[\mathcal{D}\paren{a_1 \wedge \dots \wedge a_n} = \mathcal{D}(a_1) \inter \dots \inter \mathcal{D}(a_n)\]
\end{defprop}

\begin{defprop}[Relation de Bézout]
	Il existe un \(n\)-uplet d’entiers \((u_1, \dots , u_n) \in \Z^n\) tel que  \(a_1u_1 +\dots + a_nu_n = a_1 \wedge \dots \wedge a_n\)
\end{defprop}

\begin{defprop}[Entiers premiers entre eux]
	Les entiers \(a_1, \dots , a_n\) sont dits :
	\begin{itemize}
		\item premiers entre eux dans leur ensemble si \(a_1 \wedge \dots \wedge a_n = 1\).
		\item premiers entre eux deux à deux si \(\forall (i, j) \in \interventierii{1}{n} , i \neq j \imp a_i \wedge a_j = 1\).
	\end{itemize}
\end{defprop}
\begin{dem}[existence et unicité de la forme irréductible de tout rationnel non nul]
	Montrons l'existence et unicité de la forme irréductible de tout rationnel non nul autrement dit \(\forall r \in \Qs, \exists ! (a',b') \in \Zs \times \Ns, r = \frac{a'}{b'}\)
	\begin{itemize}
		\item \existence Soit \(r \in \Qs \) alors par définition \(\exists (a,b) \in \Zs \times \Ns, r = \frac{a'}{b'}\)\\~\\
		      on note \(d = a\wedge b\) alors on note \(\begin{cases}
			      a = da' & \text{ avec } a'\in \Zs  \\
			      b = db' & \text{ avec } b' \in \Zs
		      \end{cases}\)\\~\\
		      ce qui donne \(r = \frac{da'}{db'} = \frac{a'}{b'}\) avec \(a'\wedge b' = \frac{d\paren{a'\wedge b'}}{d} = \frac{da' \wedge db'}{d} = \frac{a \wedge b}{d} = 1\)
		\item \unicite Soit \(r \in \Qs\) tel que \(r = \frac{a'}{b'} = \frac{a''}{b''}\) avec \(\begin{cases}
			      a'\wedge b'    & = 1 \\
			      a'' \wedge b'' & = 1
		      \end{cases}\)\\~\\
		      on en déduit que \(a'b'' = a'' b'\) ce qui donne \(b''\divise a''b'\) puis \(b'' \divise b'\) car \(a'\wedge b' =1\) \\~\\
		      de même \(b'\divise b''\) donc \\~\\
		      \(b'\) et \(b''\) sont associés et entier naturel et donc égaux ce qui donne \(a' = a''\) et \(b' = b''\) ce qui prouve l'unicité
	\end{itemize}
\end{dem}
\section{Nombres premiers}
\subsection{Généralités}
\begin{defi}
	Un nombre entier naturel non nul \(p\) est dit premier s’il admet uniquement deux diviseurs entiers naturels distincts (qui sont \(1\) et \(p\))
\end{defi}

\begin{defprop}
	Ensemble des nombres premiers
	L’ensemble \(\mathcal{P}\) des nombres premiers est infini.
\end{defprop}
\begin{dem}
	Par l'absurde, supposons que \(\mathcal{P}\) est fini \cad \(\mathcal{P = \accol{p_1 \dots p_n}}\)\\~\\
	On pose \(N = \paren{\prod_{i = 1}^{n} p_i} + 1\)\\~\\
	alors \(\begin{cases}
		N \in \N \\
		N \geq 2
	\end{cases}\) donc \(N\) admet un diviseur premier \(i_0\) \\~\\
	\(\exists i_0 \in \interventierii{1}{n} , \begin{cases}
		p_{i_0} \divise N \\
		p_[i_0] \divise \prod_{i=1}^{n} p_i
	\end{cases}\) donc \(p_{i_0} \divise 1\)\\~\\
	d'où \(p_{i_0} = 1\) ce qui est faux car \(p_{i_0}\) est premier\\~\\
	\conclusion \(\mathcal{P}\) est infini
\end{dem}
\subsection{Décomposition en produit de nombres premiers}

\begin{theo}
	Tout entier naturel \(n\) supérieur ou égal à \(2\) peut s’écrire de manière unique (à l’ordre près des facteurs) sous la forme
	\[n = \prod_{i=1}^{k} p_i^{\alpha_i}\]
	où \(k \in \Ns\) avec \(\forall i \in \interventierii{1}{k} , \alpha_i \in \Ns\) et \(p_i\) est un nombre premier.
\end{theo}

\begin{defprop}[Corrolaire]
	Tout entier naturel non nul \(n\) s’écrit de manière unique (à l’ordre près des facteurs) sous la forme
	\[n = \prod_{p \in \mathcal{P}}p^{\alpha_p}\]
	où \(\paren{\alpha_p}_{p \in \mathcal{P}}\) est une famille presque nulle d’entiers naturels, c’est-à-dire une famille dans laquelle tous les éléments sont nuls sauf un nombre fini d’entre eux.
\end{defprop}

\subsection{Valuation \(p\)-adique}
\begin{defprop}
	Soit \(p\) un nombre premier et \(n\) un entier naturel non nul.\\~\\
	L’entier \(\alpha_p\) qui apparaît dans la décomposition primaire de \(n\)
	\[n = \prod_{p \in \mathcal{p}} p ^{\alpha_p}\]
	est appelé valuation \(p\)-adique de \(n\) et noté \(v_p(n)\) .\\~\\
	\underline{Autrement dit}\\~\\
	\(v_p(n)\) est le plus grand entier naturel \(k\) tel que \(p^k\) divise \(n\)
\end{defprop}

\begin{defprop}[Valuation \(p\)-adique d’un produit]
	Pour tout nombre premier \(p\) et tous entiers naturels non nuls \(n\) et \(n'\), on a :
	\[v_p(nn') = v_p(n) + v_p(n')\]
\end{defprop}

\begin{defprop}[Caractérisation de la divisibilité]
	Soit \((a, b) \in (\Ns)^2\).\\~\\
	\(b\) divise \(a\) si, et seulement si, pour tout nombre premier \(p\), on a : \(v_p(b) \leq v_p(a)\)
\end{defprop}

\begin{dem}
	Soit \((a,b) \in \N^2\), procédons par double équivalence
	\begin{itemize}
		\item \impdir on suppose \(b\divise a\)\\~\\
		      alors \(\exists q \in \N, a = bq\) donc \(\forall p \in \mathcal{P}, v_p(a) = v_p(b) + v_p(q) \imp v_p(a) \geq v_p(b)\)
		\item \imprec on suppose \(\forall p \in \mathcal{P}, v_p(a) \geq v_p(b)\)\\~\\
		      alors \(p^{v_p(a)} = p^{v_p(a) - v_p(b)} \times p^{v_p(b)}\) donc \(p^{v_p(b)} \divise p^{v_p(a)}\) ainsi\\~\\
		      \(\prod_{p \in \mathcal{P}}p^{v_p(b)} \divise \prod_{p \in \mathcal{P}}p^{v_p(a)}\) donc \(b \divise a\)
	\end{itemize}
\end{dem}

\begin{defprop}[PGCD et PPCM]
	Soit \((a, b) \in (\Ns)^2\).\\~\\
	Les PGCD et PPCM des entiers a et b vérifient :
	\[a\wedge b = \prod_{ p \in \mathcal{P}} p^{\min\paren{v_p(a),v_p(b)}}\]
	\[a\vee b = \prod_{p \in \mathcal{P}}p^{\max\paren{v_p(a),v_p(b)}}\]
\end{defprop}

\begin{dem}
	Soit \((a,b) \in \paren{\Ns}^2\)
	\begin{itemize}
		\item Montrons que \(a \wedge b = \prod_{ p \in \mathcal{P}} p^{\min\paren{v_p(a),v_p(b)}}\)\\~\\
		      on note \(d = \prod_{ p \in \mathcal{P}} p^{\min\paren{v_p(a),v_p(b)}}\)\\~\\
		      \(\forall p \in \mathcal{P},v_p(d) = \min\paren{v_p(a),v_p(b)}\) \\~\\
		      donc \(\forall p \in \mathcal{P}, \begin{cases}
			      v_p(d) & \leq v_p(a) \\
			      v_p(d) & \leq v_p(b)
		      \end{cases}\) d'où \(\begin{cases}
			      d & \divise a \\
			      d & \divise b
		      \end{cases}\)\\~\\
		      on note alors \(\begin{cases}
			      a = da' & \text{ avec } a'\in \Ns  \\
			      b = db' & \text{ avec } b' \in \Ns
		      \end{cases}\)
		      Montrons alors que \(a'\wedge b' = 1\) ce qui donnerais alors \(a \wedge b = d\paren{a' \wedge b'} = d\)\\~\\
		      Soit \(k\) un diviseur commun à \(a'\) et \(b'\) différent de \(1\) alors \(k \geq 2\) donc \(k\) admet un diviseur premier \(p'\)\\~\\
		      donc \(\begin{cases}
			      p' \divise a' \\
			      p' \divise b'
		      \end{cases}\) d'où \(\begin{cases}
			      v_{p'}(p') \leq v_{p'}(a) \\
			      v_{p'}(p') \leq v_{p'}(b)
		      \end{cases}\) \ie \(\begin{cases}
			      1\leq v_{p'}(a)  & \qquad (1) \\
			      1 \leq v_{p'}(b) & \qquad (2)
		      \end{cases}\)
		      Si \(v_{p'}(a) \leq v_{p'}(b)\) alors \(v_{p'}(d) = v_{p'}(a)\) et \(v_{p'}(a) = v_{p'}(d) + v_{p'}(a')\) d'où \(v_{p'}(a') = 0\) ce qui contredit \((1)\)\\~\\
		      Si \(v_{p'}(b) < v_{p'}(a)\) alors \(v_{p'}(d) = v_{p'}(b)\) donc \(v_{p'}(b') = 0\)\\~\\
		      Donc \(k=1\) d'où \(a'\wedge b' = 1\) d'où \(a \wedge b = d\)
		\item Montrons que \(a\vee b = \prod_{p \in \mathcal{P}}p^{\max\paren{v_p(a),v_p(b)}}\)
		      on a :\begin{align*}
			      a \vee b = \frac{ab}{a\wedge b} & = \frac{\paren{\prod_{p \in \mathcal{P}} p ^{v_p(a)}}\paren{\prod_{p \in \mathcal{P}} p ^{v_p(b)}}}{\prod_{ p \in \mathcal{P}} p^{\min\paren{v_p(a),v_p(b)}}} \\
			                                      & = \prod_{p \in \mathcal{P}}p^{\max\paren{v_p(a),v_p(b)}}*
		      \end{align*}
	\end{itemize}
\end{dem}

\subsection{Congruences}
Soit \((x, y, z, t) \in \Z^4\) et \(n \in \Ns\).

\begin{defi}
	\(x\) est dit congru à \(y\) modulo \(n\) s’il existe \(k \in \Z\) tel que \(x = y + nk\) autrement dit si \(x - y \in n\Z\).\\~\\
	\underline{Notation} : \( x \equiv y \croch{n}\)
\end{defi}

\subsection{Caractérisation}
\begin{defprop}
	\(x \equiv y \croch{n}\) si, et seulement si, les restes des divisions euclidiennes de \(x\) et \(y\) par \(n\) sont égaux.
\end{defprop}

\begin{dem}
	Soit \(\paren{x,y} \in \Z\) et \(n \in \Ns\) Montrons la caractérisation par double inclusion
	\begin{itemize}
		\item \impdir On suppose \(x \equiv y \croch{n}\) on écrit la division euclidienne de \(y\) par \(n\)\\~\\
		      \(\exists (q,r) \in \Z \times \interventierii{0}{n-1}, y =nq+r\)\\~\\
		      par hypothèse on a : \(\exists k \in \Z, x = y + nk\)\\~\\
		      donc \(x = n(k+q)+r\) est la division euclidienne de \(x\) par \(n\) donc \(r\) est le reste de la division euclidienne de \(x\) et \(y\) par \(n\)
		\item \imprec On suppose que \(x\) et \(y\) ont le même reste dans la division euclidienne de \(x\) et \(y\) par \(n\) \\~\\
		      Ainsi \(\exists(k,k') \in \Z^2, \exists r \in \interventierii{0}{n-1} \begin{cases}
			      x & = nk+r \\
			      y = nk' +r
		      \end{cases}\)\\~\\
		      donc \(x-y = n(k-k')\) \ie \(x-y \in n\Z\) \ie \(x \equiv y \croch{n}\)
	\end{itemize}
\end{dem}

\subsection{Propriétés}
\begin{defprop}
	\begin{enumerate}
		\item \(x \equiv x \croch{n}\) \hfill (réflexivité)
		\item si \(x \equiv y \croch{n}\) alors \(y \equiv x \croch{n}\) \hfill(symétrie)
		\item si \(x \equiv y \croch{n} \) et \(y \equiv z \croch{n}\) alors \(x \equiv z \croch{n}\) \hfill(transitivité)
	\end{enumerate}
\end{defprop}

\subsection{Opération}
\begin{defprop}
	\begin{enumerate}
		\item  Si \(x \equiv y \croch{n} \) et \(z \equiv t \croch{n} \) alors \(x+z \equiv y+t \croch{n} \)\hfill (compatibilité avec l’addition)
		\item  Si \(x \equiv y \croch{n} \) et \(z \equiv t \croch{n} \) alors \(xz \equiv yt \croch{n} \). \hfill(compatibilité avec la multiplication)
	\end{enumerate}
\end{defprop}

\subsection{Inverse modulo \(n\)}
\begin{defprop}
	\begin{itemize}
		\item Si \(x\) et \(n\) sont premiers entre eux, il existe un couple d’entiers \((u, v)\) tel que \(ux + vn = 1\).\\~\\
		      On en déduit que
		      \[ux \equiv 1 \croch{n}\]
		      et on dit que \(u\) est un inverse de \(x\) modulo \(n\).
		\item Si \(u\) est un inverse de \(x\) modulo \(n\) alors il existe un couple d’entiers \((u, v)\) tel que \(ux + vn = 1\).
		      On en déduit que \(x\) et \(n\) sont premiers entre eux.
	\end{itemize}
\end{defprop}

\subsection{Petit Théorème de Fermat}
\begin{theo}
	Si \(p\) est un nombre premier alors :
	\begin{enumerate}
		\item \(\forall a \in \Z, ap \equiv a \croch{p}\).
		\item \(\forall a \in \Z, a \wedge p = 1 \imp a^{p-1} \equiv 1 \croch{p}\).
	\end{enumerate}
\end{theo}


\chapter{Dérivation}

\minitoc

Dans ce chapitre, \(I\) et \(J\) sont des intervalles de \(\R\), non vides et non réduits à un point.

\section{Dérivation des fonctions à valeurs réelles}
\subsection{Dérivée en un point}
Soit \(f\) une fonction définie sur \(I\), à valeurs dans \(\R\), et \(a\) un point de \(I\).

\begin{defi}[Définition avec le taux d’accroissement]
    ~\\
	\(f\) est dite dérivable en \(a\) si la fonction \(x \mapsto \frac{f (x) - f (a)}{x - a}\) admet une limite réelle \(l\) en \(a\).\\~\\
	Dans ce cas,\\~\\
	la limite \(l\) obtenue est appelée dérivée de \(f\) en a et notée\( f'(a)\).
\end{defi}

\begin{defprop}[Caractérisation de la dérivabilité en un point par D.L d’ordre \(1\)]
	\(f\) est dérivable en \(a\) si, et seulement si, il existe \((b_0, b_1) \in \R^2\) et une application \(\epsilon : I \to \R\) tel que :
	\[\forall x \in I, f (x) = b_0 + b_1(x - a) + (x - a) \epsilon(x)\text{ et }\lim_{x\to a} \epsilon(x) = 0\]
	Dans ce cas, \(b_0 = f (a)\) et \(b_1 = f '(a) \) et on dit que \(f\) admet un développement limité à l’ordre \(1\) en \(a\).
\end{defprop}

\begin{dem}
    \begin{itemize}
        \item \impdir On suppose \(f\) dérivable en \(a\) alors \(\frac{f(x)-f(a)}{x-a} \underset{x \to a}{\to} f'(a)\)\\~\\
        On pose \(\epsilon(x) = \begin{cases}
            \frac{f(x) -f(a)}{x-a} - f'(a) &\text{ si } x\neq a \\
            0 &\text{ si } x=a
        \end{cases}\) \\~\\
        alors \(\epsilon(x) \underset{x \to a}{\to} 0 \) et \(\forall x \in I \pd \accol{a}, (x-a)\epsilon(x) = f(x) - f(a) - f'(x)(x-a)\) donc \(f(x) =f(a) + f'(x)(x-a) + (x-a)\epsilon(x)\)
        \item \imprec On suppose qu'il existe \((b_0,b_1) \in \R^2\) et \(\epsilon : I \to \R\) tel que:
        \begin{itemize}
            \item si \(x\neq a \) alors \(f (x) = b_0 + b_1(x - a) + (x - a) \epsilon(x)\text{ et }\lim_{x\to a} \epsilon(x) = 0\)
            \item si \(x = a\)  : \(f(a) = b_0\)
        \end{itemize}
        donc pour \(x \neq a , \frac{f(x) - f(a)}{x-a} = b_1 + \epsilon(x) \underset{x \to a}{\to} b_1\) ainsi \(f\) dérivable en \(a\) avec \(f'(a) = b_1\)   
    \end{itemize}
\end{dem}

\begin{defprop}[Condition nécessaire de dérivabilité en un point]
	Si \(f\) est dérivable en \(a\) alors \(f\) est continue en \(a\).\\~\\
	\underline{Remarque}\\~\\
	La réciproque est FAUSSE comme le prouve l’exemple classique de la fonction valeur absolue en \(0\).
\end{defprop}
\begin{dem}
    Si \(f\) dérivable alors \(\forall x \in I, f (x) = f(a) + f'(a)(x - a) + (x - a) \epsilon(x)\text{ et }\lim_{x\to a} \epsilon(x) = 0\) donc \(f(x) \underset{x \to a}{\to} f(a)\) d'où \(f\) continue en \(a\)
\end{dem}

\begin{defprop}[Interprétations géométrique et cinématique]
	\begin{itemize}
		\item  Si \(f\) admet une dérivée au point \(a\) alors la courbe représentative de \(f\) admet une tangente en  \(M (a, f (a))\) dont la pente est égale à \(f '(a)\).
		\item Si \(f (t)\) est l’abscisse à l’instant \(t \geq 0\) d’un mobile se déplaçant sur une droite et si \(f\) admet une dérivée au point \(a \geq 0\) alors \(f '(a)\) est la vitesse instantanée de ce mobile à l’instant \(a\).
	\end{itemize}
\end{defprop}

\subsection{Dérivabilité à droite et à gauche}
Soit \(f\) une fonction définie sur \(I\), à valeurs dans \(\R\), et \(a\) un point de \(I\).
\begin{defi}
	\begin{itemize}
		\item On suppose ici que \(a\) n’est pas l’extrémité gauche de \(I\).\\~\\
		      \(f\) est dite dérivable à gauche en \(a\) si \(x \mapsto \frac{1}{ x - a} (f (x) - f (a))\) admet une limite à gauche en \(a\).\\~\\
		      La limite obtenue (unique si elle existe) est appelée dérivée à gauche de \(f\) en \(a\) et notée \(f '_g(a)\).
		\item On suppose ici que \(a\) n’est pas l’extrémité droite de \(I\).\\~\\
		      \(f\) est dite dérivable à droite en \(a\) si \(x \mapsto \frac{1}{x - a} (f (x) - f (a))\) admet une limite à droite en \(a\).\\~\\
		      La limite obtenue (unique si elle existe) est appelée dérivée à droite de \(f\) en \(a\) et notée \(f '_d(a)\).
	\end{itemize}
\end{defi}

\begin{prop}
	On suppose ici que \(a\) n’est pas extrémité de \(I\).\\~\\
	\(f\) est dérivable en \(a\) si, et seulement si, \(f\) est dérivable à gauche et à droite en \(a\) avec \(f '_g(a) = f '_d(a)\).\\~\\
	Dans ce cas, \(f '(a) = f '_g(a) = f '_d(a)\).
\end{prop}

\subsection{Condition nécessaire d’extremum local en un point intérieur}
\begin{defprop}
	Soit \(f\) une fonction définie sur \(I\), à valeurs dans \(R\).\\~\\
	Si \(f\) admet un extremum local en un point \(a\) de \(I\) qui n’est pas une extrémité de \(I\), et si \(f\) est dérivable en \(a\) alors \(f '(a) = 0\)
	\underline{Remarques}
	\begin{itemize}
		\item Les points \(a\) de \(I\) en lesquels \(f\) est dérivable avec \(f '(a) = 0\) sont dits points critiques de \(f\).
		\item La détermination des points critiques indique où des extremums sont susceptibles d’exister. Une étude complémentaire du signe de \(f (x) - f (a)\) au voisinage du point critique \(a\) est nécessaire pour conclure s’il y extremum local ou non en ce point \(a\).
		\item Il peut y avoir des extremums locaux pour \(f\) en un point extrémité \(a\) de l’intervalle \(I\) en lequel \(f\) est dérivable sans que \(f '(a)\) ne soit égal à \(0\).
	\end{itemize}
\end{defprop}

\begin{dem}
    On suppose, sans perte de généralité, que \(f\) admet un maximum local en \(a\), point de \(I\) qui n’est pas extrémité de \(I\), et que \(f\) est dérivable en \(a\). Le cas du minimum local s’en déduit en remplaçant \(f\) par \(-f\).\\~\\
    Alors, par définition d’un maximum local, il existe un réel \(\delta\) strictement positif tel que \\~\\
    \[\forall x \in \intervee{a-\delta}{a+\delta},f(x)\leq f(a)\]
    Ainsi, 
    \[\forall x \in \intervee{a-\delta}{a},\frac{f(x)-f(a)}{x-a}\geq 0\]
    et
    \[\forall x \in \intervee{a}{a+\delta},\frac{f(x)-f(a)}{x-a}\leq 0\]
    Comme \(f\) est dérivable en \(a\) et que \(a\) n'est pas extrémité de \(I\), \(f\) est dérivable à droite et à gauche en \(a\) avec
    \[f'_g(a) = f'_d(a) = f'(a)\]
    Par passage à la limite dans les inégalités précédentes, on a d’abord \(f '_g(a) \geq 0\) et \(f '_d(a) \geq 0\) puis
    \[0 \leq f '(a) \leq 0\]
    ce qui donne par antisymétrie que \(f '(a) = 0\).\\~\\
    \underline{Bilan} : Si \(f\) a un extremum local en un point \(a\) de \(\mathring{I}\) et si \(f\) est dérivable en \(a\) alors \(f '(a) = 0\).\\~\\
    \underline{Remarque} : \(\mathring{I}\) désigne l’intérieur de \(I\) ,c’est-à-dire ici, l’ensemble des points de \(I\) qui sont centres d’un intervalle ouvert inclus dans \(I\).
\end{dem}

\subsection{Dérivée sur un intervalle}
Soit \(f\) une fonction définie sur \(I\), à valeurs dans \(\R\).
\begin{defi}
	\(f\) est dite dérivable sur \(I\) si \(f\) est dérivable en tout point \(a\) de \(I\).\\~\\
	Dans ce cas,\\~\\
	la fonction qui, à tout \(a\) de \(I\) fait correspondre \(f '(a)\) est appelée application dérivée de \(f\) et notée \(f '\).\\~\\
	\underline{Notation}
	Dans la suite, on note \(\mathcal{D} (I, \R)\) l’ensemble des fonctions définies et dérivables sur \(I\), à valeurs réelles.
\end{defi}

\begin{defprop}[Opérations sur les fonctions dérivables]
	Les opérations sur les limites vues dans le chapitre “Limite et continuité” permettent de montrer que \(\mathcal{D} (I, \R)\) est stable par combinaison linéaire, produit et quotient (sous réserve que cela ait du sens).\\~\\
	Plus précisément :
	\begin{itemize}
		\item une combinaison linéaire de fonctions dérivables sur \(I\) à valeurs réelles est dérivable sur \(I\) :
		      \[\forall(f, g) \in \paren{\mathcal{D}(I,\R)}^2 , \forall(\lambda, \mu) \in \R ^2, \lambda f + \mu g \in \mathcal{D}(I,\R)\text{ et }(\lambda f + \mu g)' = \lambda f ' + \mu g'\]
		\item un produit de fonctions dérivables sur \(I\) à valeurs réelles est dérivable sur \(I\) :
		      \[\forall(f, g) \in \paren{\mathcal{D}(I,\R)}^2 , f g \in \mathcal{D}(I,\R)\text{ et }(f g)' = f 'g + f g'\]
		\item un quotient de fonctions dérivables sur \(I\) à valeurs réelles dont le dénominateur ne s’annule pas sur \(I\) est dérivable sur \(I\) :
		      \[\forall(f, g) \in \paren{\mathcal{D}(I,\R)}^2 , \forall x \in I, g(x)\neq 0 \imp \frac{f}{g}\in \mathcal{D}(I,\R)\text{ et} \paren{\frac{f}{g}}' = \frac{f 'g - f g'}{g^2}\]
	\end{itemize}
\end{defprop}

\begin{dem}[Preuve dérivée \(\paren{\frac{f}{g}}' = \frac{f'g - fg'}{g^2}\)]
    Soit \(\paren{f,g} \in\paren{\mathcal{D}(I,\R)}^2 \) avec \(\forall x \in I,g(x) \neq 0\)
    Soit  \(a \in I\), \\~\\
    Pour \(x \in I \pd \accol{a}\)
    \begin{align*}
        \frac{\frac{f}{g}(x) - \frac{f}{g}(a)}{x-a} &= \frac{\frac{f(x)g(a) - f(a)g(x)}{g(a)g(x)}}{x-a} \\
        &= \frac{f(x)g(a) - f(a)g(x)}{x-a}\times \frac{1}{g(a)g(x)} \\
        &= \frac{1}{g(a)g(x)} \times \paren{\frac{\paren{f(x) - f(a)}g(a)}{x-a} - \frac{f(a)\paren{g(x)-g(a)}}{x-a}}\\
        &= \frac{1}{g(a)g(x)} \times \paren{\frac{\paren{f(x) - f(a)}}{x-a}g(a) - f(a)\frac{\paren{g(x)-g(a)}}{x-a}}
    \end{align*}
    Par passge à la limite vers \(a\) et par définition de la dérivée on retrouve donc 
    \[\frac{\frac{f}{g}(x) - \frac{f}{g}(a)}{x-a} \underset{x \to a}{\to} \frac{1}{g^2(a)} \times \paren{f'(a)g(a) - f(a)g'(a)} = \frac{ f'(a)g(a) - f(a)g'(a)}{g^2(a)}\]
\end{dem}

\begin{defprop}[Composition de fonctions dérivables]
    Soit \(f\) une fonction définie sur \(I\) et à valeurs réelles tel que, pour tout \(x\) de \(I\),\( f (x)\) appartient à \(J\).\\~\\
    Soit \(g\) une fonction définie sur \(J\) et à valeurs réelles.\\~\\
    Si \(f\) est dérivable sur \(I\) et si \(g\) est dérivable sur \(J\) alors \(g \circ f\) est dérivable sur \(I\) avec 
    \[\forall x \in I, (g \circ f )'(x) = g' (f (x)) \times f '(x)\]
\end{defprop}

\begin{dem}
    Soit \(a \in I\). On note \(\Delta\) la fonction définie sur \(I\) par :
    \[\Delta(t) = \begin{cases}
        \frac{g(t) - g(f(a))}{t-f(a)} &\text{ si } t\neq f(a)\\
        g'(f(a)) &\text{ si } t=f(a) 
    \end{cases}\]
    Par dérivabilité de \(g\) en \(f (a)\), on a \(\lim _{t \to f(a)} \Delta(t) = g'(f(a))\) \cad \(\lim_{t\to f(a)} \Delta(t) = \delta(f(a))\) donc \(\Delta\) est continue en \(f(a)\). Comme \(f\) est continue en \(a\) puisqu'elle y est dérivable, on en déduit alors par composition que \(\lim_{x\to a} \Delta (f(x)) = \Delta (f(a))\) ce qui donne : 
    \[\lim_{x \to a} \Delta (f(x)) = g'(f(a))\] ce qui donne : 
    \[\lim_{x\to a } \Delta (f(x)) = g'(f(a))\]
    Par dérivabilité de \(f\) en \(a\), on a :
    \[\lim_{x \to a}\frac{f(x)-f(a)}{x-a} = f'(a)\]
    Comme pour tout \(x \in I \pd \accol{a}\),on peut écrire
    \[\frac{g(f(a))-g(f(a))}{x-a} = \Delta(f(x)) \times \frac{f(x)-f(a)}{x-a}\].
    On conclut par produit que la fonction \(x \mapsto \frac{g(f(x)) - g(f(a))}{x-a}\) admet une limite finie en \(a\) qui vaut \(g'(f(a)) \times f'(a)\) autrement dit que \(g \circ f\) est dérivable en \(a\) de dérivée \(g'(f(a)) \times f'(a)\).\\~\\
    \underline{Conclusion} : \(g \circ f\) est dérivable sur \(I\) de dérivée \((g' \circ f)\times f'\).
\end{dem}

\begin{defprop}[Réciproque d’une fonction dérivable]
    Soit \(f\) une fonction définie sur \(I\) et à valeurs réelles.\\~\\
    Si \(f\) est une bijection de \(I\) sur \(J = f (I)\), dérivable sur \(I\) et que sa dérivée ne s’annule pas sur \(I\) alors \(f ^{-1}\) est dérivable sur \(J\) et vérifie
    \[\forall y \in J, \paren{f ^{-1}}'(y) = \frac{1}{f ' \paren{f ^{1}(y)}}\]
\end{defprop}

\begin{dem}
    On suppose les hypothèses réunies. Alors \(f\) est continue sur \(I\) (car elle y est dérivable), à valeurs réelles et injective (car elle est bijective de \(I\) sur \(J\)). D’après une propriété du chapitre “Limite et continuité”,\(f\) est donc strictement monotone sur \(I\).\\~\\
    Par théorème de la bijection continue, on en déduit en particulier que \(f^{-1}\) est continue sur \(J\).\\~\\
    Soit \(b \in J\).\\~\\
    Pour tout \(y \in J\pd \accol{b}\), on peut écrire :
    \[\frac{f^{-1}(y)-f^{-1}(b)}{y-b} = \frac{f^{-1}(y) - f^{-1}(b)}{f(f^{-1}(y))-f(f^{-1}(b))} = \paren{\frac{f(f^{-1}(y))-f(f^{-1}(b))}{f^{-1}(y) - f^{-1}(b)}}^{-1}\]
    car \(y \neq b \) et \(f^{-1}\) injective donnent \(f^{-1}(y) \neq f^{-1}(b)\)\\~\\
    Par continuité de \(f^{-1}\) en \(b\), on a \(f^{-1}(y) \underset{y \to b}{\to} f^{-1}(b)\) et, par dérivabilité de \(f\) en \(a = f^{-1}(b)\), on a : 
    \(\frac{f(x)-f(a)}{x-a} \underset{x \to a}{\to} f'(a) = f'(f^{-1}(b))\). Une composition de limites donne donc :
    \[\frac{f(f^{-1}(y))-f(f^{-1}(b))}{f^{-1}(y) - f^{-1}(b)} \underset{y \to b}{\to} f'(f^{-1}(b))\]
    Comme \(f'(f^{-1}(b))\neq 0 \) par hypothèse sur \(f'\), par limite d'une fonction inverse, on obtient : 
    \[\paren{\frac{f(f^{-1}(y))-f(f^{-1}(b))}{f^{-1}(y) - f^{-1}(b)}}^{-1} \underset{y \to b}{\to} \paren{f'(f^{-1}(b))}^{-1}\]
    Ainsi,
    \[\frac{f^{-1}(y)-f^{-1}(b)}{y-b} \underset{y \to b}{\to} \frac{1}{f'(f^{-1}(b))} \paren{ \in \R}\]
    \underline{Conclusion} : \(f^{-1}\) est dérivable en tout \(b\) de \(J\), donc sur \(J\) ,avec 
    \[\forall b \in J,\paren{f^{-1}}'(b) = \frac{1}{f'f^{-1}(b)}\]
\end{dem}

\section{Théorèmes de Rolle et des accroissements finis}
\subsection{Théorème de Rolle}
\begin{theo}[Théorème de Rolle]
    Soit \(a\) et \(b\) deux réels tels que \(a < b\).\\~\\
    Soit \(f\) une fonction définie sur \(\intervii{a}{b}\) à valeurs réelles.\\~\\
    Si \(f\) est continue sur le segment \(\intervii{a}{b}\), dérivable sur l’intervalle ouvert \(\intervee{a}{b}\) et vérifie \(f (a) = f (b)\) alors il existe un réel \(c\) dans l’intervalle ouvert \(\intervee{a}{b}\) tel que \(f '(c) = 0\).
\end{theo}

\begin{dem}
    On suppose les hypothèse réunies\\~\\
    \(f\) étant continue sur le segment \(\intervii{a}{b}\) et à valeurs réelles, par théorème, \(f\) est bornée et atteint ses bornes.\\~\\
    On note \(m = \min f\) et \(M = \max f\) , et \((x_1, x_2) \in \intervii{a}{b}^2\) tel que \(m = f (x_1)\) et \(M = f (x_2)\).\\~\\
    On raisonne par disjonction de cas.\\~\\
\begin{itemize}
    \item Si \(m = M\) alors \(f\) est une fonction constante donc sa dérivée est la fonction nulle ; le résultat attendu est alors immédiat.
    \item Si \(m < M\) alors l’un des réels \(m\) ou \(M\) est différent de \(f (a)\). Dans la suite, on suppose, sans perte de généralité, que \(m\neq f (a)\). 
\end{itemize}
    Alors \(f (x_1)\neq f (a)\) et \(f (x_1)\neq f (b)\) (puisque \(f (a) = f (b)\)) donc \(x_1\neq a\) et \(x_1\neq b\).\\~\\
    On en déduit que \(x_1\) appartient à l’intervalle ouvert \(\intervee{a}{b}\) . Comme de plus \(f\) est dérivable en \(x_1\) et \(y\) admet un minimum global (donc un extremum local), on conclut par condition nécessaire d’extremum local en un point intérieur que \(f '(x_1) = 0\).\\~\\
    \underline{Conclusion} : Il existe un réel \(c\) dans l’intervalle ouvert \(\intervee{a}{b}\) tel que \(f '(c) = 0\).
\end{dem}


\begin{defprop}[Interprétations géométrique et cinématique]    
    Si les hypothèses du théorème de Rolle sont réunies alors :
    \begin{itemize}
        \item il existe un point en lequel la courbe représentative de \(f\) admet une tangente horizontale ;
        \item il existe un instant \(c\) en lequel la vitesse instantanée d’un mobile dont l’abscisse à l’instant \(t \geq 0\) sur une droite est donnée par \(f (t)\), est nulle.
    \end{itemize}
\end{defprop}


\subsection{Accroissements finis}
\begin{defprop}[Egalité des accroissements finis]
    Soit \(a\) et \(b\) deux réels tels que \(a < b\).\\~\\
    Soit \(f\) une fonction définie sur \(\intervii{a}{b}\) à valeurs réelles.\\~\\
    Si \(f\) est continue sur \(\intervii{a}{b}\), dérivable sur \(\intervee{a}{b}\) alors il existe \(c\) dans \(\intervee{a}{b}\) tel que :\( f (b)-f (a) = (b-a)f '(c)\).
\end{defprop}
\begin{dem}
    On suppose les hypothèses réunies et on définit
    \[g : x \mapsto f (x) -\frac{ f (b) - f (a)}{b - a }(x - a)\]
    \(g\) est définie et continue sur le segment \(\intervii{a}{b}\), dérivable sur \(\intervee{a}{b}\), à valeurs réelles avec \(g(a) = g(b)\).\\~\\
    Par théorème de Rolle, il existe donc un réel \(c\) dans \(\intervee{a}{b}\) tel que \(g'(c) = 0\) avec \(g' : x \mapsto f '(x)- \frac{f (b) - f (a)}{b - a}\)
    ce qui donne :
    \[f '(c) =\frac{ f (b) - f (a)}{b - a}\]
    et donc
    \[f (b) - f (a) = (b - a)f '(c)\]
\end{dem}

\begin{defprop}[Inégalité des accroissements finis]
    Soit \(f\) une fonction définie sur \(I\) à valeurs réelles.\\~\\
    Si \(f\) est dérivable sur \(I\) et si \(\abs{f '}\) est majorée par un réel \(k\) alors \(f\) est \(k\)-lipschitzienne, c’est-à-dire que :
    \[\forall (x, y) \in I^2, \abs{f (x) - f (y)} \leq k \abs{x - y} \]
\end{defprop}

\begin{dem}
    On suppose les hypothèses réunies et \((x, y) \in I^2\). Le cas \(x = y\) étant immédiat, on suppose sans perte de généralité que \(x < y\).\\~\\
    Alors \(f\) est continue sur \(\intervii{x}{y}\), dérivable sur \(\intervee{x}{y}\) donc, par égalité des accroissements finis, il existe un réel \(c\) dans \(\intervee{x}{y}\) tel que :
    \[f (y) - f (x) = f '(c) (y - x)\] .
    Ainsi \(\abs{f (y) - f (x)} = \abs{f '(c)} \abs{y - x}\) puis, par hypothèse sur \(\abs{f '|}\), on en déduit :
    \[\abs{f (y) - f (x)} \leq k \abs{y - x} \].
    Ceci étant vrai pour tout \((x, y) \in I^2\), \(f\) est donc \(k\)-lispchitzienne
\end{dem}

\subsection{Applications des théorèmes des accroissements finis}
Soit \(f\) une fonction définie et dérivable sur \(I\) à valeurs réelles.

\begin{defprop}[Caractérisation des applications constantes]
    \(f\) est constante si, et seulement si, pour tout \(x\) de \(I\), \(f '(x) = 0\)
\end{defprop}
\begin{dem}
    \begin{itemize}
        \item \impdir Si \(f\) est constante alors sa dérivée est nulle.
        \item \imprec Si la dérivée de \(f\) est nulle alors l’inégalité des accroissements finis donne, pour tout \((x, y) \in I^2\), \(\abs{f (x) - f (y)} \leq 0 \times \abs{ x - y}\) donc \(\abs{f (x) - f (y)} = 0\) puis \(f (x) = f (y)\) ce qui implique que f est constante.
    \end{itemize}
\end{dem}
\begin{defprop}[Caractérisation des fonctions dérivables monotones]
    \begin{enumerate}
        \item \(f\) est croissante sur \(I\) si, et seulement si, pour tout \(x\) de \(I\), \(f '(x) \geq 0\).
        \item \(f\) est décroissante sur \(I\) si, et seulement si, pour tout \(x\) de \(I\), \(f '(x) \leq 0\).
    \end{enumerate}
\end{defprop}

\begin{dem}[Caractérisation des fonctions croissantes]
    \begin{itemize}
        
        \item \impdir Si \(f\) est croissante sur \(I\) alors, pour tout \((x, y) \in I^2\) tel que \(x < y\), on a \(\frac{f (y) - f (x) }{y - x } \geq 0\).\\~\\
        Comme \(f\) est dérivable en \(x\), \(f\) est dérivable à droite en \(x\) donc, par passage à la limite dans l’inégalité précédente, on trouve \(\lim_{ y\to x^+} \frac{f (y) - f (x)}{ y - x} \geq 0\) \cad \(f ' _d(x) \geq 0\) et enfin \(f '(x) \geq 0\).
        \item \imprec on suppose que \(f '\) est positive. Soit \((x, y) \in I^2\) tel que \(x < y\). Par égalité des accroissements finis, il existe \(c \in \intervee{x}{y} \) tel que \(\frac{f (y) - f (x)}{y - x }= f'(c)\) donc, par hypothèse de positivité, on trouve \(\frac{f (y) - f (x)}{y - x} \geq 0\) ce qui prouve que \(f\) croissante
    \end{itemize}
\end{dem}

\begin{defprop}[Caractérisation des fonctions dérivables strictement monotones]
    \begin{enumerate}
        \item  \(f\) est strictement croissante sur \(I\) si, et seulement si, les conditions suivantes sont réunies :
        \begin{enumerate}
            \item pour tout \(x\) de \(I\), \(f '(x) \geq 0\).
            \item il n’existe pas de réels \(a\) et \(b\) dans \(I\) avec \(a < b\) tel que, pour tout \(x\) de \(\intervii{a}{b}\) , \(f '(x) = 0\).
        \end{enumerate}

        \item \(f\) est strictement décroissante sur \(I\) si, et seulement si, les conditions suivantes sont réunies :
        \begin{enumerate}
            \item pour tout \(x\) de \(I\), \(f '(x) \leq 0\).
            \item il n’existe pas de réels \(a\) et \(b\) dans \(I\) avec \(a < b\) tel que, pour tout \(x\) de \(\intervii{a}{b}\) , \(f '(x) = 0\).
        \end{enumerate}
    \end{enumerate}
\end{defprop}

\begin{dem}[Caractérisation des fonctions strictement croissantes]
    \begin{itemize}
        
        \item \impdir Si \(f\) est strictement croissante sur \(I\) alors \(f\) est croissante sur \(I\) donc \(f '\) est positive. Par ailleurs, si on suppose l’existence de réels \(a\) et \(b\) dans \(I\) avec \(a < b\) tels que \(f '_{|\intervii{a}{b}} = 0\) alors \(f '_{ |\intervii{a}{b}}\) est constante ce qui contredit la stricte croissance de \(f\) sur \(I\). Ainsi, il n’existe pas de réels \(a\) et \(b\) dans \(I\) avec \(a < b\) tel que, pour tout \(x\) de \(\intervii{a}{b}\) , \(f '(x) = 0\).
        \item \imprec  on suppose que, pour tout \(x\) de \(I\), f\( '(x) \geq 0\) et que de plus, il n’existe pas de segment inclus dans \(I\) sur lequel la restriction de \(f '\) est nulle. Alors \(f\) est croissante sur \(I\) ainsi il n’existe pas de segment inclus dans \(I\) tel que \(f_{ |\intervii{a}{b}}\) est une fonction constante. Si \(f\) n’est pas strictement croissante, il existe un couple \((a, b) \in I^2\) avec \( a < b\) et \(f (a) \geq f (b)\). Par croissance de \(f\) sur \(I\), on en déduit :\( \forall x \in \intervii{a}{b} , f (a) \leq f (x) \leq f (b)\) donc \(\forall x \in \intervii{a}{b} , f (a) \leq f (x) \leq f (a)\) puis \(\forall x \in \intervii{a}{b} , f (x) = f (a)\). Ainsi \(f_{ |\intervii{a}{b}}\) est constante ce qui contredit ce qui précède. On conclut donc que : \(f\) est strictement croissante sur \(I\) .
    \end{itemize}
\end{dem}

\begin{theo}[Théorème de la limite de la dérivée]
    Soit \(a\) un point de \(I\).\\~\\
    Si \(f\) est continue sur \(I\), dérivable sur \(I \pd \accol{a}\) et si \(f' _{|I\pd\accol{a}}\) admet une limite réelle \(l\) en \(a\) alors
    \[\lim _{x\to a} \frac{f (x) - f (a)}{ x - a }= l\]
    Dans ce cas :
    \begin{enumerate}
        \item \(f\) est dérivable en \(a\) avec \(f '(a) = l\) ;
        \item \(f '\) est continue en \(a\).
    \end{enumerate}
    \underline{Remarque}\\~\\
    La fonction \(f\) peut être dérivable en \(a\) sans que \(f _{|I\pd\accol{a}}\) ait une limite réelle en \(a\) (par exemple, pour la fonction \(f\) définie par \(f (0) = 0\) et \(f (x) = x^2 \sin \paren{\frac{1}{x}}\) si \(x\neq 0\)).
\end{theo}

\begin{dem}
    On suppose les hypothèses réunies et on considère un réel strictement positif \(\epsilon\). \\~\\
    Puisque \(f '_{|I\pd\accol{a}}\) a pour limite le réel \(l\), il existe un réel strictement positif \(\delta\) tel que :
    \[\forall x \in I \pd \accol{a} , \abs{x - a} \leq \delta \imp \abs{f '(x) - l} \leq \epsilon\]
    Prenons alors \(x \in I \pd \accol{a}\) tel que \(\abs{x - a} \leq \delta\).\\~\\
    \(f\) étant continue sur \(I\) et dérivable sur \(I \pd \accol{a}\), \(f\) est continue sur le segment \(\intervii{a}{x}\) ou \(\intervii{x}{a}\) (suivant que \(a < x\) ou \(a > x\)), dérivable sur l’intervalle ouvert \(\intervee{a}{x}\) ou \(\intervee{x}{a}\).\\~\\
    D’après l’égalité des accroissements finis, il existe donc un réel \(c_x\) dans \(\intervee{a}{x}\) ou \(\intervee{x}{a}\) tel que
    \[\frac{f(x)-f(a)}{x-a} = f'(c_x)\]
    Comme \(c_x\) appartient à l’intervalle ouvert d’extrémités \(a\) et \(x\), on a :
    \[c_x \in I \pd \accol{a}\text{ et } \abs{c_x - a} \leq \abs{ x - a} \leq \delta\]
    D’après ce qui a été dit précédement, on en déduit \(\abs{f '(c_x) - l} \leq \epsilon\) , c’est-à-dire :
    \[\abs{\frac{f(x) - f(a)}{x-a} - l}\leq \epsilon\]
    En résumé : 
    \[\forall x \in \Rps, \exists \delta \in \Rps, \forall x \in I \pd \accol{a}, \abs{x-a}\leq \delta \imp \abs{\frac{f(x) - f(a)}{x-a} - l}\leq \epsilon\]
    Autrement dit : \(x \mapsto \frac{f (x) - f (a)}{ x - a}\) a pour limite le réel \(l\) en \(a\) donc \(f\) est dérivable en \(a\) avec \(f'(a) = l\).\\~\\
    De plus, \(f '(a) = \lim _{x\to a} f '(x)\) donc \(f '\) est continue en \(a\).
\end{dem}

\begin{defprop}[Extension du théorème de la limite de la dérivée]
    Soit \(a\) un point de \(I\).\\~\\
    Si \(f\) est continue sur \(I\), dérivable sur \(I \pd \accol{a}\) et si \(f'_{|I\pd\accol{a}}\) admet une limite infinie \(l\) en \(a\) alors
    \[\lim _{x\to a} \frac{f (x) - f (a)}{ x - a} = l\]
\end{defprop}

\section{Classe \(\mathcal{C}^k\)}
Soit \(f\) une fonction définie sur \(I\) à valeurs réelles.
\subsection{Notations}
\begin{nota}
    On pose \(f ^{(0)} = f\) et, pour \(k \in \N\), sous réserve que cela ait du sens, \(f ^{(k+1)} = \paren{f^{(k)}}' \).
\end{nota}

\subsection{Définitions}
\begin{defi}
    Soit \(k \in \N\).
    \begin{itemize}
        \item \(f\) est dite \(k\) fois dérivable sur \(I\) si \(f ^{(k)}\) existe.
        \item \(f\) est dite de classe \(\mathcal{C}^k\) sur \(I\) si \(f\) est \(k\) fois dérivable sur \(I\) avec \(f^{(k)}\) continue sur \(I\).
        \item \(f\) est dite de classe \(\mathcal{C}^{\infty}\) sur \(I\) si, pour tout \(k \in \N\), \(f\) est de classe \(\mathcal{C}^k\) sur \(I\).
    \end{itemize}
    \underline{Remarque}\\~\\
    Soit \(k \in \N \union \accol{\pinf}\) .\\~\\
    L’ensemble des applications de classe \(\mathcal{C}^k\) sur \(I\) à valeurs dans \(R\) est souvent noté \(\mathcal{C}^k\paren{I,\R}\).
\end{defi}

\subsection{Opérations sur les fonctions de classe \(\mathcal{C}^k\) avec \(k \in \N \union \accol{\pinf}\) .}
\begin{defprop}
    \(\mathcal{C}^k\paren{I,\R}\) est stable par combinaison linéaire, produit et quotient (sous réserve que cela ait du sens).
    Plus précisément :
    \begin{itemize}
        \item une combinaison linéaire de fonctions de classe \(\mathcal{C}^k\) sur \(I\) à valeurs réelles est de classe \(\mathcal{C}^k\) sur \(I\) :
            \[\forall(f, g) \in \paren{\mathcal{C}^k(I,\R)}^2 , \forall(\lambda, \mu ) \in \R^2, \lambda f + \mu g \in \mathcal{C}^k(I,\R)\text{ et }\paren{\lambda f + \mu g}^{(k)} = \lambda f ^{(k)} + \mu g^{(k)}\]
        \item un produit de fonctions de classe Ck sur I à valeurs réelles est de classe Ck sur I :
            \[\forall(f, g) \in \paren{\mathcal{C}^k(I,\R)}^2 , f g \in \mathcal{C}^k(I,\R)\text{ et }\underbrace{(f g)^{(k)} = \sum_{i=0}^{k}\binom{k}{i}f^{(i)}g^{(k-i)} = \sum_{i=0}^k\binom{k}{i}f^{(k-i)}g^{(i)}}_{\text{formule de Leibniz}}\]
        \item un quotient de fonctions de classe \(\mathcal{C}^k\) sur \(I\) à valeurs réelles dont le dénominateur ne s’annule pas sur \(I\) est de classe \(\mathcal{C}^k\) sur \(I\).
    \end{itemize}
\end{defprop}

\subsection{Composition de fonctions de classe \(\mathcal{C}^k\) avec \(k \in \N \union \accol{\pinf}\) .}
\begin{defprop}
    Soit \(f\) une fonction définie sur \(I\) et à valeurs réelles tel que, pour tout \(x\) de \(I\), \(f (x)\) appartient à \(J\).\\~\\
    Soit \(g\) une fonction définie sur \(J\) et à valeurs réelles.\\~\\
    Si \(f\) est de classe  \(\mathcal{C}^k\) sur \(I\) et si \(g\) est de classe  \(\mathcal{C}^k\) sur \(J\) alors \(g \circ f\) est de classe \(\mathcal{C}^k\) sur \(I\).
\end{defprop}

\subsection{Réciproque d’une fonction de classe \(\mathcal{C}^k\) avec \(k \in \N \union \accol{\pinf}\)}
\begin{defprop}
    Soit \(f\) une fonction définie sur \(I\) et à valeurs réelles.\\~\\
    Si \(f\) est une bijection de \(I\) sur \(J = f (I)\), de classe \(\mathcal{C}^k\) sur \(I\) et que sa dérivée ne s’annule pas sur \(I\) alors \(f ^{-1}\) est de classe \(\mathcal{C}^k\) sur \(J\).
\end{defprop}

\section{Cas des fonctions à valeurs complexes}
\subsection{Ce qui s’étend aux fonctions complexes}
\begin{defprop}
    \begin{itemize}
        \item Dérivée en un point et sur un intervalle : définition et caractérisations, lien avec la continuité, dérivées à gauche et à droite, opérations
        \item Classe \(\mathcal{C}^k\) : définition, opérations
        \item Inégalité des accroissements finis
    \end{itemize}
\end{defprop}

\subsection{Ce qui ne s’étend pas aux fonctions complexes}

\begin{defprop}
    \begin{itemize}
        \item — Résultats utilisant la relation d’ordre :
        \begin{itemize}
            \item la notion d’extremum local (et donc la condition nécessaire d’existence d’un extremum local) 
            \item le théorème de Rolle 
            \item l’égalité des accroissements finis 
            \item les caractérisations des fonctions constantes ou monotones parmi les fonctions dérivables.
        \end{itemize}
        \item Composition de fonctions dérivables
        \item Réciproque d’une fonction dérivable
    \end{itemize}
\end{defprop}

\subsection{Quelques résultats qui s’étendent détaillés}
Soit \(f\) une fonction définie sur \(I\), à valeurs complexes.
\begin{defi}
    \begin{itemize}
        \item \(f\) est dite dérivable en \(a \in I\) si la fonction à valeurs complexes \(x \mapsto \frac{f (x) - f (a)}{x- a}\) admet une limite complexe \(l\) en a appelée nombre dérivé de \(f\) en \(a\) et notée \(l = f '(a)\).
        \item \(f\) est dite dérivable sur \(I\) si \(f\) est dérivable en tout point de \(I\).
    \end{itemize}
\end{defi}

\begin{defprop}[Caractérisation]
    \begin{itemize}
        \item \(f\) est dérivable en \(a \in I\) si, et seulement si, \(\Reel{ (f )}\) et \(\Ima{(f )}\) le sont.\\~\\
        Dans ce cas,
            \[\paren{\Reel{f }}' (a) = \paren{\Reel{ f '(a)}}\text{ et }\paren{\Ima{f }}' (a) = \paren{\Ima{ f '(a)}} \]
        — \(f\) est dérivable (respectivement de classe \(\mathcal{C}^k\)) sur \(I\) si, et seulement si, \(\Reel{ (f )}\) et \(\Ima{(f )}\) le sont.
    \end{itemize}
\end{defprop}

\begin{defprop}[Inégalité des accroissements finis]
    Si \(f\) est de classe \(\mathcal{C}^1\) sur \(I\) et si \(\abs{f'}\) est majorée par un réel \(k\) alors \(f\) est \(k\)-lipschitzienne, c’est-à-dire :
    \[\forall (x, y) \in I^2, \abs{f (x) - f (y)} \leq k \abs{x - y} \]
\end{defprop}

\begin{dem}
    \underline{Rappel} On rappelle que le théorème de Rolle et l’égalité des accroissements finis ne se généralisent pas au cas des
    fonctions à valeurs complexes (non réelles).\\~\\
    Par exemple, la fonction \(g : t \mapsto e^{2i\pi t}\) est continue sur le segment \(\intervii{0}{1}\), dérivable sur \(\intervee{0}{1}\) avec \(g(0) = g(1)\) mais sa dérivée \(g' : t \mapsto 2i\pi e^{2i\pi t}\) ne s’annule pas sur \(\intervee{0}{1}\).\\~\\
    La preuve de l’inégalité des accroissements finis dans le cas des fonctions à valeurs complexes ne peut donc se faire comme dans le cas réel. On peut tout de même démontrer cette inégalité, pour les fonctions à valeurs complexes de classe C1 sur un intervalle, en admettant des propriétés de l’intégrale d’une fonction continue sur un segment que l’on verra dans le chapitre “Intégration sur un segment”.\\~\\~\\~\\
    On suppose que \(f\) est de classe  \(\mathcal{C}^1\) sur \(I\) et que \(\abs{f '}\) est majorée par un réel \(k\).\\~\\
    Soit \((x, y) \in I^2\) avec \(x \leq y\).\\~\\
    Comme \(f '\) est continue sur le segment \(\intervii{x}{y}\) et \(f\) une primitive de \(f '\) sur \(I\), on peut écrire
    \[\abs{f(x) - f(y)} = \abs{\int_x^y f'(t)dt}\]
    puis, par propriété du module de l’intégrale,
    \[\abs{f (x) - f (y)} \leq \int^y_x \abs{f '(t)} dt\]
    Par hypothèse sur \(\abs{f '}\) et croissance de l’intégrale, on a alors
    \[\abs{f (x) - f (y)} \leq \int^y_x k dt  \ie  \abs{f (x) - f (y)} \leq k (y - x)\] 
    Ainsi :
    \[\forall (x, y) \in I^2, x \leq y \imp \abs{f (x) - f (y)} \leq k \abs{y - x}\]
    puis
    \[\forall (x, y) \in I^2, \abs{f (x) - f (y)} \leq k \abs{y - x}\] 
    Autrement dit, \(f\) est lipschitzienne de rapport \(k\).
\end{dem}


\chapter{Structure algébriques usuelles}

\minitoc

\section{généralité}
Soit \(E\) un ensemble.
\subsection{Loi de composition interne}
\begin{defprop}
    On appelle loi de composition interne sur \(E\) toute application \(f\) de \(E \times E\) dans \(E\).\\~\\
    A tout couple \((x, y)\) de \(E \times E\), est ainsi associée une unique image \(f (x, y) \in E\) souvent notée \(x \star y\) ou \(x\top y\) et appelée composé de \(x\) et \(y\) pour la loi de composition interne \(\star\) ou \(\top\).\\~\\
    \underline{Remarque}\\~\\
    Un ensemble muni d’une loi de composition interne est dit magma.
\end{defprop}

\subsection{Définitions - Propriétés}

\begin{defprop}[Associativité, commutativité]
    Une loi de composition interne \(\star\) sur \(E\) est dite :\\~\\
    \begin{enumerate}
        \item associative si \(\forall (x, y, z) \in E^3, (x \star y) \star z = x \star (y \star z)\)
        \item commutative si \(\forall (x, y) \in E^2, x \star y = y \star x\).
    \end{enumerate}
\end{defprop}

\begin{defprop}[Elément neutre]
    On dit qu’une loi de composition interne \(\star\) sur \(E\) admet un élément neutre s’il existe \(e\) dans \(E\) tel que
    \[\forall x \in E, x \star e = e \star x = x\]
    \underline{Remarques}
    \begin{enumerate}
        \item Si \(\star\) admet un élément neutre sur \(E\) alors celui-ci est UNIQUE.\\
        \item Un ensemble muni d’une loi de composition interne associative et qui admet un élément neutre est dit monoïde.
    \end{enumerate}
\end{defprop}

\begin{dem}[Unicité de l'élement neutre]
    Supposons qu'il existe deux élément neutre \(e\) et \(e'\) dans \(E\) pour la l.c.i \(\star\)\\~\\
    Alors \(\forall x \in E\begin{cases}
        x \star e & = x \\
        x \star e' &= x 
    \end{cases}\)\\~\\
    en particulier en prenant \(x = e'\) dans le premier cas et \(x = e\) dans le deuxième cas, on obtient :
    \[e = e \star e' = e' \star e = e'\]
    donc on a bien \(e = e'\), Ainsi \(\star\) admet un unique élément neutre dans \(E\).
\end{dem}

\begin{defprop}[Inversibilité]
    Soit \(\star\) une loi de composition interne sur \(E\) qui admet un élément neutre \(e\).\\~\\
    Un élément \(x\) de \(E\) est dit inversible s’il existe \(x'\) dans \(E\) tel que
    \[x \star x' = x' \star x = e\]
    \underline{Remarques}
    \begin{itemize}
        \item Si de plus la loi est associative alors :
        \begin{itemize}
            \item l’élément \(x'\) est UNIQUE et dit inverse de \(x\) ;
            \item si \(x\) et \(y\) sont des éléments de \(E\) inversibles alors l’élément \(x \star y\) est inversible d’inverse \(y' \star x'\).
        \end{itemize}
        \item Les termes "symétrisable" et "symétrique" sont parfois utilisés à la place de "inversible" et inverse".
    \end{itemize}
\end{defprop}

\begin{dem}[Unicité de l'élément inversible si \(\star\) est associative]
    ~\\~\\
    Soit \((x,x',x'') \in E^3\) tel que \(\begin{cases}
        x \star x' = x' \star x &= e \quad (1)\\
        x' \star x'' = x'' \star x' &= e \quad(2)
    \end{cases}\)\\~\\
    alors \(x''\star \paren{ x \star x'} = x'' \star e \) donc par associativité  \(\paren{x'' \star x} \star x' = x''\) donc d'après \((2)\) on a \(e \star x' = x''\) et donc \(x' = x''\)\\~\\
    \underline{Conclusion} Si la l.c.i est associative sur \(E\) et admet un neutre alors tout élement inversible admet un unique inverse
\end{dem}

\begin{defprop}[Distributivité]
    Soit \(\star\) et \(\top\) deux lois de composition interne sur\( E\).\\~\\
    On dit que la loi \(\top\) est distributive par rapport à la loi \(\star\) si :
    \[\forall (x, y, z) \in E^3, \begin{cases}
    x\top (y \star z) &= (x\top y) \star (x\top z)\\
    (y \star z) \top x &= (y\top x) \star (z\top x)
    \end{cases} \]
\end{defprop}

\begin{defprop}[Quelques exemples usuels de lois de composition interne ]
    \begin{itemize}
        \item Loi \(+\)
        \begin{itemize}
            \item sur les ensembles de nombres \(\Z, \Q, \R, \C\) ;
            \item sur les ensembles de fonctions : \(\mathcal{F}(X, \K), \mathcal{D} (I, \K)\) et \(\mathcal{C}^n (I, \K)\) ;
            \item sur les ensembles de matrices : \(\mathcal{M}_{n,p} (\K), \mathcal{S}_n (\K)\) et \(\mathcal{A}_n (\K)\).
        \end{itemize}
        \item Loi \(\times\)
        \begin{itemize}
            \item sur les ensembles de nombres \(\Q, \Qp, \R, \Rp, \C, \U\) et \(\U_n\) ;
            \item sur les ensembles de fonctions : \(\mathcal{F}(X, \K), \mathcal{D} (I, \K)\) et \(\mathcal{C}^n (I, \K)\) ;
            \item sur les ensembles de matrices : \(\mathcal{M}_{n,p} (\K)\) et \(\mathcal{GL}_n (\K)\).
        \end{itemize}
        \item Autres lois 
        \begin{itemize}
            \item sur l’ensemble \(\mathcal{P}(E)\) des parties d’un ensemble \(E\) :\( \cup, \cap\) et \(\pd\) ;
            \item sur l’ensemble des applications de \(E\) dans \(E\) avec \(E\) un ensemble : \(\circ\).
        \end{itemize}
    \end{itemize}
\end{defprop}

\subsection{Partie stable}
\begin{defprop}
    Soit \(\star\) une loi de composition interne sur \(E\).\\~\\
    On dit qu’une partie \(A\) de \(E\) est stable pour la loi \(\star\) si \(\forall (x, y) \in A^2, x \star y \in A\).
\end{defprop}

\section{Groupes, sous-groupes}
\subsection{Groupes}
\begin{defi}
    Un groupe est un ensemble \(G\) muni d’une loi de composition interne \(\star\) telle que :\\~\\
    \begin{itemize}
        \item \(\star\) est associative.
        \item \(G\) admet un élément neutre \(e_G\) pour la loi \(\star\).
        \item Tout élément \(x\) de \(G\) admet un inverse \(x'\) pour la loi \(\star\).
    \end{itemize}
    \underline{Remarques}\\
    \begin{enumerate}
        \item Il y a unicité de l’élément neutre de \(G\) et de l’inverse de tout élément de \(G\).
        \item Dans tout groupe, il y a au moins un élément : le neutre pour la loi du groupe.
        \item Si \(\star\) est commutative, on dit que \(G\) est un groupe commutatif (ou groupe abélien).
    \end{enumerate}
\end{defi}

\begin{nota}[Notations dans un groupe additif et un groupe multiplicatif]
    \begin{itemize}
        \item Lorsque la loi du groupe \(G\) est notée \(+\) alors on parle de groupe additif et on écrit :
        \begin{enumerate}
            \item \(0_G\) au lieu de \(e_G\) ;
            \item \(-x\) au lieu de\( x'\) ;
            \item \(
                nx = \begin{cases}
                    x+ \dots + x \quad(n \text{ fois }) &\text{ si } n \in \Ns \\
                    0_G &\text{ si } n = 0\\
                    -(-nx) &\text{ si } n \in \Zms
                \end{cases}\)
        \end{enumerate}
        \item Lorsque la loi du groupe \(G\) est notée \(\times\) alors on parle de groupe multiplicatif et on écrit : 
        \begin{enumerate}
            \item \(1_G\) au lieu de \(e_G\) ;
            \item \(x^{-1}\) au lieu de\( x'\) ;
            \item \(
                x^n = \begin{cases}
                    x\times \dots \times x \quad(n \text{ fois }) &\text{ si } n \in \Ns \\
                    1_G &\text{ si } n = 0\\
                    \paren{x^{-n}}^{-1} &\text{ si } n \in \Zms
                \end{cases}\)
        \end{enumerate}
    \end{itemize}
\end{nota}

\begin{defprop}[Quelques exemples usuels de groupes déjà rencontrés cette année]
    \begin{itemize}
        \item Groupes additifs
        \begin{itemize}
            \item dans les ensembles de nombres : \(\Z, \Q, \R, \C\) ;
            \item dans les ensembles de fonctions : \(\mathcal{F}(X, \K), \mathcal{D} (I, \K)\) et \(\mathcal{C}^n (I, \K)\) ;
            \item dans les ensembles de matrices : \(mathcal{M}_{n,p} (\K), \mathcal{S}_n (\K)\) et \(\mathcal{A}_n (\K)\).
        \end{itemize}
        \item Groupes multiplicatifs
        \begin{itemize}
            \item dans les ensembles de nombres : \(\Q, \Qp, \R, \Rp, \C, \U\) et \(\U_n\);
            \item dans les ensembles de matrices : \(\mathcal{GL}_n (\K)\).
        \end{itemize}
    \end{itemize}
\end{defprop}

\begin{defprop}[Groupe des permutations d’un ensemble]
    Soit \(X\) un ensemble.\\~\\
    L’ensemble des applications de \(X\) dans \(X\) qui sont des bijections est un groupe pour la loi de composition interne \(\circ\), appelé groupe des permutations de l’ensemble \(X\) et noté \(S_X\).
\end{defprop}
\begin{defprop}[Produit fini de groupes]
    Soit \((G_1, \perp)\) et \((G_2, \top)\) deux groupes.\\~\\
    Le produit cartésien \(G_1 \times G_2\) muni de la loi \(\star\) définie par :
    \[\forall(x_1, x_2) \in G_1 \times G_2, \forall(y_1, y_2) \in G_1 \times G_2, (x_1, x_2) \star (y_1, y_2) = (x_1 \perp y_1, x_2\top y_2)\]
    est un groupe, dit groupe-produit.\\~\\
    Dans ce groupe-produit,
    \begin{itemize}
        \item l’élément neutre est \((e_{G_1},e_{G_2} )\) où \(e_{G_1}\) est le neutre de \(G_1\) et \(e_{G_2}\) est le neutre de \(G_2\) ;
        \item l’inverse de \((x_1, x_2)\) de \(G_1 \times G_2\) est \((x'_1, x'_2)\) où \(x'_1\) est l’inverse de \(x_1\) et \(x'_2\) l’inverse de \(x_2\).
    \end{itemize}
    \underline{Remarques}
    \begin{itemize}
        \item on en déduit, par exemple, que :
        \begin{itemize}
            \item \(\K^2\) est un groupe additif de neutre \((0, 0)\) dans lequel \((-x, -y)\) est l’inverse de \((x, y)\) ;
            \item \(\paren{\Ks}^2\) est un groupe multiplicatif de neutre \((1, 1)\) dans lequel \((x^{-1}, y^{-1})\) est l’inverse de \((x, y)\).
        \end{itemize}
        \item La propriété s’étend à un nombre fini \(m \quad(\geq 2)\) de groupes \((G_1, \underset{1}{\perp} ),(G_2, \underset{2}{\perp} ), \dots , (G_m, \underset{m}{\perp} )\).
        \item Ainsi, par exemple, \(\K_m\) est un groupe additif et \(\paren{\Ks}^m\) est un groupe multiplicatif.
    \end{itemize}
\end{defprop}

\subsection{Sous-groupes}
\begin{defi}
    Soit \((G, \star)\) un groupe.\\~\\
    Une partie \(H\) de \(G\) est dite sous-groupe de \(G\) si les deux conditions suivantes sont réunies :
    \begin{itemize}
        \item \(H\) est une partie stable pour la loi \(\star\) ;
        \item \(H\) est un groupe pour la loi de composition interne obtenue par restriction à \(H\) de la loi de composition interne \(\star\) de \(G\).
    \end{itemize}
\end{defi}

\begin{defprop}[Caractérisation]
    Une partie \(H\) d’un groupe \((G, \star)\) est un sous-groupe de \(G\) si, et seulement si, les conditions suivantes sont réunies :
    \begin{enumerate}
        \item \(H \neq \emptyset \)\hfill \((e_G \in H)\)
        \item \(\forall (x, y) \in H^2, x \star y \in  H \) \hfill (stabilité par composition)
        \item \(\forall x \in H, x' \in H \text{ avec } x' \text{ l’inverse de } x \text{ dans } (G, \star)\) \hfill (stabilité par passage à l’inverse)
    \end{enumerate}
    \underline{Caractérisation alternative}\\
    Une partie \(H\) d’un groupe \((G, \star)\) est un sous-groupe de \(G\) si, et seulement si :
    \[H \neq \emptyset \text{ et } \forall (x, y) \in H^2, x \star y' \in H\]
\end{defprop}

\section{Morphisme de groupes}
\subsection{Morphisme}
\begin{defi}
    Une application \(f : G \to G'\) est dite morphisme de groupes si \(G\) et \(G'\) sont des groupes de lois respectives \(\star\) et \(\perp\) avec
    \[\forall (x, y) \in G \times G, f (x \star y) = f (x) \perp f (y)\]
\end{defi}

\begin{prop}
    Si \(f : G \to G'\) est un morphisme de groupes alors :\\~\\
    \begin{itemize}
        \item l’image de l’élément neutre de \(G\) par \(f\) est l’élément neutre de \(G'\), c’est-à-dire :
            \[f (e_G) = e_{G'} \]
        \item pour tout \(x \in G\), l’inverse de l’image de \(x\) par \(f\) est l’image de l’inverse de \(x\) par \(f\), c’est-à-dire :
            \[(f (x))' = f (x')\]
    \end{itemize}
\end{prop}

\begin{dem}
    Soit \(f : G \to G'\)
    \begin{itemize}
        \item Montrons que  \(f(e_G) = e_{G'} \)\\
        On a \(f(e_G) = f(e_G \star e_G) = f(e_G) \perp f(e_G)\)\\ 
        ainsi par composition par l'inverse de \(f(e_G)\), on trouve \(\paren{f(e_G)}' \perp f(e_G) = \paren{\paren{f(e_G)}' \perp f(e_G)} \perp f(e_G)\)\\ 
        donc \(e_{G'} = e_{G'} \perp f(e_G) = f(e_G)\) par associativité de \(\perp\) puis par définition de \(e_{G'}\).
        \item Montrons que l'inverse d'une image st l'image de l'inverse \\
        Soit \(x \in G\) alors : 
        \[f(x) \perp f(x') = f(x \star x') = f(e_G) = e_{G'}\]
        \[f(x') \perp f(x) = f(x' \star x) = f(e_G) = e_{G'}\]
        donc \(\paren{f(x)}' = f(x')\)
    \end{itemize}
\end{dem}

\begin{defprop}[Image directe et réciproque]
    Si \(f : G \to G'\) est un morphisme de groupes alors,
    \begin{enumerate}
        \item l’image directe de tout sous-groupe \(H\) de \(G\), est un sous-groupe de \(G'\).
        \item l’image réciproque de tout sous-groupe \(H'\) de \(G'\) est un sous-groupe de \(G\). 
    \end{enumerate}
\end{defprop}
\begin{dem}
    \begin{itemize}
        \item Montrons que si \(H\) est un sous groupe de \(G\) alors \(f(H)\) sous groupe \(G'\)
        \begin{itemize}
            \item \(f(H) \subset G'\) par définition de \(f\)
            \item \(f(H) \neq \emptyset\) car \(e_{G'} \in f(H)\) puisque \(e_{G'} = f(e_G)\) avec \(e_G \in H\)
            \item Soit \(\paren{x,y} \in \paren{f(H)}^2\) ,Montrons que \(x \perp y' \in f(H)\) 
            Par hypothèse il existe \(a\) et \(b\) dans \(H\) tel que \(\begin{cases}
                x &= f(a)\\
                y &= f(b)
            \end{cases}\)\\
            alors \begin{align*}
                x \perp y &= f(a) \perp \paren{f(b)}' \\
                &= f(a) \perp f(b')\\
                &= f(a \star b') \hfill \text{ avec } a \star b' \in H \text{ car } H \text{ sous groupe } 
            \end{align*}
            Ainsi \(x \perp y \in f(H)\) 
        \end{itemize}
        donc par caractérisation, \(f(H)\) sous groupe de \(G'\)
        \item Montrons que si \(H'\) est un sous groupe de \(G'\) alors \(f^{-1}(H)\) sous groupe \(G'\)
        \begin{itemize}
            \item \(f^{-1}(H') \subset G\) par définition de \(f^{-1}\)
            \item \(f^{-1}(H') \neq \emptyset\) car \(e_{G} \in f^{-1}(H')\) puisque \(e_{G'} = f(e_G)\) avec \(f^{-1}(e_{G'}) \in H\)
            \item Soit \(\paren{x,y} \in \paren{f^{-1}(H')}^2\) ,Montrons que \(x \star y' \in f^{-1}(H')\) 
            Par hypothèse \(\begin{cases}
                f(x) &\in H' \\
                f(y)& \in H'
            \end{cases}\)\\
            alors \begin{align*}
                f(x \perp y') &= f(x) \perp \paren{y'} \\
                &= f(x) \perp f(b)'\\
            \end{align*}
            d'où \(f(x \star y') \in H' \) car \(H'\) sous-groupe ainsi \(x \star y' \in f^{-1}(H')\)
        \end{itemize}
        Donc par caractérisation, \(f^{-1}(H')\) est un sous groupe de \(G\)
    \end{itemize}
\end{dem}
\begin{defprop}[Noyau et image d’un morphisme de groupes]
    Si \(f : G \to G'\) est un morphisme de groupes alors,
    \begin{enumerate}
        \item L’image directe \(f (G)\) est un sous-groupe particulier de \(G'\), dit image de \(f\) et noté \(\im f\) .
    \[\im f  \underset{\text{ déf}}{=} f(G) \underset{\text{déf}}{=} \accol{y \in G' \tq \exists x \in G, y = f (x)} \]
        \item L’image réciproque \(f ^{-1} (\accol{e_{G'}})\) est un sous-groupe particulier de \(G\), dit noyau de \(f\) et noté \(\ker f\)
    \[\ker f   \underset{\text{ déf}}{=} f ^{-1} (\accol{e_{G' }}) \underset{\text{ déf}}{=} \accol{x \in G \tq f (x) = e_{G'}} \]
    \end{enumerate}
\end{defprop}

\begin{defprop}[Caractérisation des morphisme injectif]
    Un morphisme de groupes \(f : G \to G'\) est injectif si, et seulement si, \(\ker f = \accol{e_G}\)
\end{defprop}

\begin{dem}
    Soit \(f : \paren{G,\star} \to \paren{G', \perp}\) un morphisme de groupe\\
    Montrons que \(f\) injectif \( \iff \ker f = \accol{e_G}\) par double implication
    \begin{itemize}
        \item \imprec On suppose \(\ker f = \accol{e_G}\)\\
        Soit \(\paren{x,y} \in G\) tel que \(f(x) = f(y)\)\\
        alors par composition par \(\paren{f(x)}'\) on a :
        \begin{align*}
            \paren{f(x)}' \perp f(x) = \paren{f(x)}' \perp f(y) &\iff e_{G'} = f(x') \perp f(y)\\
            &\iff e_{G'} = f(x' \star y)
        \end{align*}
        ainsi, \(x' \star y \in \ker f\), donc \(x' \star y = e_G\) par hypothèse puis par composition à gauche on trouve \(y = x\)\\~\\
        \underline{Conclusion} \(f\) est injective
        \item \impdir On suppose \(f\) injective\\
        si \(\ker f \neq \accol{e_G}\) alors il existe \(x \in \ker f\) tel que \(x \neq e_G\) avec \(f(x) = e_{G'}\).\\
        Donc \(f(x) = f(e_G)\) d'où \(x = e_G\) car \(f\) est injective ce qui est absurde, car on a supposé \(x \neq e_G\)\\~\\
        \underline{Conclusion} \(\ker f = \accol{e_G}\)
    \end{itemize}
\end{dem}

\subsection{Isomorphisme}
\begin{defi}
    \(f\) est dit isomorphisme de groupes si \(f\) est un morphisme de groupes et \(f\) est bijective.
\end{defi}

\begin{prop}
    Si \(f : G \to G'\) est un isomorphisme de groupes alors \(f^{-1} : G' \to G\) est un isomorphisme de groupes.
\end{prop}

\section{Anneaux, corps}

\subsection{Anneaux}
\begin{defi}
    Un anneau est un ensemble \(A\) muni de deux lois de composition interne \(\star\) et \(\perp\) telles que :
    \begin{enumerate}
        \item \((A, \star)\) est un groupe commutatif ;
        \item \(\perp\) est associative ;
        \item \(\perp\) est distributive par rapport à la loi \(\star\) ;
        \item \(A\) admet un élément neutre pour la loi \(\perp\).
    \end{enumerate}
    \underline{Remarqes}\\
    \begin{itemize}
        \item Il y a unicité de l’élément neutre pour la loi \(\perp\).
        \item Si \(\perp\) est commutative, on dit que \(A\) est un anneau commutatif.
        \item Si les lois de \(A\) sont notées \(+ \) et \(\times\), les éléments neutres de \(A\) pour les lois \(+\) et \(\times\) sont alors souvent notés respectivement \(0_A\) et \(1_A\) (ou \(0\) et \(1\) s’il n’y a pas de confusion possible) et \(1_A\) est appelé élément unité de l’anneau.
    \end{itemize}
\end{defi}

\begin{defprop}[Quelques exemples usuels d’anneaux déjà rencontrés cette année]
    \begin{itemize}
        \item Dans les ensembles de nombres : \(\Z, \Q, \R, \C\) ;
        \item Dans les ensembles de fonctions : \(\mathcal{F}(X, \K), \mathcal{D} (I, \K)\) et \(\mathcal{C}^n (I, \K)\) ;
        \item Dans les ensembles de matrices : \(\mathcal{M}_n (\K)\)
    \end{itemize}
\end{defprop}

\begin{defprop}[Calculs dans un anneau]
    Soit \((A, +, \times)\) un anneau.
    \begin{enumerate}
        \item \( \forall x \in A, 0_A \times x = x \times 0_A = 0_A \text{ et }\forall x \in A, (-1_A) \times x = x \times (-1_A) = -x\)
        \item \(\forall (x, y) \in A^2, (-x) \times y = x \times (-y) = -(x \times y) \text{ et } (-x) \times (-y) = x \times y\)
        \item \(\forall (x, y, z) \in A^3, (x - y) \times z = (x \times z) - (y \times z) \text{ et } z \times (x - y) = (z \times x) - (z \times y)\)
        \item \(\forall n \in \Ns, \forall (x, y) \in A^2, x \times y = y \times x \imp (x + y)^n = \sum_{k=0}^n \binom{k}{n}x^k \times y^{n-k} \)
        \item \(\forall n \in \Ns, \forall (x, y) \in A^2, x \times y = y \times x \imp x^n - y^n = (x - y) \times \sum^{n-1}_{k=0} x^k \times y^{n-1-k}\)
    \end{enumerate}
\end{defprop}

\begin{defprop}[Groupe des inversibles d’un anneau]
    Si \((A, +, \times)\) est un anneau alors l’ensemble\\~\\
    \[G = \accol{x \in A \tq x \text{ admet un symétrique pour la loi }\times \text{ dans }A}\]
    muni de la loi \(\times\) est un groupe, dit groupe des inversibles de l’anneau \((A, +, \times)\)
\end{defprop}

\subsection{Sous-anneaux}
\begin{defprop}
    Une partie \(H\) d’un anneau \((A, +, \times)\) est un sous-anneau de \(A\) si, et seulement si,
    \begin{enumerate}
        \item \(1_A \in H\)
        \item \(\forall(x, y) \in H2, x + y \in H\)
        \item \(\forall x \in H, -x \in H\)
        \item \(\forall(x, y) \in H2, x × y \in H\)
    \end{enumerate}
    \underline{Remarque}
    On peut remplacer les conditions \((2)\) et \((3)\) par \(\forall(x, y) \in H2, x - y \in H\).
\end{defprop}

\subsection{Morphisme d’anneaux}
\begin{defi}
    Une application \(f : A \to B\) est dite morphisme d’anneaux si \(A\) et \(B\) sont des anneaux de lois respectives \((+, \times)\) et \((\star, \perp)\) avec :
    \begin{enumerate}
        \item \(f (1_A) = 1_B\) 
        \item \(\forall (x, y) \in A^2, f (x + y) = f (x) \star f (y)\)
        \item \(\forall (x, y) \in A^2, f (x \times y) = f (x)\perp f(y)\)
    \end{enumerate}
\end{defi}

\begin{prop}
    Si \(f : A \to B\) est un morphisme d’anneaux alors \(f\) est un morphisme de groupes.
\end{prop}

\begin{defprop}[Image et noyau d’un morphisme d’anneaux]
    \begin{itemize}
        \item Si \(f : A \to B\) est un morphisme d’anneaux alors \(\im f\) est un sous-anneau de \(B\).
        \item Si \(f : A \to B\) est un morphisme d’anneaux avec \(B\neq \accol{0_B }\) alors \(\ker f\) n’est pas sous-anneau de \(A\).\\~\\
            En effet, on a \(f (1_A) = 1_B\) mais \(1_B\neq 0_B\) (sinon \(B\) serait égal à \(\accol{0_B }\)) donc \(1_A\) n’appartient pas à \(\ker f = \accol{x \in A \tq f (x) = 0_B }\) et, par conséquent, \(\ker f\) n’est pas sous-anneau de \(A\).
    \end{itemize}
\end{defprop}

\subsection{Isomorphisme d’anneaux}
\begin{defi}
    \(f\) est dit isomorphisme d’anneaux si \(f\) est un morphisme d’anneaux et \(f\) est bijective.
\end{defi}

\begin{prop}
    Si \(f : A \to B\) est un isomorphisme d’anneaux alors \(f^{-1} : B \to A\) est un isomorphisme d’anneaux.
\end{prop}

\subsection{Anneau intègre}

\begin{defi}
    On dit qu’un anneau \((A, +, \times)\) est un anneau intègre si les conditions suivantes sont réunies :
    \begin{enumerate}
        \item \(A\neq \accol{0_A}\)
        \item \(\forall (a, b) \in A^2, a \times b = 0_A \imp a = 0_A\text{ ou }b = 0_A\)
    \end{enumerate}
    \underline{Remarque}\\
    Des éléments \(a\) et \(b\) de \(A\) tels que \(a \times b = 0_A\) avec \(a\neq 0_A\) et \(b\neq 0_A\) sont dits diviseurs de \(0_A\).
\end{defi}

\begin{defprop}[Quelques exemples d’anneaux intègres/non intègres déja rencontrés cette année]
    \begin{itemize}
        \item Dans les ensembles de nombres : \(\Z, \Q, \R, \C\) sont des anneaux intègres.
        \item Dans les ensembles de fonctions : \(\mathcal{F}(X, \K), \mathcal{D} (I, \K)\) et \(\mathcal{C}^n (I, \K)\) ne sont pas des anneaux intègres.
        \item Dans les ensembles de matrices : \(\mathcal{M}_{n,p} (\K)\) n’est pas un anneau intègre.
    \end{itemize}
\end{defprop}

\subsection{Corps commutatif}
\begin{defi}
    On dit qu’un anneau \((A, +, \times)\) est un corps commutatif si les conditions suivantes sont réunies :
    \begin{enumerate}
        \item \(A\neq \accol{0_A}\).
        \item \(A\) est commutatif.
        \item tout élément de \(A\) différent de \(0_A\) admet un inverse dans \(A\) pour la loi \(\times\).
    \end{enumerate}
\end{defi}

\begin{prop}
    Tout corps commutatif est un anneau intègre.
\end{prop}

\begin{defprop}[Sous-corps]
    Une partie \(H\) d’un corps \((A, +, \times)\) est un sous-corps de \(A\) si, et seulement si, les conditions suivantes sont réunies :
    \begin{enumerate}
        \item  \(H \) est un sous-anneau de \(A\).
        \item  \(\forall x \in H, x\neq 0_A \imp x^{-1 } \in H \quad\) (où \(x^{-1}\) désigne l’inverse de \(x\) pour la loi \(\times\))
    \end{enumerate}
\end{defprop}

\chapter{Polynômes}

\minitoc
Dans ce chapitre ,\(\K\) désigne le corps \(\R\) ou \(\C\).


\section{Anneau des polynômes à une indéterminée}

\subsection{L'ensemble \(\K\croch{X}\)}

La construction de \(\K \croch{X}\) n’étant pas au programme, on se contente ici d’une présentation sommaire.


\begin{defprop}[Polynômes (formels) à coefficients dans \(\K\)]
    Une suite \(P = \paren{a_k}_{k\in \N}\) de \(\K^\N\) nulle à partir d’un certain rang est dite polynôme à coefficients dans \(\K\).\\~\\
    Pour tout \(k \in \N\), l’élément \(a_k\) est appelé coefficient de degré \(k\) de \(P\) .\\~\\
    \underline{Notations}
    \begin{itemize} 
    \item L’ensemble des polynômes à coefficients dans \(K\) est noté \(\K\croch{X}\).
    \item Le polynôme dont tous les coefficients sont nuls est dit polynôme nul et noté \(0_{\K\croch{X}}\) ou même \(0\).
    \item Le polynôme dont tous les coefficients sont nuls sauf celui de degré \(k\) qui vaut \(1\) est noté \(X^k\).
    \end{itemize} 
\end{defprop}
\begin{defprop}[égalité entre deux polynômes (formels)]
    Deux polynômes de \(\K\croch{X}\) sont égaux si, et seulement si, leurs coefficients de même degré sont égaux.
\end{defprop}

\begin{defprop}[Degré d’un polynôme (formel)]
    Le degré d’un polynôme \(P = \paren{a_k}_{k\in \N}\) de \(K\croch{X}\) est noté \(\deg(P)\) et défini de la manière suivante :
    \[\deg(P ) =\begin{cases}
    \max \accol{k \in \N \tq a_k\neq 0} &\text{ si } P\neq 0_{\K\croch{X}}\\
    \minf &\text{ si } P = 0_{\K\croch{X}}
    \end{cases}\]
\end{defprop}

\begin{defprop}[Coefficient dominant d’un polynôme (formel)]
    Soit \(P = \paren{a_k}_{k\in \N}\) un polynôme non nul de \(K\croch{X}\).
    \begin{itemize}
        \item Si \(P\) est de degré \(n\) alors \(a_n\) est dit coefficient dominant de \(P\) .
        \item Si le coefficient dominant de \(P\) est égal à \(1\), on dit que \(P\) est un polynôme unitaire.
    \end{itemize}
\end{defprop}

\subsection{L’anneau intègre \(\paren{\K \croch{X} , +, \times}\)}

\begin{defprop}[Multiplication par un scalaire, somme et produit]
    Soit \(P = \paren{a_k}_{k\in \N}\) et \(Q = \paren{b_k}_{k\in \N}\) deux polynômes de \(K\croch{X}\) et \(\lambda \in \K\)
    \begin{itemize}
        \item Le polynôme de \(K\croch{X}\) noté \(\lambda . P\) défini ci-dessous est dit polynôme multiplication de \(P\) par \(\lambda\) :
            \[\lambda . P = \paren{\lambda a_n}_{n \in \N}\]
        \item Le polynôme de \(K\croch{X}\) noté \(P + Q\) défini ci-dessous est dit polynôme somme de \(P\) et \(Q\) :
            \[P+Q = \paren{a_n + b_n}_{n \in \N}\]
        \item Le polynôme de \(K\croch{X}\) noté \(P \times Q\) défini ci-dessous est dit polynôme produit de \(P\) et \(Q\) :
            \[P \times Q = \paren{c_n}_{n \in \N}\text{ avec, pour tout } n \in \N, c_n = \sum_{k=0}^{n} a_kb_{n-k} = \sum_{k=0}^n a_{n-k} b_k\]
    \end{itemize}
    \underline{Remarque}\\~\\ 
    On peut faire l’analogie ici avec les expressions des coefficients des applications polynomiales obtenues après multiplication d’une application polynomiale par un scalaire, addition ou multiplication de deux applications polynomiales. En particulier, l’expression imposée pour les coefficients du produit de deux polynômes s’explique en pensant aux produits d’applications polynomiales.
\end{defprop}



\begin{defprop}[Notation usuelle des polynômes]
    Si \(P = \paren{a_k}_{k\in \N}\) est un polynôme non nul de \(\K \croch{X}\) alors on note
    \[P =\sum^{\deg(P )}_{k=0} a_k X^k\text{ ou }P =\sum^{\pinf}_{k=0} a_k X^k \]
    \underline{Remarques}
    \begin{itemize}
        \item La somme \(\sum^{\pinf}_{k=0} a_k X^k\) est finie car tous ses termes sont nuls sauf un nombre fini d’entre eux.
        \item Par définition du produit de deux polynômes, on a bien \(X^2 = X \times X\) et même plus généralement, \(X^k = X \times X^{k-1} = X^{k-1} \times X\) pour tout \(k \in \Ns\) ce qui justifie a posteriori la notation \(X^k\) choisie.
    \end{itemize}
\end{defprop}

\begin{defprop}[Effet des opérations polynomiales sur le degré]
    Si \(P\) et \(Q\) sont deux polynômes de \(\K \croch{X}\) et \(\lambda \in \K\) alors :
    \[ \deg(\lambda.P) = \deg(P) \text{ si } \lambda \neq 0\]
    \[ \deg(P \times Q) = \deg(P) + \deg(Q)\]
    \[\begin{cases}
        \deg(P+Q) &= \max (\deg(P),\deg(Q)) \text{ si } \deg(P) \neq \deg(Q) \\
        \deg(P+Q)& \leq \max (\deg(P),\deg(Q)) \text{ si } \deg(P) = \deg(Q)
    \end{cases}\]
\end{defprop}
\begin{dem}
    Démonstration du degré du produit de deux polynôme :\\~\\
    on a pour tout \(n\) entier naturel tel que \( n \geq s + l + 1\) avec \(s = \deg(P)\) et \(l = \deg (Q)\), \(c_n = 0\)\\
    On a aussi \begin{align*}
        c_{s+l} &= \sum_{k = 0}^{s+l} a_k + b_{s+l-k} \\
        &= a_s b_l \qquad \text{car si } k \neq s \text{ alors } k > s \text{ donc } a_s = 0 \text{ ou } s+l-k > l \text{ donc } b_{s+l-k} = 0
    \end{align*}
    or \(a_s \neq 0\) et \(b_l \neq 0\) donc \(a_s b_l \neq 0 \) ainsi \(\deg(P \times Q) = s+l = \deg(P) + \deg(Q)\)
\end{dem}
\begin{defprop}[Structure d’anneau intègre commutatif]
    \(\paren{\K \croch{X} , +, \times}\) est un anneau intègre commutatif dont l’élément neutre
    \begin{itemize}
        \item pour la loi \(+\) est le polynôme nul \((0, 0, \dots \dots )\) noté \(0_{\K\croch{X}}\) ;
        \item pour la loi \(\times\) est le polynôme \(X^0 = (1, 0, \dots , 0, \dots )\) noté \(1_{\K\croch{X}}\).
    \end{itemize}
    En particulier , on a:
    \[\forall(P,Q) \in \paren{\K \croch{X}}^2, PQ = 0_{\K\croch{X}} \imp P = 0_{\K\croch{X}} \text{ ou } Q = 0_{\K\croch{X}}\]
\end{defprop}

\subsection{L’ensemble \(\K_n \croch{X}\)}

\begin{defi}
    Pour \(n \in \N\), on note \(\K_n \croch{X}\) l’ensemble des polynômes de \(\K \croch{X}\) de degré inférieur ou égal à \(n\).
    \underline{Remarques}
    \begin{itemize}
        \item Pour tout \(n \in \N, \K_n \croch{X} \subset \K_{n+1} \croch{X}\)
        \item Les éléments de \(\K_0 \croch{X}\) sont appelés les polynômes constants.
    \end{itemize}
\end{defi}

\begin{defprop}[Structure de \(\K_n \croch{X}\)]
    \(\K_n \croch{X}\) est un sous-groupe de \(\paren{\K \croch{X} , +, \times}\).
    \underline{Remarques}
    \begin{itemize}
        \item  \(\K_n \croch{X}\) n’est pas stable pour la loi \(\times\) donc n’est pas un anneau pour les lois usuelles \(+\) et \(\times\)
        \item 
    \end{itemize}
\end{defprop}

\subsection{Composition de polynômes}
\begin{defi}
    Soit \(\forall(P,Q) \in \paren{\K \croch{X}}^2\) avec \(P = \sum_{k=0}^{\pinf} a_k X^k\).\\~\\
    On appelle polynôme composé de \(P\) et \(Q\) et on note \(P \circ Q\) le polynôme de \(\K \croch{X}\) défini par
    \[P\circ Q = \sum_{k = 0}^{\pinf} a_k Q^k\]
    \underline{Remarque}\\~\\
    On rappelle que \(Q^0 = 1_{\K \croch{X}}\)
\end{defi}

\begin{defprop}[Degré]
    Si \(P\) et \(Q\) sont des polynômes non nuls de \(\K \croch{X}\) et si \(Q\) est non constant alors
    \[\deg(P \circ Q) = \deg(P ) \times \deg(Q)\]
\end{defprop}
\section{Divisibilité et division euclidienne dans \(\K \croch{X}\)}
Soit \(\paren{A,B,C,D} \in \paren{\K \croch{X}}^4\)
\subsection{Divisibilité}
\begin{defi}
    S’il existe \(Q\) dans \(\K \croch{X}\) tel que \(A = BQ\), on dit que \(B\) divise \(A\) (ou que \(B\) est un diviseur de \(A\), ou que \(A\) est divisible par \(B\) ou encore que \(A\) est un multiple de \(B\)) et on note \(B\divise A\).\\~\\
    \underline{Remarque}\\~\\
    Si \(B\) est non nul et si \(B\) divise \(A\) alors il existe un unique \(Q\) dans \(\K \croch{X}\) tel que \(A = BQ\).
\end{defi}

\begin{defprop}[Ensembles des diviseurs et des multiples]
    \begin{itemize}
        \item On note\( \mathcal{D}(A) = \accol{B \in \K \croch{X} \tq \exists Q \in \K \croch{X} , A = BQ}\) l’ensemble des diviseurs de \(A\)
        \begin{itemize}
            \item Si \(A = 0_{\K\croch{X}}\) alors \(\mathcal{D}(A) = \K \croch{X}\).
            \item Si \(A\neq 0_{\K\croch{X}}\) alors \(\mathcal{D}(A)\) est composée de polynômes de degré \(n \leq \deg (A)\)
        \end{itemize}
        \item On note \(B\K \croch{X} = \accol{BQ \tq Q \in \K \croch{X}}\) l’ensemble des multiples de \(B\).
        \begin{itemize}
            \item Si \(B = 0_{\K\croch{X}}\) alors \(B \K \croch{X} = \accol{0_{\K\croch{X}}}\)
            \item Si \(B \neq 0_{\K \croch{X}}\) alors  \(B \K \croch{X}\) est composé de \(0_{\K \croch{X}}\) et de polynômes de degré \(n \geq \deg(B)\).
        \end{itemize}
    \end{itemize}
\end{defprop}

\begin{defprop}[Caractérisation des polynômes associés]
    \(A \divise B\) \ssi il existe \(\lambda\) dans \(\Ks\) tel que \(A = \lambda B\). (A et B sont alors dits associés)
\end{defprop}

\begin{prop}[Propriétés]
    \begin{enumerate}
        \item \(A \divise A\)
        \item Si \(A \divise B\) et \(B \divise C\) alors \(A \divise C\)
        \item Si \(A \divise B\) et \(C \divise D\) alors \(AC \divise BD\)
        \item Si \(A \divise B\) alors, pour tout \(n \in \Ns\) , \(A^n \divise B^n\)
        \item Si \(C \divise A\) et \(C \divise B\) alors, pour tout \(\paren{U,V} \in \paren{\K \croch{X}}^2\) , \(C \divise AU + BV\)
    \end{enumerate}
\end{prop}

\subsection{Division euclidienne}
\begin{theo}[Théorème de la division euclidienne]
    Pour tout \(\paren{A,B}\) de \(\paren{\K \croch{X}}^2\) avec \(B \neq 0_{\K \croch{X}}\), il existe un unique couple \(\paren{Q,R}\) de \(\paren{\K \croch{X}}^2\) tel que : 
    \[A = BQ+R \text{ et } \deg(R)< \deg(B)\]
    Dans la division euclidienne de \(A\) par \(B\), \(A\) est appelé dividende, \(B\) diviseur, \(Q\) quotient et \(R\) reste.
\end{theo}

\begin{dem}
    Soit \(\paren{A,B} \in \paren{\K \croch{X}}^2\) avec \(B \neq 0_{\K \croch{X}}\)
    \begin{itemize}
        \item \unicite\\
        On suppose qu'il existe deux couple \((Q,R)\) et \((Q_1, R_1)\) de \(\paren{\K\croch{X}}\) tel que : 
        \begin{enumerate}
            \item \(A = BQ + R\) et \(\deg(R)<\deg(B)\)
            \item \(A = BQ_1 + R_1\) et \(\deg(R_1)<\deg(B)\)
        \end{enumerate}
        Alors \(B(Q-Q_1) = R-R_1\) et donc \(\deg(B) + \deg( Q-Q_1) = \deg(R-R_1)\)\\
        avec \(\deg(R-R_1) \leq \max(\deg(R),\deg(R_1))\) donc \(\deg(R-R_1) < \deg(B)\).
        Ainsi , \(\deg(B) + \deg( Q-Q_1) < \deg(B)\) donc \(\deg( Q-Q_1) < 0\) ce qui donne \(Q - Q_1 = 0 _{\K \croch{X}}\). On en déduit \(Q = Q_1\) puis \(R = R_1\).
        \item \existence
        \begin{itemize}
            \item Dans le cas où \(B\) divise \(A\), il existe \(Q \in \K \croch{X}\) tel que \(A = BQ\) donc le couple\( (Q, 0_{\K\croch{X}})\) convient.
            \item On se place donc dans le cas où \(B\) ne divise pas \(A\) et on note \(J = \accol{\deg (A - BQ) \tq Q \in \K \croch{X}}\)
        \end{itemize}
        L’ensemble \(J\) est non vide (car il contient le degré de \(\)A) et est inclus dans \(\N\) (car il ne contient pas \(\minf\) puisque \(B\) ne divise pas \(A\) donc, quel que soit \(Q \in \K \croch{X}, A - BQ\neq 0_{\K\croch{X}}\)). L’ensemble \(J\) admet donc un minimum que l’on note \(r\) et il existe donc \(Q Q \in \K \croch{X}\) tel que \(\deg (A - BQ) = r\).\\~\\
        Montrons que le polynôme \(R = A - BQ\) est de degré \(r\) strictement inférieur au degré \(b\) de \(B\).\\~\\
        Pour cela, on peut raisonner par l’absurde, en supposant que \(\deg (R) \geq deg (B)\) donc \(r - b \geq 0\).\\~\\
        En notant \(\alpha\) le coefficient dominant de \(R\), le polynôme\( S = R - \alpha X^{r-b}B\) est alors de degré strictement inférieur à \(r\). Or \(S\) peut s’écrire aussi \(S = A - B\paren{Q + \alpha X^{r-b}}\) donc \(S = A - BT\) avec \(T \in \K \croch{X}\) ce qui prouve que \(\deg (S) \in J\) et donc que \(\deg (S) \geq r\) car \(r est le minimum de J\). Ceci contredit le résultat \(\deg (S) < r\) trouvé.\\~\\
        Ainsi, \(\deg (R) < b\) donc \(A = BQ + R\) avec \(\deg(R) < \deg (B)\) ce qui prouve l’existence attendue.
    \end{itemize}
    \conclusion il existe un unique couple \((Q, R)\) de \(\paren{\K \croch{X}}^2\) tel que : \(A = BQ + R\) et \(\deg(R) < \deg(B)\).
\end{dem}

\begin{defprop}[Caractérisation de la divisibilité]
    Soit \(\paren{A,B} \in \paren{\K \croch{X}}^2 \) avec \(B\neq 0_{\K\croch{X}}\).\\~\\
    \(B\) divise \(A\) si, et seulement si, le reste de la division euclidienne de \(A\) par \(B\) est nul.
\end{defprop}

\section{Fonctions polynomiales et racines}
\subsection{Fonction polynomiale associée à un polynôme}
\begin{defi}
    A tout polynôme \(P= \sum_{k=0}^{\pinf}\) de \(\K \croch{X}\) , on peut associer une fonction \(\wt{P} : \K \to \K\) définie par :
    \[ \forall x \in \K, \wt{P} = \sum_{k=0}^{\pinf} a_k x^k\]
    Cette fonction \(\wt{P}\) est dite fonction polynomiale associée à \(P\).\\~\\
    \underline{Remarque}\\~\\
    Par abus d’écriture, on utilise souvent la même notation pour \(P\) et \(\wt{P}\) alors que ce sont des objets de nature différente (une suite de \(\K^{\N}\) presque nulle pour l’un et une fonction de \(\K\) dans \(\K\) pour l’autre).
\end{defi}

\begin{prop}
        Soit \(\paren{P,Q} \in \paren{\K \croch{X}}^2 \) et \(\lambda \in \K\) alors : 
        \[ \wt{\lambda P} = \lambda \wt{P} \qquad \wt{P +Q} = \wt{P} + \wt{Q} \qquad \wt{P \times Q} = \wt{P} \times \wt{Q} \qquad \wt{P \circ Q} = \wt{P} \circ \wt{Q}\]
\end{prop}

\subsection{Racine (ou zéro) d’un polynôme}
\begin{defi}
    On dit que \(\alpha \in \K\) est une racine (ou un zéro) du polynôme \(P \in \K \croch{X} \text{ si } \wt{P} (\alpha) = 0\).\\~\\
    \underline{Remarques}\\~\\
    Pour \(\alpha \in \K\), l’écriture \(P (\alpha)\) n’a a priori pas de sens car \(P\) est une suite et pas une fonction. En pratique, on note tout de même \(P (\alpha)\) au lieu de \(\wt{P} (\alpha)\) et on parle d’évaluation du polynôme \(P\) en \(\alpha\) et non pas de la valeur de \(P\) en \(X = \alpha\) ce qui n’a pas de sens.
\end{defi}

\begin{defprop}[Caractérisation en termes de divisibilité]
    Soit \(\alpha \in \K\) et \( P \in \K \croch{X}\).\\~\\
    \(\alpha\) est une racine de \(P\) dans \(\K\) \ssi le polynôme \(X - \alpha\) divise \(P\)
\end{defprop}
\begin{dem}
    \begin{itemize}
        \item \impdir On suppose \(\alpha\) racine de \(P\) \\
        Par théoreme de la division euclidienne sur \(P\) par \(X - \alpha\) :
        \[\exists (Q,R) \in \paren{\K \croch{X}}^2 , P = Q\paren{X - \alpha} + R\]
        avec \(\deg(R) < \deg(X-\alpha)\) donc \(R = \beta\) avec \(\beta \in \K\).\\
        Par égalité sur les applications polynomiale on a donc \(\wt{P} = \paren{\wt{X - \alpha}}\wt{Q} + \wt{R}\) \\
        Or \(\alpha\) est racine de \(P\) donc \begin{align*}
            \wt{P}(\alpha) = 0 &\iff 0 = \paren{\wt{(\alpha - \alpha)}}\wt{Q(\alpha)} + \wt{R(\alpha)}\\
            &\iff 0 = \wt{R(\alpha)}\\
            \iff 0 = \beta
        \end{align*}
        Donc \(P = \paren{X - \alpha}Q\) ainsi \(\paren{X- \alpha} \divise P\)
        \item \imprec   On suppose que \(\paren{X - \alpha} \divise P\) \\
        alors \(\exists Q \in \K \croch{X}, P = \paren{X-\alpha}Q\) ainsi \(\wt{P} = \paren{\wt{X - \alpha}}\wt{Q}\) \\
        d'où \(\wt{P}(\alpha) = \paren{\wt{\alpha - \alpha}}\wt{Q}(\alpha) = 0\)\\
        donc \(\alpha\) est racine de \(P\). 
    \end{itemize}
    \conclusion \(\alpha\) est une racine de \(P\) dans \(\K\) \ssi le polynôme \(X - \alpha\) divise \(P\)
\end{dem}
\begin{defprop}[Propriété sur le nombre de racines]
    Soit \(P \in \K \croch{X}\).\\~\\
    \begin{itemize}
        \item Si \(P=0_{\K \croch{X}}\) alors \(P\) a une infinité de racines dans \(\K\).\\
        \item Si \(P = 0 _{\K\croch{X}}\) alors \(P\) a au plus \(\deg(P)\) racines dans \(\K\).
    \end{itemize}
    \underline{Remarque}:\\
    Le polynôme \(P\) est entièrement déterminé par la fonction polynomiale \(\wt{P}\) associée. En effet si, \(\wt{P} = \wt{Q}\) alors \(\wt{P-Q} = 0_{\K \croch{X}}\) donc \(P-Q\) a une infinité de racines et par conséquent \(P-Q = 0_{\K \croch{X}}\) puis \(P = Q\) 
\end{defprop}
\begin{dem}
    On note \(n = \deg(P)\).\\
    Supposons que \(P\) a strictement plus de \(n\) racines distinctes dans ce cas il existe \((\alpha_1, \dots \alpha_{n+1})\) qui sont racines de \(P\), alors par propriété : \[Q = \prod_{k=1}^{n+1} \paren{X - \alpha_k} \text{ divise } P\]
    donc \(\deg(Q) \leq \deg(P)\) \ie \(n+1 \leq n\) ce qui est faux.
\end{dem}

\begin{defprop}[Multiplicité d'une racine]
    Soit \(P\) un polynôme  de \(\K\croch{X}\), \(\alpha \in \K\) et \(m \in \N\).\\~\\
    On dit que \(\alpha\) est racine de multiplicité \(m\) dans \(P\) si \(\begin{cases}
        \paren{X-\alpha}^m \text{ divise } P \\
        \paren{X - \alpha} ^{m+1} \text{ divise } P
    \end{cases}\)\\
    autrement  dit s'il existe \(Q \in \K \croch{X}\) tel que: \[P = \paren{X-\alpha}^m Q \text{ avec } Q(\alpha) \neq 0\]
    \underline{Remarques}\\
    \begin{itemize}
        \item Dire que \(\alpha\) est de multiplicité \(0\) dans \(P\) signifie que \(\alpha\) n'est pas racine de \(P\).
        \item Une racine de \(P\) est dite simple (resp. double, triple,...) si sa multiplicité est 1 (resp. 2,3,...)
    \end{itemize}
\end{defprop}
\subsection{Polynômes scindés}
\begin{defi}
    Un polynôme de \(\K \croch{X}\) est dit scindé sur \(\K\) s'il peut s'écrire comme produit de polynômes de \(\K \croch{X}\) de degré \(1\) (non nécessairement distincts).
\end{defi}
\begin{defprop}[Propriété sur le degré]
    Soit \(P\) un polynôme non constant de \(\K \croch{X}\) . \\
    Si \(P\) est scindé sur \(\K\) alors le degré de \(P\) est égal à la somme des multiplicités de ses racines dans \(\K\).
\end{defprop}

\begin{defprop}[Divisibilité par un produit de polynômes distintc de degré \(1\)]
    Soit \(P \in  \K \croch{X}\) ayant \(r\) racines distinctes \(\alpha_1, \dots , \alpha_r\) avec \(r \in \Ns\)\\
    Alors \(\prod_{k=1}^{r} \paren{X-a_k}\) divise \(P\)
\end{defprop}

\begin{dem}
    Montrons cette propriété par récurrence.
    \begin{itemize}
        \item Pour \(r=1\) la propriété est vérifié
        \item Soit \(r \in \Ns\) tel que, Pour tout \(P\) de \(\K\croch{X}\) ayant \(r\) racines distinctes \(\alpha_1, \dots , \alpha_r\) on a \(\prod_{k=1}^{r} \paren{X-a_k} \divise P\)\\
        Soit \(P\) un polynôme de \(\K\croch{X}\) ayant\(r+1\) racines distinctes \(\alpha_1,\dots,\alpha_r,\alpha_{r+1}\).
        Par hypothèse de récurrence, \(\prod_{k=1}^r \paren{X-\alpha_k}\) divise \(P\) donc il existe un polynôme \(Q\) de \(K \croch{X}\) tel que \(P = Q \prod_{k=1}^{r}\paren{X-\alpha_k}\) \\.
        Comme \(\alpha_{r+1}\) est racine de \(P\), en évaluant ces polynômes en \(\alpha_{r+1}\), on trouve \(0 = Q(\alpha_{r+1})\prod_{k=1}^r \paren{\alpha_{r+1}-\alpha_k}\)\\
        donc \(Q(\alpha_{r+1}) = 0\) puisque les \(\alpha_k\) sont deux à deux distincts.\\
        Ainsi, \(Q\) a pour racine \(\alpha_{r+1}\) donc, par propriété, \(\paren{X-\alpha_{r+1}}\) divise \(Q\) i. e. il existe un
        polynôme \(S\) tel que\( Q = (X - \alpha _{r+1}) S\) ce qui donne \(P = S \prod_{k=1}^{r+1}\paren{X-\alpha_k}\)
        La propriété est donc vraie au rang \(r + 1\).
    \end{itemize}
    \conclusion Si \(P \in \K \croch{X}\) a \(r\) racines distinctes \(\alpha_1, \dots , \alpha_r\) avec \(r \in \Ns\) alors \(\prod_{k=1}^{r} \paren{X-a_k}\) divise \(P\)
\end{dem}
\section{Polynômes dérivés}
\subsection{Dérivée formelle d'un polynôme}
\begin{defi}
    ~\\
    Soit \(P =  \sum_{k=0}^{\pinf} a_k X^k\) un polynôme \(P\) noté \(P'\) défini par 
    \[P' = \sum_{k=1}^{\pinf} k a_k X^k = \sum_{k=0}^{\pinf} (k+1) a_{k+1}X^k\] 
\end{defi}

\begin{defprop}[Degré du polynôme dérivé]
    ~\\
    Si \(P\) est un polynôme de \(\K\croch{X}\) alors \(\begin{cases}
        \deg(P') = \deg(P) -1 &\text{ si } \deg(P) \geq 1\\
        P' = 0_{\K \croch{X}} & \text{ sinon }
    \end{cases}\)
\end{defprop}
\begin{defprop}[Lien avec la dérivée de la fonction polynomiale associée]
    Dans le cas particulier où \(P\) est un polynôme à coefficients réels, on a \(\wt{P'} = \paren{\wt{P}}'\)
\end{defprop}
\begin{defprop}[Opération sur les polynômes dérivés]
    Soit \(P\) et \(Q\) deux polynômes de \(\K \croch{X}\) et \(\lambda \in \K\)\\
    Alors : 
    \[\paren{ \lambda P}' = \lambda P' \quad \paren{P + Q}' = P' + Q' \quad \paren{P \times Q}' = P' \times Q + P \times Q' \quad \paren{P \circ Q}' = Q' \paren{P' \circ Q}\]
\end{defprop}

\subsection{Polynômes dérivés successifs}
\begin{defi}
    Soit \(P\) un polynôme de \(\K \croch{X}\).\\
    On pose \(P^{\paren{0}} = P\) et, pour \(k \in \N, P ^{\paren{k+1}} = \paren{ P ^{\paren{k}}}'\) appelé polynôme dérivé formel de \(P\) d’ordre \(k + 1\)
\end{defi}

\begin{defprop}[Degré des polynômes dérivés successifs]
    ~\\
    Si \(P\) est un polynôme de \(\K \croch{X}\) et \(n\) un entier naturel alors \(\begin{cases}
        \deg(P^{(n)}) = \deg(P)-n &\text{ si } \deg(P) \geq n \\
        P ^{(n)} = 0_{\K \croch{X}} & \text{ sinon}
    \end{cases}\)
\end{defprop}

\begin{defprop}[Opérations sur les polynômes dérivés successifs]
    Soit \(P\) et \(Q\) deux polynômes de \(\K \croch{X}\), \(\lambda \in \K\) et \(n \in \N\)\\
    Alors : 
    \[(\lambda P)^{(n)} = \lambda P ^{(n)}\]
    \[\paren{P + Q} ^{(n)} = P^{(n)} + Q^{(n)}\]
    \[\paren{P \times Q}^{(n)} = \sum_{k=0}^n \binom{n}{k} P ^{(k)} \times Q ^{(n-k)} = \binom{n}{k} P ^{(n-k)} \times Q ^{(k)}\qquad \text{formule de Leibniz}\] 
\end{defprop}

\begin{defprop}[Formule de Taylor polynomiale]
    Pour tout polynôme de \(P\) de \(\K \croch{X}\) et tout \(\alpha\) dans \(\K\), on a : 
    \[P = \sum_{k=0}^{\pinf} \frac{P ^{k}(\alpha)}{k!}\paren{X- \alpha}^k\]
    \underline{Remarque}\\
    On en déduit que, pour tout \(k \in \N\), le coefficient de degré \(k\) de \(P\) est \(a_k = \frac{P^{(k)}(0)}{k!}\)
\end{defprop}

\begin{dem}
    \begin{itemize}
        \item Préliminaire : \\
        Soit \(Q \in \K \croch{X}\) \\
        Montrons que :\(Q = \sum_{k=0}^{\pinf} \frac{Q ^{k}(0)}{k!}\paren{X}^k\)\\
        Soit \(Q = \sum_{i=0}^{\pinf} a_i X^i\) \\
        alors \(Q^{(k)} = \sum_{k=0}^{\pinf} a_i \paren{X^i}^{(k)}\) avec \(\paren{X^i}^{(k)} = 0_{\K \croch{X}}\) si \(k>i\)\\
        donc \begin{align*}
            Q^{(k)}(0) &= \sum_{k=0}^{\pinf} \frac{Q ^{k}(0)}{k!}\paren{X}^a_i \frac{i!}{(i-k)!} X^{i-k} \\
            &= a_k \frac{k!}{0!} +  \sum_{k=0}^{\pinf} \frac{Q ^{k}(0)}{k!}\paren{X}^a_i \frac{i!}{(i-k)!} X^{i-k} \\
            & = a_k \frac{k!}{0!}  \hfill \text{ Car en 0} X^{i-k} = 0
        \end{align*}
        donc \(a_k =  \frac{Q^{(k)}(0)}{k!}\) ainsi on conclut que : \(Q = \sum_{k=0}^{\pinf} \frac{Q ^{k}(0)}{k!}\paren{X}^k\)
        \item Preuve en \(\alpha\)\\
        On applique le préliminaire à \(Q = P \circ \paren{X+\alpha}\)\\
        alors \(Q' = \paren{P' \circ \paren{X + \alpha}}\paren{X+\alpha}' =P' \circ \paren{X + \alpha} \) donc \(Q'(0) = P'(\alpha)\)\\
        et par récurrence immédiate \(Q^{(k)} = P^{(k)}\circ \paren{X-\alpha}\)  donc \(Q^{(k)}(0) = P^{(k)}(\alpha)\)\\
        Ainsi avec le préliminaire on sait  : \(P \circ \paren{X-\alpha} = \sum_{k=0}^{\pinf} \frac{P ^{k}(\alpha)}{k!}\paren{X}^k\) \\
        d'où \(P \circ \paren{X-\alpha} \circ \paren{X + \alpha} = \sum_{k=0}^{\pinf} \frac{P ^{k}(\alpha)}{k!}\paren{X-\alpha}^k\)
    \end{itemize}
    \conclusion \(\forall P \in \K \croch{X}, \forall \alpha \in \K ,P = \sum_{k=0}^{\pinf} \frac{P ^{k}(\alpha)}{k!}\paren{X- \alpha}^k\)
\end{dem}

\begin{defprop}[Caractérisation de la multiplicité d’une racine par les polynômes dérivés successifs]
    Soit \(P\) un polynôme de \(\K \croch{X}\), \(\alpha \in \K\) et \(m \in \Ns\).\\~\\
    \(\alpha\) est racine de multiplicité \(m\) dans \(P\) \ssi \(\begin{cases}
        P^{(k)}(\alpha) = 0 &\text{ pour tout } k \in \interventierii{0}{m-1}\\
        P^{(m)}(\alpha) \neq 0
    \end{cases}\)
    \underline{Remarque}
    On en déduit que si \(\alpha\) est de multiplicité m non nulle dans \(P\) alors \(\alpha\) est de multiplicité \(m - 1\) dans \(P '\).
\end{defprop}

\begin{dem}
    Par la formule de Taylor polynomiale en \(\alpha\) on a  : \[P  = \sum_{k=0}^{m-1} \frac{P ^{k}(\alpha)}{k!}\paren{X- \alpha}^k + \paren{X-\alpha}^m \sum_{k=m}^{\pinf} \frac{P ^{k}(\alpha)}{k!}\paren{X- \alpha}^k\] 
    \ie \(P = \paren{X-\alpha}^mQ + R \)  avec \(\begin{cases}
        \sum_{k=0}^{m-1} \frac{P ^{k}(\alpha)}{k!}\paren{X- \alpha}^k &= R \text{ et } \deg(R) < \deg\paren{ \paren{X-\alpha}^m}\\
        \sum_{k=m}^{\pinf} \frac{P ^{k}(\alpha)}{k!}\paren{X- \alpha}^k & = Q
    \end{cases}\)
    On en déduit que  : \begin{align*}
        \alpha \text{ est racine de multiplicité au moins } m &\iff \paren{X-\alpha}^m \divise P \\
        &\iff \sum_{k=0}^{m-1} \frac{P ^{k}(\alpha)}{k!}\paren{X- \alpha}^k = 0_{\K \croch{X}} \\
        &\iff \forall k \interventierii{1}{m-1} P ^{(k)}(\alpha) = 0
    \end{align*}
    \conclusion \(\alpha\) est racine de multiplicité \(m\) dans \(P\) \ssi \(\begin{cases}
        P^{(k)}(\alpha) = 0 &\text{ pour tout } k \in \interventierii{0}{m-1}\\
        P^{(m)}(\alpha) \neq 0
    \end{cases}\)
\end{dem}

\section{Trois classiques incontournables}
Soit \(n \in \Ns\).
\subsection{Méthode de Horner pour l’évaluation polynomiale}
\begin{defprop}
    L’évaluation en \(\alpha \in \K\) du polynôme de degré \(n\), \(P = \sum_{k = 0}^n a_k X^k\) de \(\K \croch{X}\), peut se faire ainsi : 
    \[P(\alpha) = a_0 + \alpha \paren{a_1 + \alpha\paren{a_2 + \dots + \alpha \paren{ a_{n-1} + \alpha a_n}}}\]
    Cet algorithme dit "schéma de Horner" a une complexité linéaire (en version itérative ou récursive) alors que la méthode naïve d’évaluation a une complexité quadratique.
\end{defprop}
\subsection{Formule d’interpolation de Lagrange}
\begin{defprop}

    Soit \((x_1, \dots, x_n)\) une famille de \(n\) éléments de \(\K\) deux à deux distincts.\\
    Soit \((y_1, \dots, y_n)\) une famille de \(n\) éléments de \(\K\).\\~\\
    Il existe un unique polynôme \(P\) de \(K_{n-1}\croch{X}\) tel que : \(\forall j \in \interventierii{1}{n} , P (x_j ) = y_j\) . Ce polynôme, dit polynôme interpolateur de Lagrange, est donné par :
    \[p = y_1 L_1 + \dots y_n + L_n \qquad \text{ avec } \qquad \forall i \in \interventierii{1}{n}, L_i = \frac{\prod_{k=1,k \neq i}^{n}\paren{X - x_k}}{\prod_{k=1,k \neq i}^{n}\paren{x_i - x_k}}\]
    \underline{Remarque}: \\
    \begin{itemize}
        \item \(\forall (i,i) \in \paren{\interventierii{1}{n}}^2 L_i(x_j) = \delta_{ij}\)
        \item Plus généralement les polynômes \(Q\) de \(\K \croch{X}\) tels que \(\forall j \in \interventierii{1}{n}, Q(x_j) =  y_j\) sont les polynômes.
        \[Q = P + \paren{\prod_{k=1}^{n}\paren{X-x_k}}S\]
        où \(P\) est le polynôme interpolateur de Lagrange et \(S\) un polynôme quelconque de \( \K \croch{X}\).
    \end{itemize}
\end{defprop}
\begin{dem}
Soit \((x_1, \dots , x_n)\) une famille de \(n\) éléments de \(\K\) deux à deux distincts et \((y_1, \dots , y_n)\) une famille de \(n\) éléments de \(\K\).
\begin{itemize}
    \item \unicite \\
    On suppose qu’il existe deux polynômes \(P\) et \(Q\) de \(K_{n-1} \croch{X}\) tels que :\( \forall j \in \interventierii{1}{N} , P (x_j ) = Q(x_j ) = y_j \)\\
    Alors : \(\forall j \in \interventierii{1}{n} , (P - Q)(x_j ) = 0\) donc le polynôme \(P - Q\) a \(n\) racines distinctes.\\
    Comme \(P - Q\) appartient à \(K_{n-1} \croch{X}\), on en déduit que \(P - Q = 0_{\K\croch{X}}\) puis que \(P = Q\).
    \item \existence \\
    On exhibe ici un polynôme qui convient.\\~\\
    Pour cela, on pose, pour tout \(i \in \interventierii{1}{n},L_i = \frac{\prod_{k=1,k \neq i}^{n}\paren{X - x_k}}{\prod_{k=1,k \neq i}^{n}\paren{x_i - x_k}} \) \\
    Soit \(i \in \interventierii{1}{n}\)\\~\\
    Comme produit de \(n - 1\) polynômes de degré \(1\), \(L_i\) est un polynôme de degré \(n - 1\) donc a fortiori \(L_i\) appartient à \(\K{n-1} \croch{X}\). De plus, \(L_i(x_i) = 1\) et \(L_i(x_j ) = 0\) si \(i\neq j\) autrement dit \(L_i(x_j ) = \delta_{i,j} \).\\
    Ainsi, \(P = \sum^n _{i=1} y_iL_i\) est un polynôme de \(\K_{n-1} \croch{X}\) qui vérifie \(\forall j \in \interventierii{1}{n}, P (x_j ) = \sum^n _{i=1} y_iL_i(x_j ) = y_j\) .
\end{itemize}
\conclusion il existe un unique polynôme \(P\) de \(\K_{n-1}\croch{X}\) tel que : \(\forall j \in \interventierii{1}{n}, P (x_j ) = y_j\) .
\end{dem}
\subsection{Relations entre coefficients et racines (formules de Viète)}
\begin{defprop}
    Si \(P\) est un polynôme de \(\K \croch{X}\) de degré \(n\), scindé sur \(\K\) de racines \(\alpha_1, \dots , \alpha_n\) (répétées avec multiplicité) alors, en notant \(P = \sum_{k=0}^{n}a_k X^k\), on a :
    \[\forall i \in \interventierii{1}{n}, \sigma_i = \paren{-1}^i \frac{a_{n-i}}{a_n} \quad \text{ avec } \sigma_i = \sum_{1 \leq k_1 < k_2 <\dots< k_i \leq n} \alpha_{k_1} \alpha_{k_2} \dots \alpha_{k_i}\]
    \underline{Remarque}: \\
    Les formules concernant la somme \(\sigma_1\) et le produit des racines \(\sigma_n\) sont à connaître par coeur :
    \[\sigma_1 = \sum_{k=1}^n \alpha_k = - \frac{a_{n-1}}{a_n} \qquad \text{ et } \qquad \sigma_n = \prod_{k=1}^{n} \alpha_k = (-1)^n \frac{a_0}{a_n}\]
    Les autres sont à savoir retrouver rapidement.
\end{defprop}

\begin{dem}
    Par hypothèse sur \(P\), on peut écrire
    \[P = a_n\paren{X-\alpha_1}\paren{X-\alpha_2}\dots \paren{X-\alpha_n}\]
    ce qui donne après calculs dans l’anneau commutatif \(\K \croch{X}\)
    \[P = a_n\paren{X^n - \paren{\alpha_1 + \alpha_2 + \dots + \alpha_n}X^{n-1} + \paren{\alpha_1 \alpha_2 + \alpha_1 \alpha_3 + \dots + \alpha_{n-1} \alpha_n}X^{n-2} + \dots + \paren{-1}^n \alpha_1 \alpha_2 \dots \alpha_n}\]
    ou plus précisément 
    \[P = a_n \paren{X^n - \sigma_1 X^{n-1} + \sigma_2 X^{n-2}} + \dots \paren{-1}^{n-1} \sigma_{n-1}X + \paren{-1}^n \sigma_n\]
    avec 
    \[\sigma_i = \sum_{1 \leq k_1 < k_2 <\dots< k_i \leq n} \alpha_{k_1} \alpha_{k_2} \dots \alpha_{k_i}\]
    Par ailleurs, \(P = \sum_{k=0}^n a_k X^k\) donc, comme deux polynômes sont égaux si, et seulement si, leurs coefficients de même degré sont égaux, on trouve :
    \[\forall i \interventierii{1}{n}, (-1)^i \sigma_i a_n = a_{n-i}\] 
    et donc \[\forall i \in \interventierii{1}{n} \sigma_i = (-1)^i \frac{a_{n-1}}{a_n}\]
\end{dem}

\chapter{Analyse Asymptotique \((1)\)}

\minitoc
Dans ce chapitre, \(\K\) désigne le corps \(\R\) ou \(\C\) et \(I\) un intervalle de \(\R\), non vide et non réduit à un point.

\section{Relations de comparaison pour les fonctions}

\subsection{Définition}
\begin{defprop}[Domination]
    On dit que \(f\) est dominée par \(g\) au voisinage de \(a\) s’il existe un voisinage \(V_a\) de \(a\) et une fonction \(M : I \inter V_a \to \K\) bornée tel que
    \[\forall x \in  I \inter V_a, f (x) = M (x)g(x)\]
    On note alors \(f (x) = \underset{x\to a }{o \paren{g(x)}}\) ou \(f \underset{a}{=} o \paren{g}\).
\end{defprop}


\begin{defprop}[Négligeabilité]
    On dit que \(f\) est négligeable devant \(g\) au voisinage de \(a\) s’il existe un voisinage \(V_a\) de \(a\) et une fonction \(\epsilon : I \inter V_a \to \K\) de limite nulle en \(a\) tel que
    \[\forall x \in  I \inter V_a, f (x) = \epsilon(x)g(x)\]
    On note alors \(f (x) \underset{x\to a}{=}o \paren{g(x)}\) ou \(f \underset{a}{} o \paren{g}\).
\end{defprop}
\begin{defprop}[Equivalence]
    On dit que \(f\) est équivalente à \(g\) au voisinage de \(a\) s’il existe un voisinage \(V_a\) de \(a\) et une fonction \(u : I \inter V_a \to \K\) de limite égale à \(1\) en a tel que
    \[\forall x \in  I \inter V_a, f (x) = u(x)g(x)\]
    On note alors \(f (x) \underset{x\to a}{\sim} g(x)\) ou \(f \underset{a}{\sim} g\).
\end{defprop}

\subsection{Caractérisations pratiques}

\begin{defprop}
    Soit \(f : I \to  \K\) et \(g : I \to  \K\) deux fonctions et \(a \in  \R\) tel que \(a\) est point ou extrémité de \(I\).\\
    Dans le cas où \(g\) ne s’annule pas au voisinage de \(a\), on a les équivalences suivantes :
    \begin{enumerate}
        \item \(f \underset{a}{=} o \paren{g}\) si, et seulement si, la fonction \(\frac{f}{g}\) est bornée au voisinage de \(a\).
        \item \(f \underset{a}{=} o \paren{g}\) si, et seulement si, la fonction \(\frac{f}{g}\) a pour limite \(0\) en \(a\).
        \item \( f \underset{a}{\sim} g\) si, et seulement si, la fonction \(\frac{f}{g}\) a pour limite \(1\) en \(a\).
    \end{enumerate}
    \underline{Remarques}
    \begin{itemize}
        \item En pratique, ce sont ces caractérisations qui seront utilisées plutôt que les définitions.
        \item L’étude locale de \(f\) au voisinage de \(a \in  \R \) se ramène à l’étude de la fonction \(f (a + h)\) pour \(h \to  0\).
        \item Les équivalences précédentes sont encore valables dans le cas où \(f\) et \(g\) s’annulent en \(a\) avec \(g\) qui ne s’annule pas sur un voisinage de \(a\) privé de \(a\).
    \end{itemize}
\end{defprop}

\subsection{Lien entre les relations de comparaison}

\begin{defprop}
    Soit \(f : I \to  \K\) et \(g : I \to  \K\) deux fonctions et \(a \in  \R\) tel que \(a\) est point ou extrémité de \(I\).\\
    Alors, on a :
    \begin{enumerate}
        \item \(f \underset{a}{=} o \paren{g} \imp f \underset{a}{=} o \paren{g}\) 
        \item \(f \underset{a}{\sim} g \imp f \underset{a}{=} o \paren{g}\) 
        \item \(f \underset{a}{\sim} g \iff f \underset{a}{=} g + o \paren{g}\).
    \end{enumerate}

\end{defprop}

\subsection{Traduction des croissances comparées à l’aide des “\(o\)”}

\begin{defprop}[Au voisinage de \(\pinf\)]
    Pour tous réels strictement positifs \(\alpha, \beta\) et \(\gamma\), on a :
    \begin{itemize}
        \item \((\ln(x))^{\beta} \underset{x\to \pinf}{=}o \paren{x^{\alpha}}\) ;
        \item \(x^{\alpha} \underset{x\to \pinf}{=} o \paren{e^{\gamma x}}\) ;
        \item \( x^{\alpha} \underset{x\to \pinf}{=} o \paren{ x^{\beta}} \text{ dans le cas } \alpha < \beta\)
    \end{itemize}


\end{defprop}

\begin{defprop}[Au voisinage de \(0\)]
    Pour tous réels strictement positifs \(\alpha\) et \(\beta\), on a :
    \begin{enumerate}
        \item \(\abs{\ln(x)}^{\beta} \underset{x\to 0}{=} o \paren{\frac{1}{x^{\alpha}}}\)
        \item \(x^{\alpha} \underset{x\to 0}{=} o\paren{ x^{\beta}} \text{ dans le cas} \alpha > \beta\).
    \end{enumerate}
\end{defprop}

\subsection{Obtention et utilisation des équivalents}

\begin{defprop}[Obtention d’un équivalent par encadrement]
    Si \(f\), \(g\) et \(h\) sont à valeurs réelles et vérifient \(g \leq f \leq h\) au voisinage de \(a\) avec \( g \underset{a}{\sim} h \) alors \( f \underset{a}{\sim} h \).
\end{defprop}

\begin{defprop}
    \begin{itemize}
        \item Si \(f\underset{a}{\sim} g\) alors \(f\) et \(g\) ont même “comportement” au voisinage de \(a\), c’est-à-dire que :
        \begin{itemize}
            \item \(f\) a pour limite \(l\) en \(a\) si, et seulement si, \(g\) a pour limite \(l\) en \(a\).
            \item \(f\) n’a pas de limite en \(a\) si, et seulement si, \(g\) n’a pas de limite en \(a\).
        \end{itemize}
        \item Si \(f \underset{a}{\sim} g\) alors \(f\) et \(g\) ont le même signe au voisinage de \(a\).
    \end{itemize}
\end{defprop}

\subsection{Règles usuelles de manipulation des relations de comparaison}

\begin{defprop}[Cas des \(O\) (et des \(o\))]
    \begin{enumerate}
    \item Si \(f \underset{a}{=} O(g)\) et \(\lambda  \in  \Ks\) alors \(f \underset{a}{=}  O(\lambda  g)\) et \(\lambda f \underset{a}{=} O(g)\).
    \item Si \(f \underset{a}{=} O(g)\) et\( g \underset{a}{=} O(h)\) alors \(f \underset{a}{=} O(h)\).
    \item Si \(f \underset{a}{=} O(g)\) et\( h \underset{a}{=} O(g)\) alors \(f + h \underset{a}{=} O(g)\).
    \item Si \(f \underset{a}{=} O(g)\) alors \(f h \underset{a}{=} O(gh)\).
    \item Si \(f \underset{a}{=} O(g)\) et \(i \underset{a}{=} O(h)\) alors \(f i \underset{a}{=} O(gh)\).
    \item Si \(f \underset{a}{=} O(g)\) et \(\lim_{b} h = a\) alors \(f \circ h \underset{b}{=} O(g \circ h)\).
    \end{enumerate}
\underline{Remarques}
    \begin{itemize}
        \item Dans tout ce qui précède, on peut remplacer \(O\) par \(o\).
        \item JAMAIS de “composition des \(O\) (ou des \(o\)) à gauche” sans preuve directe
    \end{itemize}
\end{defprop}

\begin{defprop}[Cas des équivalents]
    \begin{enumerate}
    \item Si \(f \underset{a}{\sim} g\) alors \(g \underset{a}{\sim} f\) .
    \item Si \(f \underset{a}{\sim} g\) et \(g \underset{a}{\sim} h\) alors \(f \underset{a}{\sim} h\).
    \item Si \(f \underset{a}{=} O (g)\) et \(g \underset{a}{\sim} h\) alors \(f \underset{a}{=} O (h)\).
    \item Si \(f \underset{a}{=} o (g)\) et \(g \underset{a}{\sim} h\) alors \(f \underset{a}{=} o (h)\).
    \item Si \(f \underset{a}{\sim} g\) alors \(f h \underset{a}{\sim} gh\).
    \item Si \(f \underset{a}{\sim} g\) et \(i \underset{a}{\sim} h\) alors \(f i \underset{a}{\sim} gh\).
    \item Si \(f \underset{a}{\sim} g\) avec \(f\) et \(g\) strictement positives au voisinage de \(a\) alors, pour tout réel \(\beta\), \(f ^{\beta} \underset{a}{\sim} g^{\beta}\) .
    \item Si \(f \underset{a}{\sim} g\) avec \(f\) et \(g\) ne s’annulant pas au voisinage de \(a\) alors \(\frac{1}{f} \underset{a}{\sim}\frac{1}{g}\) .
    \item Si \(f \underset{a}{\sim} g\) et \(\lim_{b} h = a\) alors\( f \circ h \underset{b}{\sim}g \circ h\).
    \end{enumerate}
    JAMAIS de “composition à gauche” ni de somme d’équivalents sans preuve directe.
\end{defprop}

\section{Développements limités}
\subsection{Généralités}
    Dans cette partie, \(f : I \to  \K\) est une fonction et \(a\) un RÉEL, point ou extrémité de \(I\).
\begin{defi}
    On dit que \(f\) admet un développement limité à l’ordre \(n \in  \N\) en \(a\) (abrégé en \(DL_n(a)\)) s’il existe des éléments \(b_0, \dots, b_n\) de \(\K\) tels que :
    \[f (x) \underset{x\to a}{=} b_0 + b_1(x - a) + \dots + b_n(x - a)^n + o \paren{(x - a)^n}\]
    \underline{Remarque}\\
    En pratique, on se ramènera à la recherche d’un développement limité pour \(h \mapsto  f (a + h)\) en \(0\).
\end{defi}

\begin{defprop}[Exemple important déjà vu]
    Pour tout \(n \in  \N\), \(x \mapsto  \frac{1}{1 - x}\) admet un \(DL_n(0)\) qui est : \(\frac{1}{1 - x} \underset{x \to 0}{=} 1 + x + x^2 + \dots + x^n + o(x^n)\).
\end{defprop}

\begin{defprop}[Unicité d’un développement limité]
   S ’il existe des éléments \(b_0, \dots, b_n\) de \(\K\) tels que :
   \[f (x) \underset{x \to a}{=} b_0 + b_1(x - a) + \dots + b_n(x - a)^n + o \paren{(x - a)^n}\]
   alors ces éléments sont uniques.
   \begin{itemize}
        \item Ces éléments \(b_0, \dots, b_n\) sont appelés coefficients du \(DL_n(a)\) de \(f\) .
        \item La fonction polynomiale \(x \mapsto  b_0 + \dots + b_n(x - a)^n\) est dite partie régulière du \(DL_n(a)\) de \(f\) .
   \end{itemize}
\end{defprop}

\begin{dem}
    On raisonne par l’absurde.\\~\\
    Supposons qu’il existe \((b_0, \dots, b_n) \in \K^{n+1}\) et \((c_0, \dots, c_n) \in \K^{n+1}\) avec \((b_0, \dots, b_n)\neq (c_0, \dots, c_n)\) tel que :
    \[f (x) \underset{x \to a}{=} b_0 + b_1(x - a) + \dots + b_n(x - a)^n + o ((x - a)^n)\]
    \[f (x) \underset{x \to a}{=} c_0 + c_1(x - a) + \dots + c_n(x - a)^n + o ((x - a)^n)\]
    On note \(p\) le plus petit entier de \(\interventierii{0}{n}\) tel que \(b_p\neq c_p\).\\
    Puisque\( b_k = c_k\) pour tout \(k \in \interventierii{0}{p - 1}\) , on a alors :\\
    \[b_p(x - a)^p + \dots + b_n(x - a)^n + o ((x - a)^n) \underset{x \to a}{=} c_p(x - a)^p + \dots + c_n(x - a)^n + o ((x - a)^n)\]
    Après division par \((x - a)^p\) sur un voisinage de \(a\) privé de \(a\), on trouve :
    \[b_p + \dots + b_n(x - a){n-p} + o ((x - a)^{n-p}) = c_p + \dots + c_n(x - a)^{n-p} + o ((x - a)^{n-p})\]
    Par passage à la limite en \(a\) dans cette égalité, on obtient
    \[b_p = c_p\]
    ce qui est faux par hypothèse sur \(b_p\) et \(c_p\).\\
    On en déduit que l’hypothèse initiale est fausse ce qui permet de conclure.\\
    \conclusion si f admet un développement limité à l’ordre n au voisinage de a alors il est unique.
\end{dem}


\begin{defprop}[Troncature d’un développement limité]
    Si \(f\) admet un développement limité à l’ordre \(n \in  \N\) en \(a\) qui s’écrit \(f (x) \underset{x\to a}{=} \sum^n_{k=0} b_k(x - a)^k +o ((x - a)^n)\)
    alors \(f\) admet un développement limité à tout ordre \(m \in  \interventierii{0}{n}\) obtenu par troncature du \(DL_n(a)\) :
    \[f (x) \underset{x\to a}{=} \sum^m_{ k=0} b_k(x - a)^k + o ((x - a)^m)\] 
\end{defprop}

\subsection{Premiers résultats importants}
\begin{defprop}[Développement limité et équivalent]
    Soit \(f : I \to  \K\) une fonction et \(a\) un RÉEL, point ou extrémité de \(I\).\\
    Si \(f\) admet un développement limité à l’ordre \(n \in  \N\) en a qui s’écrit \(f (x) \underset{x\to a}{=}\sum^n_{k=p} b_k(x - a)^k +o ((x - a)^n)\) avec \(p \in  \interventierii{0}{n} \) et \(b_p\neq 0\) alors \(f (x) \underset{x \to a}{\sim} b_p(x - a)^p\).
\end{defprop}

\begin{defprop}[Cas des fonctions paires ou impaires]
    On suppose ici que \(I\) est centré en \(0\).
Si \(f : I \to  \K\) admet un développement limité à l’ordre \(n \in  \N\) en \(0\) et que
    \begin{enumerate}
        \item \(f\) est paire alors la partie régulière de son \(DL_n(0)\) ne comporte que des monômes pairs.
        \item \(f\) est impaire alors la partie régulière de son \( DL_n(0)\) ne comporte que des monômes impairs.
    \end{enumerate}

\end{defprop}

\begin{dem}
    On se place dans le cas où \(f\) est paire (preuve facile à adapter pour \(f\) impaire) et où \(f\) admet un développement limité à l’ordre \(n \in \N\) en \(0\).\\~\\
    Alors, il existe\((b_0, \dots , b_n) \in \K^{n+1}\) tel que :
    \[f (x) \underset{x \to}{=} \sum^{n}_{k=0} b_kx^k + o (x^n)\]
    Par composition à droite par la fonction \(h : x \mapsto -x\), on trouve :
    \[f (-x) \underset{x \to}{=} \sum^n_{k=0} b_k(-x)^k + o (x^n)\]
        donc
    \[f (x) \underset{x \to}{=}\sum^n_{k=0}(-1)^kb_kx^k + o (x^n)\]
    car \(f\) est paire.\\~\\
    Par unicité d’écriture du développement limité à l’ordre\( n \in \N\) de \(f\) en \(0\), on en déduit :
    \[\forall k \in \interventierii{0}{n} , b_k = (-1)^kb_k\]
    ce qui donne, pour tous les \(k\) impairs, \(b_k = -b_k\) donc \(b_k = 0\). Les coefficients de tous les monômes impairs dans le développement limité de \(f\) en \(0\) sont donc nuls.\\~\\
    \conclusion la partie régulière du \(\mathcal{DL}_n(0)\) de \(f\) ne comporte que des monômes pairs.
\end{dem}

\begin{defprop}[Caractérisation de la continuité et la dérivabilité avec un développement limité]
    Soit \(f : I \to  \K\) une fonction et \(a\) un RÉEL appartenant à \(I\).
    \begin{enumerate}
    \item \(f\) est continue en \(a\) si, et seulement si, \(f\) admet un développement limité à l’ordre \(0\) en \(a\).\\
        Dans ce cas, on \(a : f (x) \underset{x\to a}{=} f (a) + o(1)\).
    \item \(f\) est dérivable en \(a\) si, et seulement si, \(f\) admet un développement limité à l’ordre \(1\) en \(a\).\\
        Dans ce cas, on a : \(f (x) \underset{x\to a}{=} f (a) + f '(a)(x - a) + o ((x - a))\) .
    \end{enumerate}
    \underline{ATTENTION}
Ce résultat n’est pas généralisable. Ainsi, la fonction \(f\) définie sur \(\R\) par \(f (x) = x^3 \sin\paren{\frac{1}{x}}\) si \(x\neq 0\) et \(f (0) = 0\) admet un \(DL_2(0)\) qui est \(f (x) = o(x2)\) mais n’est pas deux fois dérivable en \(0\).
\end{defprop}

\subsection{Opérations sur les développements limités}
Soit \(f : I \to  \K\) et \(g : I \to  \K\) des fonctions et \(a\) un RÉEL tel que \(a\) est point ou extrémité de \(I\).

\begin{defprop}[Combinaison linéaire]
    Si \(f\) et \(g\) admettent des \(DL_n(a)\) et \((\lambda , \mu) \in  \K^2\) alors la fonction \(\lambda f + \mu g\) admet un \(DL_n(a)\) dont la partie régulière est obtenue par combinaison linéaire des parties régulières des \(DL_n(a)\) de \(f \) et \(g\).
\end{defprop}

\begin{defprop}[Produit]
    Si \(f\) et \(g\) admettent des \(DL_n(a)\) alors la fonction \(f g\) admet un \(DL_n(a)\) dont la partie régulière peut s’obtenir par troncature à l’ordre \(n\) du produit des parties régulières des \(DL_n(a)\) de \(f\) et \(g\).\\
    \underline{Remarque}\\
    En pratique, la mise en facteur des termes prépondérants dans \(f (x)\) et \(g(x)\) permet de prévoir l’ordre des développements limités à utiliser pour obtenir la précision souhaitée pour le \(DL\) de \(f g\).
\end{defprop}

\begin{defprop}[Inverse/quotient]
    Si \(f\) admet un \(DL_n(a)\) et que \(\lim_{a} f = 0\) alors la fonction \(\frac{1}{1 - f}\) admet un \(DL_n(a)\) qui peut s’obtenir par troncature à l’ordre \(n\) de la composée à droite de la partie régulière du \(DL_n(0)\) de \(x \mapsto  \frac{1}{1 - x}\) par la partie régulière du \(DL_n(a)\) de \(f\).
    \underline{Remarques}
    \begin{itemize}
    \item Cette propriété, combinée à celle vue sur le produit, permet d’obtenir des développements limités pour des quotients de fonctions. Là encore, la mise en facteur des termes prépondérants au numérateur et au dénominateur est un préalable à tout calcul.
    \item Aucun résultat général sur la composition de développements limités n’est au programme.
    \end{itemize}
\end{defprop}
\subsection{Primitivation d’un développement limité}

\begin{theo}
    Soit \(f : I \to  \K\) une fonction et \(a\) un RÉEL appartenant à \(I\).\\
    Si \(f\) est dérivable sur \(I\) et si \(f '\) admet un développement limité d’ordre \(n \in  \N\) en \(a\) de la forme
    \[f '(x) \underset{x\to a}{=} c_0 + c_1(x - a) + \dots + c_n(x - a)^n + o ((x - a)^n)\]
    alors \(f\) admet un développement limité d’ordre \(n + 1\) en \(a\) qui est
    \[f (x) \underset{x\to a}{=} f (a) + \frac{c_0}{1} (x - a) + \frac{c_1}{2} (x - a)^2 + \dots + \frac{c_n}{n + 1}(x - a)^{n+1} + o (x - a)^{n+1}\]
\end{theo}

\begin{dem}
    Preuve dans le cas où \(f\) est à valeurs dans \(\R\)\\~\\
    Montrons que \(g(x) \underset{x \to a}{=} o ((x - a)^{n+1})\) avec la fonction g\( : x \mapsto f (x) - f (a) -\sum^n_{k=0}c_k \frac{(x - a)^{k+1}}{k + 1}\)\\~\\
    D’après les théorèmes généraux, \(g\) est dérivable sur \(I\) de dérivée \(g' : x \mapsto f '(x) -\sum^n_{k=0}c_k(x - a)^k.\)
    D’après l’hypothèse faite sur \(f '\), on a donc \(g'(x) \underset{x \to a}{=} o ((x - a)^n)\) . 
    \begin{itemize}
        \item Soit \(x \in I \inter \intervee{a}{\pinf}\). Alors \(g\) est continue sur le segment \(\intervii{a}{x}\), dérivable sur \(\intervee{a}{x}\) et à valeurs réelles. D’après l’égalité des accroissements finis, il existe donc \(c_x \in \intervee{a}{x}\) tel que \(\frac{g(x) - g(a)}{x - a} = g' (c_x)\).
        \item On peut faire de même avec \(x \in I \inter \intervee{\minf}{a}\) en travaillant sur \(\intervii{x}{a}\) et \(\intervee{x}{a}\).
    \end{itemize}

Ainsi, pour tout \(x \in I \pd \accol{a}\) , il existe \(c_x \in I \pd \accol{a}\) tel que \(\frac{g(x) - g(a)}{x - a} = g' (c_x) \) avec \(\abs{c_x - a} \leq \abs{x - a}\).\\~\\
On a donc \(c_x \underset{x \to a}{ \to} a\) (par théorème d’encadrement) et on peut écrire :
\[\frac{g(x) - g(a)}{(x - a)^{n+1}} = \frac{g(x) - g(a)}{x - a} \times \frac{1}{(x - a)^n} = \frac{g' (c_x)}{(x - a)^n} = \frac{g' (c_x)}{(c_x - a)^n} \times \frac{(c_x - a)^n}{(x - a)^n}\]
avec :
\begin{itemize}
    \item \(\frac{g' (c_x)}{(c_x - a)^n} \underset{x \to a}{ \to} 0\) par composition de limites, car \(c_x \underset{x \to a}{ \to} a\) et \(\frac{g' (t)}{(t - a)^n} \underset{x \to a}{\to} 0\)
    \item \(x \mapsto \abs{\frac{(c_x - a)^n}{(x - a)^n}}\) bornée (par \(1\)) sur voisinage de \(a\) privé de \(a\).
\end{itemize}
Ainsi \(\frac{g(x) - g(a)}{(x - a)^{n+1}} \underset{x \to a}{ \to} 0\) donc \(g(x) - g(a) = o ((x - a)^{n+1})\) puis \(g(x) = o ((x - a)^{n+1})\) car \(g(a) = 0\).\\~\\
En revenant à la définition de \(g\), cela donne \(f (x) - f (a) -\sum^n_{k=0}c_k\frac{(x - a)^{k+1}}{k + 1} \underset{x \to a}{=} o ((x - a)^{n+1})\) ce qui permet de conclure.\\
\conclusion \(f (x) \underset{x \to a}{=} f (a) + \sum^n_{k=0}c_k\frac{(x - a)^{k+1}}{k + 1} + o ((x - a)^{n+1})\) \\
\ie on peut toujours primitiver un DL terme à terme
\end{dem}

\begin{defprop}[Un exemple important]
    Pour tout \(n \in  \N\), la fonction \(x \mapsto  - \ln(1 - x)\) admet un \(DL_n(0)\) qui est :
    \[- \ln(1 - x) \underset{x \to 0}{=} x + \frac{1}{2}x^2 + \dots + \frac{1}{n}x^n + o(x^n)\]
\end{defprop}

\begin{theo}[ Formule de Taylor-Young]
    Soit \(f : I \to  \K\) une fonction et \(a\) un RÉEL appartenant à \(I\).\\
    Si \(f\) est de classe \(\mathcal{C}^n\) sur \(I\) alors \(f\) admet un développement limité à l’ordre \(n\) en \(a\) donné par :
    \[f (x) \underset{x\to a}{=} f (a) + \frac{f^{(1)}(a)}{1!} (x - a) + \frac{f^{(2)}(a)}{2!} (x - a)^2 + \dots + \frac{f^{(n)}(a)}{n!} (x - a)^n + o ((x - a)^n)\]
    ce qui peut s’écrire encore
    \[f (x) \underset{x\to a}{=} \sum_{n}^{k=0}\frac{f ^{(k)}(a)}{k!} (x - a)^k + o ((x - a)^kn) \]
    \underline{Remarques}\\
    \begin{itemize}
    \item Ce théorème donne une condition suffisante d’existence d’un \(DL\) à l’ordre \(n\) en \(a\) pour \(f\) .
    \item Ce n’est pas une condition nécessaire : \cf l’exemple de la fonction 
    \[f : x \mapsto \begin{cases}
        x^3 \sin\paren{\frac{1}{x}} &\text{ si }x\neq 0\\
        0 &\text{ si }x = 0 \end{cases}\]
    qui admet bien un \(DL\) à l’ordre \(2\) en \(0\) mais n’est pas de classe \(\mathcal{C}^2\) sur \(\R\)
    \end{itemize}

\end{theo}

\begin{dem}
    On peut procéder par récurrence sur \(n \in \N\) pour montrer la propriété suivante :
    \[\forall f \in \mathcal{C}^n(I, \K), f (x) \underset{x \to a}{=}\sum^n_{k=0}\frac{f ^{(k)}(a)}{k!}k! (x - a)^k + o ((x - a)^n)\] 
    \begin{itemize}
        \item \underline{Initialisation} : \\
            On a déjà vu que : \(\forall f \in \mathcal{C}^0(I, \K), f (x) \underset{x \to a}{=} f (a) + o(1)\) (par continuité de \(f\) en \(a\)) donc la propriété est vraie au rang \(0\).
        \item \underline{Hérédité} :\\ 
        on suppose qu’il existe un entier \(n \in \N\) tel que la propriété soit vraie au rang \(n\).\\
        On considère une fonction \(f \in \mathcal{C}^{n+1}(I, \K)\). \\
        On peut appliquer l’hypothèse de récurrence à \(f '\) car \(f '\) appartient à \(\mathcal{n}(I, \K)\). Cela donne :
            \[f '(x) \underset{x \to a}{=} \sum^n_{k=0}\frac{(f ')^{k} (a)}{k!} (x - a)^k + o ((x - a)^n) \underset{x \to a}{=} \sum^n_{k=0}\frac{f ^{k+1}(a)}{k!} (x - a)^k + o ((x - a)^n) .\]
        Par théorème de primitivation des développements limités, on en déduit que :
            \[f (x) \underset{x \to a}{=} f (a) +\sum^n_{k=0} \frac{f^{(k+1)}(a)}{k!} \frac{(x - a)^{k+1}}{k + 1} + o ((x - a)^{n+1})\]
            \[f (x) \underset{x \to a}{=} f (a) + \sum^n_{k=0}\frac{f ^{(k+1)}(a)}{(k + 1)!} (x - a)^{k+1} + o ((x - a)^{n+1})\]
            ce qui donne après changement d’indice
            \[f (x) \underset{x \to a}{=} f (a) + \sum^{n+1}_{k=1}\frac{f ^{(k)}(a)}{k!} (x - a)^k + o ((x - a)^{n+1}) \underset{x \to a}{=} \sum^{n+1}_{k=0}\frac{f ^{(k)}(a)}{k!} (x - a)^k + o ((x - a)^{n+1})\]
        La propriété est donc vraie au rang \(n + 1\).\\
    \end{itemize}
    \conclusion : par principe de récurrence, la propriété est vraie pour tout entier naturel \(n\)
\end{dem}

\subsection{Développements limités usuels}
\begin{defprop}
    Soit \(n \in \N\)
    \begin{enumerate}
        \item \(\exp(x) \underset{x \to 0}{=} \sum_{k=0}^n \frac{1}{k!}x^k + o(x^n)\)
        \item \(\sh(x) \underset{x \to 0}{=} \sum_{k=0}^n \frac{1}{(2k+1)!}x^{2k+1} + o(x^{2n+1})\)
        \item \(\ch(x) \underset{x \to 0}{=} \sum_{k=0}^n \frac{1}{(2k)!}x^{2k} + o(x^{2n})\)
        \item \(\sin(x) \underset{x \to 0}{=} \sum_{k=0}^n \frac{(-1)^{k}}{(2k+1)!}x^{2k+1} + o(x^{2n+1})\)
        \item \(\ch(x) \underset{x \to 0}{=} \sum_{k=0}^n \frac{(-1)^{k}}{(2k)!}x^{2k} + o(x^{2n})\)
        \item \(\frac{1}{1-x} \underset{x \to 0}{=} \sum_{k=0}^n x^{2} + o(x^{n})\)
        \item \(\ln (1+x) \underset{x \to 0}{=} \sum_{k=0}^n \frac{(-1)^{k-1}}{k}x^k + o(x^{n})\)
        \item \(\paren{1+x}^{\alpha}\underset{x \to 0}{=} 1+ \sum_{k=0}^n \frac{\alpha (\alpha-1) \dots (\alpha -(k-1))}{k!} x^k + o(x^n) \text{ pour tout } \alpha \in \R \pd \N\)
        \item \(\arctan(x) \underset{x \to 0}{=} \sum_{k=0}^n \frac{(-1)^{k}}{(2k+1)}x^{2k+1} + o(x^{2n+1})\)
        \item \(\tan(x) \underset{x \to 0}{=} x + \frac{1}{3}x^3 + o(x^3)\)
    \end{enumerate}
\end{defprop}
\subsection{Application des développements limités à l’étude locale d’une fonction}

\begin{defprop}[Calcul d’équivalents ou de limites]
    La connaissance de DL permet de déterminer des limites de manière rapide en cas d’indétermination.
\end{defprop}

\begin{defprop}[Position relative d’une courbe et de sa tangente]
    Soit \(f\) une fonction définie sur \(I\), à valeurs RÉELLES et \(a\) un REEL appartenant à \(I\).\\
    Si \(f\) admet un développement limité à l’ordre\( n \geq 2\) en \(a\) qui s’écrit
    \[f (x) \underset{x\to a}{=} b_0 + b_1 (x - a) + \dots + b_n (x - a)^n + o ((x - a)^n)\]
    alors \(f\) est dérivable en \(a\) avec \(b_0 = f (a)\) et \(b_1 = f '(a)\) donc 
    \[f (x) - (f (a) + f '(a) (x - a)) \underset{x\to a}{=} b_2 (x - a)^2 + \dots + b_n (x - a)^n + o ((x - a)^n)\]
    Si de plus, il existe \(p \in  \N\) tel que \(2 \leq p \leq n\) et \(b_p\neq 0\), on obtient alors :
    \[f (x) - (f (a) + f '(a) (x - a)) \underset{x \to a}{\sim} b_p(x - a)^p\]
    Cela permet de connaître, au voisinage de \(a\), le signe de f\( (x)-(f (a) + f '(a) (x - a))\) donc les positions relatives de la courbe de \(f\) et de sa tangente au point \((a, f (a))\) dans un repère du plan.
\end{defprop}

\begin{defprop}[Condition nécessaire/suffisante d’existence d’un extremum local]
    Soit \(f\) une fonction définie sur \(\)I, à valeurs RÉELLES et \(a\) un point de \(I\) mais pas extrémité de \(I\).
    \begin{itemize}
        \item Si \(f\) a un extremum local en \(a\) avec \(f\) dérivable en \(a\) alors \(f '(a) = 0\) \ie \(a\) est point critique de \(f\) .        
        \item Si \(a\) est point critique de \(f\) et si \(f\) admet un développement limité à l’ordre \(n \geq 2\) en \(a\) qui s’écri
    \end{itemize}
    \[f (x) \underset{x\to a}{=} f (a) + b_p (x - a)^p + \dots + b_n (x - a)^n + o ((x - a)^n)\]
    avec \(p \in  \N\) tel que \(2 \leq p \leq n\) et \(b_p\neq 0\) alors
    \[f (x) - f (a) \underset{x \to a}{\sim} b_p(x - a)^p\]
    \(f (x) - f (a)\) est donc de signe constant localement au voisinage de \(a\) uniquement pour \(p\) pair.\\
    Dans ce cas, \(f\) admet un extremum local en \(a\) qui vaut \(f (a)\).
\end{defprop}

\begin{defprop}[Détermination d’asymptotes]
    \begin{itemize}
        \item Pour étudier une fonction \(f\) au voisinage de \(\minf\) (ou \(\pinf\)), on peut commencer par calculer un développement limité de la fonction \(g : t \mapsto  f\paren{1}{t}\) en \(0^-\) (ou \(0^+\)).
        \item En revenant à \(f\) avec \(f : x \mapsto  g\paren{1}{x}\), on obtient alors un développement asymptotique de \(f\) au voisinage de \(\minf\) (ou \(\pinf\)) qu’on peut utiliser pour déterminer d’éventuelles asymptotes à la courbe de \(f\)= dans un repère.
    \end{itemize}
\end{defprop}


\chapter{Espaces Vectoriels}

\minitoc

\chapter{Espaces vectoriels de dimension finie}

\minitoc

Dans ce chapitre, \(\K\) désigne le corps \(\R\) ou \(\C\).

\section{Existence de bases}
\subsection{Définition}
\begin{defi}
    On dit qu’un \(\K\)-espace vectoriel \(E\) est de dimension finie s’il admet une famille génératrice finie. \\
    \underline{Exemples} : \(\K^n\), \(\K_n \croch{X}\) , \(\M{n,p}\) sont de dimension finie mais pas \(\K \croch{X}\).
\end{defi}

\subsection{Algorithme de construction de bases}

\begin{defprop}
    Soit \((n, p) \in (\Ns)^2\) avec \(p \leq n\) et \((x_i)_{1\leq i\leq n}\) une famille d’un \(\K\)-espace vectoriel \(E\) de dimension finie.\\
    Si la famille \((x_1, \dots , x_p)\) est libre et si la famille \( (x_1, \dots , x_p, x_{p+1}, \dots , x_n)\) est génératrice de E alors E admet une base qui contient à la fois :
    \begin{itemize}
        \item tous les vecteurs \(x_1, \dots , x_p\) ;
        \item certains vecteurs parmi les vecteurs \(x_{p+1}, \dots , x_n\).
    \end{itemize}

\end{defprop}
\begin{dem}
    Soit \(\mathcal{F} = (x_1, \dots , x_p, x_{p+1}, \dots , x_n)\) une famille génératrice de \(E\) telle que la famille \((x_1, \dots , x_p)\) est libre.\\~\\
    \underline{Algorithme en langage naturel}\\~\\
    \(\cal{L} \leftarrow  \accol{x_1, \dots , x_p}\)\\
    Pour tout entier \(k\) allant de \(p + 1\) à \(n\), faire : \\
    \qquad Si \(L \union \accol{x_k}\) libre alors \(L \leftarrow L \union _accol{x_k}\) \\~\\
    Cet algorithme termine et donne une partie \(L\) libre composée des vecteurs \(x_1, \dots , x_p\) et de certains vecteurs parmi les vecteurs \(x_{p+1}, \dots , x_n\).\\~\\
    Notons \(\cal{B}\) la famille libre ainsi obtenue après balayage et montrons qu’elle engendre \(E\).
    \begin{itemize}
        \item Par construction de \(\cal{B}\), tout vecteur de \(\cal{F}\) est combinaison linéaire de vecteurs de \(\cal{B}\) donc appartient à \(\Vect{B}\).
        \item Par stabilité des sous-espaces vectoriels de \(E\) par combinaison linéaire, on a : \(\Vect{\cal{F}} \subset \Vect{\cal{B}}\).
        Ainsi \(E \subset \Vect{\cal{B}}\) (car \(\cal{F}\) engendre \(E\)) puis \(E = \Vect{\cal{B}}\) ce qui prouve que \(\cal{B}\) engendre \(E\).
    \end{itemize}

Conclusion : \(\cal{B}\) est une famille libre et génératrice de E donc c’est une base de E.
\end{dem}

\subsection{Théorèmes}
\begin{theo}[Base extraite]
    De toute famille génératrice d’un \(\K\)-espace vectoriel \(E\) de dimension finie différent de \(\accol{0_E}\), on peut extraire une base finie de \(E\).
\end{theo}
\begin{dem}
    \begin{enumerate}
        \item Soit \(\cal{F}\) une famille génératrice de \(E\).\\~\\
        Comme \(E\) est de dimension finie, on peut extraire de \(\cal{F}\) une famille finie génératrice \(\cal{F}'\) de \(E\).\\~\\
        En effet,\\
        par définition de la dimension finie, \(E\) admet une partie génératrice finie \(\cal{G}\). Comme \(\cal{F}\) engendre \(E\),tout vecteur \(x\) de \(\cal{G}\) est en particulier combinaison linéaire de \(\cal{F}\), autrement dit il existe un nombre fini de vecteurs de \(\cal{F}\) dont \(x\) est la combinaison linéaire. Comme \(\cal{G}\) est elle-même une partie finie,on en déduit alors qu’on peut extraire de \(\cal{F}\) une famille finie \(\cal{F}'\) telle que \(\Vect{\cal{G}} \subset \Vect{\cal{F}'}\).Vu le caractère générateur de \(\cal{G}\), on a \(E = \Vect{\cal{G}}\) donc \(E \subset \Vect{\cal{F}'}\) et enfin \(E = \Vect{\cal{F}'}\) car \(\Vect{\cal{F}'}\) est un sous-espace vectoriel de \(E\). La famille finie \(\cal{F}'\) extraite de \(\cal{F}\) engendre donc \(E\).\\~\\
        Comme \(E\) n’est pas réduit à \(\accol{0_E }\), la famille \(\cal{F}'\) contient un vecteur non nul \(x_1\) ce qui impliqueque la famille \((x_1)\) est une famille libre de \(E\). D’après l’algorithme de construction de bases, on peut alors construire une base de \(E\) qui contient \( x_1\) et certains vecteurs de la famille finie \(\cal{F}'\) donc de la famille génératrice F.\\~\\
        \conclusion \(E\) a une base finie obtenue par extraction de vecteurs d’une famille génératrice.
        \item Soit \(\cal{L}\) une famille libre de \(E\).\\
        Comme \(E\) est de dimension finie alors \(\cal{L}\) finie par propriété et \(E\) admet une famille génératrice finie, notée \(\cal{F}\), par définition. En concaténant les familles \(\cal{L}\) et \(\cal{F}\), on obtient une nouvelle famille finie génératrice de \(E\) (car surfamille de la famille génératrice de \(E\)). Par l’algorithme de construction de bases, on obtient alors une base finie de \(E\) composée des vecteurs de la famille libre \(\cal{L}\) et de certains vecteurs de la famille génératrice \(\cal{F}\).\\
        \conclusion \(E\) a une base finie obtenue par complétion d’une famille libre.
    \end{enumerate}
\end{dem}

\begin{theo}[Base incomplète]
    Toute famille libre (finie) d’un \(\K\)-espace vectoriel \(E\) de dimension finie différent de \(\accol{0_E}\) peut être complétée en une base finie de \(E\).
\end{theo}
\begin{theo}[Existence de bases en dimension finie]
    Tout \(\K\)-espace vectoriel \(E\) de dimension finie différent de \(\accol{0_E}\) admet des bases finies.
\end{theo}

\section{Dimension d’un espace vectoriel}

\subsection{Propriétés préliminaires}
\begin{defprop}[Cardinal des familles libres en dimension finie]
    Si \(E\) est un \(\K\)-espace vectoriel de dimension finie non réduit à \(\accol{0_E}\) et \(\mathcal{G}\) une famille génératrice finie de \(E\) à \(n\) élements alors toute famille libre \(\mathcal{L}\) de \(E\) a au plus \(n\) vecteurs :
    \[\Card{\mathcal{L}} \leq \Card{\mathcal{G}} \]
    \underline{Remarque}:\\
    Dans le théorème de la base incomplète, on peut donc remplacer "famille libre finie" par "famille libre".
\end{defprop}
\begin{dem}
    On note \(\cal{G}\) une partie génératrice de \(E\) ayant \(n\) vecteurs et on raisonne par l’absurde.\\~\\
    \begin{itemize}
        \item On suppose qu’il existe une partie libre \(\cal{L}\) de vecteurs de \(E\) ayant \(n + 1\) éléments.\\~\\
        Pour \(k \in \interventierii{0}{n}\), on note \(P(k)\) la propriété suivante :\\
        "il existe une partie génératrice \(\cal{G}_k\) de \(E\) comportant \(k\) éléments de \(\cal{L}\) et \(n - k\) éléments de \(\cal{G}\)".\\~\\
        \underline{Montrons que, pour tout \(k \in \interventierii{0}{n}, P(k)\) est vraie.}
        \begin{itemize}
            \item \(P(0)\) est vraie car la partie \(\cal{G}_0 = \cal{G}\) convient.\\
            \item Soit \(k \in \interventierii{0}{n-}\) tel que \(P(k)\) est vraie. Montrons que \(P(k + 1)\) est vraie.\\
             Par hypothèse de récurrence, il existe \(\cal{G}_k = \accol{l_1, \dots , l_k, g_1, \dots , g_{n-k}}\) partie génératrice de \(E\) telle que \(\forall i \in \interventierii{1}{n-k} , l_i \in \cal{L}\) et \(\forall i \in \interventierii{1}{n-k} , g_i \in \cal{G}\)\\~\\
            Comme la partie \(\cal{L}\) comporte \(n + 1\) éléments et que l’entier \(k\) est inférieur ou égal à \(n - 1\), il existe un vecteur \(l\) de \(\cal{L}\) qui est différent des vecteurs \(l_1, \dots , l_k\). Ce vecteur \(l\) appartient à \(E\) donc, par définition de \(\cal{G}_k\), il existe une famille de scalaires \((\alpha_1, \dots , \alpha_k, \beta_1, \dots , \beta_{n-k}) \in \K^n\) telle que
            \[l = \sum^k_{i=1} \alpha_il_i + \sum^{n-k}_{i=1} \beta_ig_i\]
            La partie \(\accol{l_1, \dots , l_k, l}\) est incluse dans la partie libre \(\cal{L}\) donc elle est libre et par conséquent, le vecteur \(l\) n’est pas combinaison linéaire de \(\accol{l_1, \dots , l_k}\) . Ainsi la famille de scalaires \((\beta_1, \dots , \beta_{n-k})\) est différente de \((0, \dots , 0)\) autrement dit il existe \(i \in \interventierii{1}{n-k}\) tel que \(\beta_i\) est non nul. Quitte à rénuméroter les \(\beta_i\), on peut supposer que \(\beta_{n-k}\) est non nul ce qui permet d’écrire
            \[g_{n-k} = \frac{1}{\beta_{n-k}}l - \sum^{k}_{i=1} \frac{\alpha_i}{\beta_{n-k}}li - \sum^{n-1-k}_{i=1}\frac{\beta_i}{\beta_{n-k}}g_i\]
            En notant \(l_{k+1} = l\) et \(\cal{G}_{k+1} = \accol{l_1, \dots , l_k, l_{k+1}, g_1, \dots , g_{n-k-1}}\), cela prouve que le vecteur \(g_{n-k}\) appartient à \(\Vect{\cal{G}_{k+1}}\). Comme par ailleurs, \(\cal{G}_k \pd \accol{g_{n-k}}\) est inclus dans \(\cal{G}_{k+1}\), on en déduit que \(\Vect{\cal{G}_k} \subset \Vect{\cal{G}_{k+1}}\) puis \(E \subset \Vect{\cal{G}_{k+1}}\) (car \(\cal{G}_k\) est génératrice de \(E\)) et enfin que \(E = \Vect{\cal{G}_{k+1}}\).\\~\\
            Autrement dit, la partie \(\cal{G}_{k+1}\) est génératrice de \(E\) et comporte \(k + 1\) éléments de \(\cal{L}\) et \(n - (k + 1)\) éléments de \(\cal{G}\). La propriété \(P(k + 1)\) est donc vraie.
        \end{itemize}
        Par théorème de récurrence, la propriété \(P(k)\) est donc vraie pour tout entier \(k\) de \(\interventierii{0}{n}\).\\~\\
        En particulier, \(P(n)\) est vraie. Il existe donc une partie génératrice de \(E\) composée de \(n\) vecteurs de \(\cal{L}\). Comme \(\cal{L}\) comporte \(n + 1\) vecteurs, l’un des vecteurs de \(\cal{L}\) est donc combinaison linéaire des \(n\) autres vecteurs ce qui contredit le caractère libre de\( \cal{L}\).\\
        \conclusion : il n’existe pas de parties libres à \(n + 1\) éléments dans \(E\).
        \item On suppose qu’il existe une famille libre de \(E\) ayant un nombre de vecteurs supérieur ou égal à \(n + 1\).
        Toute sous-famille de cette famille libre est alors libre par propriété. En particulier, toute sous-famille à \(n + 1\) éléments de cette famille est libre ce qui est faux d’après la propriété démontrée ci-dessus.\\
        \conclusion il n’existe pas de parties libres à plus de \(n + 1\) éléments dans E.
    \end{itemize}
    \conclusion : Si \(E\) a une famille génératrice à \(n\) vecteurs alors les familles libres de \(E\) ont au plus \(n\) vecteurs
\end{dem}
\begin{defprop}[Cardinal des bases en dimension finie]
    Si \(E\) est un \(\K\)-espace vectoriel de dimension finie non réduit à \(\accol{0_E}\) alors :
    \begin{itemize}
        \item toutes les bases de \(E\) sont finies ;
        \item toutes les bases de \(E\) ont le même nombre de vecteurs (appelé cardinal des bases).
    \end{itemize}
\end{defprop}

\subsection{Dimension}
\begin{defi}
    Soit \(E\) un \(\K\)-espace vectoriel de dimension finie.\\
    On appelle dimension de \(E\), et on note \(\dim{E}\), l’entier naturel défini de la manière suivante :
    \begin{itemize}
        \item Si \(E\neq \accol{0_E}\) alors \(\dim{E} = \Card{\mathcal{B}}\) où \(\mathcal{B}\) est une base quelconque de \(E\).
        \item Si \(E = \accol{0_E}\) alors \(\dim{E} = 0\).
    \end{itemize}
    \underline{Remarque}\\
    L’espace vectoriel nul est donc le seul \(\K\)-espace vectoriel de dimension finie égale à \(0\).
\end{defi}

\begin{defprop}[ Dimension d’espaces vectoriels déjà rencontrés]
    \begin{enumerate}
        \item \(\dim{\K^n} = n\)
        \item \(\dim{\K_n\croch{X}} = n + 1\)
        \item \(\dim{\M{n,p}} = n \times p\)
        \item \underline{Solutions d’une équation différentielle linéaire homogène d’ordre \(1\)}\\
            Soit \(I\) un intervalle de \(\R\), non vide et non réduit à un point, et \(a \in \mathcal{C(I, \K)}\).
            L’ensemble-solution sur \(I\) de \((E) : y' + a(t)y = 0\) est un \(\K\)-espace vectoriel de dimension \(1\).
        \item \underline{Solutions d’une équation différentielle linéaire homogène d’ordre \(2\) à coefficients constants}\\
            Soit \((a, b) \in \K^2\)
            L’ensemble-solution sur \(\R\) de \((E) : y'' + ay' + by = 0\) est un \(\K\)-espace vectoriel de dimension \(2\).
        \item \underline{Suites récurrentes linéaire homogène d’ordre \(2\) à coefficients constants}\\
            Soit \((a, b) \in \K \times \Ks\). L’ensemble des suites \((u_n)\) de \(\K^\N\) vérifiant\( \forall n \in \N, u_{n+2} + au_{n+1} + bu_n = 0\) est un \(\K\)-espace vectoriel de dimension \(2\).
    \end{enumerate}
\end{defprop}

\subsection{Caractérisation des bases en dimension finie}
\begin{defprop}
    Soit \(E\) un \(\K\)-espace vectoriel de dimension finie égale à \(n \in \Ns\).
    \begin{enumerate}
        \item Une famille libre de \(E\) est une base de \(E\) si, et seulement si, elle compte \(n\) vecteurs.
        \item Une famille génératrice de \(E\) est une base de \(E\) si, et seulement si, elle compte \(n\) vecteurs.
    \end{enumerate}
\end{defprop}

\subsection{Rang d’une famille finie de vecteurs}
\begin{defprop}
    On appelle rang d’une famille finie de vecteurs \((x_1, x_2, \dots , x_n)\) d’un \(\K\)-espace vectoriel \(E\), et on note \(\rg(x_1, x_2, \dots , x_n)\), la dimension de l’espace vectoriel engendré par cette famille :
    \[\rg (x_1, x_2, \dots , x_n) =  \dim{\Vect{ x_1, x_2, \dots , x_n}}.\]
    \underline{Remarque}\\
    On a \(\rg (x_1, x_2, \dots , x_n) \leq n\) avec égalité si, et seulement si, la famille \((x_1, x_2, \dots , x_n)\) est libre.
\end{defprop}

\subsection{Dimension d’un produit fini d’espaces vectoriels}
\begin{defprop}
    Si \(E_1, \dots , E_n\) sont des \(\K\)-espaces vectoriels de dimension finie alors le \(\K\)-espace vectoriel \(E_1 \times\dots\times E_n\) est de dimension finie avec
    \[\dim E_1 \times \dots \times E_n = \dim E_1 + \dots + \dim E_n\]
    \underline{Remarque}\\
    Si, pour tout \(j \in \interventierii{1}{n}\), on note, \(\mathcal{B}_j\) une base de \(E_j\) alors une base de \(E_1 \times \dots \times E_n\) est la concaténation des familles
    \[\paren{\paren{e, 0_{E_2} , \dots , 0_{E_n} }_{e\in \mathcal{B}_1} , \paren{0_{E_1}, e , \dots , 0_{E_n} }_{e\in \mathcal{B}_2} , \dots , \paren{0_{E_1}, \paren{0_{E_2}} , \dots , e }_{e\in \mathcal{B}_n}}\]
\end{defprop}

\section{Sous-espaces vectoriels en dimension finie}
\subsection{Propriétés}
\begin{defprop}[Dimension d’un sous-espace vectoriel]
    Si \(E\) est un \(\K\)-espace vectoriel de dimension finie alors
    \begin{enumerate}
        \item tout sous-espace vectoriel \(H\) de \(E\) est de dimension finie inférieure ou égale à celle de \(E\).
        \item un sous-espace vectoriel \(H\) de \(E\) est égal à \(E\) si, et seulement si, sa dimension est égale à celle de \(E\).
    \end{enumerate}
    \underline{Remarque}\\
    Une base de \(E\) obtenue en complétant une base d’un sous-espace vectoriel \(H\) de \(E\) est dite base de \(E\) adaptée à \(H\).
\end{defprop}

\begin{dem}
    On suppose que \(E\) est de dimension finie.\\~\\
    \begin{itemize}
        \item Dans le cas où \(E\) est de dimension nulle, les deux résultats sont immédiats car on a \(E = \accol{0_E }\) donc le seul sous-espace vectoriel de \(E\) est \(H = \accol{0_E }\) qui est de dimension finie nulle.
        \item On se place dans le cas où \(E\) est de dimension \(n\) avec \(n \geq 1\) autrement dit le cas où \(E\neq \accol{0_E }\).
        \begin{itemize}
            \item Dans le cas \(H = \accol{0_E }\), les deux résultats sont immédiats car \(H\) est de dimension finie égale à \(0\) donc strictement inférieure à \(n\).
            \item On se place dans le cas \(H\neq {0_E }\).
            Alors \(H\) contient un vecteur \(x_1\) différent de \(0_E\) donc \(\accol{x_1}\) est une partie libre de \(H\) (et de \(E\)).\\~\\
            \underline{Algorithme en langage naturel}\\
            \qquad \(\cal{L}_1 \leftarrow \accol{x_1}\)
            Tant que \(\cal{L}_k\) n’est pas une partie génératrice de \(H\),\\
            faire \(\cal{L}_{k+1} \leftarrow \cal{L}_k \union \accol{x_{k+1}}\) avec \(x_{k+1}\) un vecteur de \(H \pd \Vect{\cal{L}_k}\).\\~\\

            Par construction, les \(\cal{L}_k\) sont des parties libres de \(E\) donc de cardinal inférieur ou égal à \(n\) (qui est la dimension de \(E\)) et contiennent \(k\) vecteurs de \(H\). L’algorithme se termine donc (sinon la suite des cardinaux des parties \(\cal{L}_k\) serait une suite d’entiers croissante et majorée par \(n\) donc stationnaire ce qui contredirait sa stricte monotonie). Ainsi, il existe un entier \(k_0\) dans \(\interventierii{1}{k}\) tel que \(\cal{L}_{k_0}\) est une partie génératrice et libre de \(H\) donc une base de \(H\) de cardinal \(k_0 \leq n\).\\~\\
            Par définition, \(H\) est donc de dimension finie avec \(\dim H \leq \dim E\).\\~\\
            Par ailleurs, si \(\dim H = \dim E\) alors, par définition, toute base \(\cal{B}_H\) de \(H\) a pour cardinal la dimension de \(E\). Comme \(H\) est un sous-espace vectoriel de \(E\), \(\cal{B}_H\) est donc une famille libre de \(E\) qui a pour cardinal la dimension de \(E\). Par caractérisation des bases, \(\cal{B}_H\) est donc une base de \(E\) ce qui implique que \(E = \Vect{\cal{B}_H}\) donc que \(E = H\) puisque \(\cal{B}_H\) est base de \(H\).\\
            \conclusion tout sous-espace vectoriel \(H\) de \(E\) est de dimension finie telle que \(\dim H \leq \dim E\) avec égalité des dimension si, et seulement si, \(H\) est égal à \(E\).
        \end{itemize}
    \end{itemize}
\end{dem}

\begin{defprop}[Egalité de sous-espaces vectoriels en dimension finie]
    Soit \(H\) et \(G\) deux sous-espaces vectoriels d’un \(\K\)-espace vectoriel \(E\) de dimension finie.\\
   \( H = G\) si, et seulement si, \(H \subset G\) et \(\dim H = \dim G\).
\end{defprop}

\subsection{Dimension d’une somme (formule de Grassmann)}
\begin{defprop}
    Soit \(H\) et \(G\) deux sous-espaces vectoriels d’un \(\K\)-espace vectoriel \(E\).\\
    Si \(H\) et \(G\) sont de dimension finie alors \(H + G\) est de dimension finie avec
    \[\dim H + G = \dim H + \dim G - dim H \inter G\]
    \underline{Remarques}\\
    Si de plus la somme \(H + G\) est directe alors :
    \begin{itemize}
        \item \(\dim H \oplus G = \dim H + \dim G\) 
        \item la concaténation d’une base de \(H\) et d’une base de \(G\) donne une base de \(H \oplus G\) dite adaptée à la somme directe.
    \end{itemize}
        
\end{defprop}

\begin{dem}
    Soit \(E\) un \(\K\)-espace vectoriel (de dimension non nécessairement finie) et, \(H\) et \(G\) deux sous-espaces vectoriels de dimension finie de \(E\).
    \begin{itemize}
        \item Par propriété, \(H \inter G\) est sous-espace vectoriel de \(E\) inclus dans \(H\) donc sous-espace vectoriel de \(H\).\\
            Comme \(H\) est de dimension finie, \(H \inter G\) est de dimension finie donc admet une base finie \(\cal{B}_{H\inter G} = (e_i)_{i\in I}\).
            Comme \(\cal{B}_{H \inter G} = (e_i)_{i\in I}\) est une famille libre de \(H \inter G\) (donc de \(H\) et \(G\)), on peut la compléter en :
            \begin{itemize}
                \item une base finie \(\cal{B}_H\) de \(H\) en lui ajoutant des vecteurs \(h_j\) avec \(j \in J\) ;
                \item une base finie \(\cal{B}_G\) de \(G\) en lui ajoutant des vecteurs \(g_k\) avec \(k \in K\).
            \end{itemize}
            La famille \(\cal{B}\) obtenue en concaténant les familles \((e_i)_{i\in I}\) ,\((h_j )_{j\in J}\) et \((g_k)_{k\in K}\) est alors une famille génératrice du sous-espace vectoriel\( H + G\). En effet, tout élément de \(H + G\) s’écrit comme somme d’un élément de \(H\) et d’un élément de \(G\) donc comme combinaison linéaire de la famille obtenue en concaténant \(\cal{B}_H\) et \(\cal{B}_G\) et donc a fortiori comme combinaison linéaire de \(\cal{B}\).\\~\\
            Comme \(\cal{B}\) est une famille finie, \(H + G\) est alors de dimension finie par définition.
        \item Montrons que la famille \(\cal{B}\) ainsi créée est une base de \(H + G\).
            On considère une combinaison linéaire nulle de la famille \(\cal{B}\). Elle peut s’écrire sous la forme
            \[\sum_{i\in I}\alpha_ie_i + \sum_{j\in J}\beta_j h_j + \sum_{k\in K} \gamma_kg_k = 0_E \]
            avec \((\alpha_i)_{i\in I} ,(\beta_j )_{j\in J}\) et \((\gamma_k)_{k\in K}\) trois familles de scalaires.
            Alors 
            \[\underbrace{\sum_{j\in J}\beta_j h_j}_{\in H} = \underbrace{-\sum_{i\in I} \alpha_ie_i - \sum_{k\in K}\gamma_kg_k}_{\in G}\]
            donc le vecteur \(\sum_{j\in J} \beta_j h_j\) appartient à \(H \inter G\). Comme \((e_i)_{i\in I}\) est une base de \(H \inter G\), il existe donc une unique famille de scalaires \((\alpha'_i)_{i\in I}\) telle que
            \[\sum_{j\in J}\beta_j h_j = \sum_{i\in I}\alpha'_ie_i\]
            En réinjectant dans la toute première égalité, on trouve
            \[\sum_{i\in I}(\alpha_i + \alpha'_i)e_i + \sum_{k\in K}\gamma_kg_k = 0_E\]
            Comme \(\cal{B}_G\) est une base de \(G\), c’est une famille libre de \(E\) donc l’égalité précédente implique que
            \[\forall i \in I, \alpha_i + \alpha'_i = 0 \text{ et }\forall k \in K, \gamma_k = 0\]
            En réinjectant dans la trouve première égalité, on trouve 
            \[\sum_{i\in I}\alpha_ie_i + \sum_{j\in J}\beta_j h_j = 0_E\]
            Comme \(\cal{B}_H\) est une base de \(H\), c’est une famille libre de \(E\) donc l’égalité précédente implique que
            \[\forall i \in I, \alpha_i = 0\text{ et }\forall j \in J, \beta_j = 0\]
            avec de plus
            \[\forall k \in K, \gamma_k = 0\]
            En résumé, toute combinaison linéaire nulle de la famille \(\cal{B}\) a ses coefficients nuls donc, par définition, \(\cal{B}\) est libre. Comme \(\cal{B}\) est de plus une famille génératrice de \(H +G\), on en déduit que \(\cal{B}\) est une base de \(H +G\).\\~\\
            Par construction de la famille finie \(\cal{B}\), on a :
            \[\Card{\cal{B}} = \Card{\cal{B}_H} + \Card{\cal{B}_G} - \Card{\cal{B}_{H\inter G}}\]
            Par définition de la dimension d’un espace vectoriel, on a alors :
            \[\dim H + G = \dim H + \dim G - \dim H \inter G\]
    \end{itemize}
    \conclusion \(\dim H + G = \dim H + \dim G - dim H \inter G\)
\end{dem}

\subsection{Sous-espaces supplémentaires}

\begin{theo}[Théorème d’existence]
    Tout sous-espace vectoriel d’un \(\K\)-espace vectoriel de dimension finie possède au moins un supplémentaire.\\
    \underline{Remarques}\\
    \begin{itemize}
        \item Il n’y a pas unicité des supplémentaires dans un \(\K\)-espace vectoriel de dimension finie.\\
        Ainsi \(H = X^2\K_1 \croch{X}\) et \(G = (X^2 + 1)\K_1 \croch{X}\) sont des sous-espaces vectoriels différents de \(E = \K_3 \croch{X}\) tous deux supplémentaires de \(\K_1 \croch{X}\) dans l’espace vectoriel de dimension finie \(E\).
        \item L’existence d’un supplémentaire pour tout sous-espace vectoriel en dimension quelconque dépasse les ambitions du programme de MP2I-MPI.
    \end{itemize}
\end{theo}

\begin{dem}
    Soit \(E\) un \(\K\)-espace vectoriel de dimension finie et \(H\) un sous-espace vectoriel de \(E\).\\~\\
    \begin{itemize}
        \item Dans le cas où \(H = \accol{0_E }\), on a immédiatement\( E = \accol{0_E }\oplus E\) donc \(H\) admet comme supplémentaire \(E\).
        \item On se place dans le cas où \(H\neq \accol{0_E }\).
            Comme \(H\) est un sous-espace vectoriel de l’espace vectoriel de dimension finie \(E\), \(H\) est alors de dimension finie supérieure ou égale à \(1\). \(H\) admet donc une base finie \(\cal{B}_H = (e_1, \dots , e_p)\), avec \(p = \dim H\), que l’on peut compléter en une base \(\cal{B} = (e_1, \dots , e_p, e_{p+1}, \dots e_n)\) de \(E\) où \(n = \dim E\)
            Par unicité d’écriture de tout vecteur de \(E\) dans la base \(\cal{B}\) et par définition d’une somme directe, on en déduit que
            \[E = \Vect{e_1, \dots , e_p} \oplus \Vect{e_{p+1}, \dots , e_n}\]
            autrement dit, que
            \[E = H \oplus Vect (e_{p+1}, \dots , e_n)\]
            ce qui signifie que \(H\) admet \(\Vect{e_{p+1}, \dots , e_n}\) comme supplémentaire.
    \end{itemize}
    \conclusion tout sous-espace vectoriel d’un espace vectoriel de dimension finie a un supplémentaire.
\end{dem}

\begin{defprop}[Caractérisation des sous-espaces supplémentaires avec la dimension]
    Soit \(H\) et \(G\) deux sous-espaces vectoriels d’un \(\K\)-espace vectoriel E de dimension finie.\\
    \begin{enumerate}
        \item \(H\) et \(G\) sont supplémentaires si, et seulement si, \(\dim E = \dim H + \dim G\) et \(H + G = E\)
        \item \(H\) et \(G\) sont supplémentaires si, et seulement si, \(\dim E = \dim H + \dim G\) et \(H \inter G = \accol{0_E}\)
    \end{enumerate}
    \underline{Remarque}\\
        Lorsque \(H\) et \(G\) sont des sous-espaces supplémentaires de \(E\), la concaténation d’une base de \(H\) et d’une base de \(G\) donne une base de \(E\) dite adaptée à la décomposition en somme directe.
\end{defprop}

\chapter{Application Linéaire (1)}

\minitoc

Dans ce chapitre, \(\K\) désigne le corps \(\R\) ou \(\C\) et, sauf mention contraire \(E, F\) et \(G\)  sont des \(\K\)-espaces vectoriels.

\section{Généralités}
\subsection{Définitions - Notations}
\begin{defi}
    \begin{enumerate}
        \item Une application \(u\) de \(E\) dans \(F\) est dite application linéaire de \(E\) dans \(F\) si :
        \[\forall (\lambda , \mu ) \in  \K^2, \forall (x, y) \in  E^2, u(\lambda  x + \mu y) = \lambda  u(x) + \mu u(y).\]
        \item Les applications linéaires de \(E\) dans \(F\) constituent \(\cal{L}\paren{E,F}\).
        \item Les applications linéaires bijectives de \(E\) dans \(F\) sont dites \underline{isomorphismes} de \(E\) sur \(F\) .
        \item Les applications linéaires de \(E\) dans \(E\) sont dites \underline{endomorphismes} de \(E\) et constituent \(\cal{L}\paren{E}\).
        \item Les endomorphismes bijectifs de \(E\) sont dits \underline{automorphismes} de \(E\) et constituent \(\cal{GL}\paren{E}\).
    \end{enumerate}
    \underline{Remarque}\\
        Toute application \(u\) linéaire de \(E\) vers \(F\) est un morphisme de groupes additifs donc \(u (0_E ) = 0_F\) .
\end{defi}
\subsection{Opérations}
\begin{defprop}
    \begin{enumerate}
        \item La combinaison linéaire de deux applications linéaires est une application linéaire.
        \item La composée de deux applications linéaires est une application linéaire.
        \item La bijection réciproque d’un isomorphisme est un isomorphisme.
    \end{enumerate}
\end{defprop}
\subsection{Structures algébriques des ensembles d’applications linéaires}
\begin{defprop}
    \begin{enumerate}
        \item \(\paren{\cal{L}\paren{E,F}, +, .}\) est un \(\K\)-espace vectoriel car sous-espace vectoriel du \(\K\)-espace vectoriel \(\cal{F}\paren{E, F }\).
        \item \(\paren{\cal{L}\paren{E}, +, \circ }\) est un anneau dont l’élément neutre pour la loi \(\circ \) est \(\id{E}\) .
        \item \(\paren{\cal{GL}\paren{E}, \circ }\) est un groupe dit groupe linéaire.
    \end{enumerate}
\end{defprop}

\subsection{Image directe et image réciproque d’un sous-espace vectoriel}
\begin{defprop}
    Si \(u\) est une application linéaire de \(E\) dans \(F\) alors
    \begin{enumerate}
        \item l’image directe par \(u\) d’un sous-espace vectoriel de \(E\) est un sous-espace vectoriel de \(F\) ;
        \item l’image réciproque par \(u\) d’un sous-espace vectoriel de \(F\) est un sous-espace vectoriel de \(E\).
    \end{enumerate}
\end{defprop}
\subsection{Image et noyau d’une application linéaire}
\begin{defprop}[Structures algébriques]
    Si \(u\) est une application linéaire de \(E\) dans \(F\) alors :
    \begin{enumerate}
        \item \(\im u = u(E) = \accol{y \in  F \tq \exists x \in  E, y = u(x)}\) est un sous-espace vectoriel de \(F\) , appelé image de \(u\) ;
        \item \(\ker u = u^{-1} \accol{0_F } = \accol{x \in  E \tq u(x) = 0_F }\) est un sous-espace vectoriel de \(E\), appelé noyau de \(u\).
    \end{enumerate}
\end{defprop}
\begin{defprop}[Famille génératrice de l’image d’une application linéaire]
    Si \(u \in  \cal{L}\paren{E,F}\) et \(E = \Vect{(x_i)_{i\in I} } \) alors \(\im(u) = \Vect{(u(x_i))_{i\in I} }\).
\end{defprop}
\begin{defprop}[Caractérisation des applications linéaires injectives/surjectives]
  \begin{enumerate}  
        \item Une application linéaire \(u\) de \(E\) dans \(F\) est injective si, et seulement si, \(\ker u = \accol{0_E }\).
        \item Une application (linéaire) \(u\) de \(E\) dans \(F\) est surjective si, et seulement si, \(\im u = F \).
    \end{enumerate}

        \underline{Remarques}\\
        Sans la linéarité, l’équivalence \(1\) n’est pas conservée alors que l’équivalence \(2\) est conservée.
\end{defprop}

\subsection{Rang d’une application linéaire}
\begin{defi}[Définition du rang d’une application linéaire]
    Une application linéaire \(u\) de \(E\) dans \(F\) est dite de rang fini si son image \(\im(u)\) est de dimension finie.\\
    Dans ce cas, la dimension de \(\im (u)\) est appelée rang de u et notée \(\rg(u)\) :
    \[\rg(u) = \dim (\im (u))\]
\end{defi}

\begin{defprop}[Conditions suffisantes de finitude du rang]
    \begin{enumerate}
        \item Si \(u : E \to F\) est linéaire et \(E\) de dimension finie alors \(u\) est de rang fini avec \(\rg(u) \leq \dim E\).
        \item Si \(u : E \to F\) est linéaire et \(F\) de dimension finie alors \(u\) est de rang fini avec \(\rg(u) \leq \dim F\) .
    \end{enumerate}
    \end{defprop}

\begin{defprop}[Rang d’une composée]
    Soit \(u \in  \cal{L}\paren{E,F}\) et \(v \in  \cal{L}\paren{F, G}\).\\
    \begin{enumerate}
        \item Si \(u\) et \(v\) sont de rang fini alors \(v \circ u\) est de rang fini et vérifie \(\rg (v \circ u) \leq \min(\rg(u), \rg(v))\).
        \item Invariance du rang par composition par un isomorphisme
        \begin{enumerate}
            \item si \(u\) est un isomorphisme et \(v\) de rang fini alors \(v \circ u\) est de rang fini et \(\rg (v \circ u) = \rg (v)\).
            \item si \(v\) est un isomorphisme et \(u\) de rang fini alors \(v \circ u\) est de rang fini et \(\rg (v \circ u) = \rg (u)\).
        \end{enumerate}
    \end{enumerate}
\end{defprop}
\begin{dem}
    \begin{itemize}
        \item On suppose que \(u\) et \(v\) sont de rang fini.\\
        \begin{itemize}
            \item Comme \(v\) est de rang fini alors \(\im v\) est de dimension finie. L’inclusion naturelle \(\im v\circ u \subset \im v\) donne alors que \(\im v \circ u\) est de dimension finie (comme sous-espace vectoriel d’un espace de dimension finie), donc que \(v \circ u\) est de rang fini, avec \(\rg (v \circ u) \leq \rg (v)\) par définition du rang.
            \item D’autre part, \(\im v \circ u = \accol{v \circ u(x) \tq x \in  E} = \accol{v (u(x)) \tq x \in  E}\) donc on a :
                \[\im v \circ u = \accol{v(y) \tq y \in  \im u}\]
            On s’intéresse alors à \(\tilde{v} = v_{| \Im u}\) autrement dit à l’application \(\tilde{v} : \im u \to G\) définie par
                \[\forall y \in  \im u, \tilde{v}(y) = v(y)\]
            \(\tilde{v}\) est linéaire (car \(v\) l’est) d’espace de départ, \(\im u\), de dimension finie (car \(u\) est de rang fini) donc, par propriété, \(\tilde{v}\) est de rang fini avec \(\rg (\tilde{v}) \leq \dim (\im u)\) c’est-à-dire \(\dim (\im \tilde{v}) \leq \dim (\im u)\).\\~\\
            Par définition de \(\tilde{v}\), on a donc \(\im v \circ u = \im \tilde{v}\) ce qui permet de conclure que \(\dim (\im v \circ u) \leq \dim (\im u)\) autrement dit que \(\rg (v\circ u) \leq \rg (u) \).\\~\\
            Ainsi \(\rg (v \circ u) \leq \rg (v)\) et \(rg (v \circ u) \leq \rg (u)\) donc \(\rg (v \circ u) \leq \min (\rg (u) , \rg (v))\)\\~\\
            \conclusion si \(u\) et \(v\) sont de rang fini alors \(v \circ u\) est de rang fini et \(\rg (v \circ u) \leq \min(\rg(u), \rg(v))\).
        \end{itemize}
        \item Invariance du rang par composition par un isomorphisme
        \begin{enumerate}
            \item On suppose que \(u\) est un isomorphisme et que \(v\) de rang fini. \\
                On a vu au \(1\) que \(\im v \circ u = \accol{v(y) \tq y \in  \im u}\) autrement dit que \(\im v \circ u = v (\im u)\) par définition de l’image directe d’un ensemble.\\~\\
                Or \(u\) est un isomorphisme de \(E\) vers \(F\) donc est en particulier surjectif ce qui donne \(\im u = F\) .\\
                Ainsi, \(\im v \circ u = v (F )\) avec \(v (F ) = \im v\), par définition de l’image de \(v\), linéaire de \(F\) dans \(G\).\\~\\
                On en déduit que \(\im v \circ u = \im v\) ce qui prouve que \(v \circ u\) est de rang fini (car \(v\) l’est) et que \(\rg (v \circ u) = \rg (v)\) .\\~\\
                \conclusion pour \(u\) isomorphisme et \(v\) de rang fini, on a \(v \circ u\) de rang fini et \(\rg (v \circ u) =\rg (v)\).
            \item On suppose que \(v\) est un isomorphisme et \(u\) de rang fini.
                On reprend les notations et une partie des résultats de la preuve du \(1\)
                \begin{itemize}
                    \item On a \(\im v \circ u = \im\tilde{v}(\star)\) avec \(\tilde{v} = v_{|\im u}\) qui est de rang fini car elle est linéaire sur un espace de dimension finie \(\im u\).Par conséquent, \(\im \tilde{v}\) est de dimension finie et donc \(\im v \circ u\) aussi vu l’égalité écrite. Autrement dit, \(v \circ u\) est de rang fini.
                    \item Déterminons une base de \(\im \tilde{v}\) (donc de \(\im v \circ u\))\\~\\
                    Soit \((y_1, \dots , y_n)\) une base de \(\im u\) (espace de départ de \(\tilde{v}\)) avec \(n = \dim \im u\).\\~\\
                    Alors, par propriété, \(\im \tilde{v} = \Vect{\tilde{v}(y_1), \dots , \tilde{v}(y_n)} \ie (\tilde{v}(y_1), \dots , \tilde{v}(y_n))\) engendre \(\im \tilde{v}\).\\~\\
                    De plus, \((\tilde{v}(y_1), \dots , \tilde{v}(y_n))\) est une famille libre de \(G\).\\~\\
                    En effet, pour \((\lambda_1, \dots , \lambda_n) \in  \K^n\) tel que \(\lambda_1\tilde{v}(y_1) + \dots + \lambda_n\tilde{v}(y_n) = 0_G\), on a\\~\\
                    \(\lambda_1v(y_1) + \dots + \lambda_nv(y_n) = 0_G\) donc, par linéarité de \(v\),\( v (\lambda_1y_1 + \dots + \lambda_ny_n) = 0_G\) ce qui prouve que\( \lambda_1y_1 + \dots + \lambda_ny_n\) appartient au noyau de \(v\). Comme \(v\) est un isomorphisme, \(v\) est en particulier linéaire injective donc son noyau est réduit à \(\accol{0_F }\). On en déduit que\( \lambda_1y_1 + \dots + \lambda_ny_n = 0_F\) puis, par liberté de \((y_1, \dots , y_n)\) , que \(\lambda_1 = \dots = \lambda_n = 0\) ce qui prouve la liberté de \((\tilde{v}(y_1), \dots , \tilde{v}(y_n))\) .\\~\\
                    Ainsi,\( (\tilde{v}(y_1), \dots , \tilde{v}(y_n))\) est une base de \(\im \tilde{v}\) donc \(\dim \im \tilde{v} = n = \dim \im u\) puis. On obtient alors : \(\dim \im v \circ u = \dim \im u\) autrement dit \(\rg (v \circ u) = \rg (u)\) .\\
                    \conclusion pour \(v\) isomorphisme et \(u\) de rang fini, on a \(v\circ u\) de rang fini et \(\rg (v\circ u) = \rg (u)\).
                \end{itemize}
        \end{enumerate}
    \end{itemize}
\end{dem}
\section{Des applications linéaires usuelles}
\subsection{Homothéties}
\begin{defi}
   Soit\( \lambda  \in  \K\).\\
    L’application \(h_{\lambda}  = \lambda \id{E}\) est un endomorphisme de \(E\) appelé homothétie de rapport \(\lambda\) .
\end{defi}

\subsection{Projections/projecteurs}
    On suppose ici que \(E_1\) et \(E_2\) sont des sous-espaces vectoriels supplémentaires de \(E\). Alors :
    \[\forall x \in  E, \exists!(x_1, x_2) \in  E_1 \times E_2, x = x_1 + x_2\].
\begin{defi}[Définition géométrique]
    L’application \(p : E \to E\) définie par
    \[\forall x \in  E, p(x) = x_1\]
    est appelée projection (ou projecteur) sur \(E_1\) parallèlement à \(E_2\).
\end{defi}
\begin{prop}[Propriétés]
    Si \(p\) est la projection sur \(E_1\) parallèlement à \(E_2\) alors :
    \begin{enumerate}
        \item \(p\) est un endomorphisme de \(E\) qui vérifie \(p \circ p = p\) ;
        \item \(E_1 = \im p = \ker (p - \id{E} ) = \accol{x \in  E \tq p(x) = x}\) ;
        \item \(E_2 = \ker p\).
    \end{enumerate}
\end{prop}

\begin{defprop}[Caractérisation algébrique]
    Une application \(p : E \to E\) est un projecteur de \(E\) si, et seulement si, \(p\) est linéaire et \(p^2 = p\).\\
    Dans ce cas, on a :
    \begin{enumerate}
        \item \(E = \im p \oplus \ker p\) ;
        \item \(p\) est la projection sur \(\im p = \ker (p - \id{E} )\) parallèlement à \(\ker p\) ;
        \item \(\forall x \in  E, x = p(x) + (x - p(x)) \text{ avec} \begin{cases}
            p(x) &\in  \im p\\
            x - p(x) &\in  \ker p 
        \end{cases}\)
    \end{enumerate}
\end{defprop}
\begin{defprop}[Remarques]
    \begin{enumerate}
        \item Le seul projecteur de \(E\) bijectif est \(\id{E}\) .
        \item On peut avoir \(E = \im u \oplus \ker u\) sans que l’endomorphisme \(u\) de \(E\) ne soit un projecteur.
    \end{enumerate}
\end{defprop}
\subsection{Symétries}
    On suppose ici que \(E_1\) et \(E_2\) sont des sous-espaces vectoriels supplémentaires de \(E\). Alors :
    \[\forall x \in  E, \exists!(x_1, x_2) \in  E_1 \times E_2, x = x_1 + x_2\]
\begin{defi}[Définition géométrique]
    L’application \(s : E \to E\) définie par
    \[\forall x \in  E, s(x) = x_1 - x_2\]
    est appelée symétrie par rapport à \(E_1\) parallèlement à \(E_2\).\\
    \underline{Remarque}\\
    Cette application vérifie \(s = 2p - \id{E}\) où \(p\) est la projection sur \(E_1\) parallèlement à \(E_2\).
\end{defi}

\begin{prop}
    Si \(s\) est la symétrie par rapport à \(E_1\) parallèlement à \(E_2\) alors :
    \begin{enumerate}
        \item \(s\) est un endomorphisme de \(E\) qui vérifie \(s \circ s = \id{E}\) ;
        \item \(E_1 = \ker (s - \id{E} ) = \accol{x \in  E \tq s(x) = x}\) ;
        \item \(E_2 = \ker (s + \id{E} ) = \accol{x \in  E \tq s(x) = -x}\).
    \end{enumerate}
\end{prop}

\begin{defprop}[Caractérisation algébrique]
    Une application \(s : E \to E\) est une symétrie de \(E\) si, et seulement si, \(s\) est linéaire et \(s^2 = \id{E}\).\\
    Dans ce cas :
    \begin{enumerate}
        \item \(E = \ker (s - \id{E} ) \oplus \ker (s + \id{E} )\) ;
        \item \(s\) est la symétrie par rapport à \(\ker (s - \id{E} )\) parallèlement à \(\ker (s + \id{E} )\) ;
        \item \(\forall x \in  E, x = \frac{1}{2} (x + s(x)) + \frac{1}{2} (x - s(x))\text{ avec }\begin{cases}
        \frac{1}{2} (x + s(x)) &\in  \ker (s - \id{E} )\\[10pt]
        \frac{1}{2} (x - s(x)) &\in  \ker (s + \id{E} )
        \end{cases}\)
    \end{enumerate}
\end{defprop}

\begin{defprop}[Bijectivité des symétries]
    Toute symétrie \(s\) de \(E\) est un automorphisme de \(E\) dont la bijection réciproque est \(s\).
\end{defprop}

\section{Isomorphismes}
\subsection{Détermination d’une application linéaire}
\begin{defprop}[Action sur une base]
    Si \((e_i)_{i\in I}\) est une base de \(E\) et \((f_i)_{i\in I}\) une famille de \(F\) alors il existe un unique \(u \in  \cal{L}\paren{E,F}\) tel que :
    \[\forall i \in  I, u(e_i) = f_i.\]
    Autrement dit, une application linéaire est entièrement déterminée par la connaissance de son action sur les vecteurs d’une base de son espace vectoriel de départ.
\end{defprop}

\begin{defprop}[Recollement]
    Si \(E_1\) et \(E_2\) sont des sous-espaces vectoriels supplémentaires de \(E\) avec \(u_1 \in  L(E_1, F )\) et \(u_2 \in  L(E_2, F )\) alors il existe un unique \(u \in  \cal{L}\paren{E,F}\) tel que :
    \[u_1 = u_{|E_1} \text{ et }u_2 = u_{|E_2} \]
Autrement dit, une application linéaire est entièrement déterminée par la connaissance de son action sur les vecteurs de deux sous-espaces vectoriels supplémentaires de son espace vectoriel de départ.
\end{defprop}

\subsection{Caractérisations de l’injectivité, la surjectivité ou la bijectivité}
\begin{defprop}
    Soit \(u\) une application linéaire de \(E\) dans \(F\) et \((e_i)_{i\in I}\) une base de \(E\).
    \begin{enumerate}
        \item \(u\) est surjective si, et seulement si, \((u(e_i))_{i\in I}\) est une famille génératrice de \(F\) .
        \item \(u\) est injective si, et seulement si, \((u(e_i))_{i\in I}\) est une famille libre de \(F\) .
        \item \(u\) est bijective si, et seulement si, \((u(e_i))_{i\in I}\) est une base de \(F\) .
    \end{enumerate}
\end{defprop}

\subsection{Applications linéaires entre espaces de même dimension finie}
\begin{defprop}[Caractérisation des isomorphismes]    
    Si \(u\) est une application linéaire entre deux \(\K\) espaces vectoriels de même dimension finie alors les trois propositions suivantes sont deux à deux équivalentes.
    \begin{enumerate}
        \item \(u\) est injective
        \item \(u\) est surjective
        \item \(u\) est bijective.
    \end{enumerate}
\end{defprop}

\begin{defprop}[Caractérisation des automorphismes]
    
    Si \(u\) est un endomorphisme de \(E\) avec \(E\) de dimension finie alors les trois propositions suivantes sont deux à deux équivalentes.
    \begin{enumerate}
        \item \(u\) est bijective.
        \item \(u\) est inversible à droite (c’est-à-dire qu’il existe \(v \in  \cal{L}\paren{E}\) tel que\( u \circ v = \id{E}\)).
        \item \(u\) est inversible à gauche (c’est-à-dire qu’il existe \(w \in  \cal{L}\paren{E}\) tel que \(w \circ u = \id{E}\)).
    \end{enumerate}
\end{defprop}

\subsection{Espaces vectoriels isomorphes}
\begin{defi}
    Deux espaces vectoriels \(E\) et \(F\) sont dits isomorphes s’il existe un isomorphisme de \(E\) vers \(F\) .
\end{defi}

\begin{defprop}[Caractérisation par la dimension]
    Si \(E\) est de dimension finie alors \(F\) est isomorphe à \(E\) si, et seulement si, \(E\) et \(F\) ont même dimension.
\end{defprop}
\begin{defprop}[dimension de \(\cal{L}\paren{E,F}\)]
    Si \(E\) et \(F\) sont de dimension finie alors \(\cal{L}\paren{E,F}\) l’est aussi et\( \dim (\cal{L}\paren{E,F}) = \dim(E) \times \dim(F )\)
\end{defprop}
\section{Théorème du rang}
\subsection{Théorème du rang (version géométrique)}
\begin{theo}
    Si \(u\) est une application linéaire de \(E\) vers \(F\) et si \(S\) est un supplémentaire de \(\ker(u)\) dans \(E\) alors l’application \(\tilde{u} : S \to \Im u\) définie par
   \[ \forall x \in  S, \tilde{u}(x) = u(x)\]
    est un isomorphisme.\\
    \underline{Remarques}\\
    \begin{itemize}
        \item L’image d’une application linéaire est donc isomorphe à tout supplémentaire de son noyau.
        \item On dit aussi que \(u\) induit un isomorphisme de \(S\) sur \(\im(u)\).
        \item On note parfois \(\tilde{u}\) de la manière suivante \(\tilde{u} = u^{|\im(u)}_{|S}\) en parlant de bi-restriction de \(u\).
    \end{itemize}
\end{theo}

\begin{theo}[Théorème du rang (version dimension finie)]
    Si \(u\) est une application linéaire de \(E\) vers \(F\) avec \(E\) de dimension finie alors
    \begin{enumerate}
        \item \(u\) est de rang fini ;
        \item \(\dim E = \dim \ker (u) + \rg(u)\).
    \end{enumerate}
    \underline{Remarques}\\
    \begin{itemize}
        \item la dimension finie de l’espace de départ sur lequel est définie l’application linéaire suffit ici.
        \item \underline{ATTENTION}\\
            Le théorème du rang n’implique pas l’égalité \(E = \ker u \oplus \im u\) \((\star)\). En effet,
        \begin{itemize}
            \item \((\star)\) peut n’avoir aucun sens si \(E\) et \(F\) sont distincts (sens de \(x + y\) avec \(x \in  E\) et \(y \in  F\) ?) ;
            \item \((\star)\) peut être fausse si \(E\) et \(F\) sont égaux (cf \(u \in  \cal{L} (\R \croch{X})\) définie par \(\forall P \in  \R \croch{X} , u(P ) = P '\)).
        \end{itemize}
    \end{itemize}
\end{theo}

\section{Formes linéaires et hyperplans}
\subsection{Formes linéaires}
\begin{defprop}
    Toute application linéaire de \(E\) vers \(\K\) est dite forme linéaire sur \(E\).\\
    \underline{Exemple}\\
    Si \(\cal{B} = (e_i)_{i\in I}\) est une base de \(E\) alors tout \(x\) de \(E\) s’écrit de manière unique sous la forme
    \[x = \sum_{i\in I}\underbrace{e^{\star}_i (x)}_{\in \K} e_i\]
    Les fonctions \(e^{\star}_i : E \to \K\) sont des formes linéaires sur \(E\) dite formes coordonnées relativement à \(B\).
\end{defprop}
\subsection{Hyperplans}
\begin{defi}
    Un sous-espace vectoriel de \(E\) est dit hyperplan s’il est le noyau d’une forme linéaire non nulle sur \(E\).
\end{defi}

\begin{defprop}[Caractérisations des hyperplans comme supplémentaires de droites]
    Soit \(H\) un sous-espace vectoriel de \(E\).\\
    \(H\) est un hyperplan de \(E\) si, et seulement si, \(H\) est supplémentaire d’une droite de \(E\).\\
    \underline{Remarque}\\
    Dans le cas où \(H\) est un hyperplan, pour toute droite \(D\) de \(E\) non contenue dans \(H\), on a : \(E = H \oplus D\).
\end{defprop}

\begin{dem}
    \begin{itemize}
        \item On suppose que \(H\) est un hyperplan de \(E\).\\
        Alors, par définition, il existe \(\phi\) dans \(\cal{L} (E, \K)\) avec \(\phi\neq 0_{\cal{L}(E,\K)}\) tel que \(H = \ker\phi\).\\~\\
        Comme \(\phi\) n’est pas nulle, il existe un vecteur \(a\) de \(E\) tel que \(\phi(a)\neq 0\) autrement dit tel que \(a \in  E \pd H\).\\
        Le sous-espace vectoriel \(D = \Vect{a}\) est alors une droite puisque a est différent de \(0_E\) .\\~\\
        \underline{Montrons que \(E = H \oplus D\)}\\
        Soit \(x \in  E\).
        \begin{itemize}
            \item \analyse on suppose qu’il existe \((x_H , x_D) \in  H \times D\) tel que \(x = x_H + x_D\).\\~\\
            Alors : \(\exists\lambda  \in  \K, x_D = \lambda a\) et \(x_H = x - x_D\) avec \(\phi(x_H ) = 0\) par définition de \(H\).
            Par linéarité de \(\phi\), cela donne \(\phi(x) - \lambda\phi(a) = 0\) puis \(\lambda  = \frac{\phi(x)}{\phi(a)}\) car \(\phi(a)\neq 0\).\\~\\
            On en déduit donc que le seul couple \((x_H , x_D)\) possible est \(\paren{x - \frac{\phi(x)}{\phi(a)} a, \frac{\phi(x)}{\phi(a)} a}\).
            \item \synthese : on a bien \(x =\paren{x - \frac{\phi(x)}{\phi(a)} a}+ \frac{\phi(x)}{\phi(a)}a\) avec \(\frac{\phi(x)}{\phi(a)} a \in  D\) (par définition de \(D\) car \(\frac{\phi(x)}{\phi(a)} \in  \K\)) et \(x - \frac{\phi(x)}{\phi(a)} a \in  H\) (par définition de \(H\) et linéarité de \(\phi\) car \(\phi\paren{x - \frac{\phi(x)}{\phi(a)} a}= \phi(x) - \frac{\phi(x)}{\phi(a)} \phi(a) = 0\)).
        \end{itemize}
        On en déduit que tout vecteur de \(E\) s’écrit de manière unique comme somme d’un élément de \(H\) et d’un élément de \(D\). Ainsi \(E = H \oplus D\) ce qui prouve que \(H\) est supplémentaire d’une droite de \(E\).
        \item On suppose que \(H\) est supplémentaire d’une droite de \(E\) que l’on note \(D = \Vect{a}\) avec \(a\neq 0_E\) .\\~\\
        Tout \(x \in  E\) s’écrit donc de manière unique sous la forme
        \[x = x_H + \lambda_a \text{ avec } x_H \in  H \text{ et }\lambda  \in  \K\]
        On définit alors l’application linéaire \(\phi : E \to \K\) par recollement en posant :
        \[\forall x_H \in  H, \phi(x_H ) = 0 \text{ et }\forall \lambda  \in  \K, \phi(\lambda a) = \lambda \]
        Alors, \(\phi\) est une forme linéaire (par définition), non nulle car \(\phi(a) = 1(\neq 0)\) et de noyau \(H\). En effet,
        \[\phi(x) = 0 \iff \phi(x_H + \lambda a) = 0 \iff \phi(x_H ) + \phi(\lambda a) = 0 \iff \lambda  = 0 \iff x = x_H \iff x \in  H\]
        On en déduit donc, par définition, que \(H\) est un hyperplan de \(E\).
    \end{itemize}
    \conclusion : \(H\) hyperplan de \(E\) si, et seulement si, \(H\) est supplémentaire d’une droite de \(E\).
\end{dem}

\subsection{Hyperplans en dimension finie}
    Sauf mention contraire, dans cette partie, \(E\) est de dimension finie non nulle notée \(n\).
\begin{defprop}[Caractérisations des hyperplans avec la dimension]
    Un sous-espace vectoriel \(H\) de \(E\) est un hyperplan de \(E\) si, et seulement si, \(\dim(H) = \dim(E) - 1\).
\end{defprop}

\begin{defprop}[Equations d’un hyperplan en dimension finie]
    Soit \(H\) un sous-espace vectoriel de \(E\) et \(B = (e_1, \dots , e_n)\) une base de \(E\).\\
    \(H\) est un hyperplan de \(E\) si, et seulement si, il existe \((a_1, a_2, \dots , a_n) \in  \K^n \pd \accol{(0, 0, \dots , 0)}\) tel que :
        \[x \in  H \iff a_1x_1 + a_2x_2 + \dots a_nx_n = 0\]
    avec \((x_1, \dots , x_n)\) la famille des coordonnées du vecteur \(x\) de \(E\) dans la base \(\cal{B}\).\\
    L’équation \(a_1x_1 + a_2x_2 + \dots a_nx_n = 0\) est dite équation de \(H\) dans la base \(\cal{B}\)\\
    \underline{Remarque}\\
    Dans ce cas, \(H = \ker \phi\) où \(\phi\) est la forme linéaire sur \(E\) définie par
    \[\phi : x \mapsto a_1x_1 + a_2x_2 + \dots a_nx_n\].
\end{defprop}

\begin{defprop}[Comparaison des équations d’un hyperplan en dimension finie]
    Deux formes linéaires non nulles sur \(E\) de même noyau sont proportionnelles.\\
    \underline{Remarque}\\ : 
    les équations d’un même hyperplan en dimension finie diffèrent donc à une constante multiplicative non nulle près.
\end{defprop}

\begin{defprop}[Hyperplans en dimension \(2\) et \(3\)]
    \begin{itemize}
        \item Les hyperplans d’un espace vectoriel \(E\) de dimension finie égale à \(2\) sont les droites de \(E\). Leurs équations dans une base de \(E\) sont de la forme \(ax + by = 0\) avec \((a, b) \in  \K^2 \pd \accol{(0, 0)}\) en notant \(x\) et \(y\) les coordonnées d’un vecteur de \(E\) dans la base considérée.
        \item Les hyperplans d’un espace vectoriel \(E\) de dimension finie égale à \(3\) sont les plans de \(E\). Leurs équations dans une base de \(E\) sont de la forme \(ax + by + cz = 0\) avec \((a, b, c) \in  \K^3 \pd \accol{(0, 0, 0)}\) en notant \(x, y\) et \(z\) les coordonnées d’un vecteur de \(E\) dans la base considérée.
    \end{itemize}
    \underline{Remarque}\\
    On retrouve ainsi la forme des équations cartésiennes de droites vectorielles de \(\R^2\) et plans vectoriels de \(\R^3\) vues dans le chapitre “Espaces vectoriels”.
\end{defprop}
\subsection{Intersection d’hyperplans en dimension finie}
    On suppose ici que \(E\) est de dimension finie non nulle \(n\) et que \(m \in \interventierii{1}{n}\).

\begin{theo}
    \begin{enumerate}
        \item Si \(H_1, \dots , H_m\) sont des hyperplans de \(E\) alors \(\dim \biginter^{m}_{k=1}H_k \geq \dim E - m\).
        \item Si \(F\) est un sous-espace vectoriel de \(E\) de dimension \(\dim E - m\) alors il existe \(m\) hyperplans de \(E\) notés \(H_1, \dots , H_m\) tels que \(F =\biginter^m_{k=1}H_k\).
    \end{enumerate}
\end{theo}

\begin{dem}
    \begin{enumerate}
        \item On note \(F =\biginter^d{m}_{k=1}H_k\) avec \(H_1, \dots , H_m\) des hyperplans de \(E\), respectivement noyaux de \(\phi_1, \dots , \phi_m\) formes linéaires non nulles sur \(E\). On considère l’application \(u : E \to K^m\) définie par :
        \[\forall x \in  E, u(x) = (\phi_1(x), \dots , \phi_m(x)) \]
        \(u\) est linéaire sur \(E\) (car les \(\phi_i\) le sont) avec \(E\) de dimension finie. Le théorème du rang implique donc que \(\dim E = \dim \ker u + \dim \im u\).\\~\\
        Par ailleurs, \(F = \ker u\) (car \(F =\biginter^d{m}_{k=1}\phi_k\)) et \(\dim \im u \leq m\) (car \(\im u \subset K^m\) et \(\dim K^m = m\)).\\~\\
        Ainsi, \(\dim F = \dim E - \dim \im u\) donc \(\dim F \geq \dim E - m\).\\
        \conclusion : si \(H_1, \dots , H_m\) sont des hyperplans de \(E\) alors \(\dim\biginter^m_{k=1}H_k \geq \dim E - m\).
        \item Soit \(F\) un sous-espace vectoriel de \(E\) de dimension p\( = \dim E - m\).\\~\\
        Soit \(\cal{B}_F = (e_1, \dots , e_p)\) une base de \(F\) , complétée en \(\cal{B} = (e_1, \dots , e_p, e_{p+1}, \dots , e_n)\) base de \(E\).\\~\\
        Pour \(i \in \interventierii{1}{n}\) , on note \(e^{\star}_i\) la \(i_e\) forme coordonnée relativement à la base \(\cal{B}\) de \(E\).
        Soit \(x \in  E\).
        \begin{align*}
            x \in  F &\iff \forall i \in  \interventierii{p+1}{n}e^{\star}_i (x) = 0.\\
            &\iff x \in \biginter^{n}_{i=p+1} \ker e^{\star}_i \\
            &\iff x \in \biginter^{n-p}_{k=1}\ker e^{\star}_{n-k+1} \qquad (\text{ après le changement d’indice } i = n - k + 1)
        \end{align*}
        Ainsi \(F =\biginter^{n-p}_{k=1}H_k\) avec \(H_k = \ker e^{\star}_{n-k+1}\) hyperplan de \(E\) (comme noyau d’une forme linéaire non nulle sur \(E\)) et \(n - p = m\).
    \end{enumerate}
    \conclusion : si \(F\) est sous-espace vectoriel de \(E\) de dimension \(\dim E - m\) alors \(F\) est intersection de \(m\) hyperplans de \(E\).
\end{dem}

\begin{defprop}[Système d’équations d’un sous-espace vectoriel]
    Soit \(F\) un sous-espace vectoriel de \(E\) de dimension \(\dim E - m\) et \(H_1, \dots , H_m\) des hyperplans de \(E\) tels que
    \[F = \biginter^{m}_{k=1}H_k\]
    Le système d’équations obtenu en rassemblant des équations des \(m\) hyperplans \(H_i\) relativement à une base \(\cal{B}\) de \(E\) est appelé système d’équations du sous-espace vectoriel \(F\) relativement à \(B\).\\
    \underline{Exemple}
    Une droite vectorielle de \(\R^3\) a donc un système d’équations du type
    \[\begin{cases}
        ax + by + cz &= 0\\
        a'x + b'y + c'z &= 0
    \end{cases}\]
    avec \((a, b, c) \in  \R^3 \pd \accol{(0, 0, 0)}\) et \((a', b', c') \in  \R^3 \pd \accol{(0, 0, 0)}\).
\end{defprop}

\chapter{Matrice et Application Linéaire (2)}

\minitoc
Dans ce chapitre, \(\K = \R\) ou \(\C\) et \((m, n, p) \in (\Ns)^3\).

\section{Matrice d’une application linéaire en dimension finie}
\subsection{Matrice d’un vecteur et d’une famille de vecteurs}
Soit \(E\) un \(\K\)-espace vectoriel de dimension finie n muni d’une base \(\cal{B} = (e_1, \dots , e_n)\).
\begin{defprop}[Cas d’un vecteur]
Soit \(x\) un vecteur de \(E\) dont l’écriture dans la base \(\cal{B}\) de \(E\) est \(x = a_1e_1 +\dots+a_ne_n \)avec \((a_1, \dots , a_n) \in \K^n\).\\
On appelle matrice du vecteur \(x\) dans la base \(\cal{B}\), et on note \(\Mat{\cal{B}}{x}\), la matrice de \(\M{n,1}\) définie par :
\[\Mat{\cal{B}}{x}=
\begin{pmatrix}
a_{1} \\
\vdots \\
a_{i} \\
\vdots \\
a_{n}
\end{pmatrix}\]
\end{defprop}

\begin{defprop}[Cas d’une famille finie de vecteurs]
    
Soit \(\cal{F} = (x_1, \dots , x_p)\) une famille finie de \(p\) vecteurs de \(E\).\\
On appelle matrice de la famille de vecteurs \(\cal{F}\) dans la base \(\cal{B}\) et on note \(\Mat{\cal{B}}{\cal{F}}\) la matrice de \(\M{n,p}\) dont la \(j^e\) colonne contient les coordonnées de \(x_j\), le \(j^e\) vecteur de \(\cal{F}\), dans \(\cal{B}\) :
\[\Mat{\cal{B}}{\cal{F}} =
\begin{pmatrix}
a_{11} & \cdots & a_{1j} & \cdots & a_{1p} \\
\vdots &        & \vdots &        & \vdots \\
a_{i1} & \cdots & a_{ij} & \cdots & a_{ip} \\
\vdots &        & \vdots &        & \vdots \\
a_{n1} & \cdots & a_{nj} & \cdots & a_{np}
\end{pmatrix}
\begin{array}{c}
    e_1 \\
    \vdots \\
    e_i \\
    \vdots \\
    e_n
\end{array}
\]
\end{defprop}

\subsection{Matrice d’une application linéaire dans un couple de bases}
Soit E un \(\K\)-espace vectoriel de dimension finie p muni d’une base \(\cal{B} = (e_1, \dots , e_p)\).
Soit F un \(\K\)-espace vectoriel de dimension finie n muni d’une base \(\cal{C} = (f_1, \dots , f_n)\).
Soit G un \(\K\)-espace vectoriel de dimension finie m muni d’une base \(\cal{D} = (g_1, \dots , g_m)\).


\begin{defi}
    Soit \(u \in \cal{L}(E, F )\).
    On appelle matrice de l’application linéaire \(u\) relativement aux bases \(\cal{B}\) et \(\cal{C}\), et on note \(\Mat{\cal{B}, \cal{C}}{u}\), la matrice de la famille de vecteurs \((u(\cal{B}))\) dans la base \(\cal{C}\) de \(F\) :
    \[\Mat{\cal{B},\cal{C}}{u} = \Mat{\cal{C}}{u(e_1), \dots, u(e_p)}= \begin{array}{c}
  \begin{array}{cccccc}
     \substack{u(e_1) \\ \downarrow} 
     & \cdots 
     & \substack{u(e_j) \\ \downarrow} 
     & \cdots 
     & \substack{u(e_p) \\ \downarrow}
  \end{array} \\
  \left(
  \begin{array}{ccccc}
  a_{11} & \cdots &        & \cdots & a_{1p} \\
  \vdots & \ddots &        &        & \vdots \\
  a_{i1} &        & a_{ij} &        & a_{ip} \\
  \vdots &        &        & \ddots & \vdots \\
  a_{n1} & \cdots &        & \cdots & a_{np}
  \end{array}
  \right)
  \begin{array}{c}
  f_1 \\
  \vdots \\
  f_i \\
  \vdots \\
  f_n
  \end{array}
\end{array}
    \]

\end{defi}

\begin{defprop}[Coordonnées de l’image d’un vecteur par une application linéaire]
    Soit \(u \in \cal{L}(E, F )\) et \((x, y) \in E \times F\). Alors,
    \[u(x) = y \iff AX = Y \text{ avec } \begin{cases}
    A &= \Mat{\cal{B},\cal{C}}{u} \\
    X &= \Mat{\cal{B}}{u} \\
    Y &= \Mat{\cal{C}}{y} 
    \end{cases}
    \]
\end{defprop}

\begin{defprop}[Isomorphisme entre les espaces vectoriels \(\cal{L}(E, F )\) et \(\M{n,p}\)]
    Pour \(\cal{B}\) et \(\cal{C}\) des bases fixées de \(E\) et \(F\), l’application \(\psi : u \mapsto \Mat{\cal{B}, \cal{C}}{u}\) est un isomorphisme d’espaces vectoriels de \(\cal{L}(E, F )\) dans \(\M{n,p}\).\\
    \underline{Remarque} \\
    Autrement dit, \(\cal{L}(E, F )\) et \(\M{n,p}\) sont des espaces vectoriels isomorphes.
\end{defprop}


\begin{defprop}[Matrice d’une combinaison linéaire ou d’une composée d’applications linéaires]
    \begin{enumerate}
        \item \(\forall(u, v) \in (\cal{L}(E, F ))^2 , \forall(\lambda, \mu) \in K^2, \Mat{\cal{B}, \cal{C}}{\lambda u + \mu v} = \lambda \Mat{\cal{B}, \cal{C}}{u} + \mu \Mat{\cal{B}, \cal{C}}{v}\).
        \item \(\forall(u, v) \in \cal{L}(E, F ) \times \cal{L}(F, G), \Mat{\cal{B}, \cal{D}} (v \circ u) = \Mat{\cal{C}, \cal{D}}{v} \times \Mat{\cal{B}, \cal{C}}{u}\).
    \end{enumerate}
\end{defprop}

\begin{defprop}[Matrice d’un isomorphisme]
    Soit \(u \in \cal{L}(E, F )\) avec \(E\) et \(F\) de même dimension finie.\\
    \(u\) est un isomorphisme de \(E\) sur \(F\) si, et seulement si, \(\Mat{\cal{B}, \cal{C}}{u}\) est inversible avec, dans ce cas,
    \[\Mat{\cal{C}, \cal{B}}{u^{-1}} =\paren{\Mat{\cal{B}, \cal{C}}{u}}^{-1}\]

\end{defprop}

\subsection{Matrice d’un endomorphisme dans une base}
    Soit \(E\) un \(\K\)-espace vectoriel de dimension finie \(n\) muni d’une base \(\cal{B} = (e_1, \dots , e_n)\).

\begin{defi}
    Soit \(u \in \cal{L}(E)\).
    On appelle matrice de l’endomorphisme \(u\) relativement à la base \(\cal{B}\), et on note \(\Mat{\cal{B}}{u}\), la matrice \(\Mat{\cal{B}, \cal{B}}{u}\).  
\end{defi}

\begin{prop}[Isomorphisme entre les espaces vectoriels (resp. les anneaux) \(\cal{L}(E)\) et \(\M{n}\)]
    Pour \(\cal{B}\) base fixée de \(E\), l’application \(\psi : u \mapsto \Mat{\cal{B}}{u}\) est un isomorphisme d’espaces vectoriels et un isomorphisme d’anneaux de \(\cal{L}(E)\) dans \(\M{n}\).\\
    En particulier,
    \begin{enumerate}
        \item \(\forall(u, v) \in (\cal{L}(E))^2 , \forall(\lambda, \mu) \in \K^2, \Mat{\cal{B}}{\lambda u + \mu v} = \lambda \Mat{\cal{B}}{u} + \mu \Mat{\cal{B}}{v}.\)
        \item \( \forall(u, v) \in (\cal{L}(E))^2 , \Mat{\cal{B}}{v \circ u} = \Mat{\cal{B}}{v}\times \Mat{\cal{B}}{u} \)
        \item \(\Mat{\cal{B}}{\id{E}} = \cal{I}_n\).
    \end{enumerate}
\end{prop}

\begin{defprop}[Matrice d’un automorphisme]
    Soit \(u \in \cal{L}(E)\).\\~\\
    \(u\) est un automorphisme de \(E\) si, et seulement si, \(\Mat{\cal{B}}{u}\) est inversible avec, dans ce cas,
    \[\Mat{\cal{B}}{u^{-1}} =\paren{\Mat{\cal{B}}{u}}^{-1}\]
\end{defprop}

\section{Application linéaire canoniquement associée à une matrice}
\subsection{Définition}
\begin{defi}
    Soit \(A \in \M{n,p}\).\\
    On appelle application linéaire canoniquement associée à \(A\) l’application \(u\) de \(\cal{L}(\K^p, \K^n)\) telle que
    \[\Mat{\cal{B}_{can},\cal{C}_{can}}{u}= A\]
    où \(\cal{B}_{can}\) et \(\cal{C}_{can}\) sont les bases canoniques (\ie usuelles) de \(\K^p\) et \(\K^n\).
\end{defi}

\subsection{Identification usuelle de \(\K^m\) et \(\M{m,1}\)}
\begin{defprop}
    L’application de \(\K^m\) dans \(\M{m,1}\) qui à \(x \in K^m\) associe \(X = \Mat{\cal{D}_{can}}{x}\) où \(\cal{D}_{can}\) est la base usuelle de \(\K^m\) est un isomorphisme d’espaces vectoriels.\\
    \underline{Remarque} \\
    Dans la suite de ce cours, on identifiera donc souvent tout vecteur de \(\K^m\) à la matrice-colonne de ses coordonnées dans la base usuelle de \(\K^m\) autrement dit, on identifiera \(\K^m\) et \(\M{m,1}\).
\end{defprop}

\subsection{Noyau, Image, rang d’une matrice}
\begin{defi}
    On appelle noyau (resp. image, resp. rang) de \(A \in \M{n,p}\) et on note \(\ker A\) (resp. \(\im A\), resp. \(\rg A\)) le noyau (resp. l’image, resp. le rang) de l’application linéaire \(u \in \cal{L} (\K^p, \K^n)\) canoniquement associée à \(A\).\\~\\
    Autrement dit,
    \begin{enumerate}
        \item \(\ker A = \accol{x \in \K^p \tq u(x) = 0_{\K^n} } = \accol{X \in \M{p,1} \tq AX = 0_{\M{n,1}}}\) .
        \item \(\im A = \accol{u(x)\tq x \in \K^p} = \accol{AX\tq X \in \M{p,1}}\) .
        \item \(\rg A = \dim \im u = \dim \im A\).  
    \end{enumerate}
\end{defi}

\begin{prop}
    Soit \(A \in \M{n,p}\).
    \begin{enumerate}
        \item \(\im A = \Vect (\cal{C}_1, \dots , \cal{C}_p) \text{ avec } \cal{C}_1, \dots , \cal{C}_p \text{ les colonnes de  }A\).
        \item \(\rg A = \dim \Vect{\cal{C}_1, \dots , \cal{C}_p} = \rg (\cal{C}1, \dots , \cal{C}p) \text{ donc }\rg(A) \leq  \min{n, p}\).
        \item \(\ker A\) a pour système d’équations\(
        \begin{cases}            
            L_1X &= 0\\
            \dots\\
            L_nX &= 0\\
            \end{cases}
            \)
            avec \(X \in \M{p,1}\) et \(L_1, \dots , L_n\) les lignes de \(A\).
    \end{enumerate}
\end{prop}

\begin{defprop}[Caractérisations des matrices inversibles]
    Soit \(A \in \M{n}\).
    Les assertions suivantes sont deux à deux équivalentes :
    \begin{enumerate}
        \item \(A \in \GL{n}\).
        \item \(\ker A = \accol{0_{\K^n} }\).
        \item \(\im A = \K^n\) (\ie les colonnes de \(A\) engendrent \(\K^n\)).
        \item \(\rg A = n\).
        \item \(A\) est inversible à droite (\ie il existe \(B \in \M{n} \) telle que \(AB = I_n\)).
        \item \(A\) est inversible à gauche (\ie il existe \(C \in \M{n} \) telle que \(CA = I_n\)).
    \end{enumerate}
\end{defprop}
\begin{defprop}[Cas des matrices triangulaires]
    Une matrice triangulaire est inversible si, et seulement si, tous ses éléments diagonaux sont non nuls.\\~\\
    \underline{Remarque} \\
    L’inverse d’une matrice triangulaire supérieure (resp. inférieure) inversible est une matrice triangulaire supérieure (resp. inférieure) et 
    \[\text{ pour }T =\left(
    \begin{array}{cccc}
    t_{11} & \times & \dots & \times \\
    0      & t_{22} & \ddots &   \vdots     \\
    \vdots & \ddots & \ddots & \times \\
    0      & \dots  & 0 & t_{nn}
    \end{array}
    \right) \in \GL{n},\text{ on a }T^{-1} =\left(
    \begin{array}{cccc}
    t_{11}^{-1} & \star & \dots & \star \\
    0      & t_{22}^{-1} & \ddots &   \vdots     \\
    \vdots & \ddots & \ddots & \star \\
    0      & \dots  & 0 & t_{nn}^{-1}
    \end{array}
    \right)\]
\end{defprop}

\subsection{Retour sur les systèmes linéaires}
    Soit \(A = (a_{i,j} )_{\substack{1\leq i\leq n \\ 1\leq j\leq p}} \in \M{n,p}, X = (x_i) \in \M{p,1}\) et \(B = (b_i) \in  \M{n,1}\)
\begin{defprop}[Rappels]
    ~\\
    Le système linéaire \(\begin{cases}
    a_{11}x_1 + \dots + a_{1p}x_p &= b1 \\
        &\vdots \\
    a_{n1}x_1 + \dots + a_{np}x_p &= b_n
    \end{cases}\)
    d’inconnue \((x_1, \dots , x_p) \in \K^p\) est équivalent à \(AX = B\).\\
    Dans le cas où \(B = 0_{\M{n,1}}\), on dit que le système linéaire est homogène. \\
    \underline{Remarque} \\
    Autrement dit, le système linéaire est équivalent à l’équation linéaire \(u(x) = b\) où \(u\) est l’application linéaire de \(\K^p\) dans \(\K^n\) canoniquement associée à \(A\), \(x\) est le vecteur de \(\K^p\) de matrice \(X\) dans la base usuelle de \(\K^p\) et \(b\) le vecteur de \(\K^n\) de matrice \(B\) dans la base usuelle de \(\K^n\).
\end{defprop}

\begin{defprop}[Solutions d’un système linéaire]
    \begin{itemize}
        \item Cas d’un système linéaire homogène
        \begin{itemize}
            \item On appelle rang du système linéaire homogène \(AX = 0\) le rang de \(A\).
            \item L’ensemble des solutions du système homogène \(AX = 0\) est le noyau de \(A\) donc est un \(\K\)-espace vectoriel de dimension \(p - r\) où \(r\) est le rang du système.
        \end{itemize}
        \item Cas d’un système linéaire non homogène
        \begin{itemize}
            \item On dit que le système \(AX = B\) est compatible si \(B\) appartient à l’image de \(A\).
            \item Si le système \(AX = B\) est compatible alors ses solutions sont les matrices \(X_0 + Y\) avec :
            \begin{enumerate}
                \item \(X_0 \in \M{p,1}\) une solution particulière de \(AX = B\) ;
                \item \(Y \in \M{p,1}\) solution quelconque du système homogène \(AX = 0\) associé.
            \end{enumerate}
        \end{itemize}
    \end{itemize}
    L’ensemble-solution du système linéaire \(AX = B\) est donc :
    \begin{itemize}
        \item soit l’ensemble-vide (si le système n’est pas compatible) ;
        \item soit l’ensemble noté \(X_0 + \ker A\) défini par
            \[X_0 + \ker A = \accol{X_0 + Y\tq Y \in \ker A}\]
        On dit alors que c’est un \(\K\)-espace affine passant par \(X_0\) (solution particulière de \(AX = B\)) et dirigé par \(\ker A\) (espace vectoriel des solutions du système linéaire homogène \(AX = 0\) associé)
    \end{itemize}
\end{defprop}

\begin{defprop}[Cas particulier où la matrice du système est inversible]
    On se place ici dans le cas où \(n = p\).\\
    Le système linéaire \(AX = B\) admet une solution et une seule si, et seulement si, \(A\) est inversible. \\
    Dans ce cas, on dit que le système est de Cramer.
\end{defprop}

\section{Changement de bases}
Soit \(E\) un \(\K\)-espace vectoriel de dimension finie non nulle muni de deux bases \(\cal{B}\) et \(\cal{B'}\).
Soit \(F\) un \(\K\)-espace vectoriel de dimension finie non nulle muni de deux bases \(\cal{C}\) et \(\cal{C'}\).

\subsection{Matrice de passage entre deux base}
\begin{defi}
    On appelle matrice de passage de \(\cal{B}\) à \(\cal{B'}\), et on note \(\pass{\cal{B}}{\cal{B'}}\), la matrice de \(\cal{B'}\) dans la base \(\cal{B}\) :
    \[\pass{\cal{B}}{\cal{B'}} = \Mat{\cal{B}}{\cal{B'}} = 
\begin{array}{c}
  \begin{array}{cccccc}
     \substack{e'_1 \\ \downarrow} 
     & \cdots 
     & \substack{e'_j \\ \downarrow} 
     & \cdots 
     & \substack{e'_p \\ \downarrow}
  \end{array} \\[6pt]
  \left(
  \begin{array}{ccccc}
  a_{11} & \cdots & a_{1j} & \cdots & a_{1p} \\
  \vdots &        & \vdots &        & \vdots \\
  a_{i1} & \cdots & a_{ij} & \cdots & a_{ip} \\
  \vdots &        & \vdots &        & \vdots \\
  a_{n1} & \cdots & a_{pj} & \cdots & a_{np}
  \end{array}
  \right)
  \begin{array}{c}
  e_1 \\
  \vdots \\
  e_i \\
  \vdots \\
  e_n
  \end{array}
\end{array}
\]
\end{defi}

\begin{prop}
    \begin{enumerate}
        \item \(\pass{\cal{B}}{\cal{B'}}=  \Mat{\cal{B'}, \cal{B}}{\id{E}}\).
        \item \(P = \pass{\cal{B}}{\cal{B'}}\) est inversible d’inverse \(P^{-1} = \pass{\cal{B'}}{\cal{B}}\).
    \end{enumerate}
\end{prop}

\subsection{Effet des changements de bases\dots}
\begin{defprop}[\dots sur la matrice des coordonnées d’un vecteur dans une base]
    Pour \(x \in E\), on a
    \[X = P X'\]
    avec
    \[  P = \pass{\cal{B}}{\cal{B'}}, X = \Mat{B}{x}\text{ et } X' = \Mat{\cal{B'}{x}}.\]
    \underline{Remarque} \\
    Cette formule est très simple mais elle donne les coordonnées d’un vecteur dans “l’ancienne base” en fonction des coordonnées dans “la nouvelle base” ce qui n’est pas ce que l’on cherche à obtenir en général.
\end{defprop}

\begin{defprop}[\dots sur la matrice d’une application linéaire dans un couple de bases]
    Pour \(u \in \cal{L}(E, F )\), on a
    \[A' = Q^{-1}AP\]
    avec
\[A' = \Mat{\cal{B'}, \cal{C'}}{u}, Q = \pass{\cal{C}}{\cal{X}}, A = \Mat{\cal{B}, \cal{C}}{u} \text{ et }P = \pass{\cal{B}}{\cal{B'}}\]
\end{defprop}

\begin{defprop}[\dots sur la matrice d’un endomorphisme dans une base]
    Pour \(u \in \cal{L}(E)\), on a
    \[A' = P ^{-1}AP\]
    avec
\[A' = \Mat{\cal{B'}}{u}, A = \Mat{\cal{B}}{u} \text{ et }P = \pass{\cal{B}}{\cal{B'}}\]
\end{defprop}

\section{Matrices semblables et trace}
\subsection{Matrices (carrées) semblables}
    \begin{defi}
        Deux matrices \(A\) et \(A'\) de \(\M{n}\) sont dites semblables s’il existe une matrice \(P\) dans \(\GL{n}\) telle que \(A' = P ^{-1}AP\) 
    \end{defi}
\begin{defprop}[Caractérisation géométrique]
    Deux matrices \(A\) et \(A'\) de \(\M{n}\) sont semblables si, et seulement si, elles représentent le même endomorphisme dans des bases différentes.\\
    \underline{Remarque} \\
    Un des enjeux majeurs du programme d’algébre linéaire en MPI est de réduire les matrices en déterminant des matrices diagonales ou triangulaires qui leur sont semblables.
\end{defprop}

\subsection{Trace d’une matrice (carrée)}
\begin{defi}
    On appelle trace de la matrice \(A = (a_{ij} )_{1\leq i,j\leq n}\) de \(\M{n}\), et on note \(\tr(A)\), le scalaire égal à la somme des éléments diagonaux de \(A\) :
\[\tr(A) = \sum^n_{i=1}a_{ii}  \qquad(\in \K)\] .
\end{defi}

\begin{prop}
    \begin{enumerate}
        \item  \(\forall(A, B) \in (\M{n}(\K))^2 , \forall(\lambda, \mu) \in \K^2, \tr(\lambda A + \mu B) = \lambda \tr(A) + \mu \tr(B)\)
        \item  \(\forall(A, B) \in \M{n,p} \times \M{p,n}, \tr(AB) = \tr(BA)\)
        \item  Deux matrices semblables ont même trace.
    \end{enumerate}
\end{prop}
\subsection{Trace d’un endomorphisme en dimension finie}
Soit \(E\) un \(\K\)-espace vectoriel de dimension finie non nulle
\begin{defi}
    On appelle trace de l’endomorphisme \(u\) de \(E\), et on note \(\tr(u)\), la trace d’une matrice représentant \(u\) dans une base quelconque de \(E\).
\end{defi}
\begin{prop}
    \begin{enumerate}
        \item \(\forall(u, v) \in (\cal{L}(E))^2 , \forall(\lambda, \mu) \in \K^2, \tr(\lambda u + \mu v) = \lambda \tr(u) + \mu \tr(v)\)
        \item\( \forall(u, v) \in (\cal{L}(E))^2 , \tr(u \circ v) = \tr(v \circ u)\).
    \end{enumerate}
\end{prop}
\begin{defprop}[Cas particulier des projecteurs]
    La trace d’un projecteur de \(E\) est égale à son rang.
\end{defprop}

\section{Matrices (rectangles) équivalentes et rang}
\subsection{Définition}
\begin{defi}
    Deux matrices \(A\) et \(B\) de \(\M{n,p}\) sont dites équivalentes s’il existe une matrice \(U\) dans \(\GL{n}\) et une matrice \(V\) dans \(\GL{p}\) telles que \(B = U AV\) .\\
    \underline{Remarque} \\
    Les matrices d’une même application linéaire dans des couples différents de base sont équivalentes.
\end{defi}
\subsection{Invariance du rang par multiplication par une matrice inversible}
\begin{defprop}
    \begin{enumerate}
        \item \(\forall A \in \M{n,p}, \forall U \in \GL{n}, \rg U A = \rg A\)
        \item \(\forall A \in \M{n,p}, \forall V \in GL{p}, \rg AV = \rg A\)
    \end{enumerate}
    \underline{Remarques} \\
    \begin{itemize}
        \item Le rang d’une matrice est conservé par multiplication (à droite/gauche) par une matrice inversible.
        \item Des matrices équivalentes ont même rang.
    \end{itemize}
\end{defprop}
\subsection{Caractérisation des matrices de rang \(r\)}
\begin{defprop}
    Une matrice \(A\) de \(\M{n,p}\) est de rang \(r\) si, et, seulement si, \(A\) est équivalente à \(J_r =
    \left( \begin{array}{cc}
    I_r & 0_{r,p-r} \\
    0_{n-r,r} & 0_{n-r,p-r}        
    \end{array} \right)\) \\
    \underline{Remarque} \\
    Si \(u \in \cal{L}(E, F )\) est de rang \(r\) alors il existe un couple de bases dans lequel \(u\) a pour matrice \(J_r\)
\end{defprop}
\subsection{Rang et transposition}
\begin{defprop}
    Soit \(A \in \M{n,p}\).
    \begin{enumerate}
        \item Le rang de \(A\) est égal au rang de \(\trans{A}\).
        \item Le rang de \(A\) est égal au rang de la famille de ses vecteurs lignes.
    \end{enumerate}
\end{defprop}
\subsection{Rang et opérations élémentaires}
\begin{defprop}
    Les opérations élémentaires sur les colonnes (resp. lignes) d’une matrice conservent l’image (resp. le rang, resp. le noyau).
\end{defprop}
\subsection{Rang et matrices extraites}
\begin{defi}
    Toute matrice obtenue en otant des lignes ou colonnes de \(A \in \M{n,p}\) est dite matrice extraite de \(A\).
\end{defi}
\begin{prop}
    Le rang de \(A \in \M{n,p}\) est supérieur ou égal au rang de toute matrice extraite de \(A\).
\end{prop}
\begin{defprop}[Caractérisation du rang par les matrices carrées extraites]
Le rang de \(A \in \M{n,p}\) est la taille maximale des matrices extraites de \(A\) qui sont inversibles.
\end{defprop}

\chapter{Arithmétique des Polynômes}

\minitoc

Dans ce chapitre, \( \K\) désigne le corps \(\R\) ou \( \C\).\\~\\
On rappelle une définition et une notation vues dans le chapitre “Polynômes” :
\begin{itemize}
    \item  deux polynômes \(A\) et \(B\) sont dits associés, s’il existe un scalaire non nul\( \lambda\)  tel que \(A = \lambda B\).
    \item  \(\cal{D} (A)\) désigne l’ensemble des diviseurs du polynôme \(A\) de \(\K\croch{X}\) dans \(\K\croch{X}\) .
\end{itemize}

\section{Arithmétique dans \(\K\croch{X}\)}
\subsection{PGCD de deux polynômes}
\begin{defi}[Définition des PGCD]
    Soit \((A, B) \in \paren{\K\croch{X}}^2\) tel que \(B\neq 0_{\K\croch{X}}\) .\\
    Tout diviseur commun de \(A\) et \(B\) dans \(\K\croch{X}\) de degré maximal est dit PGCD des polynômes \(A\) et \(B\).\\~\\
    \underline{Remarques} \\
    \begin{itemize}
        \item Pour tout \(A \in \K\croch{X}\) , les PGCD de \(A\) et \(1_{\K\croch{X}}\) sont les associés du polynôme \(1_{\K\croch{X}}\).
        \item Pour tout \(B \in \K\croch{X} \pd \accol{0_{\K\croch{X}}}\)  , les PGCD de \(0_{\K\croch{X}}\)  et \(B\) sont les associés du polynôme \(B\).
        \item Pour tout \((A, B) \in \paren{\K\croch{X}}^2\) tel que \((A, B)\neq (0_{\K\croch{X}} , 0_{\K\croch{X}} )\), les PGCD de \(A\) et \(B\) sont les PGCD de \(B\) et \(A\).
    \end{itemize}
\end{defi}

\begin{prop}[Propriété importante des PGCD]
    Soit \((A, B) \in \paren{\K\croch{X}}^2\) tel que \(B\neq 0_{\K\croch{X}}\) .\\
    Si \(R\) est le reste de la division euclidienne de \(A\) par \(B\) alors
    \[\cal{D}(A) \inter \cal{D}(B) = \cal{D}(B) \inter \cal{D}(R)\]
    donc les PGCD de \(A\) et \(B\) sont les PGCD de \(B\) et de \(R\).\\
    \underline{Remarque} \\
    Plus généralement si \(A = BC + D\) avec \((A, B, C, D) \in \paren{\K\croch{X}}^4\) alors \(\cal{D}(A) \inter \cal{D}(B) = \cal{D}(B) \inter \cal{D}(D)\).
\end{prop}

\begin{defprop}[Algorithme d’Euclide]
    Soit \((A, B) \in \paren{\K\croch{X}}^2\) tel que \(B\neq 0_{\K\croch{X}}\) . \\~\\
    On pose :
    \[R_0 = A \text{ et }R_1 = B\]
    Pour tout \(i \in \Ns\) tel que \(R_i\neq 0_{\K\croch{X}}\) , on définit \(R_{i+1}\) comme suit :
    \[R_{i+1} \text{ est le reste de la division euclidienne de } R_{i-1} \text{ par } R_i.\]
    Alors :
    \begin{itemize}
        \item il existe \(n \in \N\) tel que \(R_{n+1} = 0_{\K\croch{X}}\)  et \(R_n\neq 0_{\K\croch{X}}\) 
        \item pour tout \(i \in \interventierii{0}{n}\) , les PGCD de \(R_{i-1}\) et \(R_i\) sont les PGCD de \(R_i\) et \(R_{i+1}\) .
    \end{itemize}
    En particulier, les PGCD de \(A\) et \(B\) sont les PGCD de \(R_n\) et \(R_{n+1}\) donc sont les polynômes associés à \(R_n\).
    \underline{Remarques} \\
    \begin{itemize}
        \item Les PGCD de \(A\) et \(B\) sont les polynômes associés au dernier reste non nul dans l’algorithme d’Euclide encore appelé algorithme des divisions euclidiennes successives.
        \item Parmi les polynômes associés au dernier reste non nul \(R_n\), un seul est unitaire.
    \end{itemize}
\end{defprop}

\begin{defprop}[    Caractérisation des PGCD
]
Soit \((A, B) \in \paren{\K\croch{X}}^2\) tel que \(B\neq 0_{\K\croch{X}}\) . \\~\\
L’ensemble des diviseurs communs à \(A\) et \(B\) est égal à l’ensemble des diviseurs d’un de leurs PGCD.
\end{defprop}

\begin{defprop}[Définition et caractérisation du PGCD]
    Soit \((A, B) \in \paren{\K\croch{X}}^2\) tel que \(B\neq 0_{\K\croch{X}}\) .\\~\\
    \begin{itemize}
        \item Le polynôme unitaire de plus haut degré de \(\K\croch{X}\) diviseur commun de \(A\) et \(B\) est appelé le PGCD de \(A\) et \(B\) et noté \(A \wedge B\).
        \item Le polynôme \(A \wedge B\) est caractérisé par les trois propriétés suivantes :
        \begin{enumerate}
            \item \(A \wedge B\) est un polynôme unitaire de \(\K\croch{X}\).
            \item \(A \wedge B \divise A\) et \(A \wedge B \divise B\).
            \item \(\forall  D \in \K\croch{X} , D \divise A\) et \(D \divise B \imp D \divise A \wedge B\).
        \end{enumerate}
    \end{itemize}
\end{defprop}
\begin{defprop}[Relation de Bézout]
    Soit \((A, B) \in \paren{\K\croch{X}}^2\) tel que \(B\neq 0_{\K\croch{X}}\) .\\~\\
    Il existe un couple de polynômes (\(U, V ) \in \paren{\K\croch{X}}^2\), dit couple de Bézout, tel que \(AU + BV = A \wedge B\).\\
    \underline{Remarque} \\
    Un tel couple, qui n’est pas unique, peut être déterminé par l’algorithme d’Euclide étendu selon le même principe que celui vu dans \(\Z\).
\end{defprop}
\subsection{PPCM de deux polynômes}
\begin{defprop}
    Soit \((A, B) \in \paren{\K\croch{X}}^2\) avec\( A\neq 0_{\K\croch{X}}\)  et \(B\neq 0_{\K\croch{X}}\) .
    \begin{itemize}
        \item Tout multiple commun non nul de \(A\) et \(B\) de degré minimal est dit PPCM de A\(\) et \(B\).
        \item L’unique polynôme unitaire de plus bas degré de \(\K\croch{X}\) qui est multiple commun de \(A\) et \(B\) est appelé le PPCM de \(A\) et \(B\) et noté \(A \wedge B\)
    \end{itemize}
\end{defprop}
\subsection{Couple de polynômes premiers entre eux}
Soit \((A, B, C) \in (\K\croch{X})^3\) avec \(A\neq 0_{\K\croch{X}}\) , \(B\neq 0_{\K\croch{X}}\)  et \(C\neq 0_{\K\croch{X}}\) .

\begin{defi}
    On dit que les polynômes \(A\) et \(B\) sont premiers entre eux si \(A \wedge B = 1_{\K\croch{X}}\)
\end{defi}

\begin{theo}[Théorème de Bézout]
    \(A\) et \(B\) sont premiers entre eux si, et seulement si, il existe \((U, V ) \in (\K\croch{X})^2\) tel que \(AU + BV = 1_{\K\croch{X}}\).
\end{theo}
\begin{defprop}[Lemme de Gauss]
Si \(C\) divise \(AB\) et si \(C\) est premier avec \(A\) alors \(C\) divise \(B\).
\end{defprop}
\begin{defprop}[Propriétés sur le produit]
    \begin{itemize}
        \item Si \(A\) et \(B\) sont premiers entre eux et divisent \(C\) alors \(AB\) divise \(C\).
        \item Si \(A\) et \(C\) sont premiers entre eux et si \(B\) et \(C\) sont premiers entre eux alors \(AB\) et \(C\) sont premiers entre eux.
    \end{itemize}
\end{defprop}

\subsection{PGCD d’un nombre fini de polynômes}
Soit \(n \in \N\) avec \(n \geq 2\) et \((A_1, \dots , A_n) \in (\K\croch{X})^n\) tel que l’un au moins des \(A_i\) est différent de \(0_{\K\croch{X}}\) .

\begin{defprop}[PGCD]
    On appelle PGCD des polynômes \(A_1, A_2, \dots\) et \(A_n\) et on note \(A_1 \wedge \dots \wedge A_n\) le polynôme unitaire de degré maximal de \(\K\croch{X}\) diviseur commun de \(A_1, A_2, \dots\) et \(A_n\).\\~\\
    On a alors :
    \[\cal{D}(A_1 \wedge \dots \wedge A_n) = \cal{D}(A_1) \inter \dots \inter \cal{D}(A_n)\].
\end{defprop}

\begin{defprop}[Relation de Bézout]
    Il existe un \(n\)-uplet de polynômes \((U_1, \dots , U_n) \in (\K\croch{X})^n\) tel que \(A_1U_1 + \dots + A_nU_n = A_1 \wedge \dots \wedge A_n\).
\end{defprop}

\begin{defprop}[Polynômes premiers entre eux]
Les polynômes\( A_1, A_2, \dots\) et \(A_n\) sont dits :
    \begin{itemize}
        \item premiers entre eux dans leur ensemble si \(A_1 \wedge \dots \wedge A_n = 1_{\K\croch{X}}\).
        \item premiers entre eux deux à deux si \(\forall (i, j) \in \interventierii{1}{n} , i\neq j \imp A_i \wedge A_j = 1_{\K\croch{X}}\).
    \end{itemize}
\end{defprop}

\section{Polynômes irréductibles}
\subsection{Théorème de D’Alembert-Gauss }
\begin{theo}
    Tout polynôme non constant de \( \C\croch{X}\) admet au moins une racine complexe. \\
    \underline{Remarques} \\
    \begin{itemize}
        \item Tout polynôme non constant de \(\C\croch{X}\) est donc scindé sur \(\C\).
        \item Deux polynômes de \(\C\croch{X}\) sont premiers entre eux si, et seulement si, ils n’ont pas de racine commune.
    \end{itemize}
\end{theo}

\subsection{Polynômes irréductibles de \(\K\croch{X}\)}
\begin{defi}
    Un polynôme \(P\) de \(\K\croch{X}\) est dit irréductible s’il n’est pas constant et que ses seuls diviseurs dans \(\K\croch{X}\) sont les polynômes constants et les polynômes associés à \(P\). \\
    \underline{Remarque} \\
    Un polynôme \(P\) de \( \K\croch{X}\) irréductible dans\( \K\croch{X}\) n’a donc comme seuls diviseurs, à une constante multiplicative près, que \(1_{\K\croch{X}}\) et lui-même (analogie avec la notion de nombre premier vue dans \(\Z\)).
\end{defi}
\begin{defprop}[Caractérisation des polynômes irréductibles]
    \begin{itemize}
        \item Les polynômes irréductibles de \(\C \croch{X}\) sont les polynômes de degré \(1\).
        \item Les polynômes irréductibles de \(\R \croch{X}\) sont les polynômes de degré \(1\) et les polynômes de degré \(2\) qui n’ont pas de racine réelle.
    \end{itemize}
\end{defprop}

\subsection{Décomposition en facteurs irréductibles}
\begin{theo}
    \begin{itemize}
        \item Tout polynôme non constant de \(\K\croch{X}\) admet au moins un diviseur irréductible dans \(\K\croch{X}\).
        \item Tout polynôme non constant \(P\) de \(\K\croch{X}\) peut s’écrire, de manière unique à l’ordre près des facteurs, sous la forme \[P = \lambda  (P_1)\alpha_1 (P_2)\alpha_2 \dots (P_m)\alpha_m\]
        avec
        \begin{enumerate}
            \item  \(\lambda\)  le coefficient dominant de \(P\) .
            \item \(m, \alpha_1, \alpha_2, \dots , \alpha_m\) des entiers naturels non nuls.
            \item \(P_1, P_2, \dots , P_m\) des polynômes irréductibles unitaires deux à deux distincts de \(\K\croch{X}\).
        \end{enumerate}
    \end{itemize}
\end{theo}

\begin{defprop}[Caractérisation de la divisibilité dans \(\C \croch{X}\)]
    Soit \((A, B) \in (\C \croch{X})^2\) avec \(B\neq 0_{\C\croch{X}}\).\\
    \(B\) divise \(A\) si, et seulement si, toute racine de \(B\) de multiplicité \(\beta\) est racine de \(A\) de multiplicité \(\alpha \geq \beta\).
\end{defprop}

\begin{defprop}[Racines conjuguées des polynômes de \(\R\croch{X}\)]
    Deux racines complexes conjuguées d’un polynôme de \(\R \croch{X}\) ont même multiplicité.
\end{defprop}

\chapter{Fractions Rationelles}

\minitoc

\section{Généralités}
La construction de l’ensemble des fractions rationnelles étant hors programme, la présentation faite ici est volontairement élémentaire.
\subsection{Le corps \(\K\paren{X}\)}
\begin{defi}
    \begin{itemize}
        \item On appelle fraction rationnelle à coefficients dans \(\K\) tout quotient (formel) du type 
        \[F = \frac{A}{B} \text{ avec }A \in \K\croch{X} \text{ et } B \in \K\croch{X} \pd \accol{0_{\K\croch{X}}}\]
        Si \(A\) et \(B\) sont premiers entre eux, on dit que la fraction rationnelle est irréductible.
        \item L’ensemble des fractions rationnelles à coefficients dans \(K\) est noté \(\K\paren{X}\).
    \end{itemize}
    \underline{Remarques} \\
    \begin{itemize}
        \item Les éléments de \(\K\paren{X}\) se manipulent comme les éléments de \(\Q\).
        \item Tout polynôme \(A\) est identifié à la fraction rationnelle \(\frac{A}{1_\K\croch{X}}\) donc \(\K \croch{X}\)  est inclus dans \(\K\paren{X}\).
        \item Les fractions rationnelles \(\frac{A_1}{B_1}\) et \(\frac{A_2}{B_2}\) sont égales si, et seulement si, \(A_1B_2\) et \(A_2B_1\) sont égaux.
    \end{itemize}
\end{defi}

\begin{defprop}[Opérations sur l’ensemble \(\K\paren{X}\)]
    ~\\
    Pour tous \(F_1 = \frac{A_1}{B_1}\) et \(F_2 = \frac{A_2}{B_2}\) éléments de \(\K\paren{X}\), on pose :
    \[F_1 + F_2 = \frac{A_1B_2 + A_2B_1}{B_1B_2} \text{ et }F_1 \times  F_2 = \frac{A_1A_2}{B_1B_2}\]
    \underline{Remarque} \\
Ces opérations, qui ne dépendent pas de l’écriture des fractions rationnelles \(F_1\) et \(F_2\), prolongent l’addition et la multiplication vues dans \(\K\croch{X}\).
\end{defprop}

\begin{defprop}[Structure de corps]
    L’ensemble \(\paren{\K\paren{X}, +, \times }\) est un corps :
    \begin{itemize}
        \item dont l’élément neutre pour l’addition est le polynôme \(0_{\K\croch{X}}\) ;
        \item dont l’élément neutre pour la multiplication est le polynôme \(1_\K\croch{X}\) ;
        \item qui contient l’anneau intègre \(\paren{\K\paren{X}, +, \times }\).
    \end{itemize}
    \underline{Remarque}\\
    Avec la loi externe \(.\) définie sur \(\K\paren{X}\) par
    \[\forall \lambda  \in K, \forall F = \frac{A}{B} \in \K\paren{X}, \lambda.F = \frac{\lambda A}{B}\]
    \(\paren{\K\paren{X}, +, . }\) est aussi un \(\K\)-espace vectoriel.
\end{defprop}

\subsection{Degré d’une fraction rationnelle}
\begin{defprop}
    ~\\
    Pour tout \(F = \frac{A}{B}\) de \(\K\paren{X}\), on définit le degré de la fraction rationnelle \(F\) , noté \(\deg(F )\), par :
    \[\deg(F ) = \deg(A) - \deg(B)\]
    \underline{Remarque} \\
    Le degré d’une fraction rationnelle, qui ne dépend pas de l’écriture de celle-ci, appartient à \(\Z \union \accol{\pinf}\) et prolonge la notion de degré vu pour les polynômes de \(\K \croch{X}\) 
\end{defprop}
\subsection{Partie entière d’une fraction rationnelle}
\begin{defprop}
    Toute fraction rationnelle \(F = \frac{A}{B}\) de \(\K\paren{X}\) s’écrit de manière unique sous la forme \(F = E + F_1\) avec :
    \begin{itemize}
        \item \(E\) un polynôme de \(\K\croch{X}\), appelé partie entière de la fraction rationnelle \(F\) ;
        \item \(F_1\) une fraction rationnelle de \(\K\paren{X}\) de degré strictement négatif.
    \end{itemize}
    Plus précisément,
    \begin{itemize}
        \item le polynôme \(E\) est le quotient de la division euclidienne de \(A\) par \(B\) ;
        \item la fraction rationnelle \(F_1\) est égale à \(\frac{\R}{B}\) où \(\R\) est le reste de la division euclidienne de \(A\) par \(B\).
    \end{itemize}
    
\end{defprop}
\subsection{Zéros et pôles d’une fraction rationnelle}
\begin{defprop}
    ~\\
    Soit \(F = \frac{A}{B}\) une fraction rationnelle irréductible de \(\K\paren{X}\) et \(m \in \N\).
    \begin{itemize}
        \item Toute racine du polynôme \(A\) de multiplicité \(m\) est dite zéro de \(F\) de multiplicité \(m\).
        \item Toute racine du polynôme \(B\) de multiplicité \(m\) est dite pôle de \(F\) de multiplicité \(m\).
    \end{itemize}
\end{defprop}
\subsection{Fonction rationnelle}
\begin{defprop}
    Soit \(F = \frac{A}{B}\) une fraction rationnelle irréductible de \(\K\paren{X}\).\\~\\
    La fonction \(\tilde{F} : x \mapsto \frac{A(x)}{B(x)}\) définie sur \(\K\) privé de l’ensemble (fini) des pôles de \(F\) et à valeurs dans \(\K\) est dite fonction rationnelle associée à la fraction rationnelle \(F\).
\end{defprop}
\section{Décomposition en éléments simples}
\subsection{Théorème de décomposition sur \(\C\) (preuve hors programme)}
\begin{defprop}
    Si \(F\) est une fraction rationnelle de \(\C\paren{X}\) de partie entière \(E\) et de pôles deux-à-deux distincts \(\lambda_1, \dots , \lambda_r\) de multiplicités \(m_1, \dots , m_r\) alors il existe une unique famille de complexes \((a_{k,p})_{(k,p)\in \interventierii{1}{r}\times \interventierii{1}{m_k}}\) telle que :
\[F = E + \sum^r_{k=1}\paren{\sum^{m_k}_{p=1} \frac{a_{k,p}}{\paren{X - \lambda_k}^p}}\]
autrement dit :
\[F = E + \underbrace{\paren{\frac{a_{1,1}}{\paren{X - \lambda_1}} + \frac{a_{1,2}}{\paren{X - \lambda_1}^2} + \dots + \frac{a_{1,m_1}}{\paren{X - \lambda_1}^{m_1}}}}_{\text{Partie polaire associée au pôle } \lambda_1} + \dots + \underbrace{\paren{\frac{a_{r,1}}{\paren{X - \lambda_r}} + \frac{a_{r,2}}{\paren{X - \lambda_r}^2} + \dots + \frac{a_{r,m_r}}{\paren{X - \lambda_r}^{m_r}}}}_{\text{Partie polaire associée au pôle } \lambda_r}\]
    \underline{Remarque}\\
Pour décomposer une fraction rationnelle de \(\C\paren{X}\) en éléments simples, il convient de l’écrire sous forme irréductible puis de décomposer son dénominateur en facteurs rréductibles de \(\C \croch{X}\), c’est-à-dire en produit de polynômes de degré \(1\).
\end{defprop}

\subsection{Théorème de décomposition sur \(\R\) (preuve hors programme)}
\begin{defprop}
    Si \(F = \frac{A}{B}\) est une fraction rationnelle irréductible de \(\R\paren{X}\) de partie entière \(E\) et que la décomposition du polynôme \(B\) en facteurs irréductibles de \(\R \croch{X}\) s’écrit 
    \[B = \beta \prod^r_{k=1} \paren{X - \lambda_k}^{m_k} \prod^s_{k=1} \paren{X^2 + b_kX + c_k}^{n_k}\]
    alors il existe d’uniques suites de réels \((a_{k,p})_{(k,p)\in \interventierii{1}{r}\times \interventierii{1}{m_k}}\), \((d_{k,p})_{(k,p)\in \interventierii{1}{s}\times \interventierii{1}{n_k}}\) et \((e_{k,p})_{(k,p)\in \interventierii{1}{s}\times \interventierii{1}{n_k}}\) telles que :
    \[F = E + \sum^r_{k=1}\paren{\sum^{m_k}_{p=1} \frac{a_{k,p}}{\paren{X - \lambda_k}^p}} + \sum^s_{k=1} \paren{\sum^{n_k}_{p=1}\frac{d_{k,p}X + e_{k,p}}{\paren{X^2 + b_kX + c_k}^p}}\]
    \underline{Remarque} \\
    Les irréductibles de \(\R \croch{X}\) sont les polynômes de degré \(1\) et les polynômes de degré \(2\) sans racine réelle. Dans la décomposition en facteurs irréductibles écrite pour \(B\) ci-dessus, les réels \(\lambda_1, \dots , \lambda_r\) sont les racines réelles deux-à-deux distinctes de \(B\) de mutiplicités respectives \(m_1, \dots , m_r\), les couples deux-à-deux distincts de réels \((b_k, c_k)\) vérifient \(b^2_k - 4c_k < 0\) et les \(n_k\) sont des entiers non nuls.
\end{defprop}
\subsection{Quelques applications des décompositions en éléments simples}
\begin{defprop}[Calcul de primitives]
    
\end{defprop}
\begin{defprop}[Calcul de dérivées successives]
    
\end{defprop}
\begin{defprop}[Calcul de sommes de séries numériques]
    
\end{defprop}

\subsection{Deux cas particuliers}
\begin{defprop}[Coefficients des éléments simples associés aux pôles simples]   
    Si \(F = \frac{A}{B}\) est une fraction rationnelle irréductible de \(\K\paren{X}\) et \(\lambda\) est un pôle simple de \(F\) alors, dans la décomposition en éléments simples de \(F\) , le coefficient de l’élément simple \(\frac{1}{X - \lambda}\)  est :
    \[a = \frac{A(\lambda )}{B'(\lambda )}\].
    \underline{Remarque} \\
    Plus généralement, lorsque \(\lambda\)  est un pôle de multiplicité \(m\) de la fraction irréductible \(F\) , l’évaluation en \(\lambda\)  de la fraction rationnelle obtenue après réduction de \(\paren{X - \lambda }^mF\) donne le coefficient de l’élément simple \(\frac{1}{\paren{X - \lambda }^m}\) dans la décomposition en éléments simples de \(F\) 
\end{defprop}

\begin{defprop}[Décomposition en éléments simples d’une fraction rationnelle du type \(\frac{B'}{B}\)]    
    Soit \(B \in \K \croch{X}  \pd \accol{0_{\R\croch{X}}}\) .
    \begin{itemize}
        \item On suppose que \(\K = \C\) et que la décomposition de \(B\) en facteurs irréductibles de \(\C \croch{X}\) s’écrit
            \[B = \beta \prod^r_{k=1}\paren{X - \lambda_k}^{m_k}\] 
            Alors, la décomposition en éléments simples de la fraction rationnelle \(F = \frac{B'}{B}\) sur \(\C\) est :
            \[\frac{B'}{B} = \sum^r_{k=1} \frac{m_k}{\paren{X - \lambda_k}}\]
        \item On suppose que \(\K = \R\) et que la décomposition de \(B\) en facteurs irréductibles de \(\R \croch{X}\) s’écrit
            \[B = \beta \prod^r_{k=1} \paren{X - \lambda_k}^{m_k} \prod^s_{k=1}\paren{X^2 + b_kX + c_k}^{n_k} \]
            Alors, la décomposition en éléments simples de la fraction rationnelle \(F = \frac{B'}{B}\) sur \(\R\) est :
            \[\frac{B'}{B} = \sum^r_{k=1}\frac{m_k}{X - \lambda_k} + \sum^s_{k=1} \frac{n_k \paren{2X + b_k}}{X^2 + b_kX + c_k}\]
    \end{itemize}
\end{defprop}

% \chapter{Analyse Asymptotique (2)}

\minitoc

Dans ce chapitre, \(\K\) désigne le corps \(\R\) ou \(\C\).

\section{Relations de comparaison pour les fonctions}
    Se reporter au chapitre \hyperref[chap:analyse-asymptotique1]{"Analyse asymptotique \((1)\)"}
\section{Relation de comparaison pour les suites}
    Sauf mention contraire, dans cette partie, \(\paren{u_n}\) et \(\paren{v_n}\) désignent deux suites à valeurs dans \(\K\).
\subsection{Définitions}
\begin{defprop}{Domination}
    On dit que \(\paren{u_n}\) est dominée par \(\paren{v_n}\) s’il existe un entier naturel \(n_0\) et une suite \(\paren{w_n}\) bornée tel que
    \[\forall n \in \interventierie{n_0}{\pinf} , u_n = v_nw_n\]
    On note alors \(u_n = \O{v_n}\).
\end{defprop}
\begin{defprop}[Négligeabilité]
    On dit que \(\paren{u_n}\) est négligeable devant \(\paren{v_n}\) s’il existe un entier naturel \(n_0\) et une suite \(\paren{w_n}\) de limite nulle tel que
    \[\forall n \in  \interventierie{n_0}{\pinf}   , u_n = v_nw_n\]
    On note alors un = \(\o{v_n}\).  
\end{defprop}
\begin{defprop}[Equivalence]
    On dit que \(\paren{u_n}\) est équivalente à \(\paren{v_n}\) s’il existe un entier naturel \(n_0\) et une suite \(\paren{w_n}\) de limite égale à \(1\) tel que
    \[\forall n \in  \interventierie{n_0}{\pinf}, u_n = v_nw_n\]
    On note alors \(u_n \sim v_n\).
\end{defprop}

\subsection{Caractérisations pratiques}
\begin{defprop}
    Dans le cas où \(\paren{v_n}\) ne s’annule pas à partir d’un certain rang, on a les équivalences suivantes :
    \begin{enumerate}
        \item \(u_n = \O{v_n}\) si, et seulement si, la suite \(\paren{\frac{u_n}{v_n}}\) est bornée.
        \item \(u_n = \o{v_n}\) si, et seulement si, la suite \(\paren{\frac{u_n}{v_n}}\) a pour limite \(0\).
        \item  \(u_n \sim v_n\) si, et seulement si, la suite \(\paren{\frac{u_n}{v_n}}\) a pour limite \(1\).
    \end{enumerate}
\end{defprop}

\subsection{Lien entre les relations de comparaison}
\begin{defprop}
    \begin{enumerate}
        \item \(u_n = \o{v_n} \imp  u_n = \O{v_n}\) ;
        \item \(u_n \sim v_n \imp  u_n = \O{v_n}\) ;
        \item \(u_n \sim v_n \iff u_n = v_n + \o{v_n}\).
    \end{enumerate}
\end{defprop}

\subsection{Croissances comparées de suites usuelles}
\begin{defprop}
    Pour tous réels strictement positifs \(\alpha\), \(\beta\) et \(\gamma\) , on a :
    \begin{enumerate}
        \item \(\paren{\ln(n)}^{\beta} = \o{n^{\alpha}}\)
        \item \(n^{\alpha} = \o{e^{\gamma n}}\)
        \item \(e^{\gamma n} = \o{n!}\) .
        \item \(n^{\alpha} = \o{n^{\beta}}\text{ dans le cas }\alpha < \beta\)
    \end{enumerate}
\end{defprop}

\subsection{Règles usuelles de manipulation des relations de comparaison}
\begin{defprop}[Cas des \(\mathscr{O}\) (et des \(o\))]
    \begin{enumerate}
        \item Si \(u_n = \O{v_n}\) et \(\lambda  \in  \Ks\) alors \(u_n = \O{\lambda v_n}\) et \(\lambda u_n = \O{v_n}\).
        \item Si \(u_n = \O{v_n}\) et \(w_n = \O{v_n}\) alors \(u_n + w_n = \O{v_n}\).
        \item Si \(u_n = \O{v_n}\) et \(v_n = \O{w_n}\) alors \(u_n = \O{w_n}\).
        \item Si \(u_n = \O{v_n}\) alors \(u_nw_n = \O{v_nw_n}\).
        \item Si \(u_n = \O{v_n}\) et \(w_n = \O{x_n}\) alors \(u_nw_n = \O{v_nx_n}\).
        \item Si \(u_n = \O{v_n}\) et \(\phi : \N \to \N\) strictement croissante alors \(u_{\phi(n)} = \O{v_{\phi(n)}}\).
    \end{enumerate}
    \underline{Remarques} \\
    \begin{itemize}
        \item Dans tout ce qui précède, on peut remplacer \(\mathscr{O}\) par \(o\).
        \item JAMAIS de "composition des \(\mathscr{O}\) (ou des \(o\)) à gauche" par une fonction sans preuve directe.
    \end{itemize}
\end{defprop}

\begin{defprop}[Cas des équivalents]
    \begin{enumerate}
        \item Si \(u_n \sim v_n\) alors \(v_n \sim u_n\).
        \item Si \(u_n \sim v_n\) et \(v_n \sim w_n\) alors \(u_n \sim w_n\).
        \item Si \(u_n = \O{\paren{v_n}}\) et \(v_n \sim w_n\) alors \(u_n = \O{w_n}\).
        \item Si \(u_n = \o{v_n}\) et \(v_n \sim w_n\) alors \(u_n = \o{w_n}\)
        \item Si \(u_n \sim v_n\) alors \(u_nw_n \sim v_nw_n\) .
        \item Si \(u_n \sim v_n\) et \(w_n \sim x_n\) alors \(u_nw_n \sim v_nx_n\) .
        \item Si \(u_n \sim v_n\) avec \(\paren{u_n}\) et \(\paren{v_n}\) strictement positives à partir d’un certain rang alors \(u^{\beta}_n \sim v^{\beta}_n\) pour tout réel \(\beta\).
        \item Si \(u_n \sim v_n\) avec \(\paren{u_n}\) et \(\paren{v_n}\) qui ne s’annulent pas à partir d’un certain rang alors \(\frac{1}{u_n}\sim \frac{1}{v_n}\)
        \item Si \(u_n \sim v_n\) et \(\phi : \N \to \N\) strictement croissante alors \(u_{\phi(n)} \sim v_{\phi(n)}\).
    \end{enumerate}
    JAMAIS de "composition à gauche" par une fonction ni de somme d’équivalents sans preuve directe.
\end{defprop}

\subsection{Obtention et utilisation des équivalents}
\begin{defprop}[Obtention d’un équivalent par encadrement]

    Si \(\paren{u_n}\) , \(\paren{v_n}\) et \(\paren{w_n}\) sont à valeurs réelles et vérifient \(v_n \leq u_n \leq w_n\) à partir d’un certain rang avec \(v_n \sim w_n\) alors \(u_n \sim v_n\).
\end{defprop}
\begin{defprop}[Limite, signe et équivalent]
    \begin{itemize}
        \item Si \(u_n \sim v_n\) alors \(\paren{u_n}\) et \(\paren{v_n}\) ont même "comportement" c’est-à-dire que :
        \begin{itemize}
            \item \(\paren{u_n}\) a pour limite \(l\) si, et seulement si, \(\paren{v_n}\) a pour limite \(l\).
            \item \(\paren{u_n}\) n’a pas de limite si, et seulement si, \(\paren{v_n}\) n’a pas de limite.
        \end{itemize}
        \item Si \(u_n \sim v_n\) alors \(u_n\) et \(v_n\) ont le même signe à partir d’un certain rang.
    \end{itemize}
\end{defprop}
\subsection{Formule de Stirling (à connaître)}
\begin{defprop}
    Un équivalent de la suite \((n!)\) est donné par la formule de Stirling suivante :
    \[n! \sim \sqrt{2\pi n} \paren{\frac{n}{e}}^n\]
    \underline{Remarques}\\
    \begin{itemize}
        \item La démonstration de ce résultat de cours n’est pas exigible mais on en verra une dans le TD sur les séries numériques.
        \item On peut déduire de cet équivalent un développement asymptotique de la suite \((\ln n!)\) qui est :
        \[\ln n! = n \ln (n) - n + \frac{1}{2} \ln \paren{2 \pi n} + \o{1}.\]
    \end{itemize}
\end{defprop}

\subsection{Utilisation des développements limités et équivalents usuels}
\begin{defprop}
    Une connaissance parfaite des développements limités et équivalents usuels vus pour les fonctions dans le chapitre \hyperref[chap:analyse-asymptotique1]{"Analyse asymptotique (1)"} est nécessaire pour traiter les problèmes d’analyse asymptotique sur les suites, étudier la nature des séries numériques en MP2I ainsi que les différents types de convergence des séries de fonctions et les problèmes d’intégrabilité des fonctions sur un intervalle en MPI.
\end{defprop}
% \chapter{Fonction Convexe}

\minitoc

Dans ce chapitre, \(I\) est un intervalle de \(\R\), non vide et non réduit à un point.

\section{Généralités}
\subsection{Définition}
\begin{defi}
    Une fonction \(f : I \to \R\) est dite convexe sur \(I\) si :
    \[\forall (x, y) \in I^2, \forall \lambda  \in \intervii{0}{1} , f \paren{\lambda x + (1 - \lambda )y} \leq \lambda f (x) + (1 - \lambda )f (y).\]
    \underline{Interprétation géométrique}\\
    \(f\) est convexe si, et seulement si, tout sous-arc de sa courbe représentative est situé au-dessous de la corde correspondante.
\end{defi}

\subsection{Inégalité de Jensen}
\begin{defprop}
    Si \(f : I \to \R\) est convexe sur \(I\) alors, pour tous \(n \in \Ns\), \((x_1, \dots , x_n) \in In\) et \((\lambda_1, \dots , \lambda_n) \in \paren{\Rp}^n\) tels que \(\lambda_1 + \dots + \lambda_n = 1\), on a :
    \[f\paren{\sum^n_{k=1}\lambda_kx_k} \leq \sum^n_{k=1}\lambda_k f(x_k)\]
\end{defprop}

\subsection{Caractérisation de la convexité par la croissance des pentes}
\begin{defprop}
    \(f : I \to \R\) est convexe sur \(I\) si, et seulement si, pour tout \(a \in I\), la fonction \(\delta_a : I \pd \accol{a} \to \R\) définie par
    \[\forall x \in I \pd \accol{a}, \delta_a(x) = \frac{f (x) - f (a)}{x - a}\]
    est croissante.\\
    \underline{Remarque} \\
    On appelle parfois \(\delta_a\) la fonction “pente en \(a\)”.
\end{defprop}

\subsection{Inégalité des trois pentes}
\begin{defprop}
    Si \(f : I \to \R\) est convexe sur \(I\) alors, pour tout \((a, b, c) \in I^3\) tel que \(a < b < c\), on a :
    \[\frac{f (b) - f (a)}{b - a }\leq \frac{ f (c) - f (a) }{c - a } \leq \frac{f (c) - f (b) }{c - b}\] 
\end{defprop}

\subsection{Position de la courbe représentative par rapport aux sécantes}
\begin{defprop}
    Soit \(f : I \to \R\) une fonction convexe et \((x, y) \in I^2\) avec \(x < y\).\\
    La courbe représentative de \(f\) est située :
    \begin{itemize}
        \item sous la sécante à la courbe aux points d’abscisse \(x\) et \(y\) sur le segment \(\intervii{x}{y} \);
        \item au-dessus de la sécante à la courbe aux points d’abscisse \(x\) et \(y\) sur \(I \pd \intervii{x}{y}\) .
    \end{itemize}
\end{defprop}

\section{Convexité et fonctions dérivables ou deux fois dérivables}
\subsection{Fonctions convexes dérivables}
\begin{defprop}[Caractérisation des fonctions convexes dérivables]
    Soit \(f : I \to \R\) une fonction dérivable sur \(I\).\\
    Alors : \(f\) est convexe sur \(I\) si, et seulement si, \(f '\) est croissante sur \(I\).
\end{defprop}
\begin{defprop}[Position de la courbe représentative par rapport aux tangentes]
    Si \(f : I \to \R\) est une fonction convexe et dérivable sur \(I\) alors la courbe représentative de \(f\) est située au-dessus de ses tangentes.
\end{defprop}

\subsection{Fonctions convexes deux fois dérivables}
\begin{defprop}
    Soit \(f : I \to \R\) une fonction deux fois dérivable sur \(I\).\\
    Alors : \(f\) est convexe sur \(I\) si, et seulement si, \(f ''\) est positive sur \(I\)
\end{defprop}
\section{Quelques exemples d’inégalités de convexité (à retrouver)}
\subsection{Inégalités usuelles}
\begin{defprop}
    \begin{enumerate}
        \item \( \forall x \in \R, \exp (x) \geq x + 1\).
        \item \(\forall (x, y) \in \R^2, \exp \paren{\frac{x + y}{2}}\leq \frac{\exp(x) + \exp(y)}{2}\) 
        \item \(\forall x \in \Rps, \ln (x) \leq x - 1\)
        \item \(\forall (x, y) \in \paren{\Rps}^2 , \ln \paren{\frac{x + y}{2}}\geq \frac{\ln(x) + \ln(y)}{2}\) 
        \item \(\forall x \in \intervii{0}{\frac{\pi}{2}}, \frac{2}{\pi}x \leq \sin (x) \leq x\)
        \item \(\forall x \in \intervee{- \frac{\pi}{2}}{\frac{\pi}{2}} \abs{\tan(x)} \geq \abs{x}\)
        \item \(\forall x \in \R, \abs{\arctan(x)} \leq \abs{x}\)
    \end{enumerate}
\end{defprop}

\subsection{Moyennes arithmétique, géométrique et harmonique (HP mais classique)}
\begin{defprop}
    Pour \((x_k)_{1\leq k\leq n} \in \paren{\Rps}^n\), on appelle moyenne :
    \begin{itemize}
        \item arithmétique des réels \(x_1, \dots , x_n\) le réel \(A_n\) défini par
            \[A_n = \frac{1}{n} \paren{\sum^n_{k=1} x_k}\]
        \item géométrique des réels \(x_1, \dots , x_n\) le réel \(G_n\) défini par
            \[G_n =\paren{\prod^n_{k=1} x_k}^{\frac{1}{n}}\]
        \item harmonique des réels \(x_1, \dots , x_n\) le réel \(H_n\) défini par
        \[H_n = \frac{n}{\sum_{k=1}^n \frac{1}{x_k}}\]
    \end{itemize}
    Ces trois réels vérifient :
    \[H_n \leq G_n \leq A_n\]
\end{defprop}
% \chapter{Séries numériques}

\minitoc

Dans tout ce chapitre, \(\K\) désigne le corps \(\R\) ou \(\C\).
\section{Généralités}
Dans cette partie, sauf mention contraire, \((u_n)_{n\in \N}\) et \((v_n)_{n\in \N}\) sont deux suites de \(\K\).
\subsection{Définition et notation d’une série}
\begin{defi}
    On pose, pour tout \(n\) de \(\N\), \(S_n =\sum^n_{k=0}u_k\) .
    \begin{enumerate}
        \item La suite \((S_n)_{n\in \N}\) est appelée série de terme général \(u_n\) et notée \(\sum_{n\geq 0}u_n\) ou plus simplement \(\sum u_n \).
        \item Pour tout \(n \in  \N\), \(S_n\) est appelé somme partielle d’ordre \(n\) de la série \(\sum_{n\geq 0}u_n\).
    \end{enumerate}
\end{defi}
\subsection{Nature d’une série}
\begin{defi}
    La série \(\sum u_n \) est dite convergente si la suite de ses sommes partielles \((S_n)_{n\in \N}\) est convergente.\\~\\
    Sinon, elle est dite divergente.
\end{defi}
\subsection{Condition nécessaire de convergence}
\begin{defprop}
    Si la série \(\sum u_n \) converge alors la suite \((u_n)_{n\in \N }\) converge vers \(0\).\\~\\
    \underline{Attention} : la réciproque est fausse comme le montre l’exemple de la série harmonique \(\sum_{n\geq 1} \frac{1}{n}\).
\end{defprop}

\begin{defprop}[Divergence grossière]
    Si la suite \((u_n)_{n\in \N }\) ne converge pas vers \(0\) alors la série \(\sum u_n \) diverge.\\~\\
    Dans ce cas,
    on dit que la série \(\sum u_n \) diverge grossièrement.
\end{defprop}

\subsection{Somme et reste d’une série convergente}
\begin{defi}
    Si la série \(\sum u_n \) converge alors :
    \begin{enumerate}
        \item la limite \(S\) de la suite \((S_n)_{n \in \N}\) est appelée somme de la série \(\sum u_n \) et notée \(S =\sum^{\pinf}_{k=0}u_k\).
        \item pour tout \(n \in  \N\), \(R_n = S - S_n\) est appelé reste d’ordre \(n\) de la série \(\sum u_n \) et noté \(R_n = \sum^{\pinf}_{k=n+1}u_k.\)
    \end{enumerate}
\end{defi}
\begin{prop}
    La suite \((R_n)_{n\in \N}\) des restes d’une série convergente a pour limite \(0\).
\end{prop}
\subsection{Lien suite-série}
\begin{defprop}
    La suite \((u_n)\) et la série télescopique \(\sum (u_{n+1} - u_n)\) ont même nature.
\end{defprop}
\subsection{Opérations algébriques sur les séries convergentes}
\begin{defprop}
    Si les séries \(\sum u_n \) et \(\sum v_n\) convergent alors, pour tout \((\alpha , \beta ) \in  \K^2\),
    \begin{enumerate}
        \item la série \(\sum(\alpha u_n + \beta v_n)\) converge,
        \item \(\sum^{\pinf}_{k=0}(\alpha u_k + \beta v_k) = \alpha \paren{\sum^{\pinf}_{k=0}u_k}+ \beta \paren{\sum^{\pinf}_{k=0}v_k}\).\hfill (linéarité de la somme)
    \end{enumerate}
    \underline{Remarque}\\
    On en déduit que l’application qui, à une série convergente associe sa somme, est linéaire.
\end{defprop}
\subsection{Cas des séries à termes complexes}
\begin{defprop}
    La série de nombres complexes \(\sum u_n \) converge si, et seulement si, les séries de nombres réels \(\sum_{n\geq 0}\Reel{u_n}\) et \(\sum_{n\geq 0}\Ima{u_n}\) convergent. Dans ce cas, les sommes de ces séries vérifient :
    \[\sum^{\pinf}_{k=0}u_k = \paren{\sum^{\pinf}_{k=0}\Reel{u_k}} + \i \paren{\sum_{n\geq 0}\Ima{u_k}}\]
\end{defprop}
\subsection{Premières séries de référence : les séries géométriques de raison \(a \in  \K\)}
\begin{defprop}
    \begin{enumerate}
        \item La série géométrique \(\sum a^n\) est convergente si, et seulement si, \(|\abs{a} < 1\).
        \item Dans le cas \(\abs{a} < 1\), on a :
        \begin{enumerate}
             \item \(\sum^{\pinf}_{k=0}a^k = \frac{1}{1 - a}\)
             \item\(\sum^{\pinf}_{k=n+1}a^k = \frac{a^{k+1}}{1 - a}\) pour tout \(n \in \N\)
        \end{enumerate}
    \end{enumerate}
\end{defprop}

\section{Séries à termes réels positifs}
\subsection{Condition nécessaire et suffisante de convergence}
\begin{defprop}
    Une série \(\sum u_n \) de nombres réels positifs converge si, et seulement si, la suite de ses sommes partielles \((S_n)_{n \in \N}\) est majorée avec, en cas de convergence :
\[\sum^{\pinf}_{k=0}u_k = \lim_{n} S_n = \sup_{n}S_n.\]
\end{defprop}

\subsection{Théorèmes de comparaison}
\begin{theo}
    \begin{enumerate}
    \item Si \((u_n)_{n\in \N }\) et \((v_n)_{n\in \N}\) sont deux suites de nombres réels telles que \(\forall n \in  \N, 0 \leq u_n \leq v_n\) alors,
    \begin{itemize}
        \item la convergence de la série \(\sum v_n\) implique la convergence de la série \(\sum u_n \).
        \item la divergence de la série \(\sum u_n \) implique la divergence celle de la série \(\sum v_n\).
    \end{itemize}
    En cas de convergence, les sommes de ces séries vérifient :
    \[\sum^{\pinf}_{k=0}u_k \leq \sum^{\pinf}_{k=0} v_k\]
    \item Si \((u_n)_{n\in \N }\) et \((v_n)_{n\in \N}\) sont deux suites de nombres réels positifs telles que \(u_n = \O{v_n}\) alors,
    \begin{itemize}
        \item la convergence de la série \(\sum v_n\) implique la convergence de la série \(\sum u_n\).
        \item la divergence de la série \(\sum u_n \) implique la divergence de la série \(\sum v_n\).
    \end{itemize}
    \item Si \((u_n)_{n\in \N }\) et \((v_n)_{n\in \N}\) sont deux suites de nombres réels positifs telles que \(u_n \sim v_n\)\\
    alors, les séries \(\sum u_n\) et \(\sum v_n\) sont de même nature.
    \end{enumerate}
    \underline{Remarques}\\
    En cas de convergence des séries avec \(u_n = \O{v_n}\) ou \(u_n \sim v_n\), on ne peut pas écrire de relation de comparaison entre leurs sommes.
\end{theo}
\subsection{Encadrement des sommes partielles par la méthode des rectangles}
\begin{defprop}
    Si \(f\) est une application continue sur \(\intervie{0}{\pinf}\), à valeurs positives et décroissante alors, pour tout entier naturel \(n\), on a : 
    \[\int^{n+1}_{0}f (t)dt \leq \sum^{n}_{k=0} f (k) \leq f (0) + \int^n_0 f (t)dt.\]
    \underline{Remarques}\\
    \begin{itemize}
        \item Le résultat est conservé si on remplace \(0\) par \(n_0 \in  \Ns\) et \(n \in  \N\) par \(n \in  \interventierie{n}{\pinf}\) .
        \item Le résultat s’adapte dans le cas où \(f\) est croissante en changeant le sens des inégalités.
    \end{itemize}
\end{defprop}
\subsection{Autres séries de référence importantes : les séries de Riemann}
\begin{defprop}
   Pour \(\alpha  \in  \R\), la série de Riemann \(\sum \frac{1}{n^{\alpha}}\)  converge si, et seulement si, \(\alpha  > 1\).\\
    \underline{Remarque}\\
    Contrairement aux séries géométriques, il n’existe pas de formule générale donnant la valeur des sommes des séries de Riemann convergentes. 
\end{defprop}

\subsection{Convergence absolue d’une série}
    Soit \((u_n)_{n\in \N }\) une suite de \(\K\).
\subsection{Définition}
\begin{defi}
    La série \(\sum u_n \) est dite absolument convergente si la série  \(\sum \abs{u_n}\) converge.
\end{defi}
\subsection{Propriété importante}
\begin{defprop}
    Si la série \(\sum u_n \) converge absolument alors la série \(\sum u_n \) converge avec, pour tout \(n \in  \N\),
    \[\abs{\sum^{\pinf}_{k=n}u_k} \leq\sum^{\pinf}_{k=n}\abs{u_k}\]
\end{defprop}

\subsection{Théorème de domination}
\begin{theo}
    Si \((u_n)_{n\in \N }\) est une suite de \(\K\) et \((v_n)_{n\in \N}\) est une suite de nombres réels positifs tel que \(u_n = \O{v_n}\) alors, la convergence de la série \(\sum v_n\) implique la convergence absolue de la série \(\sum u_n\) donc sa convergence.
\end{theo}
\subsection{Dernières séries de référence à connaître : les séries exponentielles}
\begin{defprop}
    ~\\
    Pour tout \(z \in  \K\), la série \(\sum \frac{z^n}{n!}\) converge absolument donc converge. Sa somme est notée \(\exp(z) = \sum^{\pinf}_{k=0} \frac{z^k}{k!}\) et appelée exponentielle du nombre \(z\).
\end{defprop}
\section{Séries alternées}
    Soit \((u_n)_{n\in \N }\) une suite de nombres réels.
\subsection{Définition}

\begin{defi}
    La série \(\sum u_n \) est dite série alternée si, pour tout \(n \in  \N\), \(u_{n+1}\) et \(u_n\) sont de signes contraires.
\end{defi}

\subsection{Théorème spécial pour certaines séries alternées (règle de Leibniz)}
\begin{theo}
    Si la suite \((u_n)\) converge en décroissant vers \(0\) alors :
    \begin{enumerate}
        \item la série \(\sum (-1)^n u_n\) converge ;
        \item pour tout \(n_0 \in  \N,\sum^{\pinf}_{n=n_0} (-1)^n u_n\) a même signe que son premier terme \((-1)^{n_0} u_{n_0}\) et vérifie 
        \[\abs{\sum^{\pinf}_{n=n_0}(-1)^n u_n} \leq \abs{(-1)^{n_0} u_{n_0} }.\]
    \end{enumerate}
    \underline{Remarque}\\
    Une série alternée peut être convergente sans vérifier les hypothèses du critère spécial.
\end{theo}
% \chapter{Intégration sur un segment}

\minitoc
Dans ce chapitre, \(\K\) désigne le corps \(\R\) ou \(\C\) et \(I\) un intervalle de \(\R\), non vide et non réduit à un point.

\section{Continuité uniforme}
    Soit \(f\) une fonction définie sur \(I\) et à valeurs dans \(\K\).\\
    En MP2I, la notion de continuité uniforme est introduite pour construire l’intégrale. L’étude systématique des fonctions uniformément continues n’est pas un attendu ; la présentation faite ici est donc sommaire.

\subsection{Définition}
\begin{defi}
    
\end{defi}
    \(f\) est dite uniformément continue sur \(I\) si :
    \[\forall \epsilon  \in  \Rps, \exists \delta  \in  \Rps, \forall (a, x) \in  I^2, \abs{x -  a} \leq \delta  \imp \abs{f (x) -  f (a)} \leq \epsilon \]
    \underline{Remarque}\\
    Ici le réel \(\delta\)  ne dépend que du réel \(\epsilon\)  et pas des points de \(I\) (d’où la dénomination "continuité uniforme") contrairement au réel \(\delta\)  qui apparaît dans la définition de la continuité rappelée ci-dessous qui lui dépend du réel \(\epsilon\)  et des points \(a\) de \(I\) considérés :\\~\\
    Rappel : \(f\) est continue sur \(I\) si, pour tout réel \(a\) de \(I\), la fonction \(f\) a pour limite \(f (a) \) en \(a\) :
    \[\forall a \in  I, \forall \epsilon  \in  \Rps , \exists \delta  \in  \Rps, \forall x \in  I, \abs{x -  a} \leq \delta  \imp \abs{f (x) -  f (a)} \leq \epsilon\] 
\subsection{Propriétés}
\begin{prop}
    \(f\) lipschitzienne sur \(I\) \(\imp\) \(f\) uniformément continue sur \(I \imp f\) est continue sur \(I\).\\
    \underline{Remarque}\\
    Les réciproques de ces propriétés sont fausses (cf TD).
\end{prop}
\subsection{Théorème de Heine (preuve non exigible)}
\begin{defprop}
    Toute fonction numérique, qui est continue sur un segment de \(\R\), y est uniformément continue.
\end{defprop}

\begin{dem}
    Soit \((a, b) \in \R^2\) tel que \(a < b\) et \(f : \intervii{a}{b} \to \K\) une fonction continue sur le segment \(\intervii{a}{b}\).\\~\\
    Montrons que \(f\) est uniformément continue sur le segment \(\intervii{a}{b}\) en raisonnant par l’absurde.\\~\\
    On suppose que \(f\) n’est pas uniformément continue ce qui se traduit par :
    \[\exists \epsilon > 0, \forall \delta > 0, \exists  (x, y) \in \paren{\intervii{a}{b}}^2, \abs{x - y} \leq \delta \text{ et } \abs{f (x) - f (y)} > \epsilon \]
    Pour cet \(\epsilon > 0\), on peut en déduire, en particulier que :
    \[\forall n \in \Ns, \exists  (x_n, y_n) \in \paren{\intervii{a}{b}}^2 , \abs{x_n - y_n} \leq \frac{1}{n} \text{ et } \abs{f (x_n) - f (y_n)} > \epsilon\]
    \begin{itemize}
        \item Comme la suite \((x_n)\) est à valeurs dans le segment \(\intervii{a}{b}\), elle est bornée. D’après le théorème de Bolzano-Weierstrass, on peut donc en extraire une suite \((x_{\phi(n)})\), avec \(\phi\) application strictement croissante de \(\Ns\) dans \(\Ns\), qui converge vers \(c \in \intervii{a}{b}\) ce qui implique 
        \[\abs{x_{\phi(n)} - c} \underset{n\to\pinf}{0}\]
        \item Par inégalité triangulaire vérifiée par \(\abs{.}\) (valeur absolue ou module), on a successivement 
        \[\forall n \in \Ns, \abs{y_{\phi(n)} - c }\leq \abs{y_{\phi(n)} - x_{\phi(n)}} + \abs{x_{\phi(n)} - c}\]
        \[\forall n \in \Ns, \abs{y_{\phi(n)} - c} \leq \frac{1}{\phi(n)} + \abs{x_{\phi(n)} - c}\]
        et enfin, par théorème d’encadrement :
        \[\abs{y_{\phi(n)} - c} \underset{n \to \pinf}{\to} 0\]
        Par conséquent, la suite \(\paren{y_{\phi(n)}}\) converge aussi vers \(c \in \intervii{a}{b}\) .
        \item Comme \(f\) est continue sur \(\intervii{a}{b}\) , elle l’est en \(c\). Par caractérisation séquentielle de la continuité de \(f\) en \(c\), on en déduit que les suites \(\paren{f \paren{x_{\phi(n)}}}\) et \(\paren{f\paren{ y_{\phi(n)}}}\) convergent toutes deux vers \(f (c)\) donc que la suite \(\paren{f \paren{x_{\phi(n)}}  - f \paren{ y_{\phi(n)}}}\) converge vers \(0\). Ainsi, \(\abs{f \paren{x_{\phi(n)}}  - f \paren{ y_{\phi(n)}}}\underset{ n\to\pinf}{\to} 0\) ce qui est incompatible avec les inégalités
        \[\forall n \in \Ns, \abs{f\paren{x_{\phi(n)}} - f\paren{ y_{\phi(n)}}} > \epsilon\]
    \end{itemize}
    \conclusion l’hypothèse initiale est fausse donc \(f\) est uniformément continue sur \(\intervii{a}{b}\).\\~\\
    \underline{Remarque}\\
    En revanche, la continuité sur un intervalle quelconque n’implique pas l’uniforme continuité sur cet intervalle comme le montre l’exemple simple de la fonction \(x \mapsto x^2\) sur \(\Rp\).
\end{dem}

\section{Continuité par morceaux sur un segment}
    Soit \(a\) et \(b\) deux réels tels que \(a < b\).
\subsection{Subdivision d’un segment}
\begin{defprop}
    On appelle subdivision du segment \(\intervii{a}{b}\) toute famille finie \((a_i)_{i\in \interventierii{0}{n}}\) telle que \(\underbrace{a_0}_{=a}< a_1 < \dots < \underbrace{a_n}_{=b}\). \\
    Le réel positif \(\sigma = \max_{i\in \interventierii{1}{n}} \abs{a_i -  a_{i- 1}}\) est appelé pas de cette subdivision.
\end{defprop}
\subsection{Fonctions en escalier sur un segment}
\begin{defprop}
    Une fonction \(f\) définie sur le segment \(\intervii{a}{b}\), à valeurs dans \(\K\), est dite en escalier sur le segment \(\intervii{a}{b}\) s’il existe une subdivision \((a_i)_{i\in \interventierii{0}{n}}\) de \(\intervii{a}{b}\) telle que, pour tout \(i \in  \interventierii{1}{n}\), la restriction de \(f\) à \(\intervee{a_{i-1}}{a_i}\) est une fonction constante. \\
    L’ensemble des fonctions en escalier sur \(\intervii{a}{b}\), à valeurs dans \(\K\), est noté \(\cal{E}\paren{\intervii{a}{b} , \K}\).
\end{defprop}

\subsection{Fonctions continues par morceaux sur un segment}
\begin{defi}
    Une fonction \(f\) définie sur le segment \(\intervii{a}{b}\), à valeurs dans \(\K\), est dite continue par morceaux sur le segment \(\intervii{a}{b}\) s’il existe une subdivision \((a_i)_{i\in \interventierii{0}{n}}\) de \(\intervii{a}{b}\) telle que, pour tout \(i \in  \interventierii{1}{n}\), la restriction de \(f\) à \(\intervee{a_{i-1}}{a_i}\) est prolongeable en une fonction continue sur le segment \(\intervii{a_{i-1}}{a_i}\).\\~\\
    L’ensemble des fonctions continues par morceaux sur \(\intervii{a}{b}\), à valeurs dans \(\K,\) est noté \(\cal{CM}\paren{\intervii{a}{b} , \K}\).\\
    \underline{Remarques}\\
    \begin{itemize}
        \item \(\cal{E}  \paren{\intervii{a}{b} , \K} \subset \cal{CM} \paren{\intervii{a}{b} , \K}\) et \(\cal{C} \paren{\intervii{a}{b} , \K} \subset \cal{CM} \paren{\intervii{a}{b} , \K}\).
        \item \(\cal{CM} \paren{\intervii{a}{b} , \K}\) est sous-espace vectoriel de \(\paren{\cal{F} \paren{\intervii{a}{b} , \K} , +, .}\) et sous-anneau de \(\paren{\cal{F} \paren{\intervii{a}{b} , \K} , +, \times }\).
    \end{itemize}
\end{defi}
\subsection{Approximation uniforme des fonctions continues par morceaux}
\begin{defprop}
    Soit \(f\) une fonction continue par morceaux sur le segment \(\intervii{a}{b}\), à valeurs dans \(\K\). \\
    Alors, pour tout \(\epsilon  > 0\), il existe une fonction \(\phi : \intervii{a}{b} \to \K\) en escalier sur le segment\( \intervii{a}{b}\) telle que 
    \[\sup_{t\in \intervii{a}{b}} \abs{f (t) -  \phi(t)} \leq \epsilon .\]
    \underline{Remarques} \\
    \begin{itemize}
        \item La borne supérieure écrite a du sens car toute fonction de \(\cal{CM} \paren{\intervii{a}{b}, \K}\) est bornée.
        \item On déduit du résultat précédent que :
            \[\forall f \in  \cal{CM} \paren{\intervii{a}{b}, \K} , \exists  (\phi_n) \in  \paren{\cal{E} \paren{\intervii{a}{b} , \K}}^{\N} , \sup_{t\in \intervii{a}{b}} \abs{f (t) -  \phi_n(t)} \underset{n \to \pinf}\to 0\]
        Dans ce cas, on dit que la suite de fonctions \((\phi_n)\) converge uniformément vers \(f\) sur \(\intervii{a}{b}\).
    \end{itemize}
\end{defprop}
\begin{dem}
    Soit \((a, b) \in \R^2\) tel que \(a < b\) et \(f : \intervii{a}{b} \to \K\) une fonction continue par morceaux sur le segment \(\intervii{a}{b}\).\\~\\
    Soit \(\epsilon > 0\).\\~\\
    Montrons l’existence de \(\phi : \intervii{a}{b} \to \K\) en escalier sur le segment \(\intervii{a}{b}\) telle que \(\sup_{t\in\intervii{a}{b}}\abs{f (t) - \phi(t)} \leq \epsilon\).
    \begin{itemize}
    \item Cas particulier où \(f\) est continue sur le segment \(\intervii{a}{b}\).\\~\\
        Dans ce cas, d’après le théorème de Heine, \(f\) est uniformément continue sur le segment \(\intervii{a}{b}\).\\~\\
        Il existe donc un réel \(\delta > 0\) tel que, pour tout \((x, y) \in \paren{\intervii{a}{b}}^2\), \(\abs{x - y} \leq \delta \imp \abs{f (x) - f (y)} \leq \epsilon \hfill (\star)\).\\~\\
        On note \(n\) un entier \(n \in \Ns\) tel que \(\frac{b - a}{n} \leq \delta\) (fixé dans la suite).\\~\\
        On crée alors une subdivision \((a_i)_{i\in\interventierii{0}{n}}\) du segment \(\intervii{a}{b}\) en posant, pour tout \(i \in \interventierii{0}{n}\), \(a_i = a + i\frac{b - a}{n}\)\\~\\
        On considère enfin la fonction \(\phi\), en escalier sur le segment \(\intervii{a}{b}\), définie par :
        \[\forall i \in \interventierii{1}{n}, \forall x \in \intervie{a_{i-1}}{a_i} , \phi(x) = f (a_{i-1}) \text{ et } \phi(b) = f (b)\]
        \underline{Montrons que} \(\sup_{t\in\intervii{a}{b}} \abs{f (t) - \phi(t)} \leq \epsilon\).\\~\\
        Pour tout \(t \in \intervie{a}{b}\), il existe \(i \in \interventierii{1}{n}\) tel que \(t \in \intervie{a_{i-1}}{a_i}\) donc \(\abs{t - a_{i-1}} \leq a_i - a_{i-1} \leq \frac{b - a}{n} \leq \delta\).\\~\\
        Par continuité uniforme de \(f\) , avec \((\star)\), on a : \(\abs{f (t) - \phi(t)} = \abs{f (t) - f (a_{i-1})} \leq \epsilon\). Ainsi,
        \[\forall t \in \intervii{a}{b} , \abs{f (t) - \phi(t)} \leq \epsilon\]
        (car l’inégalité est triviale pour \(t = b\)).\\~\\
        Comme la fonction\( t \mapsto \abs{f (t) - \phi(t)}\) est par \(\epsilon\) sur le segment \(\intervii{a}{b}\), elle admet une borne supérieure qui vérifie \(\sup_{t\in\intervii{a}{b}}\abs{f (t) - \phi(t)} \leq \epsilon\) (en cas d’existence, la borne supérieure est le plus petit des majorants).\\~\\
        \conclusion il existe \(\phi : \intervii{a}{b} \to \K\) en escalier sur le segment \(\intervii{a}{b}\) telle que \(\sup_{t\in\intervii{a}{b}}\abs{f (t) - \phi(t)} \leq \epsilon\).

    \item Cas général où \(f\) est continue par morceaux sur le segment \(\intervii{a}{b}\).\\~\\
        On note \((a_i)_{i\in\interventierii{0}{n}}\) une subdivision du segment \(\intervii{a}{b}\) adaptée à la fonction \(f\) .\\~\\
        Soit \(i \in \interventierii{1}{n}\).\\~\\
        Par définition, la restriction \(f_{|\intervee{a_{i-1}}{a_i}}\) est prolongeable en une fonction continue sur \(\intervii{a_{i-1}}{a_i}\).\\~\\
        D’après le point précédent, il existe donc une fonction en escalier \(\phi_i : \intervii{a_{i-1}}{a_i} \to \K\) tel que
        \[\forall t \in \intervee{a_{i-1}}{a_i} , \abs{f (t) - \phi_i(t)} \leq \epsilon\]
        On définit alors une fonction en escalier \(\phi\) sur \(\intervii{a}{b}\) en posant :
        \[\forall i \in \interventierii{1}{n}, \forall t \in \intervee{a_{i-1}}{a_i} , \phi(t) = \phi_i(t)\]
        \[\forall i \in \interventierii{0}{n}, \phi (a_i) = f (a_i)\]
        Par construction, cette fonction \(\phi\) vérifie
        \[\forall t \in \intervii{a}{b} , \forall{f (t) - \phi(t)} \leq \epsilon\]
        et on conclut, comme dans le cas précédent, que \(\sup_{t\in\intervii{a}{b}}\abs{f (t) - \phi(t)}\leq \epsilon\).\\~\\
        \conclusion il existe \(\phi : \intervii{a}{b} \to \K\) en escalier sur le segment \(\intervii{a}{b}\) telle que \(\sup_{t\in\intervii{a}{b}} \abs{f (t) - \phi(t)} \leq \epsilon\)\\~\\
    \end{itemize}
    \underline{Remarque} : vocabulaire et notation\\~\\
    Dans ce résultat, la fonction \(\phi\) dépend du réel \(\epsilon > 0\) fixé.\\~\\
    Si on applique ce résultat avec \(\epsilon = \frac{1}{n + 1}\) pour tout \(n \in N\), on en déduit l’existence d’une suite de fonctions \((\phi_n)\) en escalier sur le segment \(\intervii{a}{b}\) tel que \(\forall n \in \N\), \(\sup _{t\in\intervii{a}{b}} \abs{f (t) - \phi_n(t)} \leq \frac{1}{n + 1}\).\\~\\
    Ainsi
    \[\sup_{t\in\intervii{a}{b}}\abs{f (t) - \phi_n(t)} \underset{n\to\pinf}{\to} 0\]
    ce que l’on note plus simplement, en anticipant sur ce qui sera vu en MPI,
    \[\norme{f - \phi_n}_{\infty} \underset{n\to\pinf}{\to} 0\]
    et on dit que la suite de fonctions \((\phi_n)\) en escalier sur \(\intervii{a}{b}\) converge uniformément vers \(f\) sur \(\intervii{a}{b}\) ou encore que \(f\) est limite uniforme de la suite de fonctions \((\phi_n)\) en escalier sur \(\intervii{a}{b}\).
\end{dem}
\section{Intégrale sur un segment d’une fonction continue par morceaux}
    Soit \(a\) et \(b\) deux réels tels que \(a < b\).
\subsection{Cas particulier des fonctions en escalier sur un segment}
\begin{defprop}
    Soit \(f : \intervii{a}{b} \to \K\) une fonction en escalier sur \(\intervii{a}{b}\) et \((a_i)_{i\in \interventierii{0}{n}}\) une subdivision de \(\intervii{a}{b}\) adaptée à \(f\) .\\~\\
    Pour tout \(i \in  \interventierii{1}{n}\), on note \(\lambda_i\) la valeur prise par \(f\) sur \(\intervee{a_{i-1}}{a_i}\).\\~\\
    Alors, le scalaire \(\sum^n_{i=1}\lambda_i (a_i -  a_{i- 1})\) est :
    \begin{itemize}
        \item indépendant de la subdivision de \(\intervii{a}{b}\) adaptée à \(f\) choisie ;
        \item appelé l’intégrale de \(f\) sur \(\intervii{a}{b}\) et noté \(\int_{\intervii{a}{b}} f\) ou \(\int^b_a f\) ou encore \(\int^b_a f (t)dt\) :
        \[\int_{\intervii{a}{b}} f =\int^b_a f =\int^b_a f (t)dt = \sum^n_{i=1} \lambda_i (a_i -  a_{i- 1})\]
    \end{itemize}
    \underline{Remarque}\\
    Dans le cas \(\K = \R\), on peut interpréter cette intégrale comme "aire algébrique sous la courbe de \(f\) ".
\end{defprop}
\begin{dem}
    \begin{itemize}
        \item Convergence de la suite \(\paren{\int_{\intervii{a}{b}}\phi_n}\)\\~\\
            Par hypothèse \(\norme{f - \phi_n}_{\infty} \underset{n\to\pinf}{\to} 0\) donc, il existe un rang \(n_0\) tel que \(\forall n \in \N, n \geq n_0 \imp \norme{f - \phi_n}_{\infty} \leq 1 \).\\~\\
            Par inégalité triangulaire sur la norme \(\norme{ . }_{\pinf}\), on en déduit que :
            \[\forall n \in \N, n \geq n_0, \norme{\phi_n}_{\infty} \leq \norme{f }_{\infty} + \norme{f - \phi_n}_{\infty} \leq \norme{f }_{\infty} + 1\]
            puis, par inégalité triangulaire et croissance de l’intégrale sur un segment d’une fonction en escalier :
            \[\forall n \in \N, n \geq n_0, \abs{\int_{\intervii{a}{b}} \phi_n} \leq \int_{\intervii{a}{b}}\abs{\phi_n} \leq\int_{\intervii{a}{b}}\paren{\norme{f }_{\infty} + 1}\]
            Ainsi, la suite \(\paren{\int_{\intervii{a}{b}} \phi_n}_{n\in \N}\) à valeurs dans \(\K\) est bornée. D’après le théorème de Bolzano-Weierstrass, elle admet donc une suite extraite \(\paren{\int_{\intervii{a}{b}}\phi_{\alpha (n)}}_{n\in\N} \)convergente. On note \(l \in \K\) la limite de cette suite.\\~\\
            Soit \(n \in N\).\\
            En utilisant la linéarité, l’inégalité triangulaire et la croissance de l’intégrale sur un segment d’une fonction en escalier, on trouve successivement
            \[\int_{\intervii{a}{b}}\phi_n - \int_{\intervii{a}{b}}\phi_{\alpha (n)} =  \int _{\intervii{a}{b}} (\phi_n - \phi_{\alpha (n)}) \]
            \[\abs{\int_{\intervii{a}{b}}\phi_n -\int_{\intervii{a}{b}} \phi_{\alpha (n)}}\leq \int_{\intervii{a}{b}}\abs{\phi_n - \phi_{\alpha (n)}} \leq \int_{\intervii{a}{b}}\norme{\phi_n - \phi_{\alpha (n)}}_{\infty} \leq (b - a)\norme{\phi_n - \phi_{\alpha (n)}}_{\infty}\]
            avec, par inégalité triangulaire sur la norme,
            \[ \norme{\phi_n - \phi_{\alpha (n)}}_{\infty} \leq \norme{\phi_n - f }_{\infty} + \norme{f - \phi_{\alpha (n)}}_{\infty}\]
            donc 
            \[\abs{\int_{\intervii{a}{b}} \phi_n - \int_{\intervii{a}{b}} \phi_{\alpha (n)}} \leq (b - a) \paren{\norme{f - \phi_n}_{\infty} + \norme{f - \phi_{\alpha (n)}}_{\infty}}\]
            Comme \((b - a) \paren{\norme{f - \phi_n}_{\infty} + \norme{f - \phi_{\alpha (n)}}_{\infty}} \underset{n\to\pinf}{\to} 0\) par hypothèse, on obtient par théorème d’encadrement que 
            \[\int_{\intervii{a}{b}} \phi_n - \int_{\intervii{a}{b}} \phi_{\alpha (n)} \underset{n\to\pinf}{\to} 0\]
            puis, comme \(\int_{\intervii{a}{b}} \phi_{\alpha (n)} \underset{n\to\pinf}{\to} l\), que
            \[\int_{\intervii{a}{b}} \phi_n \underset{n\to\pinf}{\to} l\]
            \conclusion la suite \(\paren{\int_{\intervii{a}{b}} \phi_n}\) converge.
        \item Limite de la suite \(\paren{\int_{\intervii{a}{b}} \phi_n}\) indépendante du choix de la suite \((\phi_n)\).\\~\\
            On considère ici une autre suite \((\psi_n)_{n\in\N}\) de fonctions en escalier sur \(\intervii{a}{b}\) telle que \(\abs{f - \psi_n}_{\infty} \underset{n\to\pinf}{\to} 0\).
            Avec des arguments du même type que ci-dessus, on a :
            \[\abs{\int_{\intervii{a}{b}}\phi_n - \int_{\intervii{a}{b}} \psi_n} = \abs{\int_{\intervii{a}{b}} (\phi_n - \psi_n)}\leq (b - a) \norme{\phi_n - \psi_n}_{\infty} \leq (b - a)\paren{\norme{f - \phi_n}_{\infty} + \norme{f - \psi_n}_{\infty}}\]
            puis 
            \[\abs{\int_{\intervii{a}{b}} \phi_n -\int_{\intervii{a}{b}}\psi_n }\underset{n\to\pinf}{\to} 0\]
            et enfin, puisque les deux suites d’intégrales convergent,
            \[\lim_{n\to\pinf}\int_{\intervii{a}{b}}\phi_n = \lim_{n\to\pinf} \int_{\intervii{a}{b}}\psi_n\]
            \conclusion la limite de \(\paren{\int_{\intervii{a}{b}}\phi_n} \)ne dépend pas de la suite de fonctions en escalier \((\phi_n)_{n\in\N}\) qui approche uniformément \(f\) sur \(\intervii{a}{b}\).
    \end{itemize}
\end{dem}
\subsection{Cas général des fonctions continues par morceaux sur un segment}
\begin{defprop}
    Soit \(f : \intervii{a}{b} \to \K\) une fonction continue par morceaux sur \(\intervii{a}{b}\).\\~\\
    Soit \((\phi_n)\) une suite de fonctions en escalier sur \(\intervii{a}{b}\) à valeurs dans \(\K\) telle que 
    \[\sup_{t\in \intervii{a}{b}} \abs{f (t) -  \phi_n(t)} \underset{n\to\pinf}{\to}  0\]
    Alors, la suite \(\paren{\int_{\intervii{a}{b}} \phi_n}\) converge et sa limite est :
    \begin{itemize}
        \item indépendante du choix de la suite \((\phi_n)\) ;
        \item appelée l’intégrale de \(f\) sur \(\intervii{a}{b}\) et notée \(\int_{\intervii{a}{b}} f\) ou \(\int^b_a f\) ou encore \(\int^b_a f (t)dt\) :
        \[\int_{\intervii{a}{b}} f = \int^b_a f = \int^b_a f (t)dt = \lim_{n\to\pinf}\paren{\int_{\intervii{a}{b}} \phi_n}\]
    \end{itemize}
    \underline{Remarque}\\
    Cette définition prolonge bien celle vu pour les fonctions en escalier sur le segment \(\intervii{a}{b}\) car, lorsque \(f\) est en escalier sur \(\intervii{a}{b}\), on retrouve la même intégrale que celle définie dans le paragraphe précédent.
\end{defprop}
\subsection{Valeur moyenne}
\begin{defprop}
    Pour tout \(f\) de \(\cal{CM} \paren{\intervii{a}{b}, \K}\), le scalaire \(\frac{1}{b - a}\int_{\intervii{a}{b}} f\) est dit valeur moyenne de \(f\) sur \(\intervii{a}{b}\).
\end{defprop}
\subsection{Propriétés}
\begin{defprop}[Linéarité de l’intégrale]
    Si \(f : \intervii{a}{b} \to \K\) et \(g : \intervii{a}{b} \to \K\) sont deux fonctions continues par morceaux sur \(\intervii{a}{b}\) et, \(\alpha\) et \(\beta\) deux éléments de \(\K\) alors
    \[ \int_{\intervii{a}{b}} (\alpha f + \beta g) = \alpha \paren{\int_{\intervii{a}{b}} f} + \beta \paren{\int_{\intervii{a}{b}} g}\]
\end{defprop}
\begin{defprop}[Positivité (pour les fonctions à valeurs réelles)]
    Si \(f : \intervii{a}{b} \to \R\) est une fonction continue par morceaux et positive sur \(\intervii{a}{b}\) alors 
    \[ \int_{\intervii{a}{b}}f \geq 0\]
\end{defprop}
\begin{defprop}[    Croissance (pour les fonctions à valeurs réelles)]
    Si \(f : \intervii{a}{b} \to \R\) et \(g : \intervii{a}{b} \to \R\) sont des fonctions continues par morceaux telles que \(f \leq g\) alors
    \[\int_{\intervii{a}{b}} f \leq \int_{\intervii{a}{b}} g\]
\end{defprop}

\begin{defprop}[Inégalité triangulaire intégrale]
    Si \(f : \intervii{a}{b} \to \K\) est une fonction continue par morceaux sur \(\intervii{a}{b}\) alors
    \[\abs{\int_{\intervii{a}{b}} f} \leq \int_{\intervii{a}{b}} \abs{f }\]
\end{defprop}
\begin{defprop}[Relation de Chasles]
    Soit \(f : \intervii{a}{b} \to \K\) une fonction continue par morceaux sur \(\intervii{a}{b}\).\\~\\
    Alors, pour tout \((\alpha, \beta, \gamma) \in  \intervii{a}{b}\) , on a :
    \[\int^{\beta}_{\alpha} f (t) dt = \int^{\gamma}_{\alpha} f (t) dt + \int^{\beta}_{\gamma} f (t) dt\]
    avec les notations suivantes*
    \[\int^{\beta}_{\alpha} f (t) dt = \begin{cases}
        \int_{\intervii{\alpha}{\beta}} f &\text{ si } \alpha < \beta \\
        0 & \text{ si } \alpha = \beta \\
        - \int_{\intervii{\beta}{\alpha}} f &\text{ si } \alpha > \beta
    \end{cases}\]
\end{defprop}
\begin{defprop}[Nullité (pour les fonctions à valeurs réelles)]
    Si \(f : \intervii{a}{b} \to \R\) est une fonction CONTINUE, positive et d’intégrale nulle sur \(\intervii{a}{b}\) alors \(f\) est la fonction nulle. \\
    \underline{Remarque} \\
    Cette propriété n’est pas conservée dans le cas où \(f\) est seulement continue par morceaux.
\end{defprop}
\begin{defprop}[    Cas des fonctions à valeurs complexes]
    Soit \(f : \intervii{a}{b} \to \C\) une fonction continue par morceaux sur\( \intervii{a}{b}\). \\
    Alors : 
    \[\int_{\intervii{a}{b}} f = \int_{\intervii{a}{b}} \Reel{f} + \i \int_{\intervii{a}{b}} \Ima{f }\]
\end{defprop}

\begin{defprop}[Cas des fonctions paires ou impaires]
    Soit \(a\) un réel strictement positif et \(f : \intervii{-a}{a} \to \K\) une fonction continue par morceaux sur \(\intervii{-a}{a}\) .
    \begin{enumerate}
        \item Si \(f\) est paire alors \(\int^a_{- a} f (t) dt = 2 \int^a_0 f (t) dt\)
        \item Si \(f\) est impaire alors \(\int^a_{- a} f (t) dt = 0\).
    \end{enumerate}
\end{defprop}

\begin{defprop}[Cas des fonctions périodiques]
    Soit \(f : \R \to \K\) une fonction continue par morceaux sur \(\R\).\\~\\
    Si \(f\) est périodique de période \(T\) alors, pour tout réel \(a\), on a : \(\int^{a+T}_a f (t) dt = \int^T_0 f (t) dt\)
\end{defprop}
\begin{defprop}[Conséquences pratiques de la définition]
   \begin{itemize}
        \item Si \(f\) et \(g\) sont des fonctions continues par morceaux à valeurs dans \(\K\) qui coïncident sur \(\intervii{a}{b}\) privé d’un nombre fini de points alors
        \[\int_{\intervii{a}{b}} f = \int_{\intervii{a}{b}} g\]
        \item Si \(f\) est une fonction continue par morceaux à valeurs dans \(\K\) et \((a_i)_{i\in \interventierii{0}{n}}\) une subdivision de \(\intervii{a}{b}\) contenant les points de discontinuité de \(f\) alors
        \[\int_{\intervii{a}{b}} f = \sum^n_{i=1} \paren{\int_{\intervii{a_{i- 1}}{a_i}} f }\]
   \end{itemize}
\end{defprop}
\subsection{Sommes de Riemann à pas constant}
\begin{defprop}
    Si \(f : \intervii{a}{b} \to \K\) est une fonction continue par morceaux sur \(\intervii{a}{b}\) alors
    \[\lim_{n\to\pinf} \frac{b -  a}{n} \paren{\sum^{n- 1}_{k=0} f \paren{a + k \frac{b -  a}{n}}} = \int_{\intervii{a}{b}} f \]
    \underline{Remarque} \\~\\
    Dans le cas où \(\K = \R\), le réel \(\frac{b -  a}{n} \paren{\sum^{n- 1}_{k=0} f \paren{a + k \frac{b -  a}{n}}}\) correspond à la somme des aires de \(n\) rectangles de largeur \(\frac{b - a}{n}\) et de longueur \(f\paren{a + k \frac{b -  a}{n}}\) avec \(k\) qui varie dans \(\interventierii{0}{n -  1}\)
\end{defprop}
\section{Lien entre intégrale et primitive d’une fonction continue}
    Soit \(f\) une fonction définie sur \(I\), continue sur \(I\), à valeurs dans \(\K\).
\subsection{Théorème fondamental}
\begin{theo}
    \begin{enumerate}
        \item Pour tout \(a\) appartenant à \(I\), la fonction \(x \mapsto \int^x_a f (t) dt \)est :
        \begin{itemize}
            \item dérivable sur \(I\) de dérivée la fonction \(f\) ;
            \item l’unique primitive de la fonction \(f\) sur \(I\) qui s’annule en \(a\).
        \end{itemize}
        \item Pour toute primitive \(F\) de \(f\) sur \(I\), on a :
        \[\forall (a, b) \in  I^2,\int^b_a f (t) dt = F (b) -  F (a)\]
    \end{enumerate}
\end{theo}
\begin{defprop}[Existence de primitives]
    Toute fonction numérique continue sur un intervalle \(I\) admet des primitives sur \(I\) et celles-ci permettent de calculer les intégrales de cette fonction sur tout segment inclus dans \(I\).
\end{defprop}
\section{Formules de Taylor globales}
\subsection{Formule de Taylor avec reste intégral}
\begin{defprop}
    Soit \(n \in  N\). \\
    Si \(f\) est une application de classe \(\cal{C}^{n+1}\) sur \(I\) à valeurs dans \(\K\) alors, pour tout couple \((a, x) \in  I^2\), on a :
    \[f (x) = Tn(x) + Rn(x)\]
    avec
    \[ T_n(x) = \sum^n_{K=0}\frac{(x -  a)^K}{K!} f^{(K)}(a) \text{ et } R_n(x) =\int^x_a \frac{(x -  t)^n}{n!} f^{(n+1)}(t) dt\]
\end{defprop}
\subsection{Inégalité de Taylor-Lagrange}
\begin{defprop}
    Soit \(n \in  N\).\\
    Si \(f\) est une application de classe \(\cal{C}^{n+1}\) sur \(I\) à valeurs dans \(\K\) alors, pour tout couple \((a, x) \in  I^2\), on a :
    \[\abs{f (x) -  T_n(x)} \leq \frac{\abs{x -  a}^{n+1}}{(n + 1)}M_{n+1}\]
    avec
    \[M_{n+1} \text{ un majorant de }f^{(n+1)} \text{ sur } \intervii{a}{x} (\text{ ou } \intervii{x, a})\]
\end{defprop}
\begin{defprop}[Remarque]
Les deux formules précédentes ont une nature globale contrairement à la formule de Taylor-Young vue dans le chapitre \hyperref[chap:analyse-asymptotique1]{"Analyse asymptotique \((1)\)"} qui elle a une nature locale.
\end{defprop}
% \chapter{Groupe Symétrique}

\minitoc
\section{Généralités}
\subsection{Groupe symétrique}
    Soit \(n \in \Ns\).
\begin{defprop}[Rappel sur le groupe des permutations d’un ensemble]
    Soit \(X\) un ensemble.\\~\\
    L’ensemble des applications de \(X\) dans \(X\) qui sont des bijections est un groupe pour la loi de composition interne \(\circ\), appelé groupe des permutations de l’ensemble \(X\) et noté \(\cal{S}_X\) .
\end{defprop}
\begin{defprop}[Groupe symétrique]
    Le groupe des permutations de \(\interventierii{1}{n}\) est appelé groupe symétrique et souvent noté \(\cal{S}_n\) plutôt que \(\cal{S}_{\interventierii{1}{n}}\).
\end{defprop}
\subsection{Cycles, transposition}
    Soit \(n \in \Ns, n \geq 2\).
\begin{defprop}[Cycle de longueur \(p\)]
    Soit \(p \in \interventierii{2}{n}\).\\
    Tout élément \( \sigma\) de \(\cal{S}_n\) pour lequel il existe des éléments distincts \(a_1,\dots , a_p\) de \(\interventierii{1}{n}\) tel que :
    \[\sigma (a_1) = a_2, \sigma (a_2) = a_3,\dots , \sigma (a_{p-1}) = a_p, \sigma (a_p) = a_1\]
    et
    \[\sigma(k) = k \text{ si } k \in \interventierii{1}{n} \pd \accol{a_1,\dots , a_p}\]
    est dit cycle de longueur \(p\) de \(\interventierii{1}{n}\) et noté \(\sigma = (a_1a_2\dots a_p)\).
\end{defprop}
\begin{defprop}[Transposition] 
    Tout cycle de longueur \(2\) de \(\interventierii{1}{n}\) est aussi appelé transposition de \(\interventierii{1}{n}\)
\end{defprop}
\subsection{Décomposition des permutations en produit de cycles disjoints}
    Soit \(n \in \Ns, n \geq 2\)
\begin{defprop}[Cycles disjoints]
    Les cycles \((a_1a_2\dots a_p)\) et \((b_1b_2\dots b_q)\) de \(\interventierii{1}{n}\) sont dits disjoints si \(\accol{a_1,\dots , a_p} \inter \accol{b_1,\dots , b_q} = \emptyset\).\\~\\
    \underline{Remarques}\\
    \begin{itemize}
        \item Deux cycles disjoints commutent.
        \item Le terme "cycles disjoints" est parfois remplacé par "cycles à supports disjoints" ; le support d’une permutation \(\sigma\) de \(\interventierii{1}{n}\) étant l’ensemble des éléments de \(\interventierii{1}{n}\) qui ne sont pas invariants par \(\sigma\).
    \end{itemize}
\end{defprop}
\begin{defprop}[Théorème de décomposition des permutations (ADMIS)]
    Toute permutation de \(\interventierii{1}{n}\) peut s’écrire comme composée de cycles disjoints. Cette décomposition est unique, à l’ordre près des facteurs.\\
    \underline{Remarque}\\
    Le terme "produit" remplace le terme "composée" si on utilise la notation usuelle \(\sigma\sigma’\) au lieu de \(\sigma \circ \sigma'\).
\end{defprop}
\section{Signature d’une permutation}
    Soit \(n \in \Ns, n \geq 2\).
\subsection{Décomposition d’une permutation en produit de transpositions}
\begin{defprop}
    Toute permutation de \(\interventierii{1}{n}\) peut s’écrire comme composée de transpositions.\\
    \underline{Remarque}\\
    La décomposition d’une permutation de \(\interventierii{1}{n}\) comme composée de transpositions n’est pas unique.
\end{defprop}
\subsection{Théorème d’existence et unicité de la signature (ADMIS)}
\begin{theo}
    Il existe un morphisme de groupes \(\epsilon\) de \((S_n, \circ)\) dans \((\accol{-1, 1}, \times)\) qui envoie les transpositions sur \(-1\).\\
    Il est défini par :
    \[\forall\sigma \in S_n, \epsilon(\sigma) = \prod_{\accol{i,j}\in A} \frac{\sigma(j) - \sigma(i)}{j - i}\]
    où
    \[A = \accol{\accol{i, j} \tq (i, j) \in \interventierii{1}{n}^2}\] .
    Ce morphisme de groupes est unique et appelé signature de \(S_n\).\\~\\
    \underline{Remarques}\\
    \begin{itemize}
        \item La signature d’un cycle de longueur \(p\) est \((-1)^{p-1}\).
        \item La signature sera utilisée dans le chapitre "Déterminants" pour définir formellement le déterminant d’une famille de vecteurs, d’un endormorphisme ou d’une matrice carrée.
    \end{itemize}
\end{theo}

\begin{dem}
    \begin{itemize}
    \item Montrons que \(\epsilon\)  est un morphisme de groupes de \(\paren{\mathcal{S}_n, \circ}\) vers \(\accol{-1, 1}\).\\~\\
        On remarque d’abord que \(\epsilon\)  est bien définie car, pour tout \(\accol{i, j} \in A\), on a \(i\neq j\) .
        \begin{itemize}
            \item  Soit \(\sigma \in  \mathcal{S}_n\).\\~\\
                \(\psi_{\sigma}  : \accol{i, j} \mapsto \accol{\sigma (i), \sigma (j)}\) est une bijection de \(A\) vers \(A\) (de réciproque \(\psi_{\sigma^{-1}}\) ) donc, avec le changement d’indice \(\accol{k, l} = \psi_{\sigma} (\accol{i, j})\), on a \(\prod_{\accol{i,j}\in A} \abs{\sigma (j) - \sigma (i)} = \prod_{\accol{k,l}\in A}\abs{k - l} = \prod_{\accol{i,j}\in A} \abs{j - i} \) donc \(\frac{\prod_{\accol{i,j}\in A} \abs{\sigma (j) - \sigma (i)}}{\prod_{\accol{i,j}\in A} \abs{j - i}} = 1\)  ce qui donne bien \(\epsilon (\sigma ) \in  \accol{-1, 1}\).
            \item Soit \((\sigma , \sigma ') \in  (\mathcal{S}_n)^2\).\\~\\
                \[\epsilon  (\sigma  \circ \sigma ') = \prod_{\accol{i,j}\in A} \frac{\sigma  \circ \sigma '(j) - \sigma  \circ \sigma '(i)}{j - i} = \prod_{\accol{i,j}\in A} \frac{\sigma  \circ \sigma '(j) - \sigma  \circ \sigma '(i)}{\sigma '(j) - \sigma '(i)} \times \prod_{\accol{i,j}\in A} \frac{\sigma '(j) - \sigma '(i)}{j - i}\]
                Avec le changement d’indice \(\accol{k, l} = \psi_{\sigma '} (\accol{i, j})\), on a :
                \[ \epsilon  (\sigma  \circ \sigma ') = \prod_{\accol{k,l}\in A} \frac{\sigma (k) - \sigma (l)}{k - l} \times \prod_{\accol{i,j}\in A} \frac{\sigma '(j) - \sigma '(i)}{j - i} = \epsilon (\sigma ) \times \epsilon  (\sigma ')\]
                ce qui donne bien que \(\epsilon\)  est un morphisme de groupes.
        \end{itemize}
        \conclusion \(\epsilon\)  est un morphisme de groupes de \((\mathcal{S}_n, \circ)\) vers \(\accol{-1, 1}\).
    \item Montrons que \(\epsilon\)  envoie toute transposition de \(\mathcal{S}_n\) vers \(-1\).\\~\\
        Soit \(\tau = ( \quad a \quad b\quad ) \in  \mathcal{S}_n\) une transposition avec \(a < b\) et \(\accol{i, j} \in  A\).
        \begin{itemize}
            \item Si \(i \in  \accol{a,b}\) et \(j \in  \accol{a,b}\) alors
                \[\frac{\tau (j) - \tau (i)}{j - i} = \frac{\tau (b) - \tau (a)}{b - a} = \frac{a - b}{b - a} = -1\]
            \item Si \(i \notin  \accol{a,b}\) et \(j = a\) alors
                \[\frac{\tau (j) - \tau (i)}{j - i }= \frac{\tau (a) - \tau (i)}{a - i} = \frac{b - i }{a - i} \]
            \item Si \(i \notin  \accol{a,b}\) et \(j = b\) alors
                \[\frac{\tau (j) - \tau (i)}{j - i} = \frac{\tau (b) - \tau (i)}{b - i} = \frac{a - i}{b - i}\]
            \item Si \(i \notin  \accol{a,b}\) et \(j \notin  \accol{a,b}\) alors
                \[\frac{\tau (j) - \tau (i)}{j - i} = \frac{j - i}{j - i} = 1\]
        \end{itemize}
        Les ensembles
        \[A_1 = \accol{\accol{a,b}} , A_2 = \accol{\accol{i, j} \in  A \tq i \notin  \accol{a,b} \text{ et } j \in  \accol{a,b}}\]
        et
        \[A_3 = \accol{\accol{i, j} \in  A \tq i \notin  \accol{a,b} \text{ et }j \notin  \accol{a,b}}\]
        formant une partition de \(A\), on en déduit que :
        \[\epsilon (\tau ) = \prod_{\accol{i,j}\in A_1} \frac{\tau (j) - \tau (i)}{j - i} \times \prod_{\accol{i,j}\in A_2} \frac{\tau (j) - \tau (i)}{j - i} \times \prod_{\accol{i,j}\in A_3}\frac{\tau (j) - \tau (i)}{j - i}\]
        donc que
        \[\epsilon (\tau ) = -1 \times \prod_{\accol{i,j}\in A_2}\underbrace{\paren{\frac{ b - i}{a - i} \times \frac{ a - i}{b - i}}}_{=1} \times \prod_{\accol{i,j}\in A_3} 1 \]
        et enfin \(\epsilon (\tau ) = -1\).
    \item Montrons que \(\epsilon\)  est l’unique morphisme de groupes de \((\mathcal{S}_n, \circ)\) vers \(\accol{-1, 1}\) qui envoie les transpositions sur \(-1\).\\~\\
        Supposons qu’il existe un autre morphisme de groupes, noté \(\epsilon '\), de \((\mathcal{S}_n, \circ)\) vers \(\accol{-1, 1}\) qui envoie les transpositions sur \(-1\).\\~\\
        Soit \(\sigma  \in  \mathcal{S}_n\). Alors \(\sigma\)  peut s’écrire comme composée de p transpositions notées \(\tau_1, \tau_2, \dots , \tau_p\) :
        \[\sigma  = \tau_1 \circ \tau_2 \circ \dots \circ \tau_p\]
        Alors, par hypothèse sur les applications \(\epsilon\)  et \(\epsilon '\), on a :
        \[\epsilon (\sigma ) = \epsilon  (\tau_1) \epsilon  (\tau_2) \dots \epsilon  (\tau_p) = (-1)^{p} \text{ et }\epsilon '(\sigma ) = \epsilon ' (\tau_1) \epsilon ' (\tau_2) \dots \epsilon ' (\tau_p) = (-1)^p\]
        donc \(\epsilon (\sigma ) = \epsilon ' (\sigma )\).\\
        \conclusion \(\epsilon  = \epsilon '\) d’où l’unicité du morphisme de groupes vérifiant les conditions souhaitées.
    \end{itemize}
\end{dem}
% \chapter{Déterminants}

\minitoc

Dans ce chapitre, \(\K = \R\) ou \(\C\).
\section{Formes \(n-\)linéaires alternées}
Soit \(E\) un espace vectoriel sur \(\K\) de dimension finie non nulle \(n\).
\subsection{Définition}
\begin{defi}
    Une application \(f : E^n \to \K\) est dite forme \(n-\)linéaire alternée sur \(E\) si :
    \begin{enumerate}
        \item \(\forall j \in  \interventierii{1}{n} , \forall  (x_1, \dots  , x_{j-1}, x_{j+1}, \dots  , x_n) \in  E^{n-1}, (x \mapsto f \paren{x_1, \dots  , x_{j-1}, x, x_{j+1}, \dots  , x_n}) \in  \cal{L} (E, \K)\).
        \item \(f\) s’annule en tout \((x_1, \dots  , x_n) \in  E^n\) pour lequel il existe \((j, k) \in  \interventierii{1}{n}^2 \) avec \(j\neq k\) et \(x_j = x_k\).
    \end{enumerate}
\end{defi}

\subsection{Propriétés}
\begin{defprop}[Effet sur les familles liées]
    Si \(f\) est une forme \(n-\)linéaire  alternée sur \(E\) et \((x_1, \dots  , x_n)\) une famille de vecteurs de \(E\) liée alors \[f (x_1, \dots  , x_n) = 0\]
\end{defprop}
\begin{defprop}[Antisymétrie]
    Si \(f\) est une forme \(n-\)linéaire  alternée sur \(E\) alors \(f\) est antisymétrique, c’est-à-dire que :
    \[\forall  (x_1, \dots  , x_n) \in  E^n, \forall  (j, k) \in  \interventierii{1}{n}^2, j < k \imp f (x_1, \dots  , x_j , \dots  , x_k, \dots  , x_n) = -f (x_1, \dots  , x_k, \dots  , x_j , \dots  , xn) .\]
    \underline{Remarque}\\
    En notant \(\epsilon\)  la signature de \(\cal{S}_n\), l’antisymétrie de \(f\) se traduit par : pour tout \((x_1, \dots  , x_n) \in  E^n\) et pour toute transposition \(\theta\) de \(\interventierii{1}{n}\) , \(f\paren{x_{\theta (1)}, \dots  , x_{\theta (n)}} = \epsilon (\theta )f (x_1, \dots  , x_n) .\)
\end{defprop}
\begin{defprop}[Effet d’une permutation]
    Si \(f\) est une forme \(n-\)linéaire  alternée sur \(E\) alors :
    \[\forall \sigma  \in  \cal{S}_n, \forall  (x_1, \dots  , x_n) \in  E^n, f \paren{ x_{\sigma (1)}, \dots  , x_{\sigma (n)}} = \epsilon (\sigma )f \paren{x_1, \dots  , x_n} .\]
\end{defprop}

\section{Déterminant d’une famille de vecteurs dans une base}
    Soit \(E\) un espace vectoriel sur \(\K\) de dimension finie non nulle \(n\) muni d’une base \(\cal{B} = (e_1, \dots  , e_n)\) .
\subsection{Théorème}
\begin{theo}
    \begin{enumerate}
        \item Il existe une unique forme \(n-\)linéaire  alternée \(f\) sur \(E\) telle que \(f (\cal{B}) = f (e_1, \dots  , e_n) = 1\).
            Cette forme \(n-\)linéaire  alternée sur \(E\) est notée \(\det_{\cal{B}}\) et vérifie :
                \[\det_{\cal{B}}(\cal{B}) = 1.\]
        \item Toute forme \(n-\)linéaire  alternée \(g\) sur \(E\) est un multiple de \(\det_{\cal{B}}\) avec, plus précisément :
            \[g = g (\cal{B}) \det_{\cal{B}}.\]
    \end{enumerate}
\end{theo}
\begin{dem}
    On note \(\cal{A}_n\) l’ensemble des formes \(n-\)linéaires alternées sur \(E\) et on considère \(g \in \cal{A}_n\).\\~\\
    Soit \((x_1, \dots , x_n) \in E^n\).\\~\\
    Pour tout \(j \in \interventierii{1}{n}\) , on note \(x_j = \sum^n_{i=1} a_{i,j} e_i\) la décomposition de \(x_j\) dans la base \(\cal{B} = \paren{e_1, \dots , e_n}\) de \(E\).\\~\\
    Alors, par \(n-\)linéarité de \(g\), on a : \(g (x_1, \dots , x_n) = \sum_{(i_1,\dots,i_n)\in \interventierii{1}{n}^n} a_{i_1,1} \dots a_{i_n,n} \underbrace{g (e_{i_1} , \dots , e_{i_{n}} )}_{\substack{= 0\\ \text{ Si deux indices égaux}}}\)\\~\\
    Vu le caractère alterné de \(g\), on ne conserve dans la somme que les termes correspondant aux \(n-\)uplets \((i_1, \dots , i_n)\) d’éléments deux à deux distincts, c’est-à-dire les \(n-\)uplets \((\sigma (1), \dots , \sigma (n))\) avec \(\sigma\)  qui décrit le groupe symétrique \(\cal{S}_n\). Ainsi :
    \[g (x_1, \dots , x_n) = \sum_{\sigma \in\cal{S}_n} a_{\sigma (1),1} \dots a_{\sigma (n),n} g \paren{e_{\sigma (1)}, \dots , e_{\sigma (n)}}\]
    Comme \(g\) est forme \(n-\)linéaire alternée sur \(E\), on a : \(\forall \sigma  \in \cal{S}_n, g \paren{e_{\sigma (1)}, \dots , e_{\sigma (n)}  = \epsilon(\sigma )g(e_1, \dots , e_n)}\) où \(\epsilon(\sigma )\) est la signature de \(\sigma\) . Cela donne, après factorisation,
    \[\forall  (x_1, \dots , x_n) \in E^n, g (x_1, \dots , x_n) = \paren{\sum_{\sigma \in\cal{S}_n} \epsilon(\sigma ) \prod^n_{i=1} a_{\sigma (i),i}} \underbrace{g (e_1, \dots , e_n)}_{\in \K} \text{ donc } g = g (e_1, \dots , e_n)f \qquad \star\]

    en notant \(f : E^n \to \K\) la forme \(n-\)linéaire alternée définie par \(f : (x_1, \dots , x_n) \mapsto \paren{\sum_{\sigma \in\cal{S}_n}\epsilon(\sigma)\prod^n_{i=1} a_{\sigma (i),i}}\) avec \((a_{k,i})_{1\geq k\geq n}\) famille des coordonnées de \(x_k\).\\~\\
    Pour tout \(j \in \interventierii{1}{n}\) , on a \(e_j =\sum^n_{i=1}\delta_{i,j} e_i\) donc \(f (e_1, \dots , e_n) = \epsilon(\id{\cal{S}_n})\prod^n_{i=1} \delta_{\id{\N}(i),i}\) puis \(f (e_1, \dots , e_n) = 1\). La relation \((\star)\) assure de plus que \(f\) est la seule forme \(n-\)linéaire alternée sur \(E\) vérifiant cette égalité.\\~\\
    D’après \((\star)\), on a l’inclusion \(\cal{A}_n \Subset \Vect{F}\) puis \(\cal{A}_n = \Vect{f}\) (car tout \(g\) de \(\Vect{f}\) est forme \(n-\)linéaire alternée sur \(E\) puisque \(f\) l’est) avec \(\Vect{f}\) espace vectoriel de dimension \(1\) (car \(f\) n’est pas nulle puisqu’elle vaut \(1\) en \((e_1, \dots , e_n)\)).\\~\\
    \conclusion\\~\\
     L’ensemble des formes \(n-\)linéaires alternées sur \(E\) est un espace vectoriel de dimension \(1\) engendré par \(f : (x_1, \dots , x_n) \mapsto \paren{\sum_{\sigma \in\cal{S}_n} \epsilon(\sigma ) \prod^n_{i=1} a_{\sigma (i),i}}\), seule forme \(n-\)linéaire alternée sur \(E\) qui vérifie \(f (\cal{B}) = 1\).\\~\\
     En notant \(f = \det_{\cal{B}}\) (dit déterminant dans la base \(\cal{B}\) de \(E\)) , on a donc \(g = g (\cal{B}) \det_{\cal{B}} \) pour toute forme \(n-\)linéaire alternée \(g\) sur \(E\).
\end{dem}
\subsection{Déterminant d’une famille de vecteurs dans une base}
\begin{defi}
    Soit \((x_1, \dots  , x_n)\) une famille de \(n\) vecteurs de \(E\).\\~\\
    Le scalaire \(\det_{\cal{B}} (x_1, \dots  , x_n)\) est dit déterminant de la famille \((x_1, \dots  , x_n)\) dans la base \(\cal{B}\).
\end{defi}
\begin{defprop}[Expression du déterminant avec les coordonnées]
    Si \((x_1, \dots  , x_n)\) est une famille de \(n\) vecteurs de \(E\) et \(A = (a_{ij} )_{(i,j)\in \interventierii{1}{n}} = \Mat{\cal{B}}{x_1, \dots  , x_n}\) alors :
    \[\det_{\cal{B}}(x_1, \dots  , x_n) = \sum_{\sigma \in \cal{S}_n} \epsilon (\sigma ) \paren{\prod^{n}_{j=1} a_{\sigma (j)j}}\]

    \underline{Remarque}\\
    A l’issue de ce chapitre, on disposera de moyens plus pratiques que le recours à cette formule théorique pour calculer le déterminant d’une famille de vecteurs.
\end{defprop}

\begin{defprop}[Cas particuliers des dimension \(2\) et \(3\)]
    \begin{itemize}
        \item Si \(E\) est de dimension \(2\) et \((x, y)\) une famille de deux vecteurs de \(E\) de matrice \(\begin{pmatrix}
              x_{1} & y_{1} \\
              x_{2} & y_{2}
              \end{pmatrix}\) dans une base \(\cal{B}\) de \(E\) alors :
            \[\det_{\cal{B}}(x, y) = x_1y_2 - x_2y_1 \underset{\text{Not.}}{=}\begin{vmatrix}
              x_{1} & y_{1} \\
              x_{2} & y_{2}
              \end{vmatrix}\]
        \item Si \(E\) est de dimension \(3\) et \((x, y, z)\) une famille de trois vecteurs de \(E\) de matrice \(\begin{pmatrix}
              x_{1} & y_{1} & z_{1} \\
              x_{2} & y_{2} & z_{2} \\
              x_{3} & y_{3} & z_{3}
              \end{pmatrix}\) dans une base \(\cal{B}\) de \(E\) alors :
            \[\det_{\cal{B}}(x, y, z) = x_1 y_2 z_3 + x_2 y_3 z_1 + x_3 y_1 z_2 - x_3 y_2 z_1 - x_2 y_1 z_3 - x_1 y_3 z_2 \underset{\text{Not.}}{=} \begin{vmatrix}
              x_{1} & y_{1} & z_{1} \\
              x_{2} & y_{2} & z_{2} \\
              x_{3} & y_{3} & z_{3}
              \end{vmatrix}\]
    \end{itemize}
\end{defprop}
\subsection{Formule de changement de bases}
\begin{defprop}
    Si \(\cal{B}'\) est une autre base de \(E\) alors \(\det_{\cal{B}'} = \det_{\cal{B}'}(\cal{B}) \det_{\cal{B}}\) c’est-à-dire que :
    \[\forall  (x_1, \dots  , x_n) \in  E, det_{\cal{B}'} (x_1, \dots  , x_n) = \det_{\cal{B}'}(\cal{B}) \det_{\cal{B}} (x_1, \dots  , x_n) \]
\end{defprop}
\subsection{Caractérisation des bases par le déterminant}
\begin{defprop}
    Une famille \(\cal{B}' = (x_1, \dots  , x_n)\) de \(n\) vecteurs de \(E\) est une base de \(E\) si, et seulement si, \(\det_{\cal{B}}(\cal{B}')\neq 0\) avec dans ce cas,
    \[\det_{\cal{B}}(\cal{B}') = (\det_{\cal{B}'}(\cal{B}))^{-1} \]
    \underline{Remarque}\\
    En \(2^e\) année MPI, les notions d’orientation d’un \(\R-\)espace vectoriel de dimension finie et de bases directes ou indirectes seront définies en utilisant la notion de déterminant.
\end{defprop}
\section{Déterminant d’un endomorphisme ou d’une matrice carrée}
\subsection{Cas d’un endomorphisme en dimension finie}
    Soit \(E\) un espace vectoriel sur \(\K\) de dimension finie non nulle.
\begin{theo}
    Soit \(u\) un endomorphisme de \(E\).\\~\\
    Le scalaire \(\det_{\cal{B}} (u(\cal{B}))\), qui ne dépend pas de la base \(\cal{B}\) de \(\cal{E}\) considérée, est appelé déterminant de \(u\) et noté \(\det(u)\) :
    \[\det (u) = \det_{\cal{B}} (u(\cal{B})) \text{ avec }\cal{B} \text{ base quelconque de } E\]
    \underline{Exemples simples}\\
    \begin{itemize}
        \item En particulier, \( \det\paren{ 0_{\cal{L}(E)}} = 0\) et \(\det (\id{E} ) = 1\).
        \item Plus généralement,
        \begin{itemize}
            \item le déterminant d’une homothétie \(h_\alpha\)  de \(E\) de rapport \(\alpha\)  est : \(\det (h_{\alpha} ) = \alpha  \dim E\) .
            \item le déterminant d’une projection \(p\) de \(E\) autre que l’identité est : \(\det (p) = 0\).
            \item de déterminant d’une symétrie \(s\) de \(E\) est : \(\det (s) = (-1)^{\dim E-\dim F}\) où \( F = \ker(s - \id{E} )\)
        \end{itemize}
    \end{itemize}
\end{theo}
\begin{dem}
    On considère l’application \(g : E^n \to \K\) définie par
    \[\forall  (x_1, \dots , x_n) \in E^n, g (x_1, \dots , x_n) = \det_{\cal{B}'}(u(x_1), \dots , u(x_n))\]
    Montrons que \(g(\cal{B}') = \det_{\cal{B}} (u(\cal{B}))\) ce qui donnera le résultat souhaité.\\~\\
    Comme \(\det_{\cal{B}'}\) est une forme \(n-\)linéaire alternée sur \(E\) et que \(u\) est un endomorphisme de \(E\), on vérifie aisément que \(g\) est une forme \(n-\)linéaire alternée sur \(E\).\\~\\
    Par théorème, on a donc :
    \[g = g(\cal{B}) \det_{\cal{B}}\]
    En particulier :
    \[g(\cal{B}') = g(\cal{B}) \det_{\cal{B}} (\cal{B}') \]
    Or \(g(\cal{B}) = \det_{\cal{B}'} (u(\cal{B}))\) (par définition de \(g\)) et \(\det_{\cal{B}'} (u(\cal{B})) = \det_{\cal{B}'} (\cal{B}) \det_{\cal{B}}(u(\cal{B}))\) (par formule de changement de bases) donc :
    \[g(\cal{B}') = \det_{\cal{B}'} (\cal{B}) \det_{\cal{B}} (\cal{B}') \det_{\cal{B}} (u(\cal{B}))\]
    Comme \(\det_{\cal{B}}' (\cal{B}) \det_{\cal{B}} (\cal{B}') = \det_{\cal{B}'} (\cal{B}')\) (par formule de changement de bases) et \(\det_{\cal{B}'} (\cal{B}') = 1\) (par définition de \(\det_{\cal{B}'}\) ), on en déduit que :
    \[g(\cal{B}') = \det_{\cal{B}} (u(\cal{B}))\]
    autrement dit
    \[\det_{\cal{B}'} (u(\cal{B}')) = \det_{\cal{B}} (u(\cal{B})) \]
\end{dem}
\begin{prop}
    \begin{enumerate}
        \item \(\forall \lambda \in  \K, \forall u \in  \cal{L}(E), \det (\lambda u) = \lambda ^{\dim E} det (u)\)
        \item \(\forall (u, v) \in  (\cal{L}(E))^2 , \det (v \circ u) = \det (v) \det (u) = \det (u) \det (v)\)
        \item \(\forall q \in \N, \forall u \in  \cal{L}(E) , \det (u^q) = (\det u)^q \)
    \end{enumerate}
\end{prop}

\begin{defprop}[Caractérisation des automorphismes avec le déterminant]
    Un endomorphisme \(u\) de \(E\) est bijectif si, et seulement si, \(\det (u)\neq 0\) avec dans ce cas,
    \[det (u^{-1}) = (\det (u))^{-1}\]
\end{defprop}
\begin{defprop}[Morphisme de groupes de \(\cal{GL}(E)\) vers \(\Ks\)]
    L’application \(u \in  \cal{GL}(E) \mapsto \det (u)\) est un morphisme de groupes de \(\paren{\cal{GL}(E), \circ}\) vers \(\paren{\Ks, \times}\) .
\end{defprop}
\subsection{Déterminant d’une matrice carrée}
    Soit \(n \in  \Ns\).
\begin{defi}
    On appelle déterminant d’une matrice \(A \in  \M{n}\), et on note \(\det (A)\), le déterminant de l’endomorphisme de \(\K^n\) canoniquement associé à \(A\).\\~\\
    \underline{Exemples simples}\\
    En particulier, \(\det \paren{0_{\M{n}}} = 0\) et \(\det (I_n) = 1\).  
\end{defi}
\begin{defprop}[Caractère \(n-\)linéaire  alterné du déterminant par rapport aux colonnes]
    Si \(C_1, \dots  , C_n\) sont les colonnes de \(A \in  \M{n}\) et \(B_{can}\) est la base canonique de \(\K^n\) alors
    \[\det(A) = \det_{B_{can}} (C_1, \dots  , C_n)\]
    donc le déterminant d’une matrice carrée est \(n-\)linéaire  alterné par rapport aux colonnes.
\end{defprop}

\begin{prop}
    Soit \(A\) et \(\cal{B}\) deux matrices de \(\M{n}, \lambda \in  \K\) et \(q \in \N\).
    \begin{enumerate}
        \item \(\det(A) = -det(A')\) où \(A'\) est la matrice obtenue en échangeant deux colonnes de \(A\).
        \item Si deux colonnes de \(A\) sont égales (resp. si les colonnes de \(A\) sont liées) alors \(\det(A) = 0\).
        \item \(\det(\lambda A) = \lambda^n \det(A)\)
        \item \(\det(AB) = \det(A) \det(\cal{B}) = \det(\cal{B}) \det(A)\).
        \item \(\det(A^q) = (\det(A))^q\).
    \end{enumerate}
\end{prop}

\begin{defprop}[Caractérisation des matrices inversibles avec le déterminant]
    Une matrice \( A \in  \M{n}\) est inversible si, et seulement si, \(\det(A) \neq 0\) avec, dans ce cas, \[\det(A^{-1}) = (\det(A))^{-1}\]
\end{defprop}

\begin{defprop}[Morphisme de groupes de \(\cal{GL}_n(\K)\) vers \(\Ks\)]
    L’application \(A \in  \cal{GL}_n(\K) \mapsto \det(A)\) est un morphisme de groupes de \((\cal{GL}_n(\K), \times)\) vers \((\Ks, \times)\) .
\end{defprop}

\begin{defprop}[Déterminant de matrices semblables]
    Si \(A\) et \(\cal{B}\) sont des matrices semblables alors \(\det(A) = \det(\cal{B})\).
\end{defprop}
\begin{defprop}[Expression du déterminant à l’aide des coefficients de la matrice]
    Si \(A = (a_{ij} )_{(i,j)\in \interventierii{1}{n}^2} \in \M{n}\) alors
    \[\det (A) = \sum_{\sigma \in \cal{S}_n} \epsilon (\sigma ) \paren{\prod^n_{j=1} a_{\sigma(j)j}}\]
\end{defprop}

\begin{defprop}[Déterminant et transposition]
    Les déterminants d’une matrice carrée et de sa transposée sont égaux :
    \[\forall A \in  \M{n} , \det(\trans{A}) = \det(A)\]
    \underline{Remarque}\\
    Les propriétés vues sur le déterminant d’une matrice carrée relatives à ses colonnes s’étendent aux lignes.
\end{defprop}

\subsection{Calcul de déterminants en pratique}
    On note \(C_1, \dots  , C_n\) les colonnes (resp. \(L_1, \dots  , L_n\) les lignes) de \(A = (a_{ij} )_{1\leq i,j\leq n} \in  \M{n}\).
\begin{defprop}[Effet des opérations élémentaires sur un déterminant]
    \begin{enumerate}
        \item L’opération élémentaire \(C_r \leftarrow \lambda C_r\) multiplie le déterminant par \(\lambda\).
        \item L’opération élémentaire \(C_r \leftrightarrow C_s\) avec \(r\neq s\) multiplie le déterminant par \(-1\).
        \item L’opération élémentaire \(C_r \leftarrow C_r + \lambda C_s\) avec \(r\neq s\) ne modifie pas le déterminant.
    \end{enumerate}
    \underline{Remarque}\\
    On obtient le même résultat avec les opérations élémentaires sur les lignes.
\end{defprop}

\begin{defprop}[Développement suivant une colonne (ou une ligne)]
    Pour \((i, j) \in  \interventierii{1}{n}^2\), on appelle cofacteur du coefficient \(a_{ij}\) de \(A\) le scalaire noté \(A_{ij}\) défini par :
    \[A_{ij} = (-1)^{i+j} M_{ij}\]
    où \(M_{ij}\) est le déterminant de la matrice extraite de \(A\) obtenue en supprimant \(L_i\) et \(C_j\) .
    \begin{enumerate}
        \item Formule de calcul du déterminant par développement suivant la \(j^e\) colonne.
        \[\det(A) = a_{1j} A_{1j} + a_{2j} A_{2j} + \dots + a_{nj} A_{nj} = \sum^n_{i=1} a_{ij} A_{ij}\] 
        \item Formule de calcul du déterminant par développement suivant la \(i^e\) ligne.
        \[\det(A) = a_{i1}A_{i1} + a_{i2}A_{i2} + \dots + a_{in}A_{in} = \sum^n_{j=1} a_{ij} A_{ij} \]
    \end{enumerate}
    \underline{Remarque :}\\
    Sauf cas particulier, avant d’utiliser ces formules de développement, on effectuera des opérations élémentaires pour obtenir un maximum de zéros ou un facteur commun sur une ligne/colonne.
\end{defprop}
\begin{dem}
    On note \(C_1, \dots , C_n\) les colonnes (resp \(L_1, \dots , L_n\) les lignes) de \(A\) et \(\cal{B} = (e_1, \dots , e_n)\) la base usuelle de \(\K^n\).\\~\\
    Soit \(j \in \interventierii{1}{n}\).\\
    Par définition du déterminant de \(A\),on a :
    \[\det (A) = \det_{\cal{B}} (C_1, \dots , C_j , \dots , C_n) = \det_{\cal{B}}\paren{ C_1, \dots , \sum^n_{i=1} a_{ij} e_i, \dots , C_n}\]
    Par linéarité du déterminant d’une matrice par rapport à chacune des ses colonnes, on trouve :
    \[\det (A) = \sum^n_{i=1} a_{ij} \det_{\cal{B}} \paren{C_1, \dots , e_i, \dots , C_n} \qquad(\star)\]
    On note
    \[A_{ij} = \det_{\cal{B}} \paren{C_1, \dots C_{j-1}, e_i, C_{j+1}, \dots , C_n} \]
    Comme l’échange de deux colonnes multiplie le déterminant par \(-1\), en intervertissant successivement les colonnes \(j\) et \(j + 1\), puis \(j + 1\) et \(j + 2\) jusqu’aux colonnes \(n - 1\) et \(n\) ( ce qui fait \(n - j\) interversions), on trouve :
    \[A_{ij} = (-1)^{n-j} \det_{\cal{B}} \paren{C_1, \dots C_{j-1}, C_{j+1}, C_{j+2}, \dots , C_n, e_j } \].
    On procède de même avec les lignes en intervertissant successivement les lignes \(i\) et \(i + 1\), puis \(i + 1\) et \(i + 2\) jusqu’aux lignes \(n - 1\) et \(n\) (ce qui fait \(n - i\) interversions) et on trouve :
    \[A_{ij} = (-1)^{n-j} (-1)^{n-i} \det 
    \begin{pmatrix}
    a_{1,1}   & \dots & a_{1,j-1}   & a_{1,j+1}   & \dots & a_{1,n}  & 0      \\
    \vdots    &       & \vdots      & \vdots      &       & \vdots   & \vdots \\
    a_{i-1,1} & \dots & a_{i-1,j-1} & a_{i-1,j+1} & \dots & a_{i-1,n}& 0      \\
    a_{i+1,1} & \dots & a_{i+1,j-1} & a_{i+1,j+1} & \dots & a_{i+1,n}& 0      \\
    \vdots    &       & \vdots      & \vdots      &       & \vdots   & \vdots \\
    a_{n,1}   & \dots & a_{n,j-1}   & a_{n,j+1}   & \dots & a_{n,n}  & 0      \\
    a_{i,1}   & \dots & a_{i,j-1}   & a_{i,j+1}   & \dots & a_{i,n}  & 1 
    \end{pmatrix}
    \]
    Ainsi, \(A_{ij} = (-1)^{i+j} \det B_{ij}  \qquad (\star\star)\) où \(B_{ij}\) est la matrice définie par blocs
    \[B_{ij} =\begin{pmatrix}
        C_{ij} & 0_{n-1,1} \\
        D_{ij} & 1         
    \end{pmatrix}\]
    avec
    \begin{itemize}
        \item \(C_{ij}\) matrice extraite de \(A\) en otant \(L_i\) et \(C_j\)
        \item \(D_{ij}\) ligne extraite de \(L_i\) en otant sa colonne \(j\).
    \end{itemize}
    Montrons que \(\det B_{ij} = \det C_{ij}\)\\
    (cas particulier d’un résultat général qui sera montré en MPI : le déterminant d’une matrice triangulaire par blocs est égal au produit des déterminants de ses blocs diagonaux)\\~\\
    Pour simplifier, on note \(B\) (resp. \(C\)) et pas \(B_{ij}\) (resp. \(C_{ij}\) ) avec \(\cal{B} = (b_{i,j} )_{(i,j) \in\interventierii{1}{n}^2}\) et \(C = (c_{i,j})_{(i,j)\in \interventierii{1}{n-1}^2}\).\\~\\
    On sait que
    \[\det B = \sum_{\sigma \in\cal{S}_n} \epsilon(\sigma )b_{\sigma (1),1}b_{\sigma (2),2} \dots b_{\sigma (n),n}\]
    avec, par définition de \(B, b_{\sigma (n),n} = 
    \begin{cases}
        0 \text{ si }\sigma (n)\neq n \\
        1 \text{ si }\sigma (n) = n\\
    \end{cases}
    \)\\
    Ainsi :
    \[\det B = \sum_{\sigma \in T_n} \epsilon(\sigma )b_{\sigma (1),1}b_{\sigma (2),2} \dots b_{\sigma (n-1),n-1}\]
    avec \(T_n = \accol{\sigma  \in \cal{S}_n \tq \sigma (n) = n}\) (\ie l’ensemble des permutations de \(\interventierii{1}{n}\) qui laissent \(n\) invariant).\\~\\
    \(\psi : T_n \to \cal{S}_{n-1}\) définie par \(\forall \sigma  \in T_n, \psi(\sigma ) = \sigma '\) avec \(\sigma  \in \cal{S}_{n-1}\) tel que \(\forall i \in \interventierii{1}{n-1} , \sigma '(i) = \sigma (i)\)
    \begin{itemize}
        \item est bijective (de bijection réciproque \(\psi : \cal{S}_{n-1} \to T_n\) définie par \(\forall \sigma ' \in \cal{S}_{n-1}, \psi(\sigma ') = \sigma\)  avec \(\sigma  \in \cal{S}_n\) tel que \(\forall i \in \interventierii{1}{n-1} , \sigma (i) = \sigma '(i)\) et \(\sigma (n) = n\))
        \item  conserve la signature des permutations \ie \(\forall \sigma  \in T_n, \epsilon (\psi(\sigma )) = \epsilon(\sigma )\) car, comme \(\sigma\)  et \(\sigma ' = \psi(\sigma )\)coïncident sur \(\interventierii{1}{n-1}\) et que \(n\) est invariant par \(\sigma\) , les permutations \(\sigma\)  et \(\psi(\sigma )\) peuvent se décomposer en le même produit de transpositions donc ont même signature.
    \end{itemize}
    On en déduit, avec le changement d’indice bijectif \(\sigma ' = \psi(\sigma )\), que 
    \[\det B = \sum_{\sigma '\in\cal{S}_{n-1}}\epsilon(\sigma ')b_{\sigma '(1),1}b_{\sigma '(2),2} \dots b_{\sigma '(n-1),n-1}\]
    Vu le lien entre les coefficients de \(C\) et ceux de \(\cal{B}\), on en déduit que \(\det \cal{B} = \sum_{\sigma '\in\cal{S}_{n-1}} \epsilon(\sigma ')c_{\sigma '(1),1}c_{\sigma '(2),2} \dots c_{\sigma '(n-1),n-1}\) ce qui donne, par formule théorique du déterminant de \(C\) :
    \[\det B = \det C\]
    \conclusion avec les résultats \((\star)\) et \((\star\star)\) trouvés, on en déduit que :
    \[\det(A) = a_{1j} A_{1j} + \dots + a_{nj} A_{nj} = \sum^n_{i=1} a_{ij} A_{ij} \]
    où \(A_{ij} = (-1)^{i+j} M_{ij}\) où \(M_{ij} = \det (C_{ij} )\) avec \(C_{ij}\) , matrice extraite de \(A\) en otant \(L_i\) et \(C_j\).
\end{dem}

\begin{defprop}[Déterminant de matrices particulières]
    \begin{enumerate}
        \item Le déterminant d’une matrice triangulaire est le produit de ses éléments diagonaux.
        \item Déterminant de la matrice de Vandermonde pour \((x_1, \dots  , x_n) \in  \K^n \).
        \[\det\begin{pmatrix}
            1 & x_1 & \dots & x_{n-1}\\
            1 & x_2 & \dots & x_{n-1}\\
            \vdots & \vdots &&\vdots \\
            1 & x_n \dots  x^{n-1}_n 
            \end{pmatrix} = \prod_{1\leq s<t\leq n}(x_t - x_s)
        \]
    \end{enumerate}
\end{defprop}

\subsection{Comatrice}
\begin{defi}
    Soit \(A = (a_{ij} )_{(i,j)\in \interventierii{1}{n}^2} \in  \M{n}\) .\\~\\
    On appelle comatrice de \(A\), et on note \(\Com{A}\), la matrice de \(\M{n}\) définie par 
    \[\Com{A} = (A_{ij} )_{(i,j) \in \interventierii{1}{n}^2}\]
    où \(A_{ij}\) est le cofacteur de l’élément \(a_{ij}\) de \(A\).\\
    \underline{Remarque}\\
    Autrement dit, la comatrice de \(A\) est la matrice des cofacteurs de \(A\).
\end{defi}

\begin{defprop}[Relation liant \(A\) et sa comatrice]
    Pour toute matrice \(A\) de \(\M{n}\), on a :
    \[A \trans{\Com{A}} = \trans{\Com{A}} A = \det (A)I_n.\]
\end{defprop}
\begin{defprop}[Expression de l’inverse d’une matrice]
    Si \(A\) est une matrice inversible de \(\M{n}\) alors
    \[A^{-1} = \frac{1}{\det (A)} \trans{\Com{A}} \]
    \underline{Remarque}\\
    Cette formule est à bannir pour des calculs pratiques lorsque \(n \geq 3\) en raison de la lourdeur des calculs qu’elle implique.
\end{defprop}
% \chapter{Espaces préhilbertiens réels}
\minitoc
Dans ce chapitre, \(E\) est un espace vectoriel sur \(\R\)
\section{Généralités}
\subsection{Produit scalaire}
\begin{defi}
    \begin{enumerate}
        \item On appelle produit scalaire sur \(E\) toute application \(\phi\) de \(E \times E\) dans \(\R\) qui vérifie :
        \begin{enumerate}
            \item \(\forall (x, y) \in E^2, \phi(x, y) = \phi(y, x). \hfill (\phi \text{ symétrique })\)
            \item \(\forall (\lambda , \mu ) \in \R^2, \forall (x, y, z) \in E^3, \phi(x, \lambda y + \mu z) = \lambda \phi (x, y) + \mu \phi (x, z). \hfill (\phi \text{ linéaire à droite })\)
            \item \(\forall x \in E, \phi(x, x) \geq 0. \hfill (\phi \text{ positive})\)
            \item \(\forall x \in E, \phi(x, x) = 0 \imp x = 0_E . \hfill (\phi \text{ définie })\)
        \end{enumerate}
        \underline{Remarques}\\
        \begin{itemize}
            \item Le réel \(\phi(x, y)\) est souvent noté \(\scal{x}{y}\), \(\paren{x | y}\) ou encore \(x \cdot y\).
            \item L’application \(\phi\) est une forme bilinéaire symétrique définie positive sur \(E \times E\).
        \end{itemize}
        \item Un \(\R\) espace vectoriel muni d’un produit scalaire est dit :
        \begin{enumerate}
            \item espace préhilbertien réel ;
            \item espace euclidien s’il est de plus de dimension finie.
        \end{enumerate}
    \end{enumerate}
\end{defi}
\subsection{Quatre exemples usuels à connaître}
\begin{defprop}
    \begin{enumerate}
        \item L’application \(((x_1, \dots , x_n), (y_1, \dots , y_n)) \mapsto x_1y_1 + \dots + x_ny_n\) est un produit scalaire sur \(\R^n\).
        \item L’application \((X, Y ) \mapsto \trans{X}Y\) est un produit scalaire sur \(\M{n,1}{\R}\).
        \item L’application \((A, B) \mapsto \tr \paren{\trans{A}B}\) est un produit scalaire sur \(\M{n,p}{\R}\).
        \item Soit \((a, b) \in \R^2\) avec \(a < b\).
            \[ \text{ L’application }(f, g) \mapsto \int^b_a f (t)g(t) dt \text{ est un produit scalaire sur } \cal{C}([a, b] , \R)\]
    \end{enumerate}
\end{defprop}
\subsection{Inégalités remarquables}
Soit \(E\) un espace préhilbertien réel de produit scalaire noté \(\scal{.}{.}\).
\begin{defprop}[Inégalité de Cauchy-Schwarz]
    \[\forall (x, y) \in E^2, \abs{\scal{x}{y}} \leq \sqrt{\scal{x}{x}}\sqrt{\scal{y}{y}}\]
    \underline{Remarque}\\
    Il y a égalité dans l’inégalité précédente si, et seulement si, la famille \((x, y)\) est liée.
\end{defprop}
\begin{dem}
    Soit \((x,y) \in E^2\), on note \(P : \lambda \mapsto \scal{\lambda x +y}{\lambda x +y}\)\\~\\
    Ainsi \(\forall \lambda \in \R, \begin{cases}
        P(\lambda) \geq 0  \text{ par positivité du produit scalaire} \\
        P(\lambda) = \lambda \scal{x}{\lambda x +y} + \scal{y}{ \lambda x +y} = \lambda^2 \scal{x}{x} + 2\lambda \scal{x}{y} + \scal{y}{y}
    \end{cases}\)
    \begin{itemize}
        \item Cas où \(x \neq 0_E\)\\~\\
            Alors \(\scal{x}{x} \neq 0\) donc \(P\) est fonction polynomiale de degré \(2\) positive donc \(\Delta \leq 0\) avec \(\Delta = 4\scal{x}{y}^2 - 4\scal{x}{x}\scal{y}{y}\)\\
            ainsi \(\scal{x}{y}^2\leq \scal{x}{x}\scal{y}{y}\) donc \(\abs{\scal{x}{y}} \leq \sqrt{\scal{x}{x}}\sqrt{\scal{y}{y}}\)
        \item Si \(x = 0_E\) alors l'inégalité est triviale car \(\begin{cases}
            \scal{x}{y} &=0 \\
            \scal{x}{x}\scal{y}{y} &= 0
        \end{cases}\)
        
    \end{itemize}
    \underline{Etude du cas d'égalité}\\~\\
    Pour \( x = 0_E\) il y a égalité trivialement\\~\\
    Pour \(x \neq 0_E\), l'égalité implique \(\Delta = 0\) donc \(\exists \lambda_0 \in \R, P(\lambda_0) = 0\) ce qui implique par le caractère définie du produit scalaire que \(\lambda_0 x + y = O_E\) et donc que \((x,y)\) est liée\\~\\
    Ainsi si il y a égalité avec l'inégalité de Cauchy-Schwarz alors \((x,y)\) est liée\\~\\
    Réciproquement, si \((x,y)\) est liée alors sans perte de généralité \(\exists \lambda \in \R\) tel que \(x = \lambda y\)\\~\\
    \(\abs{\scal{x}{y}} = \abs{\lambda \scal{y}{y}} = \abs{y}\scal{y}{y}\)\\
    \(\sqrt{\scal{x}{x}}\sqrt{\scal{y}{y}} = \sqrt{\lambda^2\scal{y}{y}}\sqrt{\scal{y}{y}} = \abs{\lambda}\scal{y}{y}\)
    Ce qui démontre bien que \(\abs{\scal{x}{y}} =\sqrt{\scal{x}{x}}\sqrt{\scal{y}{y}} \)
\end{dem}
\begin{defprop}[négalité de Minkowski]
    \[\forall (x, y) \in E^2, \sqrt{\scal{x + y}{x+y}} \leq \sqrt{\scal{x}{y}} + \sqrt{\scal{y}{y}}\]
    \underline{Remarque}\\
    Il y a égalité dans l’inégalité précédente si, et seulement si, \(x = 0_E\) ou (\(x\neq 0_E \)et \(\exists \alpha \in \Rp, y = \alpha x\)).
\end{defprop}
\begin{dem}
    Soit \((x,y) \in E^2\)\\
    \[\scal{x+y}{x+y} = \scal{x}{x}+2\scal{x}{y} + \scal{y}{y}\]
    et \[\scal{x}{y} \leq \abs{\scal{x}{y}} \leq \sqrt{\scal{x}{x}}\sqrt{\scal{y}{y}} \qquad \text{ par Cauchy-Schwarz }\]
    Ainsi  \begin{align*}
        \scal{x+y}{x+y} &\leq \scal{x}{x}+2\sqrt{\scal{x}{x}}\sqrt{\scal{y}{y}} + \scal{y}{y} \\
        \paren{ \sqrt{\scal{x+y}{x+y}}}^2 &\leq \paren{\sqrt{\scal{x}{x}} + \sqrt{\scal{y}{y}}}^2 \\
        \sqrt{\scal{x+y}{x+y}} &\leq\sqrt{\scal{x}{x}} + \sqrt{\scal{y}{y}}
    \end{align*}
    par ailleur le cas d'égalité est vérifié que si \(\scal{x}{y} = \abs{\scal{x}{y}} = \sqrt{\scal{x}{x}}\sqrt{\scal{y}{y}}\)
    ainsi il y a égalité \ssi \(\begin{cases}
        \scal{x}{y} &\geq 0\\
        (x,y) &\text{ liée}
    \end{cases} \iff \begin{cases}
    x &= 0_E \\
    \text{ ou } \\
    x&\neq 0_E \text{ et } \exists \alpha \in \R, y = \alpha x
    \end{cases}\)
\end{dem}

\subsection{Norme associée à un produit scalaire}
Soit \(E\) un espace préhilbertien réel de produit scalaire noté \(\scal{.}{.}\).
\begin{defprop}
    \begin{itemize}
    \item L’application de \(E\) dans \(\Rp\) , notée \(\norme{.}\), définie par
        \[\forall x \in E, \norme{x} = \sqrt{\scal{x}{x}}\]
    vérifie :
    \begin{enumerate}
        \item\( \forall x \in E, \norme{x} = 0 \imp x = 0_E ; \hfill( \text{ séparation })\)
        \item\( \forall x \in E, \forall \lambda  \in \R, \norme{\lambda x} = \abs{\lambda } \norme{x} ;\hfill (\text{ homogénéité })\)
        \item\( \forall (x, y) \in E^2, \norme{x +y} \leq \norme{x} + \norme{y} . \hfill(\text{ inégalité triangulaire })\)
    \end{enumerate}~\\
    On dit que c’est une norme sur \(E\) (la notion de norme sera reprise dans un cadre plus général en MPI) et plus particulièrement que c’est la norme associée au produit scalaire \(\scal{.}{.}\).\\
    \underline{Remarque}\\
    Il y a égalité dans l’inégalité triangulaire si, et seulement si, \(x = 0_E\) ou (\(x\neq 0_E\) et \(\exists \alpha \in \Rp, y = \alpha x).\)
    \item L’application \(d : E \times E\) dans \(\R\) définie par \(\forall (x, y) \in E^2, d(x, y) = \norme{x-y}\) est dite distance associée à la norme \( \norme{.}\).
\end{itemize}
\end{defprop}
\begin{defprop}[Deux identités remarquables]
    Si \(\norme{.}\) est la norme associée au produit scalaire \(\scal{.}{.}\) sur l’espace préhilbertien réel \(E\) alors :
    \begin{enumerate}
        \item \( \forall (x, y) \in E^2, \norme{x +y }^2 = \norme{x}^2 + \norme{y}^2 + 2 \scal{x}{y} ;\)
        \item \( \forall (x, y) \in E^2, \scal{x}{y} = \frac{1}{2}\paren{\norme{x +y}^2 - \norme{x}^2 - \norme{y}^2}\) . \hfill (formule de polarisation)
    \end{enumerate}
\end{defprop}

\section{Orthogonalité}
Soit \(E\) un espace préhilbertien réel de produit scalaire \(\scal{.}{.}\) et de norme associée \(\norme{.}\).
\subsection{Vecteurs orthogonaux et vecteurs unitaires}
\begin{defprop}
    \begin{enumerate}
        \item Deux vecteurs \(x\) et \(y\) de \(E\) sont dits orthogonaux si \(\scal{x}{y} = 0\).
        \item Un vecteur \(x\) de \(E\) est dit unitaire (ou normé) si \(\norme{x} = 1\).
    \end{enumerate}
\end{defprop}
\subsection{Orthogonal d’une partie}
\begin{defi}
    L’ensemble des vecteurs de \(E\) qui sont orthogonaux à tous les vecteurs d’une partie \(F\) de \(E\) est appelé orthogonal de \(F\) et noté \(F\ortho\) :
    \[F\ortho = \accol{y \in E \tq \forall x_F \in F, \scal{x_F}{y} = 0} \]
\end{defi}
\begin{defprop}[Structure de l’orthogona]
    Si \(F\) est une partie de \(E\) alors \(F\ortho\) est un sous-espace vectoriel de \(E\).
\end{defprop}
\subsection{Familles orthogonales et familles orthonormales}
\begin{defi}
    Une famille de vecteurs de \(E\) est dite :
    \begin{enumerate}
        \item orthogonale si ses vecteurs sont orthogonaux deux à deux.
        \item orthonormale (orthonormée) si elle est orthogonale et que ses vecteurs sont unitaires (normés).
    \end{enumerate}
\end{defi}
\begin{defprop}[Liberté des familles orthogonales]
    Toute famille orthogonale de vecteurs non nuls de \(E\) est une famille libre de \(E\).\\
    \underline{Remarque}\\
    En particulier, toute famille orthonormale de vecteurs de \(E\) est libre
\end{defprop}
\begin{theo}[Théorème de Pythagore]
    Soit \(p \in \Ns, p \geq 2\).\\
    Si la famille \((x_1, \dots , x_p)\) est une famille orthogonale de \(E\) alors
    \[\norme{\sum_{k=1^p} x_k}^2 = \sum_{k=1}^p\norme{x_k}^2\]
    \underline{Remarque}\\
    La réciproque est vraie pour \(p = 2\) et fausse pour \(p \geq 3\).
\end{theo}
\begin{dem}
    Démontrons cette égalité par récurrence en supposans les hypothèses vérifié
    \begin{itemize}
        \item \(p = 2 \)\\~\\
        \[\norme{x_1 + x_2}^2 = \norme{x}^2 + \norme{y}^2 + 2\underbrace{\scal{x_1}{x_2}}_{ \substack{= 0\\ \text{ car } x_1 \text{ et } x_2 \text{ sont  orthogonaux}}}\]
        \item Supposons la propriété vérifié au rang \(n\) montrons là au rang \(n+1\)\\~\\
        \begin{align*}
            \norme{\sum_{k=1^{n+1}} x_k}^2 &= \norme{\sum_{k=1^{n}} x_k}^2 + \norme{x_{n+1}}^2 + 2\underbrace{\scal{\sum_{k=0}^n x_k}{x_{n+1}}}_{ \substack{= 0\\ \text{ car } \sum_{k=0}^n x_k \text{ et } x_2 \text{ sont orthogonaux}}} \\
            &=\sum_{k=1}^n\norme{x_k}^2 + \norme{x_{n+1}}^2\\
            &=\sum_{k=1}^{n+1}\norme{x_k}^2
        \end{align*}
    \end{itemize}
\end{dem}
\section{Bases orthonormales}
    Soit \(E\) un espace euclidien de dimension \(n \in \Ns\) de produit scalaire \(\scal{.}{.}\) et de norme associée \(\norme{.}\).

\subsection{Algorithme d’orthonormalisation de Gram-Schmidt}
\begin{defprop}
    Soit \(\cal{B} = (e_1, \dots , e_n)\) une base quelconque de \(E\).\\~\\
    \begin{itemize}
        \item Etape \(1\) : on pose \(u_1 = e_1\).
        \item Etape \(2\) : on pose \(u_2 = e_2+ \alpha_1u_1\) et on cherche \(\alpha_1 \in \R\) tel que \(\scal{u_1}{u_} = 0\).\\~\\
        Par linéarité à droite, on trouve : \(\alpha_1 = - \frac{\scal{u_1|}{e_2}}{\scal{u_1}{u_1}}\) puisque \(u_1\neq 0_E\) . Avec cette valeur de \(\alpha_1\), on obtient un vecteur \(u_2\) orthogonal à \(u_1\) tel que \(u_2\neq 0_E\) .
        \item Etape \(3\) : on pose\( u_3 = e_3 + \beta_1u_1 + \beta_2u_2\) et on cherche \((\beta_1, \beta_2) \in \R^2\) tel que \(\scal{u_1}{u_3} = 0\) et \(\scal{u_2}{u_3} = 0\).\\~\\
        Par linéarité à droite, on trouve : \(\beta_1 = - \frac{\scal{u_1}{e_3}}{\scal{u_1}{u_1 }}\) et \(\beta_2 = - \frac{\scal{u_2}{e_3}}{u_2}{u_2}\) puisque \(u_1\neq 0_E\) et \(u_2\neq 0_E\) . Avec ces valeurs de \(\beta_1\) et \(\beta_2\), on obtient un vecteur \(u_3\) orthogonal à \(u_1\) et \(u_2\) avec \(u_3\neq 0_E\) .\\~\\
        Après \(n\) étapes, on obtient \((u_1, u_2, \dots , u_n)\) famille orthogonale de vecteurs non nuls de \(E\) donc famille libre de \(E\). Cette famille de \(E\) étant orthogonale, libre et de cardinal égal à la dimension de \(E\), on en déduit que \((u_1, u_2, \dots , u_n)\)est une base orthogonale de \(E\).
    \end{itemize}
    La famille \(\cal{B}' = (e'_1, \dots , e'_n)\) définie par \(\forall i \in \interventierii{1}{n}, e'_i = \frac{1}{\norme{u_i}}u_i\) est alors une base orthonormée de \(E\).
\end{defprop}
\subsection{Existence de bases orthonormales}
\begin{defprop}
    \begin{enumerate}
        \item Tout espace euclidien de dimension non nulle admet au moins une base orthonormale.
        \item Toute famille orthonormale d’un espace euclidien peut être complétée en une base orthonormale.
    \end{enumerate}
\end{defprop}
\subsection{Calculs dans une base orthonormale}
\begin{defprop}[Coordonnées d’un vecteur]
    Si \(\cal{B} = (e_1, \dots , e_n)\) est une base orthonormée de \(E\) alors les coordonnées \((x_1, \dots , x_n)\) d’un vecteur \(x\) de \(E\) dans la base \(\cal{B}\) vérifient
        \[\forall i \in \interventierii{1}{n}, x_i = \scal{e_i}{x}\ie X = \Mat{\cal{B}}{x} =
        \begin{pmatrix}
        \scal{e_1}{x}   \\
        \dots   \\
        \scal{e_n}{x}   
        \end{pmatrix}
        \]
\end{defprop}
\begin{defprop}[Expression de la norme et du produit scalaire]
    Soit \(\cal{B}\) une base orthonormée de \(E\) et \((x, y) \in E^2\) tel que \(X = \Mat{\cal{B}}{x}\) et \(Y = \Mat{\cal{B}}{y}\).
    Alors :
    \[\scal{x}{y} = \trans{X}Y \qquad\text{ et }\qquad \norme{x} = \sqrt{\trans{X}X}\]
\end{defprop}

\section{Projection orthogonale sur un sous-espace de dimension finie}
    Soit \(E\) un espace préhilbertien réel de produit scalaire \(\scal{.}{.}\) et de norme associée \(\norme{.}\).
\subsection{Supplémentaire orthogonal d’un sous-espace de dimension finie}
\begin{defprop}
    Si \(F\) est un sous-espace vectoriel de dimension finie de \(E\) alors \(E = F \oplus F\ortho\).\\
    \underline{Remarques}\\
    \begin{enumerate}
        \item Cette propriété n’est pas conservée lorsque \(F\) est un sous-espace vectoriel de dimension non finie.
        \item Dans le cas particulier où \(E\) est euclidien, pour tout sous-espace vectoriel \(F\) de \(E\), on a :
            \[E = F \oplus F\ortho \text{ et }\dim E = \dim F + \dim F\ortho\].
    \end{enumerate}
\end{defprop}
\begin{dem}
    Supposons les hypothèses vérifié\\~\\
    Procédons par analyse-synthèse pour démontrer que \(E = F \oplus F\ortho\)
    \begin{itemize}
        \item \analyse\\~\\
            Supposons que \(E = F \oplus F\ortho\) alors tout \(x\) de \(E\) s'écrit \(x = x_F + X_{F\ortho}\)\\~\\
            Soit \(\cal{B}_F = (e_1, \dots, e_p)\) une base orthonormé de \(F\) alors \(x_F = \sum_{k=1}^p \scal{x}{e_k}e_k\)\\~\\
            avec \(\forall k \in \interventierii{1}{p}, \scal{x}{e_k} = \scal{x_F}{e_k} + \underbrace{\scal{x_{F\ortho}}{e_k}}_{ = 0}\)\\~\\
            Ainsi \(\begin{cases}
                x_F &= \sum_{k=1}^p \scal{x}{e_k}e_k \\
                x_{F\ortho} = x-x_F
            \end{cases}\)
        \item \synthese \\~\\
            Pour \(x \in E\) : \begin{itemize}
                \item \(x_F \in F\)
                \item \(x_{F\ortho} \in F\ortho\) \\
                Car \begin{align*}
                \forall k \in \interventierii{1}{P} \scal{e_k}{x_{F\ortho}} &= \scal{e_k}{x} - \scal{e_k}{x_F}\\
                &= \scal{e_k}{x} - \sum_{i=1}^p \scal{x}{e_i}\scal{e_k}{e_i}\\
                &=  \scal{e_k}{x} - \scal{x}{e_k}\\
                &= 0
                \end{align*}
            \end{itemize}
    \end{itemize}
    \conclusion Si \(F\) est un sous-espace vectoriel de dimension finie de \(E\) préhilbertien alors \(E = F \oplus F\ortho\)
\end{dem}
\subsection{Projection orthogonale sur un sous-espace de dimension finie}
\begin{defprop}
    Si \(F\) est un sous-espace vectoriel de \(E\) de dimension finie alors :
    \begin{enumerate}
        \item on peut définir la projection sur \(F\) parallèlement à \(F\ortho\) ;
        \item cette projection est appelée projection orthogonale sur \(F\) et souvent notée \(p_F\) ;
        \item pour tout \(x\) de \(E\), \(p_F (x)\) est appelé projeté orthogonal de \(x\) sur \(F\) .
    \end{enumerate}
\end{defprop}
\subsection{Détermination pratique du projeté orthogonal d’un vecteur}
\begin{defprop}
    Soit \(F\) un sous-espace vectoriel de \(E\) de dimension finie et \(p_F (x)\) le projeté orthogonal sur \(F\) de \(x \in E\).
    \begin{enumerate}
    \item Utilisation d’une base orthonormée de \(F\)\\~\\
        Si \((e_1, \dots , e_q)\) est une base orthonormée de \(F\) alors,
            \[p_F (x) = \scal{e_1}{x} e_1 + \dots + \scal{e_q}{x} e_q\].
    \item Utilisation d’une famille génératrice de \(F\)\\~\\
        En écrivant que \(p_F (x)\) est un vecteur de \(F\) tel \(x - p_F (x)\) est orthogonal à tous les vecteurs d’une famille génératrice de \(F\) , on obtient un système linéaire qui permet de déterminer \(p_F (x)\).
    \end{enumerate}
\end{defprop}
\subsection{Caractérisation du projeté orthogonal d’un vecteur}
\begin{defprop}
    Si \(F\) est un sous-espace vectoriel de \(E\) de dimension finie alors, pour tout \(x\) de \(E\), \(p_F (x)\) est l’unique vecteur \(y_0\) de \(F\) tel que
    \[\norme{x - y_0} = \min_{y\in F} \norme{x-y} .\]
    Ce minimum est appelé distance de \(x\) à \(F\) , noté \(d(x, F )\), et vérifie donc :
    \[d(x, F ) = \norme{x - p_F (x)}\]
    \underline{Remarque}\\
    On dit que le projeté orthogonal de \(x\) sur le sous-espace de dimension finie \(F\) est l’unique élément de \(F\) qui réalise la distance de \(x\) à \(F\) .
\end{defprop}
\subsection{Cas particulier des hyperplans}
    On suppose ici que \(E\) est un espace euclidien de dimension non nulle.
\begin{defprop}[Vecteur normal à un hyperplan]
    Si \(H\) est un hyperplan de \(E\) alors tout vecteur directeur de la droite \(H\ortho\) est appelé vecteur normal à \(H\).\\
    \underline{Remarque}\\
    Si \(H\) est un hyperplan de \(E\) dont l’équation dans une base orthonormée \(B\) de \(E\) est
    \[a_1x_1 + \dots + a_nx_n = 0\]
    alors le vecteur \(a\) de \(E\) de coordonnées \((a_1, \dots , a_n)\) dans la base orthonormée \(\cal{B}\) est vecteur normal à \(H\).
\end{defprop}

\begin{defprop}[Distance d’un vecteur à un hyperplan H]
    Si \(x\) est un vecteur de \(E\) et \(H\) un hyperplan de \(E\) de vecteur normal unitaire \(a\) alors
    \[d(x, H) = \norme{x - p_H (x)} = \norme{p_{H\ortho} (x)} = \abs{\scal{a}{x}}\]
\end{defprop}
% \chapter{Dénombrement - Probabilité sur un univers fini}
\minitoc
\section{Dénombrement}
    Conformément au programme officiel de MP2I, on exclut ici toute formalisation excessive. Les propriétés intuitives sont donc admises et le recours systématique à des bijections pour dénombrer n’est pas attendu.
\subsection{Cardinal d’un ensemble fin}
\begin{defi}
    Soit \(A\) un ensemble.\\~\\
    \(A\) est dit fini s’il est vide ou s’il existe un entier naturel non nul \(n\) et une bijection de \(\interventierii{1}{n}\) vers \(A\).\\~\\
    Dans ce cas,
    \begin{itemize}
        \item si \(A\) est vide, on dit que \(A\) est de cardinal égal à \(0\).
        \item si \(A\) est non vide alors l’entier \(n\) ci-dessus est unique et appelé cardinal de \(A\).
    \end{itemize}
    Notations utilisées pour le cardinal : \(\Card{A}\) ou \(\abs{A}\).
\end{defi}
\begin{prop}
    \begin{itemize}
        \item Cardinal d’une partie\\
            Si \(B\) est une partie d’un ensemble fini \(A\) alors \(B\) est un ensemble fini avec \(\Card{B} \leq \Card{A}\) [et égalité des cardinaux si, et seulement si, \(B\) = \(A\)].
        \item Opérations sur les cardinaux
        \begin{enumerate}
            \item Si \(A\) et \(B\) sont deux ensembles finis alors :
                \begin{enumerate}
                    \item \(A \union B\) est un ensemble fini et \(\Card{A \union B} = \Card{A} + \Card{B} - \Card{A \inter B}\).
                    \item \(A \times B\) est un ensemble fini et\( \Card{A \times B} = \Card{A} \times \Card{B}\).
                \end{enumerate}
            \item Soit \(A\) et \(B\) des parties d’un ensemble fini \(E\).
                \begin{enumerate}
                    \item \(A\pd B\) est un ensemble fini et \(\Card{A\pd B} = \Card{A} - \Card{A \inter B}\)
                    \item \(A\) est un ensemble fini et \(\Card{A} = \Card{E} - \Card{A}\).
                \end{enumerate}
        \end{enumerate}
        \item Caractérisation des bijections entre ensembles finis de même cardinal\\
            Une application entre deux ensembles finis de même cardinal est bijective si, et seulement si, elle est injective, si, et seulement si, elle est surjective.
        \item Cardinal de l’ensemble des applications entre deux ensembles finis\\
            Si \(A\) et \(B\) sont des ensembles finis alors l’ensemble \(B^A\) des applications de \(A\) vers \(B\) est fini avec
            \[\Card{B^A} = \Card{B}^{\Card{A}}\] .
        \item Cardinal de l’ensemble des parties d’un ensemble fini\\
            Si \(A\) est un ensemble fini alors l’ensemble \(\cal{P}(A)\) des parties de \(A\) est fini avec
            \[\Card{\cal{P}(A)} = 2^{Card{A}}.\]
    \end{itemize}
\end{prop}
\subsection{Listes et combinaisons}
\begin{defi}
    Soit \(A\) un ensemble fini et \(p\) un entier naturel.
    \begin{enumerate}
        \item On appelle \(p-\)liste (ou \(p-\)uplet) d’éléments de \(A\) tout élément \((a_1, a_2, \dots , a_p)\) de \(A_p\).
        \item On appelle \(p-\)arrangement d’éléments de \(A\) toute \(p-\)liste d’éléments distincts de \(A\).
        \item On appelle \(p-\)combinaison d’éléments de \(A\) toute partie de \(A\) à \(p\) éléments.
    \end{enumerate}
\end{defi}
\begin{prop}
    Soit \(A\) un ensemble fini de cardinal \(n\) et \(p\) un entier naturel non nul inférieur ou égal à \(n\).
    \begin{enumerate}
        \item Le nombre de \(p-\)listes d’éléments de \(A\) est égal à \(n^p\).
        \item Le nombre de \(p-\)listes d’éléments distincts de \(A\) est égal à \(\prod^{p-1}_{k=0}(n - k)= \frac{n!}{(n - p)!}\).
        \item Le nombre de \(p-\)combinaisons d’éléments de \(A\) est égal à \(\frac{n!}{p!(n - p)!} =\binom{n}{p}\).
    \end{enumerate}
    \underline{Remarques}\\
    Le nombre d’applications
    \begin{itemize}
        \item injectives d’un ensemble fini de cardinal \(p\) dans un ensemble fini de cardinal \(n\) est égal à \(\frac{n!}{(n - p)!}\)
        \item bijectives d’un ensemble fini de cardinal \(n\) dans lui-même est égal à \(n!\).
    \end{itemize}
\end{prop}
\begin{defprop}[Retour sur les formules de Pascal et du binôme (preuves combinatoires)]
    ~\\
    ~\\
    \begin{enumerate}
        \item \(\forall(n, p) \in \N^2, p \leq n \imp \binom{n}{p} + \binom{n}{p+1} = \binom{n+1}{p+1} \) \hfill (formule du triangle de Pascal)
        \item \(\forall n \in \N, \forall(a, b) \in \C^2, (a + b)^n =\sum^n_{p=0}\binom{n}{p}a^pb^{n-p}\) \hfill (formule du binôme de Newton)
    \end{enumerate}
\end{defprop}
\section{Probabilités sur un univers fini}
\subsection{Probabilités sur un univers finiExpérience aléatoire et univers}
\begin{defprop}[Expérience aléatoire]
    Une expérience qui, reproduite dans des conditions identiques, peut conduire à plusieurs résultats possibles dont on ne peut prévoir le résultat par avance est dite expérience aléatoire.
\end{defprop}
\begin{defprop}[Univers]
    L’ensemble des issues (résultats possibles ou réalisations) d’une expérience aléatoire est appelé univers (ou espace des états) et souvent noté \(\Omega\). \\~\\
    Conformément au programme de MP2I, on se limite en \(1^{\text{re}}\) année à des univers finis.
\end{defprop}
\begin{defprop}[Evénements]
    \begin{enumerate}
        \item Toute partie de l’univers \(\Omega\) est appelée événement.
        \item Parmi les parties de l’univers \(\Omega\), on trouve :
            \begin{itemize}
                \item la partie \(\Omega\), dite événement certain ;
                \item la partie \(\emptyset\), dite événement impossible ;
                \item les singletons de \(\Omega\), dits événements élémentaires souvent notés \(\accol{\omega}\) à pour \(\omega \in \Omega\).
            \end{itemize}
        \item Soient \(A\) et \(B\) deux parties de l’univers \(\Omega\) (autrement dit deux événements).
            \begin{itemize}
                \item La partie \(\conj{A} = \Omega \pd A\) est dite événement contraire de \(A\).
                \item La partie \(A \union B\) est dite événement \("A\) ou \(B"\).
                \item La partie \(A \inter B\) est dite événement \("A\) et \(B"\).
                \item Les parties \(A\) et \(B\) sont dites événements incompatibles (ou disjoints) si \(A \inter B = \emptyset\).
            \end{itemize}
        \item Soit   \( q \in \Ns\) et \(A_1, A_2, \dots , A_q\) des parties de \(\Omega\).
        L’ensemble  \(\accol{A_1, A_2, \dots , A_q}\) est dit système complet d’événements de \(\Omega\) si c’est une partition de \(\Omega\) autrement dit si :
        \begin{itemize}
            \item \(\bigunion_{i=1}^q A_i = \Omega \); \hfill (la réunion des événements du système est l’événement certain)
            \item \(\forall  i \in \interventierii{1}{q} , A_i\neq \emptyset\) ; \hfill(pas d’événenement impossible dans le système)
            \item \(\forall (i, j) \in \interventierii{1}{q}^2 , i\neq j \imp A_i \inter A_j = \emptyset.\) (événements du système incompatibles deux à deux)
        \end{itemize}
        \underline{Remarque}\\
        On note parfois \(\Omega = \bigsqcup^q_{i=1} A_i\) pour indiquer que les événements \(A_i\) sont deux à deux disjoints.
    \end{enumerate}
\end{defprop}

\subsection{Espace probabilisés finis}
\begin{defi}
    Soit \(\Omega\) un univers fini.\\~\\
    On appelle probabilité sur \(\Omega\) toute application \(\mathbb{P}\) de \(\cal{P}(\Omega)\) dans \(\intervii{0}{1}\) telle que :
    \begin{enumerate}
        \item \(\proba{\Omega} = 1\) ;
        \item \(\forall (A, B) \in \cal{P} (\Omega) \times \cal{P} (\Omega), A \inter B = \emptyset \imp P(A \union B) = P(A) + P(B)\).
    \end{enumerate}
    Le couple \((\Omega, \mathbb{P})\) est alors appelé espace probabilisé fini.
\end{defi}
\begin{prop}
    Soit \((\Omega, \mathbb{P})\) un espace probabilisé fini.
    \begin{enumerate}
        \item \(\proba{\emptyset} = 0\).\hfill (événement impossible)
        \item \(\forall A \in \cal{P}(\Omega), \proba{\conj{A}} = 1 - \proba{A}\). \hfill (événement contraire)
        \item \(\forall(A, B) \in \cal{P}(\Omega) \times \cal{P}(\Omega), \proba{A \union B} = \proba{A} + \proba{B} - \proba{A \inter B}\). \hfill(réunion d’événements)
        \item \(\forall(A, B) \in \cal{P}(\Omega) \times \cal{P}(\Omega), \proba{A \pd B} = \proba{A} - \proba{A \inter B}\) \hfill (différence d’événements)
        \item \(\forall(A, B) \in \cal{P}(\Omega) \times \cal{P}(\Omega), A \subset B \imp \proba{A} \leq \proba{B}\) \hfill (croissance des probabilités)
    \end{enumerate}
\end{prop}
\begin{defprop}[Distribution de probabilités]
    On appelle distribution de probabilités sur un ensemble fini \(I\) toute famille de réels positifs indexée par \(I\) et de somme égale à \(1\) .
\end{defprop}
\begin{defprop}[Détermination d’une probabilité]
    Soit \(n \in \Ns\).\\~\\
    Si \(\Omega = \accol{\omega_1, \dots , \omega_n}\) est un univers fini et \(p_1, \dots , p_n\) des réels positifs tels que \(p_1 + \dots + p_n = 1 \) alors il existe une probabilité \(\mathbb{P}\) et une seule sur \(\Omega\) telle que \(\forall i \in \accol{1, 2, \dots , n} , \proba{\accol{\omega_i}} = p_i\).\\

    \underline{Remarque}\\
    Autrement dit, une probabilité sur un univers fini est entièrement déterminée :
    \begin{itemize}
        \item par sa valeur sur les singletons de l’univers.
        \item par la distribution de probabilités \(\paren{\proba{\accol{\omega}}}_{\omega\in\Omega}\) .
    \end{itemize}
\end{defprop}

\begin{defprop}[Probabilité uniforme]
    Soit \((\Omega, \mathbb{P})\) un espace probabilisé fini.\\~\\
    \(\mathbb{P}\) est dite probabilité uniforme sur \(\Omega\) s’il existe \(p \in \intervii{0}{1}\) tel que \(\forall\omega \in \Omega, \proba{\accol{\Omega}} = p\).\\~\\
    Dans ce cas,
    \[p = \frac{1}{\Card{\Omega}}\text{ et }\forall\omega \in \Omega, \proba{\accol{\omega}} = \frac{1}{\Card{\Omega}}\text{ et }\forall A \in \cal{P} (\Omega) , \proba{A} = \frac{\Card{A}}{\Card{\Omega}} \]
\end{defprop}
\subsection{Probabilités conditionnelles}
Soit \((\Omega, \mathbb{P})\) un espace probabilisé fini.
\begin{defi}
    Soient \((A, B) \in \cal{P}(\Omega) \times \cal{P}(\Omega)\) tel que \(\proba{B} > 0\).\\~\\
    On appelle probabilité conditionnelle de \(A\) sachant \(B\) le réel, noté \(\proba{A|B} \) ou \(\probacond{A}{B}\), défini par
    \[\proba{A|B} = \probacond{A}{B} = \frac{\proba{A \inter B}}{\proba{B} }\]
    \underline{Remarque}\\
    Pour tout \(B \in \cal{P} (\Omega)\) tel que \(\proba{B} > 0\), l’application \(\mathbb{P}_{B} : A \mapsto \frac{\proba{A \inter B}}{\proba{B}}\) est une probabilité sur \(\Omega\) et vérifie donc toutes les propriétés précédemment énoncées sur les probabilités.\\~\\
    Ainsi, on a notamment :
    \begin{itemize}
        \item \(\forall A \in \cal{P} (\Omega) , \mathbb{P}_{B} (\conj{A}) = 1 - \mathbb{P}_{B} (A).\)
        \item \(\forall (A, C) \in \cal{P}(\Omega) \times \cal{P}(\Omega), \mathbb{P}_{B} (A \union C) = \mathbb{P}_{B} (A) + \mathbb{P}_{B} (C) - \mathbb{P}_{B} (A \inter C).\)
    \end{itemize}
\end{defi}
\begin{defprop}[Formule des probabilités composées]
    Soit \(n \in N \)tel que \(n \geq 2\).
    Si \(A_1, \dots , A_n\) sont des événements tels que \(\proba{A_1 \inter \dots \inter A_{n-1}}\) est non nulle alors 
    \[\proba{A_1 \inter A_2 \inter \dots \inter A_n} = \proba{A_1} \probacond{A_2}{A_1} \probacond{A_3}{A_1 \inter A_2} \dots \probacond{A_n}{A_1 \inter \dots \inter A_{n-1}}\]
\end{defprop}

\begin{defprop}[Formule des probabilités totales]
    Si \(\accol{A_1, \dots , A_n}\) est un système complet d’événements et \(B\) un événement quelconque alors
    \[\proba{B} = \sum^n_{i=1} \probacond{B}{A_i} \proba{A_i}\]
    avec la convention usuelle suivante conservée dans la suite du chapitre : \(\probacond{B}{A_i} \mathbb{P} (A_i) = 0\) si \(\mathbb{P} (A_i) = 0\).
\end{defprop}

\begin{defprop}[Formules de Bayes]
    Soit \(B\) un événement de probabilité non nulle.
    \begin{enumerate}
        \item Si \(A\) est un événement alors
            \[\probacond{A}{B} = \frac{\probacond{B}{A} \proba{A}}{\proba{B}}\] .
        \item Si \(\accol{A_1, \dots , A_n}\) est un système complet d’événements alors
        \[\forall j \in \interventierii{1}{n} , \frac{\probacond{A_j}{B} = \probacond{B}{A_j}\mathbb{P}(A_j)}{\sum^n_{i=1} \probacond{B}{A_i}\proba{A_i}}\]
        (conséquence immédiate de la formule de Bayes précédente et de la formule des probabilités totales)
    \end{enumerate}
\end{defprop}

\subsection{Evénements indépendants}
Soit \((\Omega, \mathbb{P})\) un espace probabilisé fini.
\begin{defprop}[Couple d’événements indépendants]
    Deux événements \(A\) et \(B\) sont dits indépendants si \(\mathbb{P} (A \inter B)\) = \(\mathbb{P} (A) \mathbb{P} (B\)) .\\
    \underline{Remarques}\\
    \begin{itemize}
        \item Si \(B\) est de probabilité non nulle, l’indépendance de \(A\) et \(B\) se traduit par \(\probacond{A}{B}  = \mathbb{P} (A)\).
        \item La notion d’indépendance est une notion probabiliste (et pas ensembliste) à ne pas confondre avec la notion ensembliste d’événements disjoints/incompatibles
    \end{itemize}
\end{defprop}

\begin{defprop}[Famille finie d’événements indépendants]
    Soit \(n \in N\) tel que \(n \geq 2\).\\~\\
    Les événements \(A_1, \dots , A_n\) sont dits indépendants si, pour tout \(I \subset \interventierii{1}{n}\),
    \[\proba{\biginter_{i\in I}A_i}= \prod_{i\in I}\paren{\proba{A_i}}\]
    \underline{Remarques} \\
    \begin{itemize}
        \item On parle parfois de mutuelle indépendance au lieu d’indépendance.
        \item Pour \(n\) entier naturel supérieur ou égal à \(2\),
        \item si les événements \(A_1, \dots , A_n\) sont indépendants alors ils sont deux à deux indépendants.
        \item la réciproque de l’implication précédente est fausse pour \(n \geq 3\). Des événements peuvent être indépendants deux à deux sans être indépendants.
    \end{itemize}
\end{defprop}

\begin{defprop}[Indépendance et événements contraires]
    \begin{itemize}
        \item Si \(A\) et \(B\) sont deux événements indépendants alors \(A\) et \(\conj{B}\) sont indépendants.
        \item Plus généralement, pour \(n\) entier naturel supérieur ou égal à \(2\),\\~\\
        si les événements \(A_1, \dots , A_n\) sont indépendants alors les événements \( A'_1, \dots , A'_n\) tels que
        \[\forall i \in \interventierii{1}{n} , A'_i \in \accol{A_i, \conj{A_i}}\]
        le sont aussi.
    \end{itemize}
\end{defprop}
% \chapter{Variables aléatoires sur un univers fini}
\minitoc 
    Dans ce chapitre, \((\Omega, \mathbb{P})\) désigne un espace probabilisé \underline{FINI}.
\section{Loi d’une variable aléatoire}
\subsection{Définitions, propriétés et notations}
    Soit \(E\) un ensemble.
\begin{defprop}[Variable aléatoire]
    Toute application \(X\) définie sur l’univers \(\Omega\) à valeurs dans \(E\) est dite variable aléatoire (réelle si \(E = \R\)).\\~\\
    Pour toute partie \(A\) de \(E\), l’image réciproque \(X^{-1} (A) = \accol{\omega \in  \Omega \tq X(\omega) \in  A}\) est une partie de \(\Omega\) donc un événement qui est noté \((X \in  A)\) ou\( \accol{X \in  A}\) :
   \[ (X \in  A) = \accol{X \in  A} = \accol{\omega \in  \Omega \tq X(\omega) \in  A}\] 
\end{defprop}
\begin{defprop}[Loi d’une variable aléatoire]
    On appelle loi de la variable aléatoire \(X : \Omega \to E\), l’application \(\mathbb{P}_{X} : \cal{P} (X(\Omega)) \to \intervii{1}{0}\)  définie par :
    \[\forall A \in  \cal{P} (X (\Omega)) , \mathbb{P}_{X} (A) = \mathbb{P} (X \in  A) .\]
    \underline{Remarques}\\
    \begin{itemize}
        \item \(\mathbb{P}_{X}\) est une probabilité sur \(X(\Omega)\).
        \item la loi de la variable aléatoire \(X : \Omega \to E\) (autrement dit \(\mathbb{P}_{X}\) ) est entièrement déterminée par la connaissance de l’ensemble fini \(X (\Omega)\) et des valeurs \(P (X = x)\) pour \(x\) variant dans \(X (\Omega)\) avec :
            \[\forall x \in  X (\Omega) , \mathbb{P}_{X} ({x}) = \mathbb{P} (X = x)\] .
        \item la loi de la variable aléatoire \(X : \Omega \to E\) (autrement dit \(\mathbb{P}_{X}\) ) est entièrement déterminée par la distribution de probabilités \((\mathbb{P}(X = x))_{x\in X(\Omega)}\) sur \(X(\Omega)\) et on a :
            \[\forall A \in  \cal{P} (X (\Omega)) , \mathbb{P}_{X} (A) = \mathbb{P} (X \in  A) = \sum_{x\in A}\mathbb{P}(X = x)\]
    \end{itemize}
\end{defprop}
\begin{nota}
    Lorsque \(X\) et \(Y\) sont deux variables aléatoires sur \(\Omega\) de même loi (\ie \(\mathbb{P}_{X} = \mathbb{P}_{Y}\) ), on note \(X \sim Y\) .\\
    \underline{Remarque}\\ 
    si \(X\) et \(Y\) sont deux variables aléatoires sur \(\Omega\) égales alors elles ont évidemment même loi. \\
    La réciproque est fausse.
\end{nota}
\subsection{Image d’une variable aléatoire par une fonction}
\begin{defprop}
    Soit \(E\) et \(F\) deux ensembles.\\~\\
    Soit \(X : \Omega \to E\) une variable aléatoire sur \(\Omega\) et \(f : E \to F\) une application.
    \begin{enumerate}
        \item L’application \(f \circ X : \Omega \to F\) est une variable aléatoire sur \(\Omega\), notée \(f (X)\).
        \item La loi de \(f (X) \) est entièrement déterminée par la connaissance de \(f\) et de la loi de \(X\) :
            \[\forall y \in  f (X (\Omega)) , \mathbb{P}_{f (X)} (\accol{y}) = \mathbb{P}(f (X) = y) = \sum_{\substack{x\in X(\Omega)\\ f(x)=y}}\mathbb{P} (X = x)\]
            \[\forall A \in  \cal{P} (f (X (\Omega))) , P_{f (X)} (A) = \mathbb{P} (f (X) \in  A) = X = \sum_{\substack{x\in X(\Omega) \\ f (x)\in A }}\mathbb{P} (X = x)\]
    \end{enumerate}
    \underline{Remarque}\\
    Si \(X\) et \(Y\) sont deux variables aléatoires sur \(\Omega\) de même loi alors, sous réserve que cela ait du sens, les variables aléatoires \(f (X)\) et \(f (Y )\) ont même loi.
\end{defprop}
\subsection{Lois usuelles}
\begin{defprop}[Loi uniforme]
    Soit \(E\) un ensemble fini non vide.\\~\\
    On dit que la variable aléatoire \(X : \Omega \to E\) suit la loi uniforme sur \(E\) (ou que \(X\) est une variable uniforme sur \(E\)) si sa loi \(\mathbb{P}_{X}\) est la probabilité uniforme sur \(E\), c’est-à-dire :
    \[\forall A \in  \cal{P} (E) , \mathbb{P}_{X} (A) = \mathbb{P} (X \in  A) = \frac{\Card{A}}{\Card{E}}\]
    On le note X\( \sim \cal{U}(E)\).
\end{defprop}
\begin{defprop}[Loi de Bernoulli]
    Soit \(p\) un réel de \(\intervii{1}{0}\) .\\~\\
    On dit que la variable aléatoire \(X : \Omega \to {0, 1}\) suit la loi de Bernoulli de paramètre \(p\) (ou que \(X\) est une variable de Bernoulli de paramètre \(p\)) si sa loi \(\mathbb{P}_{X}\) vérifie
    \[\mathbb{P}_{X} (\accol{1}) = \mathbb{P} (X = 1) = p \text{ et }\mathbb{P}_{X} (\accol{0}) = \mathbb{P} (X = 0) = 1 - p\]
    On le note \(X \sim \cal{B}(p)\).
    \begin{itemize}
    \item Indicatrice d’un événement\\
    Soit \(A\) un événement.\\
    L’indicatrice de \(A\) est une variable aléatoire de Bernoulli de paramètre \(\mathbb{P}(A) : \ind{A} \sim \cal{B}(\mathbb{P}(A))\).
    \item Interprétation/Modélisation\\
    Pour une expérience aléatoire du type "Succès-Echec", la variable aléatoire réelle prenant la valeur \(1\) en cas de succès et la valeur \(0\) en cas d’échec suit une loi de Bernoulli de paramètre \(p = \mathbb{P} (\text{ "Succès" })\) .
    \end{itemize}
\end{defprop}

\begin{defprop}[Loi binomiale]
    Soit \(n\) un entier naturel non nul et \(p\) un réel de \(\intervii{1}{0}\)  . \\
    On dit que la variable aléatoire \(X : \Omega \to \interventierii{0}{n}\) suit la loi binomiale de paramètres \(n\) et \(p\) (ou que \(X\) est une variable binomiale de paramètres \(n\) et \(p\)) si sa loi \(\mathbb{P}_{X}\) vérifie
    \[\forall k \in  \interventierii{0}{n} , \mathbb{P}_{X} (\accol{k}) = \mathbb{P} (X = k) = \binom{n}{k}p^k(1 - p)^{n-k}.\]
    On le note \(X \sim \cal{B}(n, p)\).
    \begin{itemize}
    \item Interprétations/Modélisations\\
        \begin{enumerate}
            \item Si on répète \(n\) expériences aléatoires indépendantes du type "Succès-Echec" de même probabilité de succès, la variable aléatoire réelle comptabilisant le nombre de succès obtenus après les \(n\) expériences suit la loi binomiale de paramètres \(n\) et \(p\).
            \item Si on considère "une urne" contenant uniquement des "boules blanches" en proportion \(p \in  \intervii{1}{0}\) et des "boules noires" en proportion \(1 - p\) et qu’on effectue \(n\) tirages successifs "d’une boule" dans "l’urne" avec remise après chaque tirage, la variable aléatoire réelle comptabilisant le nombre de "boules blanches" tirées après les \(n\) tirages suit la loi binomiale de paramètres \(n\) et \(p\).
        \end{enumerate}
    \end{itemize}
\end{defprop}
\subsection{Loi conditionnelle d’une variable aléatoire}
\begin{defprop}
    Soit \(X\) une variable aléatoire sur \(\Omega\) et \(A\) un événement de probabilité non nulle.\\
    On appelle "loi conditionnelle de \(X\) sachant l’événement \(A\)" l’application de \(\cal{P}(X(\Omega))\) dans \(\intervii{1}{0}\) qui, à tout \(B\) de \(\cal{P} (X(\Omega))\), associe le réel \(\mathbb{P}_A (X \in  B)\) .\\
    \underline{Remarque}\\
    Cette loi est entièrement déterminée par la distribution de probabilités \((\mathbb{P}_A(X = x))_{x\in X(\Omega)}\) sur \(X(\Omega)\) :
    \[\forall B \in  \cal{P} (X(\Omega)) , \mathbb{P}_A (X \in  B) = \sum_{x\in B}\mathbb{P}_A(X = x)\]
\end{defprop}
\subsection{Couple de variables aléatoires}
\begin{defi}
    Si \(X : \Omega \to E\) et\( Y : \Omega \to F\) sont deux variables aléatoires alors l’application \(Z : \Omega \to E \times F\) définie par
    \[\forall \omega \in  \Omega, Z(\omega) = (X(\omega), Y (\omega))\]
    est une variable aléatoire, dite couple des variables aléatoires \(X\) et \(Y\) et notée \(Z = (X, Y )\).\\~\\
    Ces notations sont conservées dans la suite de ce paragraphe.\\~\\
\end{defi}
\begin{defprop}[Loi conjointe, lois marginales]
    \begin{enumerate}
        \item La loi \(\mathbb{P}_{(X,Y )}\) du couple \((X, Y )\), dite loi conjointe de \(X\) et \(Y\) , est définie par :
            \[\forall (x, y) \in  X(\Omega) \times Y (\Omega), \mathbb{P}_{(X,Y )} (\accol{(x, y)}) = \mathbb{P} ((X, Y ) = (x, y)) = \mathbb{P} ((X = x) \inter (Y = y))\]
        \item La loi \(\mathbb{P}_{X}\) est dite première loi marginale du couple \((X, Y )\) et vérifie :
            \[\forall x \in  X(\Omega), \mathbb{P}_{X} (\accol{x}) = P(X = x) = \sum_{y\in Y (\Omega)} \mathbb{P} ((X = x) \inter (Y = y))\] .
        \item La loi \(\mathbb{P}_Y\) est dite seconde loi marginale du couple \((X, Y )\) et vérifie :
            \[\forall y \in  Y (\Omega), \mathbb{P}_Y ({y}) = \mathbb{P}(Y = y) = \sum_{x\in X(\Omega)} \mathbb{P} ((X = x) \inter (Y = y)) \]
    \end{enumerate}
    \underline{Remarque}\\
    \begin{itemize}
        \item La probabilité \(\mathbb{P}((X = x) \inter (Y = y))\) est souvent notée \(\mathbb{P}(X = x, Y = y)\)
        \item La loi conjointe permet de déterminer les lois marginales ; la réciproque est fausse.
        \item La connaissance de la loi de \(X\) et de la loi de \(Y\) sachant l’événement \(\accol{X = x}\) pour tout \(x \in  X(\Omega)\) permet de déterminer la loi conjointe de \(X\) et \(Y\) :
            \[\forall (x, y) \in  X(\Omega) \times Y (\Omega), \mathbb{P} ((X, Y ) = (x, y)) = \mathbb{P}(Y = y | X = x)\mathbb{P}(X = x)\]
    \end{itemize}
\end{defprop}

\subsection{Extension}
\begin{defprop}
    On définit de même la notion de \(n-\)uplet de variables aléatoires (avec \(n \geq 3\)) et, dans ce cadre, les notions de loi conjointe et lois marginales.
\end{defprop}
\section{Variables aléatoires indépendantes}
    Dans cette partie, \((\Omega, \mathbb{P})\) désigne un espace probabilisé fini.
\subsection{Cas de deux variables aléatoires}
\begin{defi}
    Les variables aléatoires \(X\) et \(Y\) définies sur \(\Omega\) sont dites indépendantes si, pour tout \((A, B)\) de \(\cal{P} (X(\Omega)) \times \cal{P} (Y (\Omega))\), les événements \((X \in  A)\) et \((Y \in  B)\) sont indépendants.
\end{defi}
\begin{defprop}[Caractérisation]
    Les variables aléatoires \(X\) et \(Y\) définies sur \(\Omega\) sont indépendantes si, et seulement si,
        \[\forall (x, y) \in  X(\Omega) \times Y (\Omega), \mathbb{P} ((X, Y ) = (x, y)) = \mathbb{P}(X = x)\mathbb{P}(Y = y)\]
\end{defprop}
\begin{defprop}[Images de variables aléatoires indépendantes]
    Si \(X\) et \(Y\) sont deux variables aléatoires définies sur \(\Omega\) indépendantes et si \(f\) et \(g\) sont des applications définies respectivement sur \(X(\Omega)\) et \(Y (\Omega)\) alors les variables aléatoires \(f (X)\) et \(g(Y )\) sont indépendantes.
\end{defprop}
\subsection{Cas des \(n-\)uplets de variables aléatoires avec \(n \geq 2\)}
    Dans la suite de cette partie, \(n\) désigne un entier naturel supérieur ou égal à \(2\).
\begin{defi}
    Les variables aléatoires \(X_1, \dots , X_n\) définies sur \(\Omega\) sont dites indépendantes si pour tout \((A_1, \dots , A_n)\) de \(\prod^n_{i=1} \cal{P} (X_i(\Omega))\), les événements \((X_1 \in  A_1), \dots , (X_n \in  A_n)\) sont indépendants.
\end{defi}
\begin{defprop}[Caractérisation]
    Les variables aléatoires \(X_1, \dots , X_n\) définies sur \(\Omega\) sont indépendantes si, et seulement si, 
    \[\forall (x_1, \dots , x_n) \in  X_1(\Omega) \times \dots \times X_n(\Omega), = \mathbb{P} ((X_1, \dots , X_n) = (x_1, \dots , x_n)) = \prod^n_{i=1} \mathbb{P}(X_i = x_i).\]
    \begin{itemize}
        \item Modélisation\\
        Pour modéliser \(n\) expériences aléatoires indépendantes, on peut utiliser un \(n-\)uplet \((X_1, \dots , X_n)\) de variables aléatoires indépendantes correspondant aux résultats des différentes expériences.
    \end{itemize}
\end{defprop}
\begin{defprop}[Somme de variables aléatoires de Bernoulli indépendantes]
    Si \(X_1, \dots , X_n\) sont des variables aléatoires définies sur \(\Omega\), indépendantes et de loi de Bernoulli \(\cal{B}(p)\) alors \(X_1 + \dots + X_n\) suit la loi binomiale \(\cal{B}(n, p)\).
\end{defprop}
\begin{defprop}[Lemme des coalitions]
    Si \(X_1, \dots , X_n\) sont des variables aléatoires indépendantes sur \(\Omega\) alors \(f (X_1, \dots , X_m)\) et \(g (X_{m+1}, \dots , X_n)\) sont des variables aléatoires indépendantes pour toutes applications \(f\) et \(g\) pour lesquelles cela a du sens.
\end{defprop}

\section{Espérance d’une variable aléatoire réelle ou complexe}
\subsection{Définition}
\begin{defprop}
    Si \(X : \Omega \to \\K\) est une variable aléatoire numérique telle que \(X(\Omega) = \accol{X_1, \dots , xq}\) alors le scalaire
        \[E(X) = \sum^q_{k=1} x_k\mathbb{P}(X = x_k) = \sum_{x\in X(\Omega)}x\mathbb{P}(X = x)\]
    est dit espérance de la variable aléatoire \(X\). Si \(E(X) = 0\), \(X\) est dite variable aléatoire centrée.\\
    \underline{Remarques}\\
    \begin{itemize}
        \item \(E(X)\), moyenne des valeurs de \(X\) pondérées par leurs probabilités, est un indicateur de position.
        \item Comme \(\Omega = \bigsqcup _{x\in X(\Omega)} \accol{\omega \in  \Omega \tq X(\omega) = x} = \bigsqcup _{x\in X(\Omega)} \accol{X = x}\), par additivité de \(\mathbb{P}\), on obtient \[E(X) = \sum_{\omega\in \Omega} X(\omega)\mathbb{P} (\accol{\omega})\]
        formule alternative utilisée pour les preuves sur l’espérance et pas pour des calculs pratiques.
    \end{itemize}
\end{defprop}
\subsection{Propriétés}
\begin{prop}
    Soient \(X : \Omega \to \K\) et \(Y : \Omega \to \K\) des variables aléatoires numériques et \((\alpha , \beta ) \in  \K^2\).
    \begin{itemize}
        \item \(E(\alpha X + \beta Y ) = \alpha E(X) + \beta E(Y ).\) \hfill (linéarité de l’espérance)
        \item Dans le cas \(\K = \R, X \geq 0 \imp E(X) \geq 0\). \hfill (positivité de l’espérance)
        \item Dans le cas \(\K = \R, X \leq Y \imp E(X) \leq E(Y )\). \hfill(croissance de l’espérance)
        \item \(\abs{E (X)} \leq E (\abs{X})\) . \hfill (inégalité triangulaire)
    \end{itemize}
\end{prop}
\subsection{Espérance de variables aléatoires usuelles}
\begin{defprop}
    \begin{enumerate}
        \item Si \(X\) est une variable aléatoire numérique constante égale à \(m\) alors \(E(X) = m\).
        \item Si \(X\) est une variable aléatoire de Bernoulli de paramètre \(p\), alors \(E(X) = p\).
        \item Si \(X\) est une variable aléatoire binomiale de paramètres \(n\) et \(p\), alors \(E(X) = np\).
    \end{enumerate}
    \underline{Remarque}\\
    En particulier, pour toute partie \(A\) de \(\Omega\), on a : \(E(\ind{A}) = \mathbb{P}(A)\) .
\end{defprop}
\subsection{Formule de transfert}
\begin{defprop}
    Soit \(E\) un ensemble quelconque.\\~\\
    Si \(X : \Omega \to E\) est une variable aléatoire et \(f : X(\Omega) \to \K\) une application numérique alors
    \[E(f (X)) = \sum_{x\in X(\Omega)}f (x)\mathbb{P}(X = x)\]
    \underline{Remarques}\\
    \begin{enumerate}
        \item Pour déterminer l’espérance de la variable aléatoire \(f (X)\) :
        \begin{itemize}
            \item il suffit de connaître la loi de \(X\) ;
            \item il n’est pas nécessaire de connaître la loi de \(f (X)\).
        \end{itemize}
        \item Si \((X_1, \dots , X_n)\) est un \(n-\)uplet de variables aléatoires sur \(\Omega\) et si \(f : X_1(\Omega) \times \dots \times X_n(\Omega) \to \K\) est une application numérique, la formule de transfert s’écrit :
        \[E(f (X_1, \dots , X_n)) = \sum_{(x_1,\dots,x_n)\in X_1(\Omega)\times\dots\times X_n(\Omega)} f (x_1, \dots , x_n)P ((X_1, \dots , X_n) = (x_1, \dots , x_n))\]
    \end{enumerate}
\end{defprop}
\subsection{Produit de variables aléatoires indépendantes}
\begin{defprop}
    Si \(X : \Omega \to \K\) et \(Y : \Omega \to \K\) sont des variables aléatoires numériques indépendantes alors 
        \[E(XY ) = E(X)E(Y )\].
    \underline{Remarques}\\
    \begin{itemize}
        \item Plus généralement, pour \(n \in  \Ns\), \(n \geq 3\), si \(X_1 : \Omega \to \K, \dots , X_n : \Omega \to \K\) sont des variables aléatoires numériques indépendantes alors
        \[E\paren{\prod^n_{i=1}X_i}= \prod^n_{i=1} E(X_i)\].
        \item Les réciproques sont fausses.
    \end{itemize}
\end{defprop}
\section{Variance, écart type, covariance}
\subsection{Variance et écart-type d’une variable aléatoire réelle}
\begin{defi}
    Soit \(X : \Omega \to \R\) une variable aléatoire réelle.
    \begin{enumerate}
        \item On appelle variance de \(X\) le réel, noté \(V(X)\), défini par
            \[V (X) = E (X - E(X))^2\]
            Dans le cas \( V(X) = 1\), \(X\) est dite variable aléatoire réduite.
        \item On appelle écart-type de \(X\) le réel, noté\( \sigma (X)\), défini par
            \[\sigma (X) = \sqrt{V(X)}\]
    \underline{Remarque}\\
        Les réels positifs \(V(X)\) et \(\sigma (X)\) sont des indicateurs de dispersion des valeurs de \(X\) autour de \(E(X)\) ;\\
        s’ils sont petits (resp. grands), il y a faible (resp. forte) dispersion.
    \end{enumerate}
\end{defi}
\begin{prop}
    Soient \(X : \Omega \to \R\) une variable aléatoire réelle et \((\alpha , \beta )\) un couple de réels.
    \begin{enumerate}
        \item On a : \(V(X) = E(X^2) - (E(X))^2\).
        \item On a : \(V(\alpha X + \beta ) = \alpha^2 V(X)\).
    \end{enumerate}
    \underline{Remarque}\\
    Dans le cas \(\sigma (X) > 0\), la variable aléatoire \(\frac{X - E(X)}{\sigma (X)}\) est centrée réduite.
\end{prop}
\begin{defprop}[Variance de variables aléatoires suivant des lois usuelles]
    \begin{enumerate}
        \item Si \(X\) est une variable aléatoire réelle constante égale à \(m\) alors \(V(X) = 0\).
        \item Si \(X\) est une variable aléatoire de loi de Bernoulli de paramètre \(p\), alors \(V(X) = p(1 - p)\).
        \item Si \(X\) est une variable aléatoire de loi binomiale de paramètres \(n\) et \(p\), alors \(V(X) = np(1 - p)\).
    \end{enumerate}
\end{defprop}
\subsection{Covariance de deux variables aléatoires réelles}
\begin{defi}
    Si \(X : \Omega \to \R\) et \(Y : \Omega \to \R\) sont deux variables aléatoires réelles alors le réel, noté \(\cov{X}{Y}\), défini par
    \[\cov{X}{Y} = E ((X - E(X)) (Y - E(Y )))\]
    est dit covariance de \(X\) et \(Y\) .\\~\\
    Dans le cas où \(\cov{X}{Y} = 0\), on dit que les variables aléatoires \(X\) et \(Y\) sont décorrélées.
\end{defi}
\begin{prop}
    Si \(X : \Omega \to \R\) et  \(Y : \Omega \to \R\) sont deux variables aléatoires réelles alors
    \[\cov{X}{Y} = E(XY ) - E(X)E(Y )\]
\end{prop}
\begin{defprop}[Covariance de deux variables aléatoires réelles indépendantes]
    Si \(X : \Omega \to \R\) et  \(Y : \Omega \to \R\) sont deux variables aléatoires réelles indépendantes alors \(\cov{X}{Y} = 0\).\\
    \underline{Remarque}\\
    Autrement dit, deux variables aléatoires réelles indépendantes sont décorrélées ; la réciproque est fausse.
\end{defprop}
\begin{defprop}[Variance d’une somme de deux variables aléatoires réelles]
    Si \(X : \Omega \to \R\) et  \(Y : \Omega \to \R\) sont deux variables aléatoires réelles alors :
    \[V(X + Y ) = V(X) + V(Y ) + 2\cov{X}{Y}\].
\end{defprop}
\begin{defprop}[Variance d’une somme de deux variables aléatoires réelles décorrélées]
  
    Soit \(X : \Omega \to \R\) et  \(Y : \Omega \to \R\) deux variables aléatoires réelles.\\~\\
    \(X\) et \(Y\) sont décorrélées si, et seulement si, \(V(X + Y ) = V(X) + V(Y )\).  
\end{defprop}

\begin{defprop}[Variance de \(n\) variables aléatoires réelles (programme de \(2\)e année MPI)]  
    La variance de la somme de \(n(\geq 3)\) variables aléatoires réelles sera abordée en MPI :
    \[V\paren{\sum^n_{k=1}X_k}=\sum^n_{k=1}V(Xk) + 2 \sum_{1\leq i< j\leq n} \cov{X_i}{X_j}\]
    \underline{Remarque}\\
    Si \(X\) est une variable aléatoire binomiale de paramètres \(n\) et \(p\) alors on écrire \(X = X_1 + \dots + X_n\) avec \(X_1, \dots , X_n\) des variables aléatoires de Bernoulli de même paramètre \(p\) et indépendantes. On retrouve alors la variance connue pour une variable aléatoire binomiale \(X\) avec la formule ci-dessus :
    \[V(X) =\sum^n_{k=1} V(X_i) =\sum^n_{k=1}p(1 - p) = np(1 - p)\]
\end{defprop}
\subsection{Inégalités probabilistes}
\begin{defprop}[Inégalité de Markov]
    Si \(X : \Omega \to \K\) est une variable aléatoire numérique et \(a\) un réel strictement positif alors
    \[\mathbb{P} (\abs{X}\geq a) \leq \frac{E (\abs{X})}{a}\] .
\end{defprop}
\begin{defprop}[Inégalité de Bienaymé-Tchebychev]
    Si \(X : \Omega \to \R\) est une variable aléatoire réelle et a un réel strictement positif alors
    \[\mathbb{P} (\abs{X - E(X)} \geq a) \leq \frac{V (X)}{a^2}\] .
\end{defprop}
% \chapter{fonction de deux variable réelles}
\minitoc 
Dans ce chapitre, on s’intéresse à des fonctions de deux variables réelles à valeurs réelles (vision géométrique, calculs de dérivées partielles et "règle de la chaîne" essentiellement). Ce chapitre sera entièrement repris en MPI dans un cadre plus général ; le point de vue du cours de MP2I est donc essentiellement pratique sans extension ou développement théorique.
\section{Ouverts et boules de l’espace euclidien usuel \(\R^2\)}
On munit l’espace vectoriel \(\R^2\) du produit scalaire usuel noté \(\scal{.}{.}\) et de la norme associée notée \(\norme{.}\).
\subsection{Boules pour la norme euclidienne}
\begin{defprop}
    Soit \(r \in \Rps\) et \((x_0, y_0) \in \R^2\).\\~\\
    On appelle boule ouverte de centre \(a = (x_0, y_0)\) et de rayon \(r\) pour la norme \(\norme{.}\), l’ensemble \(\cal{B}(a, r)\) défini par :
    \[\cal{B}(a, r) = \accol{(x, y) \in \R^2 \tq \norme{(x, y) - (x_0, y_0)} < r} = \accol{(x, y) \in \R^2 \tq \sqrt{(x - x_0)^2 + (y - y_0)^2} < r}\]
\end{defprop}
\subsection{Ouverts pour la norme euclidienne}
\begin{defprop}
    Soit \(\cal{O}\) une partie de \(\R^2\).\\~\\
    \(\cal{O}\) est dit ouvert de \(\R^2\) (pour la norme euclidienne) si tout point de \(\cal{O}\) est centre d’une boule ouverte (pour la norme euclidienne) incluse dans \(\cal{O}\) :
    \[\forall a \in \cal{O}, \exists r \in \Rps, \cal{B}(a, r) \subset \cal{O}\].
\end{defprop}
\subsection{Continuité des fonctions de deux variables réelles à valeurs réelles}
\begin{defprop}[Représentation graphique]
   
    A toute fonction \(f : \cal{U} \to  \R\) définie sur une partie \(\cal{U}\) de\( \R^2\) à valeurs dans \(\R\) peut être associé l’ensemble de points de\( \R^3\)
    \[S = \accol{(x, y, f (x, y)) \tq (x, y) \in \cal{U}}\]
    dit surface d’équation cartésienne \(z = f (x, y)\) permettant ainsi de représenter graphiquement \(f\) .
\end{defprop}
\begin{defprop}[Continuité sur un ouvert de \(\R^2\)]
    
    Soit \(\cal{O}\) un ouvert de \(\R^2\) et \(f : \cal{O} \to  \R\) une fonction définie sur \(\cal{O}\).\\~\\
    \(f\) est dite continue sur l’ouvert \(\cal{O}\) si \(f\) est continue en tout \((x_0, y_0)\) de \(\cal{O}\) ce qui se traduit par :
    \[\forall (x_0, y_0) \in \cal{O}, \forall \epsilon \in \Rps, \exists \delta \in \Rps, \forall (x, y) \in \cal{O}, \underbrace{\sqrt{(x - x_0)^2 + (y - y_0)^2}}_{= \norme{(x,y)-(x_0,y_0)}} \leq \delta \imp \abs{f (x, y) - f (x_0, y_0)} \leq \epsilon\]
    \underline{Remarque}\\
    Cette définition étend la notion de continuité vue pour les fonctions d’une variable réelle à valeurs réelles ; la norme euclidienne sur \(\R^2\) remplace ici la valeur absolue, norme usuelle sur \(\R\).
\end{defprop}
\begin{prop}
    \begin{enumerate}
        \item La combinaison linéaire de fonctions continues sur un ouvert de \(\R^2\) à valeurs dans \(\R\) est continue.
        \item Le produit de fonctions continues sur un ouvert de \( \R^2\) à valeurs dans \(\R\) est continu.
        \item L’inverse d’une fonction continue sur un ouvert de \(\R^2\) à valeurs dans \(\Rs\) est continue.
        \item La composée, à gauche ou à droite, d’une fonction continue sur un ouvert de \(\R^2\) à valeurs dans \(\R\) par une fonction continue est continue. Plus précisément :
        \begin{itemize}
            \item Si \(f : \cal{O} \to  \R\) est continue sur un ouvert \(\cal{O}\) de \(\R^2\) et si \(g : \cal{O}' \to  \R^2\) est continue sur un ouvert \(\cal{O}'\) de \(\R^2\) avec \(g(\cal{O}') \subset \cal{O}\) alors \(f \circ g\) est continue sur \(\cal{O}'\).
            \item Si \(f : \cal{O} \to  \R\) est continue sur un ouvert \(\cal{O}\) de \(\R^2\) et si \(h : \cal{O}' \to  \R\) est continue sur un ouvert \(\cal{O}'\) de \(\R\) avec \(f (\cal{O}) \subset \cal{O}'\) alors \(h \circ f\) est continue sur \(\cal{O}\).
        \end{itemize}
    \end{enumerate}
\end{prop}
\begin{ex}
    \begin{itemize}
        \item Les fonctions polynomiales de \(\R^2\) dans \(\R\) sont continues sur tout ouvert de \(\R^2\).
        \item Les quotients de fonctions polynomiales définies sur \(\R^2\) sont continues sur tout ouvert de \(\R^2\) où leur dénominateur ne s’annule pas.
    \end{itemize}
\end{ex}
\section{Dérivées partielles d’ordre \(1\)}
\subsection{Dérivées partielles d’ordre \(1\)}
    Soit \(f : \cal{O} \to  \R\) une fonction définie sur un ouvert \(\cal{O}\) de \(\R^2\) et \(a = (x_0, y_0)\) un point de \(\cal{O}\).
\begin{defprop}[Dérivées partielles en un point]
    \begin{itemize}
        \item On dit \(f\) admet une dérivée partielle d’ordre \(1\) en \((x_0, y_0)\) par rapport à la première place si la fonction \(f_1 : t \mapsto f (t, y_0)\) est dérivable en \(x_0\).
        Dans ce cas, la dérivée obtenue est notée \(\frac{\partial f}{\partial x} (x_0, y_0)\) :
        \[\frac{\partial f}{\partial x} (x_0, y_0) = f'_{1}(x_0) =  \lim_{t\to x_0} \frac{f (t, y_0) - f (x_0, y_0)}{t - x_0} = \lim_{h\to 0} \frac{f (h + x_0, y_0) - f (x_0, y_0)}{h} \]
        \item On dit \(f\) admet une dérivée partielle d’ordre \(1\) en \((x_0, y_0)\) par rapport à la seconde place si la fonction \(f_2 : t \mapsto f (x_0, t)\) est dérivable en \(y_0\).
        Dans ce cas, la dérivée obtenue est notée \(\frac{\partial f}{\partial y} (x_0, y_0)\) :
        \[\frac{\partial f}{\partial y} (x_0, y_0) = f'_2(y_0) = \lim_{t\to y_0} \frac{f (x_0, t) - f (x_0, y_0)}{t - y_0}= \lim_{h\to 0} \frac{f (x_0, h + y_0) - f (x_0, y_0)}{h}\]
    \end{itemize}
    \underline{Remarque}\\
    Pour \((x_0, y_0) \in \cal{O}\), les fonctions \(f_1 : t \mapsto f (t, y_0)\) et \(f_2 : t \mapsto f (x_0, t)\) sont respectivement appelées première fonction partielle et seconde fonction partielle de \(f\) en \((x_0, y_0)\).
\end{defprop}
\begin{defprop}[Dérivées partielles sur un ouvert]
    Si \(f\) admet une dérivée partielle d’ordre \(1\) en tout \((x_0, y_0)\) de \(\cal{O}\) par rapport :
    \begin{itemize}
    \item à la première place alors la fonction \((x_0, y_0) \mapsto \frac{\partial f}{\partial x} (x_0, y_0)\) définie sur \(\cal{O}\) à valeurs dans \(\R\), est appelée première fonction dérivée partielle d’ordre \(1\) et notée \(\frac{\partial f}{\partial x}\) ;
    \item à la seconde place alors la fonction \((x_0, y_0) \mapsto \frac{\partial f}{\partial y} (x_0, y_0)\) définie sur \(\cal{O}\) à valeurs dans \(\R\), est appelée seconde fonction dérivée partielle d’ordre \(1\) et notée \(\frac{\partial f}{\partial y}\)
    \end{itemize} 
\end{defprop}
\section{Existence de dérivées partielles et continuité : point de vigilance !}
\begin{defprop}
    Contrairement aux cas des fonctions d’une variable réelle à valeurs réelles pour lesquelles la dérivabilité implique la continuité, l’existence de dérivées partielles d’ordre \(1\) pour une fonction de deux variables réelles réelles à valeurs dans \(\R\) n’assure pas sa continuité.\\~\\
    \underline{Exemple}\\
    La fonction \(g\) définie sur \(\R^2\) par \(g(x, y) = \frac{xy}{x^2 + y^2}\) si \((x, y)\neq (0, 0)\) et \(g(0, 0) = 0\) admet des dérivées partielles d’ordre \(1\) en \(0\) mais n’est pas continue en \((0, 0)\) car la suite \(\paren{\frac{1}{n} , \frac{1}{n} }_{n\geq 1}\) tend vers \((0, 0)\) alors que la suite \(\paren{g\paren{ \frac{1}{n} , \frac{1}{n}}}\) ne tend pas vers \(g(0, 0)\).
\end{defprop}
\subsection{Classe \(\cal{C}^{1}\)}
\begin{defi}
    Soit \(f : \cal{O} \to  \R\) une fonction définie sur un ouvert \(\cal{O}\) de \(\R^2\).\\~\\
    \(f\) est dite de classe \(\cal{C}^{1}\) sur l’ouvert \(\cal{O}\) si ses dérivées partielles d’ordre \(1\) existent et sont continues sur \(\cal{O}\).
\end{defi}
\begin{defprop}[Opérations sur les applications de classe \(\cal{C}^{1}\)]
    \begin{enumerate}
        \item La combinaison linéaire de fonctions de classe \( \cal{C}^{1}\) sur un ouvert de \(\R^2\) à valeurs dans \(\R\) est de classe \(\cal{C}^{1}\).
        \item Le produit de fonctions de classe \(\cal{C}^{1}\) sur un ouvert de \(\R^2\) à valeurs dans \(\R\) est de classe \(\cal{C}^{1}\).
        \item L’inverse d’une fonction de classe \(\cal{C}^{1}\) sur un ouvert de \(\R^2\) à valeurs dans \(\Rs\) est de classe \(\cal{C}^{1}\).
        \item La composée, à gauche ou à droite, d’une fonction de classe \(\cal{C}^{1}\) sur un ouvert de \(\R^2\) à valeurs dans \(\R\) par une fonction de classe \(\cal{C}^{1}\) est de classe \(\cal{C}^{1}\). (preuve avec la règle de la chaîne vue au \(II\)).
    \end{enumerate}
\end{defprop}
\begin{ex}
    \begin{itemize}
        \item Les fonctions polynomiales de \(\R^2\) dans \(\R\) sont de classe \(\cal{C}^{1}\) sur tout ouvert de \(\R^2\).
        \item Les quotients de fonctions polynomiales définies sur \(\R^2\) sont de classe \(\cal{C}^{1}\) sur tout ouvert de \(\R^2\) où leur dénominateur ne s’annule pas.
    \end{itemize}
\end{ex}
\begin{defprop}[Développement limité (preuve hors programme)]
    Soit \(f : \cal{O} \to  \R\) une fonction définie sur un ouvert \(\cal{O}\) de \(\R^2\).\\~\\
    Si \(f\) est de classe \(\cal{C}^{1}\) sur l’ouvert \(\cal{O}\) alors,\\~\\
    pour tout \((x_0, y_0)\) de \(\cal{O}\) et pour tout \((h, k) \in \R^2\) tel que \((x_0 + h, y_0 + k) \in \cal{O}\), on a :
    \[f (x_0 + h, y_0 + k) = f (x_0, y_0) + \frac{\partial f}{\partial x} (x_0, y_0) h + \frac{\partial f}{\partial y} (x_0, y_0) k + \o{\norme{(h, k)}}\]
    avec \[\lim_{(h,k)\to (0,0)} \frac{\o{\norme{(h, k)}}}{\norme{(h, k)}} = 0\]
    \underline{Remarques}\\
    \begin{itemize}
        \item Pour tout \((x_0, y_0) \in \cal{O}\), le caractère "ouvert" de \(\cal{O}\) assure la possibilité de trouver une boule ouverte centrée en \((x_0, y_0)\) incluse dans \(\cal{O} \) donc de trouver des \((h, k) \in \R^2\) tel que \((x_0 + h, y_0 + k) \in \cal{O}\).
        \item Ce développement limité de \(f\) en \((x_0, y_0)\) donne une approximation locale en \((x_0, y_0)\) de la fonction \((h, k) \mapsto f (x_0 + h, y_0 + k) - f (x_0, y_0)\) par l’application linéaire
        \[ (h, k) \mapsto \frac{\partial f}{\partial x} (x_0, y_0) h + \frac{\partial f}{\partial y} (x_0, y_0) k\]
        Cela préfigure la notion de différentielle de \(f\) en \((x_0, y_0)\) qui sera vue en MPI.
        \item Si on munit \(\R^3\) de sa structure euclidienne usuelle, le plan d’équation cartésienne
        \[z - z_0 = \frac{\partial f}{\partial x} (x_0, y_0) (x - x_0) + \frac{\partial f}{\partial y} (x_0, y_0) (y - y_0)\]
        est dit plan tangent à la surface \(S\) d’équation \(z = f (x, y)\) au point \(M_0(x_0, y_0, z_0)\).
    \end{itemize}
\end{defprop}
\subsection{Classe \(\cal{C}^{1}\) et continuité}
\begin{defprop}
    Soit \(f : \cal{O} \to  \R\) une fonction définie sur un ouvert \(\cal{O}\) de \(\R^2\).\\~\\
    Si \(f\) est de classe \(\cal{C}^{1}\) sur \(\cal{O}\) alors \(f\) est continue sur \(\cal{O}\).\\
    \underline{Remarque}\\
    On rappelle que l’existence de dérivées partielles d’ordre \(1\) n’assure pas la continuité (cf \(II. 2\)).
\end{defprop}
\subsection{Gradient d’une fonction de classe \(\cal{C}^{1}\)}
\begin{defprop}[Définition par les coordonnées]
    Soit \(f : \cal{O} \to  \R\) une fonction définie et de classe \(\cal{C}^{1}\) sur un ouvert \(\cal{O}\) de \(\R^2\) et \((x_0, y_0)\) un point de \(\cal{O}\).
    On appelle gradient de \(f\) en \((x_0, y_0)\) le vecteur de \(\R^2\) noté \(\nabla f (x_0, y_0)\) défini par
    \[\nabla f (x_0, y_0) = \paren{\frac{\partial f}{\partial x} (x_0, y_0), \frac{\partial f}{\partial y} (x_0, y_0)}\]
    dans la base usuelle de \(\R^2\).
\end{defprop}
\begin{prop}
    Si \(f\) est de classe \(\cal{C}^{1}\) sur l’ouvert \(\cal{O}\) et \((x_0, y_0)\) est un point de \(\cal{O}\) alors
    \[\forall (h, k) \in \R^2, \scal{ \nabla f (x_0, y_0)}{ (h, k) } = \frac{\partial f}{\partial x} (x_0, y_0) h + \frac{\partial f}{\partial y} (x_0, y_0) k\]
    où \(\scal{}{}\) désigne le produit scalaire usuel sur \(\R^2\).\\
    \underline{Remarques}\\
    \begin{itemize}
        \item Cela résulte du fait que la base usuelle de \(\R^2\) est orthonormée pour le produit scalaire usuel.
        \item  Lorsque \(f\) est de classe \(\cal{C}^{1}\) sur l’ouvert \(\cal{O}\), le développement limité de \(f\) en tout \((x_0, y_0)\) de \(\cal{O}\) à l’ordre \(1\) peut donc s’écrire sous la forme
        \[f (x_0 + h, y_0 + k) = f (x_0, y_0) + \scal{ \nabla f (x_0, y_0)}{ (h, k) } + o(\norme{(h, k)})\]
        Lorsque \(\nabla f (x_0, y_0)\neq 0_{\R^2}\) , le vecteur gradient donne la direction dans laquelle \(f\) croît le plus vite.
    \end{itemize}
\end{prop}
\section{Extremums}
\subsection{Définitions}
\begin{defi}
    Soit \(f\) une application définie sur une partie \(\cal{U}\) de \(\R^2\), à valeurs dans \(\R\), et \(a\) un point de \(\cal{U}\).
    \begin{enumerate}
        \item On dit que \(f\) admet :
        \begin{enumerate}
            \item un maximum local en \(a = (x_0, y_0)\) s’il existe \(r > 0\) tel que :
                \[\forall (x, y) \in \cal{U} \inter \cal{B}(a, r), f (x, y) \leq f (x_0, y_0)\]
            \item un minimum local en \(a = (x_0, y_0)\) s’il existe \(r > 0\) tel que :
                \[\forall (x, y) \in\cal{U} \inter \cal{B}(a, r), f (x_0, y_0) \leq f (x, y)\]
            \item un extremum local en \(a = (x_0, y_0)\) si \(f\) admet un maximum ou un minimum local en \(a\).
        \end{enumerate}
    \item On dit que \(f\) admet :
        \begin{enumerate}
            \item un maximum global en \(a = (x_0, y_0)\) si : \(\forall (x, y) \in\cal{U}, f (x, y) \leq f (x_0, y_0)\).
            \item un minimum global en \(a = (x_0, y_0)\) si : \(\forall (x, y) \in\cal{U}, f (x_0, y_0) \leq f (x, y)\).
            \item un extremum global en \(a\) si \(f\) admet un maximum ou un minimum global en \(a\).
        \end{enumerate}
    \end{enumerate}
\end{defi}
\subsection{Condition NECESSAIRE d’existence d’un extremum local}
\begin{defprop}
    Si \(f\) est une fonction définie et de classe \(\cal{C}^{1}\) sur un ouvert \(\cal{O}\) de \(\R^2\), à valeurs réelles qui admet un extremum local en \((x_0, y_0) \in \cal{O}\) alors
    \[\nabla f ((x_0, y_0)) = 0_{\R^2}\] 
    \underline{Remarques}\\
    \begin{itemize}
        \item Tout \((x_0, y_0) \in \cal{O}\) tel que \(\nabla f ((x_0, y_0)) = 0_{\R^2}\) est dit point critique de \(f\) .
        \item La recherche d’éventuel extremum local pour \(f\) de classe \(\cal{C}^{1}\) sur un ouvert, à valeurs réelles,
        \begin{itemize}
            \item commence par la détermination des éventuels points critiques de \(f\) sur l’ouvert ;
            \item se pousuit par l’étude du signe de \(f (x, y) - f (x_0, y_0)\) au voisinage des points critiques \((x_0, y_0)\) :
            \begin{itemize}
                \item si ce signe est constant sur une boule ouverte centrée en \((x_0, y_0)\) alors il y a extremum local ;
                \item si ce n’est pas le cas, il n’y a pas d’extremum local.
            \end{itemize}
        \end{itemize}
    \end{itemize}
\end{defprop}
\section{Règle de la chaîne}
Soit \(f : \cal{O} \to  \R\) une fonction définie sur un ouvert \(\cal{O}\) de \(\R^2\).
\subsection{Dérivée selon un vecteur}
\begin{defprop}
     On dit que la fonction \(f\) est dérivable en \((x_0, y_0) \in \cal{O}\) selon le vecteur \(u = (h, k)\) de \(\R^2\) si la fonction \(t \mapsto f ((x_0, y_0) + t(h, k))\) est dérivable en \(0\). On note alors
     \[D_uf (x_0, y_0) = \lim_{t\to 0} \frac{1}{t} (f ((x_0, y_0) + t(h, k)) - f (x_0, y_0))\]
    et on dit que \(D_uf (x_0, y_0)\) est la dérivée de \(f\) en \((x_0, y_0)\) selon le vecteur \(u\).\\
    \underline{Remarque}\\
    orsqu’elles existent, les dérivées partielles premières de \(f\) en \((x_0, y_0)\) sont donc les dérivées de \(f\) en \((x_0, y_0)\) selon les deux vecteurs \(e_1\) et \(e_2\) de la base usuelle de \(\R^2\).
    \begin{itemize}
        \item Si \(f\) admet des dérivées en tout \((x_0, y_0) \in \cal{O}\) selon le vecteur \(u = (h, k)\) de \(\R^2\) alors la fonction \((x_0, y_0) \mapsto D_uf (x_0, y_0)\) définie sur \(\cal{O}\) à valeurs dans \(\R\), est appelée fonction dérivée de \(f\) selon le vecteur \(u\) et notée \(D_uf\) . On a, en particulier :
        \[D_{e_1} f = \frac{\partial f}{\partial x} \qquad \text{ et } \qquad D_{e_2} f = \frac{\partial f}{\partial y}\]
    \end{itemize}
\end{defprop}

\subsection{Expression des dérivées directionnelles avec le gradient}
\begin{defprop}
    Si \(f\) est une fonction définie et de classe \(\cal{C}^{1}\) sur un ouvert \(\cal{O}\) de \(\R^2\) et \((x_0, y_0)\) un point de \(\cal{O}\) alors \(f\) admet des dérivées en \((x_0, y_0)\) selon tout vecteur \(u = (h, k)\) de \(\R^2\) données par :
        \[ D_uf (x_0, y_0) = \scal{ \nabla f (x_0, y_0)}{u} \]
\end{defprop}
\subsection{Règle de la chaîne}
\begin{theo}
    Soit \(x\) et \(y\) des fonctions de \(I\) (intervalle non vide, non réduit à un point, de \(\R\)) dans \(\R\) tel que :
    \[\forall t \in I, (x(t), y(t)) \in \cal{O}\]
    Si \(x\) et \(y\) sont de classe \(\cal{C}^{1}\) sur \(I\) et si \(f\) est de classe \(\cal{C}^{1}\) sur l’ouvert \(\cal{O}\) alors la fonction 
        \[\psi : t \mapsto f (x(t), y(t))\]
    est de classe \(\cal{C}^{1}\) sur \(I\) et sa dérivée est :
    \[\psi' : t \mapsto \frac{\partial f}{\partial x} (x(t), y(t)) x'(t) + \frac{\partial f}{\partial y} (x(t), y(t)) y'(t)\]
    ce que l’on peut encore écrire
    \[\frac{d}{dt} (f (x(t), y(t))) = \frac{\partial f}{\partial x} (x(t), y(t)) x'(t) + \frac{\partial f}{\partial y} (x(t), y(t)) y'(t)\]
\end{theo}
\begin{defprop}[Interprétation géométrique de la règle de la chaîne]
    Avec les hypothèses et notations du \(II. 3. 1.\), en notant \(\gamma : I \mapsto \R^2\) la fonction de classe \(\cal{C}^{1}\) définie par 
    \[\forall t \in I, \gamma(t) = (x(t), y(t))\]
    la dérivée de la fonction \(f \circ \gamma\) peut s’écrire :
    \[\forall t \in I, (f \circ \gamma)' (t) = \scal{ \nabla f (\gamma(t))}{ \gamma'(t) }\]
    On parle alors de dérivée de \(f\) le long de l’arc paramétré donné par \(\gamma : t \mapsto (x(t), y(t))\).
\end{defprop}
\begin{defprop}[Application géométrique de la règle de la chaîne]
    Soit \(k \in \R\) et \((x_0, y_0) \in \cal{O}\).\\~\\
    Si \(f\) est de classe \(\cal{C}^{1}\) sur \(\cal{O}\) avec \(\nabla f (x_0, y_0)\neq 0_{\R^2}\) et \(f (x_0, y_0) = k\) alors \(\nabla f (x_0, y_0)\) est orthogonal à la courbe
    \[C_k = \accol{(x, y) \in \cal{O} \tq f (x, y) = k}\]
    appelée ligne de niveau de \(f\) .
\end{defprop}
\subsection{Propriétés}
\begin{prop}
    Soit \(x\) et \(y\) des fonctions de \(\cal{O}'\) (ouvert non vide de \(\R^2\)) dans \(\R\) tel que :
    \[\forall (u, v) \in \cal{O}', (x(u, v), y(u, v)) \in \cal{O}\]
    Si \(x\) et \(y\) sont de classe \(\cal{C}^{1}\) sur \(\cal{O}'\) et si \(f\) est de classe \(\cal{C}^{1}\) sur \(\cal{O}\) alors l’application
    \[g : (u, v) \mapsto f (x(u, v), y(u, v))\]
    est de classe \(\cal{C}^{1}\) sur \(\cal{O}'\) et ses dérivées partielles d’ordre \(1\) sont :
    \[\frac{\partial g}{\partial u}(u, v) = \frac{\partial x}{\partial u}(u, v)\frac{\partial f}{\partial x} (x(u, v), y(u, v)) + \frac{\partial y}{\partial u}(u, v)\frac{\partial f}{\partial y} (x(u, v), y(u, v)) \]
    \[\frac{\partial g}{\partial v} (u, v) = \frac{\partial x}{\partial v} (u, v)\frac{\partial f}{\partial x} (x(u, v), y(u, v)) + \frac{\partial y}{\partial v} (u, v)\frac{\partial f}{\partial y }(x(u, v), y(u, v)) \]
    Exemple important du passage en coordonnées polaires\\~\\
    Soit \(\cal{O}'\) un ouvert non vide de \(\R^2\) tel que \(\forall (r, \theta) \in \cal{O}'\), \((r \cos \theta, r \sin \theta) \in \cal{O}\).
    \begin{itemize}
        \item Les fonctions \(x : (r, \theta) \mapsto r \cos \theta\) et \(y : \theta \mapsto r \sin \theta\) sont de classe \(\cal{C}^{1}\) sur \(\cal{O}'\) car leurs dérivées partielles d’ordre \(1\) existent et sont continues.
        \item Ainsi, si \(f\) de classe \( \cal{C}^{1}\) sur \(\cal{O}\) alors \(g : (r, \theta) \mapsto f (r \cos \theta, r \sin \theta)\) est de classe \(\cal{C}^{1}\) sur \(\cal{O}'\) avec :
        \[ \frac{\partial g}{\partial r} (r, \theta) = -r \cos \theta \frac{\partial f}{\partial x} (r \cos \theta, r \sin \theta) + r \sin \theta \frac{\partial f}{\partial y} (r \cos \theta, r \sin \theta) \]
        \[ \frac{\partial g}{\partial \theta} (r, \theta) = -r \sin \theta \frac{\partial f}{\partial x} (r \cos \theta, r \sin \theta) + r \cos \theta \frac{\partial f}{\partial y }(r \cos \theta, r \sin \theta) \]
    \end{itemize}
\end{prop}

% \chapter{Famille sommables de réels ou complexes}
\minitoc 
Dans ce chapitre où \(\K\) désigne le corps \(\R\) ou \(\C\), on prolonge les calculs de sommes finies effectués en début d’année dans le chapitre "Sommes et produits finis" en présentant un cadre qui permet de sommer "en vrac" une famille infinie et procure ainsi un grand confort de calcul. On se concentre sur la pratique, vu son importance en MPI, dans les calculs d’espérance et de variance de variables aléatoires discrètes.
\section{Familles sommables de réels positifs}
\subsection{Rappel : relation d’ordre dans la demi-droite achevée \(\intervii{0}{\pinf}\)}

\begin{defprop}
    \begin{itemize}
    \item  On appelle demi-droite achevée l’ensemble noté \(\intervii{0}{\pinf}\) défini par \(\intervii{0}{\pinf} = \intervie{0}{\pinf} \union \accol{\pinf }\) .\\~\\
        Sur cette demi-droite achevée, on étend la relation d’ordre \(\leq\) , l’addition et la multiplication connues sur \(\intervie{0}{\pinf}\) avec les conventions suivantes :
        \begin{enumerate}
            \item \(\forall x \in  \intervie{0}{\pinf} , x < \pinf\) .
            \item \(\forall x \in  \intervii{0}{\pinf} , x + (\pinf ) = (\pinf ) + x = \pinf \).
            \item \(0 \times  (\pinf ) = (\pinf ) \times  0 = 0\).
            \item \(\forall x \in  \intervii{0}{\pinf} \pd {0} , x \times  (\pinf ) = (\pinf ) \times  x = \pinf\) .
        \end{enumerate}
    \item Toute partie \(X\) de \(\intervii{0}{\pinf}\) admet une borne supérieure (plus petit des majorants) notée \(\sup X\) .
        \begin{itemize}
            \item Cas où \(X\) est une partie non vide et majorée :
                \[\sup X \text{ est égale à la borne supérieure de } X \text{ vue comme partie de }\R.\]
            \item Cas où \(X\) est la partie vide ou où \(X\) est une partie non majorée :
                \[\sup X =\begin{cases}
                    0 &\text{ si } X = \emptyset\\
                    \pinf &\text{ si }X \text{ est non vide et non majorée }
                \end{cases}\]
        \end{itemize}
    \end{itemize}
\end{defprop}
\subsection{Somme d’une famille de réels positifs}
\begin{defprop}
    Soit \(I\) un ensemble quelconque.\\~\\
    On appelle somme d’une famille \((u_i)_{i\in I}\) d’éléments de \(\intervii{0}{\pinf}\), et on note \(\sum_{i\in I} u_i\), la borne supérieure dans \(\intervii{0}{\pinf}\) de l’ensemble des sommes \(\sum_{i\in F}u_i\) quand \(F\) décrit l’ensemble des parties finies de \(I\) :
    \[\sum_{i\in I}u_i = \sup \accol{\sum_{i\in F}u_i \tq F partie finie de I}\]

    \underline{Remarques}\\
    Soit \((u_i)_{i\in I}\) une famille d’éléments de \(\intervii{0}{\pinf}\).\\~\\
    \begin{itemize}
        \item Dans le cas où \(I\) est fini, la somme \(\sum_{i\in I}u_i\) coïncide avec la notion de somme finie usuelle connue.
        \item Dans le cas où \(I = \N\), la somme \(\sum_{i\in I}u_i\) coïncide avec la notion de somme de la série \(\sum u_n\) :
        \begin{itemize}
            \item dans le cas où la série \(\sum u_n\) d’éléments de \(\Rp\) converge ;
            \item dans le cas où la série \(\sum u_n\) d’éléments de \(\Rp\) diverge avec la convention \(\sum^{\pinf}_{n=0}u_n = \pinf\) .
            \[\sum_{i\in \N}u_i =\sum^{\pinf}_{n=0}u_n\]
        \end{itemize}
        \item Invariance de la somme par permutation\\~\\
            Si \(\sigma\) est une permutation de \(I\) (i.e. une bijection de \(I\) sur \(I\) ) alors
            \[\sum_{i\in I}u_i = \sum_{i\in I}u_{\sigma(i)}\].
    \end{itemize}
\end{defprop}

\subsection{Sommabilité d’une famille de réels positifs}
    Soit \(I\) un ensemble quelconque.
\begin{defi}
    La famille de réels positifs \((u_i)_{i\in I}\) est dite sommable si sa somme vérifie \(\sum_{i\in I}u_i < \pinf\) .
\end{defi}
\begin{prop}
    Si \((a_i)_{i\in I}\) et \((b_i)_{i\in I}\) sont deux familles de réels positifs tel que, pour tout \(i\) de \(I\),\( 0 \leq  a_i \leq  b_i\) et si la famille \((b_i)_{i\in I}\) est sommable alors la famille \((a_i)_{i\in I}\) est sommable et \(\sum_{i\in I}a_i \leq  \sum_{i\in I}b_i\)
\end{prop}
\subsection{Opérations}
\begin{defprop}
    Soit \(I\) un intervalle quelconque.\\~\\
    Si \((u_i)_{i\in I}\) et \((v_i)_{i\in I}\) sont deux familles de réels positifs et\( \alpha\)  un réel positif alors
    \[\sum_{i\in I}u_i + \sum_{i\in I} v_i = \sum_{i\in I}(u_i + v_i)\qquad \text{  et }\qquad \alpha  \sum_{i\in I}u_i = \sum_{i\in I}\alpha u_i\]
    \underline{Remarque}\\
    Les conventions de calcul imposées dans \(\intervii{0}{\pinf}\) donnent du sens à ces égalités y compris dans le cas où l’une des familles de réels positifs écrites n’est pas sommable.
\end{defprop}
\subsection{Théorème de sommation par paquets positif (ADMIS)}
\begin{theo}
    Soit \(I\) et \(J\) deux ensembles quelconques.\\~\\
    Si l’ensemble \(I\) est la réunion disjointe des ensembles \(I_j\) lorsque \(j\) décrit \(J\) et si \((u_i)_{i\in I}\) est une famille de réels positifs alors la somme de cette famille vérifie
    \[\sum_{i\in I}u_i = \sum_{j\in J} \paren{\sum_{i\in I_j}u_i}\]
    \underline{Remarque}\\
    Les conventions de calcul imposées dans \(\intervii{0}{\pinf}\) donnent du sens à cette égalité y compris dans le cas où l’une des familles de réels positifs écrites n’est pas sommable.
\end{theo}
\subsection{Théorème de Fubini positif}
\begin{theo}
    Soit \(J\) et \(K\) des ensembles quelconques.\\~\\
    Si \((a_{j,k})_{(j,k)\in J\times K}\)  est une famille de réels positifs alors la somme de cette famille vérifie

    \[\sum_{(j,k)\in J\times K}a_{j,k} = \sum_{j\in J}\paren{\sum_{k\in K} a_{j,k}}= \sum_{k\in K}\paren{\sum_{j\in J} a_{j,k}}\]
    \underline{Remarques}
    \begin{itemize}
        \item Les conventions de calcul imposées dans \(\intervii{0}{\pinf}\) donnent du sens à ces égalités y compris dans le cas où l’une des familles de réels positifs écrites n’est pas sommable.
        \item Ce théorème est une conséquence du théorème de sommation par paquets vu ci-dessus pour les familles de réels positifs. Il résulte de l’écriture de l’ensemble \(I = J \times K\)  comme réunion disjointe :
        \begin{itemize}
            \item  des ensembles \(A_j =\accol{(j, k) \tq k \in K  }\) lorsque \(j\) décrit \(J\) d’une part ;
            \item  des ensembles \(B_k = \accol{(j, k) \tq j \in  J}\) lorsque \(k\) décrit \(K\)   d’autre part.
        \end{itemize}
        \item Dans le cas où \(J\) et \(K\)   sont finis, on retrouve un résultat vu dans le chapitre "Sommes et produits finis" pour les sommes doubles rectangulaires.
    \end{itemize}
\end{theo}

\section{Familles sommables d’éléments de \(\K\)  }
    Soit \(I\) un ensemble quelconque.
\subsection{Généralités}
\begin{defprop}[Sommabilité]
    
    La famille \((u_i)_{i\in I}\) d’éléments de \(\K\)   est dite sommable si la famille de réels positifs \((\abs{u_i})_{i\in I}\) l’est, autrement dit si la somme \(\sum_{i\in I}\abs{u_i}\) vérifie
    \[\sum_{i\in I}\abs{u_i} < \pinf .\]
    On note \(\cal{l}^1(I, \K  )\) ou plus simplement \(\cal{l}^1(I)\) l’ensemble des familles \((u_i)_{i\in I}\) d’éléments de \(I\) sommables.
    \underline{Remarque}
    Une sous-famille d’une famille d’éléments de \(\K\) sommable est sommable.
\end{defprop}
\begin{defprop}[Somme d’une famille sommable]
    \begin{itemize}
        \item Si \((u_i)_{i\in I}\) est une famille de réels sommable, sa somme est définie par :
            \[\sum_{i\in I}u_i = X\sum_{i\in I}u^{+}_i - \sum_{i\in I}u^{-}_i\]
        où \(u^{+}_i = \frac{1}{2} (\abs{u_i} + u_i)\) et \(u^{-}_i = \frac{1}{2} (\abs{u_i} - u_i)\)
        \item si \((u_i)_{i\in I}\) est une famille de complexes sommable, sa somme est définie par :
            \[\sum_{i\in I}u_i = \sum_{i\in I}\Reel{u_i} + i\sum_{i\in I}\Ima{u_i}.\]
    \end{itemize}
    \underline{Remarques}
    \begin{itemize}
        \item Ces définitions ont du sens puisque que si \((u_i)_{i\in I}\) est une famille de réels (resp. complexes) sommable alors, d’après \(I. 3. 2.\), les familles de réels positifs \(\paren{u^{+}_i}_{i\in I}\) et \(\paren{u^{-}_i}_{i\in I}\) (resp. de réels \(\paren{\Reel{u_i}}_{i\in I}\) et \(\paren{\Ima{u_i}}_{i\in I}\) ) sont sommables puisque, pour tout \(i\) de \(I\), on a :
        \[0 \leq  u^{+}_i \leq  \abs{u_i}\qquad \text{ et } \qquad 0 \leq  u^{-}_i \leq  \abs{u_i}\]
        \[(\text{resp.} 0 \leq  \abs{\Reel{u_i}} \leq  \abs{u_i} \qquad \text{ et } \qquad 0 \leq  \abs{\Ima{u_i}} \leq  \abs{u_i})\]
        \item Approximation de la somme d’une famille sommable par une somme finie\\
        Si \((u_i)_{i\in I}\) est une famille d’éléments de \(\K\)   sommable et si \(\epsilon \in  \Rps\), il existe une partie finie \(F\) de \(I\) telle que
        \[\abs{\sum_{i\in I}u_i - \sum_{i\in F}u_i} \leq \epsilon\]
        \item Conservation de la sommabilité et invariance de la somme par permutation.\\~\\
        Si \(\sigma\) est une permutation de \(I\) et si \((u_i)_{i\in I}\) est une famille d’éléments de \(\K \)  sommable alors la famille \(\paren{u_{\sigma(i)}}_{i\in I}\) est sommable \[\sum_{i\in I}u_i = \sum_{i\in I}u_{\sigma(i)}.\]
    \end{itemize}
\end{defprop}
\subsection{Cas particulier important des familles d’éléments de \(\K\)   indexées par \(\N\)}
\begin{defprop}
    Une famille \((u_n)_{n\in \N}\) de \(\K\) est sommable si, et seulement si, la série \(\sum_{n \geq 0} u_n\) converge absolument.\\~\\
    Dans ce cas,
    \[\sum_{n\in \N}\abs{u_n} = \sum^{\pinf}_{n=0}\abs{u_n}\qquad \text{ et } \qquad \sum_{n\in \N}u_n = \sum^{\pinf}_{n=0}u_n\]
\end{defprop}
\subsection{Théorème de majoration}
\begin{defprop}
    Si \((u_i)_{i\in I}\) est une famille d’éléments de \(\K\)   et \((v_i)\) une famille sommable de réels positifs tel que, pour tout \(i \in  I\), \(\abs{ui} \leq v_i\) alors la famille \((u_i)_{i\in I}\) est sommable.
\end{defprop}
\subsection{Linéarité de la somme}
\begin{defprop}
    Soit \((u_i)_{i\in I}\) et \((v_i)_{i\in I}\) deux familles d’éléments de \(\K \)  et\( (\alpha , \beta)\) un couple d’éléments de \(\K\)  .\\~\\
    Si les familles \((u_i)_{i\in I}\) et \((v_i)_{i\in I}\) sont sommables alors la famille \((\alpha u_i + \beta v_i)_{i\in I}\) est sommable, de somme :
    \[\sum_{i\in I}(\alpha u_i + \beta v_i) = \alpha \sum_{i\in I}u_i + \beta \sum_{i\in I}v_i\]
\end{defprop}
\subsection{Théorème de sommation par paquets (ADMIS)}
\begin{theo}
    Si l’ensemble \(I\) est la réunion disjointe des ensembles \(I_j\) lorsque \(j\) décrit \(J\) et si \((u_i)_{i\in I}\) est une famille d’éléments de \(\K\) sommable alors la somme de cette famille vérifie
    \[\sum_{i\in I}u_i = \sum_{j\in J}\paren{\sum_{i\in I_j} u_i}\]
\end{theo}
\subsection{Théorème de Fubini}
\begin{defprop}
    Soit \(J\) et \(K\)   des ensembles quelconques.\\~\\
    Si \((a_{j,k})_{(j,k)\in J\times K}\)  est une famille d’éléments de \(\K\) sommable alors la somme de cette famille vérifie
    \[\sum_{(j,k)\in J\times K}a_{j,k} = \sum_{j\in J}\paren{\sum_{k\in K} a_{j,k}}= \sum_{k\in K}\paren{\sum_{j\in J} a_{j,k}}\]
\end{defprop}
\subsection{Cas particulier}
\begin{defprop}
    Si \((b_j)_{j\in J}\) et \((c_k)_{k\in K}\)  sont sommables alors \((b_j c_k)_{(j,k)\in J\times K}\) est sommable et
    \[\sum_{(j,k)\in J\times K}b_j c_k = \sum_{j\in J}b_j \times \sum_{k\in K}c_k\]
    \underline{Remarque}\\
    C’est une application du théorème de Fubini avec la famille \((a_{j,k})_{(j,k)\in J\times K}\)  où \(a_{j,k} = b_j c_k\).
\end{defprop}
\subsection{Produit de Cauchy}
\begin{defprop}
    Soit \((u_m)_{m\in \N}\) et \((v_n)_{n\in \N}\) deux suites d’éléments de \(\K\)  .\\~\\
    Pour tout \(p \in  \N\), on note :
    \[w_p = \sum_{m+n=p}u_mv_n.\]
    Si les séries \(\sum u_m\) et \(\sum v_n\) convergent absolument alors la série \(\sum w_p\) converge absolument avec
    \[\sum^{\pinf}_{p=0}w_p =\paren{\sum^{\pinf}_{m=0}u_m}\paren{\sum^{\pinf}_{n=0}v_n}\]
    La série \(\sum w_p\) est dite série produit de Cauchy des séries \(\sum u_m\) et \(\sum v_n\).\\
    \underline{Remarque}
    Ceci est une conséquence du résultat vu au \(II. 7.\) appliqué à la famille \((a_{m,n})_{(m,n)\in \N^2}\) où \(a_{m,n} = u_mv_n\) et du théorème de sommation par paquets vu au \(II. 5.\) appliqué avec la partition \((I_p)_{p\in \N}\) de \(I = \N^2\) définie par \(I_p = \accol{(m, n) \in  \N^2\tq m + n = p}\).
\end{defprop}

\end{document}