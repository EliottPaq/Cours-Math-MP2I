% Set up the document's format to A4 and the font's size to 12pt.
\documentclass[a4paper,12pt]{report}

% Set up the document's title, author and date.
\title{Maths -- MP2I}
\author{Eliott Paquet}
\date{\today}

% Set up the input's encoding to UTF-8, the document's font and language to T1 (adapted to french) and french (the grammar linter uses this parameter).
\usepackage[utf8]{inputenc}
\usepackage[T1]{fontenc}
\usepackage[french]{babel}

\usepackage[dvipsnames]{xcolor}

% Set up the document's margins.
\usepackage{geometry}
\geometry{hmargin=1.5cm,vmargin=1.5cm}

% The three main maths packages. They are used for a lot of things.
\usepackage{amssymb,amsmath}
\usepackage{mathtools}

% Useful to create nice and easy signs or variations tables.
\usepackage{tkz-tab}

% Useful to create any kind of visual representation (graph functions, illustrate geometry problems, etc)
\usepackage{tikz}
\usetikzlibrary{patterns,angles,quotes,arrows,arrows.meta,bending,matrix,calc}

% Allows to edit the itemize environment's default item document-wide.
\usepackage{enumitem}

% Allows to define \notfoo or \nfoo (not recommended) in order for \not\foo to work as wished.
\usepackage{newtxmath}
\DeclareSymbolFont{CMletters}{OML}{cmm}{m}{it}
\DeclareMathSymbol{\nu}{\mathord}{CMletters}{23}

% Makes the table of contents clickable and gives useful commands for links in general.
\usepackage[hypertexnames=false]{hyperref}
\hypersetup{colorlinks=false,linktoc=all}

% Gives the llbracket and rrbracket commands for integer intervals.
\usepackage{stmaryrd}

% Useful to insert nice-looking quotes.
\usepackage{epigraph}

% Allows to insert chapter-specific table of contents.
\usepackage{minitoc}
\mtcselectlanguage{french}
\setcounter{minitocdepth}{6}

% Useful when units are needed.
\usepackage{siunitx}
\sisetup{
locale=FR,
detect-all,
inter-unit-product=\ensuremath{\cdot},
list-final-separator={et},
list-pair-separator={et},
range-phrase={\ensuremath{\xleftrightarrow{}}},
exponent-product=\ensuremath{\cdot},
per-mode=power-positive-first
}

\usepackage[thmmarks,hyperref]{ntheorem}
\makeatletter
\let\old@thm\@thm
\usepackage[lowercase]{theoremref}
\def\@thm#1#2#3{\def\thmref@currname{#3}\old@thm{#1}{#2}{#3}}
\makeatother

% Allows whiteboard digits with \mathds
\usepackage{dsfont}

\usepackage{needspace}

% Useful for better-looking oneline fractions
\usepackage{nicefrac}

% Set up the horizontal space before the first line of a new paragraph to 2em and the vertical space between two paragraphs to 1em.
\setlength{\parindent}{0pt}
\setlength{\parskip}{1em}

% Adds 0.5em to the vertical space between two lines in an align environment. It looks better.
\addtolength{\jot}{0.5em}

% Allows align environment to break if it's too long to fit in the page where it began.
\allowdisplaybreaks[1]

% Trick to make semicolons considered like relation operators (such as =) and therefore being equidistantly spaced from the two numbers around it.
\mathcode`;=\numexpr\mathcode`;-"3000

% Commands for size-adaptative parentheses, brackets, curly brackets, absolute value and magnitude.
\newcommand{\paren}[1]{\left(#1\right)} % (x)
\newcommand{\croch}[1]{\left[#1\right]} % [x]
\newcommand{\accol}[1]{\left\lbrace#1\right\rbrace} % {x}
\newcommand{\abs}[1]{\left\lvert#1\right\rvert} % |x|
\newcommand{\norme}[1]{\left\|#1\right\|} % ||x||
\newcommand{\floor}[1]{\left\lfloor#1\right\rfloor} % ⌊x⌋
\newcommand{\ceil}[1]{\left\lceil#1\right\rceil} % ⌈x⌉

% Commands for size-adaptative intervals and integer intervals. The commands' roots are "interv" and "interventier" and the added e or i at the end mean "excluded" and "included" respectively.
\newcommand{\intervii}[2]{\left[#1;#2\right]} % [a;b]
\newcommand{\intervee}[2]{\left]#1;#2\right[} % ]a;b[
\newcommand{\intervie}[2]{\left[#1;#2\right[} % [a;b[
\newcommand{\intervei}[2]{\left]#1;#2\right]} % ]a;b]
\newcommand{\interventierii}[2]{\left\llbracket#1;#2\right\rrbracket} % non-ASCII characters needed
\newcommand{\interventieree}[2]{\left\rrbracket#1;#2\right\llbracket} % non-ASCII characters needed
\newcommand{\interventierie}[2]{\left\llbracket#1;#2\right\llbracket} % non-ASCII characters needed
\newcommand{\interventierei}[2]{\left\rrbracket#1;#2\right\rrbracket} % non-ASCII characters needed

% Commands for usually used sets.
\newcommand{\N}{\mathbb{N}} % natural integers
\newcommand{\Ns}{\mathbb{N}^*}

\newcommand{\Z}{\mathbb{Z}} % relative integers
\newcommand{\Zp}{\mathbb{Z}_+}
\newcommand{\Zs}{\mathbb{Z}^*}
\newcommand{\Zps}{\mathbb{Z}_+^*}

\newcommand{\D}{\mathbb{D}} % decimal numbers
\newcommand{\Dp}{\mathbb{D}_+}
\newcommand{\Dm}{\mathbb{D}_-}
\newcommand{\Ds}{\mathbb{D}^*}
\newcommand{\Dps}{\mathbb{D}_+^*}
\newcommand{\Dms}{\mathbb{D}_-^*}

\newcommand{\Q}{\mathbb{Q}} % rational numbers
\newcommand{\Qp}{\mathbb{Q}_+}
\newcommand{\Qm}{\mathbb{Q}_-}
\newcommand{\Qs}{\mathbb{Q}^*}
\newcommand{\Qps}{\mathbb{Q}_+^*}
\newcommand{\Qms}{\mathbb{Q}_-^*}

\newcommand{\R}{\mathbb{R}} % real numbers
\newcommand{\Rp}{\mathbb{R}_+}
\newcommand{\Rm}{\mathbb{R}_-}
\newcommand{\Rs}{\mathbb{R}^*}
\newcommand{\Rps}{\mathbb{R}_+^*}
\newcommand{\Rms}{\mathbb{R}_-^*}
\newcommand{\Rb}{\overline{\mathbb{R}}}

\newcommand{\C}{\mathbb{C}} % complex numbers
\newcommand{\Cs}{\mathbb{C}^*}

\newcommand{\K}{\mathbb{K}}
\newcommand{\Ks}{\mathbb{K}^*}

\newcommand{\A}{\mathbb{A}}
\renewcommand{\L}[2]{\mathscr{L}\paren{#1,#2}}
\newcommand{\Lendo}[1]{\mathscr{L}\paren{#1}}

\newcommand{\prem}{\mathbb{P}}

\newcommand{\U}{\mathbb{U}} % complex numbers whose modulus is 1

\renewcommand{\P}[1]{\mathscr{P}\paren{#1}} % subsets of a set
\newcommand{\Pf}[1]{\mathscr{P}_f\paren{#1}} % finite subsets of a set
\newcommand{\F}[2]{\mathscr{F}\paren{#1,#2}} % functions from 1 to 2
\newcommand{\V}[1]{\mathscr{V}\paren{#1}} % neighborhood of a number

% Redefines \Re and \Im to print Re and Im (the same way as ln or lim) instead of fraktur R and I which don't look nice and are less readable.
\renewcommand{\Re}{\operatorname{Re}}
\renewcommand{\Im}{\operatorname{Im}}
\newcommand{\Card}{\operatorname{Card}}

% Command to print an upright e for the exponential instead of a slanted e and put the exponent.
\newcommand{\e}[1]{\mathrm{e}^{#1}}

% Command to print the imaginary i with a little space on the right. This way, the exponents don't look confusing. \i normally prints a dotless i.
\renewcommand{\i}{i\mkern1mu}

% Redefines \vec such that the arrow covers the whole name of the vector.
%\renewcommand{\vec}[1]{\overrightarrow{#1}}

% Commands for 2D and 3D vectors' coordinates
\newcommand{\dcoords}[2]{\begin{pmatrix}#1\\#2\end{pmatrix}}
\newcommand{\tcoords}[3]{\begin{pmatrix}#1\\#2\\#3\end{pmatrix}}

% Redefines binom to print nicer parentheses around the numbers.
\renewcommand{\binom}[2]{\begin{pmatrix}#2\\#1\end{pmatrix}}

% Command for a QED black square. It automatically prints a whitespace before the square such that it looks nice.
\newcommand{\cqfd}{\text{ }\blacksquare}

%Acronym for "privé de" 
\newcommand{\pd}{\backslash}


% Commands with more explicit names for the best way to express divisibility (mid and nmid).
\newcommand{\divise}{\mid}
\newcommand{\notdivise}{\nmid}

% Commands that do the exact same thing but with explicit names for a complex number's conjugate and an event's negation in probability.
\newcommand{\conj}[1]{\overline{#1}}

% Command for a size-adaptative middle bar meaning "such that" (with spacing around it in order to look nice).
\newcommand{\tq}{\;\middle|\;}

% Command with an explicit name for the scalar product.
\newcommand{\scal}{\cdot}
\newcommand{\vecto}{\operatorname{_\wedge}}

% Shortcut for forcing displaystyle in inline mode.
\newcommand{\ds}{\displaystyle}

% Make the not version of implies, impliedby and iff look nice.
\newcommand{\notimp}{\centernot{\imp}}
\newcommand{\notimpr}{\centernot{\impr}}
\newcommand{\notssi}{\centernot{\ssi}}

\renewcommand{\subset}{\subseteq}
\renewcommand{\supset}{\supseteq}
\newcommand{\notsubset}{\centernot{\subset}}
\newcommand{\notsupset}{\centernot{\supset}}

% Shortcut for P(event).
\newcommand{\proba}[1]{\mathbb{P}\paren{#1}}
\newcommand{\probacond}[2]{\mathbb{P}_{#2}\paren{#1}}

% More explicit names for land (logical and) and lor (logical or).
\newcommand{\et}{\land}
\newcommand{\ou}{\lor}
\newcommand{\non}{\lnot}

% Explicitly named environment for tkz-tab tables. Automatically centers the table and handles the tikzpicture environment.
\newenvironment{tkz}
{
\begin{tikzpicture}
}
{
\end{tikzpicture}
}

% More explicitly named commands for the creation of tkz-tab tables.
\newcommand{\tableauinit}[2]{\tkzTabInit{#1}{#2}}
\newcommand{\tableausignes}[1]{\tkzTabLine{#1}}
\newcommand{\tableauvariations}[1]{\tkzTabVar{#1}}

% Shortcut for the curve and the domain of the given function.
\newcommand{\graphe}[1]{\Gamma_{#1}}
\newcommand{\ensembledef}[1]{\mathcal{D}_{#1}}

\renewcommand{\S}[1]{\mathfrak{S}_{#1}}
\newcommand{\frakA}[1]{\mathfrak{A}_{#1}}

\newcommand{\semihrule}{\rule{256.074815pt}{0.4pt}}

% Various environments that create boxes. Each one is one type of thing (example, proof, etc). Each type has its own automatic counter.
\theoremstyle{break}
\theorembodyfont{\upshape}
\theoremheaderfont{\itshape}
\theoremprework{\bigskip\needspace{\baselineskip}\color{green}\hrule\color{black}}
\theorempostwork{\bigskip}
\newtheorem{rem}{Remarque}[chapter]

\theoremstyle{break}
\theorembodyfont{\upshape}
\theoremheaderfont{\itshape}
\theoremprework{\bigskip\needspace{\baselineskip}\color{green}\hrule\color{black}}
\theorempostwork{\bigskip}
\newtheorem{ex}[rem]{Exemple}

\theoremstyle{break}
\theorembodyfont{\upshape}
\theoremheaderfont{\itshape}
\theoremprework{\bigskip\needspace{\baselineskip}\color{green}\hrule\color{black}}
\theorempostwork{\bigskip}
\newtheorem{rappel}[rem]{Rappel}

\theoremstyle{break}
\theorembodyfont{\upshape}
\theoremheaderfont{\itshape}
\theoremprework{\bigskip\needspace{\baselineskip}\color{brown}\hrule\color{black}}
\theorempostwork{\bigskip}
\newtheorem{oubli}[rem]{Oubli}

\theoremstyle{break}
\theorembodyfont{\itshape}
\theoremheaderfont{\normalfont\bfseries}
\theoremprework{\bigskip\needspace{\baselineskip}\color{orange}\hrule\color{black}}
\theorempostwork{\bigskip}
\newtheorem{formu}[rem]{Formule}

\theoremstyle{break}
\theorembodyfont{\upshape}
\theoremheaderfont{\normalfont\bfseries}
\theoremprework{\bigskip\needspace{\baselineskip}\color{blue}\hrule\color{black}}
\theorempostwork{\bigskip}
\newtheorem{defi}[rem]{Définition}

\theoremstyle{break}
\theorembodyfont{\upshape}
\theoremheaderfont{\normalfont\bfseries}
\theoremprework{\bigskip\needspace{\baselineskip}\color{blue}\hrule\color{black}}
\theorempostwork{\bigskip}
\newtheorem{reform}[rem]{Reformulation}

\theoremstyle{break}
\theorembodyfont{\upshape}
\theoremheaderfont{\normalfont\bfseries}
\theoremprework{\bigskip\needspace{\baselineskip}\color{magenta}\hrule\color{black}}
\theorempostwork{\bigskip}
\newtheorem{exo}[rem]{Exercice}

\theoremstyle{break}
\theorembodyfont{\upshape}
\theoremheaderfont{\normalfont\bfseries}
\theoremprework{\bigskip\needspace{\baselineskip}\color{magenta}\semihrule\color{green}\semihrule\color{black}}
\theorempostwork{\bigskip}
\newtheorem{exoex}[rem]{Exercice/Exemple}

\theoremstyle{break}
\theorembodyfont{\upshape}
\theoremheaderfont{\normalfont\bfseries}
\theoremprework{\bigskip\needspace{\baselineskip}\color{blue}\semihrule\color{red}\semihrule\color{black}}
\theorempostwork{\bigskip}
\newtheorem{defprop}[rem]{Définition/Propriétés}

\theoremstyle{break}
\theorembodyfont{\upshape}
\theoremheaderfont{\normalfont\bfseries}
\theoremprework{\bigskip\needspace{\baselineskip}\color{blue}\semihrule\color{red}\semihrule\color{black}}
\theorempostwork{\bigskip}
\newtheorem{deftheo}[rem]{Définition/Théorème}

\theoremstyle{break}
\theorembodyfont{\upshape}
\theoremheaderfont{\normalfont\bfseries}
\theoremprework{\bigskip\needspace{\baselineskip}\color{blue}\hrule\color{black}}
\theorempostwork{\bigskip}
\newtheorem{nota}[rem]{Notation}

\theoremstyle{break}
\theorembodyfont{\upshape}
\theoremheaderfont{\itshape}
\theoremprework{\bigskip\needspace{\baselineskip}\color{blue}\hrule}
\theorempostwork{\hrule\color{black}\needspace{\baselineskip}\bigskip}
\newtheorem*{brouill}{Brouillon}

\theoremstyle{break}
\theorembodyfont{\itshape}
\theoremheaderfont{\normalfont\bfseries}
\theoremprework{\bigskip\needspace{\baselineskip}\color{red}\hrule\color{black}}
\theorempostwork{\bigskip}
\newtheorem{theo}[rem]{Théorème}

\theoremstyle{break}
\theorembodyfont{\upshape}
\theoremheaderfont{\normalfont\bfseries}
\theoremprework{\bigskip\needspace{\baselineskip}\color{red}\hrule\color{black}}
\theorempostwork{\bigskip}
\newtheorem{prop}[rem]{Propriétés}

\theoremstyle{break}
\theorembodyfont{\itshape}
\theoremheaderfont{\normalfont\bfseries}
\theoremprework{\bigskip\needspace{\baselineskip}\color{red}\hrule\color{black}}
\theorempostwork{\bigskip}
\newtheorem{cor}[rem]{Corollaire}

\theoremstyle{break}
\theorembodyfont{\itshape}
\theoremheaderfont{\normalfont\bfseries}
\theoremprework{\bigskip\needspace{\baselineskip}\color{red}\hrule\color{black}}
\theorempostwork{\bigskip}
\newtheorem{lem}[rem]{Lemme}

\theoremstyle{break}
\theorembodyfont{\upshape}
\theoremheaderfont{\normalfont\bfseries}
\theoremprework{\bigskip\needspace{\baselineskip}\color{violet}\hrule\color{black}}
\theorempostwork{\bigskip}
\newtheorem{meth}[rem]{Méthode}

\theoremstyle{break}
\theorembodyfont{\upshape}
\theoremheaderfont{\normalfont\bfseries}
\theoremprework{\bigskip\needspace{\baselineskip}\color{violet}\hrule\color{black}}
\theorempostwork{\bigskip}
\newtheorem{appl}[rem]{Application}

\theoremstyle{break}
\theorembodyfont{\upshape}
\theoremheaderfont{\normalfont\bfseries}
\theoremprework{\bigskip\needspace{\baselineskip}\color{violet}\hrule\color{black}}
\theorempostwork{\bigskip}
\newtheorem{abus}[rem]{Abus}

\theoremstyle{break}
\theorembodyfont{\upshape}
\theoremheaderfont{\normalfont\bfseries}
\theoremprework{\bigskip\needspace{\baselineskip}\color{violet}\hrule\color{black}}
\theorempostwork{\bigskip}
\newtheorem{algo}[rem]{Algorithme}

\theoremstyle{break}
\theorembodyfont{\upshape}
\theoremheaderfont{\normalfont\bfseries}
\theoremprework{\bigskip\needspace{\baselineskip}\color{violet}\hrule\color{black}}
\theorempostwork{\bigskip}
\newtheorem{bilan}[rem]{Bilan}

\theoremstyle{break}
\theorembodyfont{\upshape}
\theoremheaderfont{\itshape}
\theoremprework{\bigskip\needspace{\baselineskip}\color{BurntOrange}\hrule\color{black}}
\theorempostwork{\bigskip}
\newtheorem{corr}[rem]{Correction}

\theoremstyle{break}
\theorembodyfont{\upshape}
\theoremheaderfont{\itshape}
\theoremsymbol{\ensuremath{\cqfd}}
\theoremprework{\bigskip\needspace{\baselineskip}\color{yellow}\hrule\color{black}}
\theorempostwork{\bigskip}
\newtheorem{dem}[rem]{Démonstration}



% Commands to make proofs easier to write
\newcommand{\impdir}{\fbox{\(\imp\)}~}
\newcommand{\imprec}{\fbox{\(\impr\)}~}
\newcommand{\incdir}{\fbox{\(\subset\)}~}
\newcommand{\increc}{\fbox{\(\supset\)}~}
\newcommand{\leqbox}{\fbox{\(\leq\)}~}
\newcommand{\geqbox}{\fbox{\(\geq\)}~}
\newcommand{\unicite}{\fbox{unicité}~}
\newcommand{\existence}{\fbox{existence}~}
\newcommand{\analyse}{\fbox{analyse}~}
\newcommand{\synthese}{\fbox{synthèse}~}
\newcommand{\conclusion}{\fbox{conclusion}~}

\renewcommand{\to}{\longrightarrow}
\renewcommand{\mapsto}{\longmapsto}

\newcommand{\fonction}[5]{\begin{array}[t]{cccc}#1 : & #2 & \to & #3 \\ & #4 & \mapsto & #5\end{array}}
\newcommand{\fonctionlambda}[4]{\begin{array}[t]{ccc}#1 & \to & #2 \\ #3 & \mapsto & #4\end{array}}

\renewcommand{\leq}{\leqslant}
\renewcommand{\geq}{\geqslant}

\newcommand{\pinf}{+\infty}
\newcommand{\minf}{-\infty}

\newcommand{\id}[1]{\mathrm{id}_{#1}}

\renewcommand{\phi}{\varphi}
\renewcommand{\epsilon}{\varepsilon}

\newcommand{\ind}[1]{\mathds{1}_{#1}}

\newcommand{\iR}{\i\R}

\newcommand{\tcheby}[2]{T_{#1}\paren{#2}}
\newcommand{\utcheby}[2]{U_{#1}\paren{#2}}

\mathcode`l="8000
\begingroup
\makeatletter
\lccode`\~=`\l
\DeclareMathSymbol{\lsb@l}{\mathalpha}{letters}{`l}
\lowercase{\gdef~{\ifnum\the\mathgroup=\m@ne \ell \else \lsb@l \fi}}%
\endgroup

\newcommand{\ensvide}{\varnothing}

\newcommand{\rond}{\circ}

\newcommand{\union}{\cup}
\newcommand{\inter}{\cap}
\newcommand{\bigunion}{\bigcup}
\newcommand{\biginter}{\bigcap}

\newcommand{\ssi}{\iff}
\newcommand{\imp}{\implies}
\newcommand{\impr}{\impliedby}

\newcommand{\excluant}{\setminus}

\newcommand{\littletaller}{\mathchoice{\vphantom{\big|}}{}{}{}}
\newcommand{\restr}[2]{{
\left.\kern-\nulldelimiterspace#1\littletaller\right|_{#2}
}}
\newcommand{\corestr}[2]{{
\left.\kern-\nulldelimiterspace#1\littletaller\right|^{#2}
}}
\newcommand{\restrbar}[1]{{
\left.\kern-\nulldelimiterspace#1\littletaller\right|
}}

\newcommand{\rel}{\mathscr{R}}

\newcommand{\classesdequiv}[1]{\nicefrac{#1}{\sim}}

\newcommand{\majo}[1]{\mathrm{majorants}\paren{#1}}
\newcommand{\mino}[1]{\mathrm{minorants}\paren{#1}}

\newcommand{\ensdiv}[1]{\operatorname{div}\paren{#1}}

\newcommand{\E}[1]{\times 10^{#1}}

\setcounter{secnumdepth}{3}

\newcommand{\guillemets}[1]{\og #1 \fg{}}

\newcommand{\prim}{^{\,\prime}}
\newcommand{\seconde}{^{\,\prime\prime}}

\newcommand{\note}[1]{\textbf{\(\star\star\) #1 \(\star\star\)}}
\newcommand{\cad}{c'est-à-dire }
\newcommand{\Cad}{C'est-à-dire }
\newcommand{\ie}{\textit{i.e.} }
\newcommand{\cf}{\textit{cf.} }
\newcommand{\Cf}{\textit{Cf.} }

\usepackage{xparse}

\NewDocumentCommand{\quantifs}{>{\SplitList{;}}m}{\ProcessList{#1}{\insertquantif}}
\newcommand{\insertquantif}[1]{#1,\;\:}

\DeclareDocumentCommand{\groupe}{m O{+}}{\paren{#1,#2}}
\DeclareDocumentCommand{\anneau}{m O{+} O{\times}}{\paren{#1,#2,#3}}
\DeclareDocumentCommand{\corps}{m O{+} O{\times}}{\paren{#1,#2,#3}}

\DeclareDocumentCommand{\poly}{O{\K} O{X}}{#1\croch{#2}}
\DeclareDocumentCommand{\polydeg}{O{\K} m O{X}}{#1_{#2}\croch{#3}}
\DeclareDocumentCommand{\fracrat}{O{\K} O{X}}{#1\paren{#2}}

\DeclareDocumentCommand{\M}{m O{\K}}{\mathscr{M}_{#1}\paren{#2}}
\DeclareDocumentCommand{\sym}{m O{\K}}{\mathscr{S}_{#1}\paren{#2}}
\DeclareDocumentCommand{\antisym}{m O{\K}}{\mathscr{A}_{#1}\paren{#2}}
\DeclareDocumentCommand{\GL}{m O{\K}}{\operatorname{GL}_{#1}\paren{#2}}
\DeclareDocumentCommand{\SL}{m O{\K}}{\operatorname{SL}_{#1}\paren{#2}}
\DeclareDocumentCommand{\Mat}{O{\fami{B}} m}{\operatorname{Mat}_{#1}\paren{#2}}
\newcommand{\pass}[2]{\mathscr{P}_{#1\to#2}}

\DeclareDocumentCommand{\contm}{O{\intervii{a}{b}} O{\K}}{\classe{0}_m\paren{#1,#2}}
\DeclareDocumentCommand{\Esc}{O{\intervii{a}{b}} O{\K}}{\operatorname{Esc}\paren{#1,#2}}

\usepackage{witharrows}

\newcommand{\croix}{^{\times}}

\usepackage{polynom}

\newcommand{\classe}[1]{\mathscr{C}^{#1}}
\newcommand{\ensclasse}[3]{\classe{#1}\paren{#2,#3}}

\newcommand{\deriv}[1]{^{\paren{#1}}}

\usepackage{derivative}
\derivset{\pdv}[delims-eval=.)]

\DeclareMathOperator{\Arctan}{Arctan}
\DeclareMathOperator{\Arcsin}{Arcsin}
\DeclareMathOperator{\Arccos}{Arccos}
\DeclareMathOperator{\cotan}{cotan}
\DeclareMathOperator{\sh}{sh}
\DeclareMathOperator{\ch}{ch}
\DeclareMathOperator{\sg}{sg}
\DeclareMathOperator{\supp}{supp}
\DeclareMathOperator{\Supp}{Supp}
\DeclareMathOperator{\rg}{rg}
\DeclareMathOperator{\tr}{tr}

\newcommand{\Hom}[2]{\operatorname{Hom}\paren{#1,#2}}
\newcommand{\Pol}[2]{\operatorname{Pol}\paren{#1,#2}}
\newcommand{\Aut}[1]{\operatorname{Aut}\paren{#1}}
\DeclareDocumentCommand{\Vect}{O{} m}{\operatorname{Vect}_{#1}\paren{#2}}

\newcommand{\diag}[1]{\operatorname{diag}\paren{#1}}

\usepackage{abstract}
\addto\captionsfrench{\renewcommand{\abstractname}{\Large Introduction}}

\newcommand{\inv}{^{-1}}
\newcommand{\etoile}{^{*}}

\newcounter{orcounter}

\newenvironment{orlist}
{
\begin{array}{|l}
\setcounter{orcounter}{0}
}
{
\end{array}
}

\newcommand{\oritem}[1]{%
\ifthenelse{\theorcounter<1}{}{\\ \text{ou} \\}#1\stepcounter{orcounter}
}

\NewDocumentCommand{\orenv}{>{\SplitList{\\}}m}{%
\begin{orlist}\ProcessList{#1}{\oritem}\end{orlist}}

\newcounter{permuitemcounter}

\newcommand{\permuitem}[1]{%
\ifthenelse{\thepermuitemcounter<1}{}{&}#1\stepcounter{permuitemcounter}}

\NewDocumentCommand{\permu}{>{\SplitList{;}}m >{\SplitList{;}}m}{%
\begin{pmatrix}\setcounter{permuitemcounter}{0}\ProcessList{#1}{\permuitem} \\ \setcounter{permuitemcounter}{0}\ProcessList{#2}{\permuitem}\end{pmatrix}}

\NewDocumentCommand{\cycle}{>{\SplitList{;}}m}{%
\begin{pmatrix}\setcounter{permuitemcounter}{0}\ProcessList{#1}{\permuitem}\end{pmatrix}}

\usepackage{pgfplots}

\DeclareDocumentCommand{\pgcd}{o o}{
\IfNoValueTF{#1}{\operatorname{pgcd}}{\operatorname{pgcd}\paren{#1,#2}}
}

\DeclareDocumentCommand{\bezout}{o o}{
\IfNoValueTF{#1}{\operatorname{bezout}}{\operatorname{bezout}\paren{#1,#2}}
}


\newcommand{\valp}[2]{v_{#1}\paren{#2}}

\newcommand{\fami}[1]{\mathscr{#1}}

\newcommand{\echange}{\leftrightarrow}

\newcommand{\trans}[1]{\prescript{t}{}{#1}}

\usepackage{mathdots}

\DeclareDocumentCommand{\detb}{O{\fami{B}}}{{\det}_{#1}}

\usepackage{cancel}

\usepackage{nicematrix}

\newcommand{\ps}[2]{\left\langle#1\tq#2\right\rangle}
\newcommand{\ortho}{^{\perp}}

\newcommand{\operp}{\mathrel{%
\begin{tikzpicture}[baseline=-0.25em]
\draw (0,0) circle (0.45em);
\draw (-0.38em,-0.25em) -- (0.38em,-0.25em);
\draw (0,-0.25em) -- (0,0.45em);
\end{tikzpicture}
}%
}

\usepackage{titletoc}
\dottedcontents{section}[5.5em]{}{3.2em}{1pc}

\newcommand{\bouleo}[2]{\mathbb{B}\paren{#1,#2}}
\newcommand{\boulef}[2]{\mathbb{B}\prim\paren{#1,#2}}
\newcommand{\sphere}[2]{\mathbb{S}\paren{#1,#2}}

\newcommand{\vdv}[2]{\operatorname{D}_{#1}#2}

\newcommand{\egqd}[1]{\underset{#1}{=}}
\newcommand{\simqd}[1]{\underset{#1}{\sim}}

\newcommand{\mediumrightarrow}{\,\begin{tikzpicture}\draw[->] (0, 0) -- (1, 0);\end{tikzpicture}\,}
\newcommand{\tendqd}[1]{\underset{#1}{\mediumrightarrow}}

\newcommand{\arr}[2]{A_{#2}^{#1}}
\newcommand{\comb}[2]{C_{#2}^{#1}}

\newcommand{\loiuniforme}[1]{\mathscr{U}\paren{#1}}
\newcommand{\loibernoulli}[1]{\mathscr{B}\paren{#1}}
\newcommand{\loibinomiale}[2]{\mathscr{B}\paren{#1,#2}}

\newcommand{\esp}[1]{\operatorname{E}\paren{#1}}
\newcommand{\vari}[1]{\operatorname{V}\paren{#1}}
\newcommand{\cov}[2]{\operatorname{Cov}\paren{#1,#2}}
\newcommand{\ecarttype}[1]{\sigma\paren{#1}}

\renewcommand{\O}[1]{\mathscr{O}\paren{#1}}
\renewcommand{\o}[1]{o\paren{#1}}

\setcounter{MaxMatrixCols}{200}

\newcommand{\Com}[1]{\operatorname{Com}#1}

\usepackage{microtype}

\newcommand{\sig}[1]{\epsilon\paren{#1}}

\ExplSyntaxOn
\RenewDocumentCommand{\v}{m}{
    \int_compare:nTF { \tl_count:n { #1 } > 1 }
    {
        \overrightarrow{#1}
    }
    {
        \vec{#1}
    }
}
\ExplSyntaxOff

\begin{document}
\renewcommand{\labelitemi}{\(\bullet\)}
\renewcommand{\labelenumi}{(\arabic{enumi})}

\everymath{\ds}

\maketitle

\begin{abstract}
	Ce document réunit l'ensemble de mes cours de Mathématiques de MP2I, ainsi que les TDs (travaux dirigés) les accompagnant. J'ai adapté certaines formulations me paraissant floues ou ne me plaisant pas mais le contenu pur des cours est strictement équivalent. Le document est organisé selon la hiérarchie suivante : chapitre, I), 1), a).

	Les éléments des tables des matières initiale et présentes au début de chaque chapitre sont cliquables (amenant directement à la partie cliquée). C'est également le cas des références à des éléments antérieurs de la forme, par exemple, \guillemets{Démonstration 5.22}.

	Dernier TD corrigé : aucun.
\end{abstract}

\dominitoc\tableofcontents

\part{Cours}

\chapter{trigonométrie (Rappels et compléments)}

\minitoc

Dans ce chapitre, on rappelle ce qui a été vu en trigonométrie au lycée et on complète avec les formules
d’addition et de duplication ainsi que l’étude de la fonction tangente.

\section{Cercle trigonométrique}

On se place dans le plan muni d'un repère orthonormé \(\paren{O,\vec{i},\vec{j}}\)

\begin{defi}[Cercle trigonométrique]

	On appelle cercle trigonométrique le cercle de centre \(O\) et de rayon \(1\)

\end{defi}

\begin{prop}[enroulement de la droite des réels sur le cercle trigonométrique]
	Soit \(M\) un point du plan. \\
	Le point \(M\) appartient au cercle trigonométrique si, et seulement si, il existe un réel \(t\) tel que les coordonnées de \(M\) dans le repère orthonormé \(\paren{O,\vec{i},\vec{j}}\) sont \(\paren{\cos t ; \sin t}\)
\end{prop}

\subsection{Relation de congruence modulo \(2\pi\) sur \(\R\)}

\begin{defi}
	Deux réels \(a\) et \(b\) sont dits congrus modulo \(2\pi\) s'il existe un entier relatif \(k\) tel que \(a-b = 2k\pi\)
	\underline{Notation} : \(a \equiv b \croch{2 \pi} \)
\end{defi}

\begin{defprop}
	On dit que la relation \(\equiv\) est une relation d'équivalence sur \(\R\) car elle vérifie les propriétés suivantes :
	\begin{enumerate}
		\item Pour tout réel x, on a : \(x \equiv x \croch{2 \pi}\). \hfill (réfléxivité)
		\item Pour tout couple de réels \(\paren{x,y}\) tel que \( x \equiv y \croch{2 \pi} \), on a :\( y \equiv x \croch{2 \pi} \) \hfill (symétrie)
		\item Pour tout triplet de réels \(\paren{x,y,z}\) tel que \(x \equiv y \croch{2 \pi} \) et \( y \equiv z \croch{2 \pi} \), on a : \( x \equiv z \croch{2 \pi} \) \hfill (transitivité)
	\end{enumerate}
\end{defprop}



\section{Cosinus et sinus}
\subsection{Formules et valeur remarquables}

\begin{formu}[Formule de base]
	Pour tout réel \(t\), on a :
	\begin{enumerate}
		\item \( \cos\paren{\pi - t} = -\cos t \) et \( \sin\paren{\pi - t} = \sin t \) \\
		\item \( \cos\paren{\pi + t} = -\cos t \) et \( \sin\paren{\pi + t} = -\sin t \) \\
		\item \( \cos\paren{\frac{\pi}{2} - t} = \sin t \) et \( \sin\paren{\frac{\pi}{2} - t} = \cos t \) \\
		\item \( \cos\paren{\frac{\pi}{2} + t} = -\sin t \) et \( \sin\paren{\frac{\pi}{2} + t} = \cos t \) \\
	\end{enumerate}
	\begin{tabular}{|c|c|c|c|c|c|}

		\hline
		\(t\)       & \(0\) & \(\frac{\pi}{6}\)      & \(\frac{\pi}{4}\)      & \(\frac{\pi}{3}\)       & \(\frac{\pi}{2}\) \\
		\hline
		\(\cos t \) & \(1\) & \(\frac{\sqrt{3}}{2}\) & \(\frac{\sqrt{2}}{2}\) & \(\frac{1}{2}\)        & \(0\)             \\
		\hline
		\(\sin t \) & \(0\) & \(\frac{1}{2}\)        & \(\frac{\sqrt{2}}{2}\) & \(\frac{\sqrt{3}}{2}\) & \(1\)             \\
		\hline
	\end{tabular}
\end{formu}

\begin{rem}
	Soient \(a\) et \(b\) des réels :
	\begin{itemize}
		\item
		      \(
		      \begin{aligned}
			      \cos a = \cos b
			      \iff\
			       & \left\{
			      \begin{aligned}
				      a & \equiv b \croch{2\pi}  \\
				        & \text{ou}              \\
				      a & \equiv -b \croch{2\pi}
			      \end{aligned}
			      \right.
			      \iff\
			       & \left\{
			      \begin{aligned}
				      \quantifs{\exists k \in \Z}  & a = b +2 k \pi   \\
				                                   & \text{ou}        \\
				      \quantifs{\exists k' \in \Z} & a = -b +2 k' \pi
			      \end{aligned}
			      \right.
		      \end{aligned}
		      \)\\
		      \item\(
		      \begin{aligned}
			      \sin a = \sin b
			      \iff\
			       & \left\{
			      \begin{aligned}
				      a & \equiv b \croch{2\pi}     \\
				        & \text{ou}                 \\
				      a & \equiv \pi-b \croch{2\pi}
			      \end{aligned}
			      \right.
			      \iff\
			       & \left\{
			      \begin{aligned}
				      \quantifs{\exists k \in \Z}  & a = b +2 k \pi      \\
				                                   & \text{ou}           \\
				      \quantifs{\exists k' \in \Z} & a = \pi-b +2 k' \pi
			      \end{aligned}
			      \right.
		      \end{aligned}
		      \)
	\end{itemize}

\end{rem}

\begin{formu} [Formule d'addition]
	Pour tout couple de réels \(\paren{a,b}\) on a :
	\begin{enumerate}
		\item \( \cos\paren{a+b} = \cos\paren{a} \cos\paren{b} - \sin\paren{a} \sin\paren{b} \) \\
		\item \( \cos\paren{a-b} = \cos\paren{a} \cos\paren{b} + \sin\paren{a} \sin\paren{b} \)\\
		\item \( \sin\paren{a+b} = \sin\paren{a} \cos\paren{b} + \cos\paren{a} \sin\paren{b} \) \\
		\item \( \sin\paren{a-b} = \sin\paren{a} \cos\paren{b} - \cos\paren{a} \sin\paren{b} \)\\
	\end{enumerate}
\end{formu}

\begin{formu}[Formule de simpson]
	Pour tout couple de réels \(\paren{a,b}\) on a :
	\begin{enumerate}
		\item \( \sin\paren{a+b} + \sin\paren{a-b} = 2\sin\paren{a} \cos\paren{b} \iff \frac{1}{2}\paren{\sin\paren{a+b} + \sin\paren{a-b}} = \sin\paren{a} \cos\paren{b}\) \\
		\item \( \cos\paren{a+b} + \cos\paren{a-b} = 2\cos\paren{a} \cos\paren{b} \iff \frac{1}{2}\paren{\cos\paren{a+b} + \cos\paren{a-b}} = \cos\paren{a} \cos\paren{b}\)

	\end{enumerate}

\end{formu}

\begin{appl}
	Calcul : \[\int_{0}^{\pi} \sin\paren{x} \cos\paren{3x} dx = \int{0}^{\pi} \frac{1}{2} \paren{\sin\paren{4x}+\sin(2x) dx} = 0\]
\end{appl}

\begin{formu}[Formule de duplication]
	Pour tout réel \(a\), on a :
	\begin{enumerate}
		\item \(\cos\paren{2a} = \cos^2\paren{a} - \sin^2\paren{a} = 2\cos^2(a)-1 = 1-\sin^2(a) \)
		\item \(\sin(2a) = 2\cos(a)\sin(a) \)
	\end{enumerate}
\end{formu}

\begin{prop}[Sinus et Cosinus]
	\begin{itemize}
		\item La fonction \(\cos\) est définie sur \(\R\), paire et périodique de période \(2\pi\). Elle est dérivable sur \(\R\) et sa dérivée vérifie \(\cos' = -\sin\)
		\item La fonction \(\sin\) est définie sur \(\R\), impaire et périodique de période \(2\pi\). Elle est dérivable sur \(\R\) et sa dérivée vérifie \(\sin' = \cos\)
	\end{itemize}
\end{prop}

\begin{prop}[Inégalité remarquable]
	Pour tout réel \(t\), on a : \(\abs{\sin(t)} \leq \abs{t}\)
\end{prop}

\section{La fonction tangente}
\begin{defi}
	La fonction \(\frac{\sin}{\cos} \) est appelée la fonction tangente et notée \(\tan\)
\end{defi}

\begin{prop}
	La fonction \(\tan\) est définie sur \(\R\backslash\accol{\frac{\pi}{2}+k\pi\tq k\in \Z}\), impaire et périodique de période \(\pi\). Elle est dérivable sur \(\R\)\(\R\backslash\accol{\frac{\pi}{2}+k\pi\tq k\in \Z}\) et sa dérivée vérifie \(\tan' = 1+\tan = \frac{1}{tan^2}\)
\end{prop}

\begin{formu}
	Pour tout réel \(t\), on a :
	\begin{enumerate}
        \item \(tan(\pi-t) = -\tan(t)\)
        \item \(tan(\pi+t) = \tan(t) \)
        \item \begin{tabular}{|c|c|c|c|c|c|}

		\hline
		\(t\)       & \(0\) & \(\frac{\pi}{6}\)      & \(\frac{\pi}{4}\)      & \(\frac{\pi}{3}\)       & \(\frac{\pi}{2}\) \\
		\hline
		\(\tan t \) & \(0\) & \(\frac{1}{\sqrt{3}}\) & \(1\) & \(\sqrt{3}\)       & NULL             \\
		\hline
	\end{tabular}
    \end{enumerate}
\end{formu}

\begin{formu}[addition et duplication]
    Pour tout couple de réels \(\paren{a,b}\) n'appartenant pas à l'ensemble \(\accol{\frac{\pi}{2}+k\pi\tq k\in \Z}\), on a :
    \begin{enumerate}
        \item Si \(a+b\) n'appartient pas à l'ensemble \(\accol{\frac{\pi}{2}+k\pi\tq k\in \Z}\) alors \(\tan(a+b) = \frac{\tan(a)+\tan(b)}{1-\tan(a) \tan(b)}\)
        \item Si \(a-b\) n'appartient pas à l'ensemble \(\accol{\frac{\pi}{2}+k\pi\tq k\in \Z}\) alors \(\tan(a-b) = \frac{\tan(a)-\tan(b)}{1+\tan(a) \tan(b)}\)
        \item Si \(2a\)  n'appartient pas à l'ensemble \(\accol{\frac{\pi}{2}+k\pi\tq k\in \Z}\) alors \(\tan(2a) = \frac{2\tan(a)}{1-\tan^2(a)} \)
    \end{enumerate}
\end{formu}

\begin{exoex}
Soit \(t\) réel n'appartenant pas à \(\accol{\frac{\pi}{4}+k\frac{\pi}{2}\tq k\in \Z}\) : 
    \begin{align*}
        \sin(t) &= 2\sin\paren{\frac{t}{2}}\cos\paren{\frac{t}{2}} \\
        &= \frac{2\sin\paren{\frac{t}{2}}}{\cos\paren{\frac{t}{2}}}\cos^2\paren{\frac{t}{2}}\\
        &= \frac{1}{1+\tan^2\paren{\frac{t}{2}}}\times 2 \tan\paren{\frac{t}{2}} \\
        &=\frac{2 \tan\paren{\frac{t}{2}}}{1+\tan^2\paren{\frac{t}{2}}} 
    \end{align*}
\end{exoex}

\chapter{Inégalité et fonction (rappel et compléments)}

\minitoc

Dans ce chapitre, sont rassemblés des rappels ou compléments sur les inégalités ainsi que des fondamentaux sur les fonctions de variable réelle à valeurs réelles (sans preuve ni évocation de continuité).

\section{Inégalité}

\subsection{Relation d'ordre sur \(\R\)}

\begin{defi}
	On dit que la relation \(\leq\) est une relation d'équivalence sur \(\R\) car elle vérifie les propriétés suivantes :
	\begin{enumerate}
		\item Pour tout réel x, on a : \(x \leq x \). \hfill (réfléxivité)
		\item Pour tout couple de réels \(\paren{x,y}\) tel que \( x \leq y  \) et \(y \leq x\), on a :\( y = x  \) \hfill (antisymétrie)
		\item Pour tout triplet de réels \(\paren{x,y,z}\) tel que \(x \leq y  \) et \( y \leq z  \), on a : \( x \leq z  \) \hfill (transitivité)
	\end{enumerate}
\end{defi}

\begin{prop}[Compatibilité avec les opérations]
	Soit \(x,y,z,t\) et \(a\) des réels.
	\begin{enumerate}
		\item Si \(x\leq y\) et \(z\leq t\) alors \(x+z\leq y +t \)
		\item Si \(x\leq y \) et \( 0 \leq a\) alors \(a x \leq a y\)
		\item Si \(x\leq y \) et \( a \leq 0\) alors \(a y \leq a x\)
		\item Si \( 0 \leq x \leq y \) et \( 0\leq z \leq t \) alors \( 0 \leq xz \leq y t \)
	\end{enumerate}
\end{prop}

\begin{nota}[Intervalles de \(\R\)]
	Les partie \(I\) de \(\R\) pouvant s’écrire sous l’une des formes suivantes sont dites intervalles de \(\R\) :
	\begin{itemize}
		\item \(I = \emptyset\) \\
		\item \(I = \accol{x \in \R\tq a \leq x \leq b} \underset{\mathrm{notation}}{=} \intervii{a}{b}\) avec \(\paren{a,b} \in \R^2 \) et \(a\leq b \) \\
		\item \(I = \accol{x \in \R\tq a \leq x < b} \underset{\mathrm{notation}}{=} \intervie{a}{b}\) avec \(\paren{a,b} \in \R\times \paren{\R \union \accol{\pinf}} \) et \(a < b\) \\
		\item \(I = \accol{x \in \R\tq a < x \leq b} \underset{\mathrm{notation}}{=} \intervei{a}{b}\) avec \(\paren{a,b} \in \paren{\R \union \accol{\minf}}\times \R \) et \(a < b\) \\
		\item \(I = \accol{x \in \R\tq a < x \leq b} \underset{\mathrm{notation}}{=} \intervee{a}{b}\) avec \(\paren{a,b} \in \paren{\R \union \accol{\minf}}\times  \paren{\R \union \accol{\pinf}} \) et \(a < b\) \\

	\end{itemize}
\end{nota}

\begin{prop}
	\begin{enumerate}
		\item Passage à l'inverse dans une inégalité
		      \[\quantifs{\forall x \in \Rps ; \forall y \in \Rps} x\leq y \iff \frac{1}{y} \leq \frac{1}{x}\]
		      \[\quantifs{\forall x \in \Rms ; \forall y \in \Rms} x\leq y \iff \frac{1}{y} \leq \frac{1}{x}\] \\
		\item Passage au carré dans une inégalité
		      \[\quantifs{\forall x \in \Rps ; \forall y \in \Rps} x\leq y \iff x^2 \leq y^2\]
		      \[\quantifs{\forall x \in \Rms ; \forall y \in \Rms} x\leq y \iff y^2 \leq x^2\] \\
		\item Passage à la racine carrée dans une inégalité
		      \[\quantifs{\forall x \in \Rp ; \forall y \in \Rp} x\leq y \iff \sqrt{x}\leq \sqrt{y}\] \\
		\item Passage à l’exponentielle ou au logarithme népérien dans une inégalité
		      \[\quantifs{\forall x \in \R ; \forall y \in \R} x\leq y \iff \e{x}\leq \e{y}\]
		      \[\quantifs{\forall x \in \Rps ; \forall y \in \Rps} x\leq y \iff \ln{x}\leq \ln{y}\] \\
	\end{enumerate}
\end{prop}

\begin{exoex}
	Montrer \(\quantifs{\forall x \in \intervii{0}{1}} x(1-x) \leq \frac{1}{4}\).
\end{exoex}

\begin{corr}[2 Méthode]
	Soit \(x \in \intervii{0}{1} \)
	\begin{enumerate}
		\item Raisonnement par équivalence
		      \[\begin{aligned}
				      x(1-x) \leq \frac{1}{4} & \iff 0 \leq \frac{1}{4}-x(1-x)     \\
				                              & \iff 0\leq x^2 -x +  \frac{1}{4}   \\
				                              & \iff 0\leq\paren{x- \frac{1}{2}}^2
			      \end{aligned}
		      \]
		      Ceci étant vrai \(\quantifs{\forall x\in \intervii{0}{1}}\) car \(\Delta = 0\) et \(x_0 =  \frac{1}{2}\), on conclut \(\quantifs{\forall x \in \intervii{0}{1}} x(1-x) \leq \frac{1}{4}\).\\
		\item étude de la fonction \(\fonction{f}{\intervii{0}{1}}{\R}{x}{\frac{1}{4}-x(1-x)}\)\\
	\end{enumerate}
\end{corr}


\begin{exoex}
    ~\\
	Montrer \(\quantifs{\forall x \in \Rps} x+\frac{1}{x}\geq 2\).
\end{exoex}

\begin{corr}
	Soit \(x \in \Rps \)

	\[\begin{aligned}
			x+\frac{1}{x}\geq 2 & \iff \frac{x^2+1}{x}\geq 2 \\
			                    & \iff x^2-2x+1\geq    0     \\
			                    & \iff (x-1)^2 \geq 0
		\end{aligned}
	\]
	Ceci étant vrai \(\quantifs{\forall x\in \Rps}\) on conclut \(\quantifs{\forall x \in \Rps} x+\frac{1}{x}\geq 2\).
\end{corr}

\begin{exoex}
    ~\\
	Encadrer \(\frac{2x^2-x+1}{x^2+\sqrt{x+2}+3}\) pour \(x \in \intervii{-1}{1}\).
\end{exoex}

\begin{corr}
	Soit \(x \in \intervii{-1}{1} \)
	\begin{enumerate}
		\item \underline{numérateur} :
		      \[\begin{aligned}
				      -1 \leq x\leq 1 & \iff 0 \leq x^2 \leq 1      \\
				                      & \iff 0 \leq 2x^2 \leq 2     \\
				                      & \iff 0 \leq 2x^2-x+1 \leq 4
			      \end{aligned}
		      \]

		\item \underline{denominateur} : \[\begin{aligned}
				      -1 \leq x\leq 1 & \iff 0 \leq x^2 \leq 1                                                      \\
				                      & \iff 4 \leq x^2 +\sqrt{x+2}+3 \leq 4+\sqrt{3}                               \\
				                      & \iff \frac{1}{4+\sqrt{3}} \leq \frac{1}{x^2 +\sqrt{x+2}+3 }\leq \frac{1}{4} \\
			      \end{aligned}
		      \]
	\end{enumerate}
	Ainsi par produit des deux inégalités on as \(0\leq\frac{2x^2-x+1}{x^2+\sqrt{x+2}+3}\leq1\) pour \(x \in \intervii{-1}{1}\).
\end{corr}

\begin{exoex}
    ~\\
	Encadrer \(\frac{x-y^2+3}{x^2+y^2-y}\) pour \(\forall \paren{x,y} \in \intervii{1}{2}^2\).
\end{exoex}

\begin{corr}
	Soit \(x \in \intervii{-1}{1} \)
	\begin{enumerate}
		\item \underline{numérateur} :
		      \[\begin{aligned}
				      1-4+3\leq x-y^2+3 \leq 2-1+4 & \iff 0 \leq x-y^2+3 \leq 5
			      \end{aligned}
		      \]

		\item \underline{denominateur} : \[\begin{aligned}
				      0 \leq y-1\leq 1 & \iff 0 \leq y^2-y \leq y                        \\
				                       & \iff 0 \leq y^2-y \leq 2                        \\
				                       & \iff 1 \leq x^2+y^2-y\leq 6                     \\
				                       & \iff \frac{1}{6} \leq \frac{1}{x^2+y^2-y}\leq 1 \\
			      \end{aligned}
		      \]
	\end{enumerate}
	Ainsi par produit des deux inégalités on as \(0\leq \frac{x-y^2+3}{x^2+y^2-y} \leq 5\) pour \(\forall \paren{x,y} \in \intervii{1}{2}^2\).
\end{corr}

\begin{defi}[Parties majorées, majorants, maximum]
	Une partie \(A\) de \(\R\) est dite majorée s’il existe un réel \(M\) tel que, pour tout réel \(x\) de \(A\), on a : \(x \leq M\). \\
	Un tel réel \(M\) est alors dit :
	\begin{itemize}
		\item majorant de \(A\) dans le cas général. \\
		\item maximum de \(A\) dans le cas particulier où \(M\) appartient à \(A\).\\
	\end{itemize}

\end{defi}

\begin{defi}[Parties minorées, minorants, minimum]
	Une partie \(A\) de \(\R\) est dite minorée s’il existe un réel \(m\) tel que, pour tout réel \(x\) de \(A\), on a : \(m\leq x\). \\
	Un tel réel \(m\) est alors dit :
	\begin{itemize}
		\item minorant  de \(A\) dans le cas général. \\
		\item minimum  de \(A\) dans le cas particulier où \(m\) appartient à \(A\).\\
	\end{itemize}

\end{defi}

\begin{exoex}
    ~\\
	Que dire de \(B = \accol{\frac{n}{n^2+1} \tq n \in \N}\) ?
\end{exoex}

\begin{corr}
	\begin{itemize}

		\item \(B\) est minorée car \( \quantifs{\forall n \in \N} 0 \leq \frac{n}{n^2+1} \) par ailleurs \(0 \in B\) donc \(0\) est un minimum. \\
		\item \(B\) est majorée par \(\frac{1}{2}\). En effent en notant \(U_n = \frac{n}{n^2+1}\), On voit que \((U_n)\) est strictement décroissante
	\end{itemize}
\end{corr}

\begin{exoex}
        ~\\
	Que dire de \(C = \accol{\frac{\e{x}}{x} \tq x \in \Rps}\) ?
\end{exoex}

\begin{corr}
	\begin{itemize}

		\item \(C\) est minorée car \( \quantifs{\forall x \in \Rps} 0 \leq \frac{\e{x}}{x} \) donc \(0\) est un minorant mais pas un minimum  \\
		\item Supposons que \(C\) est majorée alors \(\quantifs{\exists M \in \R;\forall c \in C} c\leq M \) ainsi \(\quantifs{\forall x \in \Rps} \frac{\e{x}}{x} \leq M \) donc par passage à la limite en \(\pinf\) on trouve \(\pinf \leq M\) ce qui est absurde donc \(C\) n'est pas majorée.
	\end{itemize}
\end{corr}

\begin{defi}[Parties bornées]
	Une partie \(A\) de \(\R\) est dite bornée si elle est majorée et minorée autrement dit s’il existe deux réels \(m\) et \(M\) tel que, pour tout réel \(x\) de \(A\), on a : \(m\leq x \leq M\).
\end{defi}

\section{Valeur absolue d'un réel}
\begin{defi}
	Pour tout \(x\) réel, la valeur absolue de \(x\), notée \(\abs{x}\), est définie par : \(\abs{x} = \begin{cases}
		-x & \text{si }  x < 0   \\
		x  & \text{si }  x\geq 0 \\
	\end{cases}\)
\end{defi}

\begin{prop}
	\begin{enumerate}
		\item Pour tout \(x\) réel, on a : \(0\leq\abs{x}\) et \(x\leq\abs{x}\)
		\item Pour tout couple\((x,y)\) de réels, on a : \(\abs{xy} = \abs{x}\abs{y}\)
		\item Pour tout couple \((x,y)\) de réels tel que \(y\) est non nul, on a: \(\abs{\frac{x}{y}} = \frac{\abs{x}}{\abs{y}}\)
	\end{enumerate}
\end{prop}

\begin{defprop}[Deux inéquations élémentaires]
	Pour tout réel \(x\) et tout \underline{réel positif} \(\alpha\), on a:
	\begin{enumerate}
		\item \(\abs{x}\leq \alpha \iff -\alpha \leq x \leq \alpha \iff x \in \intervii{-\alpha}{\alpha}\)
		\item \(\abs{x}\geq \alpha \iff x \leq -\alpha\text{ ou } \alpha \leq x \iff x \in \intervei{\pinf}{-\alpha}\union\intervie{\alpha}{\pinf}\)
	\end{enumerate}
\end{defprop}

\begin{defprop}[Interprétation sur la droite des réels]
	Soit \(a\) un réel et \(b\) un \underline{réel positif}. \\
	L’ensemble des réels \(x\) vérifiant \(\abs{x-a}\leq b\) (resp. \(\abs{ x-a}\geq b \)) est l’ensemble des points de la droite des
	réels situés à une distance du point \(a\) inférieure ou égale (resp. supérieure ou égale) à \(b\).
\end{defprop}

\begin{prop}[Inégalité triangulaire]
	Pour tout couple \((x,y)\) de réels, on a :
	\[\abs{x+y}\leq \abs{x}+\abs{y}\]
\end{prop}

\begin{dem} [inégalité triangulaire]
	Soit \((x,y) \in \R^2\)
	\begin{align*}
		\abs{x+y}\leq \abs{x}+\abs{y} & \iff \abs{x+y}^2\leq (\abs{x}+\abs{y})^2      \\
		                              & \iff x^2+2xy+y^2 \leq x^2+y^2+2\abs{x}\abs{y} \\
		                              & \iff xy\leq \abs{xy}
	\end{align*}
	Ce qui est vrai donc l'inégalité est bien démontrer
\end{dem}

\begin{exoex}   
     ~\\
	Encadrer \(\frac{x\cos(x)+1}{\sin(x)+3}\) pour \(x\in\intervii{-\pi}{2\pi}\)
\end{exoex}

\begin{corr}
	Soit \(x\in\intervii{-\pi}{2\pi}\)
	\begin{itemize}

		\item \underline{numérateur} : \(\abs{x\cos(x)+1}\leq \abs{x}\abs{\cos(x)}+1\leq 2\pi+1 = 2\pi+1\)
		\item \underline{dénominateur} : \(2\leq\abs{\sin(x)+3}\leq 4\)

	\end{itemize}
	Ainsi par produit des deux inégalités on as :\(0\leq\frac{\abs{x\cos(x)+1}}{\abs{\sin(x)+3}}\leq \frac{2\pi+1}{2} \)\\
	donc \(-\frac{2\pi+1}{2} \leq \frac{x\cos(x)+1}{\sin(x)+3} \leq \frac{2\pi+1}{2}\) pour \(x\in\intervii{-\pi}{2\pi}\).
\end{corr}

\begin{prop}
	Soit un couple \((x,y)\) de réels.
	\[\abs{\abs{x}-\abs{y}} \leq\abs{x-y}\]
\end{prop}

\begin{dem}
	Soit \((x,y) \in \R^2\)
	\(x =(x-y)+y\) donc \(\abs{x} \underset{\mathrm{\text{inég. triang.}}}{\leq} \abs{x-y}+\abs{y}\) d'où \(\abs{x} - \abs{y} \leq \abs{x-y}\) \\
	De même, \(y =(x-y)+x\) donc \(\abs{y} \underset{\mathrm{\text{inég. triang.}}}{\leq} \abs{x-y}+\abs{x}\) d'où \( -\abs{x-y} \leq\abs{x} - \abs{y}\)\\

	ainsi on a \(-\abs{x-y} \leq\abs{x} - \abs{y} \leq \abs{x-y}\) donc \(\abs{\abs{x}-\abs{y}} \leq\abs{x-y}\).
\end{dem}

\section{Partie entière d'un réel}
\begin{prop}
	Pour tout réel \(x\),il existe un unique entier \(n\) tel que :
	\[n\leq x < n+1\]
\end{prop}
\begin{defi}
	On appelle partie entière de \(x\), notée \(\lfloor x \rfloor\), l'unique entier \(n\) vérifiant la propriété précédente.
\end{defi}

\begin{ex}
	\(\lfloor 3.14 \rfloor = 3\), \(\lfloor -2.7 \rfloor = -3\) et \(\lfloor 5 \rfloor = 5\).
\end{ex}

\section{Généralité sur les fonctions}
\begin{defi} [Fonction]
	Une fonction de variable réelle à valeurs réelles notée \(f\) est un objet mathématique qui, à tout élément \(x\) d’une partie non vide de \(\R\), associe un et un seul nombre réel noté \(f(x)\). \\
	\underline{Notation Fonctionnelle} : \[\fonction{f}{A}{\R}{x}{f(x)}\]
\end{defi}

\begin{defi}
	Soit \(f\) une fonction de variable réelle à valeurs réelles.
	\begin{enumerate}
		\item L’ensemble des réels \(x\) pour lesquels \(f(x)\) existe est appelé ensemble/domaine de définition de \(f\) et souvent noté \(D_f = \accol{x \in \R \tq f(x) \text{existe}}\)
		\item Soit \(x \in D_f\)\\
		      La valeur réelle \(f(x)\) est appelée image de \(x\) par \(f\). \\
		\item soit \( y \in \R\) \\
		      S'il existe \(x\) dans \(D_f\) tel que \(f(x) = y\) alors \(x\) est dit antécédent de \(y\) par \(f\)
	\end{enumerate}
\end{defi}

\begin{defprop}[égalité entre fonction]
	Deux fonctions \(f\) et \(g\) de variable réelle à valeurs réelles sont dites égales si les deux conditions suivantes sont réunies :
	\begin{itemize}
		\item les fonctions \(f\) et \(g\) ont le même ensemble de définition \(D\) ;
		\item pour tout \(x\) de \(D\), \(f(x) = g(x)\).
	\end{itemize}
	dans ce cas, on note \(f = g\).
\end{defprop}

\begin{exoex}
	est-ce que les fonctions \(f\) et \(g\) définies par :
	\[f: x\mapsto\frac{1}{\sqrt{1+x}+1} \text{ et } g:  x\mapsto\frac{\sqrt{1+x}-1}{x}\]
	Sont égales ?
\end{exoex}
\begin{corr}
	Tout d'abord \(\quantifs{\forall x \in D_f\inter D_g} f(x) = g(x)\) car :
	\begin{align*} g(x) & = \frac{\sqrt{1+x}-1}{x}                                                 \\
                    & = \frac{\paren{\sqrt{1+x}-1}\paren{\sqrt{1+x}+1}}{x\paren{\sqrt{1+x}+1}} \\
                    & = \frac{1+x-1}{x\paren{\sqrt{1+x}+1}}                                    \\
                    & = \frac{x}{x\paren{\sqrt{1+x}+1}}                                        \\
                    & = \frac{1}{\sqrt{1+x}+1} = f(x)
	\end{align*}
	Donc \(f = g\) sur \(D_f\inter D_g\) mais
	\(D_f = \intervei{-1}{\pinf}\) or \(D_g = \intervie{-1}{\pinf}\pd\accol{0}\) donc \(D_f \neq D_g\) donc \(f \neq g\).
\end{corr}

\begin{defi}[représentation graphique d'une fonction]
	Dans le plan muni d’un repère orthonormé \((O, \vec{i}, \vec{j})\), l’ensemble de points \(\mathcal{C}_f\) défini par
	\[
		\mathcal{C}_f = \accol{ M(x; f(x)) \tq x \in D_f }
	\]
	est appelé représentation graphique de \(f\) (ou courbe représentative de \(f\)).
\end{defi}

\begin{defi}[Parité,imparité et périodicité d'une fonction]
	\begin{itemize}
		\item Une fonction \(f\) est dite paire si, pour tout \(x\) de son domaine de définition, on a : \(f(-x) = f(x)\).
		\item Une fonction \(f\) est dite impaire si, pour tout \(x\) de son domaine de définition, on a : \(f(-x) = -f(x)\).
		\item Une fonction \(f\) est dite périodique de période \(T\) si, pour tout \(x\) de son domaine de définition, on a : \(f(x+T) = f(x)\).
	\end{itemize}
\end{defi}
\begin{exo}
	Montrer que toute fonction de \(\R\) peut s'écrire de manière unique comme la somme d'une fonction paire et d'une fonction impaire.
\end{exo}

\begin{corr}[Analyse-synthèse]
	Soit \(f : \R \mapsto \R \) une fonction quelqu'onque
	\begin{itemize}
		\item \analyse :  Supposons qu'il existe \(\begin{cases}
			      p:\R \mapsto \R \text{ paire} \\
			      i:\R \mapsto \R \text{ impaire}
		      \end{cases}\) telles que \(f = p + i\) \\
		      Ainsi \(\forall x \in \R \begin{cases}
			      f(x) = p(x) + i(x) \hfill (1)                  \\
			      f(-x) = p(-x) + i(-x) = p(x) - i(x) \hfill (2) \\
		      \end{cases}\) \\
		      \begin{itemize}
			      \item\(\frac{1}{2}\paren{\text{(1)+(2)}}\) donne \(p:x\mapsto \frac{f(x)+f(-x)}{2}\) \\
			      \item \(\frac{1}{2}\paren{\text{(1)-(2)}}\) donne \(i:x\mapsto \frac{f(x)-f(-x)}{2}\) \\
		      \end{itemize}
		\item \synthese : vérifions que le seul couple trouvé convient :
		      \begin{itemize}
			      \item \(\forall x \in \R, f(x) = p(x)+i(x)\)\\
			      \item \(p(-x) = p(x) \text{ et } i(-x) = -i(x)\)\\
		      \end{itemize}
	\end{itemize}
	Ainsi \(f\) s'écrit de manière unique comme la somme d'une fonction paire et impaire
\end{corr}

\begin{defi} [opération et composition]
	Soit \(f\) et \(g\) deux fonctions de variable réelle à valeurs réelles de domaines de définition \(D_f\) et \(D_g\).
	\begin{itemize}
		\item La somme de \(f\) et \(g\) est la fonction, notée \(f + g\), définie par \(f + g : x \mapsto f(x) + g(x)\). \\
		      Son domaine de définition \(D_{f+g}\) vérifie : \(D_{f+g} = D_f \inter D_g\).
		\item  La multiplication de \(f\) par le réel \(\alpha\) est la fonction, notée \(\alpha f\), définie par \(\alpha f : x \mapsto \alpha f(x)\). \\
		      Son domaine de définition \(D_{\alpha f}\) vérifie : \(D_{\alpha f} = D_f\) si \(\alpha \neq 0\).
		\item Le produit de \(f\) et \(g\) est la fonction, notée \(f g\), définie par \(f g : x \mapsto f(x)g(x)\). \\
		      Son domaine de définition \(D_{fg}\) vérifie : \(D_{fg} = D_f \inter D_g\).
		\item Le quotient de \(f\) par \(g\) est la fonction , notée \(frac{f}{g}\), définie par \(frac{f}{g} : x \mapsto \frac{f(x)}{g(x)}\). \\
		      Son domaine de définition \(D_{frac{f}{g}}\) vérifie : \(D_{frac{f}{g}} = D_f \inter \accol{x \in D_g | g(x) \neq 0}\).
		\item La composée de \(g\) et \(f\) est la fonction, notée \(g \circ f\), définie par \(g \circ f : x \mapsto g(f(x))\). \\
		      Son domaine de définition \(D_{g \circ f}\) vérifie : \(D_{g \circ f} = \accol{x \in D_f | f(x) \in D_g}\).
	\end{itemize}
\end{defi}

\begin{exoex}
	Domaine de définition de : \(\fonction{f}{D_f}{\R}{x}{\sqrt{x-\frac{1}{x}}} \)
\end{exoex}

\begin{corr}
	Soit \(x \in D_f\) alors \(x-\frac{1}{x} \geq 0 \iff x\neq 0\) et \(\frac{x^2-1}{x} = \frac{(x-1)(x+1)}{x} \geq 0\)\\
	% Assurez-vous d'avoir \usepackage{tkz-tab} dans le préambule
	\begin{tikzpicture}
		\tkzTabInit{$x$/1,$(x-1)(x+1)$/1,$x$/1,$f$/1}{$-\infty$,$-1$,$0$,$1$,$+\infty$}
		\tkzTabLine{,+,0,-,t,-,0,+,}
		\tkzTabLine{,-,t,-,0,+,t,+,}
		\tkzTabLine{,-,0,+,d,-,0,+}


	\end{tikzpicture}\\
	ainsi on voit bien que \(D_f = \intervie{-1}{0}\union\intervie{1}{\pinf}\)
\end{corr}

\section{Fonction et relation d'ordre}
\begin{defi}[Monotonie]
	Soit \(f\) une fonction de variable réelle à valeurs réelles et \(D\) une partie de son domaine de définition \(D_f\).
	\begin{enumerate}
		\item \(f\) est dite \textbf{croissante} sur \(D\) si, pour tout \((x, y) \in D^2\) tel que \(x \leq y\), on a \(f(x) \leq f(y)\).
		\item \(f\) est dite \textbf{décroissante} sur \(D\) si, pour tout \((x, y) \in D^2\) tel que \(x \leq y\), on a \(f(x) \geq f(y)\).
		\item \(f\) est dite \textbf{strictement croissante} sur \(D\) si, pour tout \((x, y) \in D^2\) tel que \(x < y\), on a \(f(x) < f(y)\).
		\item \(f\) est dite \textbf{strictement décroissante} sur \(D\) si, pour tout \((x, y) \in D^2\) tel que \(x < y\), on a \(f(x) > f(y)\).
	\end{enumerate}
	\textbf{Remarque :} \(f\) est dite \textbf{monotone} (resp. \textbf{strictement monotone}) sur \(D\) si elle est croissante ou décroissante (resp. strictement croissante ou strictement décroissante) sur \(D\).
\end{defi}

\begin{rem}[Application de la définition]
	Sous réserve que cela ait du sens :
	\begin{itemize}
		\item La somme de deux fonctions croissantes(resp. décroissantes) est croissante(resp. décroissante).
		\item La composée de deux fonctions croissantes(resp. décroissantes) est croissante(resp. décroissante).
		\item La composée d'une fonction croissante et d'une fonction décroissante est décroissante
		\item Le produit de deux fonctions \underline{positives} croissantes (resp. décroissantes) est croissante(resp. décroissante).
	\end{itemize}
\end{rem}


\begin{defi}
	Soit \(f\) une fonction de variable réelle à valeurs réelles de domaine de définition \(D_f\). \\
	Soit \(D\) une partie non vide de \(D_f\).
	\begin{enumerate}
		\item \(f\) est dite \textbf{majorée} sur \(D\) si l'ensemble \(\accol{f(x) \tq x \in D}\) est majoré, c'est-à-dire s'il existe un réel \(M\) tel que, pour tout réel \(x\) de \(D\), on a : \(f(x) \leq M\).\\
		      Un tel réel \(M\) est alors dit :
		      \begin{itemize}
			      \item \textbf{majorant} de \(f\) sur \(D\) dans le cas général.
			      \item \textbf{maximum} de \(f\) sur \(D\) dans le cas particulier où il existe \(x_0\) dans \(D\) tel que \(M = f(x_0)\).
		      \end{itemize}
		\item \(f\) est dite \textbf{minoriée} sur \(D\) si l'ensemble \(\accol{f(x) \tq x \in D}\) est minoré, c'est-à-dire s'il existe un réel \(m\) tel que, pour tout réel \(x\) de \(D\), on a : \(m \leq f(x)\).\\
		      Un tel réel \(m\) est alors dit :
		      \begin{itemize}
			      \item \textbf{minorant} de \(f\) sur \(D\) dans le cas général.
			      \item \textbf{minimum} de \(f\) sur \(D\) dans le cas particulier où il existe \(x_0\) dans \(D\) tel que \(m = f(x_0)\).
		      \end{itemize}
		\item \(f\) est dite \textbf{bornée} sur \(D\) si \(f\) est majorée et minoriée sur \(D\), c'est-à-dire s'il existe deux réels \(m\) et \(M\) tels que, pour tout réel \(x\) de \(D\), on a : \(m \leq f(x) \leq M\).\end{enumerate}
\end{defi}

\begin{prop}
	Soit \(f\) une fonction de variable réelle à valeurs réelles de domaine de définition \(D_f\). \\
	Alors \(f\) est bornée sur \(D\) si, et seulement si, la fonction \(\abs{f}\) est majorée sur \(D\).
\end{prop}
\section{Dérivation des fonctions d'une variable réelle}

\begin{defi}[dérivée en un point]
	Soit \(f\) une fonction de variable réelle à valeurs réelles de domaine de définition \(D_f\) et \(x_0\) un point de \(D_f\). \\
	\(f\) est dite dérivable en \(x_0\) si la fonction \(x\mapsto \frac{f(x)-f(x_0)}{x-x_0}\) admet une limite finie en \(x_0\). \\
	Dans ce cas, on note \(f'(x_0)\) la valeur de cette limite et on l'appelle la dérivée de \(f\) en \(x_0\). \\
	Cela reient à déterminer si la fonction \(h\mapsto \frac{f(x_0+h)-f(x_0)}{h}\) admet une limite finie en \(0\).\\
\end{defi}

\begin{defi}{fonction dérivée}
	\(f\) est dite dérivable sur \(D_f\) si elle est dérivable en tout point de \(D_f\). \\
	Dans ce cas, la fonction \(x\mapsto f'(x)\) est appelée fonction dérivée de \(f\) et notée \(f'\). \\
\end{defi}

\begin{defprop}[équation de la tangente]
	On se place dans le plan muni d’un repère orthonormé \((O, \vec{i}, \vec{j})\). \\
	Soit \(f\) une fonction de variable réelle à valeurs réelles et \(C_f\) la courbe représentative de \(f\). \\
	Soit \(x_0\) un point de \(D_f\) .\\
	Si \(f\) est dérivable en \(x_0\), alors la tangente à la courbe \(C_f\) au point \(M(x_0, f(x_0))\) est la droite d’équation :
	\[y = f'(x_0)(x-x_0) + f(x_0)\]
\end{defprop}

\begin{defprop}[opération sur les fonctions dérivable]
	Soit \(I\) et \(J\) des intervalles de \(\R\) non vide et non réduits à un point. \\
	\begin{enumerate}
		\item \underline{Combinaison linéaire} : \\
		      Soit \(f\) et \(g\) deux fonctions définies sur \(I\) et à valeurs réelles et \((\alpha, \beta)\) deux réels. \\
		      Si \(f\) et \(g\) sont dérivables sur \(I\), alors \(\alpha f + \beta g\) est dérivable sur \(I\) et sa dérivée vérifie :
		      \[\alpha f + \beta g' = \alpha f' + \beta g'\]
		\item \underline{Produit} : \\
		      Soit \(f\) et \(g\) deux fonctions définies sur \(I\) et à valeurs réelles. \\
		      Si \(f\) et \(g\) sont dérivables sur \(I\), alors \(f g\) est dérivable sur \(I\) et sa dérivée vérifie :
		      \[(f g)' =f'g+fg'\]
		      \item\underline{quotient} :\\
		      Soit \(f\) et \(g\) deux fonctions définies sur \(I\) et à valeurs réelles tel que \(g\) est non nulle sur \(I\). \\
		      Si \(f\) et \(g\) sont dérivables sur \(I\), alors \(\frac{f}{g}\) est dérivable et sa dérivée vérifie :
		      \[\paren{\frac{f}{g}}' = \frac{f'g-fg'}{g^2}\]
		\item \underline{Composition} : \\
		      Soit \(f\) une fonction définie sur \(I\) et à valeurs réelle tel que, pour tout \(x\) de \(I\), \(f(x)\) appartient à \(J\)\\
		      Soit \(g\) une fonction définie sur \(J\) et à valeurs réelles. \\
		      Si \(f\) est dérivable sur \(I\) et \(g\) dérivable sur \(J\), alors la composée \(g \circ f\) est dérivable sur \(I\) et sa dérivée vérifie :
		      \[\paren{g \circ f}' = g' \circ f \times f'\]
	\end{enumerate}
\end{defprop}

\begin{defprop}[Caractérisation des fonctions constantes ou monotones]
	Soit \(f\) une fonction définie sur un intervalle \(I\) et à valeurs réelles. \\
	\begin{enumerate}
		\item \(f\) est constante sur \(I\) si, et seulement si, pour tout \(x\) de \(I\),\(f'(x)=0\).
		\item \(f\) est croissante sur \(I\) si, et seulement si, pour tout \(x\) de \(I\), \(f'(x) \geq 0\).
		\item \(f\) est décroissante sur \(I\) si, et seulement si, pour tout \(x\) de \(I\), \(f'(x) \leq 0\).
		\item \(f\) est strictement croissante sur \(I\) si, et seulement si, les deux conditions suivante sont réunies :
		      \begin{enumerate}
			      \item pour tout \(x\) de \(I\), \(f'(x) \geq 0\) ;
			      \item il n'existe pas de réels \(a\) et \(b\) dans \(I\) avec \(a < b\) tels que pour tout \(x\) de \(\intervii{a}{b}\), on a \(f'(x) = 0\).
		      \end{enumerate}
		\item \(f\) est strictement décroissante sur \(I\) si, et seulement si, les deux conditions suivante sont réunies :
		      \begin{enumerate}
			      \item pour tout \(x\) de \(I\), \(f'(x) \leq 0\) ;
			      \item il n'existe pas de réels \(a\) et \(b\) dans \(I\) avec \(a < b\) tels que pour tout \(x\) de \(\intervii{a}{b}\), on a \(f'(x) = 0\).
		      \end{enumerate}
	\end{enumerate}
\end{defprop}

\begin{defprop}[dérivées usuelles]
    ~\\
	\begin{tabular}{|c|c|c|}

		\hline
		\textbf{Fonction}                    & \textbf{Domaine de dérivabilitée}                 & \textbf{Fonction dérivée}                                             \\
		\hline
		\(x\mapsto a\) avec \(a \in \R\)     & \(\R\)                                            & \(x\mapsto 0\)                                                        \\
		\hline
		\(x\mapsto x^n\) avec \(n \in \Ns\)  & \(\R\)                                            & \(x\mapsto nx^{n-1}\)                                                 \\
		\hline
		\(x\mapsto x^-n\) avec \(n \in \Ns\) & \(\Rs\)                                           & \(x\mapsto -nx^{-n-1}\)                                               \\
		\hline
		\(x\mapsto \sqrt{x}\)                & \(\Rps\)                                          & \(x\mapsto \frac{1}{2\sqrt{x}}\)                                      \\
		\hline
		\(x\mapsto \e{x}\)                   & \(\R\)                                            & \(x\mapsto \e{x}\)                                                    \\
		\hline
		\(x\mapsto \ln(x)\)                  & \(\Rps\)                                          & \(x\mapsto \frac{1}{x}\)                                              \\
		\hline
		\(x\mapsto \sin(x)\)                 & \(\R\)                                            & \(x\mapsto \cos(x)\)                                                  \\
		\hline
		\(x\mapsto \cos(x)\)                 & \(\R\)                                            & \(x\mapsto -\sin(x)\)                                                 \\
		\hline
		\(x\mapsto \tan(x)\)                 & \(\R\pd\accol{\frac{\pi}{2}+2k\pi \tq k \in \Z}\) & \(x\mapsto \frac{1}{\cos^2(x)} \) ou \(x\mapsto \frac{1}{\cos^2(x)}\) \\
		\hline
	\end{tabular}



\end{defprop}

\begin{exoex}
    ~\\
	Calculer\(\int_{\frac{\pi}{4}}^{\frac{\pi}{3}} \frac{\sin^3(x)}{\cos^5(x)} dx\)
\end{exoex}

\begin{corr}
	\begin{align*}
		\int_{\frac{\pi}{4}}^{\frac{\pi}{3}} \frac{\sin^3(x)}{\cos^5(x)} dx & = \int_{\frac{\pi}{4}}^{\frac{\pi}{3}} \tan^3(x) \times \frac{1}{\cos^2(x)} dx \\
		                                                                    & = \int_{\frac{\pi}{4}}^{\frac{\pi}{3}} \tan^3(x) \times \paren{\tan^2(x)+1} dx \\
		                                                                    & = \croch{\frac{1}{4}\paren{\tan^4(x) }}_{\frac{\pi}{4}}^{\frac{\pi}{3}}        \\
		                                                                    & = \frac{1}{4}\paren{\tan^4\paren{\frac{\pi}{3}} - \tan^4\paren{\frac{\pi}{4}}} \\
		                                                                    & = \frac{1}{4}\paren{\paren{\sqrt{3}}^4 - 1^4}                                  \\
		                                                                    & = 2                                                                            \\
	\end{align*}
\end{corr}

\begin{defprop}[étude pratique d'une fonction]
	Le plan d'étude d'une fonction \(f\) est en général le suivant:
	\begin{itemize}
		\item Détermination du domaine de définition de \(f\)
		\item Réduction éventuelles du domaine d'étude selon les propriétés de \(f\) (parité, périodicité, etc.)
		\item Limites aux bornes du domaine d'étude
		\item Etude de la monotonie (le plus souvent,mais pas uniquement, après calcul de la dérivée de \(f\) et détermination du signe de celle-ci )
		\item Construction du tableau de variation de \(f\)(limites aux bornes, valeurs remarquables, variations)
		\item Tracé de la courbe représentative de \(f\)
	\end{itemize}
\end{defprop}
\begin{defprop}[dérivées d'odre supériéur]
	Soit \(f\) une fonction définie sur un intervalle \(I\) et à valeurs réelles. \\
	On note
	\[f^{(0)} = f\]
	puis, pour tout entier naturel \(k\) tel que la fonction\(f^{(k)}\) existe et est déribable sur \(I\), on pose :
	\[f^{(k+1)} = \paren{f^{(k)}}'\]
	Si \(n\) est un entier naturel, tel que la fonction \(f^{(n)}\) existe alors on dit que \(f\) est \(n\)-fois dérivable sur \(I\) et que \(f^{(n)}\) est la dérivée d'ordre \(n\) (ou dérivée \(n\)-ième) de \(f\).\\

\end{defprop}
\begin{defi}[Fonction réciproque]
	Soit \(f\) une fonction définie sur un intervalle \(I\) à valeurs dans \(J\)
	Si, pour tout y de \(J\), l’équation \(y = f(x)\) admet une unique solution \(x\) dans \(I\) notée \(x = f^{-1}(y)\) alors :
	\begin{itemize}
		\item la fonction \(f\) est dite bijection de \(I\) sur \(J\)
		\item la fonction \(f^{-1}\) ainsi définie sur \(J\) et à valeurs dans \(I\), est dite bijection réciproque de \(f\).
	\end{itemize}
	\underline{Exemples}:
	\begin{itemize}
		\item \(\sqrt{}\) est une bijection de \(\Rp\) sur \(\Rp\) de bijection réciproque \(f : \Rp \to \Rp\) définie par \(f(x) = x^2\).
		\item \(\exp\) est une bijection de \(\R\) sur \(\Rps\) de bijection réciproque la fonction \(\ln\)
	\end{itemize}
\end{defi}
\begin{prop}[Propriétés de la bijection réciproque]
	Si \(f\) est une bijection de \(I\) sur \(J\) de bijection réciproque notée \(f^{-1}\) alors on a :
	\begin{enumerate}
		\item pour tout \(x\) de \(I\), \(f(f^{-1}(x)) = x\) ;
		\item pour tout \(y\) de \(J\), \(f^{-1}(f(y)) = y\).
	\end{enumerate}

\end{prop}

\begin{defprop}[représentation graphique]
	on se place dans le plan muni d’un repère orthonormé \((O, \vec{i}, \vec{j})\). \\
	Si \(f\) est une bijection de \(I\) sur \(J\) alors la courbe représentative de \(f\) et de sa bijection réciproque \(f^{-1}\) sont symétriques par rapport à la droite d’équation \(y = x\).
\end{defprop}

\begin{defprop}[dérivée de la bijection réciproque]
	Soit \(f\) une bijection de \(I\) sur \(J\)  et si \(f\) est dérivable sur \(I\) alors sa bijection réciproque \(f^{-1}\) est dérivable en tout point y de \(J\) tel que \(f'(f^{-1}(y)) \neq 0\) avec, dasn ce cas : \[(f^{-1})'(y) = \frac{1}{f'(f^{-1}(y))}\]
\end{defprop}
\begin{dem}
	Soit \(f\) une bijection de \(I\) sur \(J\), soit \(y\) in \(J\) tel que \(f'(f^{-1}(y)) \neq 0\). \\
	on sait que \(f(f^{-1}(y)) = y\) donc en appliquant la définition de la dérivée de fonction composée on a :
	\[(f(f^{-1}(y)))' = (y)' \iff f'(f^{-1}(y))\times (f^{-1}(y))' = 1 \iff (f^{-1}(y))' = \frac{1}{f'(f^{-1}(y))}\]
\end{dem}
\begin{defprop}[Trois fonction usuelles trigonométriques]
	\begin{itemize}
		\item \underline{Fonction \(\Arccos\)} : \\
		      La fonction \(\Arccos\) est la réciproque de la fonction \(\fonction{c}{\intervii{0}{\pi}}{\intervii{-1}{1}}{x}{\cos(x)}\) et est donc définie sur \(\intervii{-1}{1}\) à valeurs dans \(\intervii{0}{\pi}\) et dérivable sur \(\intervee{-1}{1}\) de dérivée: \[\arccos':x\mapsto \frac{-1}{\sqrt{1-x^2}}\]\\
		\item \underline{Fonction \(\Arcsin\)} : \\
		      La fonction \(\Arccos\) est la réciproque de la fonction \(\fonctionlambda{\intervii{-\frac{\pi}{2}}{\frac{\pi}{2}}}{\intervii{-1}{1}}{x}{\sin(x)}\) et est donc définie sur \(\intervii{-1}{1}\) à valeurs dans \(\intervii{-\frac{\pi}{2}}{\frac{\pi}{2}}\) et dérivable sur \(\intervee{-1}{1}\) de dérivée: \[\arcsin':x\mapsto \frac{1}{\sqrt{1-x^2}}\]\\
		\item \underline{Fonction \(\Arctan\)} : \\
		      La fonction \(\Arccos\) est la réciproque de la fonction \(\fonctionlambda{\intervee{-\frac{\pi}{2}}{\frac{\pi}{2}}}{\R}{x}{\tan(x)}\) et est donc définie sur \(\R\) à valeurs dans \(\intervee{-\frac{\pi}{2}}{\frac{\pi}{2}}\) et dérivable sur \(\R\) de dérivée: \[\arctan':x\mapsto \frac{1}{1+x^2}\]\\
	\end{itemize}
\end{defprop}


\begin{dem}[démonstration de la dérivée de la fonction \(\Arccos\)]
	Soit \(y\in\intervii{-1}{1}\), on note \(\fonction{c}{\intervii{0}{\pi}}{\intervii{-1}{1}}{x}{\cos(x)}\)
	\begin{align*}
		c'(c^{-1}(y)) & = -\sin(c^{-1}(y))                                                                                                \\
		              & = -\sqrt{\sin^2(c^{-1}(y))} \qquad  \text{car }c^{-1}(y)\in \intervii{0}{\pi} \text{ donc } \sin(c^{-1}(y))\geq 0 \\
		              & = -\sqrt{1-\cos^2(c^{-1}(y))}                                                                                     \\
		              & = -\sqrt{1-y^2}                                                                                                   \\
	\end{align*}
	Ainsi d'après la définition de la dérivée de la bijection réciproque on a : \(\Arccos'(y) = \frac{-1}{\sqrt{1-y^2}}\)
\end{dem}

\begin{rem}[démonstration d'une relation intéressante entre \(\Arctan(x)\) et \(\Arctan\paren{{\frac{1}{x}}}\)]
	Soit \(f:x\mapsto \Arctan{\paren{\frac{1}{x}}}\), on as \(D_f = \R\pd\accol{0}\) et \(f\) dérivable sur \(D_f\)
	\begin{align*}
		f'(x) & = \Arctan'\paren{\frac{1}{x}} \times \paren{\frac{1}{x}}'         \\
		      & = \frac{1}{1+\paren{\frac{1}{x}}^2} \times \paren{\frac{-1}{x^2}} \\
		      & = \frac{-1}{x^2+1}
	\end{align*}
	On remarque que \(\quantifs{\forall x \in \Rs}f'(x) = -\Arctan'\paren{x}\) ainsi \(\quantifs{\forall x \in \Rps}f'(x) +\Arctan'\paren{x} = 0\) donc \(\quantifs{\forall x \in \Rs}\paren{f(x)+\Arctan\paren{x}}'=0\) \\
	Ainsi il existe \(c\) un réel tel que \(\quantifs{\forall x \in \Rps}f(x) + \Arctan\paren{x} = c\)
	\begin{align*}
		\text{Pour } x = 1, f(1) + \Arctan(1) & = c \\
		f(1) + \frac{\pi}{4}                  & = c \\
		c = \frac{\pi}{2}
	\end{align*}
	Ainsi \(\quantifs{\forall x \in \Rps} \Arctan\paren{\frac{1}{x}} + \Arctan\paren{x} = \frac{\pi}{2}\) \\
	De manière analogue on trouve  \(\quantifs{\forall x \in \Rms} \Arctan\paren{\frac{1}{x}} + \Arctan\paren{x} = -\frac{\pi}{2}\) \\
\end{rem}

\chapter{Calcul algébrique (rappels et compléments)}

\minitoc
\section{Sommes et produit finis}
\begin{nota}
	Soit \(\paren{a_i}_{i\in I}\) une famille de réels indexée par un ensemble \(I\) fini. \\
	La somme (resp. le produit) de tous les réels de la famille est notée \(\sum_{i\in I} a_i\) (resp. \(\prod_{i\in I} a_i\)). \\
	\begin{itemize}
		\item Si \(I\) est l'ensemble vide, on convient que : \(\sum_{i\in I} a_i = 0\) et \(\prod_{i\in I} a_i = 1\).
		\item Si \(I = \accol{1,2,\ldots,n}\) avec \(n\) un entier naturel non nul, on note \(\sum_{i=1}^n a_i\)  ou \(\sum_{1\leq i \leq n} a_i\) au lieu de \(\sum_{i\in I} a_i\) (resp. \(prod_{i=1}^n a_i\) ou \(\prod_{1\leq i \leq n} a_i\) au lieu de \(\prod_{i\in I} a_i\)).
	\end{itemize}
\end{nota}
\begin{prop}[opération et calcul par paquets]
	\begin{itemize}
		\item Pour toutes familles \(\paren{a_i}_{i\in I}\) et \(\paren{b_i}_{i\in I}\) de réels indexées par \(I\) et pour tout couple\((\alpha,\beta)\) de réels, on a :
		      \[ \sum_{i\in I} \paren{\alpha a_i + \beta b_i} = \alpha \sum_{i\in I} a_i + \beta \sum_{i\in I} b_i \qquad \text{ et } \qquad \prod_{i\in I} \paren{a_i b_i} = \paren{\prod_{i\in I} a_i}\paren{\prod_{b_i}}\]
		\item Pour toute famille \(\paren{a_i}_{i\in I}\) de réels indexée par \(I\) avec \(I=I_1 \union I_2\) et \(I_1\cap I_2 = \emptyset\), on a :
		      \[ \sum_{i\in I} a_i = \sum_{i\in I_1} a_i + \sum_{i\in I_2} a_i \qquad \text{ et } \qquad  \prod_{i\in I} a_i = \prod_{i\in I_1} a_i \prod_{i\in I_2} a_i \]
	\end{itemize}
\end{prop}

\begin{exoex}
	~\\
	Calculer : \(\sum_{k=1}^{2n} (-1)^k k \) avec \(n \in \N\)
\end{exoex}

\begin{corr}
	\begin{align*}
		\sum_{k=1}^{2n} (-1)^k k & = \sum_{k=0}^{n-1} (-1)^{2k+1} (2k-+) + \sum_{k=1}^{n} (-1)^{2k} (2k) \\
		                         & = -\sum_{k=0}^{n-1} (2k+1) + \sum_{k=1}^{n} 2k                        \\
		                         & = -\paren{2\sum_{k=0}^{n-1} k + n} + 2\sum_{k=1}^{n} k                \\
		                         & = -\paren{2\frac{(n-1)n}{2} + n} + 2\frac{n(n+1)}{2}                  \\
		                         & = n\paren{n+1-n+1-1}                                                  \\
		                         & = n                                                                   \\
	\end{align*}
\end{corr}
\begin{defprop}[téléscopage]
	Soit \(\paren{b_i}_{1 \leq i \leq n}\) une famille \underline{finie} de réels avec \(n\) supérieur ou égal à \(2\).
	\begin{enumerate}
		\item La somme \(\sum_{i=1}^n b_{i+1}-b_i\) est dire somme télescopique et vaut \(b_{n+1}-b_1\).
		\item Si tous les \(b_i\) sont non nuls, le produit \(\prod_{i=1}^n \frac{b_{i+1}}{b_i}\) est dit produit télescopique et vaut \(\frac{b_{n+1}}{b_1}\).
	\end{enumerate}
\end{defprop}

\begin{defprop}[Somme usuelles]
	Pour tout entier naturel \(n\) et tout réel \(x\) différent de \(1\), on a :
	\[\sum_{k=0}^n k = \frac{n(n+1)}{2} \qquad \sum_{k=0}^n k^2 = \frac{n(n+1)(2n+1)}{6} \qquad \sum_{k=0}^n x^k = \frac{x^{n+1}-1}{x-1}\]
\end{defprop}

\begin{defprop}[Factorisation de \(a^n-b^n\) ]
	Pour tout \(n\) entier naturel non nul et tout couple \((a,b)\) de réels, on a :
	\begin{align*}
		a^n-b^n & = (a-b)\paren{a^{n-1} + a^{n-2}b + \ldots + ab^{n-2} + b^{n-1}} \\
		        & = (a-b)\sum_{k=0}^{n-1} a^{n-1-k}b^k                            \\
		        & = (a-b)\sum_{k=0}^{n-1} a^kb^{n-1-k}                            \\
	\end{align*}
\end{defprop}

\begin{defprop}[coefficients binomiaux]
	Soit \(n\) un entier naturel non et \(k\) entière relatif, on a:
	\begin{enumerate}
		\item \(\binom{n}{k} = \binom{n}{n-k} \hfill \text{(symétrie)}\)
		\item \(binom{n}{k} +\binom{n}{k+1} = \binom{n+1}{k+1} \hfill \text{(relation de Pascal)}\)
		\item \(\binom{n}{k} \) est un entier naturel
	\end{enumerate}
\end{defprop}

\begin{defprop}[Formule du binôme de Newton]
	Pour tout couple \((a,b)\) de réels et tout entier naturel \(n\), on a :
	\[\paren{a+b}^n = \sum_{k=0}^n \binom{n}{k} a^{n-k}b^k = \sum_{k=0}^n \binom{n}{k} a^{k}b^{n-k}\]
\end{defprop}


\section{Cas des sommes doubles finies}
\begin{defi}
	Soit \(A\) un ensemble fini de couples et \((a_{i,j})_{(i,j)\in A}\) une famille de réels indexée par \(A\). La somme de tous les réels de la famille \((a_{i,j})_{(i,j)\in A}\) est notée \(\sum_{(i,j)\in A} a_{i,j}\) et appelée somme double. \\
	\underline{Remarque} : Si \(A\) est l'ensemble vide, on convient que \(\sum_{(i,j)\in A} a_{i,j} = 0\)
\end{defi}
\begin{defprop}[Sommes double rectangulaires]
	Dans le cas où \(A = \accol{1,2,\ldots,n}\times \accol{1,2,\ldots,m}\) avec \(n\) et \(m\) des entiers naturels non nuls,
	\begin{itemize}
		\item la somme double \(\sum_{(i,j)\in A} a_{i,j}\) est rectangulaire
		\item le somme double \(\sum_{(i,j)\in A} a_{i,j}\) s'écrit aussi \(\sum_{\substack{1 \leq i \leq n \\ 1 \leq j \leq m}} a_{i,j}\)
		\item la somme double \(\sum_{(i,j)\in A} a_{i,j}\) vaut  :
		      \[ sum_{(i,j)\in A} a_{i,j} = \sum_{\substack{1 \leq i \leq n \\ 1 \leq j \leq m}} a_{i,j} = \sum_{i=1}^n \paren{\sum_{j=1}^m a_{i,j}} = \sum_{j=1}^m \paren{\sum_{i=1}^n a_{i,j}} \]
		\item si \((b_i)_{1\leq i \leq n}\) et \((c_j)_{1\leq j \leq m}\) sont des familles finies de réels, alors : \[\paren{\sum_{i=1}^n b_i}\paren{\sum_{j=1}^m c_j} = \sum_{\substack{1 \leq i \leq n \\ 1 \leq j \leq m}} b_i c_j\]
	\end{itemize}
\end{defprop}

\begin{defprop}[somme double triangulaire]
	Dans le cas où \(A = \accol{(i,j) \in \N^2 | 1 \leq i \leq j \leq n}\) avec \(n\) un entier naturel non nul,
	\begin{itemize}
		\item La somme double \(\sum_{(i,j)\in A} a_{i,j}\) est dite triangulaire.
		\item La somme double \(\sum_{(i,j)\in A} a_{i,j}\) s'écrit aussi \(\sum_{1\leq i \leq j \leq n} a_{i,j}\) et vaut:
		      \[\sum_{(i,j)\in A} a_{i,i} = \sum_{1\leq i \leq j \leq n} a_{i,j} = \sum_{i=1}^n \paren{\sum_{j=i}^n a_{i,j}} = \sum_{j=1}^n \paren{\sum_{i=1}^j a_{i,j}}\]
	\end{itemize}
\end{defprop}


\section{Système linéaire de deux équations à deux inconnues}
\begin{defprop}[rappel de première]
	Dans le plan \(\R^2\) muni d’un repère orthonormé \((O,\vec{i},\vec{j})\), toute droite \(D\) admet une équation de la forme \[ax + by = c\]
	où \(a\), \(b\) et \(c\) sont des réels tels que \((a,b)\neq (0,0)\). \\
	Avec ces notations,
	\begin{itemize}
		\item le vecteur \(\vec{n}\) de coordonnées \((a,b)\) est un vecteur normal à \(D\) ;
		\item le vecteur \(\vec{u}\) de coordonnées \((-b,a)\) est un vecteur directeur de \(D\).
	\end{itemize}
\end{defprop}

\begin{defprop}[Système linéaire de deux équations à deux inconnues]
	Soit \(a\), \(b\), \(c\), \(a'\), \(b'\) et \(c'\) des réels. Le système d’équations
	\[
		(S) :
		\begin{cases}
			ax + by = c \\
			a'x + b'y = c'
		\end{cases}
	\]
	d’inconnues les réels \(x\) et \(y\) est dit système linéaire de deux équations à deux inconnues.
\end{defprop}

\begin{defprop}[Interprétation géométrique]
	Dans le cas où \((a,b)\neq (0,0)\) et \((a',b')\neq (0,0)\), résoudre le système \((S)\) revient à  déterminer l’intersection entre deux droites \(D\) et \(D'\) du plan.
	Trois cas se présentent :
	\begin{itemize}
		\item Les droites sont confondues donc \((S)\) a une infinité de solutions qui forment une droite ;
		\item Les droites sont sécantes donc \((S)\) a une unique solution ;
		\item Les droites sont parallèles non confondues donc \((S)\) n’a pas de solutions.
	\end{itemize}
\end{defprop}


\section{Système linéaire de trois équations à trois inconnues}

\begin{defprop}[rappel de terminale]
	Dans l'espace \(\R^3\) muni d’un repère orthonormé \((O,\vec{i},\vec{j},\vec{k})\), tout plan \(P\) admet une équation de la forme
	\[ax + by + cz = d\]
	où \(a\), \(b\), \(c\) et \(d\) sont des réels tels que \((a,b,c)\neq (0,0,0)\)
	\begin{itemize}
		\item le vecteur \(\vec{n}\) de coordonnées \((a,b,c)\) est un vecteur normal à \(P\) ;
		\item deux vecteurs non colinéaires pris parmi les vecteurs de coordonnées \((-b,a,0)\), \((0,-c,b)\) et \((-c,0,a)\) donnent la direction de \(P\).
	\end{itemize}
\end{defprop}

\begin{defprop}[Système linéaire de deux équations à trois inconnues]
	Soit \(a\), \(b\), \(c\), \(d\), \(a'\), \(b'\), \(c'\) et \(d'\) des réels. Le système d’équations
	\[
		(S) :
		\begin{cases}
			ax + by + cz = d \\
			a'x + b'y + c'z = d'
		\end{cases}
	\]
	d’inconnues les réels \(x\), \(y\) et \(z\) est dit système linéaire de deux équations à trois inconnues.
\end{defprop}

\begin{defprop}[Interprétation géométrique]
	Dans le cas où \((a,b,c)\neq (0,0,0)\) et \((a',b',c')\neq (0,0,0)\), résoudre le système \((S)\) revient à déterminer l’intersection entre deux plans \(P\) et \(P'\) de l’espace.
	Trois cas se présentent :
	\begin{itemize}
		\item Les plans sont confondus donc \((S)\) a une infinité de solutions qui forment un plan ;
		\item Les plans sont sécants donc \((S)\) a une infinité de solutions qui forment une droite ;
		\item Les plans sont parallèles non confondus donc \((S)\) n’a pas de solutions.
	\end{itemize}
\end{defprop}
\begin{defprop}[Système linéaire de trois équations à trois inconnues]
	Soit \(a\), \(b\), \(c\), \(d\), \(a'\), \(b'\), \(c'\), \(d'\), \(a''\), \(b''\), \(c''\) et \(d''\) des réels. Le système d’équations
	\[ (S) : \begin{cases}
			ax + by + cz = d     \\
			a'x + b'y + c'z = d' \\
			a''x + b''y + c''z = d''
		\end{cases} \]
	d’inconnues les réels \(x\), \(y\) et \(z\) est dit système linéaire de trois équations à trois inconnues.
\end{defprop}

\begin{defprop}[Interprétation géométrique]
	Dans le cas où \((a,b,c)\neq (0,0,0)\), \((a',b',c')\neq (0,0,0)\) et \((a'',b'',c'')\neq (0,0,0)\), résoudre le système \((S)\) revient à déterminer l’intersection entre trois plans \(P\), \(P'\) et \(P''\) de l’espace.
	Cela conduit à distinguer huit cas de figures qui donnent quatre types d’ensemble-solution pour \((S)\) :
	\begin{itemize}
		\item Le système \((S)\) a une infinité de solutions qui forment un plan ;
		\item Le système \((S)\) a une infinité de solutions qui forment une droite ;
		\item Le système \((S)\) a une unique solution ;
		\item Le système \((S)\) n’a pas de solutions.
	\end{itemize}
\end{defprop}

\section{Algorithme du Pivot}
\begin{rem} [Remarque préliminaire]
	En cycle terminal, de petits systèmes linéaires ont été rencontrés et résolus dans des cas simples, le plus souvent par “substitution”. \\
	En MP2I, nous utiliserons en priorité la méthode de résolution par “pivot”. Plus efficace et élégante, cette technique sera reprise au semestre 2 dans le chapitre “Matrices” pour résoudre plus généralement des systèmes linéaires de \(n\) équations à \(p\) inconnues.
\end{rem}

\begin{defprop}[Opérations élémentaires]
	On reprend les notations des paragraphes III. et IV. et on note \(L_i\) la \(i\)-ème ligne du système \((S)\).\\
	On appelle opérations élémentaires sur les lignes du système linéaire \((S)\) :
	\begin{enumerate}
		\item l’échange de deux lignes distinctes : \(L_i \leftrightarrow L_j\) avec \(i\neq j\) ;
		\item la multiplication d'une ligne par un réel non nul : \(L_i \leftarrow \lambda L_i\) avec \(\lambda\neq 0\) ;
		\item l'addition à une ligne du produit d'une autre ligne par un réel non nul : \(L_i \leftarrow L_i + \lambda L_j\) avec \(i\neq j\) et \(\lambda\neq 0\).
	\end{enumerate}
\end{defprop}

\begin{prop}[Propriété importante]
	Toute opération élémentaire sur les lignes d’un système linéaire le transforme en un système linéaire équivalent c’est-à-dire un système ayant le même ensemble de solutions.
\end{prop}

\begin{defprop}[résolution d'un système linéaire par la méthode du pivot]
	La résolution d’un système linéaire par la méthode du pivot se déroule en deux phases :
	\begin{itemize}
		\item \underline{phase de descente} : en effectuant des opérations élémentaires sur les lignes du système, on transforme le système en un système de forme “triangulaire” ou “trapézoïdale” comme, par exemple,
		      \[(S1) : \begin{cases} a_1x+b_1y = c_1 \\ b'_1y = c'_1 \end{cases}\]
		      \[(S2) : \begin{cases} a_1x+b_1y+c_1z = d_1 \\ b'_1y+c'_1z = d'_1 \end{cases}\]
		      \[(S3) : \begin{cases} a_1x+b_1y+c_1z = d_1 \\ b'_1y+c'_1z = d'_1 \\ c''_1z = d''_1 \end{cases}\]
		\item \underline{phase de remontée} : Le système obtenu est équivalent au système initial ; il est facile à résoudre ce qui permet d’obtenir l’ensemble des solutions du système initial. Dans cette phase de remontée, on peut au choix :
		      \begin{itemize}
			      \item effectuer des substitutions successives (moins élégant) ;
			      \item utiliser à nouveau des opérations élémentaires sur les lignes pour réduire le système sous forme “diagonale” (plus élégant et facile à coder).
		      \end{itemize}
	\end{itemize}
\end{defprop}
\begin{rem}
	Les opérations élémentaires effectuées lors de la résolution d’un système linéaire par la méthode du pivot (phases de descente et de remontée) doivent systématiquement être indiquées en marge du système étudié pour faciliter la lecture des correcteurs et permettre de retrouver les éventuelles erreurs de calcul.
\end{rem}

\begin{rem}[Pour aller plus loin (pour ceux qui ont suivi l’option maths expertes)]
	\begin{itemize}
		\item Les petits systèmes linéaires décrits au III. et IV. peuvent se traduire matriciellement par une équation matricielle du type \(AX = B\) avec \(A\) et \(B\) des matrices à préciser et \(X\) une matrice colonne inconnue.
		\item L’effet des opérations élémentaires sur les lignes de ces systèmes peut se traduire matriciellement par des multiplications de la matrice \(A\) à gauche par des matrices inversibles bien
	\end{itemize}
\end{rem}


%finir de taper les notes de cours 


\end{document}